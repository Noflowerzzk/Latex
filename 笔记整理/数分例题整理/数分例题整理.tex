\special{dvipdfmx:config z 0} %取消PDF压缩,加快速度,最终版本生成的时候最好把这句话注释掉


\documentclass{ctexart}
\usepackage{newtxtext, amsmath, amssymb}
\usepackage[left=0.5in, right=0.5in, top=0.9in, bottom=0.9in]{geometry}
\usepackage{amsthm} % 设置定理环境,要在amsmath后用
\usepackage{xpatch} % 用于去除amsthm定义的编号后面的点
\usepackage[strict]{changepage} % 提供一个 adjustwidth 环境
\usepackage{multicol} % 用于分栏
\usepackage[fontsize=14pt]{fontsize} % 字号设置
\usepackage{esvect} % 实现向量箭头输入(显示效果好于\vec{}和\overrightarrow{}),格式为\vv{⟨向量符号⟩}或\vv*{⟨向量符号⟩}{⟨下标⟩} 
\usepackage[
            colorlinks, % 超链接以颜色来标识,而并非使用默认的方框来标识
            linkcolor=mlv,
            anchorcolor=mlv,
            citecolor=mlv % 设置各种超链接的颜色
            ]
            {hyperref} % 实现引用超链接功能
\usepackage{tcolorbox} % 盒子效果
\usepackage{hyperref}
\usepackage{fancyhdr}
\usepackage{extarrows}
\tcbuselibrary{most} % tcolorbox宏包的设置,详见宏包说明文档
\everymath{\displaystyle}  % 让所有数学模式都使用 \displaystyle


\geometry{a4paper,centering,scale=0.8}

% 保证 leftmark 始终为 subsection 名称
\newcommand{\chaptermark}{\markboth{}{}}
\renewcommand{\sectionmark}[1]{\markboth{}{}}
\renewcommand{\subsectionmark}[1]{\markboth{#1}{}}

\pagestyle{fancy}%清除原页眉页脚样式
\fancyhf{}

\fancyhead[L]{\leftmark}
\fancyhead[R]{\thepage}

% ————————————————————————————————————自定义符号————————————————————————————————————
\newcommand{\。}{.} % 示例:将指令\。定义为全角句点。
\newcommand{\zte}{\mathrm{e}} % 正体e
\newcommand{\ztd}{\mathrm{d}} % 正体d
\newcommand{\zti}{\mathrm{i}} % 正体i
\newcommand{\ztj}{\mathrm{j}} % 正体j
\newcommand{\ztk}{\mathrm{k}} % 正体k
\newcommand{\CC}{\mathbb{C}} % 黑板粗体C
\newcommand{\QQ}{\mathbb{Q}} % 黑板粗体Q
\newcommand{\RR}{\mathbb{R}} % 黑板粗体R
\newcommand{\ZZ}{\mathbb{Z}} % 黑板粗体Z
\newcommand{\NN}{\mathbb{N}} % 黑板粗体N
\newcommand{\ds}{^\prime}
\newcommand{\dds}{^{\prime\prime}}
\newcommand{\dx}{\mathrm{d}x}
\newcommand{\dd}{\mathrm{d}}
% ————————————————————————————————————自定义符号————————————————————————————————————

% ————————————————————————————————————自定义颜色————————————————————————————————————
\definecolor{mlv}{RGB}{40, 137, 124} % 墨绿
\definecolor{qlv}{RGB}{240, 251, 248} % 浅绿
\definecolor{slv}{RGB}{33, 96, 92} % 深绿
\definecolor{qlan}{RGB}{239, 249, 251} % 浅蓝
\definecolor{slan}{RGB}{3, 92, 127} % 深蓝
\definecolor{hlan}{RGB}{82, 137, 168} % 灰蓝
\definecolor{qhuang}{RGB}{246, 250, 235} % 浅黄
\definecolor{shuang}{RGB}{77, 82, 59} % 深黄
\definecolor{qzi}{RGB}{240, 243, 252} % 浅紫
\definecolor{szi}{RGB}{49, 55, 70} % 深紫
% 将RGB换为rgb,颜色数值取值范围改为0到1
% ————————————————————————————————————自定义颜色————————————————————————————————————

% ————————————————————————————————————盒子设置————————————————————————————————————
% tolorbox提供了tcolorbox环境,其格式如下:
% 第一种格式:\begin{tcolorbox}[colback=⟨背景色⟩, colframe=⟨框线色⟩, arc=⟨转角弧度半径⟩, boxrule=⟨框线粗⟩]   \end{tcolorbox}
% 其中设置arc=0mm可得到直角;boxrule可换为toprule/bottomrule/leftrule/rightrule可分别设置对应边宽度,但是设置为0mm时仍有细边,若要绘制单边框线推荐使用第二种格式
% 方括号内加上title=⟨标题⟩, titlerule=⟨标题背景线粗⟩, colbacktitle=⟨标题背景线色⟩可为盒子加上标题及其背景线
\newenvironment{kuang1}{
    \begin{tcolorbox}[colback=hlan, colframe=hlan, arc = 1mm]
    }
    {\end{tcolorbox}}
\newenvironment{kuang2}{
    \begin{tcolorbox}[colback=hlan!5!white, colframe=hlan, arc = 1mm]
    }
    {\end{tcolorbox}}
% 第二种格式:\begin{tcolorbox}[enhanced, colback=⟨背景色⟩, boxrule=0pt, frame hidden, borderline={⟨框线粗⟩}{⟨偏移量⟩}{⟨框线色⟩}]   {\end{tcolorbox}}
% 将borderline换为borderline east/borderline west/borderline north/borderline south可分别为四边添加框线,同一边可以添加多条
% 偏移量为正值时,框线向盒子内部移动相应距离,负值反之
\newenvironment{kuang3}{
    \begin{tcolorbox}[enhanced, breakable, colback=hlan!5!white, boxrule=0pt, frame hidden,
        borderline south={0.5mm}{0.1mm}{hlan}]
    }
    {\end{tcolorbox}}
\newenvironment{lvse}{
    \begin{tcolorbox}[enhanced, breakable, colback=qlv, boxrule=0pt, frame hidden,
        borderline west={0.7mm}{0.1mm}{slv}]
    }
    {\end{tcolorbox}}
\newenvironment{lanse}{
    \begin{tcolorbox}[enhanced, breakable, colback=qlan, boxrule=0pt, frame hidden,
        borderline west={0.7mm}{0.1mm}{slan}]
    }
    {\end{tcolorbox}}
\newenvironment{huangse}{
    \begin{tcolorbox}[enhanced, breakable, colback=qhuang, boxrule=0pt, frame hidden,
        borderline west={0.7mm}{0.1mm}{shuang}]
    }
    {\end{tcolorbox}}
\newenvironment{zise}{
    \begin{tcolorbox}[enhanced, breakable, colback=qzi, boxrule=0pt, frame hidden,
        borderline west={0.7mm}{0.1mm}{szi}]
    }
    {\end{tcolorbox}}
% tcolorbox宏包还提供了\tcbox指令,用于生成行内盒子,可制作高光效果
\newcommand{\hl}[1]{
    \tcbox[on line, arc=0pt, colback=hlan!5!white, colframe=hlan!5!white, boxsep=1pt, left=1pt, right=1pt, top=1.5pt, bottom=1.5pt, boxrule=0pt]
{\bfseries \color{hlan}#1}}
% 其中on line将盒子放置在本行(缺失会跳到下一行),boxsep用于控制文本内容和边框的距离,left、right、top、bottom则分别在boxsep的参数的基础上分别控制四边距离
% ————————————————————————————————————盒子设置————————————————————————————————————

% ————————————————————————————————————自定义字体设置————————————————————————————————————
\setCJKfamilyfont{xbsong}{方正小标宋简体}
\newcommand{\xbsong}{\CJKfamily{xbsong}}
% CTeX宏集还预定义了\songti、\heiti、\fangsong、\kaishu、\lishu、\youyuan、\yahei、\pingfang等字体命令
% 由于未知原因,一些设备可能无法调用这些字体,故此文档暂时未使用
% ————————————————————————————————————自定义字体设置————————————————————————————————————

% ————————————————————————————————————各级标题设置————————————————————————————————————
\ctexset{
    % 修改 section。
    section={   
    % 设置标题编号前后的词语,使用name={⟨前部分⟩,⟨后部分⟩}参数进行设置。    
        name={\textbf{第},\textbf{章}\hspace{18pt}},
    % 使用number参数设置标题编号,\arabic设置为阿拉伯数字,\chinese设置为中文,\roman设置为小写罗马字母,\Roman设置为大写罗马字母,\alph设置为小写英文,\Alph设置为大写英文。
        number={\textbf{\chinese{section}}},
    % 参数format设置标题整体的样式。包括标题主题、编号以及编号前后的词语。
    % 参数format还可以设置标题的对齐方式。居中对齐\centering,左对齐\raggedright,右对齐\hfill
        format=\color{hlan}\centering\zihao{2}, % 设置 section 标题为正蓝色、黑体、居中对齐、小二号字
    % 取消编号后的空白。编号后有一段空白,参数aftername=\hspace{0pt}可以用来控制编号与标题之间的距离。
        aftername={}
    },
    % 修改 subsection。
    subsection={   
        name={\S \hspace{6pt}, \hspace{8pt}},
        number={\arabic{section}.\arabic{subsection}},
        format=\color{hlan}\centering\zihao{-2}, % 设置 subsection 标题为黑体、三号字
        aftername=\hspace{0pt}
    },
    subsubsection={   
        name={,、},
        number={\chinese{subsubsection}},
        format=\raggedright\color{hlan}\zihao{3},
        aftername=\hspace{0pt}
    },
    part={   
        name={第,部分},
        number={\chinese{part}},
        format=\color{white}\centering\bfseries\zihao{-1},
        aftername=\hspace{1em}
    }
}
% 各个标题设置的各行结尾一定要记得写逗号!!!!!!!!!!!!!!!!!!!!!!!!!!!!!!!!!!!!!!!!!!
% ————————————————————————————————————各级标题设置————————————————————————————————————

% ————————————————————————————————————定理类环境设置————————————————————————————————————
\newtheoremstyle{t}{0pt}{}{}{\parindent}{\bfseries}{}{1em}{} % 定义新定理风格。格式如下:
%\newtheoremstyle{⟨风格名⟩}
%                {⟨上方间距⟩} % 若留空,则使用默认值
%                {⟨下方间距⟩} % 若留空,则使用默认值
%                {⟨主体字体⟩} % 如 \itshape
%                {⟨缩进长度⟩} % 若留空,则无缩进;可以使用 \parindent 进行正常段落缩进
%                {⟨定理头字体⟩} % 如 \bfseries
%                {⟨定理头后的标点符号⟩} % 如点号、冒号
%                {⟨定理头后的间距⟩} % 不可留空,若设置为 { },则表示正常词间间距;若设置为 {\newline},则环境内容开启新行
%                {⟨定理头格式指定⟩} % 一般留空
% 定理风格决定着由 \newtheorem 定义的环境的具体格式,有三种定理风格是预定义的,它们分别是:
% plain: 环境内容使用意大利斜体,环境上下方添加额外间距
% definition: 环境内容使用罗马正体,环境上下方添加额外间距
% remark: 环境内容使用罗马正体,环境上下方不添加额外间距
\theoremstyle{t} % 设置定理风格
\newtheorem{dyhj}{\color{slv} 定义}[subsection] % 定义定义环境,格式为\newtheorem{⟨环境名⟩}{⟨定理头文本⟩}[⟨上级计数器⟩]或\newtheorem{⟨环境名⟩}[⟨共享计数器⟩]{⟨定理头文本⟩},其变体\newtheorem*不带编号
\newtheorem{dlhj}{\color{shuang} 定理}[subsection]
\newtheorem{lthj}{\color{szi} 例}[subsection]
\newtheorem*{jiehj}{\color{slan} 解}
\newtheorem*{zmhj}{\color{slan} 证明}
\newtheorem{ylhj}{\color{shuang} 引理}[subsection]
\newtheorem*{ylzmhj}{\color{slan} 引理的证明}
\newenvironment{dy}{\begin{lvse}\begin{dyhj}}{\end{dyhj}\end{lvse}}
\newenvironment{zm}{\begin{lanse}\begin{zmhj}}{$\hfill \square$\end{zmhj}\end{lanse}}
\newenvironment{dl}{\begin{huangse}\begin{dlhj}}{\end{dlhj}\end{huangse}}
\newenvironment{lt}{\begin{zise}\begin{lthj}}{\end{lthj}\end{zise}}
\newenvironment{yl}{\begin{huangse}\begin{ylhj}}{\end{ylhj}\end{huangse}}
% ————————————————————————————————————定理类环境设置————————————————————————————————————

% ————————————————————————————————————目录设置————————————————————————————————————
\setcounter{tocdepth}{4} % 设置在 ToC 的显示的章节深度
\setcounter{secnumdepth}{3} % 设置章节的编号深度
% 数值可选选项:-1 part 0 chapter 1 section 2 subsection 3 subsubsection 4 paragraph 5 subparagraph
\renewcommand{\contentsname}{\bfseries 目\hspace{1em}录}
% ————————————————————————————————————目录设置————————————————————————————————————

\everymath{\displaystyle} % 设置所有数学公式显示为行间公式的样式

\begin{document}

\quad\\
{
\Huge \bfseries 
\begin{kuang3}
    \color{hlan}
    \centering
    数分例题整理
\end{kuang3}
}

\thispagestyle{empty} % 取消这一页的页码
\newpage

% ————————————————————————————————————页码设置————————————————————————————————————
\pagenumbering{Roman}
\setcounter{page}{1}

\begin{multicols}{2} % 设置环境中内容为两栏。数字为栏数
    {
    \tableofcontents
    }
\end{multicols}

\newpage
\setcounter{page}{1}
\pagenumbering{arabic}
% ————————————————————————————————————页码设置————————————————————————————————————

% 指令\addcontentsline{toc}{⟨章节层级⟩}{⟨标题名⟩}可将“⟨标题名⟩”加入目录。如为无章节层级的标题,可先用\phantomsection指令添加section分级

\begin{kuang1}
    \part{实数基本定理与极限}
\end{kuang1}

\begin{kuang2}
    \section{实数的定义}
\end{kuang2}

\begin{kuang3}
    \subsection{自然数与其定义}
\end{kuang3}

\subsubsection{自然数的定义}

\begin{dy}[Peano公理]
    (略)    
\end{dy}

\begin{dy}[自然数加法与乘法] 
    \textbf{自然数加法}定义为映射 $+: \NN \times \NN \rightarrow \NN$, 满足以下性质: 
    \begin{itemize}
        \item $a + b = b + a$ 
        \item $a + 1 = S(a)$
        \item $a + S(b) = S(a + b)$
    \end{itemize} 
    \textbf{自然数的乘法}定义为映射 $\cdot: \NN \times \NN \rightarrow \NN$, 满足以下性质: 
    \begin{itemize}
        \item $a \cdot b = b \cdot a$
        \item $a \cdot 1 = a$
        \item $a \cdot S(b) = a + (a \cdot b)$
    \end{itemize}
\end{dy}

\begin{dy}[自然数的大小关系]
    $a < b$ 当且仅当存在$c \in \NN, b = a + c$.
\end{dy}

\begin{kuang3}
    \subsection{实数的定义}
\end{kuang3}
(略)

\begin{kuang3}
    \subsection{实数基本定理}
\end{kuang3}

\begin{lt}
    证明: $\RR$ 不可列.
    \begin{zmhj}
        使用闭区间套定理. \\
        反证法. 假设 $\RR$ 可列, 记 $\RR = \{x_1, x_2, \cdots, x_n, \cdots\}$.
        \begin{itemize}
            \item [(1)] 取 $[a_1, b_1]$ 使得 $x_1 \notin [a_1, b_1]$.
            \item [(2)] 三分 $[a_1, b_1]$ 得三个小区间, 三者必有其一不含 $x_2$. 记为 $[a_2, b_2]$
            $\vdots $
        \end{itemize}
        由此得到一个闭区间套 $\{[a_n, b_n]\}_{n = 1}^{\infty}$. 由闭区间套定理, $\exists \xi \in \RR, \forall n \in \NN, \xi \in [a_n, b_n]$. \\
        但对 $\forall k \in \NN, x_k \notin [a_k, b_k]$, 所以 $\displaystyle{x_k \notin \bigcap_{n = 1}^\infty[a_n, b_n] }$. 故 $\displaystyle{\RR \cap \left(\bigcap_{n = 1}^\infty[a_n, b_n]\right)} = \varnothing $, 矛盾!
    \end{zmhj}
\end{lt}

\begin{kuang2}
    \section{数列极限与相关计算}
\end{kuang2}

\begin{kuang3}
    \subsection{数列极限与相关计算}
\end{kuang3}

\begin{lt}
    证明: $\displaystyle{\lim_{n \to \infty}\sqrt[n]{n}} = 1$. 
    \begin{zmhj}
        令 $\sqrt[n]{n} = 1 + y_n$, 有 
        \[
            n = (1 + y_n)^n > 1 + \frac{n(n - 1)}{2}y_n.
        \]
        故 
        \[
            \left\lvert \sqrt[n]{n} - 1\right\rvert = \left\lvert y_n\right\rvert < \sqrt{\frac{2}{n}}, \forall n \in \NN. 
        \]
        对任意 $\varepsilon > 0$, 取 $N = \left[\frac{2}{\varepsilon^2}\right] + 1$, 则对 $n < N$ 有 $\left\lvert \sqrt[n]{n} - 1\right\rvert < \varepsilon$
    \end{zmhj}
\end{lt}

\begin{lt}
    判断以下命题是否正确. 若正确, 给出证明; 若不正确, 给出反例:  \\
    数列 ${a_n} $收敛的充要条件是,对任意正正数 $p$, 都有 $\displaystyle{\lim_{n \to \infty} \left\lvert a_n - a_{n + p}\right\rvert = 0}$. 
    \begin{jiehj}
        反例如下: 令 $a_n = \sqrt{n}$, 则 $\forall p > 0$
        \[
            \left\lvert a_{n + p} - a_n\right\rvert = \frac{p}{\sqrt{n + p} + \sqrt{n}} \rightarrow 0,
        \]
        但显然该数列不收敛.
    \end{jiehj}
\end{lt}

\begin{lt}
    求极限: $\lim_{n \to \infty}\sin\left(\sqrt{4n^2 + n}\pi \right)$.
    \begin{jiehj}
        \begin{align*}
            \sin\left(\sqrt{4n^2 + n}\pi\right) = \ & \sin\left(\left(\sqrt{4n^2 + n} - 2n\right)\pi\right) \\
            = \ & \sin \left(\frac{n}{\sqrt{4n^2 + n} + 2n}\pi\right) \\
            = \ & \sin \left(\frac{1}{\sqrt{4 + \frac{1}{n}} + 2}\pi\right)
        \end{align*}
        所以 $$\lim_{n \to \infty}\sin\left(\sqrt{4n^2 + n}\pi \right) = \sin \frac{\pi}{4} = \frac{\sqrt{2}}{2}$$
    \end{jiehj}
\end{lt}

\begin{kuang2}
    \section{函数极限、连续相关定理}
\end{kuang2}

\begin{kuang3}
    \subsection{函数极限}
\end{kuang3}

\begin{dl}\label{单调函数任意一点左右极限均存在}
    单调函数任意一点左右极限均存在
\end{dl}

\begin{zm}
    不妨 $f(x)$ 在 $(a, b)$ 上单增,对任意 $x_0 \in (a, b)$, $\{f(x) | x \in (a, x_0)\}$ 有上确界 $\alpha$. \\
    对任意 $x \in (a, x_0), f(x) \leqslant \alpha$, 但 $\forall \varepsilon > 0, \exists x' \in (a, x_0), f(x') > \alpha$. 由 $f(x)$ 单调性,$\forall x \in (x', x_0), -\varepsilon < f(x') \leqslant f(x) - \alpha \leqslant 0$, 即 $\lim_{x \to x_0^-}f(x) = \alpha$. \\
    右极限同理. 
\end{zm}


\begin{kuang3}
    \subsection{连续函数与间断}
\end{kuang3}

\begin{yl}[单调函数的不连续点]\label{单调函数的不连续点必然是跳跃间断点}
    单调函数的不连续点必然是跳跃间断点
\end{yl}

\begin{zm}
    由\textbf{定理\ref{单调函数任意一点左右极限均存在}}即得 
\end{zm}


\begin{dl}\label{单调函数至多有可列个间断点}
    单调函数至多有可列个间断点
\end{dl}

\begin{zm}
    由单调性及间断点的性质,$\lim_{x \to x_0^-}f(x) < f(x_0) < \lim_{x \to x_0^+}f(x)$. \\
    由有理数的稠密性,在每个间断点 $x_0$ 的区间(由单调性,它们两两不交) \\ $\left(\lim_{x \to x_0^-}f(x), \lim_{x \to x_0^+}f(x)\right)$, 必存在一个有理数,用这个有理数代表这个区间. 则这些有理数与这些间断点一一对应. \\
    因此间断点至多有可列个. 
\end{zm}

\begin{lt}
    $f: \RR \to \RR$ 是一个连续函数. 定义 $L(f) = \{x \in \RR | f(x) = 0\}$. 证明:若 $L(f)$ 非空,则 $L(f)$ 是一个闭集. (闭集是包含所有聚点的集合)
    \begin{jiehj}
        
    \end{jiehj}
\end{lt}

\begin{kuang1}
    \part{微分与积分}
\end{kuang1}

\begin{kuang2}
    \section{函数微分、导数相关定理}
\end{kuang2}

\begin{kuang3}
    \subsection{导数基本性质与中值定理}
\end{kuang3}

\subsubsection{中值定理}

\begin{lt}
    $f(x) \in C[a, b] \cap D(a, b)$, 其中 $a > 0$ 且 $f(a) = 0$. 证明: $\exists \xi \in (a, b), f(\xi) = \frac{b - \xi}{a}f'(\xi)$.
    \begin{jiehj}
        令 $F(x) = (b - x)^af(x)$, 易知 $F(x) \in C[a, b] \cap D(a, b), F(a) = F(b) = 0$, 由 Rolle 定理, $\exists \xi \in (a, b), F'(\xi) = 0 \Rightarrow f(\xi) = \frac{b - \xi}{a}f'(\xi)$
    \end{jiehj} 
\end{lt}

用 Rolle 定理证明等式的基本方法:
\begin{itemize}
    \item 将等号一端改写成只有零,形成 $g(\xi) = 0$
    \item 求积分 $G(x) = \int g(x) \dx$
    \item 验证 $G(x)$ 满足 Rolle 中值定理条件
\end{itemize}

例如上题,化简为 $\frac{f'(x)}{f(x)} - \frac{a}{b - x} = 0$, 有 $G(x) = (b - x)^af(x)$

\begin{lt}
    $f(x)$ 在 $[0, 1]$ 上二阶可导, $f(0) = f(1), f'(1) = 1$, 求证:存在 $\xi \in (0, 1), f''(\xi) = 2$.
    \begin{jiehj}
        $f''(x) - 2 = 0$,有 $f'(x) - 2x = C_1$,由条件 $f'(1) = 1$,有 $C_1 = -1$,化简有 $f'(x) - 2x + 1 = 0$,故 $F(x) = f(x) - x^2 + x$. 下略.
    \end{jiehj}
\end{lt}

\begin{lt}
    证明 $2^x - x^2 = 1$ 有且仅有三个实根
    \begin{jiehj}
        有三个解证明略.\\
        假设有四个解,反复使用中值定理可以证明最后的导函数没有解。
    \end{jiehj}
\end{lt}

\begin{lt}
    $f(x)$ 在 $(-\infty, + \infty)$ 上可微,且 $f(0) = 0$, $\left\lvert f'(x)\right\rvert \leqslant p\left\lvert f(x)\right\rvert $, 证明 $f(x) \equiv 0, x \in \RR$
    \begin{zmhj}
        先考虑 $x \in \left[0, \frac{1}{2p}\right]$ 的情形. $\left\lvert f(x)\right\rvert $ 在 $\left[0, \frac{1}{2p}\right]$ 的最大值为 $\left\lvert f(x_0)\right\rvert = M \geqslant 0$\\
        \[
            M = \left\lvert f(x) - f(0)\right\rvert = \left\lvert f'(\xi)x_0\right\rvert \leq \frac{1}{2p} \cdot p \left\lvert f(\xi)\right\rvert \leqslant \frac{1}{2}M
        \]
        因此 $M = 0$. 对其它区间同理归纳即可.
    \end{zmhj}
\end{lt}

\subsubsection{导函数的性质}

\begin{dl}[导数的 Darboux 定理(介值性)]
    $f(x)$ 可导且 $f'(a) \neq f'(b)$, 则对任意介于 $f'(a), f'(b)$ 之间的 $r$, 都存在 $\xi \in (a, b)$, 使得 $r = f'(\xi)$
\end{dl}

两种证明方法

\begin{zm}
    不妨 $f'(a) < f'(b)$, 对任意 $r \in (f'(a), f'(b))$, 令 $F(x) = f(x) - rx$, 有 $F(a) < 0, F(b) > 0$. 故由极限保号性,存在 $\delta > 0, \forall x \in (a, a + \delta), \frac{F(x) - F(a)}{x - a} < 0$, 即 $F(x) < F(a)$, $F(a)$ 不是最小值. \\
    同理 $F(b)$ 也不是最小值. 因此最小值 $F(\xi)$ 在开区间 $(a, b)$ 上取到,由费马定理, $F'(\xi) = 0 \Rightarrow f'(\xi) = r$.
\end{zm}

\begin{zm}
    假设同上,作函数 
    \begin{align*}
        F(x) = \left\{\begin{matrix}
            \frac{f(x) - f(a)}{x - a}, \quad & x \neq a \\
            f'(a), \quad & x = a
        \end{matrix}\right., \quad 
        G(x) = \left\{ \begin{matrix}
            \frac{f(x) - f(b)}{x - b}, \quad & x \neq b \\
            f'(b), \quad & x = b
        \end{matrix}\right.
    \end{align*}
    $r$ 要么在 $F(a), F(b)$ 之间,要么在 $G(a), G(b)$ 之间. \\
    如果 $r$ 在 $F(a), F(b)$ 之间,由连续函数的介值定理,存在 $x_0 \in (a, b), r = \frac{f(x_0) - f(a)}{x_0 - a}$ 再对 $F(x)$ 用 Lagrange 中值定理即可. $r$ 在 $G(a), G(b)$ 之间同理. 
\end{zm}

\begin{kuang3}
    \subsection{Taylor 公式}
\end{kuang3}

\subsubsection{推导与证明}

\begin{yl}
    若 $r(x_0) = r'(x_0) = r''(x_0) = \cdots = r^{(n)}(x_0) = 0$,则 $r(x) = o((x - x_0)^n)\ (x \to x_0)$.
\end{yl}

\begin{zm}
    对 $n$ 归纳. \\
    当 $n = 1$ 时,$r(x) = r(x_0) + r'(x_0)(x - x_0) + o(x - x_0) = o(x - x_0)$ \\
    假设当 $n \geqslant 1$ 有 $r^{(n)}(x) = o((x - x_0)^n)$ 成立. \\
    则当 $n + 1$ 时,由 Lagrange 中值定理,
    \[
        r(x) = r(x_0) + r'(x_0 + \theta(x - x_0))\cdot(x - x_0) = r'(x_0 + \theta(x - x_0))\cdot(x - x_0) 
    \]
    当 $x \to x_0$ 时,
    \[
        r'(x_0 + \theta(x - x_0))\cdot(x - x_0) = o((x - x_0)^n)\cdot (x - x_0) = o((x - x_0)^{n + 1})
    \]
    由归纳原理,原命题成立. 
\end{zm}


\begin{dl}[带 Lagrange 余项的泰勒公式]
    设 $f(x)$ 在 $[a, b]$ 上$n$阶可导,在 $(a, b)$ 上有 $n + 1$ 阶导. 设 $x_0$ 为 $[a, b]$ 内一点,则对任意 $x \in [a, b]$ 存在 $\theta \in (0, 1)$, 使得
    \begin{align*}
        f(x) & = f(x_0) + f'(x_0)(x - x_0) + \frac{1}{2}f''(x_0)(x - x_0)^2 +  \\
        & \cdots + \frac{1}{n!}f^{(n)}(x_0)(x - x_0)^n + \frac{1}{(n + 1)!}f^{(n + 1)}(x_0 + \theta(x - x_0))(x - x_0)^{(n + 1)}
    \end{align*}
\end{dl}

\begin{zm}
    令 $r(x) = f(x) - f(x_0) + f'(x_0)(x - x_0) + \frac{1}{2}f''(x_0)(x - x_0)^2 + \cdots + \frac{1}{n!}f^{(n)}(x_0)(x - x_0)^n$, 有 $r(x_0) = r'(x_0) = r''(x_0) = \cdots = r^{(n)}(x_0) = 0$. \\
    由 Cauchy 中值定理,
    \begin{align*}
        \frac{r(x)}{(x - x_0)^{n + 1}} & = \frac{r(x) - r(x_0)}{(x - x_0)^{n + 1} - (x_0 - x_0)^{n + 1}} \\
        & = \frac{r'(\xi_1) - r'(x_0)}{(n + 1)((\xi_1 - x_0)^n - (x_0 - x_0)^n)} & \xi_1 \in (x_0, x) \text{或} (x, x_0) \\
        & =  \frac{r''(\xi_2) - r''(x_0)}{(n + 1)\cdot n\cdot ((\xi - x_0)^{(n - 1)} - (x_0 - x_0)^{(n - 1)})} & \xi_1 \in (x_0, \xi_1) \text{或} (x, \xi_1) \\
        & = \cdots \\
        & = \frac{1}{(n + 1)!}\cdot \frac{r^{(n)}(\xi_n)}{\xi_n - x_0} & \xi_n \in (x_0, \xi_{n - 1}) \text{或} (\xi_{n - 1}, x_0) \\
        & = \frac{1}{(n + 1)!}r^{(n + 1)}(\xi) & \xi \in (x_0, \xi_{n}) \text{或} (\xi_{n}, x_0) \\
        & = \frac{1}{(n + 1)!}f^{(n + 1)}(\xi)
    \end{align*}
    因此 $r(x) = \frac{1}{(n + 1)!}f^{(n + 1)}(\xi)(x - x_0)^{n + 1}$  
\end{zm}

\begin{lt}[Taylor 公式的应用套路]
    $f(x)$ 在 $[a, b]$ 上二阶可导, $f(a) = f(b) = 0$, 证明:
    \[
        \max_{a \leqslant x \leqslant b}\left\lvert f(x)\right\rvert \leqslant \frac{1}{8} (b - a)^2 \max_{a \leqslant x \leqslant b}\left\lvert f''(x)\right\rvert 
    \]
    \begin{zmhj}
        由最值存在定理, $\exists x_0 \in [a, b], \left\lvert f(x_0)\right\rvert = \max_{a \leqslant x \leqslant b}\left\lvert f(x)\right\rvert$, 且 $f'(x_0) = 0$ \\
        把 $f(x)$ 在 $x = x_0$ 处展开,有
        \[
            f(x) = f(x_0) + f'(x_0)(x - x_0) + \frac{1}{2}f''(\xi)(x - x_0)^2, \quad \xi \text{在} x, x_0 \text{之间}
        \]
        代入 $x = a, x = b$ 有:
        \begin{align*}
            & \frac{1}{2}f''(\xi_1)(a - x_0)^2 = f(a) - f(x_0) - f'(x_0)(a - x_0) & a < \xi_1 < x_0 \\
            & \frac{1}{2}f''(\xi_2)(x_0 - b)^2 = f(b) - f(x_0) - f'(x_0)(b - x_0) & x_0 < \xi_2 < b
        \end{align*}
        相加有
        \begin{align*}
            \left\lvert f''(\xi_1)\right\rvert  + \left\lvert f''(\xi_2)\right\rvert  =\  & 2\left\lvert f(x_0)\right\rvert \left(\frac{1}{(a - x_0)^2} + \frac{1}{(b - x_0)^2}\right) \\
            \geqslant \ & 2\left\lvert f(x_0)\right\rvert\left(\frac{(1 + 1)^3}{(a - x_0 + x_0 - b)^2}\right)\\
            = \ & \frac{16}{(a - b)^2}\left\lvert f(x_0)\right\rvert
        \end{align*}
        而
        \begin{align*}
            \max_{a \leqslant x \leqslant b}\left\lvert f''(x)\right\rvert  \geqslant\ & \frac{1}{2}\left(\left\lvert f''(\xi_1)\right\rvert  + \left\lvert f''(\xi_2)\right\rvert\right) \\
            = \ & \frac{8}{(a - b)^2}\left\lvert f(x_0)\right\rvert \\
            = \ & \frac{8}{(a - b)^2}\max_{a \leqslant x \leqslant b}\left\lvert f(x)\right\rvert
        \end{align*}
        即 \[\max_{a \leqslant x \leqslant b}\left\lvert f(x)\right\rvert \leqslant \frac{1}{8}(a - b)^2\max_{a \leqslant x \leqslant b}\left\lvert f''(x)\right\rvert \] \qed
    \end{zmhj}
\end{lt}

\begin{lt}
    $f(x)$ 在 $(-\infty, +\infty)$ 上二阶可导, $\left\lvert f(x)\right\rvert < k_0, \left\lvert f''(x)\right\rvert < k_1$, 证明:
    \[
        \left\lvert f'(x)\right\rvert \leqslant \sqrt{2k_0k_1}
    \]
    \begin{zmhj}
        由 Taylor 公式,
        \begin{align*}
            f(x + h) &= f(x) + f'(x)h + \frac{1}{2}f''(\xi_1)h^2 \\
            f(x - h) &= f(x) - f'(x)h + \frac{1}{2}f''(\xi_2)h^2
        \end{align*}
        化简有:
        \[
            f'(x) = \frac{f(x + h) - f(x - h)}{2h} + \frac{1}{4}\left(f''(\xi_2) - f(\xi_1)\right)h
        \]
        因此
        \[
            \left\lvert f'(x)\right\rvert \leqslant \frac{\left\lvert f(x + h)\right\rvert + \left\lvert f(x - h)\right\rvert}{2h} + \frac{1}{4}\left(\left\lvert f''(\xi_2)\right\rvert  + \left\lvert f''(\xi_1)\right\rvert \right)h \leqslant \frac{k_0}{h} + \frac{h}{2}k_1 \qquad \forall h > 0
        \]
        因此
        \[
            \left\lvert f'(x)\right\rvert \leqslant \min_{h \in (0, + \infty)} \left(\frac{k_0}{h} + \frac{h}{2}k_1\right) = \sqrt{2k_0k_1}
        \]
        \qed
    \end{zmhj}
\end{lt}

\begin{kuang3}
    \subsection{凸函数与 Lipschitz 条件}
\end{kuang3}

\subsubsection{凸函数与二阶导的关系}

\begin{dy}[凸函数]
    对某函数 $f(x)$ 定义域的任意区间 $[a, b]$ ,有任意 $\lambda \in [0, 1]$, 满足
    \[
        f(\lambda x_1 + (1 - \lambda) x_2) \leq \lambda f(x_1) + (1 - \lambda) f(x_2)
    \]
    则成 $f(x)$ 为其定义区间上的\textbf{下凸函数}
\end{dy}

\begin{dl}
    $f(x)$ 在区间 $I$ 上二阶可导,则 $\forall x \in I, f\dds(x) \geqslant 0$ 是 $f(x)$ 在 $I$ 上下凸的充要条件. 
\end{dl}

\begin{zm}
    充分性: \\
    任取 $x_1, x_2 \in I$,不妨设 $x_1 < x_2$。任取 $\lambda \in (0,1)$,根据 Lagrange 中值定理
    \begin{align*}
        & \lambda f(x_1) + (1 - \lambda) f(x_2) - f(\lambda x_1 + (1 - \lambda)x_2) \\
        = & -\lambda \left( f(\lambda x_1 + (1 - \lambda)x_2) - f(x_1) \right) + (1 - \lambda) \left( f(x_2) - f(\lambda x_1 + (1 - \lambda)x_2) \right) \\
        = & -\lambda f'(\xi_1) \cdot (1 - \lambda)(x_2 - x_1) + (1 - \lambda)f'(\xi_2) \cdot \lambda(x_2 - x_1) \quad \\
        & \qquad (x_1 < \xi_1 < \lambda x_1 + (1 - \lambda)x_2 < \xi_2 < x_2) \\
        = & \ \lambda (1 - \lambda) f''(\xi) (x_2 - x_1)(\xi_2 - \xi_1) \quad (\xi_1 \leq \xi \leq \xi_2) \\
        \geqslant & \  0
    \end{align*}
    必要性: 假设 $f(x)$ 是 $I$ 上的下凸函数,且处处二阶可导。取 $x_0 \in I$,则 $\forall \Delta x > 0$,有
    \begin{align*}
        \frac{f(x_0 + \Delta x) + f(x_0 - \Delta x) - 2f(x_0)}{\Delta x^2} \geq 0 
        \qquad \left( x_0 = \frac{1}{2} (x_0 - \Delta x) + \frac{1}{2}(x_0 + \Delta x) \right)
    \end{align*}
    另一方面,根据带 Peano 余项的 Taylor 公式, 
    \begin{align*}
        & \frac{f(x_0 + \Delta x) + f(x_0 - \Delta x) - 2f(x_0)}{\Delta x^2} \\
        = & \ \frac{1}{\Delta x^2} \Bigg( f(x_0) + f'(x_0) \Delta x + \frac{1}{2} f''(x_0) \Delta x^2 + o(\Delta x^2)  \\
        & \  + f(x_0) + f'(x_0)(-\Delta x) + \frac{1}{2} f''(x_0)(-\Delta x)^2 + o(\Delta x^2) - 2f(x_0) \Bigg)\\
        = &\  f''(x_0) + \frac{o(\Delta x^2)}{\Delta x^2} \quad (\Delta x \to 0) \\
        = & \ f(x_0) \\
        \geqslant & \ 0
    \end{align*}
\end{zm}


\subsubsection{Lipschitz 条件}

\begin{dy}[局部 Lipschitz 函数]
    对任意 $x_0 \in D_f$, 存在邻域 $(x_0  - \delta, x_0 + \delta)$, 与常数 $C$ ($\delta, C$ 均依赖于 $x_0$), 使得
    \[
        \forall x, x\ds \in (x_0  - \delta, x_0 + \delta) \cap D_f: \quad \left\lvert f(x) - f(x\ds)\right\rvert \leqslant C\left\lvert x - x\ds\right\rvert  
    \]
    则称 $f(x)$ 为\textbf{局部 Lipschitz 函数}.
\end{dy}

\begin{dy}[Lipschitz 函数]
    若存在常数 $C$, 使得
    \[
        \forall x, x\ds \in D_f: \quad \left\lvert f(x) - f(x\ds)\right\rvert \leqslant C\left\lvert x - x\ds\right\rvert
    \]
    则称 $f(x)$ 为\textbf{Lipschitz 函数}.
\end{dy}

又由限覆盖定理,闭区间上的局部 Lipschitz 函数是Lipschitz 函数.

\begin{dl}
    Lipschitz 函数是连续函数.  
\end{dl}

\subsubsection{开区间上的凸函数}

\begin{yl}\label{凸函数另一定义}
    $f(x)$ 是 $(a, b)$ 上的下凸函数当且仅当任意 $(x_1, x_2) \subseteq (a, b)$, 任意 $x \in (x_1, x_2)$, 有
    \[
        \frac{f(x) - f(x_1)}{x - x_1} \leqslant \frac{f(x_2) - f(x_1)}{x_2 - x_1} \leqslant \frac{f(x_2) - f(x)}{x_2 - x}
    \]
\end{yl}

\begin{zm}
    由 $x = \frac{x_2 - x}{x_2 - x_1}x_1 + \frac{x - x_1}{x_2 - x_1}x_2$, 有
    \begin{align*}
        f(x) \text{是下凸函数} \Leftrightarrow\quad 
        & f(x) \leqslant \frac{x_2 - x}{x_2 - x_1}f(x_1) + \frac{x - x_1}{x_2 - x_1}f(x_2)\quad (\forall a < x_1 < x < x_2 < b)\\
        \Leftrightarrow\quad & \frac{f(x) - f(x_1)}{x - x_1} \leqslant \frac{f(x_2) - f(x_1)}{x_2 - x_1}
    \end{align*}
    另一边同理.
\end{zm}

\begin{yl}
    闭区间上的下凸函数有界.
\end{yl}

\begin{zm}
    设闭区间 $[a, b]$, 任取 $x \in [a, b]$, $\exists \lambda \in (0, 1), x = \lambda a + (1 - \lambda) b, f(x) \leqslant \lambda f(a) + (1 - \lambda )f(b) \leqslant \max\{f(a), f(b)\}$. \\
    又将 $f(x)$ 连续延拓到 $[c, d] \supseteq [a, b]$ 上且保证其在 $[c, d]$ 上下凸,有 $f(x)$ 在 $[c, d]$ 上有上界. \\
    由于 $f(a) \leqslant \lambda_1f(x) + (1 - \lambda_1)f(c)$, 有 $f(x) \geqslant \frac{1}{\lambda_1}f(a) - \left(\frac{1}{\lambda_1} - 1\right)f(c)$. \\
    同理 $f(x) \geqslant \frac{1}{\lambda_2}f(b) - \left(\frac{1}{\lambda_2} - 1\right)f(d)$. 即 $f(x)$ 有下界.
\end{zm}

\begin{dl}
    开区间上的凸函数必为连续函数. 
\end{dl}

\begin{zm}
    对任意 $\varepsilon > 0$, 任意 $x_0 \in (a, b)$, 存在 $\delta > 0, [x_0 - 2\delta, x_0 + 2\delta] \subseteq (a, b)$. \\
    任取 $x > y \in (x_0 - \delta, x_0 + \delta)$, $x = \lambda (x_0 + 2\delta) + (1 - \lambda)y, \lambda \in (0, 1)$. \\
    由凸函数性质, $f(x) \leqslant \lambda f(x_0 + 2\delta) + (1 - \lambda)f(y) \Leftrightarrow f(x) - f(y)\leqslant \lambda[f(x_0 + 2\delta) - f(y)]$. \\
    由\textcolor{shuang}{\textbf{引理 3.3.2}}, $\left\lvert f(x)\right\rvert < M, \exists M > 0$. 故 $\left\lvert f(y) - f(x)\right\rvert \leqslant \lambda[f(x_0 + 2\delta) - f(y)] \leqslant 2\lambda M$. \\
    又由 $x = \lambda (x_0 + 2\delta) + (1 - \lambda)y \Rightarrow \left\lvert x - y\right\rvert = \lambda(x_0 + 2\delta - y) > \lambda\delta$, 故 $2\lambda M < \frac{\left\lvert x - y\right\rvert}{\delta}$. \\
    代入有 $\left\lvert f(x) - f(y)\right\rvert < \frac{\left\lvert x - y\right\rvert }{\delta}$, 满足 \textbf{局部 Lipschitz 性质},因此 $f(x)$ 连续.
\end{zm}


\begin{dl}
    开区间上的凸函数必为 Lipschitz 函数.
\end{dl}

\begin{lt}
    设函数 $f(x)$ 是 $(a, b)$ 上的下凸函数,证明:
    \begin{enumerate}
        \item [(1)] 在每个 $x \in (a, b)$,$f'_-(x)$ 与 $f'_+(x)$ 均存在,且 $f'_-(x) \leqslant f'_+(x)$.
        \item [(2)] 若 $a < x_1 < x_2 < b$,则有 $f'_+(x_1) \leqslant f'_-(x_2)$.
        \item [(3)] $f(x)$ 不可导的点至多有可列个.
    \end{enumerate}
    \begin{zmhj} \quad
        \begin{enumerate}
            \item [(1)] 取 $x_0 \in (a, b)$, 由\textbf{引理\ref{凸函数另一定义}}, $\frac{f(x_0) - f(x_0 - \Delta x)}{\Delta x} \geqslant \frac{f(x_0) - f(x_0 - 2 \Delta x_0)}{2\Delta x}$, 表明 $F_-(\Delta x) = \frac{f(x_0) - f(x_0 - \Delta x)}{\Delta x}$ 关于 $\Delta x$ 单减. \\
            又取定 $x' > x_0, F_-(\Delta x) \leqslant \frac{f(x') - f(x_0)}{x' - x_0}$, $F_-(\Delta x)$ 在 $\Delta x \to 0^+$ 时单调递增且有上界,故极限存在,即 $f\ds_-(x_0) = \lim_{\Delta x \to 0}F_-(\Delta x)$ 存在. 同理 $f\ds_+(x_0)$ 也存在.\\
            $f\ds_-(x_0) = \lim_{x \to x_0^-}\frac{f(x_0) - f(x)}{x_0 - x} \leqslant \lim_{x \to x_0^+}\frac{f(x) - f(x_0)}{x - x_0} = f\ds_+(x_0)$ \quad(由\textbf{引理\ref{凸函数另一定义}}) 
            \item [(2)] 任取 $x \in (x_1, x_2)$, \\
            有 $f'_+(x_1) = \lim_{x \to x_1^+}\frac{f(x) - f(x_1)}{x - x_1} \leqslant \frac{f(x) - f(x_1)}{x - x_1} \leqslant \frac{f(x_2) - f(x)}{x_2 - x} \leqslant \lim_{x \to x_2^-}\frac{f(x_2) - f(x)}{x_2 - x} = f'_-(x_2)$
            \item [(3)] 由 $(2)$ 知,如果存在一点 $x_0 \in (a, b)$, 使得 $f'_-(x_0) < f'_+(x_0)$, 则开区间 $(f'_-(x_0), f'_+(x_0))$ 中不含 $f'_-(x), f'_+(x), \ \forall x \in (a, b)$ 的所有值.|\\
            假设有两点 $x_1 < x_2$, 有 $f'_-(x_i) < f'_+(x_i), \ i \in \{1, 2\}$. 则易得 $f'_-(x_1) < f'_+(x_1) \leqslant f'_-(x_2) < f'_+(x_2)$. \\
            由\textbf{引理\ref{单调函数的不连续点必然是跳跃间断点}}, 同理有这样的 $x_0$ 有可列个. 
        \end{enumerate}
    \end{zmhj}
\end{lt}

\subsubsection{相关不等式}

\begin{dl}[Young 不等式]
    $p, q$ 不等于 $0$ 或 $1$, $\frac{1}{p} + \frac{1}{q} = 1$, 则对任意正数 $a, b$, 有
    \[
        ab \leqslant \frac{1}{p}a^p + \frac{1}{q}b^q  \quad  p > 1
    \]
    \[
        ab \geqslant \frac{1}{p}a^p + \frac{1}{q}b^q  \quad  p < 1
    \]
\end{dl}

\begin{zm}
    求导即可.
\end{zm}

\begin{dl}[H$\ddot{\mathrm{o}}$lder 不等式]
    $p, q$ 不等于 $0$ 或 $1$, $\frac{1}{p} + \frac{1}{q} = 1$, 则对任意正数数列 $\{a_k\}_{k = 1}^{n}, \{b_k\}_{k = 1}^{n}$, 有
    \[
        \sum_{i = 1}^n a_ib_i \leqslant \left(\sum_{i = 1}^{n}a_i^p\right)^{\frac{1}{p}} \cdot \left(\sum_{i = 1}^{n}b_i^q\right)^{\frac{1}{q}}  \quad p > 1
    \]
    \[
        \sum_{i = 1}^n a_ib_i \geqslant \left(\sum_{i = 1}^{n}a_i^p\right)^{\frac{1}{p}} \cdot \left(\sum_{i = 1}^{n}b_i^q\right)^{\frac{1}{q}}  \quad p < 1
    \]
\end{dl}

\begin{zm}
    令 $A = \sum_{i = 1}^{n}a_i^p, B = \sum_{i = 1}^{n}b_i^q$ \\
    $p > 1$ 时,对 $\forall i \in \{1, 2, \cdots , n\}$, 由 \textbf{Young 不等式},
    \[
        \frac{a_ib_i}{A^{\frac{1}{p}}B^{\frac{1}{q}}} \leqslant \frac{1}{p}\frac{a_i^p}{A} + \frac{1}{q}\frac{b_i^p}{B}
    \]
    故 
    \[
        \frac{1}{A^{\frac{1}{p}}B^{\frac{1}{q}}}\sum_{i = 1}^{n}a_ib_i \leqslant \frac{1}{p} + \frac{1}{q} = 1
    \]
    即
    \[
        \sum_{i = 1}^{n}a_ib_i \leqslant A^{\frac{1}{p}}B^{\frac{1}{q}} = \left(\sum_{i = 1}^{n}a_i^p\right)^{\frac{1}{p}} \cdot \left(\sum_{i = 1}^{n}b_i^q\right)^{\frac{1}{q}}
    \]
\end{zm}

\begin{zm}[用 Jesen 不等式的证明]
    $p > 1$ 时, $f(x) = x^p$ 为下凸函数. \\
    由于 $q = \frac{p}{p - 1}$, 把 $x_iy_i$ 分解为
    \[
        x_iy_i = y_i^q\left(\sum_{i = 1}^{n}y_i^q\right)^{-1} \cdot x_iy_i^{1 - q}\left(\sum_{i = 1}^{n}y_i^q\right)
    \]
    注意到
    \[
        \sum_{i = 1}^{n}\left(y_i^q\left(\sum_{i = 1}^{n}y_i^q\right)^{-1}\right) = 1
    \]
    由 \textbf{Jesen 不等式},
    \begin{align*}
        \left(\sum_{i = 1}^{n}x_iy_i\right)^p\  =\ & \left\{\sum_{i = 1}^{n}\left[y_i^q\left(\sum_{i = 1}^{n}y_i^q\right)^{-1} \cdot x_iy_i^{1 - q}\left(\sum_{i = 1}^{n}y_i^q\right)\right]\right\}^p \\
        \leqslant\ & \sum_{i = 1}^{n}y_i^q\left(\sum_{i = 1}^{n}y_i^q\right)^{-1} \cdot x_i^p y_i^{p - pq}\left(\sum_{i = 1}^{n}y_i^q\right)^p \\
        = \ & \left(\sum_{i = 1}^{n}x_i^p\right)\cdot \left(\sum_{i = 1}^{n}y_i^q\right)^{p - 1} \\
    \end{align*}
    因此,
    \begin{align*}
        \sum_{i = 1}^{n}x_iy_i =  \left(\sum_{i = 1}^{n}a_i^p\right)^{\frac{1}{p}} \cdot \left(\sum_{i = 1}^{n}b_i^q\right)^{\frac{1}{q}}
    \end{align*}
\end{zm}

\begin{kuang2}
    \section{函数不定积分、定积分相关定理}
\end{kuang2}

\begin{kuang3}
    \subsection{不定积分}
\end{kuang3}

(全是计算题)
\subsubsection{计算计巧}

\begin{lt}[凑微分技巧] \quad
    \begin{itemize}
        \item $\frac{x \dx}{\sqrt{1 + x^2}} = \dd \left(\sqrt{1 + x^2}\right)$
        \item $\int x^{2k + 1}\sqrt{a + bx^2} \dx \xlongequal[]{t = \sqrt{a + bx^2}} \cdots, (b x \dx = t \dd t)$
    \end{itemize}
\end{lt}

\begin{lt}[分部积分法则] \quad
    \begin{itemize}
        \item $\int f(x) \ln g(x) \dx = \int \ln g(x) \dd F(x)$
        \item $\int f(x) \arctan g(x) \dx$ 分两种情况. \begin{itemize}
            \item $f(x)$ 能凑微分,则化为 $\int \arctan g(x) \dd F(x)$
            \item $f(x)$ 不能凑微分,则换元 $t = \arctan g(x)$
        \end{itemize} 
    \end{itemize}
\end{lt}

\begin{dl}[用于求分段函数积分]
    $$
        \int f(x) \dx = \int_{a}^{x} f(t) \dd t
    $$
\end{dl}

\begin{dl}[记不住就完了!] \quad
    \begin{itemize}
        \item 
        \[
            \int \sqrt{a^2 \pm x^2} \dx = \frac{x\sqrt{x^2 \pm a^2}}{2} \pm \frac{a^2}{2} \ln\left\lvert x + \sqrt{x^2 \pm a^2}\right\lvert + C
        \]
        \item $
            I_n = \int \frac{\dx}{\left(x^2 + a^2\right)^n}
        $ \\ 则
        \[
            I_n = 
            \left\{
                \begin{matrix}
                    \frac{1}{a}\arctan \frac{x}{a} + C &\qquad n = 1 \\
                    \frac{2n - 3}{2a^2(n - 1)}I_{n - 1} + \frac{1}{2a^2(n - 1)}\frac{x}{\left(x^2 + a^2\right)^{n - 1}} & \qquad n \geqslant 2
                \end{matrix}
            \right.
        \]
    \end{itemize}
    
\end{dl}

\subsubsection{一些例题}

\begin{lt}[正余弦齐次分式函数]
    计算 $\int \frac{7 \sin x + \cos x}{3 \sin x + 4 \cos x} \dx$.
    \begin{jiehj}
        由于 $7 \sin x + \cos x = 3 \sin x + 4 \cos x - \left(3 \sin x + 4 \cos x\right)'$, 有 
        \[
            \int \frac{7 \sin x + \cos x}{3 \sin x + 4 \cos x} \dx = \int \dx - \int \frac{\dd \left(3 \sin x + 4 \cos x\right)}{3 \sin x + 4 \cos x} = x - \ln \left|3 \sin x + 4 \cos x \right| + C
        \]
    \end{jiehj}
\end{lt}

\begin{kuang3}
    \subsection{定积分}
\end{kuang3}

\subsubsection{定积分的定义与 Darboux 和}

\begin{dy}[Riemann 和与定积分]
    (略)
\end{dy}

\begin{dy}[Darboux 和]
    (略)
\end{dy}

\subsubsection{定积分存在的充要条件 ---- Darboux 定理}

\begin{yl}[Darboux 引理]
    对于 [a, b] 上的有界函数 f(x),成立
    \[
        \lim_{\lambda(P) \to 0} \overline{S}(P) = \overline{\int_a^b} f(x) \dx, \quad \lim_{\lambda(P) \to 0} \underline{S}(P) = \underline{\int_a^b} f(x) \dx.
    \]
    确切地说,对任意 $\varepsilon > 0$,存在 $\delta > 0$,使得对任意划分 P,只要 $0 < \lambda(P) < \delta$,则有
    $$
        \left|\overline{S}(P) - \overline{\int_a^b} f(x)  \dx\right| < \varepsilon,\quad     \left|\underline{S}(P) - \underline{\int_a^b} f(x)  \dx\right| < \varepsilon
    $$ 
\end{yl}

\begin{dl}[可积的充要条件, Darboux 定理]
    设 $f(x)$ 在 $[a, b]$ 上有界,则以下命题等价:
    \begin{enumerate}
        \item $f(x)$ 在 $[a, b]$ 上 Riemann 可积;
        \item $\overline{\int_a^b} f(x) \dx = \underline{\int_a^b} f(x) \dx$;
        \item $\lim_{\lambda(P) \to 0} \left( \overline{S}(P) - \underline{S}(P) \right) = 0$;
        \item 存在一列划分 $\{P_l\}_{l=1}^{\infty}$,使得 $\lim_{l \to \infty} \left( \overline{S}(P_l) - \underline{S}(P_l) \right) = 0$.
        \item [4'.] 对任意 $\varepsilon > 0$,存在划分 $P$,使得 $\overline{S}(P) - \underline{S}(P) < \varepsilon$。
    \end{enumerate}
    最后,若 $f(x)$ 可积,则成立

    $$
        \int_a^b f(x) \dx = \overline{\int_a^b} f(x) \dx = \underline{\int_a^b} f(x) \dx.
    $$ 
\end{dl}

\subsubsection{定积分存在的充要条件 ---- Lebesgue 判别法}

先给出一个较弱的定理
\begin{dl}
    闭区间上的连读函数可积.
\end{dl}

\begin{zm}
    由一致连续性即得.
\end{zm}

\begin{dy}[区间的测度与零测集]
    定义开、闭区间 $I = (a, b)$ 的长度 $\left\lvert I\right\rvert = b - a$ \\
    设集合 $S \subseteq \mathbb{R}$。如果对于任意 $\varepsilon > 0$,存在至多可列个开区间 $\{I_n\}_{n=1}^{\infty}$,使得
    $$ S \subset \bigcup_{n=1}^{\infty} I_n \quad \text{且} \quad \sum_{n=1}^{\infty} |I_n| < \varepsilon, $$ 
    则称 $S$ 是一个零测集。
\end{dy}

\begin{lt}
    证明: $\RR$ 中的可列集是零测集. 
    \begin{zmhj}
        设 $S \subset \mathbb{R}$ 是可列集,记
        $$ S = \{x_1, x_2, x_3, \ldots, x_n, \ldots\}. $$ 
        任取 $\varepsilon > 0$,对每个 $n$,记
        $$ I_n = \left(x_n - \frac{\varepsilon}{2^{n+2}}, x_n + \frac{\varepsilon}{2^{n+2}}\right). $$ 
        则显然
        $$ S \subset \bigcup_{n=1}^{\infty} I_n, $$ 
        且
        $$ \sum_{n=1}^{\infty} |I_n| = \sum_{n=1}^{\infty} \frac{\varepsilon}{2^{n+1}} = \varepsilon. $$ 
        所以,$S$ 是零测集。
    \end{zmhj}
\end{lt}

\textbf{注:}如果在某区间上,某数学性质在一个零测集之外均成立,则称该区间上该性质\textbf{几乎处处成立}. \\

再给出一道相关例题 

\begin{lt}\label{可积当且仅当对任意}
    证明:对任意有界函数 $f(x)$, $f(x)$ 可积当且仅当对任意 $\varepsilon, \sigma > 0$, 存在一划分 $P$, 使得
    \[
        \sum_{\omega_i \geqslant \varepsilon}\omega_i \Delta x_i < \sigma 
    \]
    \begin{zmhj}
        $f(x)$ 可积等价于 $\lim_{\lambda(P) \to 0}\sum_{i = 1}^{p}w_i\Delta x_i = 0$. 令 $\max_{x \in [a, b]} f(x) = M, \min_{x \in [a, b]} f(x) = m$ \\
        \begin{itemize}
            \item 充分性: \\
            对任意 $\varepsilon > 0$, 存在 $\delta = \sqrt{\frac{\varepsilon}{2(p - 1)}}, \sigma = \frac{\varepsilon}{2(p - 1)(M - m)}, \varepsilon_0 = \sqrt{\frac{\varepsilon}{2(p - 1)}}$ 取分划 $P$ 满足 $\lambda(P) < \delta$, 有
            \begin{align*}
                \sum_{i = 1}^{p}w_i\Delta x_i = \sum_{w_i \geqslant \varepsilon_0}w_i\Delta x_i + \sum_{w_i < \varepsilon_0}w_i\Delta x_i < (p - 1)(M - m)\sigma + (p - 1)\varepsilon_0\delta = \varepsilon
            \end{align*}
            即 $\lim_{\lambda(P) \to 0}\sum_{i = 1}^{p}w_i\Delta x_i = 0$, $f(x)$ 可积.
            \item 必要性:\\
            取定 $\varepsilon_0 > 0$, 存在 $\sigma = \min \{b - a, \sqrt{\varepsilon_0}\}, \varepsilon = \frac{\varepsilon_0}{\sigma}$.假设对任意分划 $P$ 满足 $\sum_{w_i \geqslant \varepsilon}\Delta x_i \geqslant \sigma$, 有
            \begin{align*}
                \sum_{i = 1}^{p}w_i\Delta x_i = \sum_{w_i \geqslant \varepsilon_0}w_i\Delta x_i + \sum_{w_i < \varepsilon_0}w_i\Delta x_i \geqslant \varepsilon \sigma + 0 = \varepsilon_0
            \end{align*}
            表明 $f(x)$ 不可积,矛盾!因此对任意 $\varepsilon > 0, \sigma > 0$, 存在划分 $P$, $\sum_{w_i \geqslant \varepsilon}\Delta x_i < \sigma$
        \end{itemize}
    \end{zmhj}
\end{lt}

\begin{dl}[Lebesgue 判别法]
    $f(x)$ 是 $[a, b]$ 上的有界函数, $f(x)$ 在 $[a, b]$ 上可积的充要条件是 $f(x)$ 的间断点集是一个可列集. ($f(x)$ 几乎处处连续)
\end{dl}

\begin{zm}
    首先我们定义符号:对于任意 $x \in D_f$, 
    \[
        \omega_f(x, \varepsilon) := \sup_{x_1, x_2 \in (x - \varepsilon, x + \varepsilon)}\left\lvert f(x_1) - f(x_2)\right\rvert 
    \]
    以及由于 $\omega_f(x, \varepsilon)$ 关于 $\varepsilon$ 单增由下界,由单调有界定理,当 $\varepsilon \to 0^+$ 的极限存在. 因此定义
    \[
        \omega_f(x) := \lim_{\varepsilon \to 0^+}\omega_f(x, \varepsilon)
    \]
    因此我们易得如下引理:
    \begin{yl}
        $f(x)$ 在 $x = x_0$ 处连续当且仅当 $\omega_f(x_0) = 0$.
    \end{yl}
    \textbf{充分性}: \\
    $f(x)$ 在 $[a, b]$ 上有界,记 $\left\lvert f(x)\right\rvert \leqslant M$.  \\
    设 $K \subseteq [a, b]$ 是 $f(x)$ 所有间断点的集合,则 $K$ 为零测集. \\
    任取 $\varepsilon > 0$.则由零测集的定义,存在至多可列个开区间 $\{I_k\}_{k=1}^{\infty}$,使得
    \[
        K \subset \bigcup_{k=1}^{\infty} I_k, \quad \sum_{k=1}^{\infty} |I_k| < \frac{\varepsilon}{4M}. 
    \]
    另一方面,对任意 $x \in [a, b] \setminus K$,由于 $x$ 是连续点,所以有
    \[ 
        \omega_f(x) = \lim_{\delta \to 0^+} \omega_f(x, \delta) = 0. 
    \] 
    由极限的定义,存在 $\delta_x > 0$,使得
    \[ 
        \omega_f(x, 2\delta_x) := \sup_{x', x'' \in (x-2\delta_x, x+2\delta_x) \cap [a, b]} |f(x') - f(x'')| < \frac{\varepsilon}{2(b-a)}
    \] 
    显然,$\{(x - \delta_x, x + \delta_x)\}_{x \in [a, b] \setminus K} \cup \{I_k\}_{k=1}^{\infty}$ 是 $[a, b]$ 的一个开覆盖,因为它分别地覆盖了 $[a, b] \setminus K$ 与 $K$ 中的每一个点.根据有限覆盖定理,它有有限子覆盖,记为
    $$ [a, b] \subset \bigcup_{j=1}^{N} (y_j - \delta_{y_j}, y_j + \delta_{y_j}) \cup \bigcup_{l=1}^{M} I_{k_l}. $$ 
    该子覆盖中所有开区间的端点构成了 $[a, b]$ 的一个划分 $P : a = x_0 < x_1 < x_2 < \cdots < x_n = b$ ($[a, b]$ 之外的端点忽略).该划分所决定的小区间 $[x_{i-1}, x_i]$ 可以分为两类:
    \begin{itemize}
        \item [\uppercase\expandafter{\romannumeral1}.] $(x_{i-1}, x_i)$ 包含于某个 $I_{k_l}$.则所有这类小区间 $[x_{i-1}, x_i]$ 的长度之和不超过所有 $I_{k_l}$ 的长度之和,从而小于 $\frac{\varepsilon}{4M}$. 记这些 $i$ 构成的集合为 $\Lambda_1$
        \item [\uppercase\expandafter{\romannumeral2}.] $(x_{i-1}, x_i)$ 不包含于任何 $I_{k_l}$.根据覆盖关系,可以通过左端点 $x_{i-1}$ 的位置讨论 $[x_{i-1}, x_i]$ 的位置:
        \begin{itemize}
            \item $x_{i-1}$ 属于某个 $I_{k_l}$.由于 $x_i$ 是一切 $x_{i-1}$ 右侧端点中距离 $x_{i-1}$ 最近的那一个,所以它不比 $I_{k_l}$ 的右端点更靠右.此时 $(x_{i-1}, x_i) \subset I_{k_l}$.因此该情形不会发生.
            \item $x_{i-1}$ 不属于任意 $I_{k_l}$,根据覆盖,它必属于某个 $(y_j - \delta_{y_j}, y_j + \delta_{y_j})$.利用类似于上一种情形的讨论,必有
            $$ (x_{i-1}, x_i) \subset (y_j - \delta_{y_j}, y_j + \delta_{y_j}). $$ 
            从而
            $$ [x_{i-1}, x_i] \subset (y_j - 2\delta_{y_j}, y_j + 2\delta_{y_j}). $$ 
        \end{itemize}
        可见,第二类区间上函数的振幅小于 $\frac{\varepsilon}{2(b-a)}$. 记这些 $i$ 构成的集合为 $\Lambda_2$
    \end{itemize}
    最后,对于该划分 $P$
    $$ S(P) - S(P) = \sum_{i \in \Lambda_1} \omega_i \Delta x_i + \sum_{i \in \Lambda_2} \omega_i \Delta x_i < 2M \cdot \frac{\varepsilon}{4M} + \frac{\varepsilon}{2(b-a)} \cdot (b-a) = \varepsilon. $$
    \textbf{必要性}: \\
    先证明一个引理:
    \begin{yl}
        如果 $f(x)$ 在 $[a, b]$ 上可积,则对任意 $\sigma > 0$, 集合
        \[
            \left\{x \in [a, b]\ |\ \omega_f(x) > \sigma\right\}
        \]
        是零测集.
    \end{yl}
    由 \textbf{例题 \ref{可积当且仅当对任意}} 立得. \\
    注意到
    \begin{align*}
        x \in \left\{x \in [a, b]\ | \ \omega_f(x) > 0\right\} &\Leftrightarrow \exists n \in \NN, \ \omega_f(x) > \frac{1}{n} \\
        &\Leftrightarrow x \in \bigcup_{n = 1}^{+\infty}\left\{x \in [a, b]\ \left| \ \omega_f(x) > \frac{1}{n}\right.\right\}  
    \end{align*}
    因此 
    \[
        \left\{x \in [a, b]\ | \ \omega_f(x) > 0\right\} = \bigcup_{n = 1}^{+\infty}\left\{x \in [a, b]\ \left| \ \omega_f(x) > \frac{1}{n}\right.\right\} 
    \]
    即 $\left\{x \in [a, b]\ | \ \omega_f(x) > 0\right\}$ 是可列集.
\end{zm}

\begin{lt}
    闭区间上的单调有界函数可积.
    \begin{zmhj}
        由 \textbf{定理 \ref{单调函数至多有可列个间断点}} 立得.
    \end{zmhj}
\end{lt}

\subsubsection{积分中值定理与微积分基本定理}

\begin{dl}[积分第一中值定理]
    设 $f(x), g(x)$ 为 $[a, b]$ 上的可积函数,且 $g(x)$ 在 $[a, b]$ 上不变号. 记 $M = \sup_{x \in [a, b]} f(x), m = \inf_{x \in [a, b]}f(x)$, 则存在 $\eta \in [m, M]$, 使得.
    \[
        \int_{a}^{b}f(x)g(x)\dx = \eta \int_{a}^{b}g(x)\dx
    \]
    特别地,当 $f(x)$ 在 $[a, b]$ 上连续,则存在 $\xi \in [a, b]$, 
    \[
        \int_{a}^{b}f(x)g(x)\dx = f(\xi) \int_{a}^{b}g(x)\dx
    \]
\end{dl}

\begin{dy}
    有界函数 $f(x)$ 在 $[a, b]$ 上可积,则对任意 $x \in [a, b]$, $F(x) = \int_{a}^{x}f(x)\dx$ 为 $f$ 的变上限积分.
\end{dy}

关于 $F(x)$ 有性质:

\begin{dl}\quad
    \begin{itemize}
        \item $F(x)$ 有 Lipschitz 性质.
        \item 若 $f(x)$ 连续, $F(x)$ 可导,且 $F'(x) = f(x)$
    \end{itemize}
\end{dl}

\begin{zm}\quad
    \begin{itemize}
        \item 由 $\left\lvert f(x)\right\rvert \leqslant M$,
        \begin{align*}
            \left\lvert f(x) - f(y)\right\rvert =& \left\lvert \int_{x}^{y}f(x)\dx\right\rvert \\
            \leqslant& \int_{x}^{y}\left\lvert f(x)\right\rvert \dx \\
            \leqslant& \int_{x}^{y} M \dx = M\left\lvert y - x\right\rvert 
        \end{align*}
        即证.
        \item 考虑 $x_0 \in (a, b)$ 的情形. 任取 $\varepsilon > 0$. 根据 $f(x)$ 在 $x_0$ 的连续性,存在 $\delta > 0$,使得
        $$ \forall x \in (x_0 - \delta, x_0 + \delta) : |f(x) - f(x_0)| < \varepsilon. $$ 
        对任意 $\Delta x$,根据定积分的区间可加性与第一中值定理:
        $$ \frac{F(x_0 + \Delta x) - F(x_0)}{\Delta x} = \frac{1}{\Delta x} \int_{x_0}^{x_0 + \Delta x} f(t) \, dt = f(\xi), $$ 
        其中 $\xi \in [x_0, x_0 + \Delta x]$ 或 $\xi \in [x_0 + \Delta x, x_0]$. 所以,只要 $0 < |\Delta x| < \delta$,就有
        $$ \left| \frac{F(x_0 + \Delta x) - F(x_0)}{\Delta x} - f(x_0) \right| = |f(\xi) - f(x_0)| < \varepsilon. $$ 
        由定义
        $$ F'(x_0) = \lim_{\Delta x \to 0} \frac{F(x_0 + \Delta x) - F(x_0)}{\Delta x} = f(x_0). $$ 
    \end{itemize}
\end{zm}

\begin{dl}[微积分第一基本定理]
    若 $f(x)$ 是 $[a, b]$ 上的连续函数,则其变上限积分 $F(x) = \int_{a}^{x}f(x)\dx$ 是 $f(x)$ 的一个原函数.
\end{dl}

\begin{dl}[微积分第二基本定理, Newton-Leibniz 公式]\label{微积分第二基本定理}
    设 $f(x)$ 在 $[a, b]$ 上连续, $F(x)$ 是 $f(x)$ 的一个原函数,则
    \[
        \int_{a}^{b}f(x)\dx = F(b) - F(a) = F(x) \bigg|_a^b
    \]
\end{dl}

\textbf{定理 \ref{微积分第二基本定理}}可以加强为:

\begin{dl}
    设函数 $F(x)$ 在 $[a, b]$ 上连续,在 $(a, b)$ 上可导。函数 $f(x)$ 在 $[a, b]$ 上可积,且有 $F'(x) = f(x)$, $x \in (a, b)$。那么 
    \[ 
        \int_a^b f(x) \dx = F(b) - F(a)
    \]
\end{dl}

\end{document}