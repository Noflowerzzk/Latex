\documentclass{ctexart}
\usepackage{newtxtext, amsmath, amssymb}
\usepackage[left=0.5in, right=0.5in, top=0.9in, bottom=0.9in]{geometry}
\usepackage{amsthm} % 设置定理环境,要在amsmath后用
\usepackage{xpatch} % 用于去除amsthm定义的编号后面的点
\usepackage[strict]{changepage} % 提供一个 adjustwidth 环境
\usepackage{multicol} % 用于分栏
\usepackage[fontsize=14pt]{fontsize} % 字号设置
\usepackage{esvect} % 实现向量箭头输入(显示效果好于\vec{}和\overrightarrow{}),格式为\vv{⟨向量符号⟩}或\vv*{⟨向量符号⟩}{⟨下标⟩} 
\usepackage[
            colorlinks, % 超链接以颜色来标识,而并非使用默认的方框来标识
            linkcolor=mlv,
            anchorcolor=mlv,
            citecolor=mlv % 设置各种超链接的颜色
            ]
            {hyperref} % 实现引用超链接功能
\usepackage{tcolorbox} % 盒子效果
\usepackage{hyperref}
\usepackage{fancyhdr}
\usepackage{rotating}
\usepackage{bm} % 加载 bm 包
\usepackage{circuitikz}
\tcbuselibrary{most} % tcolorbox宏包的设置,详见宏包说明文档
\everymath{\displaystyle}  % 让所有数学模式都使用 \displaystyle


\geometry{a4paper,centering,scale=0.8}

% 保证 leftmark 始终为 subsection 名称
\newcommand{\chaptermark}{\markboth{}{}}
\renewcommand{\sectionmark}[1]{\markboth{}{}}
\renewcommand{\subsectionmark}[1]{\markboth{#1}{}}

\pagestyle{fancy}%清除原页眉页脚样式
\fancyhf{}

\fancyhead[L]{\leftmark}
\fancyhead[R]{\thepage}

% ————————————————————————————————————自定义符号————————————————————————————————————
\newcommand{\。}{.} % 示例:将指令\。定义为全角句点。
\newcommand{\zte}{\mathrm{e}} % 正体e
\newcommand{\ztd}{\mathrm{d}} % 正体d
\newcommand{\zti}{\mathrm{i}} % 正体i
\newcommand{\ztj}{\mathrm{j}} % 正体j
\newcommand{\ztk}{\mathrm{k}} % 正体k
\newcommand{\CC}{\mathbb{C}} % 黑板粗体C
\newcommand{\QQ}{\mathbb{Q}} % 黑板粗体Q
\newcommand{\RR}{\mathbb{R}} % 黑板粗体R
\newcommand{\ZZ}{\mathbb{Z}} % 黑板粗体Z
\newcommand{\NN}{\mathbb{N}} % 黑板粗体N
\newcommand{\ds}{^\prime}
\newcommand{\dds}{^{\prime\prime}}
\newcommand{\tr}{\mathrm{tr}}
\newcommand{\rr}{\mathrm{r}}
\newcommand{\T}{\mathrm{T}}
\newcommand{\A}{\bm{A}}
\newcommand{\B}{\bm{B}}
\newcommand{\C}{\bm{C}}
\newcommand{\D}{\bm{D}}
\newcommand{\E}{\bm{E}}
\newcommand{\OO}{\bm{O}}
\newcommand{\al}{\bm{\alpha}}
\newcommand{\bt}{\bm{\beta}}
\newcommand{\diag}{\mathrm{diag}}
\newcommand{\rdots}{\begin{rotate}{80}$\ddots$\end{rotate}}
\newcommand{\xxxg}{\textbf{线性相关}}
\newcommand{\xxwg}{\textbf{线性无关}}
% ————————————————————————————————————自定义符号————————————————————————————————————

% ————————————————————————————————————自定义颜色————————————————————————————————————
\definecolor{mlv}{RGB}{40, 137, 124} % 墨绿
\definecolor{qlv}{RGB}{240, 251, 248} % 浅绿
\definecolor{slv}{RGB}{33, 96, 92} % 深绿
\definecolor{qlan}{RGB}{239, 249, 251} % 浅蓝
\definecolor{slan}{RGB}{3, 92, 127} % 深蓝
\definecolor{hlan}{RGB}{82, 137, 168} % 灰蓝
\definecolor{qhuang}{RGB}{246, 250, 235} % 浅黄
\definecolor{shuang}{RGB}{77, 82, 59} % 深黄
\definecolor{qzi}{RGB}{240, 243, 252} % 浅紫
\definecolor{szi}{RGB}{49, 55, 70} % 深紫
% 将RGB换为rgb,颜色数值取值范围改为0到1
% ————————————————————————————————————自定义颜色————————————————————————————————————

% ————————————————————————————————————盒子设置————————————————————————————————————
% tolorbox提供了tcolorbox环境,其格式如下:
% 第一种格式:\begin{tcolorbox}[colback=⟨背景色⟩, colframe=⟨框线色⟩, arc=⟨转角弧度半径⟩, boxrule=⟨框线粗⟩]   \end{tcolorbox}
% 其中设置arc=0mm可得到直角;boxrule可换为toprule/bottomrule/leftrule/rightrule可分别设置对应边宽度,但是设置为0mm时仍有细边,若要绘制单边框线推荐使用第二种格式
% 方括号内加上title=⟨标题⟩, titlerule=⟨标题背景线粗⟩, colbacktitle=⟨标题背景线色⟩可为盒子加上标题及其背景线
\newenvironment{kuang1}{
    \begin{tcolorbox}[colback=hlan, colframe=hlan, arc = 1mm]
    }
    {\end{tcolorbox}}
\newenvironment{kuang2}{
    \begin{tcolorbox}[colback=hlan!5!white, colframe=hlan, arc = 1mm]
    }
    {\end{tcolorbox}}
% 第二种格式:\begin{tcolorbox}[enhanced, colback=⟨背景色⟩, boxrule=0pt, frame hidden, borderline={⟨框线粗⟩}{⟨偏移量⟩}{⟨框线色⟩}]   {\end{tcolorbox}}
% 将borderline换为borderline east/borderline west/borderline north/borderline south可分别为四边添加框线,同一边可以添加多条
% 偏移量为正值时,框线向盒子内部移动相应距离,负值反之
\newenvironment{kuang3}{
    \begin{tcolorbox}[enhanced, breakable, colback=hlan!5!white, boxrule=0pt, frame hidden,
        borderline south={0.5mm}{0.1mm}{hlan}]
    }
    {\end{tcolorbox}}
\newenvironment{lvse}{
    \begin{tcolorbox}[enhanced, breakable, colback=qlv, boxrule=0pt, frame hidden,
        borderline west={0.7mm}{0.1mm}{slv}]
    }
    {\end{tcolorbox}}
\newenvironment{lanse}{
    \begin{tcolorbox}[enhanced, breakable, colback=qlan, boxrule=0pt, frame hidden,
        borderline west={0.7mm}{0.1mm}{slan}]
    }
    {\end{tcolorbox}}
\newenvironment{huangse}{
    \begin{tcolorbox}[enhanced, breakable, colback=qhuang, boxrule=0pt, frame hidden,
        borderline west={0.7mm}{0.1mm}{shuang}]
    }
    {\end{tcolorbox}}
\newenvironment{zise}{
    \begin{tcolorbox}[enhanced, breakable, colback=qzi, boxrule=0pt, frame hidden,
        borderline west={0.7mm}{0.1mm}{szi}]
    }
    {\end{tcolorbox}}
% tcolorbox宏包还提供了\tcbox指令,用于生成行内盒子,可制作高光效果
\newcommand{\hl}[1]{
    \tcbox[on line, arc=0pt, colback=hlan!5!white, colframe=hlan!5!white, boxsep=1pt, left=1pt, right=1pt, top=1.5pt, bottom=1.5pt, boxrule=0pt]
{\bfseries \color{hlan}#1}}
% 其中on line将盒子放置在本行(缺失会跳到下一行),boxsep用于控制文本内容和边框的距离,left、right、top、bottom则分别在boxsep的参数的基础上分别控制四边距离
% ————————————————————————————————————盒子设置————————————————————————————————————

% ————————————————————————————————————自定义字体设置————————————————————————————————————
\setCJKfamilyfont{xbsong}{方正小标宋简体}
\newcommand{\xbsong}{\CJKfamily{xbsong}}
% CTeX宏集还预定义了\songti、\heiti、\fangsong、\kaishu、\lishu、\youyuan、\yahei、\pingfang等字体命令
% 由于未知原因,一些设备可能无法调用这些字体,故此文档暂时未使用
% ————————————————————————————————————自定义字体设置————————————————————————————————————

% ————————————————————————————————————各级标题设置————————————————————————————————————
\ctexset{
    % 修改 section。
    section={   
    % 设置标题编号前后的词语,使用name={⟨前部分⟩,⟨后部分⟩}参数进行设置。    
        name={\textbf{第},\textbf{章}\hspace{18pt}},
    % 使用number参数设置标题编号,\arabic设置为阿拉伯数字,\chinese设置为中文,\roman设置为小写罗马字母,\Roman设置为大写罗马字母,\alph设置为小写英文,\\bm{A}lph设置为大写英文。
        number={\textbf{\chinese{section}}},
    % 参数format设置标题整体的样式。包括标题主题、编号以及编号前后的词语。
    % 参数format还可以设置标题的对齐方式。居中对齐\centering,左对齐\raggedright,右对齐\hfill
        format=\color{hlan}\centering\zihao{2}, % 设置 section 标题为正蓝色、黑体、居中对齐、小二号字
    % 取消编号后的空白。编号后有一段空白,参数aftername=\hspace{0pt}可以用来控制编号与标题之间的距离。
        aftername={}
    },
    % 修改 subsection。
    subsection={   
        name={\S \hspace{6pt}, \hspace{8pt}},
        number={\arabic{section}.\arabic{subsection}},
        format=\color{hlan}\centering\zihao{-2}, % 设置 subsection 标题为黑体、三号字
        aftername=\hspace{0pt}
    },
    subsubsection={   
        name={,、},
        number={\chinese{subsubsection}},
        format=\raggedright\color{hlan}\zihao{3},
        aftername=\hspace{0pt}
    },
    part={   
        name={第,部分},
        number={\chinese{part}},
        format=\color{white}\centering\bfseries\zihao{-1},
        aftername=\hspace{1em}
    }
}
% 各个标题设置的各行结尾一定要记得写逗号!!!!!!!!!!!!!!!!!!!!!!!!!!!!!!!!!!!!!!!!!!
% ————————————————————————————————————各级标题设置————————————————————————————————————

% ————————————————————————————————————定理类环境设置————————————————————————————————————
\newtheoremstyle{t}{0pt}{}{}{\parindent}{\bfseries}{}{1em}{} % 定义新定理风格。格式如下:
%\newtheoremstyle{⟨风格名⟩}
%                {⟨上方间距⟩} % 若留空,则使用默认值
%                {⟨下方间距⟩} % 若留空,则使用默认值
%                {⟨主体字体⟩} % 如 \itshape
%                {⟨缩进长度⟩} % 若留空,则无缩进;可以使用 \parindent 进行正常段落缩进
%                {⟨定理头字体⟩} % 如 \bfseries
%                {⟨定理头后的标点符号⟩} % 如点号、冒号
%                {⟨定理头后的间距⟩} % 不可留空,若设置为 { },则表示正常词间间距;若设置为 {\newline},则环境内容开启新行
%                {⟨定理头格式指定⟩} % 一般留空
% 定理风格决定着由 \newtheorem 定义的环境的具体格式,有三种定理风格是预定义的,它们分别是:
% plain: 环境内容使用意大利斜体,环境上下方添加额外间距
% definition: 环境内容使用罗马正体,环境上下方添加额外间距
% remark: 环境内容使用罗马正体,环境上下方不添加额外间距
\theoremstyle{t} % 设置定理风格
\newtheorem{dyhj}{\color{slv} 定义}[subsection] % 定义定义环境,格式为\newtheorem{⟨环境名⟩}{⟨定理头文本⟩}[⟨上级计数器⟩]或\newtheorem{⟨环境名⟩}[⟨共享计数器⟩]{⟨定理头文本⟩},其变体\newtheorem*不带编号
\newtheorem{dlhj}{\color{shuang} 定理}[subsection]
\newtheorem{lthj}{\color{szi} 例}[subsection]
\newtheorem*{jiehj}{\color{slan} 解}
\newtheorem*{zmhj}{\color{slan} 证明}
\newtheorem{ylhj}{\color{shuang} 引理}[subsection]
\newtheorem*{ylzmhj}{\color{slan} 引理的证明}
\newenvironment{dy}{\begin{lvse}\begin{dyhj}}{\end{dyhj}\end{lvse}}
\newenvironment{zm}{\begin{lanse}\begin{zmhj}}{$\hfill \square$\end{zmhj}\end{lanse}}
\newenvironment{dl}{\begin{huangse}\begin{dlhj}}{\end{dlhj}\end{huangse}}
\newenvironment{lt}{\begin{zise}\begin{lthj}}{\end{lthj}\end{zise}}
\newenvironment{yl}{\begin{huangse}\begin{ylhj}}{\end{ylhj}\end{huangse}}
% ————————————————————————————————————定理类环境设置————————————————————————————————————

% ————————————————————————————————————目录设置————————————————————————————————————
\setcounter{tocdepth}{4} % 设置在 ToC 的显示的章节深度
\setcounter{secnumdepth}{3} % 设置章节的编号深度
% 数值可选选项:-1 part 0 chapter 1 section 2 subsection 3 subsubsection 4 paragraph 5 subparagraph
\renewcommand{\contentsname}{\bfseries 目\hspace{1em}录}
% ————————————————————————————————————目录设置————————————————————————————————————

\everymath{\displaystyle} % 设置所有数学公式显示为行间公式的样式

\begin{document}

\quad\\
{
\Huge \bfseries 
\begin{kuang3}
    \color{hlan}
    \centering
    线代例题整理
\end{kuang3}
}

\thispagestyle{empty} % 取消这一页的页码
\newpage

% ————————————————————————————————————页码设置————————————————————————————————————
\pagenumbering{Roman}
\setcounter{page}{1}

\begin{multicols}{2} % 设置环境中内容为两栏。数字为栏数
    {
    \tableofcontents
    }
\end{multicols}

\newpage
\setcounter{page}{1}
\pagenumbering{arabic}
% ————————————————————————————————————页码设置————————————————————————————————————


\begin{kuang2}
    \section{定义和关键性质}
\end{kuang2}

\begin{kuang3}
    \subsection{矩阵相关运算}
\end{kuang3}

\subsubsection{矩阵的转置、伴随、迹、行列式、秩}

\begin{itemize}
    \item 对加法:
    \begin{itemize}
        \item $(\bm{A} + \bm{B})^\T = \bm{A}^\T + \bm{B}^\T$
        \item $\tr(\bm{A} + \bm{B}) = \tr(\bm{A}) + \tr(\bm{B})$
    \end{itemize}
    \item 对乘法:
    \begin{itemize}
        \item $(\bm{A}\bm{B})^T = \bm{B}^T\bm{A}^T$
        \item $(\bm{A}\bm{B})^* = \bm{B}^*\bm{A}^*$
        \item $(\bm{A}\bm{B})^{-1} = \bm{B}^{-1}\bm{A}^{-1}$
        \item $\tr(\bm{A}\bm{B}) = \tr(\bm{A})\tr(\bm{B})$
        \item $\left\lvert \bm{A}\bm{B}\right\rvert  = \left\lvert \bm{A}\right\rvert \left\lvert \bm{B}\right\rvert $
        \item $\bm{P}, \bm{Q}$ 可逆, $\rr(\A) = \rr(\bm{P}\A) = \rr(\A\bm{Q}) = \rr(\bm{P}\A\bm{Q})$
        \item $\A_{m \times n}, \B_{n \times s}$, $\rr(\A) + \rr(\B) - n \leqslant \rr(\A\B) \leqslant\min\left\{\rr(\A), \rr(\B)\right\}$
    \end{itemize}
    \item 组合:
    \begin{itemize}
        \item $\left\lvert \bm{A}^\T\right\rvert = \left\lvert \A\right\rvert $
        \item $\left\lvert \bm{A}\right\rvert \bm{A}^{-1} = \bm{A}^* \Leftrightarrow \bm{A}^{-1} = \frac{\A^*}{\left\lvert \A\right\rvert }$
        \item $\left\lvert \A^*\right\rvert = \left\lvert \A\right\rvert^{n - 1} $
        \item $\left\lvert \A^{-1}\right\rvert =\frac{1}{\left\lvert \A\right\rvert} $
        \item $\left(\A^*\right)^* = \left\lvert \A\right\rvert^{n - 2}\A $
    \end{itemize}
\end{itemize}

\subsubsection{分块矩阵}

\begin{itemize}
    \item 分块对角阵:
    \begin{itemize}
        \item $\A = \diag(\A_1\ \A_2\ \cdots\ \A_n) \Rightarrow \A^{-1} = \diag(\A_1^{-1}\ \A_2^{-1}\ \cdots\ \A_n^{-1})$
        \item $
            \A = \begin{pmatrix}
                & & & \A_1 \\
                & & \A_2 & \\
                & \rdots  & & \\
                \A_n & & &
            \end{pmatrix} \Rightarrow
            \A^{-1} = \begin{pmatrix}
                & & & \A_n^{-1} \\
                & & \rdots & \\
                & \A_2^{-1}  & & \\
                \A_1^{-1} & & &
            \end{pmatrix}
        $
        \item $
            \A = \begin{pmatrix}
                \B & \D \\
                \OO & \C
            \end{pmatrix}
        $, $\B, \C$ 可逆,则 \[
            \A^{-1} = \begin{pmatrix}
                \B^{-1} & -\B^{-1}\D\C^{-1} \\
                \OO & \C^{-1}
            \end{pmatrix}
        \]
    \end{itemize}
    \item 分块矩阵的秩:
    \begin{itemize}
        \item $\rr \begin{pmatrix}
            \A & \OO \\
            \OO & \B
        \end{pmatrix} = \rr(\A) + \rr(\B)
        $
        \item $\rr \begin{pmatrix}
            \A & \OO \\
            \C & \B
        \end{pmatrix} \geqslant \rr(\A) + \rr(\B)
        $
        \item $\max\left\{\rr(\A), \rr(\B)\right\} \leqslant \rr\left(\A \ \B\right) \leqslant \rr(\A) + \rr(\B)$
    \end{itemize}
\end{itemize}

\subsubsection{线性相关与极大线性无关组}

\begin{itemize}
    \item 线性相关:
    \begin{itemize}
        \item $\al\ \xxxg \Leftrightarrow \al = \bm{0}$
        \item 向量组里有 $\bm{0}$, 则必然 \xxxg
        \item 两个向量 \xxxg $\Leftrightarrow$ 二者对应成比例
        \item 部分 \xxxg $\Rightarrow$ 整体 \xxxg; 整体 \xxwg $\Rightarrow$ 部分 \xxwg
        \item $\left(\al_1\ \al_2\ \cdots\ \al_n \right)\bm{X} = \bm{0}$ 有非零解 $\Leftrightarrow \rr\left(\al_1\ \al_2\ \cdots\ \al_n\right) < n \Leftrightarrow$ $\al_1\ \al_2\ \cdots\ \al_n$ \xxxg \\
        $\left(\al_1\ \al_2\ \cdots\ \al_n \right)\bm{X} = \bm{0}$ 只有零解 $\Leftrightarrow \rr\left(\al_1\ \al_2\ \cdots\ \al_n\right) = n \Leftrightarrow$ $\al_1\ \al_2\ \cdots\ \al_n$ \xxwg
        \item $n$ 个 $m$ 维向量,$m < n$ 时 \xxxg \\
        \xxwg 的 $m$ 维向量最多有 $n$ 个
        \item $\begin{pmatrix}
            \al_i \\
            \bt_i
        \end{pmatrix}$ \xxxg $\Rightarrow \al_i$ \xxxg \\
        $\al_i$ \xxwg $\Rightarrow \begin{pmatrix}
            \al_i \\
            \bt_i
        \end{pmatrix}$ \xxwg
    \end{itemize}
\end{itemize}

\begin{kuang2}
    \section{例题们}
\end{kuang2}
dfdf
$\bm{A}\bm{x} = \bm{\beta}\Omega $
$R_1 = 2  \Omega$

\[\left\{
\begin{array}{cccccccc}
    +i_a &- i_{CF} &- i_{CD} & & & & & = 0\\
    +i_a & & &- i_3 & -i_{ED} & & & = 0 \\
     & & & & +i_{ED} & -i_{FE} & -i_c & = 0 \\
     &+i_{CF} & & & & -i_{FE} & -i_c & = 0 \\
    (R_1 + R_2)i_a & & & R_3i_3 & & & & \ \ = u_s \\
     & -R_4i_{CF} & & R_3i_3 & & -R_5i_{FE} & & = 0 \\
     & & & & & R_5i_{FE} & (R_6 + R_7)i_c & = 0
\end{array}\right.
\]


\end{document}