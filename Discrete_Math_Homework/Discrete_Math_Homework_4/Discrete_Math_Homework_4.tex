\documentclass{article}
\usepackage{amsmath}  % 数学符号包
\usepackage{amssymb}  % 更多数学符号
\usepackage{enumitem} % 列表样式
\usepackage{fancyhdr} % 页眉设置
\usepackage{geometry} % 页面设置
% \usepackage[UTF8]{ctex}
\usepackage{bm}
\usepackage{amsthm}


\geometry{a4paper, margin=1in}


\pagestyle{fancy}
\fancyhf{}
\fancyhead[C]{Discrete Math Homework 4}
\fancyhead[R]{2024.10.11}


\title{Discrete Math Homework 4}
\author{noflowerzzk}
\date{2024.10.11}


\begin{document}
\maketitle

\section{Answer\_1}

These are first order logic propositions: a), f), i), j).

\section{Anewer\_2}

a), b), d), e), f).

\section{Answer\_3}

a).

\section{Answer\_4}

c), d), e)

\section{Anewer\_5}

\begin{itemize}
    \item[a)]
    \begin{proof}
        $
            [\![ \forall x \exists y R(x, y) ]\!]_\mathcal{J} = \mathbf{T}
        $
        iff. for all $a \in \mathbb{N}$, exists $a + 1 \in \mathbb{N}$, Let $\mathcal{J^\prime} = \mathcal{J}_1 [x \mapsto a][y \mapsto a + 1]$ 
        \[
            [\![ R(x, y) ]\!]_{\mathcal{J}_1 [x \mapsto a][y \mapsto a + 1]} = \mathcal{J}^\prime (R)(\mathcal{J}^\prime(x), \mathcal{J}^\prime(y)) = \mathcal{J}^\prime(R)(a, a + 1) 
        \]
        i.e. $ a < a + 1 $ and it's obviously true.
    \end{proof}
    \item[b)]
    \begin{proof}
         $ [\![ \exists y R(x, y) ]\!]_{\mathcal{J}_2} = \mathbf{T} $ 
        iff. exists $b \in \mathbb{N}$, 
        $$
            [\![ \exists y R(x, y) ]\!]_{\mathcal{J}_2[y \mapsto b]} = 
            \mathcal{J}_2[y \mapsto b] (R)(\mathcal{J}_2[y \mapsto b](x), \mathcal{J}_2[y \mapsto b](y)) = \mathcal{J}_2[y \mapsto b] (R)(0, b) = \mathbf{T}
        $$
        But it's obvious that for all $b \in N$, $b \geqslant 0$, $\mathcal{J}_2[y \mapsto b] (R)(0, b) = \mathbf{F}$. So $ [\![ \exists y R(x, y) ]\!]_{\mathcal{J}_2} = \mathbf{F} $ 
        \qedhere
    \end{proof}
    \item[c)]
    \begin{proof}
        To proof $[\![ \forall x \exists y R(x, y) ]\!]_{\mathcal{J}_3} = \mathbb{F}$, we only need to give a counterexample. \\
        Noting that if we let $\mathcal{J}^\prime = \mathcal{J}_3[x \mapsto 0]$, \\
        The proposition $ [\![ \forall x \exists y R(x, y) ]\!]_{\mathcal{J}^\prime} $ is the same as the one in problem b). \\
        So for all  $a \in \mathbb{N}$, s.t. $\mathcal{J}^\prime = \mathcal{J}_3[y \mapsto a]$, 
        $$ [\![ \forall x \exists y R(x, y) ]\!]_{\mathcal{J}^\prime} = \mathcal{J}^\prime(R)(\mathcal{J}^\prime(x), \mathcal{J}^\prime(y)) = \mathcal{J}^\prime(R)(0, a) = \mathbf{F}$$ \\
        That indicates that $ [\![ \forall x \exists y R(x, y) ]\!]_{\mathcal{J}_3} = \mathbf{F} $
    \end{proof}
\end{itemize}

\end{document}