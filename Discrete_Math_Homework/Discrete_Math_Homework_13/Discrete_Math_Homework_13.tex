\documentclass{article}
\usepackage{amsmath}  % 数学符号包
\usepackage{amssymb}  % 更多数学符号
\usepackage{enumitem} % 列表样式
\usepackage{fancyhdr} % 页眉设置
\usepackage{geometry} % 页面设置
% \usepackage[UTF8]{ctex}
\usepackage{bm}
\usepackage{amsthm}
\everymath{\displaystyle}  % 让所有数学模式都使用 \displaystyle
\newcommand{\lb}{\left\llbracket}
\newcommand{\rb}{\right\rrbracket}
\newcommand{\RR}{\mathbb{R}}
\newcommand{\NN}{\mathbb{N}}
\newcommand{\QQ}{\mathbb{Q}}

\geometry{a4paper, margin=1in}


\pagestyle{fancy}
\fancyhf{}
\fancyhead[C]{Dscrete Math Homework 13}
\fancyhead[R]{2024.11.28}


\title{Dscrete Math Homework 13}
\author{noflowerzzk}
\date{2024.11.28}


\begin{document}
\maketitle

\section{}

\begin{proof}\quad
    \begin{itemize}
        \item Noiting that $0 = \varnothing$, so $\forall x(x\subseteq A\wedge\neg x=\varnothing\rightarrow\exists y(y\in x \wedge\forall z(z\in x\to y\in z\lor y=z)))\qquad (*)$ is always true. So $0$ is $\in$-well-ordered.
        \item If $x \subseteq n$, then according to the condition, $x$ satisfies the expression $(*)$. \\
        If $x = x' \cup \{n\}$, where $x' \subseteq n$, let $y$ be the $\in$-least element of $x'$, so $y \in x' \subseteq n \Rightarrow y \in \{n\}$. And for any $z \in x'$, it holds that $y \in z \lor y = z$. \\
        In summary, $n \cup \{n\}$ is also $\in$-well-ordered.
    \end{itemize} 
\end{proof}

\section{}

\begin{proof}
    Obviously $\mathrm{Inducive}(u) \rightarrow u \subseteq v$, $\mathrm{Inducive}(u) \rightarrow v \subseteq u$. \\
    So $\forall a(a \in v \rightarrow a \in u), \forall a(a \in v \rightarrow a \in u)$, i.e. $u = v$.
\end{proof}

\section{}

\begin{proof}\quad
    \begin{itemize}
        \item Noting that $\varnothing \in u, \varnothing \in v$, so $varnothing \in u \cap v$. 
        \item $\forall a \in u, a \cup \{a\} \in u, \forall a \in v, a \cup \{a\} \in u$, so $\forall a \in u \cap v$, we have $a \cup \{a\} \in u, a \cup \{a\} \in v$, i.e. $a \cup \{a\} \in u \cap v$.
    \end{itemize}
    So  $u \cap v$ is alse inducive.
\end{proof}

\section{}

Let $V$ denote $ \left\{x \in u\ | \ \forall v (v \subseteq u \land \mathrm{Inducive}(v)\rightarrow x \in v)\right\}$

\begin{itemize}
    \item \begin{proof}
        Given $v \subseteq u \land \mathrm{Inducive}(v)$, we know that $\forall x \in V, x \in v$. Noting that $v$ is inducive, $x \cup \{x\} \in v$. Due to the arbitrariness of $v$, $x \cup \{x\} \in V$. And obviously $\varnothing \in V$. So $V = \left\{x \in u\ | \ \forall v (v \subseteq u \land \mathrm{Inducive}(v)\rightarrow x \in v)\right\}$ is also inducive. 
    \end{proof}
    \item \begin{proof}
        If there exists an inducive subset $v_0$ of $u$ that $v_0 \subseteq V$, we prove that $v_0 = V$. \\
        According to the definition of $V$, for any $x \in V$, due to $v_0$ is inducive, $x \in v_0$, so $V \subseteq v_0$. So $V = v_0$, $V$ is the smallest inducive subset of $u$.
    \end{proof}
\end{itemize}

\section{}

\begin{proof}
    Let $K$ denote the set of all inducive sets, $T_u$ denote the smallest inducive subseteq of the inducive set $u$. \\
    Then we prove that $\bigcap_{u \in K}T_u$ is the smallest inducive set. \\
    According to \textbf{3}, $\bigcap_{u \in K}T_u$ is an inducive set. \\
    For any inducive set $u' \subseteq \bigcap_{u \in K}T_u, \mathrm{Inducive}(u')$. Noting that $u' \in K$, so $T_{u'} \subseteq u'$, so $\bigcap_{u \in K}T_u \subseteq u'$, i.e. $u' = \bigcap_{u \in K}T_u$. \\
    So $\bigcap_{u \in K}T_u$ is the smallest inducive set.
\end{proof}

\section{}

\begin{proof}\quad
    \begin{itemize}
        \item [a)] Noting that $0 = \varnothing \in X$, so $\varnothing = 0 \in \NN \cap X$. \\
        Also, for any $n \in \NN \cap X$, $n \cup \{n\} \in X$, so $\NN \cap X$ is an inducive set. 
        \item [b)] Noting that $\NN \cap X \subseteq \NN$, $\NN \cap X$ is an inducive set and $\NN$ is the smallest inducive set, we have $\NN = \NN \cap X$, i.e. $\forall n \in \NN$, $n \in X$ always holds.        
    \end{itemize}
\end{proof}

\end{document}