\documentclass{article}
\usepackage{amsmath}  % 数学符号包
\usepackage{amssymb}  % 更多数学符号
\usepackage{enumitem} % 列表样式
\usepackage{fancyhdr} % 页眉设置
\usepackage{geometry} % 页面设置
% \usepackage[UTF8]{ctex}
\usepackage{bm}
\usepackage{amsthm}


\geometry{a4paper, margin=1in}


\pagestyle{fancy}
\fancyhf{}
\fancyhead[C]{Discrete Math Homework 5}
\fancyhead[R]{2024.10.23}


\title{Discrete Math Homework 5}
\author{noflowerzzk}
\date{2024.10.23}


\begin{document}

\maketitle

\section{}

\begin{proof}
    For all $\mathcal{J}$ is a $\mathcal{S}$-interpretation, satisfies 
    $[\![\exists x \forall y (R(x, y))]\!]_\mathcal{J} = \mathbf{T}$. So exists an $a$ in $\mathcal{J}$'s domain,  for all $b$ in $\mathcal{J}$'s domain, $[\![R(x, y)]\!]_{\mathcal{J}[x \mapsto a][y \mapsto b]} = \mathbf{T}$. \\
    Then let $a = b$, we have $[\![R(x, y)]\!]_{\mathcal{J}[x \mapsto a][y \mapsto a]} = \mathcal{J}(R)(a, a) = \mathbf{T}$. That means that $[\![R(x, x)]\!]_{\mathcal{J}[x \mapsto a]} = \mathbf{T}$.
    So for all $\mathcal{J}$, $[\![\exists x(R(x, x))]\!]_{\mathcal{J}} = \mathbf{T}$, i.e. 
    \[
        \exists x \forall y (R(x, y)) \models \exists x(R(x, x))
    \]
\end{proof}

\end{document}