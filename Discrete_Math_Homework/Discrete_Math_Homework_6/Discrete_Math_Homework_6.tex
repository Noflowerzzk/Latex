\documentclass{article}
\usepackage{amsmath}  % 数学符号包
\usepackage{amssymb}  % 更多数学符号
\usepackage{enumitem} % 列表样式
\usepackage{fancyhdr} % 页眉设置
\usepackage{geometry} % 页面设置
% \usepackage[UTF8]{ctex}
\usepackage{bm}
\usepackage{amsthm}
\usepackage{stmaryrd}


\geometry{a4paper, margin=1in}


\pagestyle{fancy}
\fancyhf{}
\fancyhead[C]{Discrete Math Homework 6}
\fancyhead[R]{2024.10.27}


\title{Discrete Math Homework 6}
\author{noflowerzzk}
\date{2024.10.27}


\begin{document}
\maketitle

\section{}

\begin{proof}
    Let $\mathcal{J}, \text{where} \mathcal{J}(p) = \mathbf{T}, \mathcal{J}(q) = \mathbf{F}, \mathcal{J}(r) = \mathbf{F} $, we have 
    $$
        [\![ (p \land q) \rightarrow r ]\!]_\mathcal{J} = \mathbf{T}, 
        [\![(p \rightarrow r) \land (q \rightarrow r)]\!]_\mathcal{J} = \mathbf{F}
    $$
    That means that they are not equivalent.
\end{proof}

\section{}
They are not logically equivalent.
\begin{proof}
    Let $\mathcal{J}$ is a $S$-interpretation, where: \\
    \begin{itemize}
        \item[-] $\mathcal{J}$'s domain is $\mathbb{Z}$, \\
        \item[-] $\mathcal{J}(P)(x) = \mathbf{T}$ iff. $x$ is an even number, \\
        \item[-] $\mathcal{J}(Q)(x) = \mathbf{T}$ iff. $x$ is an odd number. \\
    
    \end{itemize}
    Then for all $a \in \mathbb{Z}$, $ [\![\forall x(P(x) \rightarrow Q(x))]\!]_\mathcal{J} = \mathbf{T} $ iff. 
    $$ \mathcal{J}(P)(a) \rightarrow \mathcal{J}(Q)(a) = \mathbf{T} $$ (for all $a \in \mathbb{Z}$). \\
    Let $a = 2$, $\mathcal{J}(P)(a) = \mathbf{T}$, and $\mathcal{J}(Q)(a) = \mathbf{F}$, which means that $ [\![\forall x(P(x) \rightarrow Q(x))]\!]_\mathcal{J} = \mathbf{F} $.\\
    However, $[\![\forall x (P(x))]\!]_\mathcal{J} = \mathcal{J}(P)(a) = \mathbf{F}$, $[\![\forall x (Q(x))]\!]_\mathcal{J} = \mathcal{J}(Q)(a) = \mathbf{T}$, ($a = 2$).\\
    So $[\![\forall x (P(x)) \rightarrow \forall x (Q(x))]\!]_\mathcal{J} = \mathbf{T} \neq [\![\forall x(P(x) \rightarrow Q(x))]\!]_\mathcal{J}$. \\
    So they are not equivalent.
\end{proof} 

\section{}
\begin{proof}
    For all $\mathcal{J}$ is a $S$-interpretation, $\mathcal{J}$'s domain is $A$.  \\
    Then 
    $ [\![\exists x (P(x) \rightarrow \forall y P(y))]\!]_\mathcal{J} = \mathbf{T}$ if forall $a \in A$, $\mathcal{J}^\prime = \mathcal{J}[x \mapsto a][y \mapsto a]$\\
    \[
        [\![(P(x) \rightarrow  QPy))]\!]_{\mathcal{J}^\prime} = \mathcal{J}^\prime(P)(a) \rightarrow \mathcal{J}^\prime(P)(a) = \mathcal{J}(P)(a) \rightarrow \mathcal{J}(P)(a) = \mathbf{T}
    \]
    If $\mathcal{J}(P)(a) = \mathbf{T}$, then $\mathcal{J}(P)(a) \rightarrow \mathcal{J}(P)(a) = \mathbf{T}$. \\
    If $\mathcal{J}(P)(a) = \mathbf{F}$, then $\mathcal{J}(P)(a) \rightarrow \mathcal{J}(P)(a) = \mathbf{T}$. \\
    So $\exists x (P(x) \rightarrow \forall y P(y))$ is true on any interpretation.
\end{proof}

\end{document}