\documentclass{article}
\usepackage{amsmath}  % 数学符号包
\usepackage{amssymb}  % 更多数学符号
\usepackage{enumitem} % 列表样式
\usepackage{fancyhdr} % 页眉设置
\usepackage{geometry} % 页面设置
% \usepackage[UTF8]{ctex}
\usepackage{bm}
\usepackage{amsthm}
\usepackage{amssymb}
\usepackage{booktabs}

\geometry{a4paper, margin=1in}


\pagestyle{fancy}
\fancyhf{}
\fancyhead[C]{Discrete Math Homework 2}
\fancyhead[R]{2024.9.28}


\title{Discrete Math Homework 2}
\author{Noflowerzzk}
\date{2024.9.28}


\begin{document}

\maketitle

\section{Answer\_1}

\renewcommand{\arraystretch}{1} % 行高设为1以消除空白

\begin{itemize}
    \item[a)] 
        \[
            \begin{array}{c|c|c|c|c|c|c}
                \toprule
                p & q & \neg p & \neg q & \neg p \lor q & \neg q \lor p & (\neg p \lor q) \land (\neg q \lor p) \\
                \hline
                T & T & F & F & T & T & T \\
                T & F & F & T & F & T & F \\
                F & T & T & F & T & F & F \\
                F & F & T & T & T & T & T \\
                \bottomrule
            \end{array}
        \]
    \item[b)]
        \[
            \begin{array}{c|c|c|c|c|c|c}
                \toprule
                p & q & \neg p & \neg q & \neg p \lor q & \neg q \lor p & (\neg p \lor q) \lor (\neg q \lor p) \\
                \hline
                T & T & F & F & T & T & T \\
                T & F & F & T & F & T & T \\
                F & T & T & F & T & F & T \\
                F & F & T & T & T & T & T \\
                \bottomrule
            \end{array}
        \]
    \item[c)]
        \[
            \begin{array}{c|c|c|c|c}
                \toprule
                p & q & r & q \lor r & p \land (q \lor r) \\
                \hline
                T & T & T & T & T \\
                T & T & F & T & T \\
                T & F & T & T & T \\
                T & F & F & F & F \\
                F & T & T & T & F \\
                F & T & F & T & F \\
                F & F & T & T & F \\
                F & F & F & F & F \\
                \bottomrule
            \end{array}
        \]
    \item[d)]
        \[
            \begin{array}{c|c|c|c|c|c}
                \toprule
                p & q & r & p \land q & p \land r & (p \land q) \lor (p \land r) \\
                \hline
                T & T & T & T & T & T \\
                T & T & F & T & F & T \\
                T & F & T & F & T & T \\
                T & F & F & F & F & F \\
                F & T & T & F & F & F \\
                F & T & F & F & F & F \\
                F & F & T & F & F & F \\
                F & F & F & F & F & F \\
                \bottomrule
            \end{array}
        \]
    \item[e)]
        \[
            \begin{array}{c|c|c|c}
                \toprule
                p & q & p \land q & \neg(p \land q) \\
                \hline
                T & T & T & F \\
                T & F & F & T \\
                F & T & F & T \\
                F & F & F & T \\
                \bottomrule
            \end{array}
        \]
    \item[f)]
        \[
            \begin{array}{c|c|c|c|c}
                \toprule
                p & q & \neg p & \neg q & \neg p \lor \neg q \\
                \hline
                T & T & F & F & F \\
                T & F & F & T & T \\
                F & T & T & F & T \\
                F & F & T & T & T \\
                \bottomrule
            \end{array}
        \]
    \item[g)]
        \[
            \begin{array}{c|c|c|c|c|c|c}
                \toprule
                p & q & p \land q & \neg p & \neg q & \neg p \land \neg q & (p \land q) \lor (\neg p \land \neg q) \\
                \hline
                T & T & T & F & F & F & T \\
                T & F & F & F & T & F & F \\
                F & T & F & T & F & F & F \\
                F & F & F & T & T & T & T \\
                \bottomrule
            \end{array}
        \]
    Obviously, $p \land (q \lor r) \equiv (p \land q) \lor (p \land r)$, $\lnot(p \land q) \equiv \lnot p \lor \lnot q$
\end{itemize}

\section{Answer\_2}

\begin{proof}
    Writing the truth table:
    \[
        \begin{array}{c|c|c|c|c|c|c|c}
            \toprule
            p & q & r & \lnot p & \lnot q & p \lor q & \lnot p \lor r & \lnot q \lor r \\
            \hline
            T & T & T & F & F & T & T & T \\
            T & T & F & F & F & T & F & F \\
            T & F & T & F & T & T & T & T \\
            T & F & F & F & T & T & F & T \\
            F & T & T & T & F & T & T & T \\
            F & T & F & T & F & T & T & F \\
            F & F & T & T & T & F & T & T \\
            F & F & F & T & T & F & T & T \\
            \bottomrule
        \end{array}    
    \]

    Noting that with under truth assignment $\mathcal{J}$ in line 1 and line 5, for any $\phi \in \Phi $, $[\![\phi]\!]_\mathcal{J} = \text{T}$.
    and $\mathcal{J}(r) = \text{T}$. That means $\Phi \models r$.
    \qedhere
\end{proof}

\section{Answer\_3}

\begin{itemize}
    \item[a)]
        \begin{proof}
            $\phi \models \psi$ means that under any trhth assignmeny $\mathcal{J}$, where $[\![\phi]\!]_\mathcal{J} = \text{T}$, \\
            we have $[\![\psi]\!]_\mathcal{J} = \text{T}$. \\
            Assuming a truth $\mathcal{J}$ s.t. $[\![\phi]\!]_\mathcal{J} = \text{T}$, then $[\![\psi]\!]_\mathcal{J} = \text{T}$, so 
            $$ [\![\phi \land \psi]\!]_\mathcal{J} = [\![\land]\!]_\mathcal{J}([\![\phi]\!]_\mathcal{J}, [\![\psi]\!]_\mathcal{J}) = \text{T} = [\![\phi]\!]_\mathcal{J}$$
            Assuming $[\![\phi]\!]_\mathcal{J} = \text{F}$, then
            $$ [\![\phi \land \psi]\!]_\mathcal{J} = [\![\land]\!]_\mathcal{J}([\![\phi]\!]_\mathcal{J}, [\![\psi]\!]_\mathcal{J}) = \text{F} = [\![\phi]\!]_\mathcal{J}$$
            That means $\phi \land \psi \equiv \phi$.
            \\
            Similarly,
            Assuming $[\![\phi]\!]_\mathcal{J} = \text{T}$, then $[\![\psi]\!]_\mathcal{J} = \text{T}$
            $$ [\![\phi \lor \psi]\!]_\mathcal{J} = [\![\lor]\!]_\mathcal{J}([\![\phi]\!]_\mathcal{J}, [\![\psi]\!]_\mathcal{J}) = \text{T} = [\![\psi]\!]_\mathcal{J}$$
            Assuming $[\![\phi]\!]_\mathcal{J} = \text{F}$, then
            $$ [\![\phi \lor \psi]\!]_\mathcal{J} = [\![\lor]\!]_\mathcal{J}([\![\phi]\!]_\mathcal{J}, [\![\psi]\!]_\mathcal{J})  = [\![\psi]\!]_\mathcal{J}$$
            That means $\phi \lor \psi = \psi$.
            \qedhere
        \end{proof}
    \item[b)]
        We will now prove the absorption rule.
        \begin{proof}
            For all $\mathcal{J}$ s.t. $[\![\phi \land \psi]\!]_\mathcal{J} = \text{T}$, we know that $[\![\phi]\!]_\mathcal{J} = \text{T}$, which means that $\phi \land \psi \models \phi$. \\
            So $$ \phi \lor (\phi \land \psi) \equiv (\phi \land \psi) \lor \phi \equiv \phi $$
            \qedhere
        \end{proof}
        \begin{proof}
            For all $\mathcal{J}$ s.t. $[\![\phi]\!]_\mathcal{J} = \text{T}$, we know that $[\![\phi \lor \psi]\!]_\mathcal{J} = \text{T}$, which means that $\phi \models \phi \lor \psi$. \\
            So $$ \phi \land (\phi \lor \psi) \equiv \phi $$
            \qedhere
        \end{proof}
\end{itemize}

\end{document}