\documentclass{article}
\usepackage{amsmath}  % 数学符号包
\usepackage{amssymb}  % 更多数学符号
\usepackage{enumitem} % 列表样式
\usepackage{fancyhdr} % 页眉设置
\usepackage{geometry} % 页面设置
\usepackage[UTF8]{ctex}
\usepackage{bm}
\usepackage{amsthm}
\everymath{\displaystyle}  % 让所有数学模式都使用 \displaystyle
\newcommand{\lb}{\left\llbracket}
\newcommand{\rb}{\right\rrbracket}

\geometry{a4paper, margin=1in}


\pagestyle{fancy}
\fancyhf{}
\fancyhead[C]{思考题二}
\fancyhead[R]{2025.3.2}


\title{思考题二}
\author{Noflowerzzk}
\date{2025.3.2}


\begin{document}
\maketitle

对其中的某个点 $P$, 球心是 $O$, $OP = r$,球半径为 $R$,球的质量面密度为 $\sigma$. 取一个顶点为 $P$ 的对角锥体,其底各自在球面上,沿其底取小圆环,宽度为 $\Delta x$,其轴线为直线 $OP$. 设两个锥体的顶角为 $2\theta$,其中一个母线长为 $d$. \\
则得到上部圆环有几何关系 
   $ \cos \theta = - \frac{d^2 + r^2 - R^2}{2dr} $
又由圆的性质,有 $d d' = (R - r) (R + r)$

上方引力的合力为 
\begin{align*}
    F_{\text{上}} = G\frac{m \cdot \Delta x \cdot 2\pi d\sin \theta \sigma}{d^2}\cos \theta = 
\end{align*}
下方引力的合力为 $F_{\text{下}} = G\frac{m \cdot \Delta x \cdot 2\pi d'\sin \theta \sigma}{d'^2}\cos \theta$

代入计算发现 $F_{\text{上}} = F_{\text{下}}$,即上下两个圆环的引力相等。对球的所有圆环计算,故整个球的引力为 $0$.

\end{document}