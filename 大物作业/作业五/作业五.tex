\documentclass{article}
\usepackage{amsmath}  % 数学符号包
\usepackage{amssymb}  % 更多数学符号
\usepackage{enumitem} % 列表样式
\usepackage{fancyhdr} % 页眉设置
\usepackage{geometry} % 页面设置
\usepackage[UTF8]{ctex}
\usepackage{bm}
\usepackage{amsthm}
\everymath{\displaystyle}  % 让所有数学模式都使用 \displaystyle
\newcommand{\lb}{\left\llbracket}
\newcommand{\rb}{\right\rrbracket}
\newcommand{\dd}{\mathrm{d}}
\newcommand{\kgmps}{\ \mathrm{kg \cdot m / s}}


\geometry{a4paper, margin=1in}


\pagestyle{fancy}
\fancyhf{}
\fancyhead[C]{作业五}
\fancyhead[R]{2025.3.21}


\title{作业五}
\author{Noflowerzzk}
\date{2025.3.21}


\begin{document}
\maketitle

\section*{3 - 2}

\begin{itemize}
    \item [(1)] $p = m \frac{\dd r}{\dd t} = m\omega (-a\sin \omega t \boldsymbol{i} + b \cos \omega t \boldsymbol{j})$
    \item [(2)] $I = \Delta p = 0$. 
\end{itemize}

\section*{3 - 5}

\begin{itemize}
    \item [(1)] $p = \sqrt{p_1^2 + p_2^2} = 1.36 \times 10^{-22} \kgmps$, 与 $p_1$ 夹角为 $152$ °
    \item [(2)] $E_k = \frac{p^2}{2m} = 1.59 \times 10^{-19} \mathrm{J}$
\end{itemize}

\section*{3 - 7}

设空气密度为 $\rho$, 帆面积为 $S$, 时间 $\dd t$ 内, $\dd m = \rho S(v_0 - v)\dd t$, 故 $F = \frac{\dd p}{\dd t} = \rho S (v_0 - v)^2$, 因此 $P = \rho S v (v_0 - v)^2$, 即当 $v = \frac{1}{3}v_0$ 是功率最大.

\section*{3 - 8}

易知 $v_A$ 沿 $AB$ 方向,沿绳方向速度相等, $v_A = v_B\cos \theta$. 设 $B$ 的方向和 $AB$ 夹角为 $\theta$ \\
对 $A$ 分析有 $m_Av_A = I_{AB}$, 对 $B$ 分析有 $m_Bv_B\cos \theta = I\cos \alpha - I_{AB}, m_Bv_B\sin \theta = I\sin \alpha$. \\
由上面的式子解得 $I = \frac{m_B(m_A + m_B)v_B}{\sqrt{m_B^2 \cos^2 \alpha + (m_A + m_B)^2\sin^2 \alpha}}$, $\theta = \arctan \left(\frac{m_A + m_B}{m_B}\tan \alpha\right)$

\section*{3 - 9}

由动量定理及二炮弹在空中飞行时间相同,有 $x_1m + x_2m = x_0(2m)$, 得 $x_2 = \frac{3}{2}x_0$.

\section*{3 - 14}

$k = \frac{mg}{x_0} = 20 \mathrm{N/m}$. 油灰冲撞时动量守恒,有 $(m_1 + m_2)v = m_2\sqrt{2gh}$, 又能量守恒有 $\frac{1}{2}(m_1 + m_2)v^2 + \frac{1}{2}kx_0^2 + (m_1 + m_2)gx = \frac{1}{2}k(x_0 + x)^2$, 解得 $x = 0.3 \mathrm{m}$.

\section*{3 - 15}

\begin{itemize}
    \item [(1)] $A$ 开始运动瞬间速度为 $v_0$, 则 $\frac{1}{2}kx^2 = \frac{1}{2}mv_0^2$, $v_0 = \sqrt{\frac{kx_0^2}{3m}}$. 又分离后 $A, B$ 动量守恒,二块速度相等时速度为 $\frac{1}{4}\sqrt{\frac{kx_0^2}{3m}}$
    \item [(2)] 即 $A, B$ 速度相等时,能量守恒有 $\frac{1}{2}m_2v_0^2 = \frac{1}{2}(m_1 + m_2)v^2 + \frac{1}{2}kx^2$, 得伸长量为 $x = \frac{1}{2}x_0$
\end{itemize}

\section*{补充题1}

\begin{itemize}
    \item [(1)] 能量守恒有 $mgl = \frac{1}{2}Mv_1^2 + \frac{1}{2}mv_2^2$, 水平方向动量守恒有 $Mv_1 = mv_2$ 解得 $v_1 = \sqrt{2\frac{M(M + m)}{m}v_1^2}$. 
    \item [(2)] 能量守恒有 $\frac{1}{2}mgl = \frac{1}{2}Mv_M^2 + \frac{1}{2}m\left(v_M - \frac{1}{2}v_m\right)^2 + \frac{1}{2}m\left(\frac{\sqrt{3}}{2}v_m\right)^2$ 以及同上的水平方向动量守恒,解得 $v_M = m\sqrt{\frac{gl}{4M^2 + 7mM + 3m^2}}$
    \item [(3)] 对木板, $A_M = \frac{1}{2}Mv_1^2$, $A + A_M = 0$, 解得 $A = -\frac{m^2gl}{M + m}$
\end{itemize}

\section*{补充题2}

碰撞瞬间动量守恒,有 $m\sqrt{2gh} = (M_1 + m)v_1$. 此时 $h_1 = -\frac{M_1g}{k}$. 恰脱离时, $h_2 = \frac{M_2g}{k}$. 整个反弹过程能量守恒,有 $\frac{1}{2}(M_1 + m)v_1^2 + (M_1 + m)gh_1 + \frac{1}{2}kh_1^2 = (M_1 + m)gh_2 + \frac{1}{2}kx_2^2$. 联立得 $h = \frac{g}{2k}\frac{(M_1 + m)(M_1 + M_2)(M_1 + M_2 + m)}{m^2}$.

\section*{补充题3}

\begin{itemize}
    \item [(1)] $F = - \frac{\dd E_p}{\dd t} = \frac{V}{r^2}\mathrm{e}^{-\frac{r}{r_0}}$.
    \item [(2)] $r = r_0$ 时 $F = \frac{V}{r_0^2}$. $F = 0.01\frac{V}{r_0^2}$ 时 $r = 2.3 \times 10^{-14} \mathrm{m}$
    \item [(3)] $F = \frac{Vr_0}{r^2}$. 当 $F = \frac{1}{100}F(r_0)$ 时有 $r = 1.5 \times 10^{-15} \mathrm{m}$
\end{itemize}

\end{document}