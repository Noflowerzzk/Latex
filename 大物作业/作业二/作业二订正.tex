\documentclass{article}
\usepackage{amsmath}  % 数学符号包
\usepackage{amssymb}  % 更多数学符号
\usepackage{enumitem} % 列表样式
\usepackage{fancyhdr} % 页眉设置
\usepackage{geometry} % 页面设置
\usepackage[UTF8]{ctex}
\usepackage{bm}
\usepackage{amsthm}
\everymath{\displaystyle}  % 让所有数学模式都使用 \displaystyle
\newcommand{\lb}{\left\llbracket}
\newcommand{\rb}{\right\rrbracket}


\geometry{a4paper, margin=1in}


\pagestyle{fancy}
\fancyhf{}
\fancyhead[C]{作业二订正}
\fancyhead[R]{2025.3.9}


\title{作业二订正}
\author{noflowerzzk}
\date{2025.3.9}


\begin{document}
\maketitle

\section*{1 - 30}

正方形边长为 $a = \frac{vT}{4}$. 后来时,迎风时间 $t_1 = \frac{T}{4(1 - k)}$, 顺风时间 $t_2 = \frac{T}{4(1 + k)}$, 两边时间各为 $t_3 = \frac{T}{4\sqrt{1 - k^2}}$, 因此时间差为 $T\left(\frac{1}{4(1 - k)} + \frac{1}{4(1 + k)} + \frac{1}{2\sqrt{1 - k^2}} - 1\right) = \frac{T}{2}\left(\frac{1}{\sqrt{1 - k^2}} - 1\right)\left(\frac{1}{\sqrt{1 - k^2}} + 2\right)$

时间为 $\frac{T}{4(1 - k)} + \frac{T}{4(1 + k)} + \frac{T}{2\sqrt{1 - k^2}} - T$

\section*{1 - 24 补交}

对切点分析,把实际速度按沿杆、垂直杆的方向分解,得 $v_t = v\sin \theta$. 故 $\omega = \frac{v_t}{R/\tan \theta} = \frac{v \sin^2 \theta}{R\cos \theta}$

\end{document}