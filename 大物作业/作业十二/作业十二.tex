\documentclass{article}
\usepackage{amsmath}  % 数学符号包
\usepackage{amssymb}  % 更多数学符号
\usepackage{enumitem} % 列表样式
\usepackage{fancyhdr} % 页眉设置
\usepackage{geometry} % 页面设置
\usepackage[UTF8]{ctex}
\usepackage{bm}
\usepackage{amsthm}
\everymath{\displaystyle}  % 让所有数学模式都使用 \displaystyle
\newcommand{\lb}{\left\llbracket}
\newcommand{\rb}{\right\rrbracket}


\geometry{a4paper, margin=1in}


\pagestyle{fancy}
\fancyhf{}
\fancyhead[C]{作业十二}
\fancyhead[R]{2025.5.18}


\title{作业十二}
\author{Noflowerzzk}
\date{2025.5.18}


\begin{document}
\maketitle

\section*{6 - 14}

\begin{itemize}
    \item [(1)] 显然为 $A_2 - A_1$
    \item [(2)] 计算的表达式为 $x = (A_2 - A_1)\sin\frac{2\pi}{T}t$
\end{itemize}

\section*{6 - 15}

\begin{itemize}
    \item [(1)] 由图得 $A = 10 \mathrm{cm}$
    \item [(2)] 相位差为 $\frac{\pi}{2}$
\end{itemize}

\section*{6 - 24}

\begin{itemize}
    \item [(1)] $x^2 + y^2 - xy = 12$, 椭圆
    \item [(2)] $x + y = 0$, 直线
    \item [(3)] $x^2 + y^2 = 16$, 圆
\end{itemize}

\section*{8 - 1}

使用天数为9天

\section*{8 - 2}

积分得管子内分子数为 $N = \frac{\ln 5}{800}\frac{p_0SL}{k_B}$. 故压强为 $p = \frac{\ln 5}{8}p_0$

\section*{8 - 3}

时间为 $6\ln 10$ 分钟

\section*{8 - 4}

$\omega = \sqrt{\frac{2gH_0}{3L^2}}$

\section*{8 - 17}

方均艮速率为 $\sqrt{\frac{3k_BT}{m}}$, 内能为 $\frac{1}{2}\rho v^2$

\section*{8 - 18}

比例为 $3 : 5$

\section*{8 - 19}

比例为 $\frac{5}{6}$

\section*{8 - 20}

平均动能 $\frac{MpV}{2N_Am}$

\section*{8 - 21}

\begin{itemize}
    \item [(1)] $A$ 内能 $\frac{3}{2}p_0V_0$, $B$ 内能为 $\frac{5}{2}p_0V_0$.
    \item [(2)] 平均温度为 $\frac{8p_0V_0}{13R}$
\end{itemize}

\end{document}