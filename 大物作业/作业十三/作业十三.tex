\documentclass{article}
\usepackage{amsmath}  % 数学符号包
\usepackage{amssymb}  % 更多数学符号
\usepackage{enumitem} % 列表样式
\usepackage{fancyhdr} % 页眉设置
\usepackage{geometry} % 页面设置
\usepackage[UTF8]{ctex}
\usepackage{bm}
\usepackage{amsthm}
\everymath{\displaystyle}  % 让所有数学模式都使用 \displaystyle
\newcommand{\lb}{\left\llbracket}
\newcommand{\rb}{\right\rrbracket}
\newcommand{\dd}{\mathrm{d}}
\newcommand{\ee}{\mathrm{e}}

\geometry{a4paper, margin=1in}


\pagestyle{fancy}
\fancyhf{}
\fancyhead[C]{作业十三}
\fancyhead[R]{2025.5.22}


\title{作业十三}
\author{Noflowerzzk}
\date{2025.5.22}


\begin{document}
\maketitle

\section*{8 - 5}

极值点为 $v = 50$m/s \\
约为 $3.6 \times 10^8$ \\
平均速率为 $31.25$m/s

\section*{8 - 6}

曲线 $1$ 对应的为 $v = \sqrt{\frac{2k_BT_1}{m}}$. \\
百分比之差为 $1 - S_0$.

\section*{8 - 8}

\begin{itemize}
    \item [(1)] 约为 $1.44 \times 10^{10}$
    \item [(2)] 约为 $6.4 \times 10^8$
    \item [(3)] 平均速率约为 $54$m/s
    \item [(4)] 平均速率为 $80$m/s
\end{itemize}

\section*{8 - 9}

单位冲量为 $\dd I = \int_{0}^{+\infty}2mv_x nf(v)\dd v v_x \dd t \dd S$, 故计算得压强为 $nk_BT$.

\section*{8 - 10}

\begin{itemize}
    \item [(1)] $T_1 = 300$K 时 $v = 394$m/s, $T_2 = 600$K 时 $v = 588$m/s
    \item [(2)] 代入分布函数得占比为 $0.15\%$
    \item [(3)] 同理为 $0.042\%$
\end{itemize}

\section*{8 - 11}

$\dd m = \left\lvert p_1 - p_2\right\rvert S\sqrt{\frac{M_{\mathrm{mol}}}{2\pi RT}}$

\section*{8 - 14}

$f(\varepsilon) = \frac{2}{\sqrt{\pi}}\left(k_BT\right)^{-\frac{3}{2}}\sqrt{\varepsilon}\ee^{-\frac{\varepsilon}{k_BT}}$ \\
最概然动能为 $0$, 平均动能为 $\frac{3}{2}k_BT$

\section*{8 - 12}

需要 $3600$s

\section*{8 - 13}

\begin{align*}
    p_1 = \frac{p_0}{2}\left(\ee^{-\frac{S}{2V}\sqrt{\frac{8k_BT}{\pi m}}t} + 1\right) \\
    p_2 = \frac{1}{2}p_0\left(1 - \ee^{-\frac{S}{2V}\sqrt{\frac{8k_BT}{\pi m}}t}\right)
\end{align*}

\section*{8 - 15}

\begin{itemize}
    \item [(1)] 刚性:$U = \frac{5}{4}v_p^2$. 非刚性双原子分子, $U = \frac{7}{4}v_p^2$.
    \item [(2)] 刚性 $E = \frac{5}{4}mv_p^2$, 非刚性 $E = \frac{7}{4}mv_p^2$.
\end{itemize}

\section*{8 - 22}

内能增加 $\frac{3}{4}nk_BT$

\section*{8 - 23}

氢气分子的平均速度较大,易从大气层中逃逸,故大气中氢气含量不断减少

\section*{8 - 25}

$2308$m

\end{document}