\documentclass{article}
\usepackage{amsmath}  % 数学符号包
\usepackage{amssymb}  % 更多数学符号
\usepackage{enumitem} % 列表样式
\usepackage{fancyhdr} % 页眉设置
\usepackage{geometry} % 页面设置
\usepackage[UTF8]{ctex}
\usepackage{bm}
\usepackage{amsthm}
\everymath{\displaystyle}  % 让所有数学模式都使用 \displaystyle
\newcommand{\lb}{\left\llbracket}
\newcommand{\rb}{\right\rrbracket}


\geometry{a4paper, margin=1in}


\pagestyle{fancy}
\fancyhf{}
\fancyhead[C]{作业十一}
\fancyhead[R]{2025.5.6}


\title{作业十一}
\author{Noflowerzzk}
\date{2025.5.6}


\begin{document}
\maketitle

\section*{6 - 1}

震动方程为 $x = 0.12 \cos \left(\frac{4\pi}{3}t + \frac{\pi}{3}\right)$m. \\
$\Delta t = 0.5$s

\section*{6 - 2}

由图显然 $T = 3$s

\section*{6 - 4}

\begin{itemize}
    \item [(1)] $\frac{\mathrm{d}^2\theta}{\mathrm{d}t^2} + \frac{g}{l}\theta = 0$, 角频率为 $\sqrt{\frac{g}{l}} = 3.13$rad/s, $f = 0.498$Hz, $T = 2.01$s
    \item [(2)] $\theta = 0.0876\cos \left(3.13t + 226.8°\right)$rad
\end{itemize}

\section*{6 - 6}

位置 $x = 6\sqrt{3}$m, 速度 $-6\pi$m/s,加速度 $-6\sqrt{3}\pi^2 \mathrm{m/s^2}$

\section*{6 - 7}

$\Delta \phi = (2k + 1)\pi$

\section*{6 - 10}

$\frac{\mathrm{d}^2x}{\mathrm{d}t^2} + \frac{2g}{L}x = 0$, 因此做简谐运动.

\section*{6 - 11}

由于 $\frac{\mathrm{t}^2\theta}{\mathrm{d}t^2} + \frac{3g}{2L}\theta = 0$, 故其周期为 $2\pi\sqrt{\frac{2L}{3g}}$ \\
计算同样有 $\frac{\mathrm{t}^2\theta}{\mathrm{d}t^2} + \frac{3g}{2L}\theta = 0$, 故其周期不变.

\section*{6 - 13}

动能占 $\frac{3}{4}$, 势能占 $\frac{1}{4}$. \\
各占一半时, $x = \pm \frac{\sqrt{2}}{2}A$.

\section*{6 - 16}

平衡位置有 $x_0 = \frac{mg}{k}$. 又 $\frac{1}{2}kA^2 = \frac{1}{2}mv^2$ 得 $A = \sqrt{\frac{mv^2}{k}}$. 计算得外力左总为 $W = \frac{\pi}{2}mvg\sqrt{\frac{m}{k}} + \frac{m^2g^2}{2k} + mv^2$.

\section*{6 - 18}

最大速度为 $v_{\mathrm{m}} = \mu g\sqrt{\frac{m_1 + m_2}{k}}$

\section*{6 - 19}

\begin{itemize}
    \item [(1)] 圆频率为 $\sqrt{\frac{M + m}{k}}$, 由于总能量不变,振幅不变。
    \item [(2)] 圆频率为 $\sqrt{\frac{M + m}{k}}$, 计算后得振幅变为 $A' = \sqrt{\frac{M}{M + m}}A$
\end{itemize}

\section*{6 - 21}

在弹簧原长时,在 $x$ 处取微元,有 $\mathrm{d}m = \frac{m_s}{l_0}\mathrm{d}x$. 弹簧伸长 $x_0$ 时,该位置位移为 $u \frac{xx_0}{l_0}$,此时 $\mathrm{d}m$ 的动能为 $\mathrm{d}E_k = \frac{m_s}{2l_0^3}\left(\frac{\mathrm{d}x}{\mathrm{d}t}\right)^2x^2\mathrm{d}x$,计算得总机械能为 $E = \frac{1}{2}\left(\frac{m_s}{2} + m\right)\dot{x}^2 + \frac{1}{2}kx^2$. 机械能守恒,有 $\ddot{x} + \frac{k}{\frac{m_s}{3} + m}x = 0$, 故周期为 $2\pi\sqrt{\frac{\frac{m_s}{3} + m}{k}}$

\section*{6 - 22}

由于 $A_1 = A_0 \mathrm{e}^{-\beta t}$, 得 $\beta = 0.11$, 计算得时间为 $t = 20.96$s

\section*{6 - 23}

有阻尼,则周期为 $T = \frac{2\pi}{\sqrt{\omega^2 - \beta^2}}$, 又 $0.9 A = A\mathrm{e}^{-\beta T} $, 解得 $T = 1.00014T_0$

$S = 4\sum_{n = 1}^{\infty}A_i = 400$cm

\end{document}