\documentclass{article}
\usepackage{amsmath}  % 数学符号包
\usepackage{amssymb}  % 更多数学符号
\usepackage{enumitem} % 列表样式
\usepackage{fancyhdr} % 页眉设置
\usepackage{geometry} % 页面设置
% \usepackage[UTF8]{ctex}
\usepackage{bm}
\usepackage{amsthm}


\geometry{a4paper, margin=1in}

% 定义定理环境
\newtheorem{theorem}{Theorem}   % 定义“定理”环境
\newtheorem{proposition}{Proposition} % 定义“命题”环境


\pagestyle{fancy}
\fancyhf{}
\fancyhead[C]{Mutual Implication of Theorems in the Real Number System}
\fancyhead[R]{}


\title{Mutual Implication of Theorems in the Real Number System}
\author{Noflowerzzk}
\date{}


\begin{document}

\maketitle

\section{Supremum Axiom \& Monotone Convergence Theorem}
\subsection{Supremum Axiom$\Rightarrow$Monotone Convergence Theorem}

\begin{proof}
    Take an in increasing sequence $\{ x_n \} $ s.t. $\forall n\in \mathbb{N}\*, x_n < M$ \\
    By Supremum Axiom, we know that $\{ x_n \}$ has a supremum $L = \sup {\{x_n\}}$ \\
    For $\forall \varepsilon > 0, \exists n_0 $ s.t.
    $$ x_{n_0} + \varepsilon > L $$ 
    Because $\{x_n\}$ is monotonically increasing, for all $n > n_0$, we have
    $$ L - x_n < \varepsilon $$
    i.e.
    $$ \left\lvert L - x_n\right\rvert < \varepsilon $$
    That means
    $$ \lim_{n \to \infty} x_n = L = \sup{\{x_n\}} $$
    \qedhere
\end{proof}

\subsection{Monotone Convergence Theorem$\Rightarrow$Supremum Axiom}

\begin{proof}
    \textit{Referring to} 
    \textbf{The Nested Closed Interval Theorem}
\end{proof}

\section{Monotone Convergence Theorem \& The Nested Closed Interval Theorem}

\subsection{Monotone Convergence Theorem $\Rightarrow$ The Nested Closed Interval Theorem}

\begin{theorem}
    For a sequence of losed intervals $[a_n, b_n]$ with the following properties:
    \begin{itemize}
        \item[(1)] $\displaystyle{ [a_{n + 1}, b_{n + 1}] \subset [a_n, b_n] }$
        \item[(2)] $\displaystyle{ \lim_{n \to \infty} a_n -b_n = 0 }$ 
    \end{itemize}
    Then there exist a unique $\xi$
    $$\bigcap_{i = 1}^{\infty}[a_n, b_n] = \{\xi\}$$
\end{theorem}

\begin{proof}
    For nested closed intervals $\{[a_n, b_n ] \}$, obviously 
    $$\bigcap_{i = 1}^{\infty}[a_n, b_n] $$ is not empty.
    Noting that
    $$ a_1 \leqslant  a_2 \leqslant \cdots \leqslant a_n \leqslant \cdots \leqslant b_n \leqslant \cdots \leqslant b_2 \leqslant b_1 $$
    So we have 
    $$ \lim_{n \to \infty}a_n = A, \lim_{n \to \infty}b_n = B $$
    If there exist $$ \xi , \xi^{\prime}\in \bigcap_{i = 1}^{\infty}[a_n, b_n]$$
    Because $\lim_{n \to \infty}(a_n - b_n) = 0$, we have $A = B$. 
    And $A \leqslant \xi, \xi^{\prime} \leqslant B$, 
    So $\xi = \xi^{\prime}$. 
    \qedhere
\end{proof}

\subsection{The Nested Closed Interval Theorem$\Rightarrow$Monotone Convergence Theorem}



\section{Supremum Axiom $\&$  The Nested Closed Interval Theorem}

\subsection{The Nested Closed Interval Theorem $\Rightarrow$ Supremum Axiom}

\begin{proof}
    Assume a number set $A$. When $A$ is finit, obviously its supremum is $\max A$. \\
    If $A$ is infinit, without loss of generality, let $A$ has upper bounds. Let $B$ be the set of upper bounds of $A$. \\
    We choose $a_1 \in A$, and let $C = \{x| x > a_1, x\notin B\}$. \\
    Then choose $c_1 \in C, b_1 \in B$, we have $c_1 < b_1$. \\
    If $\displaystyle{\frac{c_1 + b_ 1}{2} \in C}$, let $c_2 = \displaystyle{\frac{c_1 + b_ 1}{2}}, b_2 = b_1$.
    Otherwise let $c_2 = c_1, b_2 = \displaystyle{\frac{c_1 + b_1}{2}}$.
    By analogy, we have construct a sequence of closed intervals $$\{[c_n, b_n]\}$$ which satisfies all the condition of The Nested Closed Interval Theorem. \\
    So we have $$ \bigcap_{i = 1}^{\infty}[c_n, b_n] = \{\xi\}, \lim_{n \to \infty} c_n =\lim_{n \to \infty} b_n = \xi $$
    We will now prove that $\xi = \sup A$. \\
    According to the definition of limit, $\forall \varepsilon > 0, \exists N, \forall n > N, \xi - c_n < \varepsilon $. \\
    Noting that $c_n$ is not the upper bound of $A$, \\
    so $\exists \varphi \in A, \varphi > \xi - \varepsilon$.
    That means $\xi$ is the supremum of $A$.
\end{proof}

\subsection{Supremum Axiom $\Rightarrow$ The Nested Closed Interval Theorem}

\begin{proof}
    \textit{Referring to} 
    \textbf{Monotone Convergence Theorem}
\end{proof}


\section{Cauchy's convergence test $\&$ Supremum Axiom}
\subsection{Cauchy's convergence test $\Rightarrow$ Supremum Axiom}

\begin{proof}
    Assume that $S$ is a set with upper bounds. \\
    According to Archimedean property, for any $a > 0$, there exist $k$, $ka = \lambda_a$ is the upper bound of $S$ while $\lambda_a - a = (k - 1)a$ is not. \\
    Now let $a_n = \frac{1}{n}$, then we have a sequence of ${\lambda_n}$. \\
    Noting that exist $a^{\prime} > \lambda_n - \frac{1}{n}$. \\
    Also, $\lambda_m \geqslant a^{\prime}$, so we have
    $$\lambda_n - \lambda_m < \frac{1}{n}$$
    Simiarly, 
    $$\lambda_m - \lambda_n < \frac{1}{m}$$
    So
    $$\left\lvert \lambda_m - \lambda_n \right\rvert < \max\left\{\frac{1}{n}, \frac{1}{m}\right\}$$
    So according to \textbf{Cauchy's convergence test}, ${\lambda_n}$ have a limit. \\
    Now we will prove that the limit is the supremum of $S$. \\
    Let $\displaystyle{\lambda = \lim_{n \to \infty }\lambda_n}$.
    Obviously $\lambda$ is the upper bound of $S$. Also $\exists a^{\prime} > \lambda - \delta$. \\
    So $\lambda$ is the supremum of $S$.\qedhere 
\end{proof}

\subsection{Supremum Axiom $\Rightarrow$ Cauchy's convergence test}

\begin{proof}
    
\end{proof}



\end{document}