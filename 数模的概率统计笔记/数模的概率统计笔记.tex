\special{dvipdfmx:config z 0} %取消PDF压缩,加快速度,最终版本生成的时候最好把这句话注释掉


\documentclass{ctexart}
\usepackage{newtxtext, amsmath, amssymb}
\usepackage[left=0.5in, right=0.5in, top=0.9in, bottom=0.9in]{geometry}
\usepackage{amsthm} % 设置定理环境,要在amsmath后用
\usepackage{xpatch} % 用于去除amsthm定义的编号后面的点
\usepackage[strict]{changepage} % 提供一个 adjustwidth 环境
\usepackage{multicol} % 用于分栏
\usepackage[fontsize=14pt]{fontsize} % 字号设置
\usepackage{esvect} % 实现向量箭头输入(显示效果好于\vec{}和\overrightarrow{}),格式为\vv{⟨向量符号⟩}或\vv*{⟨向量符号⟩}{⟨下标⟩} 
\usepackage[
            colorlinks, % 超链接以颜色来标识,而并非使用默认的方框来标识
            linkcolor=mlv,
            anchorcolor=mlv,
            citecolor=mlv % 设置各种超链接的颜色
            ]
            {hyperref} % 实现引用超链接功能
\usepackage{tcolorbox} % 盒子效果
\usepackage{hyperref}
\usepackage{fancyhdr}
\usepackage{extarrows}
\tcbuselibrary{most} % tcolorbox宏包的设置,详见宏包说明文档
\everymath{\displaystyle}  % 让所有数学模式都使用 \displaystyle


\geometry{a4paper,centering,scale=0.8}

% 保证 leftmark 始终为 subsection 名称
\newcommand{\chaptermark}{\markboth{}{}}
\renewcommand{\sectionmark}[1]{\markboth{}{}}
\renewcommand{\subsectionmark}[1]{\markboth{#1}{}}

\pagestyle{fancy}%清除原页眉页脚样式
\fancyhf{}

\fancyhead[L]{\leftmark}
\fancyhead[R]{\thepage}

% ————————————————————————————————————自定义符号————————————————————————————————————
\newcommand{\。}{.} % 示例:将指令\。定义为全角句点。
\newcommand{\zte}{\mathrm{e}} % 正体e
\newcommand{\ztd}{\mathrm{d}} % 正体d
\newcommand{\zti}{\mathrm{i}} % 正体i
\newcommand{\ztj}{\mathrm{j}} % 正体j
\newcommand{\ztk}{\mathrm{k}} % 正体k
\newcommand{\CC}{\mathbb{C}} % 黑板粗体C
\newcommand{\QQ}{\mathbb{Q}} % 黑板粗体Q
\newcommand{\RR}{\mathbb{R}} % 黑板粗体R
\newcommand{\ZZ}{\mathbb{Z}} % 黑板粗体Z
\newcommand{\NN}{\mathbb{N}} % 黑板粗体N
\newcommand{\ds}{^\prime}
\newcommand{\dds}{^{\prime\prime}}
\newcommand{\dx}{\mathrm{d}x}
\newcommand{\dd}{\mathrm{d}}
% ————————————————————————————————————自定义符号————————————————————————————————————

% ————————————————————————————————————自定义颜色————————————————————————————————————
\definecolor{mlv}{RGB}{40, 137, 124} % 墨绿
\definecolor{qlv}{RGB}{240, 251, 248} % 浅绿
\definecolor{slv}{RGB}{33, 96, 92} % 深绿
\definecolor{qlan}{RGB}{239, 249, 251} % 浅蓝
\definecolor{slan}{RGB}{3, 92, 127} % 深蓝
\definecolor{hlan}{RGB}{82, 137, 168} % 灰蓝
\definecolor{qhuang}{RGB}{246, 250, 235} % 浅黄
\definecolor{shuang}{RGB}{77, 82, 59} % 深黄
\definecolor{qzi}{RGB}{240, 243, 252} % 浅紫
\definecolor{szi}{RGB}{49, 55, 70} % 深紫
% 将RGB换为rgb,颜色数值取值范围改为0到1
% ————————————————————————————————————自定义颜色————————————————————————————————————

% ————————————————————————————————————盒子设置————————————————————————————————————
% tolorbox提供了tcolorbox环境,其格式如下:
% 第一种格式:\begin{tcolorbox}[colback=⟨背景色⟩, colframe=⟨框线色⟩, arc=⟨转角弧度半径⟩, boxrule=⟨框线粗⟩]   \end{tcolorbox}
% 其中设置arc=0mm可得到直角;boxrule可换为toprule/bottomrule/leftrule/rightrule可分别设置对应边宽度,但是设置为0mm时仍有细边,若要绘制单边框线推荐使用第二种格式
% 方括号内加上title=⟨标题⟩, titlerule=⟨标题背景线粗⟩, colbacktitle=⟨标题背景线色⟩可为盒子加上标题及其背景线
\newenvironment{kuang1}{
    \begin{tcolorbox}[colback=hlan, colframe=hlan, arc = 1mm]
    }
    {\end{tcolorbox}}
\newenvironment{kuang2}{
    \begin{tcolorbox}[colback=hlan!5!white, colframe=hlan, arc = 1mm]
    }
    {\end{tcolorbox}}
% 第二种格式:\begin{tcolorbox}[enhanced, colback=⟨背景色⟩, boxrule=0pt, frame hidden, borderline={⟨框线粗⟩}{⟨偏移量⟩}{⟨框线色⟩}]   {\end{tcolorbox}}
% 将borderline换为borderline east/borderline west/borderline north/borderline south可分别为四边添加框线,同一边可以添加多条
% 偏移量为正值时,框线向盒子内部移动相应距离,负值反之
\newenvironment{kuang3}{
    \begin{tcolorbox}[enhanced, breakable, colback=hlan!5!white, boxrule=0pt, frame hidden,
        borderline south={0.5mm}{0.1mm}{hlan}]
    }
    {\end{tcolorbox}}
\newenvironment{lvse}{
    \begin{tcolorbox}[enhanced, breakable, colback=qlv, boxrule=0pt, frame hidden,
        borderline west={0.7mm}{0.1mm}{slv}]
    }
    {\end{tcolorbox}}
\newenvironment{lanse}{
    \begin{tcolorbox}[enhanced, breakable, colback=qlan, boxrule=0pt, frame hidden,
        borderline west={0.7mm}{0.1mm}{slan}]
    }
    {\end{tcolorbox}}
\newenvironment{huangse}{
    \begin{tcolorbox}[enhanced, breakable, colback=qhuang, boxrule=0pt, frame hidden,
        borderline west={0.7mm}{0.1mm}{shuang}]
    }
    {\end{tcolorbox}}
\newenvironment{zise}{
    \begin{tcolorbox}[enhanced, breakable, colback=qzi, boxrule=0pt, frame hidden,
        borderline west={0.7mm}{0.1mm}{szi}]
    }
    {\end{tcolorbox}}
% tcolorbox宏包还提供了\tcbox指令,用于生成行内盒子,可制作高光效果
\newcommand{\hl}[1]{
    \tcbox[on line, arc=0pt, colback=hlan!5!white, colframe=hlan!5!white, boxsep=1pt, left=1pt, right=1pt, top=1.5pt, bottom=1.5pt, boxrule=0pt]
{\bfseries \color{hlan}#1}}
% 其中on line将盒子放置在本行(缺失会跳到下一行),boxsep用于控制文本内容和边框的距离,left、right、top、bottom则分别在boxsep的参数的基础上分别控制四边距离
% ————————————————————————————————————盒子设置————————————————————————————————————

% ————————————————————————————————————自定义字体设置————————————————————————————————————
\setCJKfamilyfont{xbsong}{方正小标宋简体}
\newcommand{\xbsong}{\CJKfamily{xbsong}}
% CTeX宏集还预定义了\songti、\heiti、\fangsong、\kaishu、\lishu、\youyuan、\yahei、\pingfang等字体命令
% 由于未知原因,一些设备可能无法调用这些字体,故此文档暂时未使用
% ————————————————————————————————————自定义字体设置————————————————————————————————————

% ————————————————————————————————————各级标题设置————————————————————————————————————
\ctexset{
    % 修改 section。
    section={   
    % 设置标题编号前后的词语,使用name={⟨前部分⟩,⟨后部分⟩}参数进行设置。    
        name={\textbf{第},\textbf{章}\hspace{18pt}},
    % 使用number参数设置标题编号,\arabic设置为阿拉伯数字,\chinese设置为中文,\roman设置为小写罗马字母,\Roman设置为大写罗马字母,\alph设置为小写英文,\Alph设置为大写英文。
        number={\textbf{\chinese{section}}},
    % 参数format设置标题整体的样式。包括标题主题、编号以及编号前后的词语。
    % 参数format还可以设置标题的对齐方式。居中对齐\centering,左对齐\raggedright,右对齐\hfill
        format=\color{hlan}\centering\zihao{2}, % 设置 section 标题为正蓝色、黑体、居中对齐、小二号字
    % 取消编号后的空白。编号后有一段空白,参数aftername=\hspace{0pt}可以用来控制编号与标题之间的距离。
        aftername={}
    },
    % 修改 subsection。
    subsection={   
        name={\S \hspace{6pt}, \hspace{8pt}},
        number={\arabic{section}.\arabic{subsection}},
        format=\color{hlan}\centering\zihao{-2}, % 设置 subsection 标题为黑体、三号字
        aftername=\hspace{0pt}
    },
    subsubsection={   
        name={,、},
        number={\chinese{subsubsection}},
        format=\raggedright\color{hlan}\zihao{3},
        aftername=\hspace{0pt}
    },
    part={   
        name={第,部分},
        number={\chinese{part}},
        format=\color{white}\centering\bfseries\zihao{-1},
        aftername=\hspace{1em}
    }
}
% 各个标题设置的各行结尾一定要记得写逗号!!!!!!!!!!!!!!!!!!!!!!!!!!!!!!!!!!!!!!!!!!
% ————————————————————————————————————各级标题设置————————————————————————————————————

% ————————————————————————————————————定理类环境设置————————————————————————————————————
\newtheoremstyle{t}{0pt}{}{}{\parindent}{\bfseries}{}{1em}{} % 定义新定理风格。格式如下:
%\newtheoremstyle{⟨风格名⟩}
%                {⟨上方间距⟩} % 若留空,则使用默认值
%                {⟨下方间距⟩} % 若留空,则使用默认值
%                {⟨主体字体⟩} % 如 \itshape
%                {⟨缩进长度⟩} % 若留空,则无缩进;可以使用 \parindent 进行正常段落缩进
%                {⟨定理头字体⟩} % 如 \bfseries
%                {⟨定理头后的标点符号⟩} % 如点号、冒号
%                {⟨定理头后的间距⟩} % 不可留空,若设置为 { },则表示正常词间间距;若设置为 {\newline},则环境内容开启新行
%                {⟨定理头格式指定⟩} % 一般留空
% 定理风格决定着由 \newtheorem 定义的环境的具体格式,有三种定理风格是预定义的,它们分别是:
% plain: 环境内容使用意大利斜体,环境上下方添加额外间距
% definition: 环境内容使用罗马正体,环境上下方添加额外间距
% remark: 环境内容使用罗马正体,环境上下方不添加额外间距
\theoremstyle{t} % 设置定理风格
\newtheorem{dyhj}{\color{slv} 定义}[subsection] % 定义定义环境,格式为\newtheorem{⟨环境名⟩}{⟨定理头文本⟩}[⟨上级计数器⟩]或\newtheorem{⟨环境名⟩}[⟨共享计数器⟩]{⟨定理头文本⟩},其变体\newtheorem*不带编号
\newtheorem{dlhj}{\color{shuang} 定理}[subsection]
\newtheorem{lthj}{\color{szi} 例}[subsection]
\newtheorem*{jiehj}{\color{slan} 解}
\newtheorem*{zmhj}{\color{slan} 证明}
\newtheorem{ylhj}{\color{shuang} 引理}[subsection]
\newtheorem*{ylzmhj}{\color{slan} 引理的证明}
\newenvironment{dy}{\begin{lvse}\begin{dyhj}}{\end{dyhj}\end{lvse}}
\newenvironment{zm}{\begin{lanse}\begin{zmhj}}{$\hfill \square$\end{zmhj}\end{lanse}}
\newenvironment{dl}{\begin{huangse}\begin{dlhj}}{\end{dlhj}\end{huangse}}
\newenvironment{lt}{\begin{zise}\begin{lthj}}{\end{lthj}\end{zise}}
\newenvironment{yl}{\begin{huangse}\begin{ylhj}}{\end{ylhj}\end{huangse}}
% ————————————————————————————————————定理类环境设置————————————————————————————————————

% ————————————————————————————————————目录设置————————————————————————————————————
\setcounter{tocdepth}{4} % 设置在 ToC 的显示的章节深度
\setcounter{secnumdepth}{3} % 设置章节的编号深度
% 数值可选选项:-1 part 0 chapter 1 section 2 subsection 3 subsubsection 4 paragraph 5 subparagraph
\renewcommand{\contentsname}{\bfseries 目\hspace{1em}录}
% ————————————————————————————————————目录设置————————————————————————————————————

\everymath{\displaystyle} % 设置所有数学公式显示为行间公式的样式

\begin{document}

\quad\\
{
\Huge \bfseries 
\begin{kuang3}
    \color{hlan}
    \centering
    数模的概率统计笔记
\end{kuang3}
}

\quad \\
{
\bf
\begin{kuang3}
    \color{hlan}
    \centering
    Noflowerzzk    
\end{kuang3}
}

\thispagestyle{empty} % 取消这一页的页码
\newpage

% ————————————————————————————————————页码设置————————————————————————————————————
\pagenumbering{Roman}
\setcounter{page}{1}

\begin{multicols}{2} % 设置环境中内容为两栏。数字为栏数
    {
    \tableofcontents
    }
\end{multicols}

\newpage
\setcounter{page}{1}
\pagenumbering{arabic}
% ————————————————————————————————————页码设置————————————————————————————————————

\begin{kuang1}
    \part{概率论}
\end{kuang1}

\begin{kuang2}
    \section{随机变量}
\end{kuang2}

\begin{kuang3}
    \subsection{随机变量的数值性质}
\end{kuang3}

\begin{dy} \quad
    \begin{itemize}
        \item 协方差 $\mathrm{cov}(X, Y) = E((X - E(X))(Y - E(Y)))$
        \item 相关系数/标准化协方差 $\rho(X, Y) = \frac{\mathrm{cov}(X, Y)}{\sqrt{D(X)}\sqrt{D(Y)}}$
        \item 变异系数 $\delta_X = \frac{\sqrt{D(X)}}{\left\lvert E(X)\right\rvert }$
        \item $k$ 阶原点矩 $E(X^k)$
        \item $k$ 阶中心矩 $E\left((X - E(X))^k\right)$
    \end{itemize}
\end{dy}

\begin{kuang3}
    \subsection{大数定律}
\end{kuang3}

\begin{dl}[Chebyshev 不等式]
    \[
        P(\left\lvert X - E(X)\right\rvert \geqslant \varepsilon) \leqslant \frac{D(X)}{\varepsilon^2} 
    \]
\end{dl}

\begin{dy}[依概率收敛]
    $\exists c, \forall \varepsilon > 0, \lim_{n \to +\infty}P\left(\left\lvert X_n - c\right\rvert \leqslant \varepsilon\right) = 1$, 则称随机变量序列 $\{X_n\}$ 依概率收敛于 $c$, 记作 $X_n \xlongrightarrow[]{P}c$
\end{dy}

\begin{dl}[Chebyshev 大数定律]
    随机变量序列 $\{X_n\}$ 两两不(线性)相关,且 $D(X_i)$ 有一致上界 $c$ (即 $D(X_i) < c$), 则有
    \[
        \overline{X} = \frac{1}{n}\sum_{i = 1}^{n}X_i \xlongrightarrow{P}\frac{1}{n}\sum_{i = 1}^{n}E(X_i)
    \]
\end{dl}

\begin{dl}[相互独立同分布(辛钦)大数定律]
    $\{X_i\}$ 相互独立同分布, $E(X_i) = \mu$ 有限,则
    \[
        \overline{X} = \frac{1}{n}\sum_{i = 1}^{n}X_i \xlongrightarrow{P}\mu
    \]
\end{dl}

\begin{kuang3}
    \subsection{中心极限定理}
\end{kuang3}

\begin{dl}[列维-林德伯格中心极限定理]
    $\{X_i\}$ 相互独立同分布,$D(X_i) = \sigma^2$ 有限,$E(X_i) = \mu$,则
    \[
        \lim_{n \to \infty}P\left(\frac{\sum_{i = 1}^{n}X_i - n\mu}{\sqrt{n}\sigma}\right) = \Phi(x)
    \]
    即当 $n$ 充分大时,可以认为 $\sum_{i = 1}^{n}X_i \overset{\text{近似}}{\sim} N(n\mu, n\sigma^2)$ 或者 $\bar{X} \overset{\text{近似}}{\sim} N\left(\mu, \frac{\sigma^2}{n}\right)$
\end{dl}

\begin{kuang1}
    \part{数理统计}
\end{kuang1}

\begin{kuang2}
    \section{统计量}
\end{kuang2}

\begin{kuang3}
    \subsection{三大分布}
\end{kuang3}

\subsubsection{$\chi^2$ 分布}

\begin{dy}
    设 $\{X_i\}_{i = 1}^n$ 为相互独立的标准正态分布随机变量,称随机变量 $Y = \sum_{i = 1}^{n}X_i^2$ 服从自由度为 $n$ 的 $\chi^2$ 分布,记为 $Y \sim \chi^2(n)$. \\
    $\chi^2$ 分布的密度函数为
    \begin{align*}
        f(y) = \begin{cases}
            \frac{1}{2^{\frac{n}{2}}\Gamma\left(\frac{n}{2}\right)}y^{\frac{n}{2} - 1}\mathrm{e}^{-\frac{y}{2}}, & \quad y > 0 \\
            0,& \quad \text{其它}
        \end{cases}
    \end{align*}
\end{dy}

\begin{dl}
    $Y \sim \chi^2(n)$ 有以下性质
    \begin{itemize}
        \item $E(Y) = n, D(Y) = 2n$
        \item 可加性, $X \sim \chi^2(m), Y \sim \chi^2(n)$, $X, Y$ 相互独立,则 $X + Y \sim \chi^2(m + n)$
    \end{itemize}
\end{dl}

\subsubsection{$t$ 分布(学生氏分布)}

\begin{dy}
    设 $X, Y$ 相互独立, $X \sim N(0, 1), Y \sim \chi^2(n)$, 则称 $T = \frac{X}{\sqrt{Y/n}}$ 服从自由度为 $n$ 的 $t$ 分布. \\
    $t$ 分布的密度函数为
    \begin{align*}
        f(x) = \frac{\Gamma\left(\frac{n + 1}{2}\right)}{\sqrt{\pi n }\Gamma\left(\frac{n}{2}\right)}\left(1 + \frac{x^2}{n}\right)^{-\frac{n + 1}{2}}
    \end{align*}
\end{dy}

$t(n)$的密度函数与标准正态分布 $N(0, 1)$ 密度很相似, 它们都是关于原点对称, 单峰偶函数, 在 $x = 0$ 处达到极大. 但 $t(n)$ 的峰值低于
$N(0, 1)$ 的峰值, $t(n)$ 的密度函数尾部都要比 $N(0, 1)$ 的两侧尾部粗一些. 容易证明:
\[
    \lim_{n \to \infty} f(x) = \Phi(x)
\]



\end{document}