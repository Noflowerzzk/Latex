\documentclass{article}
\usepackage{amsmath}  % 数学符号包
\usepackage{amssymb}  % 更多数学符号
\usepackage{enumitem} % 列表样式
\usepackage{fancyhdr} % 页眉设置
\usepackage{geometry} % 页面设置
\usepackage[UTF8]{ctex}
\usepackage{bm}
\usepackage{amsthm}
\everymath{\displaystyle}  % 让所有数学模式都使用 \displaystyle
\newcommand{\lb}{\left\llbracket}
\newcommand{\rb}{\right\rrbracket}


\geometry{a4paper, margin=1in}


\pagestyle{fancy}
\fancyhf{}
\fancyhead[C]{作业四}
\fancyhead[R]{2025.3.13}


\title{作业四}
\author{Noflowerzzk}
\date{2025.3.13}


\begin{document}
\maketitle

\section*{P72 T1}

\begin{itemize}
    \item [(9)] 取 $x_n = \frac{2}{3^n\pi}$, 有 $u_n(x_n) \to 2^n$, $u_n(x)$ 在 $(0, +\infty)$ 不一致收敛到0,故函数项级数不一致收敛. \\
    当 $x \geq \delta$ 时,$\left\lvert u_n(x)\right\rvert \leq \frac{1}{\delta}\frac{2^n}{3^n}$. 由于 $\sum_{n = 1}^{\infty} \frac{1}{\delta}\frac{2^n}{3^n}$ 收敛,故原函数项级数收敛.
    \item [(10)] 令 $a_n = \frac{1}{\sqrt{n}}, b_n = \sin x \sin nx$. $a_n$ 对任意固定的 $x$ 单减趋于0, $\sum_{k = 1}^{n}b_k = \cos \frac{x}{2} \left(\cos \left(n + \frac{1}{2}\right)x - \cos \frac{x}{2}\right)$ 有界,故原级数收敛.
    \item [(11)] 由于取 $n, m = 2n$, $x = \frac{1}{\sqrt{n}}$
    \begin{align*}
        \sum_{k = n}^{m}u_k > \frac{nx^2}{(1 + x^2)^{2n}} = \frac{1}{\left(1 + \frac{1}{n}\right)^{2n}} > \frac{1}{\mathrm{e}^2}
    \end{align*}
    由 Cauchy 收敛准则知级数不收敛.
    \item [(12)] 令 $a_n = \frac{x^2}{(1 + x^2)^n}$ 单减收敛到0, $b_n = (-1)^n$ 构成的部分和序列有界,故原级数一致收敛.
\end{itemize}

\section*{P73 T2}

\begin{proof}
    由于 $\frac{\cos nx}{n^2 + 1} \leq \frac{1}{n^2 + 1}$ 而 $\sum_{n = 1}^{\infty}\frac{1}{n^2 + 1}$ 收敛,故 $f(x)$ 连续. \\
    又设 $\sigma(x) = \sum_{n = 1}^{\infty}\left(\frac{\cos nx}{n^2 + 1}\right)' = -\sum_{n = 1}^{\infty}\frac{n\sin nx}{n^2 + 1}$. \\
    由于 $x \in [2a, 2\pi - 2a]$ 有 
    \begin{align*}
        \left\lvert -\sum_{n = 1}^{\infty}\frac{n\sin nx}{n^2 + 1}\right\rvert &= \frac{\left\lvert \cos \left(n + \frac{1}{2}\right)x - \cos \frac{1}{2}x\right\rvert }{\left\lvert 2\sin a\right\rvert } \leq \frac{1}{\sin a}
    \end{align*}
    故 $-\sum_{n = 1}^{\infty}\frac{n\sin nx}{n^2 + 1}$ 在 $(0, 2\pi)$ 上内闭一致收敛. 因此 $\sigma(x)$ 在 $(0, 2\pi)$ 上连续. 又 $f'(x) = \sigma(x)$, $f(x)$ 在 $(0, 2\pi)$ 上有连续导数.
\end{proof}

\section*{P73 T3}

\begin{proof}
    对任意闭区间 $[m, M]$ 上的 $x$, 有 $\sum_{n = 1}^{\infty}n\mathrm{e}^{-nx}$ 收敛,故 $\sum_{n = 1}^{\infty}n\mathrm{e}^{-nx}$ 一致收敛, $f(x)$ 连续. \\
    令 $\sigma_n(x) = \sum_{n = 1}^{\infty}(n\mathrm{e}^{-nx})^{(n)}$, 同理其也一致收敛,即 $\sigma_n(x)$ 一致收敛. 有显然 $\sigma(x) = f^{(n)}(x)$ 有 $f(x)$ 有各阶连续导函数.
\end{proof}

\section*{P73 T4}

\begin{proof}
    令 $f(x) = \sum_{n = 1}^{\infty}\frac{1}{n^x}$. 任意 $[m, M] \subseteq (1, +\infty)$, 有 $\sum_{n = 1}^{\infty}\frac{1}{n^x} \leq \sum_{n = 1}^{\infty}\frac{1}{n^m}$ 收敛. $f(x)$ 在 $(1, +\infty)$ 上内闭一致收敛。 $f(x)$ 连续. \\
    由于 $\left(\frac{1}{n^x}\right)^{(n)} = (-1)^k\frac{\ln^k n}{n^x}$, 同理 $\sum_{n = 1}^{\infty}(-1)^k\frac{\ln^k n}{n^x}$ 在 $(0, +\infty)$ 上内闭一致收敛. 故 $f(x)$ 在 $(1, +\infty)$ 上有各阶连续导函数. \\
    令 $g(x) = \sum_{n = 1}^{\infty}\frac{(-1)^n}{n^x}$, 同理 $g(x)$ 在 $(1, +\infty)$ 上内闭一致收敛,故 $g(x)$ 连续. \\
    由于 $\left(\frac{(-1)^n}{n^x}\right)^{(k)} = (-1)^{n + 1}\frac{\ln^k n}{n^x}$. 同理也有其内闭一致收敛,故 $g(x)$ 在 $(1, +\infty)$ 的各阶连续导函数.
\end{proof}

\section*{P73 T6}

\begin{proof}
    \begin{itemize}
        \item [(1)] 对固定的 $x < 1$ 有 $\frac{1}{n^x}$ 关于 $n$ 单减且小于 1. 由 Abel 判别法, $\sum_{n = 1}^{\infty}$ 在 $[0, \delta)$ 上一致收敛. 故和函数连续,即原表达式成立.
        \item [(2)] 同理有 $\sum_{n = 1}^{\infty}a_nx^n$ 一致收敛. 因此
        \begin{align*}
            \int_{0}^{1}\sum_{n = 1}^{\infty}a_nx^n = \sum_{n = 1}^{\infty}\frac{a_n}{n + 1}
        \end{align*}
    \end{itemize}
\end{proof}

\section*{P73 T7}

\begin{proof}
    由于 $v_n(x)$ 连续. $\sum_{n = 1}^{\infty}v_n(x)$, 由 Cauchy 收敛准则,对任意 $\varepsilon > 0$, 存在 $N$, 任意 $m > n > N$, $\left\lvert \sum_{k = n}^{m}u_k(x)\right\rvert \leq \sum_{k = n}^{m}v_k(x) \leq \varepsilon$, 故 $\sum_{k = n}^{m}u_n(x)$ 一致收敛.
\end{proof}

\section*{P73 T9}

\begin{proof}
    假设 $\sum_{n = 1}^{\infty}$ 在 $(a, a + \delta)$ 上一致收敛. 则由 Cauchy 收敛准则,任意 $\varepsilon > 0$, 存在 $N, \forall m > n > N$, $\left\lvert \sum_{k = n}^{m}u_k(x)\right\rvert \leq \varepsilon$. 则当 $x \to a$ 时, $\left\lvert \sum_{k = n}^{m}u_k(a)\right\rvert \leq \varepsilon$ 与 $\sum_{n = 1}^{\infty}$ 发散矛盾!因此 $\sum_{n = 1}^{\infty}$ 在 $(a, a + \delta)$ 上不一致收敛.
\end{proof}

\section*{P73 T10}

\begin{proof}
    已知 $\ln \left(1 + \frac{x}{n \ln^2 n}\right)$ 在 $[-a, a]$ 上单增. 故
    \begin{align*}
        \ln \left(1 + \frac{-a}{n \ln^2 n}\right) \leq \ln \left(1 + \frac{x}{n \ln^2 n}\right) \leq \ln \left(1 + \frac{a}{n \ln^2 n}\right)
    \end{align*}
    $n$ 充分大时, $\ln \left(1 + \frac{x}{n \ln^2 n}\right) \thicksim \frac{x}{n \ln^2 n}$
    而 $\lim_{n \to \infty}\sum_{n = 2}^{\infty}\frac{\pm a}{n \ln^2 n} $ 收敛. 因此原和函数一致收敛.
\end{proof}

\section*{P73 T12}

\begin{itemize}
    \item [(1)] 由于 $\sum_{n = 1}^{\infty}\frac{1}{\sqrt{n^3 + n}}$ 收敛且 $\cos nx$ 有界,则 $\sum_{n = 1}^{\infty}\frac{\cos nx}{\sqrt{n^3 + n}}$ 一致收敛,记为 $f(x)$, 故 $f(x)$ 连续.
    \item [(2)] \begin{align*}
        F(x) &= \int_{0}^{x}f(t)\mathrm{d}t \\
        &= \sum_{n = 1}^{\infty}\int_{0}^{x}\frac{\cos nx}{\sqrt{n^3 + n}}\mathrm{d}t = \sum_{n = 1}^{\infty}\frac{\sin nx}{n \sqrt{n^3 + n}}
    \end{align*}
    故 
    \begin{align*}
        F\left(\frac{\pi}{2}\right) &= \frac{\sqrt{2}}{2} - \frac{1}{3\sqrt{30}} \sum_{n = 3}^{\infty}\frac{(-1)^{n - 1}}{(2n - 1)\sqrt{(2n - 1)^3 + 2n - 1}} \\
    \end{align*}
    因此 $\frac{\sqrt{2}}{2} - \frac{1}{15} < \frac{\sqrt{2}}{2} - \frac{1}{3\sqrt{30}} < F\left(\frac{\pi}{2}\right) < \frac{\sqrt{2}}{2}$
\end{itemize}

\end{document}