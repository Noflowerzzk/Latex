\documentclass{article}
\usepackage{amsmath}  % 数学符号包
\usepackage{amssymb}  % 更多数学符号
\usepackage{enumitem} % 列表样式
\usepackage{fancyhdr} % 页眉设置
\usepackage{geometry} % 页面设置
\usepackage[UTF8]{ctex}
\usepackage{bm}
\usepackage{amsthm}
\everymath{\displaystyle}  % 让所有数学模式都使用 \displaystyle
\newcommand{\lb}{\left\llbracket}
\newcommand{\rb}{\right\rrbracket}


\geometry{a4paper, margin=1in}


\pagestyle{fancy}
\fancyhf{作业二}
\fancyhead[C]{}
\fancyhead[R]{2025.02.27}


\title{作业二}
\author{noflowerzzk}
\date{2025.02.27}


\begin{document}
\maketitle

\section*{P114 T2}

令 $t = \frac{y}{x} \in \mathbb{R}$, 有 
\begin{align*}
    f(t) = \frac{x^3}{\left(x^2 + y^2\right)^\frac{3}{2}} = \frac{1}{\left(1 + t^2\right)^\frac{3}{2}}
\end{align*}

故 $f(x) = \frac{1}{\left(1 + x^2\right)^\frac{3}{2}}$

\section*{P114 T4}

\begin{itemize}
    \item [(1)]
    由于 $\lim_{(x, y) \to (0, 0); y = 0}f(x, y) = 1$ 而 $\lim_{(x, y) \to 0; x = 0}f(x, y) = -1$, 故极限不存在.
    \item [(2)]
    由于 $\lim_{(x, y) \to (0, 0); y = 0}f(x, y) = 0$ 而 $\lim_{(x, y) \to (0, 0); y = x} = \frac{1}{2}$, 故极限不存在.
    \item [(3)]
    由于 $\lim_{(x, y) \to (0, 0); y = 1}f(x, y) = 0$ 而 $\lim_{(x, y) \to (0, 0); 0 < y < x^2} f(x, y) = 1$, 故极限不存在.
    \item [(4)]
    极限存在. 因为 $\frac{3}{4} + \frac{3}{8} > 1$. 且极限为 0.
\end{itemize}

\section*{P114 T7}

\begin{itemize}
    \item [(1)]
    \begin{align*}
        \lim_{(x, y) \to (0, 1)}\frac{1 - xy}{x^2 + y^2} &= \frac{1 - 0}{0 + 1} = 1
    \end{align*}
    \item [(3)]
    \begin{align*}
        \lim_{(x, y) \to (0, 0)} \frac{\sqrt{1 + xy} - 1}{xy} &= \lim_{t \to 0}\frac{\sqrt{1 + t} - 1}{t} = \frac{1}{2}
    \end{align*}
    \item [(5)]
    \begin{align*}
        \lim_{(x, y) \to (0, 0)}\frac{\ln \left(x^2 + \mathrm{e}^{y^2}\right)}{x^2 + y^2} &= \lim_{(x, y) \to (0, 0)}\frac{x^2 + \mathrm{e}^{y^2} - 1}{x^2 + y^2} \\
        &= \lim_{(x, y) \to (0, 0)} \frac{x^2 + y^2}{x^2 + y^2} = 1
    \end{align*}
    \item [(7)]
    \begin{align*}
        \lim_{(x, y) \to (0, 0)} \frac{1 - \cos (x^2 + y^2)}{(x^2 + y^2)x^2 y^2} &= \lim_{(x, y) \to (0, 0)}\frac{1}{2}\frac{(x^2 + y^2)^2}{(x^2 + y^2)x^2 y^2} \\
        &= \lim_{(x, y) \to (0, 0)}\frac{1}{2}\left(\frac{1}{x^2 } + \frac{1}{y^2}\right) = \infty
    \end{align*}
\end{itemize}

\section*{P115 T11}

\begin{proof}
    $x \neq y$ 时,
    由 Lagrange 中值定理,有存在 $\xi$ 在 $x, y$ 之间,
    \begin{align*}
        F(x, y) = \frac{f(x) - f(y)}{x - y} = f'(\xi) \to f'(c).
    \end{align*}
    $x = y$ 时,
    \begin{align*}
        F(x, y) = f'(x) \to f'(c)
    \end{align*}
\end{proof}

\section*{补充题1}

\begin{proof}
    设集合 $A \subseteq X$ 是度量空间 $(X, d)$ 的一个子集,且存在有限个点 $x_1, x_2, \cdots, x_n$ 构成 $A$ 的一个 $\varepsilon$-网. \\
    我们证明, $\{x_k\}_{k = 1}^n$ 是 $\overline{A}$ 的一个 $2\varepsilon$-网. \\
    $\forall y \in \overline{A}, \exists a \in A, d(a, y) < \varepsilon$. 同时, $\exists x_i \in A, a \in B(x_i, \varepsilon)$, 即 $d(a, x_i) < \varepsilon$. 故由三角不等式, $d(x_i, y) < 2\varepsilon$. \\
    因此$\{x_k\}_{k = 1}^n$ 是 $\overline{A}$ 的一个 $2\varepsilon$-网.
\end{proof}

\section*{补充题2}

\begin{proof}
    反证法. 假设存在数列 $\{x_n\}_{n = 1}^{+\infty}$ 是 Cauchy 列但其不收敛到 $X$ 中. \\
    由 Cauchy 列的定义,对 $\varepsilon > 0$, 存在 $N > 0$,任意 $m, n > N, d(x_m, x_n) < \varepsilon/2$. 故取 $A = \{x_k\ |\ k \in \mathbb{N}\}$, $A \subseteq \bigcup_{i = 1}^N$, $\{x_1, \cdots, x_N\}$ 是 $A$ 的一个 $\varepsilon$-网. 故由补充题1,有$\{x_1, \cdots, x_N\}$ 也是 $\overline{A}$ 的一个 $2\varepsilon$-网. 也即 $\overline{A}$ 是 $X$ 中的完全有界闭集,其为列紧集. \\
    因此 $\overline{A}$ 有收敛子列,收敛到 $x \in \overline{A}$. 由收敛的定义易推出原 Cauchy 列一定也收敛到 $\overline{A} \subseteq X$, 矛盾! \\
    因此原集合为完备集.
\end{proof}

\end{document}