\documentclass{article}
\usepackage{amsmath}  % 数学符号包
\usepackage{amssymb}  % 更多数学符号
\usepackage{enumitem} % 列表样式
\usepackage{fancyhdr} % 页眉设置
\usepackage{geometry} % 页面设置
\usepackage[UTF8]{ctex}
\usepackage{bm}
\usepackage{amsthm}
\everymath{\displaystyle}  % 让所有数学模式都使用 \displaystyle
\newcommand{\lb}{\left\llbracket}
\newcommand{\rb}{\right\rrbracket}
\newcommand{\ee}{\mathrm{e}}
\newcommand{\dd}{\mathrm{d}}
\newcommand{\dx}{\dd x}
\newcommand{\sif}{\sum_{n = 1}^{\infty}}
\newcommand{\nti}{\lim_{n \to +\infty}}
\newcommand{\sifz}{\sum_{n = 0}^{\infty}}


\geometry{a4paper, margin=1in}


\pagestyle{fancy}
\fancyhf{}
\fancyhead[C]{作业十四}
\fancyhead[R]{2024.12.25}


\title{作业十四}
\author{noflowerzzk}
\date{2024.12.25}


\begin{document}
\maketitle

\section*{P38 T1}

\begin{itemize}
    \item [(1)] 由于 $\sif x_n^+$ 发散, $\sif x_n$ 不绝对收敛. \\
    又 $\sif x_n^- = \sif \frac{1}{(2n)!}$ 收敛. 故 $\sif x_n$ 不条件收敛.
    \item [(2)] $\sif \left\lvert x_n\right\rvert = \sif \frac{1}{n + x}$ 不收敛, $\sif x_n$ 不绝对收敛. \\
    由于 $\lim_{n \to +\infty}\frac{1}{n + x} = 0$, $\sif \frac{-1^{n + 1}}{n + x}$ 为 Lebniz 级数,条件收敛.
    \item [(3)] 对级数 $\sif \sin \frac{x}{n}$, 设当 $n > N$ 时 $\sin \frac{x}{n} \geqslant \frac{2}{\pi}x$,则 $n_0 > N$ 时部分和 $\sum_{i = 1}^{n}\left\lvert \sin \frac{x}{i}\right\rvert  \geqslant \sum_{i = 1}^{n}\sin \frac{x}{i} \geqslant \sum_{i = 1}^{N}\sin \frac{x}{i} + \frac{2}{\pi}\sum_{i = N + 1}^{n}\frac{x}{i}$ 无上界. 故 $\sif (-1)^{n + 1}\sin \frac{x}{n}$ 不绝对收敛. \\
    而 $\lim_{n \to \infty}\sin \frac{x}{n} = 0$, $\sif (-1)^{n + 1}\sin \frac{x}{n}$ 为 Lebniz 级数,条件收敛.
    \item [(4)] 由于 $\lim_{n \to +\infty}\frac{(-1)^{n + 1}}{\sqrt[n]{n}} = 1 \neq 0$, 不绝对收敛,不条件收敛.
    \item [(5)] 由于 $\sum_{i = 1}^{n}\frac{\ln^2 i}{i} \geqslant \sum_{i = 1}^{n}\frac{1}{i}$, 不绝对收敛. \\
    又 $\lim_{n \to +\infty}\frac{\ln^2 n}{n} = \nti \ln \frac{n + 1}{n}(\ln n + \ln (n + 1)) = \nti \frac{\ln n + \ln (n + 1)}{n} = 0$, $\sum_{n = 2}^{\infty}(-1)^{n + 1}\frac{\ln^2 n}{n}$ 是 Lebniz 级数,条件收敛.
    \item [(6)] 由 $\left\lvert \frac{1}{\sqrt{n}}\cos\frac{n \pi}{3}\right\rvert \geqslant \frac{1}{2\sqrt{n}}\ (n = 6k + 2 \text{ 或 } n = 6k + 4)$, 故 $\sif \frac{1}{\sqrt{n}}\cos\frac{n \pi}{3}$ 不绝对收敛. \\
    而 $S_{6n} = \sum_{i = 1}^{2n}\frac{(-1)^{i - 1}}{2\sqrt{3i - 2}} + \sum_{i = 1}^{2n}\frac{(-1)^{i}}{2\sqrt{3i - 1}} + \sum_{i = 1}^{2n}\frac{(-1)^i}{\sqrt{3i}}$,故 $\sif \frac{1}{\sqrt{n}}\cos\frac{n \pi}{3}$ 条件收敛.
    \item [(7)] 当 $x \in \left(k\pi - \frac{\pi}{6}, k\pi + \frac{\pi}{6}\right)$ 时, $4 \sin^2 x < 1$, $\sif \frac{4^n \sin^{2n} x}{n}$ 收敛,原级数绝对收敛. \\
    当 $x = k\pi \pm \frac{\pi}{6}$ 时,原级数即为 $\sif \frac{(-1)^{n + 1}}{n}$, 条件收敛. \\
    当 $x \in \left(k\pi + \frac{\pi}{6}, k\pi + \frac{5\pi}{6}\right)$ 时, $\nti (-1)^{n + 1} \frac{4^n \sin^{2n} x}{n} = \infty$, 不收敛. 
    \item [(8)] 
    \begin{itemize}
        \item $x = \frac{k\pi}{2}$ 时,原式 $= 0$, 绝对收敛.
        \item $x \neq \frac{k\pi}{2}, p > 1$ 时, $\left\lvert \frac{\sin (n + 1)x \cos(n - 1)x}{n^p}\right\rvert \leqslant \frac{1}{n^p}$, 绝对收敛.
        \item $x \neq \frac{k\pi}{2}, p \in (0, 1]$ 时,$\frac{\sin (n + 1)x \cos(n - 1)x}{n^p} = \frac{\sin 2nx}{2n^p} + \frac{\sin 2x}{2n^p}$, $\sif \frac{\sin 2x}{2n^p}$ 发散. 由于 $\frac{\sin 2nx}{2n^p}$ 中, $\nti \frac{1}{2n^p} = 0$, $\sum_{i = 1}^{n}\sin 2ix = \frac{1}{\cos x}\left(\sin (2n + 1)x - \sin x\right)$ 有界,故由 Dirichlet 判别法, $\sif \frac{\sin 2nx}{2n^p}$ 收敛,故原级数发散.
        \item $x \neq \frac{k\pi}{2}, p \leqslant 0$ 时,显然 $\nti \frac{\sin (n + 1)x \cos(n - 1)x}{n^p} \neq 0$, 不收敛.
    \end{itemize}
\end{itemize}

\section*{P38 T2}

\begin{itemize}
    \item [(1)] 
    \begin{align*}
        S_{6n} - S_{3n} &= \sum_{k = n + 1}^{2n}\left( \frac{1}{3k - 2} + \frac{1}{3k - 1} - \frac{1}{3k}\right) \\
        & \geqslant \sum_{k = n + 1}^{2n}\frac{1}{3k - 2} \\
        & \geqslant \frac{1}{3}\sum_{k = n + 1}^{2n}\frac{1}{k - 1} \\
        & \geqslant \frac{1}{6}
    \end{align*}
    由 Cauchy 收敛原理,该级数不收敛.
    \item [(2)] 相似的,
    \begin{align*}
        S_{6n} - S_{3n} \leqslant \frac{1}{6}
    \end{align*}    
    由 Cauchy 收敛原理,该级数不收敛.
\end{itemize}

\section*{P39 T4}

不成立. 反例 $x_n = \frac{1}{n}$ \\

\section*{P39 T5}

不一定. \\
去 $x_n = \frac{(-1)^n}{\sqrt{n}}$, $\sif x_n$ 收敛. 取 $y_n = x_n + \frac{1}{n}$, 有 $\nti \frac{x_n}{y_n} = 1$, 但 $\sif y_n$ 不收敛.

\section*{P39 T6}

不一定. 取 
\[
    x_n = \begin{cases}
        \frac{1}{n},\quad & 2 \mid n \\
        \frac{1}{n^2}, \quad & 2 \nmid n
    \end{cases}
\]

\section*{P39 T7}

是.
\begin{proof}
    由 Lebniz 定理, $\nti x_n = A > 0$. 因此当 $n > N$ 时, $x_n > \frac{A}{2}$. 
    \begin{align*}
        \sif \left(\frac{1}{1 + x_n}\right)^n & \leqslant \sum_{i = 1}^{N}\left(\frac{1}{1 + x_i}\right)^i + \sum_{i = N + 1}^{\infty}\left(\frac{1}{1 + \frac{A}{2}}\right)^i \\
        & < \sum_{i = 1}^{N}\left(\frac{1}{1 + x_i}\right)^i + \sum_{i = N + 1}^{\infty}\left(\frac{1}{1 + \frac{A}{2}}\right)^N
    \end{align*}
    是收敛的.
\end{proof}

\section*{P39 T8}

\begin{proof}
    \begin{align*}
        \sif \frac{x_n}{n^\alpha} = \sif \frac{x_n}{n^{\alpha_0}}\cdot\frac{1}{n^{\alpha - \alpha_0}}
    \end{align*}
    由于 $\sum_{i = 1}^{n} \frac{x_i}{i^{\alpha_0}}$ 有界, $\frac{1}{n^{\alpha - \alpha_0}}$ 单调递减且趋近于 $0$, 由 D-A 判别法, $\sif \frac{x_n}{n^\alpha}$ 收敛.
\end{proof}

\section*{P39 T9}

\begin{proof}
    \begin{align*}
        \sum_{i = 1}^{n}x_i = \sum_{i = 1}^{n - 1}i(x_i - x_{i + 1}) + nx_n
    \end{align*}
    由于 $\sum_{i = 1}^{n - 1}i(x_i - x_{i + 1})$, $nx_n$ 均收敛,$\sum_{i = 1}^{n}x_i$ 收敛.
\end{proof}

\section*{P39 T10}

\begin{proof}
    \begin{align*}
        \sum_{i = m}^{n}x_iy_i = \sum_{i = 1}^{n - 1}(x_i - x_{i + 1})\sum_{j = 1}^{i}y_j + x_n\sum_{j = 1}^{n}y_j
    \end{align*}
    故对任意 $\varepsilon > 0$, $\exists N > 0, \forall n > m > N$, $\left\lvert \sum_{i = m + 1}^{n}y_i\right\rvert < \varepsilon$, $x_n \leqslant M$. 又 $\sif \left\lvert x_{n + 1} - x_n\right\rvert = M_0$因此
    \begin{align*}
        \left\lvert \sum_{i = m + 1}^{n}x_iy_i\right\rvert &= \left\lvert \sum_{i = m + 1}^{n}(x_i - x_{i + 1})\sum_{j = m + 1}^{i}y_j + x_n \sum_{j = m + 1}^{n}y_j \right\rvert \\
        &\leqslant (M_0 + M)\varepsilon
    \end{align*}
    由 Cauchy 收敛准则, $\sif x_ny_n$ 收敛.
\end{proof}

\section*{P39 T11}

\begin{proof}
    由 L'Hospital 定理, $f(0) = f'(0) = 0$. 又 $f(x) = f(0) + f'(0)x + \frac{1}{2}f''(0)x^2 + o(x^2) = \frac{1}{2}f''(0)x^2 + o(x^2)$, 因此若 $f''(0) \neq 0$, $\nti \left|f\left(\frac{1}{n}\right)\right| \bigg/ \left(\frac{1}{2}f''(0)\cdot\frac{1}{n^2}\right) = 1$, 而 $\sif \frac{1}{2}f''(0)\frac{1}{n^2}$ 收敛,有比较判别法显然 $\sif f\left(\frac{1}{n}\right)$ 绝对收敛. \\
    又 $f''(0) = 0$, 显然成立.        
\end{proof}

\section*{P39 T15}

\begin{itemize}
    \item [(1)]
    \begin{proof}
        令 $c_n = \sum_{i = 0}^{n}\frac{1}{i!}\cdot\frac{(-1)^{n}}{(n - i)!}$ 有
        \begin{align*}
            c_n = \frac{1}{n!}\sum_{i = 0}^{n}\binom{n}{i}(-1)^i = 0\quad (n \geqslant 1) \\
            c_0 = 1
        \end{align*}
        因此 $\sifz \frac{1}{n!} \cdot \sifz \frac{(-1)^n}{n!} = 1$
    \end{proof}
    \item [(2)]
    \begin{proof}
        令 $c_n = \sum_{i = 0}^{n}q^i\cdot q_{n - i} = (n + 1)q^n$, 则原式化为
        \begin{align*}
            \left(\sifz q^n\right)^2 = \nti \sum_{k = 0}^{n}(n + 1)q^n = \frac{1}{(q - 1)^2}
        \end{align*}
    \end{proof}
\end{itemize}

\end{document}