\documentclass{article}
\usepackage{amsmath}  % 数学符号包
\usepackage{amssymb}  % 更多数学符号
\usepackage{enumitem} % 列表样式
\usepackage{fancyhdr} % 页眉设置
\usepackage{geometry} % 页面设置
\usepackage[UTF8]{ctex}
\usepackage{bm}
\usepackage{amsthm}
\everymath{\displaystyle}  % 让所有数学模式都使用 \displaystyle
\newcommand{\lb}{\left\llbracket}
\newcommand{\rb}{\right\rrbracket}
\newcommand{\dd}{\mathrm{d}}

\geometry{a4paper, margin=1in}


\pagestyle{fancy}
\fancyhf{}
\fancyhead[C]{下作业八}
\fancyhead[R]{2025.4.10}


\title{下作业八}
\author{Noflowerzzk}
\date{2025.4.10}


\begin{document}
\maketitle

\section*{P145 T1}

\begin{itemize}
    \item [(8)] 
    \begin{align*}
        \frac{\partial u}{\partial x} &= yf_1\left(xy, \frac{x}{y}\right) + \frac{1}{y}f_2\left(xy, \frac{x}{y}\right) \\
        \frac{\partial u}{\partial y} &= xf_1\left(xy, \frac{x}{y}\right) - \frac{x}{y^2}f_2\left(xy, \frac{x}{y}\right) \\
        \frac{\partial^2 u}{\partial x \partial y} &= f_1\left(xy, \frac{x}{y}\right) - \frac{1}{y^2}f_2\left(xy, \frac{x}{y}\right) + xyf_{11}\left(xy, \frac{x}{y}\right) - \frac{x}{y^2}f_{22}\left(xy, \frac{x}{y^3}\right) \\
        \frac{\partial^2 u}{\partial y^2} &= \frac{2x}{y^3}f_2\left(xy, \frac{x}{y}\right) + x^2f_{11}\left(xy + \frac{x}{y}\right) - \frac{2x^2}{y}f_{12}\left(xy + \frac{x}{y}\right) + \frac{x^2}{y^4}f_{22}\left(xy + \frac{x}{y}\right)
    \end{align*}
    \item [(9)] \begin{align*}
        \frac{\partial u}{\partial x} &= 2xf'(x^2 + y^2 + z^2) \\
        \frac{\partial u}{\partial y} &= 2yf'(x^2 + y^2 + z^2) \\
        \frac{\partial u}{\partial z} &= 2zf'(x^2 + y^2 + z^2) \\
        \frac{\partial^2 u}{\partial x^2} &= 2f'(x^2 + y^2 + z^2) + 4x^2f''(x^2 + y^2 + z^2) \\
        \frac{\partial^2 u}{\partial x \partial y} &= 4xyf''(x^2 + y^2 + z^2) 
    \end{align*}
    \item [(10)] 
    \begin{align*}
        \frac{\partial w}{\partial u} &= f_x + f_y + vf_z \\
        \frac{\partial w}{\partial v} &= f_x - f_y + uf_z \\
        \frac{\partial^2 w}{\partial u \partial v} = f_{xx} + (u + v)f_{xz} - f_{yy} + (u - v)f_{yz} + f_z + uvf_{zz}
    \end{align*}
\end{itemize}

\section*{P144 T9}

\begin{proof}
    \begin{itemize}
        \item [(1)] 由于 $\frac{\partial f(tx, ty)}{\partial t} = nt^{n - 1}f(x, y) = xf_1(tx, ty) + yf_2(tx, ty)$. 代入 $t = 1$ 有 $x\frac{\partial f}{\partial x} + y\frac{\partial f}{\partial y} = nf$
        \item [(2)] 易得 $n = 1$. $x\frac{\partial z}{\partial x} + y\frac{\partial z}{\partial y} = z = \sqrt{x^2 + y^2}$
    \end{itemize}
\end{proof}

\section*{P144 T11}

$f'(u,v) = 2uv$ 


 $$
\begin{pmatrix}
\cos \theta & -r \sin \theta \\
r \cos \theta & \sin \theta
\end{pmatrix}
$$ 

所以

 $$
(fg)'(r,0) = f'(g(r,0)) g'(r,0) = 
\begin{pmatrix}
2r \cos \theta & -2r \sin \theta \\
\sin \theta & r \cos \theta
\end{pmatrix}
\begin{pmatrix}
r \cos \theta \\
r \sin \theta
\end{pmatrix}
=
\begin{pmatrix}2r^2 \cos^2 \theta \\
    2r^2 \cos \theta \sin \theta
    \end{pmatrix}
    $$ 

\section*{P144 T17}

取 $x = r\cos \theta, y = r\sin \theta$, 有
\begin{align*}
    \frac{\partial}{\partial r}f(x, y) = \frac{1}{r}(f_x(x, y) + f_y(x, y)) = 0
\end{align*}
故 $f(x, y)$ 只与 $\theta$ 有关. 又 $\lim_{(x, y) \to (0, 0)} = f(0, 0)$ 是常数,故对任意 $\theta$, 有 $f(x, y)$ 是常数.

\section*{P151 T2}

$f(x, y) = -14 - 13(x - 1) - 6(y - 2) + 5(x - 1)^2 - 12(x - 1)(y - 2) + 4(y - 2)^2 + 3(x - 1)^3 - 2(x - 1)^2(y - 2) - 2(x - 1)(y - 2)^2 + (y - 2)^3$

\section*{P151 T3}

$f(x, y) = xy - \frac{1}{2}xy^2 + 0\left(\left(\sqrt{x^2 + y^2}\right)^3\right)$

\section*{P151 T4}

$f(x, y) = 1 + (x + y) + \frac{1}{2}(x + y)^2 + \cdots + \frac{1}{n!}(x + y)^n + R_n$, $R_n = \frac{1}{(n + 1)!}(x + y)^{n + 1}\mathrm{e}^{\theta(x + y)}, \theta \in (0, 1)$.

\section*{P189 T1}

\begin{itemize}
    \item [(2)] $\begin{cases}
        f_x = 0 \\
        f_y = 0
    \end{cases}$ 得驻点 $(x, y) = (0, 0), (1, 1), (-1, -1)$. 由于 $f_{xx} = 2(6x^2 - 1), f_{xy} = -2, f_{yy} = 2(6y^2 - 1)$, 有 $H = 4(6x^2 - 1)(6y^2 - 1) - 4$, 得 $(1, 1), (-1, -1)$ 是极值点. 又 $f(x, x)$ 在 $(0, 0)$ 附近小于0, $f(x, -x)$ 附近大于0, 故 $f(x, y)$ 在 $(0, 0)$ 变号,不是极值点.
    \item [(4)] 求得驻点 $(0, 0), (1, 1), (-1, 1), \left(\frac{\sqrt{2}}{2}, \frac{3}{8}\right), \left(\frac{-\sqrt{2}}{2}, \frac{3}{8}\right)$. 又 $H = 2(30x^4 - 12x^2y - 2y) - (4x^3 + 2x)^2$ 得 $\left(\frac{\sqrt{2}}{2}, \frac{3}{8}\right), \left(\frac{-\sqrt{2}}{2}, \frac{3}{8}\right)$ 上取极小值 $-\frac{1}{64}$.
    \item [(6)] 求得驻点 $\left(2^\frac{1}{4}, 2^\frac{1}{2}, 2^\frac{3}{4}\right)$, 又 Hesse 矩阵正定,故取得极小值 $4\times 2^\frac{1}{4}$
\end{itemize}

\section*{P189 T2}

\begin{align*}
    f(x, y, z) = \frac{1}{2}(x - 2y)^2 + \frac{1}{2}(x + 2z) + y^2
\end{align*}
最小值为 0, 当 $x = y = z = 0$ 是取到.

\section*{P189 T6}

\begin{align*}
    S = \frac{R^2}{2}(\sin \alpha_1 + \sin \alpha_2 - \sin (\alpha_1 + \alpha_2))
\end{align*}

求导得当 $\alpha_1 = \alpha_2 = \frac{2\pi}{3}$ 有 $S_{\text{max}} = \frac{3\sqrt{3}}{4}R^2$.

\section*{P189 T11}

设圆为单位圆,两个顶角为 $2\alpha, 2\beta$.
\begin{align*}
    S = \cot \alpha + \cot \beta + \tan (\alpha + \beta) 
\end{align*}

求偏导计算得 $\alpha = \beta = \frac{\pi}{2} - \alpha - \beta$ 时取到极值. 即 $\alpha = \beta = \frac{\pi}{6}$.

\section*{P189 T12}

同理设单位圆,各边圆心角为 $\alpha_i$.

\begin{align*}
    S = \frac{1}{2}(\sin \alpha_1 + \cdots + \sin \alpha_n)
\end{align*}

求偏导计算得

\begin{align*}
    \frac{\partial S}{\partial \alpha_k} = \frac{1}{2}(\cos \alpha_k - \cos (\alpha_1 + \cdots + \alpha_{n - 1})) = 0
\end{align*}

解得 $\alpha_k = \frac{2\pi}{n}$, 即正 $n$ 边形时面积最大.

\end{document}