\documentclass{article}
\usepackage{amsmath}  % 数学符号包
\usepackage{amssymb}  % 更多数学符号
\usepackage{enumitem} % 列表样式
\usepackage{fancyhdr} % 页眉设置
\usepackage{geometry} % 页面设置
\usepackage[UTF8]{ctex}
\usepackage{bm}
\usepackage{amsthm}
\everymath{\displaystyle}  % 让所有数学模式都使用 \displaystyle
\newcommand{\lb}{\left\llbracket}
\newcommand{\rb}{\right\rrbracket}


\geometry{a4paper, margin=1in}


\pagestyle{fancy}
\fancyhf{}
\fancyhead[C]{作业三}
\fancyhead[R]{2025.3.5}


\title{作业三}
\author{Noflowerzzk}
\date{2025.3.5}


\begin{document}
\maketitle

\section*{P60 T3}

\begin{proof} \quad
    \begin{itemize}
        \item [(1)] 取 $S(x) \equiv 0$. 有 $\left\lvert S_n(x) - S(x)\right\rvert = \frac{x}{1 + n^2x^2} \to 0$, $(n \to 0)$. 故 $S_n(x) \rightrightarrows S(x) \equiv 0 $. $S_n(x)$ 在 $\mathbb{R}$ 上一致收敛.
        \item 令 $T_n(x) = \frac{\mathrm{d}}{\mathrm{d}x}S_n(x) = \frac{1 - n^2x^2}{(1 + n^2x^2)^2}$. 取 $T(x) = \lim_{n \to \infty}T_n(x) = \begin{cases}
            1 & x = 0 \\
            0 & x \neq 0
        \end{cases}$
        取 $x = \frac{1}{2n}$, $T_n(x) - T(x) = \frac{12}{25} \neq 0$. 故不一致收敛.
        \item [(3)] $x = 0$ 时, $\frac{\mathrm{d}}{\mathrm{d}x}\lim_{n \to \infty}S_n(x) = 0$ 而 $\lim_{n \to \infty}\frac{\mathrm{d}}{\mathrm{d}x}S_n(x) = 1$.
    \end{itemize}
\end{proof}

\section*{P61 T4}

$S_n'(x) = \frac{x^{n - 1}}{1 + x^{2n}}$, $\lim_{n \to \infty}S_n'(x) = \frac{1}{2}$, 但是 $S'(1) = 0$, 不成立.

\section*{P61 T5}

\begin{itemize}
    \item [(1)] $S(x) = \lim_{n \to \infty} S_n(x) = 0$, $S_n'(x) = n^\alpha \mathrm{e}^{-nx}(1 - nx) = 0$ 有 $x = \frac{1}{n}$. 此时计算 $d(S_n(x), S(x)) = $
    \item [(2)] $\int_{0}^{1}\lim_{n \to \infty}S_n(x)\mathrm{d}x = \int_{0}^{1}S(x)\mathrm{d}x = 0$, $\lim_{n \to \infty}\int_{0}^{1}S_n(x)\mathrm{d}x = n^{\alpha - 2} - n^{\alpha - 1}(1 - nx)$, 而 $\lim_{n \to \infty}\mathrm{e}^{-nx}(1 - nx) = \begin{cases}
        0, & x \in (0, 1] \\
        1, & x = 0
    \end{cases}$
    故当 $\alpha < 0$ 时有上式成立.
\end{itemize}

\section*{P118 T3}

\begin{proof}
    由于 $f(x, y)$ 是初等函数,故其连续. \\
    而由于 $d\left(\left(1 - \frac{1}{n}, 1 - \frac{1}{n}\right), \left(1 - \frac{1}{2n}, 1 - \frac{1}{2n}\right)\right) \to 0$, 而 
    \begin{align*}
        \lim_{n \to \infty}f\left(1 - \frac{1}{n}, 1 - \frac{1}{n}\right) - \lim_{n \to \infty}f\left(1 - \frac{1}{2n}, 1 - \frac{1}{2n}\right) = \lim_{n \to \infty} \frac{(4n - 3)n^2}{(4n - 1)(2n - 1)} = \infty
    \end{align*}
    故 $f(x, y)$ 不一致连续.
\end{proof}

\section*{P118 T5}

\begin{proof}
    \begin{itemize}
        \item [(1)] 任取 $(x_0, y_0)$. 由题意知,存在 $R > 0$, 任意 $x^2 + y^2 > R^2$ 有 $f(x, y) > f(x_0, y_0)$. 同时, $f(x, y)$ 在 $\overline{B}((0, 0), R)$ 上有界且连续,有最小值. 故 $f(x, y)$ 在 $\mathbb{R}^2$ 上有最小值.
        \item [(2)] 任取 $(x_0, y_0)$. 若 $f(x, y) \equiv 0$, 则显然成立. \\
        否则取 $(x_0, y_0)$ 使得 $f(x_0, y_0) \neq 0$. \\
        若 $f(x_0, y_0) > 0$, 存在 $R > 0$, 任意 $x^2 + y^2 > R^2$, 有 $f(x, y) < f(x_0, y_0)$. 而 $f(x, y)$ 在闭集 $\overline{B}((0, 0), R)$ 中必有最大值. 故 $f(x, y)$ 在 $\mathbb{R}^2$ 上有最大值. \\
        若 $f(x_0, y_0) < 0$, 取 $-f(x, y)$ 即可.
    \end{itemize}
\end{proof}

\section*{P118 T6}

\begin{proof}
    由于任意 $\left\lVert \boldsymbol{x}\right\rVert = 1$ $f(\boldsymbol{x})$ 有界,故 $f(\boldsymbol{x})$ 有最大值和最小值,设为 $a, b$. \\
    现任取 $\boldsymbol{x} \in \mathbb{R}^n$有 
    \begin{align*}
        f(\boldsymbol{x}) = f\left(\left\lVert \boldsymbol{x}\right\rVert \frac{\boldsymbol{x}}{\left\lVert \boldsymbol{x}\right\rVert}\right) = \left\lVert \boldsymbol{x}\right\rVert f\left(\frac{\boldsymbol{x}}{\left\lVert \boldsymbol{x}\right\rVert}\right) \leq b \left\lvert \boldsymbol{x}\right\rvert \\
        f(\boldsymbol{x}) = f\left(\left\lVert \boldsymbol{x}\right\rVert \frac{\boldsymbol{x}}{\left\lVert \boldsymbol{x}\right\rVert}\right) = \left\lVert \boldsymbol{x}\right\rVert f\left(\frac{\boldsymbol{x}}{\left\lVert \boldsymbol{x}\right\rVert}\right) \geq a \left\lvert \boldsymbol{x}\right\rvert
    \end{align*}
\end{proof}

\section*{P118 T7}

\begin{proof}
    $\forall \boldsymbol{x} \in \overline{A}$, 若 $\boldsymbol{x} \in A$, 则显然 $f(\boldsymbol{x}) \in f(A) \subseteq \overline{f(A)}$ \\
    若 $\boldsymbol{x} \in \partial A$, 则 $\forall \varepsilon > 0$, $\exists \delta > 0, \boldsymbol{x}_0 \in A$, $\left\lVert \boldsymbol{x} - \boldsymbol{x}_0\right\rVert < \delta$ 时, 有 $\left\lVert f(\boldsymbol{x}) - f(\boldsymbol{x}_0)\right\rVert < \varepsilon$. 故 $f(\boldsymbol{x}) \in \overline{f(A)}$ \\
    故 $f(\overline{A}) \subseteq \overline{f(A)}$.
\end{proof}

\section*{P118 T8}

\begin{proof}
    \begin{itemize}
        \item [(1)] 任取 $\boldsymbol{\xi} \in \partial D$, 存在一个点列 $\{\boldsymbol{x}_n\}_{n = 1}^{+\infty}$ 收敛到 $\boldsymbol{\xi}$. 由于 $f(\boldsymbol{x})$ 一致连续,故由于 $\{\boldsymbol{x}_n\}$ 是 Cauchy 列, 故 $\{f(\boldsymbol{x}_n)\}$ 也是 Cauchy 列. 故 $\{f(\boldsymbol{x}_n)\}$ 收敛. 设其收敛到 $g(\boldsymbol{\xi})$. \\
        由于收敛的定义,$\forall \varepsilon > 0$, $\exists N > 0$, 当 $n > N$ 时,有 $\left\lVert f(\boldsymbol{x}_n) - g(\boldsymbol{\xi})\right\rVert < \frac{\varepsilon}{2}$. 对任意 $\boldsymbol{x} \in D$, 可以取上述 Cauchy 列且存在 $n > N$ $\left\lVert \boldsymbol{x} - \boldsymbol{x}_n\right\rVert < \frac{\varepsilon}{2}$. 故有
        \begin{align*}
            \left\lVert f(\boldsymbol{x}) - g(\boldsymbol{\xi})\right\rVert \leq \left\lVert f(\boldsymbol{x}) - f(\boldsymbol{x}_n)\right\rVert + \left\lVert f(\boldsymbol{x}_n) - g(\boldsymbol{\xi})\right\rVert < \varepsilon
        \end{align*}
        故取 $\tilde{f}(\boldsymbol{x}) = \begin{cases}
            f(\boldsymbol{x}), & \boldsymbol{x} \in D \\
            g(\boldsymbol{x}), & \boldsymbol{x} \in \partial D
        \end{cases}$ 在 $\overline{D}$ 上连续.
        \item [(2)] 易知 $\tilde{f}(\boldsymbol{x})$ 在 $\overline{D}$ 上连续,故 $\tilde{f}(\boldsymbol{x})$ 在 $\overline{D}$ 上有界. 设其界为 $\left\lvert \tilde{f}(\boldsymbol{x})\right\rvert \leq M$. \\
        因此 $$\left\lvert f(\boldsymbol{x})\right\rvert \leq \left\lvert f(\boldsymbol{x}) - \tilde{f}(\boldsymbol{x})\right\rvert + \left\lvert \tilde{f}(\boldsymbol{x})\right\rvert \leq \varepsilon + M$$
        故 $f(\boldsymbol{x})$ 在 $D$ 上有界.
    \end{itemize}
\end{proof}

\end{document}