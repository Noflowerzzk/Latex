\documentclass{article}
\usepackage{amsmath}  % 数学符号包
\usepackage{amssymb}  % 更多数学符号
\usepackage{enumitem} % 列表样式
\usepackage{fancyhdr} % 页眉设置
\usepackage{geometry} % 页面设置
\usepackage[UTF8]{ctex}
\usepackage{bm}
\usepackage{amsthm}
\everymath{\displaystyle}  % 让所有数学模式都使用 \displaystyle
\newcommand{\lb}{\left\llbracket}
\newcommand{\rb}{\right\rrbracket}
\newcommand{\dd}{\mathrm{d}}
\newcommand{\nti}{\sum_{n = 1}^{\infty}}


\geometry{a4paper, margin=1in}


\pagestyle{fancy}
\fancyhf{}
\fancyhead[C]{作业七}
\fancyhead[R]{2025.4.3}


\title{作业七}
\author{Noflowerzzk}
\date{2025.4.3}


\begin{document}
\maketitle

\section*{P135 T1}

\begin{align*}
    (11) & \quad \frac{\partial u}{\partial x} = (3x^2 + y^2 + z^2) e^{x(x^2 + y^2 + z^2)}, \quad \frac{\partial u}{\partial y} = 2xy e^{x(x^2 + y^2 + z^2)}, \quad \frac{\partial u}{\partial z} = 2xz e^{x(x^2 + y^2 + z^2)}. \\
    (12) & \quad \frac{\partial u}{\partial x} = \frac{y}{z} x^{z-1}, \quad \frac{\partial u}{\partial y} = \frac{\ln x}{z} x^{\frac{y}{z}}, \quad \frac{\partial u}{\partial z} = -\frac{y \ln x}{z^2} x^{\frac{y}{z}}. \\
    (13) & \quad \frac{\partial u}{\partial x} = -\frac{x}{(x^2 + y^2 + z^2)^{\frac{3}{2}}}, \quad \frac{\partial u}{\partial y} = -\frac{y}{(x^2 + y^2 + z^2)^{\frac{3}{2}}}, \quad \frac{\partial u}{\partial z} = -\frac{z}{(x^2 + y^2 + z^2)^{\frac{3}{2}}}. \\
    (14) & \quad \frac{\partial u}{\partial x} = y^z x^{z-1}, \quad \frac{\partial u}{\partial y} = z y^{z-1} x y^z \ln x, \quad \frac{\partial u}{\partial z} = y^z x^z \ln x \ln y. \\
    (15) & \quad \frac{\partial u}{\partial x_i} = a_i, \quad i = 1, 2, \cdots, n. \\
    (16) & \quad \frac{\partial u}{\partial x_i} = \sum_{j=1}^{n} a_{ij} y_j, \quad i = 1, 2, \cdots, n, \quad \frac{\partial u}{\partial y_j} = \sum_{i=1}^{n} a_{ij} x_i, \quad j = 1, 2, \cdots, n.
\end{align*}

\section*{P135 T3}

\begin{proof}
    由于
    $$
        \frac{\partial z}{\partial x} = \frac{1}{y^2} e^{\frac{x}{y^2}}, \quad \frac{\partial z}{\partial y} = -\frac{2x}{y^3} e^{\frac{x}{y^2}},
    $$ 
    所以
    $$
        2x \frac{\partial z}{\partial x} + y \frac{\partial z}{\partial y} = 0.
    $$ 
\end{proof}

\section*{P135 T12}

\begin{proof}
    
$$
\lim_{(x,y) \to (0,0)} f(x,y) = \lim_{(x,y) \to (0,0)} \sqrt[3]{xy} = 0 = f(0,0),
$$ 

 $$
f_x(0,0) = \lim_{\Delta x \to 0} \frac{\sqrt[3]{\Delta x \cdot 0} - 0}{\Delta x} = 0, \quad f_y(0,0) = \lim_{\Delta y \to 0} \frac{\sqrt[3]{0 \cdot \Delta y} - 0}{\Delta y} = 0,
$$ 
所以函数在原点  $(0,0)$  连续且可偏导。取  $v = (\cos \alpha, \sin \alpha)$ ,则在该方向,

$$
\frac{\dd f}{\dd v} = \lim_{t \to 0^+} \frac{f(0 + t \cos \alpha, 0 + t \sin \alpha) - f(0,0)}{t}
$$ 

 $$
= \lim_{t \to 0^+} \frac{\sqrt[3]{t \cos \alpha \cdot t \sin \alpha}}{t} = \lim_{t \to 0^+} \frac{\sqrt[3]{\sin 2\alpha}}{\sqrt[3]{2t}},
$$ 
当  $\sin 2\alpha = 0$ ,即  $\alpha = \frac{k\pi}{2}$  时,极限存在且为零;当  $\sin 2\alpha \neq 0$ ,即  $\alpha \neq \frac{k\pi}{2}$  时,极限不存在。所以除方向  $\boldsymbol{e_i}$,$\boldsymbol{-e_i} (i = 1,2)$  外,在原点的沿其他方向的方向导数都不存在。

\end{proof}

\section*{P135 T13}

\begin{proof}
    \begin{align*}
        \frac{\left\lvert xy\right\rvert}{\sqrt{x^2 + y^2}} \leq \sqrt{x^2 + y^2} \to 0
    \end{align*} 故 $$\lim_{(x,y) \to (0,0)} \frac{\left\lvert xy\right\rvert}{\sqrt{x^2 + y^2}} = 0$$. 由此可得函数在原点连续。
    又 $f_x(0, 0) = f_y(0, 0) = 0$,但 $f(\Delta x, \Delta y) - f(0, 0) - (f_x(0, 0)\Delta x + f_y(0, 0)\Delta y) \neq o(\sqrt{\Delta x^2 + \Delta y^2})$. 故 $f(x)$ 在 $(0, 0)$ 不可微。
\end{proof}

\section*{P135 T16}

\begin{itemize}
    \item [(4)] \begin{align*}
        \frac{\partial^4 u}{\partial x^4} &= -\frac{3 \cdot 2a^3}{(ax + by + cz)^4} \frac{\partial (ax + by + cz)}{\partial x} = -\frac{6a^4}{(ax + by + cz)^4}, \\
        \frac{\partial^4 u}{\partial x^2 \partial y^2} &= \frac{\partial^4 u}{\partial y^2 \partial x^2} = -\frac{3 \cdot 2a^2b}{(ax + by + cz)^4} \frac{\partial (ax + by + cz)}{\partial y} = -\frac{6a^2b^2}{(ax + by + cz)^4}.
    \end{align*}
    \item [(5)] \begin{align*}
        \quad \frac{\partial^{p+q} z}{\partial x^p \partial y^q} &= \frac{\partial^p}{\partial x^p} \left( \frac{\partial^q z}{\partial y^q} \right) = \frac{\partial^p}{\partial x^p} \left( (x-a)^p \frac{\partial^q (y-b)^q}{\partial y^q} \right) \\
        \quad &= \frac{\mathrm{d}^p (x-a)^p}{\mathrm{d}x^p} \frac{\mathrm{d}^q (y-b)^q}{\mathrm{d}y^q} = p! \, q! \, . \\
    \end{align*}
    \item [(6)] 由 Leibniz 公式可得
    \begin{align*}
        \quad \frac{\partial^{p+q+r} u}{\partial x^p \partial y^q \partial z^r} &= \frac{\partial^p (x e^x)}{\partial x^p} \frac{\partial^q (y e^y)}{\partial y^q} \frac{\partial^r (z e^z)}{\partial z^r} \\
        & \quad = \frac{\mathrm{d}^p (x e^x)}{\mathrm{d}x^p} \frac{\mathrm{d}^q (y e^y)}{\mathrm{d}y^q} \frac{\mathrm{d}^r (z e^z)}{\mathrm{d}z^r} \\
        & \quad = (x+p) e^x \cdot (y+q) e^y \cdot (z+r) e^z \\
        & \quad = (x+p)(y+q)(z+r) e^{x+y+z}.
    \end{align*}
\end{itemize}

由 $x$ 的偏导数可得 $f(x, y) = -x \sin y - \frac{1}{y}\ln (1 - xy) + g(y)$. 令 $x = 0$ 有 $g(y) = 2 \sin y + y^3$. 因此 $f(x, y) = (2 - x)\sin y - \frac{1}{y}\ln (1 - xy) + y^3$

\section*{P135 T20}

\begin{itemize}
    \item [(1)] \begin{align*}
        f_1'(x, y, z) = (1, 0, 0), \quad f_2'(x, y, z) = (0, 1, 0), \quad f_3'(x, y, z) = (0, 0, 1), \\
    \end{align*}
    因此 $\boldsymbol{f}$ 的导数为单位阵.
    \item [(2)] $f_x'(x, y, z) = (1, 0, 0)$, 则 $f_x(x, y, z) = x + C_1$. 同理 $f_y(x, y, z) = y + C_2, f_z(x, y, z) = z + C_3$.
    \item [(3)] 类似 $(2)$ 有 $f_x(x, y, z) = \int p(x) \dd x, f_y(x, y, z) = \int q(y) \dd y, f_z(x, y, z) = \int r(z) \dd z$.
\end{itemize}

\end{document}