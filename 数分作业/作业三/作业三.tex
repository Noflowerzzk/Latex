\documentclass{article}
\usepackage{amsmath}  % 数学符号包
\usepackage{amssymb}  % 更多数学符号
\usepackage{enumitem} % 列表样式
\usepackage{fancyhdr} % 页眉设置
\usepackage{geometry} % 页面设置
\usepackage[UTF8]{ctex}
\usepackage{bm}
\usepackage{amsthm}


\geometry{a4paper, margin=1in}


\pagestyle{fancy}
\fancyhf{}
\fancyhead[C]{数分作业三}
\fancyhead[R]{2024.10.16}


\title{数分作业三}
\author{noflowerzzl}
\date{2024.10.16}


\begin{document}
\maketitle

\section{}

\begin{itemize}
    \item[(1)]
    \[
        \lim_{n \to \infty}\left(1 - \frac{1}{n}\right)^n 
        = \lim_{n \to \infty}\left(\left(1 + \frac{1}{n - 1}\right)^{n - 1} \left(1 + \frac{1}{n - 1}\right)\right)^{-1}
        = \frac{1}{\text{e}}    
    \]
    \item[(b)]
    \[
        \lim_{n \to \infty}\left(1 + \frac{1}{n + 1}\right)^{n}
        = \lim_{n \to \infty}\left(1 + \frac{1}{n + 1}\right)^{n + 1}\frac{n + 1}{n + 2}
        = \text{e}
    \]
    \item[(c)]\[
        \lim_{n \to \infty}\left(1 + \frac{1}{2n}\right)^n
        = \lim_{n \to \infty}\sqrt{\left(1 + \frac{1}{2n}\right)^{2n}}
        = \sqrt{\text{e}}
    \]
    \item[(4)]\[
        \lim_{n \to \infty}\left(1 + \frac{1}{n^2}\right)^n
        = \lim_{n \to \infty}\left(\left(1 + \frac{1}{n^2}\right)^{n^2}\right)^{\frac{1}{n}}
        = \lim_{n \to \infty}\text{e}^{\frac{1}{n}} = 1
    \]
    \item[(5)]
    由$\displaystyle{\left(1 + \frac{1}{n} - \frac{1}{n^2}\right)^n < \left(1 + \frac{1}{n}\right)^n}$, \\
    且$\displaystyle{\left(1 + \frac{1}{n} - \frac{1}{n^2}\right)^n
     = \left(1 + \frac{n - 1}{n^2}\right)^{\frac{n^2}{n - 1}\frac{n - 1}{n}}
     > \left(1 + \frac{n - 1}{n^2}\right)^{\frac{n^2}{n - 1}}}$ \\
    由夹逼原理,
    \[
        e = \lim_{n \to \infty}\left(1 + \frac{1}{n}\right)^n 
        = \lim_{n \to \infty}\left(1 + \frac{n - 1}{n^2}\right)^{\frac{n^2}{n - 1}}   
    \]
    所以
    \[
        \lim_{n \to \infty}\left(1 + \frac{1}{n} - \frac{1}{n^2}\right)^n = e
    \]
\end{itemize}

\section{}

\begin{itemize}
    \item[(1)]
    收敛.
    \begin{proof}
        由$x_1 = \sqrt{2} < 2, x_{n + 1} = \sqrt{x_n + 2} < 2 \Leftrightarrow x_n < 2 $, 归纳得$x_n < 2$. \\
        又$x_n < x_{n + 1} \Leftrightarrow (x_n + 1)(x_n - 2) < 0$, 由$0 < x < 2$知其成立.
        所以$x_n$单调递增有上界, 极限存在, 设为$A$. \\
        有$A = \sqrt{A + 2} \Rightarrow A = 2$.
    \end{proof}
    \item[(2)]
    收敛.
    \begin{proof}
        由$x_1 = \sqrt{2} < 2, x_{n + 1} = \sqrt{2x_n} < 2 \Leftrightarrow x_n < 2 $, 归纳得$x_n < 2$. \\
        又$x_n < x_{n + 1} \Leftrightarrow x_n^2 < 2x_n $, 由$0 < x < 2$知其成立.
        所以$x_n$单调递增有上界, 极限存在, 设为$A$. \\
        有$A = \sqrt{2A} \Rightarrow A = 2$.
    \end{proof}
    \item[(3)]
    收敛.
    \begin{proof}
        由$x_1 = \sqrt{2} > -1, x_{n + 1} = \frac{-1}{x_n + 2} > -1$, 归纳得$x_n > -1$ \\
        又$x_n > x_{n + 1} \Leftrightarrow (x_n + 1)^2 > 0 $, 由$x > -1$知其成立.
        所以$x_n$单调递减有下界, 极限存在, 设为$A$. \\
        有$A = \frac{-1}{A + 2} \Rightarrow A = -1$.
    \end{proof}
    \item[(4)]
    收敛.
    \begin{proof}
        由$x_1 = 1 < 4, x_{n + 1} = \sqrt{4 + 3x_n} < 4 \Leftrightarrow x_n < 4 $, 归纳得$x_n < 4$. \\
        又$x_n < x_{n + 1} \Leftrightarrow (x_n - 4)(x_n + 1) < 0 $, 由$0 < x < 4$知其成立.
        所以$x_n$单调递增有上界, 极限存在, 设为$A$. \\
        有$A = \sqrt{4 + 3A} \Rightarrow A = 4$.
    \end{proof}
    \item[(5)]
    收敛.
    \begin{proof}
        由$x_1 \in (0, 1), x_{n + 1} = 1 - \sqrt{1 - x_n} \in (0, 1) \Leftrightarrow x_n \in (0, 1) $, 归纳得$x_n \in (0, 1)$. \\
        又$x_n > x_{n + 1} \Leftrightarrow 1 - x_n > (1 - x_n)^2 $, 由$0 < x < 1$知其成立.
        所以$x_n$单调递减有下界, 极限存在, 设为$A < 1$. \\
        有$A = 1 - \sqrt{1 - A} \Rightarrow A = 0$.
    \end{proof}
    \item[(6)]
    收敛.
    \begin{proof}
        由$x_1 \in (0, 1), x_{n + 1} = x_n(1 - x_n) \in (0, 1) \Leftrightarrow x_n \in (0, 1) $, 归纳得$x_n \in (0, 1)$. \\
        又$x_n > x_{n + 1} \Leftrightarrow x_n > x_n(1 - x_n) $, 由$0 < x$知其成立.
        所以$x_n$单调递减有下界, 极限存在, 设为$A < 1$. \\
        有$A = A(1 - A) \Rightarrow A = 0$.
    \end{proof}
\end{itemize}

% 手动定义章节编号
\section*{15}
% 将条目添加到目录中,指定章编号和名称
\addcontentsline{toc}{section}{15}

\begin{proof}
    充分性: \\
    $\forall \varepsilon > 0$, $\exists N_1 > 0$, $n > N_1 \text{时,} \left\lvert \text{diam}A_n\right\rvert < \varepsilon $. \\
    由$\text{diam}A_k$定义, $\forall p, q > n > N_1, \left\lvert x_p - x_q\right\rvert \leqslant \left\lvert \text{diam}A_n\right\rvert < \varepsilon$. \\
    由柯西收敛准则, ${x_n}$收敛. \\
    必要性: \\
    由于$\{x_n\}$收敛, 由柯西收敛准则, $\forall \varepsilon > 0, \exists N > 0, p, q > N$时 $\left\lvert x_p - x_q\right\rvert < \varepsilon$. \\
    则$\text{diam}A_n = \sup {\left\lvert x_p - x_q\right\rvert } < \varepsilon$. \\
    即$\lim_{n \to \infty}\text{diam}A_n = 0$.
\end{proof}

% 手动定义章节编号
\section*{16}
% 将条目添加到目录中,指定章编号和名称
\addcontentsline{toc}{section}{16}

\begin{proof}
    设$\{x_n\}$单调递且有上界. \\
    假设其不收敛, 则$\exists \varepsilon_0 > 0, \forall N > 0, \exists m, n > N, \left\lvert x_m - x_n\right\rvert \geqslant \varepsilon_0$. \\
    取$N_1 = 1$, $\left\lvert x_{n_1} - x_{m_1}\right\rvert \geqslant \varepsilon_0$, 且$n_1 < m_1$\\
    取$N_2 = m_1$, $\left\lvert x_{n_2} - x_{m_2}\right\rvert \geqslant \varepsilon_0$, 且$n_2 < m_2$ \\
    $\cdots$ \\
    取$N_k = m_{k - 1}$, $\left\lvert x_{n_k} - x_{m_k}\right\rvert \geqslant \varepsilon_0$, 且$n_k < m_k$. \\
    有$x_{n_k} - x_{n_1} > k\varepsilon_0$, $\{x_{n_k}\}$无上界, 矛盾!
    所以$\{x_n\}$ 收敛.
\end{proof}

\end{document}