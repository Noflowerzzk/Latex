\documentclass{article}
\usepackage{amsmath}  % 数学符号包
\usepackage{amssymb}  % 更多数学符号
\usepackage{enumitem} % 列表样式
\usepackage{fancyhdr} % 页眉设置
\usepackage{geometry} % 页面设置
\usepackage[UTF8]{ctex}
\usepackage{bm}
\usepackage{amsthm}
\usepackage{listings}
\usepackage{xcolor}
\everymath{\displaystyle}  % 让所有数学模式都使用 \displaystyle
\newcommand{\lb}{\left\llbracket}
\newcommand{\rb}{\right\rrbracket}


\definecolor{codebg}{rgb}{0.95, 0.95, 0.95}    % 代码背景色
\definecolor{codeframe}{rgb}{0.8, 0.8, 0.8}    % 代码框边框色
\definecolor{codetext}{rgb}{0.1, 0.1, 0.1}    % 代码默认文本色
\definecolor{keywordcolor}{rgb}{0, 0, 0.8}    % 关键字颜色
\definecolor{commentcolor}{rgb}{0.3, 0.6, 0.3} % 注释颜色
\definecolor{stringcolor}{rgb}{0.8, 0, 0}     % 字符串颜色
\definecolor{numcolor}{rgb}{0.6, 0.2, 0.8}    % 数字颜色
\definecolor{funcname}{rgb}{0.75, 0.15, 0.25} % 函数名颜色

\usepackage{listings}
\lstdefinestyle{codestyle}{
    backgroundcolor=\color{codebg},
    frame=single,
    rulecolor=\color{codeframe},
    basicstyle=\ttfamily\small\color{codetext},
    keywordstyle=\color{keywordcolor}\bfseries,
    commentstyle=\color{commentcolor}\itshape,
    stringstyle=\color{stringcolor},
    numberstyle=\tiny\color{gray},
    identifierstyle=\color{codetext},
    emph={lu_decomposition, np},
    emphstyle=\color{funcname}\bfseries,
    numbers=left,
    stepnumber=1,
    breaklines=true,
    showstringspaces=false
}

\geometry{a4paper, margin=1in}


\pagestyle{fancy}
\fancyhf{}
\fancyhead[C]{作业五}
\fancyhead[R]{2025.3.23}


\title{作业五}
\author{Noflowerzzk}
\date{2025.3.23}

\begin{document}
\maketitle

\textcolor{gray}{本文代码均用 Python 实现}

\section{}

\begin{lstlisting}[style=codestyle, language=Python, title=LU 分解代码]
    import numpy as np
    
    def lu_decomposition(A):
        """ 对方阵 A 进行 LU 分解,返回 L 和 U """
        n = len(A)
        L = np.eye(n)
        U = A.astype(float).copy()
        
        for i in range(n):
            for j in range(i+1, n):
                if U[i, i] == 0:
                    raise ValueError("不存在 LU 分解!")
                factor = U[j, i] / U[i, i]
                L[j, i] = factor
                U[j, i:] -= factor * U[i, i:]
                
        return L, U
    
    # 测试
    A = np.array([[2, -1, 3],
              [1, 2, 1],
              [2, 4, -2]], dtype=float)
    L, U = lu_decomposition(A)
    
    print("L:")
    print(L)
    print("U:")
    print(U)
\end{lstlisting}

测试输出结果为

\begin{lstlisting}[style=codestyle, language=Python,title=输出结果]
    L:
        [[1.  0.  0. ]
        [0.5 1.  0. ]
        [1.  2.  1. ]]
    R:
        [[ 2.  -1.   3. ]
        [ 0.   2.5 -0.5]
        [ 0.   0.  -4. ]]
\end{lstlisting}

因此 $\det A = \det R = -20$.

\section{}

\begin{lstlisting}[style=codestyle, language=Python, title=Cholesky 分解代码]
    import numpy as np

    def cholesky_decomposition(A):
        """ 对称正定矩阵 A 进行 Cholesky 分解,返回 L """
        n = A.shape[0]
        L = np.zeros_like(A)

        for i in range(n):
            for j in range(i + 1):
                sum_k = sum(L[i, k] * L[j, k] for k in range(j))

                if i == j:
                    L[i, j] = np.sqrt(A[i, i] - sum_k)
                else:
                    L[i, j] = (A[i, j] - sum_k) / L[j, j]

        return L

    # 测试
    A = np.array([[5, -2, 0],
                [-2, 3, -1],
                [0, -1, 1]], dtype=float)

    L = cholesky_decomposition(A)
    print("L:\n", L)
    print("L.T:\n", L.T)    
\end{lstlisting}

测试输出结果为

\begin{lstlisting}[style=codestyle, language=Python,title=输出结果]
    L:
        [[ 2.23606798  0.          0.        ]
        [-0.89442719  1.4832397   0.        ]
        [ 0.         -0.67419986  0.73854895]]
    L.T:
        [[ 2.23606798 -0.89442719  0.        ]
        [ 0.          1.4832397  -0.67419986]
        [ 0.          0.          0.73854895]]
\end{lstlisting}

\section{}

仅需 $A$ 是正定阵即可,即其各阶顺序主子式大于0. 计算得结果为 $a \in (-\sqrt{3}, \sqrt{3})$.

\section{}

$u_1 = \begin{pmatrix}
    2 \\ -1 \\ 2
\end{pmatrix}$

$u_2 = \alpha_2 - \frac{\alpha_2 \cdot u_1}{u_1 \cdot u_1}u_1 = \frac{2}{9}\begin{pmatrix}
    -11 \\ 10 \\ 16
\end{pmatrix}$ \\

单位化即有 $Q = \begin{pmatrix}
    \frac{2}{3} & \frac{-11\sqrt{53}}{159} \\
    -\frac{1}{3} & \frac{10\sqrt{53}}{159} \\
    \frac{2}{3} & \frac{16\sqrt{53}}{159}
\end{pmatrix}$

$R = \begin{pmatrix}
    3 & -\frac{3}{7} \\
    0 & \frac{2\sqrt{53}}{3}
\end{pmatrix}$

\begin{lstlisting}[style=codestyle, language=Python, title=Householder 变换计算 QR 分解代码]
    import numpy as np

    def householder_qr(A):
        """ 使用 Householder 变换计算矩阵 A 的 QR 分解 """
        m, n = A.shape
        Q = np.eye(m)
        R = A.copy().astype(float)

        for k in range(n):
            # 选取列向量
            x = R[k:, k]
            
            # 计算法向量 v
            e1 = np.zeros_like(x)
            e1[0] = np.linalg.norm(x) * (1 if x[0] >= 0 else -1)
            v = x + e1
            v = v / np.linalg.norm(v)

            # 构造 Householder 变换矩阵 H_k = E - 2uu.T
            H_k = np.eye(m)
            H_k[k:, k:] -= 2.0 * np.outer(v, v)

            # 更新 R 和 Q
            R = H_k @ R
            Q = Q @ H_k.T  # 注意 Q 需要不断右乘 H_k 的转置

        return Q, R

    # 测试
    A = np.array([[3, 14, 9],
                [6, 43, 3],
                [6, 22, 15]], dtype=float)

    Q_A, R_A = householder_qr(A)

    print("Q_A:\n", Q_A)
    print("R_A:\n", R_A)

    B = np.array([[1, 1, 1],
                [2, -1, -1],
                [2, -4, 10]], dtype=float)

    Q_B, R_B = householder_qr(B)

    print("Q_B\n", Q_B)
    print("R_B:\n", R_B)
\end{lstlisting}

输出结果为

\begin{lstlisting}[style=codestyle, language=Python,title=输出结果]
    Q_A:
        [[-0.33333333  0.13333333  0.93333333]
        [-0.66666667 -0.73333333 -0.13333333]
        [-0.66666667  0.66666667 -0.33333333]]
    R_A:
        [[-9.00000000e+00 -4.80000000e+01 -1.50000000e+01]
        [ 1.77635684e-16 -1.50000000e+01  9.00000000e+00]
        [-4.21884749e-16 -6.21724894e-16  3.00000000e+00]]
    Q_B
        [[-0.33333333 -0.66666667  0.66666667]
        [-0.66666667 -0.33333333 -0.66666667]
        [-0.66666667  0.66666667  0.33333333]]
    R_B:
        [[-3.00000000e+00  3.00000000e+00 -6.33333333e+00]
        [-6.66133815e-16 -3.00000000e+00  6.33333333e+00]
        [ 6.66133815e-16  4.44089210e-16  4.66666667e+00]]
\end{lstlisting}

\textcolor{gray}{极小数值视为 0}

因此 $B^{-1} = R^{-1}Q^T$, 计算结果为

\begin{lstlisting}[style=codestyle, language=Python,title=B 的逆]
    [[ 3.33333333e-01  3.33333333e-01  3.81822733e-18]
    [ 5.23809524e-01 -1.90476190e-01 -7.14285714e-02]
    [ 1.42857143e-01 -1.42857143e-01  7.14285714e-02]]
\end{lstlisting}

\section{}

\begin{lstlisting}[style=codestyle, language=Python,title=Givens 变换计算 QR 分解代码]
    import numpy as np

    def givens_rotation(A):
        """ 使用 Givens 变换计算矩阵 A 的 QR 分解 """
        m, n = A.shape
        Q = np.eye(m)
        R = A.copy().astype(float)
    
        for j in range(n):
            for i in range(m - 1, j, -1):  # 从底部向上遍历行
                if R[i, j] != 0:  # 只在非零元素时进行旋转
                    a, b = R[j, j], R[i, j]
                    r = np.hypot(a, b)  # 计算 sqrt(a^2 + b^2) 避免溢出
                    c, s = a / r, -b / r
                    
                    # 构造 Givens 旋转矩阵
                    G = np.eye(m)
                    G[[j, i], [j, i]] = c
                    G[j, i], G[i, j] = -s, s
    
                    R = G @ R
                    Q = Q @ G.T
    
        return Q, R
    
    # 测试
    A = np.array([[2, 2, 1],
                  [0, 2, 2],
                  [2, 1, 2]], dtype=float)
    
    Q, R = givens_rotation(A)
    
    print("Q:\n", Q)
    print("R:\n", R)    
\end{lstlisting}

\begin{lstlisting}[style=codestyle, language=Python,title=输出结果]
    Q:
        [[ 0.70710678  0.23570226 -0.66666667]
        [ 0.          0.94280904  0.33333333]
        [ 0.70710678 -0.23570226  0.66666667]]
    R:
        [[ 2.82842712e+00  2.12132034e+00  2.12132034e+00]
        [ 0.00000000e+00  2.12132034e+00  1.64991582e+00]
        [ 0.00000000e+00 -4.96469267e-17  1.33333333e+00]]
\end{lstlisting}



\end{document}