\documentclass{article}
\usepackage{amsmath}  % 数学符号包
\usepackage{amssymb}  % 更多数学符号
\usepackage{enumitem} % 列表样式
\usepackage{fancyhdr} % 页眉设置
\usepackage{geometry} % 页面设置
\usepackage{bm}
\usepackage[UTF8]{ctex}
\usepackage{cases}


\geometry{a4paper, margin=1in}

% \makeatletter
% \newcommand\xleftrightarrow[2][]{%
%   \ext@arrow 9999{\longleftrightarrowfill@}{#1}{#2}}
% \newcommand\longleftrightarrowfill@{%
%   \arrowfill@\leftarrow\relbar\rightarrow}
% \makeatother


\pagestyle{fancy}
\fancyhf{}
\fancyhead[C]{Exercise 02}
\fancyhead[R]{2024.9.26}


\title{Exercise 02}
\author{Noflowerzzk}
\date{2024.9.26}


\begin{document}

\maketitle

\section{Answer\_2}


\begin{itemize}
    \item [(1)] \[
        \boldsymbol{A} = 
        \begin{pmatrix}
            a_{11} & a_{11} & \cdots & a_{11} \\
            a_{21} & a_{21} & \cdots & a_{21} \\
            \vdots & \vdots &        & \vdots \\
            a_{n1} & a_{n1} & \cdots & a_{n1}
        \end{pmatrix}
    \]

    \item [(2)] \[
        \boldsymbol{A} = 
        \begin{pmatrix}
            0 & 0 \\
            0 & 0 \\
            0 & 2
        \end{pmatrix}
    \]
    \item[(3)] \[
        \boldsymbol{A}=
        \begin{pmatrix}
            2 & -1 & 0 & 0 \\
            -1 & 2 & -1 & 0 \\
            0 & -1 & 2 & -1 \\
            0 & 0 & -1 & 2
        \end{pmatrix}
    \]
    \item[(4)] \[
        \boldsymbol{A}=
        \begin{pmatrix}
            1 & 0 & 0 & 0 \\            
            1 & 1 & 0 & 0 \\
            1 & 1 & 1 & 0 \\
            1 & 1 & 1 & 1 \\
        \end{pmatrix}
    \]
\end{itemize}

\section{Answer\_3}

\text{
    由矩阵相等的定义, 有
}


\begin{numcases}{}
    a + 2b = 4 \\
    2a - b = -2 \\
    2c + d = 4 \\
    c - 2d = -3
\end{numcases}

\text{
    由$(1)(2)$解得
    $
        \begin{cases}
            a = 0 \\
            b = 2
        \end{cases}
    $
    ,
    由$(3)(4)$解得
    $
        \begin{cases}
            c = 1 \\
            d = 2
        \end{cases}
    $
}

\section{Answer\_4}


\begin{itemize}
    \item[(1)] 
        $
        \begin{pmatrix}
            1 & 2 & 2 & -1 \\
            2 & 3 & 3 & 1 \\
            3 & 4 & 4 & 3
        \end{pmatrix}
        \xrightarrow[r_3 - 2r_1]{r_2 - 2r_1}
        \begin{pmatrix}
            1 & 2 & 2 & -1 \\
            0 & -1 & -1 & 3 \\
            0 & -2 & -2 & 6
        \end{pmatrix}
        \xrightarrow{r_3 - 2r_2}
        \begin{pmatrix}     
            1 & 2 & 2 & -1 \\
            0 & -1 & -1 & 3 \\
            0 & 0 & 0 & 0
        \end{pmatrix}
        \xrightarrow{-r_2}   
        \begin{pmatrix}     
            1 & 2 & 2 & -1 \\
            0 & 1 & 1 & -3 \\
            0 & 0 & 0 & 0
        \end{pmatrix}
        \xrightarrow{r_1 - 2r_2}
        \begin{pmatrix}     
            1 & 0 & 0 & 5 \\
            0 & 1 & 1 & -3 \\
            0 & 0 & 0 & 0
        \end{pmatrix}
        $

    \item[(2)]
        $
        \begin{pmatrix}
            0 & 2 & -3 & 1 \\
            0 & 3 & -4 & 3 \\
            0 & 4 & -7 & -1
        \end{pmatrix}
        \xrightarrow[r_4 - 2r_1]{r_2 - 1.5r_1}
        \begin{pmatrix}
            0 & 2 & -3 & 1 \\
            0 & 0& \frac{1}{2} & \frac{3}{2} \\
            0 & 0 & -1 & -3
        \end{pmatrix}
        \xrightarrow{r_3 + 2r_2}
        \begin{pmatrix}
            0 & 2 & -3 & 1 \\
            0 & 0& \frac{1}{2} & \frac{3}{2} \\
            0 & 0 & 0 & 0
        \end{pmatrix}
        \xrightarrow[\frac{1}{2}r_1]{2r_2}        
        \begin{pmatrix}
            0 & 1 & -\frac{3}{2} & \frac{1}{2} \\
            0 & 0 & 1 & 3 \\
            0 & 0 & 0 & 0
        \end{pmatrix}
        \xrightarrow{r_1 + \frac{3}{2}r_2}
        \begin{pmatrix}
            0 & 1 & 0 & 5 \\
            0 & 0 & 1 & 3 \\
            0 & 0 & 0 & 0
        \end{pmatrix}
        $

    \item[(3)]
        $
        \begin{pmatrix}
            1 & -1 & 3 & -4 & 3 \\
            3 & -3 & 5 & -4 & 1 \\
            2 & -2 & 3 & -2 & 6 \\
            3 & -3 & 4 & -2 & -1
        \end{pmatrix}
        \xrightarrow[r_2 - 3r_1]{r_3 - 2r_1, r_4 - 3r_1}
        \begin{pmatrix}
            1 & -1 & 3 & -4 & 3 \\
            0 & 0 & -4 & 8 & -8 \\
            0 & 0 & -3 & 6 & 0 \\
            0 & 0 & -5 & 10 & -10
        \end{pmatrix}
        \xrightarrow{\frac{1}{-4}r_2}
        \begin{pmatrix}
            1 & -1 & 3 & -4 & 3 \\
            0 & 0 & 1 & -2 & 2 \\
            0 & 0 & -3 & 6 & 0 \\
            0 & 0 & -5 & 10 & -10
        \end{pmatrix}
        \xrightarrow[r_3 + 3r_2]{r_4 + 5r_2}
        \begin{pmatrix}
            1 & -1 & 3 & -4 & 3 \\
            0 & 0 & 1 & -2 & 2 \\
            0 & 0 & 0 & 0 & 6 \\
            0 & 0 & 0 & 0 & 0
        \end{pmatrix}
        \xrightarrow{\frac{1}{6}r_3}
        \begin{pmatrix}
            1 & -1 & 3 & -4 & 3 \\
            0 & 0 & 1 & -2 & 2 \\
            0 & 0 & 0 & 0 & 1 \\
            0 & 0 & 0 & 0 & 0
        \end{pmatrix}
        \xrightarrow{r_1 - 3r_3}
        \begin{pmatrix}
            1 & -1 & 0 & 2 & -3 \\
            0 & 0 & 1 & -2 & 2 \\
            0 & 0 & 0 & 0 & 1 \\
            0 & 0 & 0 & 0 & 0
        \end{pmatrix}
        \xrightarrow[r_1 + 3 r_3]{r_2 - 2r_3}
        \begin{pmatrix}
            1 & -1 & 0 & 2 & 0 \\
            0 & 0 & 1 & -2 & 0 \\
            0 & 0 & 0 & 0 & 1 \\
            0 & 0 & 0 & 0 & 0
        \end{pmatrix}
        $
        \item[(5)]
        $
        \begin{pmatrix}
            1 & 2 & 3 & 4 \\
            2 & 3 & 4 & 5 \\
            5 & 4 & 5 & 2
        \end{pmatrix}
        \xrightarrow{r_2 - 2r_1, r_3 - 5r_1}
        \begin{pmatrix}
            1 & 2 & 3 & 4 \\
            0 & -1 & -2 & -3 \\
            0 & -6 & -10 & -18
        \end{pmatrix}
        \xrightarrow{r_2 \times (-1)}
        \begin{pmatrix}
            1 & 2 & 3 & 4 \\
            0 & 1 & 2 & 3 \\
            0 & -6 & -10 & -18
        \end{pmatrix}
        \xrightarrow{r_3 + 6r_2}
        \begin{pmatrix}
            1 & 2 & 3 & 4 \\
            0 & 1 & 2 & 3 \\
            0 & 0 & 2 & 0
        \end{pmatrix}
        \xrightarrow{\frac{1}{2}r_3}
        \begin{pmatrix}
            1 & 2 & 3 & 4 \\
            0 & 1 & 2 & 3 \\
            0 & 0 & 1 & 0
        \end{pmatrix}
        \xrightarrow{r_2 - 2r_3}
        \begin{pmatrix}
            1 & 2 & 3 & 4 \\
            0 & 1 & 0 & 3 \\
            0 & 0 & 1 & 0
        \end{pmatrix}
        \xrightarrow{r_1 - 3r_3, r_1 - 2r_2}
        \begin{pmatrix}
            1 & 0 & 0 & -2 \\
            0 & 1 & 0 & 3 \\
            0 & 0 & 1 & 0
        \end{pmatrix}
        $
        \item[(6)]
        $
        \begin{pmatrix}
            1 & 1 & 3 & 3 \\
            0 & 2 & -1 & 2 \\
            1 & -2 & 2 & 3 \\
            0 & 1 & 1 & 4
        \end{pmatrix}
        \xrightarrow{r_3 - r_1}
        \begin{pmatrix}
            1 & 1 & 3 & 3 \\
            0 & 2 & -1 & 2 \\
            0 & -3 & -1 & 0 \\
            0 & 1 & 1 & 4
        \end{pmatrix}
        \xrightarrow{r_4 - \frac{1}{2}r_2, r_3 + \frac{3}{2}r_2}
        \begin{pmatrix}
            1 & 1 & 3 & 3 \\
            0 & 2 & -1 & 2 \\
            0 & 0 & -\frac{5}{2} & -3 \\
            0 & 0 & \frac{3}{2} & 3
        \end{pmatrix}
        \xrightarrow[\frac{2}{3}r_4]{-2r_3}
        \begin{pmatrix}
            1 & 1 & 3 & 3 \\
            0 & 2 & -1 & 2 \\
            0 & 0 & 5 & 6 \\
            0 & 0 & 1 & 2
        \end{pmatrix}
        \xrightarrow{r_3 \leftrightarrow r_4}
        \begin{pmatrix}
            1 & 1 & 3 & 3 \\
            0 & 2 & -1 & 2 \\
            0 & 0 & 1 & 2 \\
            0 & 0 & 5 & 6 
        \end{pmatrix}
        \xrightarrow[subscript]{r_4 + 5r_3}
        \begin{pmatrix}
            1 & 1 & 3 & 3 \\
            0 & 2 & -1 & 2 \\
            0 & 0 & 1 & 2 \\
            0 & 0 & 0 & -4 
        \end{pmatrix}
        \xrightarrow[r_1 - 3 r_3]{r_2 + r_3, -\frac{1}{4}r_4}
        \begin{pmatrix}
            1 & 1 & 0 & -3 \\
            0 & 2 & 0 & 4 \\
            0 & 0 & 1 & 2 \\
            0 & 0 & 0 & 1 
        \end{pmatrix}
        \xrightarrow{\frac{1}{2}r_2}
        \begin{pmatrix}
            1 & 1 & 0 & -3 \\
            0 & 1 & 0 & 2 \\
            0 & 0 & 1 & 2 \\
            0 & 0 & 0 & 1 
        \end{pmatrix}
        \xrightarrow{r_1 - r_2}
        \begin{pmatrix}
            1 & 0 & 0 & -5 \\
            0 & 1 & 0 & 2 \\
            0 & 0 & 1 & 2 \\
            0 & 0 & 0 & 1 
        \end{pmatrix}
        \xrightarrow[r_3 - 2r_4]{R_1 + 5r_4, r_2 - 2r_4}
        \begin{pmatrix}
            1 & 0 & 0 & 0 \\
            0 & 1 & 0 & 0 \\
            0 & 0 & 1 & 0 \\
            0 & 0 & 0 & 1 
        \end{pmatrix}
        % \overset{\text{上方文字}}{\underset{\text{下方文字}}{\xleftrightarrow{\hspace{2cm}}}}
        $
        \item[(7)]
        $
        \begin{pmatrix}
            1 & 2 & 0 & 3 \\
            4 & 7 & 1 & 10 \\
            0 & 1 & -1 & 2 \\
            2 & 3 & 1 & 4
        \end{pmatrix}
        \xrightarrow[r_4 - 2r_1]{r_2 - 4r_1}
        \begin{pmatrix}
            1 & 2 & 0 & 3 \\
            0 & -1 & 1 & -2 \\
            0 & 1 & -1 & 2 \\
            0 & -1 & 1 & -2
        \end{pmatrix}
        \xrightarrow{r_2 \leftrightarrow r_3}
        \begin{pmatrix}
            1 & 2 & 0 & 3 \\
            0 & 1 & -1 & 2 \\
            0 & -1 & 1 & -2 \\
            0 & -1 & 1 & -2
        \end{pmatrix}
        \xrightarrow[r_4 + r_2]{r_3 + r_2}
        \begin{pmatrix}
            1 & 2 & 0 & 3 \\
            0 & 1 & -1 & 2 \\
            0 & 0 & 0 & 0 \\
            0 & 0 & 0 & 0
        \end{pmatrix}
        \xrightarrow{r_1 - 2r_2}
        \begin{pmatrix}
            1 & 0 & 2 & -1 \\
            0 & 1 & -1 & 2 \\
            0 & 0 & 0 & 0 \\
            0 & 0 & 0 & 0
        \end{pmatrix}        
        $
        \item[(8)]
        $
        \begin{pmatrix}
            1 & 1 & 3 \\
            -1 & 2 & 3 \\
            1 & 3 & 7
        \end{pmatrix}
        \xrightarrow{r_2 + r_1, \ r_3 - r_1}
        \begin{pmatrix}
            1 & 1 & 3 \\
            0 & 3 & 6 \\
            0 & 2 & 4
        \end{pmatrix}
        \xrightarrow{\frac{1}{3}r_2}
        \begin{pmatrix}
            1 & 1 & 3 \\
            0 & 1 & 2 \\
            0 & 2 & 4
        \end{pmatrix}
        \xrightarrow{r_3 - 2r_2, \ r_1 - r_2}
        \begin{pmatrix}
            1 & 0 & 1 \\
            0 & 1 & 2 \\
            0 & 0 & 0
        \end{pmatrix}           
        $
\end{itemize}

\section{Answer\_5}

\begin{itemize}
    \item[(1)]
    该方程组的增广矩阵是
    $$\tilde{A} = \begin{pmatrix}
        2 & 4 & -1 & 6 \\
        1 & -2 & 1 & 4 \\
        3 & 6 & 2 & -1
        \end{pmatrix}$$
    化简为行阶梯形矩阵为
    $$\begin{pmatrix}
        1 & 2 & -\frac{1}{2} & 3 \\
        0 & 1 & -\frac{3}{8} & -\frac{1}{4} \\
        0 & 0 & \frac{7}{2} & -10
        \end{pmatrix}$$
    所以$r = s = n$, 方程组有唯一解.
    \item[(2)]
    该方程组的增广矩阵是
    $$\tilde{A} = \begin{pmatrix}
        2 & 4 & -1 & 6 \\
        1 & 2 & 1 & 4 \\
        3 & 6 & 2 & -1
        \end{pmatrix}$$
    化简为行阶梯形矩阵为
    $$\begin{pmatrix}
        1 & 2 & -\frac{1}{2} & 3 \\
        0 & 0 & 3 & 2 \\
        0 & 0 & 0 & 1
        \end{pmatrix}
        $$
    所以$r > s$, 方程组无解.
    \item[(3)]
    该方程组的增广矩阵是
    $$\tilde{A} = \begin{pmatrix}
        2 & 4 & -1 & 6 \\
        1 & 2 & 1 & 3 \\
        3 & 6 & 2 & 9
        \end{pmatrix}
        $$
    化简为行阶梯形矩阵为
    $$\begin{pmatrix}
        1 & 2 & -\frac{1}{2} & 3 \\
        0 & 0 & 1 & 0 \\
        0 & 0 & 0 & 0
        \end{pmatrix}
        $$
    所以$r = s < n$, 方程组有无数解.
    \item[(4)]
    该方程组的增广矩阵是
    $$\tilde{A} = \begin{pmatrix}
        3 & -2 & 1 & -2 \\
        6 & -4 & 2 & -5 \\
        -9 & 6 & -3 & 6
        \end{pmatrix}        
        $$
    化简为行阶梯形矩阵为
    $$\begin{pmatrix}
        1 & -\frac{2}{3} & \frac{1}{3} & -\frac{2}{3} \\
        0 & 0 & 0 & 1 \\
        0 & 0 & 0 & 0
        \end{pmatrix}        
        $$
    所以$r > s$, 方程组无解.
    \item[(5)]
    该方程组的增广矩阵是
    $$\tilde{A} = \begin{pmatrix}
        2 & 4 & a \\
        3 & 6 & 5
        \end{pmatrix}        
    $$
    化简为行阶梯形矩阵为
    $$
    \begin{pmatrix}
        3 & 6 & 5 \\
        0 & 0 & a - 5
    \end{pmatrix}
    $$
    方程组无解$\Leftrightarrow  r > s \Leftrightarrow  a - 5 \neq 0 \Leftrightarrow a \neq 5$.
    \item[(6)]
    该方程组的增广矩阵是
    $$\tilde{A} = \begin{pmatrix}
        3 & a & 3 \\
        a & 3 & 5
        \end{pmatrix}        
    $$
    化简为行阶梯形矩阵为
    $$
    \begin{pmatrix}
        3 & a & 3 \\
        0 & 9 - a^2 & 15 - 3a
    \end{pmatrix}
    $$
    方程组无解 $\Leftrightarrow r > s \Leftrightarrow \begin{cases}
        9 - a^2 = 0 \\
        15 - 3a \neq 0
    \end{cases} \Leftrightarrow a = \pm 3$. 

\end{itemize}

\section{Answer\_6}

\begin{itemize}
    \item[(1)]
    该方程组的增广矩阵是
    $$\tilde{A} = \begin{pmatrix}
        1 & 1 & 2 & -1 & 0 \\
        2 & 1 & 1 & -1 & 0 \\
        2 & 2 & 1 & 2 & 0
    \end{pmatrix}       
    $$
    化简为行阶梯形矩阵为
    $$
    \begin{pmatrix}
        1 & 0 & 0 & -\frac{4}{3} & 0 \\
        0 & 1 & 0 & 3 & 0 \\
        0 & 0 & 1 & -\frac{4}{3} & 0
    \end{pmatrix}
    $$
    所以方程组的解为:
    $$
    \begin{cases} 
        x_1 = \frac{4}{3}x_4 \\
        x_2 = -3x_4 \\
        x_3 = \frac{4}{3} \\
        x_4 = x_4
    \end{cases}
    $$
    \item[(4)]
    该方程组的增广矩阵是
    $$\tilde{A} = \begin{pmatrix}
        3 & 4 & -5 & 7 & 0 \\
        2 & -3 & 3 & -2 & 0 \\
        4 & 11 & -13 & 16 & 0 \\
        7 & -2 & 1 & 3 & 0
    \end{pmatrix}
    $$
    进行一定的初等行变换, 并将未知量按次序$x_4, X_3, X_2, X_1$得到:
    $$
    \begin{pmatrix}
        1 & 0 & 0 & 0 & 0 \\
        0 & 1 & 0 & 17 & 0 \\
        0 & 0 & 1 & 5 & 0 \\
        0 & 0 & 0 & 1 & 0
    \end{pmatrix}
    $$
    所以$r = s = n$, 原方程组只有零解, 即
    $$
    \begin{cases}
        x_1 = 0 \\
        x_2 = 0 \\
        x_3 = 0 \\
        x_4 = 0 
    \end{cases}
    $$
\end{itemize}

\section{Answer\_7}

\begin{itemize}
    \item[(1)]
    该方程组的增广矩阵是
    $$\tilde{A} = \begin{pmatrix}
        1 & 2 & 3 & 1 \\
        2 & 2 & 5 & 2 \\
        3 & 5 & 1 & 3
    \end{pmatrix}
    $$
    化简为行阶梯形矩阵为
    $$
    \begin{pmatrix}
        1 & 0 & 0 & 1 \\
        0 & 1 & 0 & 0 \\
        0 & 0 & 1 & 0
    \end{pmatrix}
    $$
    所以方程组的解为:
    $$
    \begin{cases} 
        x_1 = 1 \\
        x_2 = 0 \\
        x_3 = 0
    \end{cases}
    $$
    \item[(2)]
    该方程组的增广矩阵是
    $$\tilde{A} = \begin{pmatrix}
        1 & -2 & -1 & 2 \\
        2 & -1 & -3 & 1 \\
        3 & 2 & -5 & 0
    \end{pmatrix}
    $$
    化简为简化行阶梯形矩阵为
    $$
    \begin{pmatrix}
        1 & 0 & 0 & 5 \\
        0 & 1 & 0 & 0 \\
        0 & 0 & 1 & 3
    \end{pmatrix}
    $$
    所以方程组的解为:
    $$
    \begin{cases} 
        x_1 = 5 \\
        x_2 = 0 \\
        x_3 = 3
    \end{cases}
    $$
    \item[(3)]
    该方程组的增广矩阵是
    $$\tilde{A} = \begin{pmatrix}
        4 & 2 & -1 & 2 \\
        3 & -1 & 2 & 10 \\
        11 & 3 & 0 & 8
    \end{pmatrix}    
    $$
    化简为简化行阶梯形矩阵为
    $$
    \begin{pmatrix}
        1 & 0 & 0.3 & 0 \\
        0 & 1 & -1.1 & 0 \\
        0 & 0 & 0 & 1
    \end{pmatrix}
    $$
    由于$r > s$, 方程组无解.
    \item[(4)]
    该方程组的增广矩阵是
    $$\tilde{A} = \begin{pmatrix}
        2 & 3 & 1 & 4 \\
        1 & -2 & 4 & -5 \\
        3 & 8 & -2 & 13 \\
        4 & -1 & 9 & -6
    \end{pmatrix}    
    $$
    化简为简化行阶梯形矩阵为
    $$
    \begin{pmatrix}
        1 & 0 & 2 & -1 \\
        0 & 1 & -1 & 2 \\
        0 & 0 & 0 & 0 \\
        0 & 0 & 0 & 0
    \end{pmatrix}
    $$
    所以方程组的解为:
    $$
    \begin{cases} 
        x_1 = -1 - 2x_3 \\
        x_2 = 2 + x_3 \\
        x_3 = x_3
    \end{cases}
    $$
    \item[(5)]
    该方程组的增广矩阵是
    $$\tilde{A} = \begin{pmatrix}
        2 & 1 & -1 & 1 & 1 \\
        4 & 2 & -2 & 1 & 2 \\
        2 & 1 & -1 & -1 & 1
    \end{pmatrix}    
    $$
    化简为简化行阶梯形矩阵为
    $$
    \begin{pmatrix}
        1 & \frac{1}{2} & -\frac{1}{2} & 0 & \frac{1}{2} \\
        0 & 0 & 0 & 1 & 0 \\
        0 & 0 & 0 & 0 & 0
    \end{pmatrix}
    $$
    所以方程组的解为:
    $$
    \begin{cases} 
        x_1 = -\frac{1}{2}x_2 + \frac{1}{2}x_3 + \frac{1}{2} \\
        x_2 = x_2 \\
        x_3 = x_3
        x_4 = 0
    \end{cases}
    $$
\end{itemize}

\section{Answer\_8}

\begin{itemize}
    \item[(1)]
    该方程组的增广矩阵是
    $$
    \begin{pmatrix}
        -2 & 1 & 1 & -2 \\
        1 & -2 & 1 & p \\
        1 & 1 & -2 & p^2
    \end{pmatrix}
    $$
    化简为行阶梯形矩阵为
    $$
    \begin{pmatrix}
        -2 & 1 & 1 & -2 \\
        0 & 0 & 3 & p + 3 \\
        0 & 0 & 0 & p^2 + p - 2
    \end{pmatrix}
    $$
    所以,
    \begin{itemize}
        \item 方程组无解 $\Leftrightarrow p^2 + p - 2 \neq 0 \Leftrightarrow p \neq -2 \text{且} p \neq 1$
        \item 由于$ n = r \geqslant s$, 方程组不可能有无穷解.
        \item 方程组有唯一解$\Leftrightarrow p^2 + p - 2 = 0 \Leftrightarrow p = -2 \text{或} p = 1$
            \begin{itemize}
                \item $p = -2$时, 有
                $$\begin{cases}
                    x_1 = \frac{7 + 3x_2}{6} \\
                    x_2 = x_2 \\
                    x_3 = \frac{1}{3}
                \end{cases}
                $$
                \item $p = 1$时, 有
                $$
                \begin{cases}
                    x_1 = \frac{10 + 3x_2}{6} \\
                    x_2 = x_2 \\
                    x_3 = \frac{4}{3}
                \end{cases}
                $$
            \end{itemize}
    \end{itemize}
    \item[(2)]
    该方程组的增广矩阵是
    $$
    \tilde{A} = \begin{pmatrix}
        1 & 2 & 0 & 3 \\
        4 & 7 & 1 & 10 \\
        0 & 1 & -1 & q \\
        1 & 1 & p & p^2
    \end{pmatrix}
    $$
    化简为行阶梯形矩阵为
    $$
    \begin{pmatrix}
        1 & -2 & 0 & 3 \\
        0 & -1 & 1 & -2 \\
        0 & 0 & p - 1 & 0 \\
        0 & 0 & 0 & q - 2
        \end{pmatrix}
    $$
    \begin{itemize}
        \item 方程组无解 $\Leftrightarrow r > s \Leftrightarrow q \neq 2$
        \item 方程组有无数解 $\Leftrightarrow r = s < n \Leftrightarrow q = 2 \text{且} p = 1$.
            此时方程组的解为
            $$\begin{cases}
                x_1 = x_3 + 7 \\
                x_2 = x_3 + 2 \\
                x_3 = x_3
            \end{cases}
            $$
        \item 方程组有唯一解 $\Leftrightarrow r = s = n \Leftrightarrow q = 2 \text{且} p \neq 1 $.
            此时方程组的解为
            $$
            \begin{cases}
                x_1 = 7 \\
                x_2 = 2 \\
                x_3 = 0
            \end{cases}
            $$
    \end{itemize}

    \item[(3)]
    该方程组的增广矩阵是
    $$
    \tilde{A} = \begin{pmatrix}
        p & 1 & 1 & 1 \\
        1 & p & 1 & p \\
        1 & 1 & p & p^2
    \end{pmatrix}
    $$
    化简为行阶梯形矩阵为
    $$
    \begin{pmatrix}
        (p - 1)(p + 2) & 0 & 0 & -p^2 + 1 \\
        0 & (p - 1)(p + 2) & 0 & p - 1 \\
        0 & 0 & (p - 1)(p + 2) & (p + 1)^2(p - 1)
    \end{pmatrix}
    $$
    \begin{itemize}
        \item 方程组无解 $\Leftrightarrow r > s \Leftrightarrow p = -2$.
        \item 方程组有无数解 $\Leftrightarrow r = s < n \Leftrightarrow p = 1$.
            此时方程组的解为
            $$\begin{cases}
                x_1 = 1 - x_2 - x_3\\
                x_2 = x_2 \\
                x_3 = x_3
            \end{cases}
            $$
        \item 方程组有唯一解 $\Leftrightarrow r = s = n \Leftrightarrow p \neq -2 \text{且} p \neq 1 $.
            此时方程组的解为
            $$
            \begin{cases}
                x_1 = -\frac{1 + p}{2 + p} \\
                x_2 = \frac{1}{p + 2} \\
                x_3 = \frac{(p + 1)^2}{p + 2}
            \end{cases}
            $$
    \end{itemize}
    \item[(4)]
    该方程组的增广矩阵是
    $$
    \tilde{A} = \begin{pmatrix}
        1 + p & 1 & 1 & 0 \\
        1 & 1 + p & 1 & p \\
        1 & 1 & 1 + p & p^2
    \end{pmatrix}    
    $$
    化简为行阶梯形矩阵为
    $$
    \begin{pmatrix}
        p(p + 3) & 0 & 0 & -p^2 - p \\
        0 & p(p + 3) & 0 & -p \\
        0 & 0 & p(p + 3) & -p
    \end{pmatrix}
    $$
    \begin{itemize}
        \item 方程组无解 $\Leftrightarrow r > s \Leftrightarrow p = -3$.
        \item 方程组有无数解 $\Leftrightarrow r = s < n \Leftrightarrow p = 0$.
            此时方程组的解为
            $$\begin{cases}
                x_1 = - x_2 - x_3\\
                x_2 = x_2 \\
                x_3 = x_3
            \end{cases}
            $$
        \item 方程组有唯一解 $\Leftrightarrow r = s = n \Leftrightarrow p \neq -3 \text{且} p \neq 0 $.
            此时方程组的解为
            $$
            \begin{cases}
                x_1 = -\frac{1 + p}{3 + p} \\
                x_2 = -\frac{p}{p + 3} \\
                x_3 = -\frac{1}{p + 3}
            \end{cases}
            $$
    \end{itemize}

\end{itemize}

\end{document}