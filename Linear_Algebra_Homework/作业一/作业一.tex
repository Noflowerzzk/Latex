\documentclass{article}
\usepackage{amsmath}  % 数学符号包
\usepackage{amssymb}  % 更多数学符号
\usepackage{enumitem} % 列表样式
\usepackage{fancyhdr} % 页眉设置
\usepackage{geometry} % 页面设置
\usepackage[UTF8]{ctex}
\usepackage{bm}
\usepackage{amsthm}
\everymath{\displaystyle}  % 让所有数学模式都使用 \displaystyle
\newcommand{\lb}{\left\llbracket}
\newcommand{\rb}{\right\rrbracket}


\geometry{a4paper, margin=1in}


\pagestyle{fancy}
\fancyhf{}
\fancyhead[C]{作业 1}
\fancyhead[R]{2025.2.23}


\title{作业 1}
\author{noflowerzzk}
\date{2025.2.23}


\begin{document}
\maketitle

\section{}

\begin{proof}
    一方面,任意 $f(x) = a_0 + a_1x + \cdots + a_nx^n$, 有若 $2 | n$, 则 $g(x) = a_0 + a_2x^2 + \cdots + a_nx^n \in U$, $h(x) = a_1x + \cdots + a_{n - 1}x^{n - 1} \in W$. 故 $U + W = V$. \\
    另一方面,显然 $U \cap W = \{0\}$. \\
    因此 $U \oplus W = V$.
\end{proof}

\section{}

\begin{proof}
    \item [(1) $\Rightarrow$ (2)]
    对 $n$ 归纳. \\
    $n = 2$ 时,上课已经证明. \\
    假设 $n = k \geq 2$ 时,有 $\bigoplus_{i = 1}^k W_i = V$ 有 $\forall \alpha \in V$, 有唯一分解 $\alpha = \alpha_1 + \cdots + \alpha_k$, $\alpha_i \in W_i$. 则 $n = k + 1$ 时,有 $V = \bigoplus_{i = 1}^{k + 1}W_i$, 有 $W_{k + 1} \cap \sum_{i = 1}^{k}W_i = \{0\}$. 故由归纳假设 $n = 2$ 情形, $\forall \alpha \in V$, $\alpha$ 可唯一分解为 $\alpha = \alpha_k' + \alpha_{k + 1}$, 其中 $\alpha_k' \in \sum_{i = 1}^{k}W_i, \alpha_{k + 1} \in W_{k + 1}$. 又由归纳假设, $\alpha_k'$ 有在 $\sum_{i = 1}^{k}W_i$ 的唯一分解 $\alpha_k' = \sum_{i = 1}^{k}\alpha_i$, 且 $\alpha_{k + 1}$ 与 $\alpha_i\ (i \leq k)$ 互不相同. 故是唯一分解.
    \item [(2) $\Rightarrow$ (3)] 显然
    \item [(3) $\Rightarrow$ (4)] 对 $n$ 归纳,$n = 2$ 时,上课已经证明. \\
    假设 $n = k$ 时, $W_i, i = 1, 2, \cdots k$ 的基构成 $\sum_{i = 1}^{k}W_i$ 的一组基,则 $n = k + 1$ 时,由于 $\mathbf{0}$ 在 $W_1, \cdots W_k$ 中有唯一分解,且由于 $\mathbf{0}$ 在 $\sum_{i = 1}^{k}W_i$
    和 $W_{k + 1}$ 中有唯一分解,由归纳假设, $\sum_{i = 1}^{k}W_i$
    和 $W_{k + 1}$ 的基构成 $\sum_{i = 1}^{k + 1}W_i$ 的基. 因此 $W_i$ 的基构成 $\sum_{i = 1}^{l + 1}W_i$ 的基. 
    \item [(4) $\Rightarrow$ (1)]
    对 $n$ 归纳, $n = 2$ 时,上课已经证明. \\
    假设 $n = k \geq 2$ 时有 $W_i \cap \sum_{j = 1}^{i - 1}W_i = \{0\}$,则 $n = k + 1$ 时,由 $n = 2$ 的结论,有 $W_{k + 1} \cap \sum_{i = 1}^{k}W_i = \{0\}$, 即有 $\bigoplus_{i = 1}^{k + 1}W_i$
\end{proof}

\section{}

\begin{proof}
    由维数公式, $\dim U \leq \sum_{i = 1}^{n}\dim U_i$. 故 "=" 成立有 $\forall i, \dim U_i + \dim (U_1 + \cdots + U_{i - 1}) = \dim (U_1 + \cdots + U_i)$, 即根据 $n = 2$ 的情形, $U_i \cap \sum_{j = 1}^{i - 1}U_j = \{0\}$, 即 $V = \bigoplus_{i = 1}^{n}U_i$.
\end{proof}

\section{}

\begin{proof}
    一方面,任意 $\alpha, \alpha' \in W$,有 $\forall u \in U$, $(\alpha, u) = (\alpha', u) = 0$. 故 $(\alpha + \alpha', u) = 0$, 有 $\aleph + \alpha' \in W$. \\
    另一方面,任意 $k \in F$, 有 $(k\alpha, u) = k(\alpha, u) = 0$, $k\alpha \in W$. 故 $W$ 是 $V$ 的子空间.
\end{proof}

\section{}

$\mathbf{C}$ 到 $\mathbf{B}$ 的过渡矩阵是 
\[
    P = 
    \begin{pmatrix}
        1 & 3 & 0 \\
        -2 & -5 & 2 \\
        1 & 4 & 3
    \end{pmatrix}
\]
因此 $\mathbf{B}$ 到 $\mathbf{C}$ 的过渡矩阵为
\[
    P^{-1} = 
    \begin{pmatrix}
        -23 & -9 & 6 \\
        8 & 3 & -2 \\
        -3 & -1 & 1
    \end{pmatrix}
\]

$1 + 2t$ 在 $\mathbf{C}$ 中为 $\alpha = (1, 2, 0)^T$, 故在 $\mathbf{B}$ 中为
\[
    P\alpha = 
    \begin{pmatrix}
        7 \\ -12 \\ 9
    \end{pmatrix}
\]

\section{}

$\mathbf{C}$ 到 $\mathbf{B}$ 的过渡矩阵是 
\[
    P = 
    \begin{pmatrix}
        1 & 2 & 1 \\
        0 & 1 & 2 \\
        -3 & -5 & 0
    \end{pmatrix}
\]

因此 $\mathbf{B}$ 到 $\mathbf{C}$ 的过渡矩阵为
\[
    P^{-1} = 
    \begin{pmatrix}
        10 & -5 & 3 \\
        -6 & 3 & -2 \\
        3 & -1 & 1
    \end{pmatrix}
\]

$t^2$ 在 $\mathbf{C}$ 中为 $\alpha = (0, 0, 1)^T$, 故在 $\mathbf{B}$ 中为
\[
    P\alpha = 
    \begin{pmatrix}
        1 \\ 2 \\ 0
    \end{pmatrix}
\]

\section{}

\begin{itemize}
    \item [(1)]
    \[
        P
        \begin{pmatrix}
            u_1  u_2 \\ u_3
        \end{pmatrix} = 
        \begin{pmatrix}
            v_1 \\ v_2 \\ v_3
        \end{pmatrix}
    \]
    故由 
    \[
        P^{-1} = \begin{pmatrix}
            5 & 8 & 5 \\
            -3 & -5 & -3 \\
            -2 & -2 & -1
        \end{pmatrix}
    \]
    有
    \[
        v_1 = \begin{pmatrix}
            -109 \\ 60 \\ 61
        \end{pmatrix}\ 
        v_2 = \begin{pmatrix}
            67 \\ -37 \\ -37
        \end{pmatrix}\   
        v_3 = \begin{pmatrix}
            27 \\ -16 \\ -16
        \end{pmatrix}
    \]
    \item [(2)]
    同理有 
    \[
        w_1 = \begin{pmatrix}
            -11 \\ 10 \\ 1
        \end{pmatrix}\ 
        w_2 = \begin{pmatrix}
            46 \\ -31 \\ -19
        \end{pmatrix}\   
        w_3 = \begin{pmatrix}
            -63 \\ 40 \\ 30
        \end{pmatrix}
    \]
\end{itemize}

\section{}

\begin{proof}
    易验证 $2^{k + 3} - 2\cdot 2^{k + 2} + 9 \cdot 2^{k + 1} - 18 \cdot 2^k = 0$. \\
    令 $z = 3\left(\cos \frac{\pi}{2} + \mathrm{i} \sin\frac{\pi}{2}\right)$, 有 $z^k = 3^k \cos \frac{k\pi}{2} + \mathrm{i}\sin \frac{k\pi}{2}$.
    由于
    \begin{align*}
        z^{k + 3} - 2z^{k + 2} + 9 z^{k + 1} - 18z^{k} = z^{k}\left(z^3 - 2z^2 + 9z - 18\right).
    \end{align*}
    又 $z$ 是方程 $z^3 - 2z^2 + 9z - 18 = 0$ 的根,故 $z^{k + 3} - 2z^{k + 2} + 9 z^{k + 1} - 18z^{k} = 0$, 即其实部虚部均为 0.

    取 $k = 0$, 有 Casorati 矩阵 
    \[
        C(0) = \begin{pmatrix}
            1 & 1 & 1 \\
            2 & 3 & 0 \\
            4 & 0 & -9
        \end{pmatrix}
    \]
    易验证 $C(0)$ 可逆,故三个型号线性无关,构成其解集的一组基.
\end{proof}

\section{}

\begin{itemize}
    \item [(1)] 由于 $\dim(S + T) = \dim S + \dim T - \dim(S \cap T)$. 由于 $\dim (S + T) \leq 10$, $1 \leq \dim (S \cap T)$. 又 $\dim (S \cap T ) \leq \dim S$, 有 $\dim (S \cap T)$ 取值为 1 或 2.
    \item [(2)] $1 \leq \dim (S \cap T) \leq 2$ 有 $\dim (S + T)$ 取值为 7 或 8.
    \item [(3)] $\dim S^\bot + \dim S = 10$, $\dim S^\bot = 8$
\end{itemize}

\end{document}