\documentclass{article}
\usepackage{amsmath}  % 数学符号包
\usepackage{amssymb}  % 更多数学符号
\usepackage{enumitem} % 列表样式
\usepackage{fancyhdr} % 页眉设置
\usepackage{geometry} % 页面设置
\usepackage[UTF8]{ctex}
\usepackage{graphicx}
\usepackage{pgf}
\usepackage{caption}
% \usepackage{bm}
\usepackage{amsthm}
\usepackage{minted}  % 用于显示代码
% \usepackage{pythontex}  % 用于运行 Python 代码
\usepackage[backend=biber,style=gb7714-2015]{biblatex}

\definecolor{mybgcolor}{rgb}{0.95, 0.95, 0.95}  % RGB 范围在 [0, 1] 之间

\everymath{\displaystyle}  % 让所有数学模式都使用 \displaystyle
\newcommand{\lb}{\left\llbracket}
\newcommand{\rb}{\right\rrbracket}
\newcommand{\bs}[1]{\boldsymbol{#1}}
\newcommand{\RR}{\mathbb{R}}
\newcommand{\EE}{\mathbb{E}}
\newcommand{\dd}{\mathrm{d}}
\newcommand{\WW}{\bs{W}_\mathrm{whiten}}
\newcommand{\XC}{\bs{X}_\mathrm{c}}
\newcommand{\np}{\indent\par}
% \defbibheading{bibliography}[参考文献]{\section*{#1}}

\def\mathdefault#1{#1}

\geometry{a4paper, margin=1in}


\pagestyle{fancy}
\fancyhf{}
\fancyhead[C]{独立成分分析(ICA)与其在音频信号分离中的应用}
\fancyhead[R]{2025.5.3}


\title{独立成分分析(ICA)与其在音频信号分离中的应用}
\author{张桢锴}
\date{2025.5.3}


\begin{document}
\maketitle

\begin{abstract}
    本文简要介绍了独立成分分析(ICA)特别是 FastICA 算法的基本原理与实现流程。ICA 利用信号的非高斯性与统计独立性,从多个观测信号中恢复原始独立源信号。文中详细阐述了数学建模、信号预处理(中心化与白化)、固定点迭代过程,并通过代码与实验演示了 FastICA 在模拟音频信号分离中的实际效果,显示其在语音分离与降噪等的应用。 \np
    \noindent{\textbf{关键词:}独立成分分析;FastICA算法;音频处理;噪音消除}
\end{abstract}

\section{前置知识}

\subsection*{高斯分布与非高斯分布}

ICA依赖源信号的非高斯性(Non-Gaussianity)作为分离的依据 \np
\textbf{高斯信号}指信号的幅值分布服从高斯分布(正态分布),\textbf{非高斯信号}指幅值分布不服从高斯分布,可能呈现双峰、尖峰、重尾或不对称性(如语音、图像边缘、脑电信号). 而我们关注的恰恰是这些具有不对称性的非高斯信号. \np
由中心极限定理,多个独立随机变量的混合信号趋于服从高斯分布,因此通过最大化分离信号的非高斯性,能分离处出原始的信号.

\subsection*{信号的统计独立性}
把信号 $s_1(t), s_2(t)$ 视为随机变量,则其相互独立为 $p(s_1, s_2) = p(s_1)p(s_2)$. \np
同理 $n$ 个信号相互独立即为 $p(s_1, s_2, \cdots, s_n) = \prod_{i = 1}^{n}p(s_i)$. \np

\section{独立成分分析(ICA)的基本原理}

在现实生活中,观测信号(例如麦克风接收的声音信号)$\bs{X}$ 为源信号(例如各个环境中的声源) $\bs{S}$ 的瞬时线性组合,且源信号可以认为相互统计独立。我们试图从观测信号中提取出各个相互独立的源信号,并求出其混合矩阵 $\bs{A}$. 

\subsection{数学模型}

设观测信号由 $n$ 个相互统计独立的源信号线性组合而成,数学形式可表述为:
\begin{align*}
    \bs{x} &= \bs{A}\bs{s} \\
    \bs{s} &= \left(s_1(t), s_2(t), \cdots, s_n(t)\right)^T
\end{align*}

其中:
\begin{itemize}
    \item 信号源 $s_i(t)$ 表示第 $i$ 个信号在时间 $t$ 的实际值。其中 $p(s_1, s_2, \cdots, s_n) = \prod_{i = 1}^{n}p(s_i)$, 即 $s_i$ 相互独立.
    \item $\bs{x}$ 为观测信号向量,包含 $m$ 个通道在时间 $t$ 的采样值.
    \item $\bs{s}$ 为独立源信号向量,即 $n$ 个信号源在 $T$ 个时间点的实际值. 
    \item $\bs{A} \in \RR^{m \times n}$ 为未知的混合矩阵,表征原信号到观测信号的线性映射关系。一般假设 $n \leq m$, 以保证模型可辨识性.
\end{itemize}

对于 $T$ 个时间点的采样值,观测信号作为列向量构成观测信号矩阵 $\bs{X} = (\bs{x}_1, \cdots, \bs{x}_T)$, 同理源信号构成源信号矩阵 $\bs{S} = (\bs{s}_1, \cdots, \bs{s}_T)$, 同样有
\begin{align*}
    \bs{X} = \bs{AS}
\end{align*}

\subsection{ICA 的目标}

这里我们不加证明的断言,在 ICA 的假设中,信号的\textbf{非高斯性}等价于信号之间的\textbf{独立性}. \np
为尽可能还原源信号,我们的目标是找到解混矩阵 $\bs{W}$, 使分离出的信号 $\bs{Y} = \bs{W}^T\bs{X}$ 尽可能的接近源信号 $\bs{S}$. 也即对 $\bs{W}$ 的每个列向量 $\bs{w}_i$,投影结果 $\bs{y}_i = \bs{w}_i^T\bs{X}$ 尽量满足非高斯性.\np
对于非线性函数 $G(x)$ (例如 $\log_2\cosh x$,用于放大非高斯特征), 最大化负熵近似 $J(y) = \left(E(G(y)) - E(G(v))\right)^2$ 能够使 $y$ 的非高斯性最大. 其中 $v$ 为满足标准正态分布的随机变量.

\section{FastICA 算法}

\textbf{FastICA} 是一种基于固定点迭代的独立成分分析(ICA)算法. 其核心思想是通过最大化非高斯性来估计独立成分. 

\subsection{预处理}

预处理的目的是将观测信号各个行(每个观测点不同时间采集的样本)的均值变为 $0$, 方差变为 $1$.

\subsubsection*{中心化}

将观测信号 $\bs{X} \in \RR^{n \times T}$ 零均值化,即
\begin{align*}
    \bs{X}_\mathrm{c} = \bs{X} - \EE (\bs{X})
\end{align*} 
\par 其中 $\EE (\bs{X})$ 为 $\bs{X}$ 每个行的均值组成的列向量.

\subsubsection*{白化}

对中心化后的数据进行白化,使数据矩阵正交. \np
\begin{enumerate}
    \item 计算协方差矩阵
    \begin{align*}
        \bs{C} = T\cdot\mathrm{Cov}(\XC) = \XC\XC^T
    \end{align*}
    如果 $\bs{C}$ 半正定但不正定,则表明 $\XC$ 中有线性相关的维度,除去冗余维度即可. \np
    若 $\bs{C}$ 正定,则其特征值均大于零.
    \item 对 $\bs{C}$ 相似对角化,即
    \begin{align*}
        \bs{C} = \bs{E}\bs{\Lambda}\bs{E}^T
    \end{align*}
    考虑 $\bs{C}$ 正定的情况,$\bs{E}$ 为单位正交向量矩阵, $\bs{\Lambda}$ 为特征值对角阵.
    \item 构造白化矩阵,即
    \begin{align*}
        \bs{W}_\mathrm{whiten} = \bs{\Lambda}^{-1/2}\bs{E}^T
    \end{align*}
    \item 得到白化后的数据
    \begin{align*}
        \tilde{\bs{X}} = \bs{W}_\mathrm{whiten}\XC
    \end{align*}
\end{enumerate}
\par 此时 $\tilde{\bs{X}}\tilde{\bs{X}}^T = \XC^T\bs{E}\bs{\Lambda}^{-1/2}\bs{\Lambda}^{-1/2}\bs{E}^T\XC = \bs{I}$, 即 $\tilde{\bs{X}}$ 是正交阵.

\subsection{固定点迭代提取独立成分}

我们要提取 $n$ 个独立成分,即分离矩阵 $\bs{W} = \begin{pmatrix}
    \bs{\omega}_1^T \\ \vdots \\ \bs{\omega}_n^n
\end{pmatrix}$
使得 $\bs{Y} = \bs{W}\tilde{\bs{X}}$ 为目标原信号矩阵的近似. \np
对 $\tilde{\bs{X}}$ 的某个列向量 $\bs{x}$,我们要最大化 $J_G(\bs{\omega}^T\bs{x}) = \left(E(G(\bs{\omega}^T\bs{x})) - E(G(v))\right)^2$. 限定 $\left\lVert \bs{\omega}\right\rVert = 1$,为使用 Lagrange 乘数法,对 $\bs{\omega}$ 求偏导得
\begin{align*}
    \frac{\partial \left(J_G(\bs{\omega}^T\bs{x}) - \lambda\left(\left\lVert \bs{\omega}\right\rVert^2 - 1 \right)\right)}{\partial \bs{\omega}} = 2 \EE \left(\bs{x}G'(\bs{\omega^T \bs{x}})\right) - 2 \lambda \bs{\omega}
\end{align*}
使上式为 $0$ 的 $\bs{\omega}$ 即为最优的 $\bs{\omega}$. 使用牛顿法 $x_{n + 1} = x_n - \frac{f(x_n)}{f'(x_n)}$, 能得到上述方程的近似解. 计算偏导
\begin{align*}
    \frac{\partial \left(\EE \left(\bs{x}G'(\bs{\omega^T \bs{x}})\right) - \lambda \bs{\omega}\right)}{\partial \bs{\omega}} = \EE \left(\bs{x}\bs{x}^TG''\left(\bs{\omega^T}\bs{x}\right)\right) - \lambda \bs{I} \approx \EE \left(G''\left(\bs{\omega}^T\bs{x}\right)\right)\bs{I} - \lambda \bs{I}
\end{align*}
代入牛顿法公式,用 $\bs{\omega}_n$ 近似 $\lambda$,并进一步化简最后得
\begin{align*}
    \bs{\omega}_{n + 1} = \EE \left(\bs{x}G'\left(\bs{\omega}_n^T\bs{x}\right)\right) - \EE \left(G''(\bs{\omega}_n^T\bs{x})\right)\bs{\omega}_n
\end{align*}
对多个 $\bs{x}$ 的计算方法类似. 算法的具体过程如下:

\begin{enumerate}
    \item 随机生成正交阵 $\bs{W}^{(0)} \in \RR^{n \times n}$. 一般采用随机生成矩阵结合 QR 分解的方法实现.
    \item 固定点迭代更新. 
    \begin{align*}
        \bs{W}^{(n + 1)'} &= \EE \left[\tilde{\bs{X}}G'\left(\bs{W}^{(n)T}\tilde{\bs{X}}\right)\right] - \EE\left[G''\left(\bs{W}^{(n)T}\tilde{\bs{X}}\right)\right]\bs{W}^{(n)} \\
        \bs{W}^{(n + 1)} &= \left(\bs{W}^{(n + 1)'} \bs{W}^{(n + 1)'T}\right)^{-1/2}\bs{W}^{(n + 1)'}
    \end{align*}
    第二个式子的归一化与白化过程基本相同,保持了 $\bs{W}$ 的正交性.
    \item 不断迭代直至收敛. 例如 $\Delta = \left\lVert \bs{W}^{(n + 1)} - \bs{W}^{(n)}\right\rVert_F$ 小于特定值.
\end{enumerate}

此时的 $\bs{W}$ 即为所求解混矩阵.

\section{FastICA 算法对模拟音频信号的分离}

首先使用代码进行正弦波、方波与锯齿波的生成,并随机生成一些噪声模拟声音信号叠加到原波形上. 
FastICA 处理的 Python 代码如下(部分)

\begin{minted}[frame=single, fontsize=\normalsize, bgcolor=mybgcolor]{python}
def g(x): return np.tanh(x)
def g_prime(x): return 1 - np.tanh(x) ** 2

def whiten(X):
    X -= X.mean(axis=0)
    cov = np.cov(X, rowvar=False)
    d, E = np.linalg.eigh(cov)
    D_inv = np.diag(1. / np.sqrt(d))
    return X @ E @ D_inv @ E.T

def fastica(X, n_components, max_iter=200, tol=1e-4):
    X = whiten(X)
    n_samples, n_features = X.shape
    W = np.zeros((n_components, n_features))

    for i in range(n_components):
        w = np.random.rand(n_features)
        w /= np.linalg.norm(w)
        for _ in range(max_iter):
            wx = X @ w
            gwx = g(wx)
            g_wx = g_prime(wx)
            w_new = (X.T @ gwx - g_wx.mean() * w) / n_samples
            for j in range(i):
                w_new -= np.dot(w_new, W[j]) * W[j]
            w_new /= np.linalg.norm(w_new)
            if np.abs(np.abs(np.dot(w_new, w)) - 1) < tol:
                break
            w = w_new
        W[i, :] = w
    return X @ W.T

S_est = fastica(X, n_components=3)
\end{minted}

\begin{figure}[ht]
    \centering
    \raisebox{-1\height}{\resizebox{0.7\textwidth}{!}{%% Creator: Matplotlib, PGF backend
%%
%% To include the figure in your LaTeX document, write
%%   \input{<filename>.pgf}
%%
%% Make sure the required packages are loaded in your preamble
%%   \usepackage{pgf}
%%
%% Also ensure that all the required font packages are loaded; for instance,
%% the lmodern package is sometimes necessary when using math font.
%%   \usepackage{lmodern}
%%
%% Figures using additional raster images can only be included by \input if
%% they are in the same directory as the main LaTeX file. For loading figures
%% from other directories you can use the `import` package
%%   \usepackage{import}
%%
%% and then include the figures with
%%   \import{<path to file>}{<filename>.pgf}
%%
%% Matplotlib used the following preamble
%%   \def\mathdefault#1{#1}
%%   \everymath=\expandafter{\the\everymath\displaystyle}
%%   
%%   \ifdefined\pdftexversion\else  % non-pdftex case.
%%     \usepackage{fontspec}
%%   \fi
%%   \makeatletter\@ifpackageloaded{underscore}{}{\usepackage[strings]{underscore}}\makeatother
%%
\begingroup%
\makeatletter%
\begin{pgfpicture}%
\pgfpathrectangle{\pgfpointorigin}{\pgfqpoint{6.000000in}{5.000000in}}%
\pgfusepath{use as bounding box, clip}%
\begin{pgfscope}%
\pgfsetbuttcap%
\pgfsetmiterjoin%
\definecolor{currentfill}{rgb}{1.000000,1.000000,1.000000}%
\pgfsetfillcolor{currentfill}%
\pgfsetlinewidth{0.000000pt}%
\definecolor{currentstroke}{rgb}{1.000000,1.000000,1.000000}%
\pgfsetstrokecolor{currentstroke}%
\pgfsetdash{}{0pt}%
\pgfpathmoveto{\pgfqpoint{0.000000in}{0.000000in}}%
\pgfpathlineto{\pgfqpoint{6.000000in}{0.000000in}}%
\pgfpathlineto{\pgfqpoint{6.000000in}{5.000000in}}%
\pgfpathlineto{\pgfqpoint{0.000000in}{5.000000in}}%
\pgfpathlineto{\pgfqpoint{0.000000in}{0.000000in}}%
\pgfpathclose%
\pgfusepath{fill}%
\end{pgfscope}%
\begin{pgfscope}%
\pgfsetbuttcap%
\pgfsetmiterjoin%
\definecolor{currentfill}{rgb}{1.000000,1.000000,1.000000}%
\pgfsetfillcolor{currentfill}%
\pgfsetlinewidth{0.000000pt}%
\definecolor{currentstroke}{rgb}{0.000000,0.000000,0.000000}%
\pgfsetstrokecolor{currentstroke}%
\pgfsetstrokeopacity{0.000000}%
\pgfsetdash{}{0pt}%
\pgfpathmoveto{\pgfqpoint{0.709829in}{3.729963in}}%
\pgfpathlineto{\pgfqpoint{5.820000in}{3.729963in}}%
\pgfpathlineto{\pgfqpoint{5.820000in}{4.617500in}}%
\pgfpathlineto{\pgfqpoint{0.709829in}{4.617500in}}%
\pgfpathlineto{\pgfqpoint{0.709829in}{3.729963in}}%
\pgfpathclose%
\pgfusepath{fill}%
\end{pgfscope}%
\begin{pgfscope}%
\pgfpathrectangle{\pgfqpoint{0.709829in}{3.729963in}}{\pgfqpoint{5.110171in}{0.887537in}}%
\pgfusepath{clip}%
\pgfsetroundcap%
\pgfsetroundjoin%
\pgfsetlinewidth{1.003750pt}%
\definecolor{currentstroke}{rgb}{0.800000,0.800000,0.800000}%
\pgfsetstrokecolor{currentstroke}%
\pgfsetdash{}{0pt}%
\pgfpathmoveto{\pgfqpoint{0.942110in}{3.729963in}}%
\pgfpathlineto{\pgfqpoint{0.942110in}{4.617500in}}%
\pgfusepath{stroke}%
\end{pgfscope}%
\begin{pgfscope}%
\definecolor{textcolor}{rgb}{0.150000,0.150000,0.150000}%
\pgfsetstrokecolor{textcolor}%
\pgfsetfillcolor{textcolor}%
\pgftext[x=0.942110in,y=3.598018in,,top]{\color{textcolor}{\sffamily\fontsize{11.000000}{13.200000}\selectfont\catcode`\^=\active\def^{\ifmmode\sp\else\^{}\fi}\catcode`\%=\active\def%{\%}$\mathdefault{0}$}}%
\end{pgfscope}%
\begin{pgfscope}%
\pgfpathrectangle{\pgfqpoint{0.709829in}{3.729963in}}{\pgfqpoint{5.110171in}{0.887537in}}%
\pgfusepath{clip}%
\pgfsetroundcap%
\pgfsetroundjoin%
\pgfsetlinewidth{1.003750pt}%
\definecolor{currentstroke}{rgb}{0.800000,0.800000,0.800000}%
\pgfsetstrokecolor{currentstroke}%
\pgfsetdash{}{0pt}%
\pgfpathmoveto{\pgfqpoint{1.523101in}{3.729963in}}%
\pgfpathlineto{\pgfqpoint{1.523101in}{4.617500in}}%
\pgfusepath{stroke}%
\end{pgfscope}%
\begin{pgfscope}%
\definecolor{textcolor}{rgb}{0.150000,0.150000,0.150000}%
\pgfsetstrokecolor{textcolor}%
\pgfsetfillcolor{textcolor}%
\pgftext[x=1.523101in,y=3.598018in,,top]{\color{textcolor}{\sffamily\fontsize{11.000000}{13.200000}\selectfont\catcode`\^=\active\def^{\ifmmode\sp\else\^{}\fi}\catcode`\%=\active\def%{\%}$\mathdefault{250}$}}%
\end{pgfscope}%
\begin{pgfscope}%
\pgfpathrectangle{\pgfqpoint{0.709829in}{3.729963in}}{\pgfqpoint{5.110171in}{0.887537in}}%
\pgfusepath{clip}%
\pgfsetroundcap%
\pgfsetroundjoin%
\pgfsetlinewidth{1.003750pt}%
\definecolor{currentstroke}{rgb}{0.800000,0.800000,0.800000}%
\pgfsetstrokecolor{currentstroke}%
\pgfsetdash{}{0pt}%
\pgfpathmoveto{\pgfqpoint{2.104093in}{3.729963in}}%
\pgfpathlineto{\pgfqpoint{2.104093in}{4.617500in}}%
\pgfusepath{stroke}%
\end{pgfscope}%
\begin{pgfscope}%
\definecolor{textcolor}{rgb}{0.150000,0.150000,0.150000}%
\pgfsetstrokecolor{textcolor}%
\pgfsetfillcolor{textcolor}%
\pgftext[x=2.104093in,y=3.598018in,,top]{\color{textcolor}{\sffamily\fontsize{11.000000}{13.200000}\selectfont\catcode`\^=\active\def^{\ifmmode\sp\else\^{}\fi}\catcode`\%=\active\def%{\%}$\mathdefault{500}$}}%
\end{pgfscope}%
\begin{pgfscope}%
\pgfpathrectangle{\pgfqpoint{0.709829in}{3.729963in}}{\pgfqpoint{5.110171in}{0.887537in}}%
\pgfusepath{clip}%
\pgfsetroundcap%
\pgfsetroundjoin%
\pgfsetlinewidth{1.003750pt}%
\definecolor{currentstroke}{rgb}{0.800000,0.800000,0.800000}%
\pgfsetstrokecolor{currentstroke}%
\pgfsetdash{}{0pt}%
\pgfpathmoveto{\pgfqpoint{2.685085in}{3.729963in}}%
\pgfpathlineto{\pgfqpoint{2.685085in}{4.617500in}}%
\pgfusepath{stroke}%
\end{pgfscope}%
\begin{pgfscope}%
\definecolor{textcolor}{rgb}{0.150000,0.150000,0.150000}%
\pgfsetstrokecolor{textcolor}%
\pgfsetfillcolor{textcolor}%
\pgftext[x=2.685085in,y=3.598018in,,top]{\color{textcolor}{\sffamily\fontsize{11.000000}{13.200000}\selectfont\catcode`\^=\active\def^{\ifmmode\sp\else\^{}\fi}\catcode`\%=\active\def%{\%}$\mathdefault{750}$}}%
\end{pgfscope}%
\begin{pgfscope}%
\pgfpathrectangle{\pgfqpoint{0.709829in}{3.729963in}}{\pgfqpoint{5.110171in}{0.887537in}}%
\pgfusepath{clip}%
\pgfsetroundcap%
\pgfsetroundjoin%
\pgfsetlinewidth{1.003750pt}%
\definecolor{currentstroke}{rgb}{0.800000,0.800000,0.800000}%
\pgfsetstrokecolor{currentstroke}%
\pgfsetdash{}{0pt}%
\pgfpathmoveto{\pgfqpoint{3.266076in}{3.729963in}}%
\pgfpathlineto{\pgfqpoint{3.266076in}{4.617500in}}%
\pgfusepath{stroke}%
\end{pgfscope}%
\begin{pgfscope}%
\definecolor{textcolor}{rgb}{0.150000,0.150000,0.150000}%
\pgfsetstrokecolor{textcolor}%
\pgfsetfillcolor{textcolor}%
\pgftext[x=3.266076in,y=3.598018in,,top]{\color{textcolor}{\sffamily\fontsize{11.000000}{13.200000}\selectfont\catcode`\^=\active\def^{\ifmmode\sp\else\^{}\fi}\catcode`\%=\active\def%{\%}$\mathdefault{1000}$}}%
\end{pgfscope}%
\begin{pgfscope}%
\pgfpathrectangle{\pgfqpoint{0.709829in}{3.729963in}}{\pgfqpoint{5.110171in}{0.887537in}}%
\pgfusepath{clip}%
\pgfsetroundcap%
\pgfsetroundjoin%
\pgfsetlinewidth{1.003750pt}%
\definecolor{currentstroke}{rgb}{0.800000,0.800000,0.800000}%
\pgfsetstrokecolor{currentstroke}%
\pgfsetdash{}{0pt}%
\pgfpathmoveto{\pgfqpoint{3.847068in}{3.729963in}}%
\pgfpathlineto{\pgfqpoint{3.847068in}{4.617500in}}%
\pgfusepath{stroke}%
\end{pgfscope}%
\begin{pgfscope}%
\definecolor{textcolor}{rgb}{0.150000,0.150000,0.150000}%
\pgfsetstrokecolor{textcolor}%
\pgfsetfillcolor{textcolor}%
\pgftext[x=3.847068in,y=3.598018in,,top]{\color{textcolor}{\sffamily\fontsize{11.000000}{13.200000}\selectfont\catcode`\^=\active\def^{\ifmmode\sp\else\^{}\fi}\catcode`\%=\active\def%{\%}$\mathdefault{1250}$}}%
\end{pgfscope}%
\begin{pgfscope}%
\pgfpathrectangle{\pgfqpoint{0.709829in}{3.729963in}}{\pgfqpoint{5.110171in}{0.887537in}}%
\pgfusepath{clip}%
\pgfsetroundcap%
\pgfsetroundjoin%
\pgfsetlinewidth{1.003750pt}%
\definecolor{currentstroke}{rgb}{0.800000,0.800000,0.800000}%
\pgfsetstrokecolor{currentstroke}%
\pgfsetdash{}{0pt}%
\pgfpathmoveto{\pgfqpoint{4.428060in}{3.729963in}}%
\pgfpathlineto{\pgfqpoint{4.428060in}{4.617500in}}%
\pgfusepath{stroke}%
\end{pgfscope}%
\begin{pgfscope}%
\definecolor{textcolor}{rgb}{0.150000,0.150000,0.150000}%
\pgfsetstrokecolor{textcolor}%
\pgfsetfillcolor{textcolor}%
\pgftext[x=4.428060in,y=3.598018in,,top]{\color{textcolor}{\sffamily\fontsize{11.000000}{13.200000}\selectfont\catcode`\^=\active\def^{\ifmmode\sp\else\^{}\fi}\catcode`\%=\active\def%{\%}$\mathdefault{1500}$}}%
\end{pgfscope}%
\begin{pgfscope}%
\pgfpathrectangle{\pgfqpoint{0.709829in}{3.729963in}}{\pgfqpoint{5.110171in}{0.887537in}}%
\pgfusepath{clip}%
\pgfsetroundcap%
\pgfsetroundjoin%
\pgfsetlinewidth{1.003750pt}%
\definecolor{currentstroke}{rgb}{0.800000,0.800000,0.800000}%
\pgfsetstrokecolor{currentstroke}%
\pgfsetdash{}{0pt}%
\pgfpathmoveto{\pgfqpoint{5.009052in}{3.729963in}}%
\pgfpathlineto{\pgfqpoint{5.009052in}{4.617500in}}%
\pgfusepath{stroke}%
\end{pgfscope}%
\begin{pgfscope}%
\definecolor{textcolor}{rgb}{0.150000,0.150000,0.150000}%
\pgfsetstrokecolor{textcolor}%
\pgfsetfillcolor{textcolor}%
\pgftext[x=5.009052in,y=3.598018in,,top]{\color{textcolor}{\sffamily\fontsize{11.000000}{13.200000}\selectfont\catcode`\^=\active\def^{\ifmmode\sp\else\^{}\fi}\catcode`\%=\active\def%{\%}$\mathdefault{1750}$}}%
\end{pgfscope}%
\begin{pgfscope}%
\pgfpathrectangle{\pgfqpoint{0.709829in}{3.729963in}}{\pgfqpoint{5.110171in}{0.887537in}}%
\pgfusepath{clip}%
\pgfsetroundcap%
\pgfsetroundjoin%
\pgfsetlinewidth{1.003750pt}%
\definecolor{currentstroke}{rgb}{0.800000,0.800000,0.800000}%
\pgfsetstrokecolor{currentstroke}%
\pgfsetdash{}{0pt}%
\pgfpathmoveto{\pgfqpoint{5.590043in}{3.729963in}}%
\pgfpathlineto{\pgfqpoint{5.590043in}{4.617500in}}%
\pgfusepath{stroke}%
\end{pgfscope}%
\begin{pgfscope}%
\definecolor{textcolor}{rgb}{0.150000,0.150000,0.150000}%
\pgfsetstrokecolor{textcolor}%
\pgfsetfillcolor{textcolor}%
\pgftext[x=5.590043in,y=3.598018in,,top]{\color{textcolor}{\sffamily\fontsize{11.000000}{13.200000}\selectfont\catcode`\^=\active\def^{\ifmmode\sp\else\^{}\fi}\catcode`\%=\active\def%{\%}$\mathdefault{2000}$}}%
\end{pgfscope}%
\begin{pgfscope}%
\pgfpathrectangle{\pgfqpoint{0.709829in}{3.729963in}}{\pgfqpoint{5.110171in}{0.887537in}}%
\pgfusepath{clip}%
\pgfsetroundcap%
\pgfsetroundjoin%
\pgfsetlinewidth{1.003750pt}%
\definecolor{currentstroke}{rgb}{0.800000,0.800000,0.800000}%
\pgfsetstrokecolor{currentstroke}%
\pgfsetdash{}{0pt}%
\pgfpathmoveto{\pgfqpoint{0.709829in}{3.797783in}}%
\pgfpathlineto{\pgfqpoint{5.820000in}{3.797783in}}%
\pgfusepath{stroke}%
\end{pgfscope}%
\begin{pgfscope}%
\definecolor{textcolor}{rgb}{0.150000,0.150000,0.150000}%
\pgfsetstrokecolor{textcolor}%
\pgfsetfillcolor{textcolor}%
\pgftext[x=0.383555in, y=3.744769in, left, base]{\color{textcolor}{\sffamily\fontsize{11.000000}{13.200000}\selectfont\catcode`\^=\active\def^{\ifmmode\sp\else\^{}\fi}\catcode`\%=\active\def%{\%}$\mathdefault{-2}$}}%
\end{pgfscope}%
\begin{pgfscope}%
\pgfpathrectangle{\pgfqpoint{0.709829in}{3.729963in}}{\pgfqpoint{5.110171in}{0.887537in}}%
\pgfusepath{clip}%
\pgfsetroundcap%
\pgfsetroundjoin%
\pgfsetlinewidth{1.003750pt}%
\definecolor{currentstroke}{rgb}{0.800000,0.800000,0.800000}%
\pgfsetstrokecolor{currentstroke}%
\pgfsetdash{}{0pt}%
\pgfpathmoveto{\pgfqpoint{0.709829in}{4.182152in}}%
\pgfpathlineto{\pgfqpoint{5.820000in}{4.182152in}}%
\pgfusepath{stroke}%
\end{pgfscope}%
\begin{pgfscope}%
\definecolor{textcolor}{rgb}{0.150000,0.150000,0.150000}%
\pgfsetstrokecolor{textcolor}%
\pgfsetfillcolor{textcolor}%
\pgftext[x=0.501843in, y=4.129138in, left, base]{\color{textcolor}{\sffamily\fontsize{11.000000}{13.200000}\selectfont\catcode`\^=\active\def^{\ifmmode\sp\else\^{}\fi}\catcode`\%=\active\def%{\%}$\mathdefault{0}$}}%
\end{pgfscope}%
\begin{pgfscope}%
\pgfpathrectangle{\pgfqpoint{0.709829in}{3.729963in}}{\pgfqpoint{5.110171in}{0.887537in}}%
\pgfusepath{clip}%
\pgfsetroundcap%
\pgfsetroundjoin%
\pgfsetlinewidth{1.003750pt}%
\definecolor{currentstroke}{rgb}{0.800000,0.800000,0.800000}%
\pgfsetstrokecolor{currentstroke}%
\pgfsetdash{}{0pt}%
\pgfpathmoveto{\pgfqpoint{0.709829in}{4.566521in}}%
\pgfpathlineto{\pgfqpoint{5.820000in}{4.566521in}}%
\pgfusepath{stroke}%
\end{pgfscope}%
\begin{pgfscope}%
\definecolor{textcolor}{rgb}{0.150000,0.150000,0.150000}%
\pgfsetstrokecolor{textcolor}%
\pgfsetfillcolor{textcolor}%
\pgftext[x=0.501843in, y=4.513507in, left, base]{\color{textcolor}{\sffamily\fontsize{11.000000}{13.200000}\selectfont\catcode`\^=\active\def^{\ifmmode\sp\else\^{}\fi}\catcode`\%=\active\def%{\%}$\mathdefault{2}$}}%
\end{pgfscope}%
\begin{pgfscope}%
\definecolor{textcolor}{rgb}{0.150000,0.150000,0.150000}%
\pgfsetstrokecolor{textcolor}%
\pgfsetfillcolor{textcolor}%
\pgftext[x=0.328000in,y=4.173731in,,bottom,rotate=90.000000]{\color{textcolor}{\sffamily\fontsize{12.000000}{14.400000}\selectfont\catcode`\^=\active\def^{\ifmmode\sp\else\^{}\fi}\catcode`\%=\active\def%{\%}Amplitude}}%
\end{pgfscope}%
\begin{pgfscope}%
\pgfpathrectangle{\pgfqpoint{0.709829in}{3.729963in}}{\pgfqpoint{5.110171in}{0.887537in}}%
\pgfusepath{clip}%
\pgfsetroundcap%
\pgfsetroundjoin%
\pgfsetlinewidth{1.003750pt}%
\definecolor{currentstroke}{rgb}{0.298039,0.447059,0.690196}%
\pgfsetstrokecolor{currentstroke}%
\pgfsetdash{}{0pt}%
\pgfpathmoveto{\pgfqpoint{0.942110in}{4.230957in}}%
\pgfpathlineto{\pgfqpoint{0.944433in}{4.246364in}}%
\pgfpathlineto{\pgfqpoint{0.946757in}{4.212866in}}%
\pgfpathlineto{\pgfqpoint{0.949081in}{4.200154in}}%
\pgfpathlineto{\pgfqpoint{0.951405in}{4.212064in}}%
\pgfpathlineto{\pgfqpoint{0.953729in}{4.202453in}}%
\pgfpathlineto{\pgfqpoint{0.956053in}{4.204095in}}%
\pgfpathlineto{\pgfqpoint{0.958377in}{4.215728in}}%
\pgfpathlineto{\pgfqpoint{0.960701in}{4.262652in}}%
\pgfpathlineto{\pgfqpoint{0.963025in}{4.196886in}}%
\pgfpathlineto{\pgfqpoint{0.965349in}{4.208560in}}%
\pgfpathlineto{\pgfqpoint{0.967673in}{4.151677in}}%
\pgfpathlineto{\pgfqpoint{0.969997in}{4.242722in}}%
\pgfpathlineto{\pgfqpoint{0.974645in}{4.165883in}}%
\pgfpathlineto{\pgfqpoint{0.976969in}{4.203169in}}%
\pgfpathlineto{\pgfqpoint{0.979293in}{4.172835in}}%
\pgfpathlineto{\pgfqpoint{0.981617in}{4.230386in}}%
\pgfpathlineto{\pgfqpoint{0.983941in}{4.221094in}}%
\pgfpathlineto{\pgfqpoint{0.986265in}{4.232433in}}%
\pgfpathlineto{\pgfqpoint{0.990913in}{4.180676in}}%
\pgfpathlineto{\pgfqpoint{0.993237in}{4.185516in}}%
\pgfpathlineto{\pgfqpoint{0.997885in}{4.266496in}}%
\pgfpathlineto{\pgfqpoint{1.000209in}{4.218198in}}%
\pgfpathlineto{\pgfqpoint{1.002533in}{4.230693in}}%
\pgfpathlineto{\pgfqpoint{1.004857in}{4.266400in}}%
\pgfpathlineto{\pgfqpoint{1.007181in}{4.284813in}}%
\pgfpathlineto{\pgfqpoint{1.009505in}{4.240818in}}%
\pgfpathlineto{\pgfqpoint{1.011829in}{4.236794in}}%
\pgfpathlineto{\pgfqpoint{1.014153in}{4.277118in}}%
\pgfpathlineto{\pgfqpoint{1.016476in}{4.252532in}}%
\pgfpathlineto{\pgfqpoint{1.018800in}{4.265503in}}%
\pgfpathlineto{\pgfqpoint{1.021124in}{4.221367in}}%
\pgfpathlineto{\pgfqpoint{1.023448in}{4.312421in}}%
\pgfpathlineto{\pgfqpoint{1.025772in}{4.313974in}}%
\pgfpathlineto{\pgfqpoint{1.030420in}{4.257594in}}%
\pgfpathlineto{\pgfqpoint{1.032744in}{4.262831in}}%
\pgfpathlineto{\pgfqpoint{1.035068in}{4.279638in}}%
\pgfpathlineto{\pgfqpoint{1.037392in}{4.308020in}}%
\pgfpathlineto{\pgfqpoint{1.039716in}{4.261377in}}%
\pgfpathlineto{\pgfqpoint{1.042040in}{4.286777in}}%
\pgfpathlineto{\pgfqpoint{1.044364in}{4.258929in}}%
\pgfpathlineto{\pgfqpoint{1.046688in}{4.298376in}}%
\pgfpathlineto{\pgfqpoint{1.049012in}{4.292686in}}%
\pgfpathlineto{\pgfqpoint{1.051336in}{4.295949in}}%
\pgfpathlineto{\pgfqpoint{1.053660in}{4.351783in}}%
\pgfpathlineto{\pgfqpoint{1.058308in}{4.288054in}}%
\pgfpathlineto{\pgfqpoint{1.060632in}{4.269113in}}%
\pgfpathlineto{\pgfqpoint{1.062956in}{4.325177in}}%
\pgfpathlineto{\pgfqpoint{1.065280in}{4.283916in}}%
\pgfpathlineto{\pgfqpoint{1.067604in}{4.324310in}}%
\pgfpathlineto{\pgfqpoint{1.069928in}{4.323411in}}%
\pgfpathlineto{\pgfqpoint{1.072252in}{4.334918in}}%
\pgfpathlineto{\pgfqpoint{1.074576in}{4.312860in}}%
\pgfpathlineto{\pgfqpoint{1.076900in}{4.277411in}}%
\pgfpathlineto{\pgfqpoint{1.081548in}{4.300176in}}%
\pgfpathlineto{\pgfqpoint{1.083872in}{4.250415in}}%
\pgfpathlineto{\pgfqpoint{1.086195in}{4.283326in}}%
\pgfpathlineto{\pgfqpoint{1.088519in}{4.358514in}}%
\pgfpathlineto{\pgfqpoint{1.090843in}{4.316672in}}%
\pgfpathlineto{\pgfqpoint{1.093167in}{4.314938in}}%
\pgfpathlineto{\pgfqpoint{1.095491in}{4.381451in}}%
\pgfpathlineto{\pgfqpoint{1.097815in}{4.316887in}}%
\pgfpathlineto{\pgfqpoint{1.100139in}{4.343119in}}%
\pgfpathlineto{\pgfqpoint{1.102463in}{4.306880in}}%
\pgfpathlineto{\pgfqpoint{1.104787in}{4.354360in}}%
\pgfpathlineto{\pgfqpoint{1.107111in}{4.318146in}}%
\pgfpathlineto{\pgfqpoint{1.109435in}{4.332441in}}%
\pgfpathlineto{\pgfqpoint{1.111759in}{4.333594in}}%
\pgfpathlineto{\pgfqpoint{1.114083in}{4.323780in}}%
\pgfpathlineto{\pgfqpoint{1.116407in}{4.340188in}}%
\pgfpathlineto{\pgfqpoint{1.118731in}{4.323735in}}%
\pgfpathlineto{\pgfqpoint{1.121055in}{4.328424in}}%
\pgfpathlineto{\pgfqpoint{1.125703in}{4.288396in}}%
\pgfpathlineto{\pgfqpoint{1.128027in}{4.329811in}}%
\pgfpathlineto{\pgfqpoint{1.130351in}{4.340978in}}%
\pgfpathlineto{\pgfqpoint{1.132675in}{4.382858in}}%
\pgfpathlineto{\pgfqpoint{1.134999in}{4.312155in}}%
\pgfpathlineto{\pgfqpoint{1.137323in}{4.358391in}}%
\pgfpathlineto{\pgfqpoint{1.139647in}{4.375412in}}%
\pgfpathlineto{\pgfqpoint{1.141971in}{4.314112in}}%
\pgfpathlineto{\pgfqpoint{1.144295in}{4.345589in}}%
\pgfpathlineto{\pgfqpoint{1.146619in}{4.307734in}}%
\pgfpathlineto{\pgfqpoint{1.148943in}{4.365433in}}%
\pgfpathlineto{\pgfqpoint{1.151267in}{4.375697in}}%
\pgfpathlineto{\pgfqpoint{1.153591in}{4.377107in}}%
\pgfpathlineto{\pgfqpoint{1.155914in}{4.381622in}}%
\pgfpathlineto{\pgfqpoint{1.158238in}{4.426714in}}%
\pgfpathlineto{\pgfqpoint{1.160562in}{4.352072in}}%
\pgfpathlineto{\pgfqpoint{1.162886in}{4.389674in}}%
\pgfpathlineto{\pgfqpoint{1.165210in}{4.393513in}}%
\pgfpathlineto{\pgfqpoint{1.167534in}{4.362706in}}%
\pgfpathlineto{\pgfqpoint{1.169858in}{4.373826in}}%
\pgfpathlineto{\pgfqpoint{1.172182in}{4.395277in}}%
\pgfpathlineto{\pgfqpoint{1.174506in}{4.344550in}}%
\pgfpathlineto{\pgfqpoint{1.176830in}{4.363415in}}%
\pgfpathlineto{\pgfqpoint{1.179154in}{4.346842in}}%
\pgfpathlineto{\pgfqpoint{1.181478in}{4.380851in}}%
\pgfpathlineto{\pgfqpoint{1.183802in}{4.355588in}}%
\pgfpathlineto{\pgfqpoint{1.186126in}{4.390857in}}%
\pgfpathlineto{\pgfqpoint{1.188450in}{4.388013in}}%
\pgfpathlineto{\pgfqpoint{1.190774in}{4.368683in}}%
\pgfpathlineto{\pgfqpoint{1.193098in}{4.360599in}}%
\pgfpathlineto{\pgfqpoint{1.195422in}{4.456486in}}%
\pgfpathlineto{\pgfqpoint{1.202394in}{4.349680in}}%
\pgfpathlineto{\pgfqpoint{1.204718in}{4.445550in}}%
\pgfpathlineto{\pgfqpoint{1.207042in}{4.381967in}}%
\pgfpathlineto{\pgfqpoint{1.209366in}{4.394589in}}%
\pgfpathlineto{\pgfqpoint{1.211690in}{4.419685in}}%
\pgfpathlineto{\pgfqpoint{1.214014in}{4.365219in}}%
\pgfpathlineto{\pgfqpoint{1.216338in}{4.388257in}}%
\pgfpathlineto{\pgfqpoint{1.218662in}{4.394370in}}%
\pgfpathlineto{\pgfqpoint{1.220986in}{4.428195in}}%
\pgfpathlineto{\pgfqpoint{1.223310in}{4.419538in}}%
\pgfpathlineto{\pgfqpoint{1.225633in}{4.406105in}}%
\pgfpathlineto{\pgfqpoint{1.227957in}{4.381404in}}%
\pgfpathlineto{\pgfqpoint{1.230281in}{4.437115in}}%
\pgfpathlineto{\pgfqpoint{1.232605in}{4.428696in}}%
\pgfpathlineto{\pgfqpoint{1.234929in}{4.410497in}}%
\pgfpathlineto{\pgfqpoint{1.237253in}{4.428109in}}%
\pgfpathlineto{\pgfqpoint{1.239577in}{4.419621in}}%
\pgfpathlineto{\pgfqpoint{1.241901in}{4.378975in}}%
\pgfpathlineto{\pgfqpoint{1.244225in}{4.436518in}}%
\pgfpathlineto{\pgfqpoint{1.246549in}{4.377213in}}%
\pgfpathlineto{\pgfqpoint{1.248873in}{4.446768in}}%
\pgfpathlineto{\pgfqpoint{1.251197in}{4.447701in}}%
\pgfpathlineto{\pgfqpoint{1.253521in}{4.446391in}}%
\pgfpathlineto{\pgfqpoint{1.258169in}{4.422145in}}%
\pgfpathlineto{\pgfqpoint{1.262817in}{4.477226in}}%
\pgfpathlineto{\pgfqpoint{1.265141in}{4.434206in}}%
\pgfpathlineto{\pgfqpoint{1.267465in}{4.456963in}}%
\pgfpathlineto{\pgfqpoint{1.269789in}{4.435625in}}%
\pgfpathlineto{\pgfqpoint{1.272113in}{4.459233in}}%
\pgfpathlineto{\pgfqpoint{1.274437in}{4.441533in}}%
\pgfpathlineto{\pgfqpoint{1.276761in}{4.459014in}}%
\pgfpathlineto{\pgfqpoint{1.279085in}{4.444574in}}%
\pgfpathlineto{\pgfqpoint{1.281409in}{4.458888in}}%
\pgfpathlineto{\pgfqpoint{1.283733in}{4.405521in}}%
\pgfpathlineto{\pgfqpoint{1.286057in}{4.462672in}}%
\pgfpathlineto{\pgfqpoint{1.288381in}{4.412579in}}%
\pgfpathlineto{\pgfqpoint{1.290705in}{4.410898in}}%
\pgfpathlineto{\pgfqpoint{1.293029in}{4.448590in}}%
\pgfpathlineto{\pgfqpoint{1.295353in}{4.428673in}}%
\pgfpathlineto{\pgfqpoint{1.297676in}{4.441005in}}%
\pgfpathlineto{\pgfqpoint{1.300000in}{4.430997in}}%
\pgfpathlineto{\pgfqpoint{1.302324in}{4.510619in}}%
\pgfpathlineto{\pgfqpoint{1.304648in}{4.381265in}}%
\pgfpathlineto{\pgfqpoint{1.309296in}{4.494061in}}%
\pgfpathlineto{\pgfqpoint{1.311620in}{4.422198in}}%
\pgfpathlineto{\pgfqpoint{1.313944in}{4.468595in}}%
\pgfpathlineto{\pgfqpoint{1.316268in}{4.436139in}}%
\pgfpathlineto{\pgfqpoint{1.318592in}{4.490355in}}%
\pgfpathlineto{\pgfqpoint{1.320916in}{4.455178in}}%
\pgfpathlineto{\pgfqpoint{1.323240in}{4.480092in}}%
\pgfpathlineto{\pgfqpoint{1.327888in}{4.422582in}}%
\pgfpathlineto{\pgfqpoint{1.330212in}{4.450322in}}%
\pgfpathlineto{\pgfqpoint{1.332536in}{4.442162in}}%
\pgfpathlineto{\pgfqpoint{1.334860in}{4.408888in}}%
\pgfpathlineto{\pgfqpoint{1.337184in}{4.477502in}}%
\pgfpathlineto{\pgfqpoint{1.339508in}{4.398721in}}%
\pgfpathlineto{\pgfqpoint{1.341832in}{4.467419in}}%
\pgfpathlineto{\pgfqpoint{1.344156in}{4.502255in}}%
\pgfpathlineto{\pgfqpoint{1.346480in}{4.407928in}}%
\pgfpathlineto{\pgfqpoint{1.348804in}{4.430238in}}%
\pgfpathlineto{\pgfqpoint{1.351128in}{4.389633in}}%
\pgfpathlineto{\pgfqpoint{1.353452in}{4.466648in}}%
\pgfpathlineto{\pgfqpoint{1.355776in}{4.420467in}}%
\pgfpathlineto{\pgfqpoint{1.358100in}{4.451143in}}%
\pgfpathlineto{\pgfqpoint{1.360424in}{4.462387in}}%
\pgfpathlineto{\pgfqpoint{1.362748in}{4.498764in}}%
\pgfpathlineto{\pgfqpoint{1.365072in}{4.486850in}}%
\pgfpathlineto{\pgfqpoint{1.367395in}{4.439781in}}%
\pgfpathlineto{\pgfqpoint{1.369719in}{4.483564in}}%
\pgfpathlineto{\pgfqpoint{1.372043in}{4.493604in}}%
\pgfpathlineto{\pgfqpoint{1.374367in}{4.420595in}}%
\pgfpathlineto{\pgfqpoint{1.376691in}{4.456826in}}%
\pgfpathlineto{\pgfqpoint{1.379015in}{4.438045in}}%
\pgfpathlineto{\pgfqpoint{1.381339in}{4.475419in}}%
\pgfpathlineto{\pgfqpoint{1.383663in}{4.471026in}}%
\pgfpathlineto{\pgfqpoint{1.385987in}{4.479819in}}%
\pgfpathlineto{\pgfqpoint{1.388311in}{4.436369in}}%
\pgfpathlineto{\pgfqpoint{1.390635in}{4.490073in}}%
\pgfpathlineto{\pgfqpoint{1.392959in}{4.427239in}}%
\pgfpathlineto{\pgfqpoint{1.395283in}{4.445783in}}%
\pgfpathlineto{\pgfqpoint{1.397607in}{4.447962in}}%
\pgfpathlineto{\pgfqpoint{1.399931in}{4.470960in}}%
\pgfpathlineto{\pgfqpoint{1.402255in}{4.468524in}}%
\pgfpathlineto{\pgfqpoint{1.404579in}{4.452755in}}%
\pgfpathlineto{\pgfqpoint{1.406903in}{4.415798in}}%
\pgfpathlineto{\pgfqpoint{1.409227in}{4.465206in}}%
\pgfpathlineto{\pgfqpoint{1.413875in}{4.447925in}}%
\pgfpathlineto{\pgfqpoint{1.416199in}{4.475028in}}%
\pgfpathlineto{\pgfqpoint{1.418523in}{4.469339in}}%
\pgfpathlineto{\pgfqpoint{1.420847in}{4.493555in}}%
\pgfpathlineto{\pgfqpoint{1.423171in}{4.447226in}}%
\pgfpathlineto{\pgfqpoint{1.425495in}{4.447217in}}%
\pgfpathlineto{\pgfqpoint{1.427819in}{4.503608in}}%
\pgfpathlineto{\pgfqpoint{1.430143in}{4.438129in}}%
\pgfpathlineto{\pgfqpoint{1.432467in}{4.446649in}}%
\pgfpathlineto{\pgfqpoint{1.434791in}{4.431225in}}%
\pgfpathlineto{\pgfqpoint{1.437114in}{4.493961in}}%
\pgfpathlineto{\pgfqpoint{1.439438in}{4.484650in}}%
\pgfpathlineto{\pgfqpoint{1.441762in}{4.447272in}}%
\pgfpathlineto{\pgfqpoint{1.444086in}{4.438751in}}%
\pgfpathlineto{\pgfqpoint{1.446410in}{4.390809in}}%
\pgfpathlineto{\pgfqpoint{1.448734in}{4.460879in}}%
\pgfpathlineto{\pgfqpoint{1.451058in}{4.417769in}}%
\pgfpathlineto{\pgfqpoint{1.453382in}{4.428029in}}%
\pgfpathlineto{\pgfqpoint{1.455706in}{4.456116in}}%
\pgfpathlineto{\pgfqpoint{1.458030in}{4.452315in}}%
\pgfpathlineto{\pgfqpoint{1.460354in}{4.462928in}}%
\pgfpathlineto{\pgfqpoint{1.462678in}{4.398191in}}%
\pgfpathlineto{\pgfqpoint{1.465002in}{4.462401in}}%
\pgfpathlineto{\pgfqpoint{1.467326in}{4.490381in}}%
\pgfpathlineto{\pgfqpoint{1.469650in}{4.481590in}}%
\pgfpathlineto{\pgfqpoint{1.474298in}{4.404800in}}%
\pgfpathlineto{\pgfqpoint{1.476622in}{4.457932in}}%
\pgfpathlineto{\pgfqpoint{1.478946in}{4.399004in}}%
\pgfpathlineto{\pgfqpoint{1.483594in}{4.476870in}}%
\pgfpathlineto{\pgfqpoint{1.485918in}{4.424466in}}%
\pgfpathlineto{\pgfqpoint{1.488242in}{4.504608in}}%
\pgfpathlineto{\pgfqpoint{1.492890in}{4.463046in}}%
\pgfpathlineto{\pgfqpoint{1.495214in}{4.480250in}}%
\pgfpathlineto{\pgfqpoint{1.497538in}{4.429934in}}%
\pgfpathlineto{\pgfqpoint{1.499862in}{4.431326in}}%
\pgfpathlineto{\pgfqpoint{1.502186in}{4.430596in}}%
\pgfpathlineto{\pgfqpoint{1.504510in}{4.414669in}}%
\pgfpathlineto{\pgfqpoint{1.506834in}{4.455923in}}%
\pgfpathlineto{\pgfqpoint{1.509157in}{4.424816in}}%
\pgfpathlineto{\pgfqpoint{1.511481in}{4.375903in}}%
\pgfpathlineto{\pgfqpoint{1.513805in}{4.434286in}}%
\pgfpathlineto{\pgfqpoint{1.516129in}{4.404761in}}%
\pgfpathlineto{\pgfqpoint{1.518453in}{4.414239in}}%
\pgfpathlineto{\pgfqpoint{1.520777in}{4.454445in}}%
\pgfpathlineto{\pgfqpoint{1.523101in}{4.454986in}}%
\pgfpathlineto{\pgfqpoint{1.525425in}{4.479605in}}%
\pgfpathlineto{\pgfqpoint{1.527749in}{4.450282in}}%
\pgfpathlineto{\pgfqpoint{1.530073in}{4.387412in}}%
\pgfpathlineto{\pgfqpoint{1.532397in}{4.449792in}}%
\pgfpathlineto{\pgfqpoint{1.534721in}{4.398277in}}%
\pgfpathlineto{\pgfqpoint{1.537045in}{4.478951in}}%
\pgfpathlineto{\pgfqpoint{1.539369in}{4.427343in}}%
\pgfpathlineto{\pgfqpoint{1.541693in}{4.415610in}}%
\pgfpathlineto{\pgfqpoint{1.544017in}{4.415866in}}%
\pgfpathlineto{\pgfqpoint{1.546341in}{4.464783in}}%
\pgfpathlineto{\pgfqpoint{1.548665in}{4.385374in}}%
\pgfpathlineto{\pgfqpoint{1.550989in}{4.436021in}}%
\pgfpathlineto{\pgfqpoint{1.557961in}{4.393108in}}%
\pgfpathlineto{\pgfqpoint{1.560285in}{4.405064in}}%
\pgfpathlineto{\pgfqpoint{1.562609in}{4.437373in}}%
\pgfpathlineto{\pgfqpoint{1.564933in}{4.440469in}}%
\pgfpathlineto{\pgfqpoint{1.569581in}{4.397269in}}%
\pgfpathlineto{\pgfqpoint{1.571905in}{4.392558in}}%
\pgfpathlineto{\pgfqpoint{1.574229in}{4.391034in}}%
\pgfpathlineto{\pgfqpoint{1.576553in}{4.408873in}}%
\pgfpathlineto{\pgfqpoint{1.578876in}{4.383779in}}%
\pgfpathlineto{\pgfqpoint{1.583524in}{4.442275in}}%
\pgfpathlineto{\pgfqpoint{1.585848in}{4.430604in}}%
\pgfpathlineto{\pgfqpoint{1.588172in}{4.403104in}}%
\pgfpathlineto{\pgfqpoint{1.590496in}{4.391312in}}%
\pgfpathlineto{\pgfqpoint{1.592820in}{4.405582in}}%
\pgfpathlineto{\pgfqpoint{1.595144in}{4.456607in}}%
\pgfpathlineto{\pgfqpoint{1.599792in}{4.395672in}}%
\pgfpathlineto{\pgfqpoint{1.602116in}{4.383713in}}%
\pgfpathlineto{\pgfqpoint{1.606764in}{4.375643in}}%
\pgfpathlineto{\pgfqpoint{1.609088in}{4.393761in}}%
\pgfpathlineto{\pgfqpoint{1.613736in}{4.394562in}}%
\pgfpathlineto{\pgfqpoint{1.616060in}{4.414435in}}%
\pgfpathlineto{\pgfqpoint{1.618384in}{4.358704in}}%
\pgfpathlineto{\pgfqpoint{1.620708in}{4.375316in}}%
\pgfpathlineto{\pgfqpoint{1.623032in}{4.330289in}}%
\pgfpathlineto{\pgfqpoint{1.625356in}{4.380031in}}%
\pgfpathlineto{\pgfqpoint{1.627680in}{4.352245in}}%
\pgfpathlineto{\pgfqpoint{1.630004in}{4.382729in}}%
\pgfpathlineto{\pgfqpoint{1.632328in}{4.386700in}}%
\pgfpathlineto{\pgfqpoint{1.634652in}{4.332690in}}%
\pgfpathlineto{\pgfqpoint{1.636976in}{4.327763in}}%
\pgfpathlineto{\pgfqpoint{1.639300in}{4.328344in}}%
\pgfpathlineto{\pgfqpoint{1.641624in}{4.372409in}}%
\pgfpathlineto{\pgfqpoint{1.643948in}{4.351061in}}%
\pgfpathlineto{\pgfqpoint{1.646272in}{4.400598in}}%
\pgfpathlineto{\pgfqpoint{1.648595in}{4.390660in}}%
\pgfpathlineto{\pgfqpoint{1.650919in}{4.387230in}}%
\pgfpathlineto{\pgfqpoint{1.653243in}{4.368526in}}%
\pgfpathlineto{\pgfqpoint{1.655567in}{4.392360in}}%
\pgfpathlineto{\pgfqpoint{1.657891in}{4.356582in}}%
\pgfpathlineto{\pgfqpoint{1.660215in}{4.346419in}}%
\pgfpathlineto{\pgfqpoint{1.662539in}{4.395924in}}%
\pgfpathlineto{\pgfqpoint{1.664863in}{4.311976in}}%
\pgfpathlineto{\pgfqpoint{1.669511in}{4.331855in}}%
\pgfpathlineto{\pgfqpoint{1.671835in}{4.358087in}}%
\pgfpathlineto{\pgfqpoint{1.674159in}{4.350137in}}%
\pgfpathlineto{\pgfqpoint{1.676483in}{4.355206in}}%
\pgfpathlineto{\pgfqpoint{1.678807in}{4.346118in}}%
\pgfpathlineto{\pgfqpoint{1.681131in}{4.352162in}}%
\pgfpathlineto{\pgfqpoint{1.683455in}{4.313426in}}%
\pgfpathlineto{\pgfqpoint{1.685779in}{4.323730in}}%
\pgfpathlineto{\pgfqpoint{1.688103in}{4.354950in}}%
\pgfpathlineto{\pgfqpoint{1.690427in}{4.330459in}}%
\pgfpathlineto{\pgfqpoint{1.692751in}{4.350286in}}%
\pgfpathlineto{\pgfqpoint{1.697399in}{4.315978in}}%
\pgfpathlineto{\pgfqpoint{1.699723in}{4.318589in}}%
\pgfpathlineto{\pgfqpoint{1.702047in}{4.316650in}}%
\pgfpathlineto{\pgfqpoint{1.704371in}{4.292780in}}%
\pgfpathlineto{\pgfqpoint{1.706695in}{4.336838in}}%
\pgfpathlineto{\pgfqpoint{1.715991in}{4.299084in}}%
\pgfpathlineto{\pgfqpoint{1.718314in}{4.295336in}}%
\pgfpathlineto{\pgfqpoint{1.720638in}{4.310614in}}%
\pgfpathlineto{\pgfqpoint{1.722962in}{4.301687in}}%
\pgfpathlineto{\pgfqpoint{1.725286in}{4.347370in}}%
\pgfpathlineto{\pgfqpoint{1.727610in}{4.277182in}}%
\pgfpathlineto{\pgfqpoint{1.729934in}{4.250599in}}%
\pgfpathlineto{\pgfqpoint{1.732258in}{4.323124in}}%
\pgfpathlineto{\pgfqpoint{1.734582in}{4.262635in}}%
\pgfpathlineto{\pgfqpoint{1.736906in}{4.262040in}}%
\pgfpathlineto{\pgfqpoint{1.741554in}{4.281155in}}%
\pgfpathlineto{\pgfqpoint{1.743878in}{4.314301in}}%
\pgfpathlineto{\pgfqpoint{1.746202in}{4.293907in}}%
\pgfpathlineto{\pgfqpoint{1.748526in}{4.289057in}}%
\pgfpathlineto{\pgfqpoint{1.750850in}{4.318286in}}%
\pgfpathlineto{\pgfqpoint{1.753174in}{4.259973in}}%
\pgfpathlineto{\pgfqpoint{1.755498in}{4.254548in}}%
\pgfpathlineto{\pgfqpoint{1.757822in}{4.301401in}}%
\pgfpathlineto{\pgfqpoint{1.760146in}{4.232974in}}%
\pgfpathlineto{\pgfqpoint{1.764794in}{4.272225in}}%
\pgfpathlineto{\pgfqpoint{1.767118in}{4.274510in}}%
\pgfpathlineto{\pgfqpoint{1.769442in}{4.269706in}}%
\pgfpathlineto{\pgfqpoint{1.771766in}{4.220154in}}%
\pgfpathlineto{\pgfqpoint{1.774090in}{4.260010in}}%
\pgfpathlineto{\pgfqpoint{1.776414in}{4.226456in}}%
\pgfpathlineto{\pgfqpoint{1.778738in}{4.242216in}}%
\pgfpathlineto{\pgfqpoint{1.781062in}{4.223861in}}%
\pgfpathlineto{\pgfqpoint{1.783386in}{4.220169in}}%
\pgfpathlineto{\pgfqpoint{1.785710in}{4.248312in}}%
\pgfpathlineto{\pgfqpoint{1.788034in}{4.199383in}}%
\pgfpathlineto{\pgfqpoint{1.790357in}{4.240068in}}%
\pgfpathlineto{\pgfqpoint{1.792681in}{4.250034in}}%
\pgfpathlineto{\pgfqpoint{1.797329in}{4.220393in}}%
\pgfpathlineto{\pgfqpoint{1.799653in}{4.248414in}}%
\pgfpathlineto{\pgfqpoint{1.801977in}{4.229997in}}%
\pgfpathlineto{\pgfqpoint{1.804301in}{4.188153in}}%
\pgfpathlineto{\pgfqpoint{1.806625in}{4.222697in}}%
\pgfpathlineto{\pgfqpoint{1.808949in}{4.220400in}}%
\pgfpathlineto{\pgfqpoint{1.811273in}{4.194899in}}%
\pgfpathlineto{\pgfqpoint{1.813597in}{4.233533in}}%
\pgfpathlineto{\pgfqpoint{1.815921in}{4.210594in}}%
\pgfpathlineto{\pgfqpoint{1.818245in}{4.217696in}}%
\pgfpathlineto{\pgfqpoint{1.820569in}{4.210647in}}%
\pgfpathlineto{\pgfqpoint{1.822893in}{4.214159in}}%
\pgfpathlineto{\pgfqpoint{1.825217in}{4.168811in}}%
\pgfpathlineto{\pgfqpoint{1.827541in}{4.175186in}}%
\pgfpathlineto{\pgfqpoint{1.829865in}{4.200686in}}%
\pgfpathlineto{\pgfqpoint{1.832189in}{4.197840in}}%
\pgfpathlineto{\pgfqpoint{1.834513in}{4.206685in}}%
\pgfpathlineto{\pgfqpoint{1.836837in}{4.237431in}}%
\pgfpathlineto{\pgfqpoint{1.839161in}{4.242154in}}%
\pgfpathlineto{\pgfqpoint{1.841485in}{4.233790in}}%
\pgfpathlineto{\pgfqpoint{1.846133in}{4.200359in}}%
\pgfpathlineto{\pgfqpoint{1.848457in}{4.193030in}}%
\pgfpathlineto{\pgfqpoint{1.850781in}{4.181871in}}%
\pgfpathlineto{\pgfqpoint{1.855429in}{4.208500in}}%
\pgfpathlineto{\pgfqpoint{1.857753in}{4.170523in}}%
\pgfpathlineto{\pgfqpoint{1.860076in}{4.171703in}}%
\pgfpathlineto{\pgfqpoint{1.862400in}{4.200970in}}%
\pgfpathlineto{\pgfqpoint{1.864724in}{4.203290in}}%
\pgfpathlineto{\pgfqpoint{1.867048in}{4.208509in}}%
\pgfpathlineto{\pgfqpoint{1.869372in}{4.151343in}}%
\pgfpathlineto{\pgfqpoint{1.871696in}{4.140760in}}%
\pgfpathlineto{\pgfqpoint{1.874020in}{4.181277in}}%
\pgfpathlineto{\pgfqpoint{1.876344in}{4.185238in}}%
\pgfpathlineto{\pgfqpoint{1.878668in}{4.175989in}}%
\pgfpathlineto{\pgfqpoint{1.880992in}{4.162025in}}%
\pgfpathlineto{\pgfqpoint{1.883316in}{4.156902in}}%
\pgfpathlineto{\pgfqpoint{1.885640in}{4.119260in}}%
\pgfpathlineto{\pgfqpoint{1.887964in}{4.138832in}}%
\pgfpathlineto{\pgfqpoint{1.890288in}{4.143434in}}%
\pgfpathlineto{\pgfqpoint{1.892612in}{4.141707in}}%
\pgfpathlineto{\pgfqpoint{1.897260in}{4.189391in}}%
\pgfpathlineto{\pgfqpoint{1.899584in}{4.165727in}}%
\pgfpathlineto{\pgfqpoint{1.901908in}{4.106572in}}%
\pgfpathlineto{\pgfqpoint{1.904232in}{4.149562in}}%
\pgfpathlineto{\pgfqpoint{1.906556in}{4.147545in}}%
\pgfpathlineto{\pgfqpoint{1.908880in}{4.122356in}}%
\pgfpathlineto{\pgfqpoint{1.911204in}{4.135752in}}%
\pgfpathlineto{\pgfqpoint{1.913528in}{4.116655in}}%
\pgfpathlineto{\pgfqpoint{1.915852in}{4.155779in}}%
\pgfpathlineto{\pgfqpoint{1.918176in}{4.125392in}}%
\pgfpathlineto{\pgfqpoint{1.920500in}{4.175575in}}%
\pgfpathlineto{\pgfqpoint{1.922824in}{4.073655in}}%
\pgfpathlineto{\pgfqpoint{1.925148in}{4.168471in}}%
\pgfpathlineto{\pgfqpoint{1.927472in}{4.157906in}}%
\pgfpathlineto{\pgfqpoint{1.929795in}{4.126144in}}%
\pgfpathlineto{\pgfqpoint{1.932119in}{4.072403in}}%
\pgfpathlineto{\pgfqpoint{1.934443in}{4.091115in}}%
\pgfpathlineto{\pgfqpoint{1.936767in}{4.083698in}}%
\pgfpathlineto{\pgfqpoint{1.939091in}{4.087603in}}%
\pgfpathlineto{\pgfqpoint{1.941415in}{4.114255in}}%
\pgfpathlineto{\pgfqpoint{1.943739in}{4.071888in}}%
\pgfpathlineto{\pgfqpoint{1.946063in}{4.101872in}}%
\pgfpathlineto{\pgfqpoint{1.948387in}{4.039849in}}%
\pgfpathlineto{\pgfqpoint{1.950711in}{4.100777in}}%
\pgfpathlineto{\pgfqpoint{1.953035in}{4.052542in}}%
\pgfpathlineto{\pgfqpoint{1.955359in}{4.101953in}}%
\pgfpathlineto{\pgfqpoint{1.957683in}{4.069234in}}%
\pgfpathlineto{\pgfqpoint{1.960007in}{4.061968in}}%
\pgfpathlineto{\pgfqpoint{1.966979in}{4.075777in}}%
\pgfpathlineto{\pgfqpoint{1.969303in}{4.111595in}}%
\pgfpathlineto{\pgfqpoint{1.971627in}{4.082935in}}%
\pgfpathlineto{\pgfqpoint{1.973951in}{4.074219in}}%
\pgfpathlineto{\pgfqpoint{1.976275in}{4.046526in}}%
\pgfpathlineto{\pgfqpoint{1.978599in}{4.040370in}}%
\pgfpathlineto{\pgfqpoint{1.980923in}{4.041400in}}%
\pgfpathlineto{\pgfqpoint{1.983247in}{4.078986in}}%
\pgfpathlineto{\pgfqpoint{1.985571in}{4.034476in}}%
\pgfpathlineto{\pgfqpoint{1.987895in}{4.033831in}}%
\pgfpathlineto{\pgfqpoint{1.990219in}{4.120664in}}%
\pgfpathlineto{\pgfqpoint{1.992543in}{4.043917in}}%
\pgfpathlineto{\pgfqpoint{1.994867in}{4.034107in}}%
\pgfpathlineto{\pgfqpoint{1.997191in}{4.030023in}}%
\pgfpathlineto{\pgfqpoint{1.999514in}{4.033234in}}%
\pgfpathlineto{\pgfqpoint{2.001838in}{4.060169in}}%
\pgfpathlineto{\pgfqpoint{2.004162in}{4.052314in}}%
\pgfpathlineto{\pgfqpoint{2.006486in}{4.029530in}}%
\pgfpathlineto{\pgfqpoint{2.011134in}{4.068625in}}%
\pgfpathlineto{\pgfqpoint{2.013458in}{4.078729in}}%
\pgfpathlineto{\pgfqpoint{2.015782in}{4.027034in}}%
\pgfpathlineto{\pgfqpoint{2.018106in}{4.063049in}}%
\pgfpathlineto{\pgfqpoint{2.022754in}{4.004216in}}%
\pgfpathlineto{\pgfqpoint{2.025078in}{4.049892in}}%
\pgfpathlineto{\pgfqpoint{2.027402in}{4.051013in}}%
\pgfpathlineto{\pgfqpoint{2.029726in}{4.011672in}}%
\pgfpathlineto{\pgfqpoint{2.032050in}{4.042749in}}%
\pgfpathlineto{\pgfqpoint{2.036698in}{3.999578in}}%
\pgfpathlineto{\pgfqpoint{2.039022in}{4.063695in}}%
\pgfpathlineto{\pgfqpoint{2.041346in}{3.983096in}}%
\pgfpathlineto{\pgfqpoint{2.043670in}{4.044597in}}%
\pgfpathlineto{\pgfqpoint{2.045994in}{4.029343in}}%
\pgfpathlineto{\pgfqpoint{2.050642in}{3.938857in}}%
\pgfpathlineto{\pgfqpoint{2.052966in}{3.993972in}}%
\pgfpathlineto{\pgfqpoint{2.055290in}{3.972401in}}%
\pgfpathlineto{\pgfqpoint{2.057614in}{4.007928in}}%
\pgfpathlineto{\pgfqpoint{2.059938in}{4.019742in}}%
\pgfpathlineto{\pgfqpoint{2.062262in}{3.917635in}}%
\pgfpathlineto{\pgfqpoint{2.064586in}{4.010902in}}%
\pgfpathlineto{\pgfqpoint{2.066910in}{4.006526in}}%
\pgfpathlineto{\pgfqpoint{2.071557in}{3.955912in}}%
\pgfpathlineto{\pgfqpoint{2.073881in}{4.013146in}}%
\pgfpathlineto{\pgfqpoint{2.076205in}{3.993792in}}%
\pgfpathlineto{\pgfqpoint{2.078529in}{4.010599in}}%
\pgfpathlineto{\pgfqpoint{2.080853in}{4.012032in}}%
\pgfpathlineto{\pgfqpoint{2.083177in}{3.977379in}}%
\pgfpathlineto{\pgfqpoint{2.085501in}{3.969752in}}%
\pgfpathlineto{\pgfqpoint{2.087825in}{3.986889in}}%
\pgfpathlineto{\pgfqpoint{2.090149in}{3.948009in}}%
\pgfpathlineto{\pgfqpoint{2.092473in}{3.981175in}}%
\pgfpathlineto{\pgfqpoint{2.094797in}{4.052454in}}%
\pgfpathlineto{\pgfqpoint{2.097121in}{3.956324in}}%
\pgfpathlineto{\pgfqpoint{2.099445in}{3.915617in}}%
\pgfpathlineto{\pgfqpoint{2.101769in}{3.967032in}}%
\pgfpathlineto{\pgfqpoint{2.104093in}{3.972775in}}%
\pgfpathlineto{\pgfqpoint{2.106417in}{3.954658in}}%
\pgfpathlineto{\pgfqpoint{2.108741in}{3.972277in}}%
\pgfpathlineto{\pgfqpoint{2.111065in}{3.939847in}}%
\pgfpathlineto{\pgfqpoint{2.113389in}{3.925715in}}%
\pgfpathlineto{\pgfqpoint{2.115713in}{3.971948in}}%
\pgfpathlineto{\pgfqpoint{2.118037in}{3.964854in}}%
\pgfpathlineto{\pgfqpoint{2.120361in}{3.951619in}}%
\pgfpathlineto{\pgfqpoint{2.122685in}{3.962055in}}%
\pgfpathlineto{\pgfqpoint{2.125009in}{3.997799in}}%
\pgfpathlineto{\pgfqpoint{2.127333in}{3.914818in}}%
\pgfpathlineto{\pgfqpoint{2.129657in}{3.965208in}}%
\pgfpathlineto{\pgfqpoint{2.131981in}{3.911271in}}%
\pgfpathlineto{\pgfqpoint{2.136629in}{3.977805in}}%
\pgfpathlineto{\pgfqpoint{2.138953in}{3.978576in}}%
\pgfpathlineto{\pgfqpoint{2.141276in}{3.939175in}}%
\pgfpathlineto{\pgfqpoint{2.143600in}{3.951091in}}%
\pgfpathlineto{\pgfqpoint{2.145924in}{3.954560in}}%
\pgfpathlineto{\pgfqpoint{2.148248in}{3.955191in}}%
\pgfpathlineto{\pgfqpoint{2.150572in}{3.990260in}}%
\pgfpathlineto{\pgfqpoint{2.152896in}{3.972758in}}%
\pgfpathlineto{\pgfqpoint{2.157544in}{3.919710in}}%
\pgfpathlineto{\pgfqpoint{2.159868in}{3.897377in}}%
\pgfpathlineto{\pgfqpoint{2.162192in}{3.961689in}}%
\pgfpathlineto{\pgfqpoint{2.164516in}{3.910868in}}%
\pgfpathlineto{\pgfqpoint{2.166840in}{3.963242in}}%
\pgfpathlineto{\pgfqpoint{2.169164in}{3.910774in}}%
\pgfpathlineto{\pgfqpoint{2.171488in}{4.006810in}}%
\pgfpathlineto{\pgfqpoint{2.173812in}{3.904617in}}%
\pgfpathlineto{\pgfqpoint{2.176136in}{3.965725in}}%
\pgfpathlineto{\pgfqpoint{2.178460in}{3.907982in}}%
\pgfpathlineto{\pgfqpoint{2.180784in}{3.917753in}}%
\pgfpathlineto{\pgfqpoint{2.183108in}{3.903799in}}%
\pgfpathlineto{\pgfqpoint{2.185432in}{3.945418in}}%
\pgfpathlineto{\pgfqpoint{2.187756in}{3.918709in}}%
\pgfpathlineto{\pgfqpoint{2.190080in}{3.990065in}}%
\pgfpathlineto{\pgfqpoint{2.192404in}{3.909200in}}%
\pgfpathlineto{\pgfqpoint{2.194728in}{3.971796in}}%
\pgfpathlineto{\pgfqpoint{2.197052in}{3.906256in}}%
\pgfpathlineto{\pgfqpoint{2.199376in}{3.968280in}}%
\pgfpathlineto{\pgfqpoint{2.201700in}{3.919549in}}%
\pgfpathlineto{\pgfqpoint{2.204024in}{3.962977in}}%
\pgfpathlineto{\pgfqpoint{2.208672in}{3.915287in}}%
\pgfpathlineto{\pgfqpoint{2.210995in}{3.906809in}}%
\pgfpathlineto{\pgfqpoint{2.213319in}{3.957566in}}%
\pgfpathlineto{\pgfqpoint{2.215643in}{3.931249in}}%
\pgfpathlineto{\pgfqpoint{2.217967in}{3.929948in}}%
\pgfpathlineto{\pgfqpoint{2.220291in}{3.946104in}}%
\pgfpathlineto{\pgfqpoint{2.222615in}{3.940337in}}%
\pgfpathlineto{\pgfqpoint{2.224939in}{3.949142in}}%
\pgfpathlineto{\pgfqpoint{2.227263in}{3.926029in}}%
\pgfpathlineto{\pgfqpoint{2.229587in}{3.870791in}}%
\pgfpathlineto{\pgfqpoint{2.231911in}{3.931236in}}%
\pgfpathlineto{\pgfqpoint{2.234235in}{3.915463in}}%
\pgfpathlineto{\pgfqpoint{2.236559in}{3.942427in}}%
\pgfpathlineto{\pgfqpoint{2.238883in}{3.924458in}}%
\pgfpathlineto{\pgfqpoint{2.241207in}{3.916100in}}%
\pgfpathlineto{\pgfqpoint{2.243531in}{3.915563in}}%
\pgfpathlineto{\pgfqpoint{2.245855in}{3.972899in}}%
\pgfpathlineto{\pgfqpoint{2.250503in}{3.872441in}}%
\pgfpathlineto{\pgfqpoint{2.252827in}{3.970468in}}%
\pgfpathlineto{\pgfqpoint{2.255151in}{3.928656in}}%
\pgfpathlineto{\pgfqpoint{2.262123in}{3.881811in}}%
\pgfpathlineto{\pgfqpoint{2.264447in}{3.892920in}}%
\pgfpathlineto{\pgfqpoint{2.266771in}{3.927940in}}%
\pgfpathlineto{\pgfqpoint{2.269095in}{3.907595in}}%
\pgfpathlineto{\pgfqpoint{2.271419in}{3.905114in}}%
\pgfpathlineto{\pgfqpoint{2.276067in}{3.950608in}}%
\pgfpathlineto{\pgfqpoint{2.278391in}{3.913054in}}%
\pgfpathlineto{\pgfqpoint{2.280715in}{3.919675in}}%
\pgfpathlineto{\pgfqpoint{2.283038in}{3.946171in}}%
\pgfpathlineto{\pgfqpoint{2.285362in}{3.910433in}}%
\pgfpathlineto{\pgfqpoint{2.294658in}{3.847204in}}%
\pgfpathlineto{\pgfqpoint{2.299306in}{3.931532in}}%
\pgfpathlineto{\pgfqpoint{2.301630in}{3.910706in}}%
\pgfpathlineto{\pgfqpoint{2.303954in}{3.940961in}}%
\pgfpathlineto{\pgfqpoint{2.306278in}{3.905246in}}%
\pgfpathlineto{\pgfqpoint{2.308602in}{3.910836in}}%
\pgfpathlineto{\pgfqpoint{2.310926in}{3.922009in}}%
\pgfpathlineto{\pgfqpoint{2.313250in}{3.911454in}}%
\pgfpathlineto{\pgfqpoint{2.315574in}{3.930327in}}%
\pgfpathlineto{\pgfqpoint{2.317898in}{3.936115in}}%
\pgfpathlineto{\pgfqpoint{2.320222in}{3.909393in}}%
\pgfpathlineto{\pgfqpoint{2.322546in}{3.860735in}}%
\pgfpathlineto{\pgfqpoint{2.324870in}{3.915103in}}%
\pgfpathlineto{\pgfqpoint{2.327194in}{3.906331in}}%
\pgfpathlineto{\pgfqpoint{2.329518in}{3.922936in}}%
\pgfpathlineto{\pgfqpoint{2.331842in}{3.876986in}}%
\pgfpathlineto{\pgfqpoint{2.334166in}{3.911491in}}%
\pgfpathlineto{\pgfqpoint{2.338814in}{3.947268in}}%
\pgfpathlineto{\pgfqpoint{2.341138in}{3.860997in}}%
\pgfpathlineto{\pgfqpoint{2.343462in}{3.928836in}}%
\pgfpathlineto{\pgfqpoint{2.345786in}{3.934345in}}%
\pgfpathlineto{\pgfqpoint{2.348110in}{3.867790in}}%
\pgfpathlineto{\pgfqpoint{2.350434in}{3.872141in}}%
\pgfpathlineto{\pgfqpoint{2.352757in}{3.900978in}}%
\pgfpathlineto{\pgfqpoint{2.355081in}{3.883230in}}%
\pgfpathlineto{\pgfqpoint{2.357405in}{3.910287in}}%
\pgfpathlineto{\pgfqpoint{2.359729in}{3.906737in}}%
\pgfpathlineto{\pgfqpoint{2.362053in}{3.897729in}}%
\pgfpathlineto{\pgfqpoint{2.364377in}{3.925155in}}%
\pgfpathlineto{\pgfqpoint{2.366701in}{3.863257in}}%
\pgfpathlineto{\pgfqpoint{2.371349in}{3.910963in}}%
\pgfpathlineto{\pgfqpoint{2.373673in}{3.917287in}}%
\pgfpathlineto{\pgfqpoint{2.375997in}{3.899716in}}%
\pgfpathlineto{\pgfqpoint{2.380645in}{3.926180in}}%
\pgfpathlineto{\pgfqpoint{2.382969in}{3.893596in}}%
\pgfpathlineto{\pgfqpoint{2.385293in}{3.883267in}}%
\pgfpathlineto{\pgfqpoint{2.387617in}{3.930554in}}%
\pgfpathlineto{\pgfqpoint{2.389941in}{3.896142in}}%
\pgfpathlineto{\pgfqpoint{2.392265in}{3.896201in}}%
\pgfpathlineto{\pgfqpoint{2.394589in}{3.897449in}}%
\pgfpathlineto{\pgfqpoint{2.396913in}{3.944088in}}%
\pgfpathlineto{\pgfqpoint{2.399237in}{3.901302in}}%
\pgfpathlineto{\pgfqpoint{2.401561in}{3.896000in}}%
\pgfpathlineto{\pgfqpoint{2.403885in}{3.958410in}}%
\pgfpathlineto{\pgfqpoint{2.406209in}{3.872750in}}%
\pgfpathlineto{\pgfqpoint{2.408533in}{3.936249in}}%
\pgfpathlineto{\pgfqpoint{2.410857in}{3.968535in}}%
\pgfpathlineto{\pgfqpoint{2.415505in}{3.919483in}}%
\pgfpathlineto{\pgfqpoint{2.417829in}{3.938409in}}%
\pgfpathlineto{\pgfqpoint{2.420153in}{3.911484in}}%
\pgfpathlineto{\pgfqpoint{2.422476in}{3.868129in}}%
\pgfpathlineto{\pgfqpoint{2.424800in}{3.966978in}}%
\pgfpathlineto{\pgfqpoint{2.427124in}{3.868799in}}%
\pgfpathlineto{\pgfqpoint{2.429448in}{3.966906in}}%
\pgfpathlineto{\pgfqpoint{2.431772in}{3.931153in}}%
\pgfpathlineto{\pgfqpoint{2.434096in}{3.920863in}}%
\pgfpathlineto{\pgfqpoint{2.436420in}{3.957463in}}%
\pgfpathlineto{\pgfqpoint{2.438744in}{3.920169in}}%
\pgfpathlineto{\pgfqpoint{2.441068in}{3.969217in}}%
\pgfpathlineto{\pgfqpoint{2.443392in}{3.957792in}}%
\pgfpathlineto{\pgfqpoint{2.445716in}{3.968203in}}%
\pgfpathlineto{\pgfqpoint{2.448040in}{3.937151in}}%
\pgfpathlineto{\pgfqpoint{2.450364in}{3.997853in}}%
\pgfpathlineto{\pgfqpoint{2.452688in}{3.911041in}}%
\pgfpathlineto{\pgfqpoint{2.455012in}{3.921698in}}%
\pgfpathlineto{\pgfqpoint{2.459660in}{3.991128in}}%
\pgfpathlineto{\pgfqpoint{2.464308in}{3.900207in}}%
\pgfpathlineto{\pgfqpoint{2.466632in}{3.945448in}}%
\pgfpathlineto{\pgfqpoint{2.468956in}{3.926106in}}%
\pgfpathlineto{\pgfqpoint{2.471280in}{3.937668in}}%
\pgfpathlineto{\pgfqpoint{2.473604in}{3.960051in}}%
\pgfpathlineto{\pgfqpoint{2.475928in}{3.927432in}}%
\pgfpathlineto{\pgfqpoint{2.478252in}{3.927450in}}%
\pgfpathlineto{\pgfqpoint{2.480576in}{3.963245in}}%
\pgfpathlineto{\pgfqpoint{2.482900in}{3.953709in}}%
\pgfpathlineto{\pgfqpoint{2.485224in}{3.990550in}}%
\pgfpathlineto{\pgfqpoint{2.489872in}{3.925110in}}%
\pgfpathlineto{\pgfqpoint{2.492195in}{3.910629in}}%
\pgfpathlineto{\pgfqpoint{2.494519in}{3.957063in}}%
\pgfpathlineto{\pgfqpoint{2.496843in}{3.947314in}}%
\pgfpathlineto{\pgfqpoint{2.499167in}{3.972221in}}%
\pgfpathlineto{\pgfqpoint{2.501491in}{3.982359in}}%
\pgfpathlineto{\pgfqpoint{2.503815in}{3.984273in}}%
\pgfpathlineto{\pgfqpoint{2.506139in}{3.983866in}}%
\pgfpathlineto{\pgfqpoint{2.510787in}{3.944300in}}%
\pgfpathlineto{\pgfqpoint{2.513111in}{3.970972in}}%
\pgfpathlineto{\pgfqpoint{2.515435in}{3.977497in}}%
\pgfpathlineto{\pgfqpoint{2.517759in}{4.016145in}}%
\pgfpathlineto{\pgfqpoint{2.520083in}{4.015993in}}%
\pgfpathlineto{\pgfqpoint{2.522407in}{3.942404in}}%
\pgfpathlineto{\pgfqpoint{2.524731in}{3.973133in}}%
\pgfpathlineto{\pgfqpoint{2.527055in}{3.965883in}}%
\pgfpathlineto{\pgfqpoint{2.529379in}{4.000741in}}%
\pgfpathlineto{\pgfqpoint{2.531703in}{3.957525in}}%
\pgfpathlineto{\pgfqpoint{2.534027in}{3.990451in}}%
\pgfpathlineto{\pgfqpoint{2.536351in}{3.990880in}}%
\pgfpathlineto{\pgfqpoint{2.538675in}{3.992873in}}%
\pgfpathlineto{\pgfqpoint{2.540999in}{3.951488in}}%
\pgfpathlineto{\pgfqpoint{2.543323in}{4.030991in}}%
\pgfpathlineto{\pgfqpoint{2.545647in}{4.032600in}}%
\pgfpathlineto{\pgfqpoint{2.547971in}{3.991524in}}%
\pgfpathlineto{\pgfqpoint{2.550295in}{3.991278in}}%
\pgfpathlineto{\pgfqpoint{2.552619in}{4.016060in}}%
\pgfpathlineto{\pgfqpoint{2.554943in}{3.969963in}}%
\pgfpathlineto{\pgfqpoint{2.557267in}{3.961752in}}%
\pgfpathlineto{\pgfqpoint{2.559591in}{4.054269in}}%
\pgfpathlineto{\pgfqpoint{2.561915in}{3.997140in}}%
\pgfpathlineto{\pgfqpoint{2.564238in}{4.034578in}}%
\pgfpathlineto{\pgfqpoint{2.566562in}{3.980246in}}%
\pgfpathlineto{\pgfqpoint{2.568886in}{4.008092in}}%
\pgfpathlineto{\pgfqpoint{2.571210in}{3.987708in}}%
\pgfpathlineto{\pgfqpoint{2.573534in}{4.030984in}}%
\pgfpathlineto{\pgfqpoint{2.575858in}{3.995984in}}%
\pgfpathlineto{\pgfqpoint{2.578182in}{4.013316in}}%
\pgfpathlineto{\pgfqpoint{2.580506in}{3.986873in}}%
\pgfpathlineto{\pgfqpoint{2.582830in}{4.000599in}}%
\pgfpathlineto{\pgfqpoint{2.585154in}{4.023976in}}%
\pgfpathlineto{\pgfqpoint{2.587478in}{3.988953in}}%
\pgfpathlineto{\pgfqpoint{2.589802in}{4.027424in}}%
\pgfpathlineto{\pgfqpoint{2.592126in}{4.004146in}}%
\pgfpathlineto{\pgfqpoint{2.594450in}{4.005561in}}%
\pgfpathlineto{\pgfqpoint{2.596774in}{4.041836in}}%
\pgfpathlineto{\pgfqpoint{2.599098in}{4.019446in}}%
\pgfpathlineto{\pgfqpoint{2.601422in}{4.012787in}}%
\pgfpathlineto{\pgfqpoint{2.603746in}{3.996887in}}%
\pgfpathlineto{\pgfqpoint{2.606070in}{4.045159in}}%
\pgfpathlineto{\pgfqpoint{2.610718in}{4.042552in}}%
\pgfpathlineto{\pgfqpoint{2.613042in}{4.022653in}}%
\pgfpathlineto{\pgfqpoint{2.615366in}{4.020410in}}%
\pgfpathlineto{\pgfqpoint{2.617690in}{4.077742in}}%
\pgfpathlineto{\pgfqpoint{2.620014in}{4.039211in}}%
\pgfpathlineto{\pgfqpoint{2.622338in}{4.074635in}}%
\pgfpathlineto{\pgfqpoint{2.624662in}{4.062764in}}%
\pgfpathlineto{\pgfqpoint{2.626986in}{4.084202in}}%
\pgfpathlineto{\pgfqpoint{2.629310in}{4.052207in}}%
\pgfpathlineto{\pgfqpoint{2.631634in}{4.044450in}}%
\pgfpathlineto{\pgfqpoint{2.633957in}{4.032240in}}%
\pgfpathlineto{\pgfqpoint{2.636281in}{4.038357in}}%
\pgfpathlineto{\pgfqpoint{2.638605in}{4.063935in}}%
\pgfpathlineto{\pgfqpoint{2.640929in}{4.026637in}}%
\pgfpathlineto{\pgfqpoint{2.647901in}{4.102709in}}%
\pgfpathlineto{\pgfqpoint{2.652549in}{4.070264in}}%
\pgfpathlineto{\pgfqpoint{2.654873in}{4.123526in}}%
\pgfpathlineto{\pgfqpoint{2.657197in}{4.052652in}}%
\pgfpathlineto{\pgfqpoint{2.659521in}{4.076909in}}%
\pgfpathlineto{\pgfqpoint{2.661845in}{4.123914in}}%
\pgfpathlineto{\pgfqpoint{2.664169in}{4.074982in}}%
\pgfpathlineto{\pgfqpoint{2.666493in}{4.116130in}}%
\pgfpathlineto{\pgfqpoint{2.668817in}{4.058531in}}%
\pgfpathlineto{\pgfqpoint{2.671141in}{4.041810in}}%
\pgfpathlineto{\pgfqpoint{2.673465in}{4.116634in}}%
\pgfpathlineto{\pgfqpoint{2.675789in}{4.081005in}}%
\pgfpathlineto{\pgfqpoint{2.678113in}{4.115552in}}%
\pgfpathlineto{\pgfqpoint{2.680437in}{4.045525in}}%
\pgfpathlineto{\pgfqpoint{2.682761in}{4.116991in}}%
\pgfpathlineto{\pgfqpoint{2.687409in}{4.127942in}}%
\pgfpathlineto{\pgfqpoint{2.689733in}{4.101022in}}%
\pgfpathlineto{\pgfqpoint{2.692057in}{4.140098in}}%
\pgfpathlineto{\pgfqpoint{2.694381in}{4.072065in}}%
\pgfpathlineto{\pgfqpoint{2.696705in}{4.134411in}}%
\pgfpathlineto{\pgfqpoint{2.701353in}{4.118323in}}%
\pgfpathlineto{\pgfqpoint{2.703676in}{4.136554in}}%
\pgfpathlineto{\pgfqpoint{2.706000in}{4.169757in}}%
\pgfpathlineto{\pgfqpoint{2.708324in}{4.096884in}}%
\pgfpathlineto{\pgfqpoint{2.712972in}{4.105673in}}%
\pgfpathlineto{\pgfqpoint{2.717620in}{4.145329in}}%
\pgfpathlineto{\pgfqpoint{2.719944in}{4.143403in}}%
\pgfpathlineto{\pgfqpoint{2.722268in}{4.102520in}}%
\pgfpathlineto{\pgfqpoint{2.726916in}{4.157378in}}%
\pgfpathlineto{\pgfqpoint{2.729240in}{4.147857in}}%
\pgfpathlineto{\pgfqpoint{2.731564in}{4.170223in}}%
\pgfpathlineto{\pgfqpoint{2.733888in}{4.151302in}}%
\pgfpathlineto{\pgfqpoint{2.736212in}{4.152019in}}%
\pgfpathlineto{\pgfqpoint{2.738536in}{4.145532in}}%
\pgfpathlineto{\pgfqpoint{2.740860in}{4.104044in}}%
\pgfpathlineto{\pgfqpoint{2.743184in}{4.185655in}}%
\pgfpathlineto{\pgfqpoint{2.745508in}{4.176517in}}%
\pgfpathlineto{\pgfqpoint{2.750156in}{4.173459in}}%
\pgfpathlineto{\pgfqpoint{2.752480in}{4.119757in}}%
\pgfpathlineto{\pgfqpoint{2.754804in}{4.186283in}}%
\pgfpathlineto{\pgfqpoint{2.757128in}{4.197136in}}%
\pgfpathlineto{\pgfqpoint{2.759452in}{4.202221in}}%
\pgfpathlineto{\pgfqpoint{2.761776in}{4.175713in}}%
\pgfpathlineto{\pgfqpoint{2.764100in}{4.209296in}}%
\pgfpathlineto{\pgfqpoint{2.766424in}{4.205114in}}%
\pgfpathlineto{\pgfqpoint{2.768748in}{4.195644in}}%
\pgfpathlineto{\pgfqpoint{2.771072in}{4.156229in}}%
\pgfpathlineto{\pgfqpoint{2.773395in}{4.170723in}}%
\pgfpathlineto{\pgfqpoint{2.775719in}{4.193509in}}%
\pgfpathlineto{\pgfqpoint{2.778043in}{4.234646in}}%
\pgfpathlineto{\pgfqpoint{2.780367in}{4.164858in}}%
\pgfpathlineto{\pgfqpoint{2.782691in}{4.184157in}}%
\pgfpathlineto{\pgfqpoint{2.785015in}{4.236201in}}%
\pgfpathlineto{\pgfqpoint{2.787339in}{4.186722in}}%
\pgfpathlineto{\pgfqpoint{2.789663in}{4.180453in}}%
\pgfpathlineto{\pgfqpoint{2.791987in}{4.199378in}}%
\pgfpathlineto{\pgfqpoint{2.796635in}{4.194933in}}%
\pgfpathlineto{\pgfqpoint{2.798959in}{4.231304in}}%
\pgfpathlineto{\pgfqpoint{2.801283in}{4.219487in}}%
\pgfpathlineto{\pgfqpoint{2.803607in}{4.240036in}}%
\pgfpathlineto{\pgfqpoint{2.805931in}{4.202089in}}%
\pgfpathlineto{\pgfqpoint{2.808255in}{4.249651in}}%
\pgfpathlineto{\pgfqpoint{2.810579in}{4.241580in}}%
\pgfpathlineto{\pgfqpoint{2.812903in}{4.279520in}}%
\pgfpathlineto{\pgfqpoint{2.817551in}{4.220671in}}%
\pgfpathlineto{\pgfqpoint{2.822199in}{4.256873in}}%
\pgfpathlineto{\pgfqpoint{2.824523in}{4.240227in}}%
\pgfpathlineto{\pgfqpoint{2.826847in}{4.247893in}}%
\pgfpathlineto{\pgfqpoint{2.829171in}{4.230605in}}%
\pgfpathlineto{\pgfqpoint{2.831495in}{4.258058in}}%
\pgfpathlineto{\pgfqpoint{2.833819in}{4.266023in}}%
\pgfpathlineto{\pgfqpoint{2.836143in}{4.235394in}}%
\pgfpathlineto{\pgfqpoint{2.838467in}{4.258487in}}%
\pgfpathlineto{\pgfqpoint{2.840791in}{4.298264in}}%
\pgfpathlineto{\pgfqpoint{2.843115in}{4.235306in}}%
\pgfpathlineto{\pgfqpoint{2.845438in}{4.237158in}}%
\pgfpathlineto{\pgfqpoint{2.850086in}{4.226018in}}%
\pgfpathlineto{\pgfqpoint{2.854734in}{4.274688in}}%
\pgfpathlineto{\pgfqpoint{2.857058in}{4.203093in}}%
\pgfpathlineto{\pgfqpoint{2.861706in}{4.293387in}}%
\pgfpathlineto{\pgfqpoint{2.864030in}{4.318482in}}%
\pgfpathlineto{\pgfqpoint{2.866354in}{4.240928in}}%
\pgfpathlineto{\pgfqpoint{2.868678in}{4.281879in}}%
\pgfpathlineto{\pgfqpoint{2.871002in}{4.283545in}}%
\pgfpathlineto{\pgfqpoint{2.873326in}{4.320885in}}%
\pgfpathlineto{\pgfqpoint{2.875650in}{4.299990in}}%
\pgfpathlineto{\pgfqpoint{2.877974in}{4.258886in}}%
\pgfpathlineto{\pgfqpoint{2.880298in}{4.340527in}}%
\pgfpathlineto{\pgfqpoint{2.882622in}{4.261275in}}%
\pgfpathlineto{\pgfqpoint{2.884946in}{4.229339in}}%
\pgfpathlineto{\pgfqpoint{2.887270in}{4.280726in}}%
\pgfpathlineto{\pgfqpoint{2.889594in}{4.284569in}}%
\pgfpathlineto{\pgfqpoint{2.891918in}{4.322609in}}%
\pgfpathlineto{\pgfqpoint{2.894242in}{4.302866in}}%
\pgfpathlineto{\pgfqpoint{2.896566in}{4.308067in}}%
\pgfpathlineto{\pgfqpoint{2.898890in}{4.324149in}}%
\pgfpathlineto{\pgfqpoint{2.901214in}{4.313212in}}%
\pgfpathlineto{\pgfqpoint{2.903538in}{4.279718in}}%
\pgfpathlineto{\pgfqpoint{2.905862in}{4.327727in}}%
\pgfpathlineto{\pgfqpoint{2.908186in}{4.306754in}}%
\pgfpathlineto{\pgfqpoint{2.910510in}{4.313351in}}%
\pgfpathlineto{\pgfqpoint{2.912834in}{4.376365in}}%
\pgfpathlineto{\pgfqpoint{2.915157in}{4.342409in}}%
\pgfpathlineto{\pgfqpoint{2.917481in}{4.279232in}}%
\pgfpathlineto{\pgfqpoint{2.919805in}{4.343531in}}%
\pgfpathlineto{\pgfqpoint{2.922129in}{4.347376in}}%
\pgfpathlineto{\pgfqpoint{2.924453in}{4.345803in}}%
\pgfpathlineto{\pgfqpoint{2.926777in}{4.363830in}}%
\pgfpathlineto{\pgfqpoint{2.931425in}{4.296139in}}%
\pgfpathlineto{\pgfqpoint{2.933749in}{4.353356in}}%
\pgfpathlineto{\pgfqpoint{2.936073in}{4.309857in}}%
\pgfpathlineto{\pgfqpoint{2.938397in}{4.351053in}}%
\pgfpathlineto{\pgfqpoint{2.940721in}{4.299977in}}%
\pgfpathlineto{\pgfqpoint{2.943045in}{4.363562in}}%
\pgfpathlineto{\pgfqpoint{2.945369in}{4.298847in}}%
\pgfpathlineto{\pgfqpoint{2.947693in}{4.398079in}}%
\pgfpathlineto{\pgfqpoint{2.952341in}{4.340499in}}%
\pgfpathlineto{\pgfqpoint{2.954665in}{4.320909in}}%
\pgfpathlineto{\pgfqpoint{2.956989in}{4.350601in}}%
\pgfpathlineto{\pgfqpoint{2.959313in}{4.347816in}}%
\pgfpathlineto{\pgfqpoint{2.961637in}{4.379961in}}%
\pgfpathlineto{\pgfqpoint{2.963961in}{4.347024in}}%
\pgfpathlineto{\pgfqpoint{2.966285in}{4.392042in}}%
\pgfpathlineto{\pgfqpoint{2.968609in}{4.352821in}}%
\pgfpathlineto{\pgfqpoint{2.970933in}{4.369679in}}%
\pgfpathlineto{\pgfqpoint{2.973257in}{4.338149in}}%
\pgfpathlineto{\pgfqpoint{2.975581in}{4.370121in}}%
\pgfpathlineto{\pgfqpoint{2.977905in}{4.420922in}}%
\pgfpathlineto{\pgfqpoint{2.980229in}{4.343988in}}%
\pgfpathlineto{\pgfqpoint{2.982553in}{4.376355in}}%
\pgfpathlineto{\pgfqpoint{2.984876in}{4.367637in}}%
\pgfpathlineto{\pgfqpoint{2.987200in}{4.388059in}}%
\pgfpathlineto{\pgfqpoint{2.989524in}{4.374315in}}%
\pgfpathlineto{\pgfqpoint{2.991848in}{4.405352in}}%
\pgfpathlineto{\pgfqpoint{2.994172in}{4.334957in}}%
\pgfpathlineto{\pgfqpoint{2.996496in}{4.403661in}}%
\pgfpathlineto{\pgfqpoint{2.998820in}{4.358044in}}%
\pgfpathlineto{\pgfqpoint{3.001144in}{4.400203in}}%
\pgfpathlineto{\pgfqpoint{3.003468in}{4.378934in}}%
\pgfpathlineto{\pgfqpoint{3.005792in}{4.381700in}}%
\pgfpathlineto{\pgfqpoint{3.008116in}{4.327068in}}%
\pgfpathlineto{\pgfqpoint{3.012764in}{4.397829in}}%
\pgfpathlineto{\pgfqpoint{3.015088in}{4.425543in}}%
\pgfpathlineto{\pgfqpoint{3.017412in}{4.376191in}}%
\pgfpathlineto{\pgfqpoint{3.019736in}{4.351990in}}%
\pgfpathlineto{\pgfqpoint{3.022060in}{4.415692in}}%
\pgfpathlineto{\pgfqpoint{3.024384in}{4.385645in}}%
\pgfpathlineto{\pgfqpoint{3.026708in}{4.407249in}}%
\pgfpathlineto{\pgfqpoint{3.029032in}{4.411080in}}%
\pgfpathlineto{\pgfqpoint{3.031356in}{4.421425in}}%
\pgfpathlineto{\pgfqpoint{3.033680in}{4.349389in}}%
\pgfpathlineto{\pgfqpoint{3.036004in}{4.402623in}}%
\pgfpathlineto{\pgfqpoint{3.038328in}{4.387693in}}%
\pgfpathlineto{\pgfqpoint{3.040652in}{4.424444in}}%
\pgfpathlineto{\pgfqpoint{3.042976in}{4.418773in}}%
\pgfpathlineto{\pgfqpoint{3.045300in}{4.388956in}}%
\pgfpathlineto{\pgfqpoint{3.047624in}{4.432600in}}%
\pgfpathlineto{\pgfqpoint{3.049948in}{4.425546in}}%
\pgfpathlineto{\pgfqpoint{3.052272in}{4.373694in}}%
\pgfpathlineto{\pgfqpoint{3.054596in}{4.353978in}}%
\pgfpathlineto{\pgfqpoint{3.056919in}{4.433636in}}%
\pgfpathlineto{\pgfqpoint{3.059243in}{4.397448in}}%
\pgfpathlineto{\pgfqpoint{3.061567in}{4.430269in}}%
\pgfpathlineto{\pgfqpoint{3.066215in}{4.391932in}}%
\pgfpathlineto{\pgfqpoint{3.068539in}{4.449519in}}%
\pgfpathlineto{\pgfqpoint{3.070863in}{4.407948in}}%
\pgfpathlineto{\pgfqpoint{3.073187in}{4.418089in}}%
\pgfpathlineto{\pgfqpoint{3.075511in}{4.422531in}}%
\pgfpathlineto{\pgfqpoint{3.077835in}{4.400299in}}%
\pgfpathlineto{\pgfqpoint{3.082483in}{4.460732in}}%
\pgfpathlineto{\pgfqpoint{3.084807in}{4.396062in}}%
\pgfpathlineto{\pgfqpoint{3.087131in}{4.451136in}}%
\pgfpathlineto{\pgfqpoint{3.089455in}{4.436006in}}%
\pgfpathlineto{\pgfqpoint{3.091779in}{4.404822in}}%
\pgfpathlineto{\pgfqpoint{3.094103in}{4.445925in}}%
\pgfpathlineto{\pgfqpoint{3.096427in}{4.439638in}}%
\pgfpathlineto{\pgfqpoint{3.098751in}{4.413448in}}%
\pgfpathlineto{\pgfqpoint{3.101075in}{4.460663in}}%
\pgfpathlineto{\pgfqpoint{3.103399in}{4.426384in}}%
\pgfpathlineto{\pgfqpoint{3.108047in}{4.412636in}}%
\pgfpathlineto{\pgfqpoint{3.110371in}{4.454820in}}%
\pgfpathlineto{\pgfqpoint{3.112695in}{4.437876in}}%
\pgfpathlineto{\pgfqpoint{3.115019in}{4.462712in}}%
\pgfpathlineto{\pgfqpoint{3.117343in}{4.443815in}}%
\pgfpathlineto{\pgfqpoint{3.119667in}{4.494244in}}%
\pgfpathlineto{\pgfqpoint{3.121991in}{4.411968in}}%
\pgfpathlineto{\pgfqpoint{3.126638in}{4.444485in}}%
\pgfpathlineto{\pgfqpoint{3.128962in}{4.435061in}}%
\pgfpathlineto{\pgfqpoint{3.131286in}{4.432849in}}%
\pgfpathlineto{\pgfqpoint{3.133610in}{4.481197in}}%
\pgfpathlineto{\pgfqpoint{3.135934in}{4.408022in}}%
\pgfpathlineto{\pgfqpoint{3.138258in}{4.443111in}}%
\pgfpathlineto{\pgfqpoint{3.140582in}{4.423289in}}%
\pgfpathlineto{\pgfqpoint{3.147554in}{4.468922in}}%
\pgfpathlineto{\pgfqpoint{3.149878in}{4.468942in}}%
\pgfpathlineto{\pgfqpoint{3.152202in}{4.475596in}}%
\pgfpathlineto{\pgfqpoint{3.154526in}{4.476421in}}%
\pgfpathlineto{\pgfqpoint{3.156850in}{4.420982in}}%
\pgfpathlineto{\pgfqpoint{3.159174in}{4.441392in}}%
\pgfpathlineto{\pgfqpoint{3.161498in}{4.451146in}}%
\pgfpathlineto{\pgfqpoint{3.163822in}{4.487122in}}%
\pgfpathlineto{\pgfqpoint{3.166146in}{4.452022in}}%
\pgfpathlineto{\pgfqpoint{3.168470in}{4.457802in}}%
\pgfpathlineto{\pgfqpoint{3.170794in}{4.419591in}}%
\pgfpathlineto{\pgfqpoint{3.173118in}{4.435705in}}%
\pgfpathlineto{\pgfqpoint{3.175442in}{4.434281in}}%
\pgfpathlineto{\pgfqpoint{3.177766in}{4.440561in}}%
\pgfpathlineto{\pgfqpoint{3.180090in}{4.440282in}}%
\pgfpathlineto{\pgfqpoint{3.182414in}{4.483582in}}%
\pgfpathlineto{\pgfqpoint{3.184738in}{4.498391in}}%
\pgfpathlineto{\pgfqpoint{3.187062in}{4.444883in}}%
\pgfpathlineto{\pgfqpoint{3.189386in}{4.444616in}}%
\pgfpathlineto{\pgfqpoint{3.191710in}{4.429574in}}%
\pgfpathlineto{\pgfqpoint{3.194034in}{4.452247in}}%
\pgfpathlineto{\pgfqpoint{3.196357in}{4.431685in}}%
\pgfpathlineto{\pgfqpoint{3.198681in}{4.492743in}}%
\pgfpathlineto{\pgfqpoint{3.201005in}{4.450300in}}%
\pgfpathlineto{\pgfqpoint{3.203329in}{4.496510in}}%
\pgfpathlineto{\pgfqpoint{3.207977in}{4.430586in}}%
\pgfpathlineto{\pgfqpoint{3.210301in}{4.462814in}}%
\pgfpathlineto{\pgfqpoint{3.212625in}{4.477027in}}%
\pgfpathlineto{\pgfqpoint{3.214949in}{4.424261in}}%
\pgfpathlineto{\pgfqpoint{3.221921in}{4.481964in}}%
\pgfpathlineto{\pgfqpoint{3.224245in}{4.420016in}}%
\pgfpathlineto{\pgfqpoint{3.226569in}{4.469998in}}%
\pgfpathlineto{\pgfqpoint{3.228893in}{4.450780in}}%
\pgfpathlineto{\pgfqpoint{3.231217in}{4.466267in}}%
\pgfpathlineto{\pgfqpoint{3.233541in}{4.455438in}}%
\pgfpathlineto{\pgfqpoint{3.235865in}{4.472334in}}%
\pgfpathlineto{\pgfqpoint{3.238189in}{4.467065in}}%
\pgfpathlineto{\pgfqpoint{3.240513in}{4.453974in}}%
\pgfpathlineto{\pgfqpoint{3.242837in}{4.431843in}}%
\pgfpathlineto{\pgfqpoint{3.245161in}{4.447476in}}%
\pgfpathlineto{\pgfqpoint{3.247485in}{4.473592in}}%
\pgfpathlineto{\pgfqpoint{3.252133in}{4.397949in}}%
\pgfpathlineto{\pgfqpoint{3.254457in}{4.450946in}}%
\pgfpathlineto{\pgfqpoint{3.259105in}{4.420569in}}%
\pgfpathlineto{\pgfqpoint{3.261429in}{4.465192in}}%
\pgfpathlineto{\pgfqpoint{3.263753in}{4.450710in}}%
\pgfpathlineto{\pgfqpoint{3.266076in}{4.499791in}}%
\pgfpathlineto{\pgfqpoint{3.268400in}{4.462333in}}%
\pgfpathlineto{\pgfqpoint{3.270724in}{4.457372in}}%
\pgfpathlineto{\pgfqpoint{3.273048in}{4.463703in}}%
\pgfpathlineto{\pgfqpoint{3.275372in}{4.454362in}}%
\pgfpathlineto{\pgfqpoint{3.277696in}{4.457564in}}%
\pgfpathlineto{\pgfqpoint{3.280020in}{4.476636in}}%
\pgfpathlineto{\pgfqpoint{3.282344in}{4.448798in}}%
\pgfpathlineto{\pgfqpoint{3.286992in}{4.495470in}}%
\pgfpathlineto{\pgfqpoint{3.289316in}{4.478263in}}%
\pgfpathlineto{\pgfqpoint{3.291640in}{4.431385in}}%
\pgfpathlineto{\pgfqpoint{3.293964in}{4.428725in}}%
\pgfpathlineto{\pgfqpoint{3.296288in}{4.495909in}}%
\pgfpathlineto{\pgfqpoint{3.298612in}{4.464741in}}%
\pgfpathlineto{\pgfqpoint{3.300936in}{4.476432in}}%
\pgfpathlineto{\pgfqpoint{3.303260in}{4.445414in}}%
\pgfpathlineto{\pgfqpoint{3.305584in}{4.477794in}}%
\pgfpathlineto{\pgfqpoint{3.307908in}{4.427118in}}%
\pgfpathlineto{\pgfqpoint{3.310232in}{4.467414in}}%
\pgfpathlineto{\pgfqpoint{3.314880in}{4.411658in}}%
\pgfpathlineto{\pgfqpoint{3.317204in}{4.437551in}}%
\pgfpathlineto{\pgfqpoint{3.321852in}{4.392318in}}%
\pgfpathlineto{\pgfqpoint{3.324176in}{4.451150in}}%
\pgfpathlineto{\pgfqpoint{3.326500in}{4.467214in}}%
\pgfpathlineto{\pgfqpoint{3.328824in}{4.430864in}}%
\pgfpathlineto{\pgfqpoint{3.331148in}{4.445311in}}%
\pgfpathlineto{\pgfqpoint{3.333472in}{4.491702in}}%
\pgfpathlineto{\pgfqpoint{3.335796in}{4.436199in}}%
\pgfpathlineto{\pgfqpoint{3.338119in}{4.420940in}}%
\pgfpathlineto{\pgfqpoint{3.340443in}{4.477999in}}%
\pgfpathlineto{\pgfqpoint{3.342767in}{4.411177in}}%
\pgfpathlineto{\pgfqpoint{3.345091in}{4.432010in}}%
\pgfpathlineto{\pgfqpoint{3.347415in}{4.483732in}}%
\pgfpathlineto{\pgfqpoint{3.349739in}{4.401188in}}%
\pgfpathlineto{\pgfqpoint{3.352063in}{4.451242in}}%
\pgfpathlineto{\pgfqpoint{3.356711in}{4.426282in}}%
\pgfpathlineto{\pgfqpoint{3.359035in}{4.432665in}}%
\pgfpathlineto{\pgfqpoint{3.361359in}{4.426315in}}%
\pgfpathlineto{\pgfqpoint{3.363683in}{4.405081in}}%
\pgfpathlineto{\pgfqpoint{3.366007in}{4.409698in}}%
\pgfpathlineto{\pgfqpoint{3.368331in}{4.386438in}}%
\pgfpathlineto{\pgfqpoint{3.370655in}{4.389030in}}%
\pgfpathlineto{\pgfqpoint{3.372979in}{4.456673in}}%
\pgfpathlineto{\pgfqpoint{3.377627in}{4.409846in}}%
\pgfpathlineto{\pgfqpoint{3.379951in}{4.404297in}}%
\pgfpathlineto{\pgfqpoint{3.382275in}{4.402277in}}%
\pgfpathlineto{\pgfqpoint{3.384599in}{4.427001in}}%
\pgfpathlineto{\pgfqpoint{3.386923in}{4.404995in}}%
\pgfpathlineto{\pgfqpoint{3.389247in}{4.414863in}}%
\pgfpathlineto{\pgfqpoint{3.391571in}{4.367177in}}%
\pgfpathlineto{\pgfqpoint{3.393895in}{4.422751in}}%
\pgfpathlineto{\pgfqpoint{3.396219in}{4.365941in}}%
\pgfpathlineto{\pgfqpoint{3.398543in}{4.390528in}}%
\pgfpathlineto{\pgfqpoint{3.400867in}{4.364194in}}%
\pgfpathlineto{\pgfqpoint{3.403191in}{4.423502in}}%
\pgfpathlineto{\pgfqpoint{3.407838in}{4.374105in}}%
\pgfpathlineto{\pgfqpoint{3.410162in}{4.371141in}}%
\pgfpathlineto{\pgfqpoint{3.412486in}{4.461270in}}%
\pgfpathlineto{\pgfqpoint{3.419458in}{4.401204in}}%
\pgfpathlineto{\pgfqpoint{3.421782in}{4.406357in}}%
\pgfpathlineto{\pgfqpoint{3.424106in}{4.371640in}}%
\pgfpathlineto{\pgfqpoint{3.426430in}{4.428405in}}%
\pgfpathlineto{\pgfqpoint{3.431078in}{4.352250in}}%
\pgfpathlineto{\pgfqpoint{3.433402in}{4.389061in}}%
\pgfpathlineto{\pgfqpoint{3.435726in}{4.394852in}}%
\pgfpathlineto{\pgfqpoint{3.438050in}{4.385314in}}%
\pgfpathlineto{\pgfqpoint{3.440374in}{4.398953in}}%
\pgfpathlineto{\pgfqpoint{3.442698in}{4.386829in}}%
\pgfpathlineto{\pgfqpoint{3.445022in}{4.365049in}}%
\pgfpathlineto{\pgfqpoint{3.447346in}{4.386452in}}%
\pgfpathlineto{\pgfqpoint{3.449670in}{4.360275in}}%
\pgfpathlineto{\pgfqpoint{3.451994in}{4.411286in}}%
\pgfpathlineto{\pgfqpoint{3.454318in}{4.359806in}}%
\pgfpathlineto{\pgfqpoint{3.456642in}{4.396994in}}%
\pgfpathlineto{\pgfqpoint{3.458966in}{4.415502in}}%
\pgfpathlineto{\pgfqpoint{3.461290in}{4.370706in}}%
\pgfpathlineto{\pgfqpoint{3.463614in}{4.367983in}}%
\pgfpathlineto{\pgfqpoint{3.465938in}{4.399723in}}%
\pgfpathlineto{\pgfqpoint{3.470586in}{4.363754in}}%
\pgfpathlineto{\pgfqpoint{3.472910in}{4.304524in}}%
\pgfpathlineto{\pgfqpoint{3.475234in}{4.340862in}}%
\pgfpathlineto{\pgfqpoint{3.477557in}{4.347720in}}%
\pgfpathlineto{\pgfqpoint{3.479881in}{4.328592in}}%
\pgfpathlineto{\pgfqpoint{3.482205in}{4.384481in}}%
\pgfpathlineto{\pgfqpoint{3.484529in}{4.342403in}}%
\pgfpathlineto{\pgfqpoint{3.486853in}{4.395057in}}%
\pgfpathlineto{\pgfqpoint{3.489177in}{4.302691in}}%
\pgfpathlineto{\pgfqpoint{3.491501in}{4.394414in}}%
\pgfpathlineto{\pgfqpoint{3.493825in}{4.354393in}}%
\pgfpathlineto{\pgfqpoint{3.496149in}{4.395612in}}%
\pgfpathlineto{\pgfqpoint{3.498473in}{4.291741in}}%
\pgfpathlineto{\pgfqpoint{3.500797in}{4.364160in}}%
\pgfpathlineto{\pgfqpoint{3.505445in}{4.347829in}}%
\pgfpathlineto{\pgfqpoint{3.507769in}{4.289063in}}%
\pgfpathlineto{\pgfqpoint{3.510093in}{4.277237in}}%
\pgfpathlineto{\pgfqpoint{3.512417in}{4.343460in}}%
\pgfpathlineto{\pgfqpoint{3.514741in}{4.293435in}}%
\pgfpathlineto{\pgfqpoint{3.517065in}{4.307036in}}%
\pgfpathlineto{\pgfqpoint{3.519389in}{4.277228in}}%
\pgfpathlineto{\pgfqpoint{3.521713in}{4.321597in}}%
\pgfpathlineto{\pgfqpoint{3.524037in}{4.329978in}}%
\pgfpathlineto{\pgfqpoint{3.526361in}{4.308685in}}%
\pgfpathlineto{\pgfqpoint{3.528685in}{4.339803in}}%
\pgfpathlineto{\pgfqpoint{3.531009in}{4.316708in}}%
\pgfpathlineto{\pgfqpoint{3.533333in}{4.349196in}}%
\pgfpathlineto{\pgfqpoint{3.535657in}{4.308344in}}%
\pgfpathlineto{\pgfqpoint{3.540305in}{4.341485in}}%
\pgfpathlineto{\pgfqpoint{3.542629in}{4.291061in}}%
\pgfpathlineto{\pgfqpoint{3.544953in}{4.284548in}}%
\pgfpathlineto{\pgfqpoint{3.547277in}{4.307844in}}%
\pgfpathlineto{\pgfqpoint{3.549600in}{4.261500in}}%
\pgfpathlineto{\pgfqpoint{3.554248in}{4.287583in}}%
\pgfpathlineto{\pgfqpoint{3.556572in}{4.272314in}}%
\pgfpathlineto{\pgfqpoint{3.558896in}{4.283767in}}%
\pgfpathlineto{\pgfqpoint{3.561220in}{4.335596in}}%
\pgfpathlineto{\pgfqpoint{3.563544in}{4.258518in}}%
\pgfpathlineto{\pgfqpoint{3.565868in}{4.293521in}}%
\pgfpathlineto{\pgfqpoint{3.568192in}{4.289331in}}%
\pgfpathlineto{\pgfqpoint{3.570516in}{4.311187in}}%
\pgfpathlineto{\pgfqpoint{3.572840in}{4.260151in}}%
\pgfpathlineto{\pgfqpoint{3.575164in}{4.281120in}}%
\pgfpathlineto{\pgfqpoint{3.577488in}{4.262293in}}%
\pgfpathlineto{\pgfqpoint{3.579812in}{4.293721in}}%
\pgfpathlineto{\pgfqpoint{3.582136in}{4.297487in}}%
\pgfpathlineto{\pgfqpoint{3.584460in}{4.275329in}}%
\pgfpathlineto{\pgfqpoint{3.586784in}{4.309981in}}%
\pgfpathlineto{\pgfqpoint{3.591432in}{4.212031in}}%
\pgfpathlineto{\pgfqpoint{3.598404in}{4.307693in}}%
\pgfpathlineto{\pgfqpoint{3.600728in}{4.208204in}}%
\pgfpathlineto{\pgfqpoint{3.603052in}{4.259908in}}%
\pgfpathlineto{\pgfqpoint{3.605376in}{4.236219in}}%
\pgfpathlineto{\pgfqpoint{3.607700in}{4.285324in}}%
\pgfpathlineto{\pgfqpoint{3.610024in}{4.253426in}}%
\pgfpathlineto{\pgfqpoint{3.612348in}{4.339564in}}%
\pgfpathlineto{\pgfqpoint{3.614672in}{4.232752in}}%
\pgfpathlineto{\pgfqpoint{3.616996in}{4.203293in}}%
\pgfpathlineto{\pgfqpoint{3.621643in}{4.290811in}}%
\pgfpathlineto{\pgfqpoint{3.623967in}{4.150723in}}%
\pgfpathlineto{\pgfqpoint{3.626291in}{4.231916in}}%
\pgfpathlineto{\pgfqpoint{3.628615in}{4.224750in}}%
\pgfpathlineto{\pgfqpoint{3.630939in}{4.255896in}}%
\pgfpathlineto{\pgfqpoint{3.633263in}{4.220656in}}%
\pgfpathlineto{\pgfqpoint{3.635587in}{4.228484in}}%
\pgfpathlineto{\pgfqpoint{3.637911in}{4.232334in}}%
\pgfpathlineto{\pgfqpoint{3.640235in}{4.217666in}}%
\pgfpathlineto{\pgfqpoint{3.642559in}{4.181638in}}%
\pgfpathlineto{\pgfqpoint{3.647207in}{4.222358in}}%
\pgfpathlineto{\pgfqpoint{3.651855in}{4.210606in}}%
\pgfpathlineto{\pgfqpoint{3.654179in}{4.245422in}}%
\pgfpathlineto{\pgfqpoint{3.656503in}{4.179628in}}%
\pgfpathlineto{\pgfqpoint{3.661151in}{4.194953in}}%
\pgfpathlineto{\pgfqpoint{3.665799in}{4.226063in}}%
\pgfpathlineto{\pgfqpoint{3.668123in}{4.150111in}}%
\pgfpathlineto{\pgfqpoint{3.675095in}{4.224587in}}%
\pgfpathlineto{\pgfqpoint{3.677419in}{4.201167in}}%
\pgfpathlineto{\pgfqpoint{3.679743in}{4.231602in}}%
\pgfpathlineto{\pgfqpoint{3.682067in}{4.139583in}}%
\pgfpathlineto{\pgfqpoint{3.684391in}{4.219366in}}%
\pgfpathlineto{\pgfqpoint{3.686715in}{4.209793in}}%
\pgfpathlineto{\pgfqpoint{3.691362in}{4.151799in}}%
\pgfpathlineto{\pgfqpoint{3.693686in}{4.173629in}}%
\pgfpathlineto{\pgfqpoint{3.696010in}{4.170103in}}%
\pgfpathlineto{\pgfqpoint{3.698334in}{4.148044in}}%
\pgfpathlineto{\pgfqpoint{3.700658in}{4.205293in}}%
\pgfpathlineto{\pgfqpoint{3.702982in}{4.131027in}}%
\pgfpathlineto{\pgfqpoint{3.705306in}{4.117753in}}%
\pgfpathlineto{\pgfqpoint{3.709954in}{4.158737in}}%
\pgfpathlineto{\pgfqpoint{3.712278in}{4.170212in}}%
\pgfpathlineto{\pgfqpoint{3.714602in}{4.165951in}}%
\pgfpathlineto{\pgfqpoint{3.716926in}{4.098189in}}%
\pgfpathlineto{\pgfqpoint{3.719250in}{4.168083in}}%
\pgfpathlineto{\pgfqpoint{3.721574in}{4.118566in}}%
\pgfpathlineto{\pgfqpoint{3.723898in}{4.118080in}}%
\pgfpathlineto{\pgfqpoint{3.726222in}{4.173921in}}%
\pgfpathlineto{\pgfqpoint{3.728546in}{4.082740in}}%
\pgfpathlineto{\pgfqpoint{3.730870in}{4.147188in}}%
\pgfpathlineto{\pgfqpoint{3.733194in}{4.112542in}}%
\pgfpathlineto{\pgfqpoint{3.735518in}{4.172819in}}%
\pgfpathlineto{\pgfqpoint{3.737842in}{4.108945in}}%
\pgfpathlineto{\pgfqpoint{3.742490in}{4.179261in}}%
\pgfpathlineto{\pgfqpoint{3.744814in}{4.115768in}}%
\pgfpathlineto{\pgfqpoint{3.747138in}{4.159828in}}%
\pgfpathlineto{\pgfqpoint{3.749462in}{4.133177in}}%
\pgfpathlineto{\pgfqpoint{3.751786in}{4.146949in}}%
\pgfpathlineto{\pgfqpoint{3.754110in}{4.064186in}}%
\pgfpathlineto{\pgfqpoint{3.756434in}{4.077926in}}%
\pgfpathlineto{\pgfqpoint{3.758757in}{4.140108in}}%
\pgfpathlineto{\pgfqpoint{3.761081in}{4.151186in}}%
\pgfpathlineto{\pgfqpoint{3.763405in}{4.104280in}}%
\pgfpathlineto{\pgfqpoint{3.770377in}{4.079773in}}%
\pgfpathlineto{\pgfqpoint{3.772701in}{4.084884in}}%
\pgfpathlineto{\pgfqpoint{3.775025in}{4.094774in}}%
\pgfpathlineto{\pgfqpoint{3.779673in}{4.029165in}}%
\pgfpathlineto{\pgfqpoint{3.784321in}{4.104597in}}%
\pgfpathlineto{\pgfqpoint{3.786645in}{4.060534in}}%
\pgfpathlineto{\pgfqpoint{3.788969in}{4.097989in}}%
\pgfpathlineto{\pgfqpoint{3.791293in}{4.040220in}}%
\pgfpathlineto{\pgfqpoint{3.793617in}{4.117453in}}%
\pgfpathlineto{\pgfqpoint{3.795941in}{4.076029in}}%
\pgfpathlineto{\pgfqpoint{3.798265in}{4.059494in}}%
\pgfpathlineto{\pgfqpoint{3.800589in}{4.141231in}}%
\pgfpathlineto{\pgfqpoint{3.802913in}{4.091744in}}%
\pgfpathlineto{\pgfqpoint{3.805237in}{4.083088in}}%
\pgfpathlineto{\pgfqpoint{3.807561in}{4.069875in}}%
\pgfpathlineto{\pgfqpoint{3.809885in}{4.079203in}}%
\pgfpathlineto{\pgfqpoint{3.812209in}{4.064459in}}%
\pgfpathlineto{\pgfqpoint{3.814533in}{4.019256in}}%
\pgfpathlineto{\pgfqpoint{3.819181in}{4.120111in}}%
\pgfpathlineto{\pgfqpoint{3.821505in}{4.093068in}}%
\pgfpathlineto{\pgfqpoint{3.823829in}{4.021421in}}%
\pgfpathlineto{\pgfqpoint{3.826153in}{4.034711in}}%
\pgfpathlineto{\pgfqpoint{3.830800in}{4.015751in}}%
\pgfpathlineto{\pgfqpoint{3.833124in}{4.073316in}}%
\pgfpathlineto{\pgfqpoint{3.835448in}{3.979998in}}%
\pgfpathlineto{\pgfqpoint{3.837772in}{4.067436in}}%
\pgfpathlineto{\pgfqpoint{3.840096in}{4.055625in}}%
\pgfpathlineto{\pgfqpoint{3.842420in}{4.000226in}}%
\pgfpathlineto{\pgfqpoint{3.844744in}{3.994301in}}%
\pgfpathlineto{\pgfqpoint{3.847068in}{4.018444in}}%
\pgfpathlineto{\pgfqpoint{3.849392in}{4.055569in}}%
\pgfpathlineto{\pgfqpoint{3.854040in}{4.013213in}}%
\pgfpathlineto{\pgfqpoint{3.856364in}{4.014566in}}%
\pgfpathlineto{\pgfqpoint{3.858688in}{3.992657in}}%
\pgfpathlineto{\pgfqpoint{3.861012in}{4.043837in}}%
\pgfpathlineto{\pgfqpoint{3.863336in}{4.005471in}}%
\pgfpathlineto{\pgfqpoint{3.865660in}{4.040104in}}%
\pgfpathlineto{\pgfqpoint{3.867984in}{4.032767in}}%
\pgfpathlineto{\pgfqpoint{3.870308in}{4.032731in}}%
\pgfpathlineto{\pgfqpoint{3.872632in}{4.011364in}}%
\pgfpathlineto{\pgfqpoint{3.874956in}{4.022217in}}%
\pgfpathlineto{\pgfqpoint{3.877280in}{4.052135in}}%
\pgfpathlineto{\pgfqpoint{3.881928in}{3.993179in}}%
\pgfpathlineto{\pgfqpoint{3.884252in}{3.965922in}}%
\pgfpathlineto{\pgfqpoint{3.886576in}{3.995451in}}%
\pgfpathlineto{\pgfqpoint{3.888900in}{3.975522in}}%
\pgfpathlineto{\pgfqpoint{3.891224in}{3.984826in}}%
\pgfpathlineto{\pgfqpoint{3.893548in}{4.026722in}}%
\pgfpathlineto{\pgfqpoint{3.898196in}{3.959587in}}%
\pgfpathlineto{\pgfqpoint{3.900519in}{3.966063in}}%
\pgfpathlineto{\pgfqpoint{3.902843in}{4.030754in}}%
\pgfpathlineto{\pgfqpoint{3.905167in}{3.953948in}}%
\pgfpathlineto{\pgfqpoint{3.907491in}{4.008900in}}%
\pgfpathlineto{\pgfqpoint{3.909815in}{3.947570in}}%
\pgfpathlineto{\pgfqpoint{3.912139in}{3.978547in}}%
\pgfpathlineto{\pgfqpoint{3.914463in}{3.991754in}}%
\pgfpathlineto{\pgfqpoint{3.916787in}{4.018428in}}%
\pgfpathlineto{\pgfqpoint{3.921435in}{3.976675in}}%
\pgfpathlineto{\pgfqpoint{3.923759in}{3.992076in}}%
\pgfpathlineto{\pgfqpoint{3.926083in}{3.998041in}}%
\pgfpathlineto{\pgfqpoint{3.928407in}{3.958896in}}%
\pgfpathlineto{\pgfqpoint{3.930731in}{3.945567in}}%
\pgfpathlineto{\pgfqpoint{3.933055in}{3.995599in}}%
\pgfpathlineto{\pgfqpoint{3.935379in}{3.942920in}}%
\pgfpathlineto{\pgfqpoint{3.937703in}{3.928651in}}%
\pgfpathlineto{\pgfqpoint{3.940027in}{3.984505in}}%
\pgfpathlineto{\pgfqpoint{3.942351in}{3.940092in}}%
\pgfpathlineto{\pgfqpoint{3.944675in}{3.921627in}}%
\pgfpathlineto{\pgfqpoint{3.946999in}{3.960550in}}%
\pgfpathlineto{\pgfqpoint{3.949323in}{3.978780in}}%
\pgfpathlineto{\pgfqpoint{3.951647in}{3.946771in}}%
\pgfpathlineto{\pgfqpoint{3.953971in}{3.967378in}}%
\pgfpathlineto{\pgfqpoint{3.956295in}{3.933857in}}%
\pgfpathlineto{\pgfqpoint{3.958619in}{3.955606in}}%
\pgfpathlineto{\pgfqpoint{3.963267in}{3.932272in}}%
\pgfpathlineto{\pgfqpoint{3.965591in}{3.908916in}}%
\pgfpathlineto{\pgfqpoint{3.967915in}{3.969931in}}%
\pgfpathlineto{\pgfqpoint{3.970238in}{3.969201in}}%
\pgfpathlineto{\pgfqpoint{3.972562in}{3.892282in}}%
\pgfpathlineto{\pgfqpoint{3.974886in}{3.900810in}}%
\pgfpathlineto{\pgfqpoint{3.977210in}{3.916924in}}%
\pgfpathlineto{\pgfqpoint{3.979534in}{3.968551in}}%
\pgfpathlineto{\pgfqpoint{3.981858in}{3.968216in}}%
\pgfpathlineto{\pgfqpoint{3.984182in}{3.900454in}}%
\pgfpathlineto{\pgfqpoint{3.986506in}{3.918935in}}%
\pgfpathlineto{\pgfqpoint{3.988830in}{3.961483in}}%
\pgfpathlineto{\pgfqpoint{3.993478in}{3.922423in}}%
\pgfpathlineto{\pgfqpoint{3.995802in}{3.956120in}}%
\pgfpathlineto{\pgfqpoint{3.998126in}{3.917739in}}%
\pgfpathlineto{\pgfqpoint{4.000450in}{3.906339in}}%
\pgfpathlineto{\pgfqpoint{4.002774in}{3.916533in}}%
\pgfpathlineto{\pgfqpoint{4.005098in}{3.893246in}}%
\pgfpathlineto{\pgfqpoint{4.007422in}{3.915729in}}%
\pgfpathlineto{\pgfqpoint{4.009746in}{3.903948in}}%
\pgfpathlineto{\pgfqpoint{4.012070in}{3.943166in}}%
\pgfpathlineto{\pgfqpoint{4.014394in}{3.952711in}}%
\pgfpathlineto{\pgfqpoint{4.016718in}{3.967540in}}%
\pgfpathlineto{\pgfqpoint{4.019042in}{3.941008in}}%
\pgfpathlineto{\pgfqpoint{4.021366in}{3.892514in}}%
\pgfpathlineto{\pgfqpoint{4.023690in}{3.952953in}}%
\pgfpathlineto{\pgfqpoint{4.028338in}{3.977677in}}%
\pgfpathlineto{\pgfqpoint{4.030662in}{3.894744in}}%
\pgfpathlineto{\pgfqpoint{4.032986in}{3.881658in}}%
\pgfpathlineto{\pgfqpoint{4.035310in}{3.900131in}}%
\pgfpathlineto{\pgfqpoint{4.037634in}{3.970031in}}%
\pgfpathlineto{\pgfqpoint{4.039957in}{3.897591in}}%
\pgfpathlineto{\pgfqpoint{4.042281in}{3.926878in}}%
\pgfpathlineto{\pgfqpoint{4.044605in}{3.938257in}}%
\pgfpathlineto{\pgfqpoint{4.046929in}{3.942031in}}%
\pgfpathlineto{\pgfqpoint{4.049253in}{3.898837in}}%
\pgfpathlineto{\pgfqpoint{4.051577in}{3.886570in}}%
\pgfpathlineto{\pgfqpoint{4.053901in}{3.886023in}}%
\pgfpathlineto{\pgfqpoint{4.056225in}{3.864630in}}%
\pgfpathlineto{\pgfqpoint{4.058549in}{3.920115in}}%
\pgfpathlineto{\pgfqpoint{4.060873in}{3.929698in}}%
\pgfpathlineto{\pgfqpoint{4.063197in}{3.890072in}}%
\pgfpathlineto{\pgfqpoint{4.065521in}{3.943103in}}%
\pgfpathlineto{\pgfqpoint{4.067845in}{3.909989in}}%
\pgfpathlineto{\pgfqpoint{4.070169in}{3.935545in}}%
\pgfpathlineto{\pgfqpoint{4.074817in}{3.864777in}}%
\pgfpathlineto{\pgfqpoint{4.077141in}{3.943919in}}%
\pgfpathlineto{\pgfqpoint{4.079465in}{3.862546in}}%
\pgfpathlineto{\pgfqpoint{4.084113in}{3.930038in}}%
\pgfpathlineto{\pgfqpoint{4.086437in}{3.911320in}}%
\pgfpathlineto{\pgfqpoint{4.088761in}{3.947651in}}%
\pgfpathlineto{\pgfqpoint{4.091085in}{3.930768in}}%
\pgfpathlineto{\pgfqpoint{4.098057in}{3.865680in}}%
\pgfpathlineto{\pgfqpoint{4.100381in}{3.883908in}}%
\pgfpathlineto{\pgfqpoint{4.102705in}{3.885787in}}%
\pgfpathlineto{\pgfqpoint{4.105029in}{3.839276in}}%
\pgfpathlineto{\pgfqpoint{4.107353in}{3.870732in}}%
\pgfpathlineto{\pgfqpoint{4.109677in}{3.869897in}}%
\pgfpathlineto{\pgfqpoint{4.114324in}{3.975233in}}%
\pgfpathlineto{\pgfqpoint{4.116648in}{3.900431in}}%
\pgfpathlineto{\pgfqpoint{4.118972in}{3.904517in}}%
\pgfpathlineto{\pgfqpoint{4.121296in}{3.883875in}}%
\pgfpathlineto{\pgfqpoint{4.123620in}{3.964067in}}%
\pgfpathlineto{\pgfqpoint{4.125944in}{3.909793in}}%
\pgfpathlineto{\pgfqpoint{4.128268in}{3.958302in}}%
\pgfpathlineto{\pgfqpoint{4.130592in}{3.859853in}}%
\pgfpathlineto{\pgfqpoint{4.135240in}{3.937175in}}%
\pgfpathlineto{\pgfqpoint{4.137564in}{3.923112in}}%
\pgfpathlineto{\pgfqpoint{4.139888in}{3.890173in}}%
\pgfpathlineto{\pgfqpoint{4.142212in}{3.928750in}}%
\pgfpathlineto{\pgfqpoint{4.144536in}{3.931116in}}%
\pgfpathlineto{\pgfqpoint{4.146860in}{3.901289in}}%
\pgfpathlineto{\pgfqpoint{4.149184in}{3.926039in}}%
\pgfpathlineto{\pgfqpoint{4.151508in}{3.897583in}}%
\pgfpathlineto{\pgfqpoint{4.153832in}{3.932793in}}%
\pgfpathlineto{\pgfqpoint{4.156156in}{3.911744in}}%
\pgfpathlineto{\pgfqpoint{4.158480in}{3.857789in}}%
\pgfpathlineto{\pgfqpoint{4.160804in}{3.875869in}}%
\pgfpathlineto{\pgfqpoint{4.165452in}{3.926825in}}%
\pgfpathlineto{\pgfqpoint{4.167776in}{3.886402in}}%
\pgfpathlineto{\pgfqpoint{4.170100in}{3.949307in}}%
\pgfpathlineto{\pgfqpoint{4.172424in}{3.941290in}}%
\pgfpathlineto{\pgfqpoint{4.174748in}{3.880889in}}%
\pgfpathlineto{\pgfqpoint{4.177072in}{3.930148in}}%
\pgfpathlineto{\pgfqpoint{4.179396in}{3.863022in}}%
\pgfpathlineto{\pgfqpoint{4.181719in}{3.927462in}}%
\pgfpathlineto{\pgfqpoint{4.184043in}{3.916447in}}%
\pgfpathlineto{\pgfqpoint{4.186367in}{3.895475in}}%
\pgfpathlineto{\pgfqpoint{4.188691in}{3.920632in}}%
\pgfpathlineto{\pgfqpoint{4.191015in}{3.893322in}}%
\pgfpathlineto{\pgfqpoint{4.193339in}{3.918261in}}%
\pgfpathlineto{\pgfqpoint{4.195663in}{3.912825in}}%
\pgfpathlineto{\pgfqpoint{4.197987in}{3.871819in}}%
\pgfpathlineto{\pgfqpoint{4.200311in}{3.919271in}}%
\pgfpathlineto{\pgfqpoint{4.202635in}{3.847133in}}%
\pgfpathlineto{\pgfqpoint{4.204959in}{3.900352in}}%
\pgfpathlineto{\pgfqpoint{4.207283in}{3.855819in}}%
\pgfpathlineto{\pgfqpoint{4.209607in}{3.952901in}}%
\pgfpathlineto{\pgfqpoint{4.211931in}{3.898777in}}%
\pgfpathlineto{\pgfqpoint{4.214255in}{3.919496in}}%
\pgfpathlineto{\pgfqpoint{4.216579in}{3.913859in}}%
\pgfpathlineto{\pgfqpoint{4.218903in}{3.940818in}}%
\pgfpathlineto{\pgfqpoint{4.223551in}{3.903420in}}%
\pgfpathlineto{\pgfqpoint{4.225875in}{3.923204in}}%
\pgfpathlineto{\pgfqpoint{4.228199in}{3.888965in}}%
\pgfpathlineto{\pgfqpoint{4.230523in}{3.959805in}}%
\pgfpathlineto{\pgfqpoint{4.232847in}{3.867520in}}%
\pgfpathlineto{\pgfqpoint{4.235171in}{3.926367in}}%
\pgfpathlineto{\pgfqpoint{4.237495in}{3.932145in}}%
\pgfpathlineto{\pgfqpoint{4.239819in}{3.931303in}}%
\pgfpathlineto{\pgfqpoint{4.242143in}{3.898201in}}%
\pgfpathlineto{\pgfqpoint{4.244467in}{3.921332in}}%
\pgfpathlineto{\pgfqpoint{4.246791in}{3.906673in}}%
\pgfpathlineto{\pgfqpoint{4.249115in}{3.903670in}}%
\pgfpathlineto{\pgfqpoint{4.251438in}{3.946465in}}%
\pgfpathlineto{\pgfqpoint{4.253762in}{3.915982in}}%
\pgfpathlineto{\pgfqpoint{4.256086in}{3.936899in}}%
\pgfpathlineto{\pgfqpoint{4.258410in}{3.969521in}}%
\pgfpathlineto{\pgfqpoint{4.260734in}{3.916712in}}%
\pgfpathlineto{\pgfqpoint{4.263058in}{3.918491in}}%
\pgfpathlineto{\pgfqpoint{4.265382in}{3.926388in}}%
\pgfpathlineto{\pgfqpoint{4.267706in}{3.926783in}}%
\pgfpathlineto{\pgfqpoint{4.270030in}{3.963115in}}%
\pgfpathlineto{\pgfqpoint{4.272354in}{3.877100in}}%
\pgfpathlineto{\pgfqpoint{4.274678in}{3.941232in}}%
\pgfpathlineto{\pgfqpoint{4.277002in}{3.935483in}}%
\pgfpathlineto{\pgfqpoint{4.279326in}{3.967723in}}%
\pgfpathlineto{\pgfqpoint{4.281650in}{3.937169in}}%
\pgfpathlineto{\pgfqpoint{4.283974in}{3.958296in}}%
\pgfpathlineto{\pgfqpoint{4.286298in}{3.926840in}}%
\pgfpathlineto{\pgfqpoint{4.288622in}{3.963346in}}%
\pgfpathlineto{\pgfqpoint{4.290946in}{3.921996in}}%
\pgfpathlineto{\pgfqpoint{4.293270in}{3.984427in}}%
\pgfpathlineto{\pgfqpoint{4.295594in}{3.936030in}}%
\pgfpathlineto{\pgfqpoint{4.297918in}{3.934024in}}%
\pgfpathlineto{\pgfqpoint{4.300242in}{3.978878in}}%
\pgfpathlineto{\pgfqpoint{4.302566in}{3.959849in}}%
\pgfpathlineto{\pgfqpoint{4.304890in}{3.952346in}}%
\pgfpathlineto{\pgfqpoint{4.307214in}{3.996898in}}%
\pgfpathlineto{\pgfqpoint{4.309538in}{3.931617in}}%
\pgfpathlineto{\pgfqpoint{4.311862in}{3.955429in}}%
\pgfpathlineto{\pgfqpoint{4.314186in}{3.951773in}}%
\pgfpathlineto{\pgfqpoint{4.316510in}{3.912286in}}%
\pgfpathlineto{\pgfqpoint{4.318834in}{3.988964in}}%
\pgfpathlineto{\pgfqpoint{4.321158in}{4.023957in}}%
\pgfpathlineto{\pgfqpoint{4.323481in}{4.025560in}}%
\pgfpathlineto{\pgfqpoint{4.325805in}{3.932991in}}%
\pgfpathlineto{\pgfqpoint{4.328129in}{3.980265in}}%
\pgfpathlineto{\pgfqpoint{4.330453in}{3.929719in}}%
\pgfpathlineto{\pgfqpoint{4.332777in}{3.934810in}}%
\pgfpathlineto{\pgfqpoint{4.335101in}{3.983680in}}%
\pgfpathlineto{\pgfqpoint{4.337425in}{3.986042in}}%
\pgfpathlineto{\pgfqpoint{4.339749in}{3.950439in}}%
\pgfpathlineto{\pgfqpoint{4.342073in}{3.976681in}}%
\pgfpathlineto{\pgfqpoint{4.344397in}{4.017544in}}%
\pgfpathlineto{\pgfqpoint{4.346721in}{3.970924in}}%
\pgfpathlineto{\pgfqpoint{4.349045in}{3.997701in}}%
\pgfpathlineto{\pgfqpoint{4.351369in}{3.970436in}}%
\pgfpathlineto{\pgfqpoint{4.353693in}{3.958887in}}%
\pgfpathlineto{\pgfqpoint{4.356017in}{3.998960in}}%
\pgfpathlineto{\pgfqpoint{4.358341in}{4.001752in}}%
\pgfpathlineto{\pgfqpoint{4.360665in}{3.955874in}}%
\pgfpathlineto{\pgfqpoint{4.362989in}{4.023784in}}%
\pgfpathlineto{\pgfqpoint{4.365313in}{4.036387in}}%
\pgfpathlineto{\pgfqpoint{4.367637in}{3.978486in}}%
\pgfpathlineto{\pgfqpoint{4.369961in}{3.984246in}}%
\pgfpathlineto{\pgfqpoint{4.372285in}{4.033944in}}%
\pgfpathlineto{\pgfqpoint{4.374609in}{4.014006in}}%
\pgfpathlineto{\pgfqpoint{4.376933in}{4.044380in}}%
\pgfpathlineto{\pgfqpoint{4.379257in}{4.053019in}}%
\pgfpathlineto{\pgfqpoint{4.381581in}{3.973573in}}%
\pgfpathlineto{\pgfqpoint{4.383905in}{4.082231in}}%
\pgfpathlineto{\pgfqpoint{4.386229in}{3.973736in}}%
\pgfpathlineto{\pgfqpoint{4.390877in}{4.034385in}}%
\pgfpathlineto{\pgfqpoint{4.393200in}{3.996819in}}%
\pgfpathlineto{\pgfqpoint{4.395524in}{4.031096in}}%
\pgfpathlineto{\pgfqpoint{4.397848in}{4.020479in}}%
\pgfpathlineto{\pgfqpoint{4.400172in}{3.978584in}}%
\pgfpathlineto{\pgfqpoint{4.402496in}{4.009112in}}%
\pgfpathlineto{\pgfqpoint{4.404820in}{3.952362in}}%
\pgfpathlineto{\pgfqpoint{4.407144in}{4.018617in}}%
\pgfpathlineto{\pgfqpoint{4.409468in}{4.037494in}}%
\pgfpathlineto{\pgfqpoint{4.411792in}{4.015688in}}%
\pgfpathlineto{\pgfqpoint{4.414116in}{4.008676in}}%
\pgfpathlineto{\pgfqpoint{4.416440in}{3.978659in}}%
\pgfpathlineto{\pgfqpoint{4.418764in}{4.030011in}}%
\pgfpathlineto{\pgfqpoint{4.421088in}{4.028568in}}%
\pgfpathlineto{\pgfqpoint{4.423412in}{4.063842in}}%
\pgfpathlineto{\pgfqpoint{4.425736in}{4.030513in}}%
\pgfpathlineto{\pgfqpoint{4.428060in}{4.064673in}}%
\pgfpathlineto{\pgfqpoint{4.430384in}{4.032861in}}%
\pgfpathlineto{\pgfqpoint{4.432708in}{4.022144in}}%
\pgfpathlineto{\pgfqpoint{4.435032in}{4.045344in}}%
\pgfpathlineto{\pgfqpoint{4.439680in}{3.999407in}}%
\pgfpathlineto{\pgfqpoint{4.442004in}{4.050502in}}%
\pgfpathlineto{\pgfqpoint{4.444328in}{4.019306in}}%
\pgfpathlineto{\pgfqpoint{4.446652in}{4.090984in}}%
\pgfpathlineto{\pgfqpoint{4.448976in}{4.058998in}}%
\pgfpathlineto{\pgfqpoint{4.451300in}{4.087674in}}%
\pgfpathlineto{\pgfqpoint{4.453624in}{4.089685in}}%
\pgfpathlineto{\pgfqpoint{4.455948in}{4.086385in}}%
\pgfpathlineto{\pgfqpoint{4.458272in}{4.061924in}}%
\pgfpathlineto{\pgfqpoint{4.460596in}{4.060041in}}%
\pgfpathlineto{\pgfqpoint{4.462919in}{4.107096in}}%
\pgfpathlineto{\pgfqpoint{4.465243in}{4.052587in}}%
\pgfpathlineto{\pgfqpoint{4.467567in}{4.035916in}}%
\pgfpathlineto{\pgfqpoint{4.472215in}{4.105424in}}%
\pgfpathlineto{\pgfqpoint{4.474539in}{4.030449in}}%
\pgfpathlineto{\pgfqpoint{4.479187in}{4.073668in}}%
\pgfpathlineto{\pgfqpoint{4.481511in}{4.051057in}}%
\pgfpathlineto{\pgfqpoint{4.483835in}{4.095284in}}%
\pgfpathlineto{\pgfqpoint{4.486159in}{4.054592in}}%
\pgfpathlineto{\pgfqpoint{4.488483in}{4.057107in}}%
\pgfpathlineto{\pgfqpoint{4.490807in}{4.102816in}}%
\pgfpathlineto{\pgfqpoint{4.493131in}{4.096069in}}%
\pgfpathlineto{\pgfqpoint{4.495455in}{4.069205in}}%
\pgfpathlineto{\pgfqpoint{4.497779in}{4.068828in}}%
\pgfpathlineto{\pgfqpoint{4.500103in}{4.133981in}}%
\pgfpathlineto{\pgfqpoint{4.502427in}{4.051213in}}%
\pgfpathlineto{\pgfqpoint{4.504751in}{4.089750in}}%
\pgfpathlineto{\pgfqpoint{4.507075in}{4.096173in}}%
\pgfpathlineto{\pgfqpoint{4.509399in}{4.075595in}}%
\pgfpathlineto{\pgfqpoint{4.511723in}{4.143835in}}%
\pgfpathlineto{\pgfqpoint{4.514047in}{4.155350in}}%
\pgfpathlineto{\pgfqpoint{4.518695in}{4.082541in}}%
\pgfpathlineto{\pgfqpoint{4.523343in}{4.170586in}}%
\pgfpathlineto{\pgfqpoint{4.525667in}{4.102117in}}%
\pgfpathlineto{\pgfqpoint{4.527991in}{4.109646in}}%
\pgfpathlineto{\pgfqpoint{4.532638in}{4.149435in}}%
\pgfpathlineto{\pgfqpoint{4.534962in}{4.159234in}}%
\pgfpathlineto{\pgfqpoint{4.539610in}{4.082179in}}%
\pgfpathlineto{\pgfqpoint{4.541934in}{4.095205in}}%
\pgfpathlineto{\pgfqpoint{4.544258in}{4.161807in}}%
\pgfpathlineto{\pgfqpoint{4.546582in}{4.192752in}}%
\pgfpathlineto{\pgfqpoint{4.548906in}{4.168878in}}%
\pgfpathlineto{\pgfqpoint{4.551230in}{4.169930in}}%
\pgfpathlineto{\pgfqpoint{4.553554in}{4.127045in}}%
\pgfpathlineto{\pgfqpoint{4.555878in}{4.106524in}}%
\pgfpathlineto{\pgfqpoint{4.560526in}{4.169106in}}%
\pgfpathlineto{\pgfqpoint{4.562850in}{4.103247in}}%
\pgfpathlineto{\pgfqpoint{4.565174in}{4.168987in}}%
\pgfpathlineto{\pgfqpoint{4.567498in}{4.169679in}}%
\pgfpathlineto{\pgfqpoint{4.569822in}{4.140054in}}%
\pgfpathlineto{\pgfqpoint{4.572146in}{4.167140in}}%
\pgfpathlineto{\pgfqpoint{4.574470in}{4.156950in}}%
\pgfpathlineto{\pgfqpoint{4.576794in}{4.164209in}}%
\pgfpathlineto{\pgfqpoint{4.581442in}{4.198839in}}%
\pgfpathlineto{\pgfqpoint{4.583766in}{4.172037in}}%
\pgfpathlineto{\pgfqpoint{4.586090in}{4.201396in}}%
\pgfpathlineto{\pgfqpoint{4.588414in}{4.175598in}}%
\pgfpathlineto{\pgfqpoint{4.590738in}{4.123570in}}%
\pgfpathlineto{\pgfqpoint{4.593062in}{4.130712in}}%
\pgfpathlineto{\pgfqpoint{4.595386in}{4.201190in}}%
\pgfpathlineto{\pgfqpoint{4.600034in}{4.170208in}}%
\pgfpathlineto{\pgfqpoint{4.602358in}{4.266715in}}%
\pgfpathlineto{\pgfqpoint{4.604681in}{4.232531in}}%
\pgfpathlineto{\pgfqpoint{4.607005in}{4.253342in}}%
\pgfpathlineto{\pgfqpoint{4.609329in}{4.153316in}}%
\pgfpathlineto{\pgfqpoint{4.611653in}{4.148343in}}%
\pgfpathlineto{\pgfqpoint{4.616301in}{4.204669in}}%
\pgfpathlineto{\pgfqpoint{4.618625in}{4.267546in}}%
\pgfpathlineto{\pgfqpoint{4.620949in}{4.203275in}}%
\pgfpathlineto{\pgfqpoint{4.623273in}{4.258694in}}%
\pgfpathlineto{\pgfqpoint{4.625597in}{4.219222in}}%
\pgfpathlineto{\pgfqpoint{4.627921in}{4.211303in}}%
\pgfpathlineto{\pgfqpoint{4.630245in}{4.247832in}}%
\pgfpathlineto{\pgfqpoint{4.632569in}{4.171541in}}%
\pgfpathlineto{\pgfqpoint{4.634893in}{4.234648in}}%
\pgfpathlineto{\pgfqpoint{4.637217in}{4.197674in}}%
\pgfpathlineto{\pgfqpoint{4.639541in}{4.208347in}}%
\pgfpathlineto{\pgfqpoint{4.644189in}{4.263288in}}%
\pgfpathlineto{\pgfqpoint{4.646513in}{4.247651in}}%
\pgfpathlineto{\pgfqpoint{4.648837in}{4.253451in}}%
\pgfpathlineto{\pgfqpoint{4.653485in}{4.200694in}}%
\pgfpathlineto{\pgfqpoint{4.655809in}{4.251209in}}%
\pgfpathlineto{\pgfqpoint{4.658133in}{4.225708in}}%
\pgfpathlineto{\pgfqpoint{4.660457in}{4.259358in}}%
\pgfpathlineto{\pgfqpoint{4.662781in}{4.263637in}}%
\pgfpathlineto{\pgfqpoint{4.665105in}{4.254629in}}%
\pgfpathlineto{\pgfqpoint{4.667429in}{4.253300in}}%
\pgfpathlineto{\pgfqpoint{4.669753in}{4.282591in}}%
\pgfpathlineto{\pgfqpoint{4.672077in}{4.272034in}}%
\pgfpathlineto{\pgfqpoint{4.674400in}{4.254430in}}%
\pgfpathlineto{\pgfqpoint{4.676724in}{4.262407in}}%
\pgfpathlineto{\pgfqpoint{4.679048in}{4.308236in}}%
\pgfpathlineto{\pgfqpoint{4.681372in}{4.255304in}}%
\pgfpathlineto{\pgfqpoint{4.683696in}{4.231724in}}%
\pgfpathlineto{\pgfqpoint{4.688344in}{4.268766in}}%
\pgfpathlineto{\pgfqpoint{4.690668in}{4.238329in}}%
\pgfpathlineto{\pgfqpoint{4.692992in}{4.316509in}}%
\pgfpathlineto{\pgfqpoint{4.695316in}{4.253308in}}%
\pgfpathlineto{\pgfqpoint{4.699964in}{4.317836in}}%
\pgfpathlineto{\pgfqpoint{4.702288in}{4.301701in}}%
\pgfpathlineto{\pgfqpoint{4.704612in}{4.273842in}}%
\pgfpathlineto{\pgfqpoint{4.706936in}{4.308512in}}%
\pgfpathlineto{\pgfqpoint{4.709260in}{4.274990in}}%
\pgfpathlineto{\pgfqpoint{4.711584in}{4.292121in}}%
\pgfpathlineto{\pgfqpoint{4.713908in}{4.293820in}}%
\pgfpathlineto{\pgfqpoint{4.716232in}{4.252343in}}%
\pgfpathlineto{\pgfqpoint{4.718556in}{4.301476in}}%
\pgfpathlineto{\pgfqpoint{4.720880in}{4.297725in}}%
\pgfpathlineto{\pgfqpoint{4.723204in}{4.313784in}}%
\pgfpathlineto{\pgfqpoint{4.725528in}{4.268771in}}%
\pgfpathlineto{\pgfqpoint{4.727852in}{4.321564in}}%
\pgfpathlineto{\pgfqpoint{4.730176in}{4.322638in}}%
\pgfpathlineto{\pgfqpoint{4.732500in}{4.317882in}}%
\pgfpathlineto{\pgfqpoint{4.734824in}{4.327932in}}%
\pgfpathlineto{\pgfqpoint{4.737148in}{4.353870in}}%
\pgfpathlineto{\pgfqpoint{4.739472in}{4.364093in}}%
\pgfpathlineto{\pgfqpoint{4.741796in}{4.361845in}}%
\pgfpathlineto{\pgfqpoint{4.744119in}{4.322330in}}%
\pgfpathlineto{\pgfqpoint{4.746443in}{4.360069in}}%
\pgfpathlineto{\pgfqpoint{4.748767in}{4.348219in}}%
\pgfpathlineto{\pgfqpoint{4.751091in}{4.303211in}}%
\pgfpathlineto{\pgfqpoint{4.753415in}{4.347315in}}%
\pgfpathlineto{\pgfqpoint{4.755739in}{4.337249in}}%
\pgfpathlineto{\pgfqpoint{4.758063in}{4.334044in}}%
\pgfpathlineto{\pgfqpoint{4.760387in}{4.380868in}}%
\pgfpathlineto{\pgfqpoint{4.765035in}{4.328456in}}%
\pgfpathlineto{\pgfqpoint{4.767359in}{4.313775in}}%
\pgfpathlineto{\pgfqpoint{4.769683in}{4.358960in}}%
\pgfpathlineto{\pgfqpoint{4.772007in}{4.377001in}}%
\pgfpathlineto{\pgfqpoint{4.774331in}{4.360118in}}%
\pgfpathlineto{\pgfqpoint{4.776655in}{4.379660in}}%
\pgfpathlineto{\pgfqpoint{4.778979in}{4.357805in}}%
\pgfpathlineto{\pgfqpoint{4.781303in}{4.313620in}}%
\pgfpathlineto{\pgfqpoint{4.783627in}{4.392012in}}%
\pgfpathlineto{\pgfqpoint{4.785951in}{4.387722in}}%
\pgfpathlineto{\pgfqpoint{4.788275in}{4.373313in}}%
\pgfpathlineto{\pgfqpoint{4.790599in}{4.349323in}}%
\pgfpathlineto{\pgfqpoint{4.792923in}{4.401635in}}%
\pgfpathlineto{\pgfqpoint{4.795247in}{4.354123in}}%
\pgfpathlineto{\pgfqpoint{4.797571in}{4.367491in}}%
\pgfpathlineto{\pgfqpoint{4.799895in}{4.327910in}}%
\pgfpathlineto{\pgfqpoint{4.802219in}{4.350705in}}%
\pgfpathlineto{\pgfqpoint{4.804543in}{4.435484in}}%
\pgfpathlineto{\pgfqpoint{4.809191in}{4.363696in}}%
\pgfpathlineto{\pgfqpoint{4.811515in}{4.370007in}}%
\pgfpathlineto{\pgfqpoint{4.813838in}{4.380758in}}%
\pgfpathlineto{\pgfqpoint{4.818486in}{4.333317in}}%
\pgfpathlineto{\pgfqpoint{4.820810in}{4.407126in}}%
\pgfpathlineto{\pgfqpoint{4.823134in}{4.412605in}}%
\pgfpathlineto{\pgfqpoint{4.825458in}{4.322180in}}%
\pgfpathlineto{\pgfqpoint{4.827782in}{4.401343in}}%
\pgfpathlineto{\pgfqpoint{4.830106in}{4.390317in}}%
\pgfpathlineto{\pgfqpoint{4.832430in}{4.330882in}}%
\pgfpathlineto{\pgfqpoint{4.834754in}{4.420079in}}%
\pgfpathlineto{\pgfqpoint{4.837078in}{4.388331in}}%
\pgfpathlineto{\pgfqpoint{4.839402in}{4.444622in}}%
\pgfpathlineto{\pgfqpoint{4.841726in}{4.338277in}}%
\pgfpathlineto{\pgfqpoint{4.844050in}{4.386200in}}%
\pgfpathlineto{\pgfqpoint{4.848698in}{4.361158in}}%
\pgfpathlineto{\pgfqpoint{4.855670in}{4.432463in}}%
\pgfpathlineto{\pgfqpoint{4.857994in}{4.341860in}}%
\pgfpathlineto{\pgfqpoint{4.860318in}{4.424133in}}%
\pgfpathlineto{\pgfqpoint{4.862642in}{4.332625in}}%
\pgfpathlineto{\pgfqpoint{4.864966in}{4.404017in}}%
\pgfpathlineto{\pgfqpoint{4.867290in}{4.407431in}}%
\pgfpathlineto{\pgfqpoint{4.869614in}{4.431300in}}%
\pgfpathlineto{\pgfqpoint{4.871938in}{4.401460in}}%
\pgfpathlineto{\pgfqpoint{4.874262in}{4.415488in}}%
\pgfpathlineto{\pgfqpoint{4.876586in}{4.401193in}}%
\pgfpathlineto{\pgfqpoint{4.878910in}{4.399302in}}%
\pgfpathlineto{\pgfqpoint{4.881234in}{4.377347in}}%
\pgfpathlineto{\pgfqpoint{4.883558in}{4.391408in}}%
\pgfpathlineto{\pgfqpoint{4.885881in}{4.433331in}}%
\pgfpathlineto{\pgfqpoint{4.888205in}{4.391328in}}%
\pgfpathlineto{\pgfqpoint{4.890529in}{4.427769in}}%
\pgfpathlineto{\pgfqpoint{4.892853in}{4.427381in}}%
\pgfpathlineto{\pgfqpoint{4.895177in}{4.412514in}}%
\pgfpathlineto{\pgfqpoint{4.897501in}{4.422881in}}%
\pgfpathlineto{\pgfqpoint{4.899825in}{4.401796in}}%
\pgfpathlineto{\pgfqpoint{4.902149in}{4.446989in}}%
\pgfpathlineto{\pgfqpoint{4.904473in}{4.384537in}}%
\pgfpathlineto{\pgfqpoint{4.909121in}{4.455533in}}%
\pgfpathlineto{\pgfqpoint{4.911445in}{4.426187in}}%
\pgfpathlineto{\pgfqpoint{4.913769in}{4.452157in}}%
\pgfpathlineto{\pgfqpoint{4.916093in}{4.453524in}}%
\pgfpathlineto{\pgfqpoint{4.918417in}{4.411889in}}%
\pgfpathlineto{\pgfqpoint{4.923065in}{4.464616in}}%
\pgfpathlineto{\pgfqpoint{4.925389in}{4.427238in}}%
\pgfpathlineto{\pgfqpoint{4.927713in}{4.450431in}}%
\pgfpathlineto{\pgfqpoint{4.930037in}{4.431637in}}%
\pgfpathlineto{\pgfqpoint{4.932361in}{4.440464in}}%
\pgfpathlineto{\pgfqpoint{4.934685in}{4.476346in}}%
\pgfpathlineto{\pgfqpoint{4.937009in}{4.411071in}}%
\pgfpathlineto{\pgfqpoint{4.939333in}{4.455336in}}%
\pgfpathlineto{\pgfqpoint{4.941657in}{4.411954in}}%
\pgfpathlineto{\pgfqpoint{4.943981in}{4.455194in}}%
\pgfpathlineto{\pgfqpoint{4.946305in}{4.448782in}}%
\pgfpathlineto{\pgfqpoint{4.948629in}{4.414478in}}%
\pgfpathlineto{\pgfqpoint{4.950953in}{4.445223in}}%
\pgfpathlineto{\pgfqpoint{4.953277in}{4.426770in}}%
\pgfpathlineto{\pgfqpoint{4.955600in}{4.469800in}}%
\pgfpathlineto{\pgfqpoint{4.957924in}{4.451447in}}%
\pgfpathlineto{\pgfqpoint{4.960248in}{4.417754in}}%
\pgfpathlineto{\pgfqpoint{4.962572in}{4.437375in}}%
\pgfpathlineto{\pgfqpoint{4.964896in}{4.479278in}}%
\pgfpathlineto{\pgfqpoint{4.967220in}{4.446929in}}%
\pgfpathlineto{\pgfqpoint{4.969544in}{4.442671in}}%
\pgfpathlineto{\pgfqpoint{4.974192in}{4.488957in}}%
\pgfpathlineto{\pgfqpoint{4.978840in}{4.418166in}}%
\pgfpathlineto{\pgfqpoint{4.981164in}{4.449447in}}%
\pgfpathlineto{\pgfqpoint{4.983488in}{4.450581in}}%
\pgfpathlineto{\pgfqpoint{4.985812in}{4.488852in}}%
\pgfpathlineto{\pgfqpoint{4.988136in}{4.432922in}}%
\pgfpathlineto{\pgfqpoint{4.990460in}{4.481297in}}%
\pgfpathlineto{\pgfqpoint{4.995108in}{4.430686in}}%
\pgfpathlineto{\pgfqpoint{4.997432in}{4.448136in}}%
\pgfpathlineto{\pgfqpoint{4.999756in}{4.503308in}}%
\pgfpathlineto{\pgfqpoint{5.004404in}{4.430895in}}%
\pgfpathlineto{\pgfqpoint{5.006728in}{4.451384in}}%
\pgfpathlineto{\pgfqpoint{5.009052in}{4.427099in}}%
\pgfpathlineto{\pgfqpoint{5.011376in}{4.448491in}}%
\pgfpathlineto{\pgfqpoint{5.013700in}{4.446887in}}%
\pgfpathlineto{\pgfqpoint{5.016024in}{4.459832in}}%
\pgfpathlineto{\pgfqpoint{5.018348in}{4.418858in}}%
\pgfpathlineto{\pgfqpoint{5.020672in}{4.450938in}}%
\pgfpathlineto{\pgfqpoint{5.022996in}{4.456202in}}%
\pgfpathlineto{\pgfqpoint{5.025319in}{4.414936in}}%
\pgfpathlineto{\pgfqpoint{5.027643in}{4.463553in}}%
\pgfpathlineto{\pgfqpoint{5.029967in}{4.461613in}}%
\pgfpathlineto{\pgfqpoint{5.032291in}{4.483830in}}%
\pgfpathlineto{\pgfqpoint{5.034615in}{4.480445in}}%
\pgfpathlineto{\pgfqpoint{5.036939in}{4.409740in}}%
\pgfpathlineto{\pgfqpoint{5.039263in}{4.442584in}}%
\pgfpathlineto{\pgfqpoint{5.041587in}{4.444975in}}%
\pgfpathlineto{\pgfqpoint{5.043911in}{4.443986in}}%
\pgfpathlineto{\pgfqpoint{5.048559in}{4.453459in}}%
\pgfpathlineto{\pgfqpoint{5.050883in}{4.423252in}}%
\pgfpathlineto{\pgfqpoint{5.053207in}{4.414185in}}%
\pgfpathlineto{\pgfqpoint{5.055531in}{4.416899in}}%
\pgfpathlineto{\pgfqpoint{5.057855in}{4.456932in}}%
\pgfpathlineto{\pgfqpoint{5.062503in}{4.421750in}}%
\pgfpathlineto{\pgfqpoint{5.064827in}{4.494072in}}%
\pgfpathlineto{\pgfqpoint{5.067151in}{4.413341in}}%
\pgfpathlineto{\pgfqpoint{5.069475in}{4.407108in}}%
\pgfpathlineto{\pgfqpoint{5.071799in}{4.479182in}}%
\pgfpathlineto{\pgfqpoint{5.074123in}{4.469871in}}%
\pgfpathlineto{\pgfqpoint{5.076447in}{4.465842in}}%
\pgfpathlineto{\pgfqpoint{5.078771in}{4.422501in}}%
\pgfpathlineto{\pgfqpoint{5.081095in}{4.512608in}}%
\pgfpathlineto{\pgfqpoint{5.083419in}{4.421757in}}%
\pgfpathlineto{\pgfqpoint{5.085743in}{4.477620in}}%
\pgfpathlineto{\pgfqpoint{5.088067in}{4.455665in}}%
\pgfpathlineto{\pgfqpoint{5.090391in}{4.466867in}}%
\pgfpathlineto{\pgfqpoint{5.092715in}{4.413182in}}%
\pgfpathlineto{\pgfqpoint{5.095039in}{4.413378in}}%
\pgfpathlineto{\pgfqpoint{5.097362in}{4.449893in}}%
\pgfpathlineto{\pgfqpoint{5.099686in}{4.430592in}}%
\pgfpathlineto{\pgfqpoint{5.102010in}{4.446061in}}%
\pgfpathlineto{\pgfqpoint{5.104334in}{4.431470in}}%
\pgfpathlineto{\pgfqpoint{5.106658in}{4.449265in}}%
\pgfpathlineto{\pgfqpoint{5.108982in}{4.519578in}}%
\pgfpathlineto{\pgfqpoint{5.111306in}{4.425699in}}%
\pgfpathlineto{\pgfqpoint{5.113630in}{4.432285in}}%
\pgfpathlineto{\pgfqpoint{5.115954in}{4.460221in}}%
\pgfpathlineto{\pgfqpoint{5.118278in}{4.396110in}}%
\pgfpathlineto{\pgfqpoint{5.120602in}{4.440607in}}%
\pgfpathlineto{\pgfqpoint{5.122926in}{4.427967in}}%
\pgfpathlineto{\pgfqpoint{5.125250in}{4.464719in}}%
\pgfpathlineto{\pgfqpoint{5.127574in}{4.478891in}}%
\pgfpathlineto{\pgfqpoint{5.132222in}{4.438125in}}%
\pgfpathlineto{\pgfqpoint{5.134546in}{4.445052in}}%
\pgfpathlineto{\pgfqpoint{5.136870in}{4.397142in}}%
\pgfpathlineto{\pgfqpoint{5.141518in}{4.484953in}}%
\pgfpathlineto{\pgfqpoint{5.143842in}{4.446958in}}%
\pgfpathlineto{\pgfqpoint{5.146166in}{4.429628in}}%
\pgfpathlineto{\pgfqpoint{5.148490in}{4.499692in}}%
\pgfpathlineto{\pgfqpoint{5.150814in}{4.440896in}}%
\pgfpathlineto{\pgfqpoint{5.153138in}{4.441487in}}%
\pgfpathlineto{\pgfqpoint{5.155462in}{4.418128in}}%
\pgfpathlineto{\pgfqpoint{5.157786in}{4.450901in}}%
\pgfpathlineto{\pgfqpoint{5.160110in}{4.400425in}}%
\pgfpathlineto{\pgfqpoint{5.162434in}{4.419013in}}%
\pgfpathlineto{\pgfqpoint{5.164758in}{4.449060in}}%
\pgfpathlineto{\pgfqpoint{5.169405in}{4.369485in}}%
\pgfpathlineto{\pgfqpoint{5.171729in}{4.394852in}}%
\pgfpathlineto{\pgfqpoint{5.174053in}{4.399239in}}%
\pgfpathlineto{\pgfqpoint{5.176377in}{4.371854in}}%
\pgfpathlineto{\pgfqpoint{5.178701in}{4.451917in}}%
\pgfpathlineto{\pgfqpoint{5.181025in}{4.400628in}}%
\pgfpathlineto{\pgfqpoint{5.183349in}{4.488102in}}%
\pgfpathlineto{\pgfqpoint{5.185673in}{4.417063in}}%
\pgfpathlineto{\pgfqpoint{5.187997in}{4.431499in}}%
\pgfpathlineto{\pgfqpoint{5.190321in}{4.429146in}}%
\pgfpathlineto{\pgfqpoint{5.192645in}{4.472216in}}%
\pgfpathlineto{\pgfqpoint{5.194969in}{4.428956in}}%
\pgfpathlineto{\pgfqpoint{5.197293in}{4.491556in}}%
\pgfpathlineto{\pgfqpoint{5.199617in}{4.391475in}}%
\pgfpathlineto{\pgfqpoint{5.201941in}{4.409333in}}%
\pgfpathlineto{\pgfqpoint{5.204265in}{4.402156in}}%
\pgfpathlineto{\pgfqpoint{5.206589in}{4.400692in}}%
\pgfpathlineto{\pgfqpoint{5.211237in}{4.449788in}}%
\pgfpathlineto{\pgfqpoint{5.213561in}{4.441741in}}%
\pgfpathlineto{\pgfqpoint{5.215885in}{4.392604in}}%
\pgfpathlineto{\pgfqpoint{5.218209in}{4.375444in}}%
\pgfpathlineto{\pgfqpoint{5.222857in}{4.407280in}}%
\pgfpathlineto{\pgfqpoint{5.225181in}{4.385121in}}%
\pgfpathlineto{\pgfqpoint{5.227505in}{4.411249in}}%
\pgfpathlineto{\pgfqpoint{5.229829in}{4.370371in}}%
\pgfpathlineto{\pgfqpoint{5.232153in}{4.428893in}}%
\pgfpathlineto{\pgfqpoint{5.234477in}{4.400826in}}%
\pgfpathlineto{\pgfqpoint{5.236800in}{4.401103in}}%
\pgfpathlineto{\pgfqpoint{5.239124in}{4.425855in}}%
\pgfpathlineto{\pgfqpoint{5.241448in}{4.421837in}}%
\pgfpathlineto{\pgfqpoint{5.243772in}{4.339467in}}%
\pgfpathlineto{\pgfqpoint{5.246096in}{4.428907in}}%
\pgfpathlineto{\pgfqpoint{5.248420in}{4.367543in}}%
\pgfpathlineto{\pgfqpoint{5.250744in}{4.367578in}}%
\pgfpathlineto{\pgfqpoint{5.253068in}{4.400951in}}%
\pgfpathlineto{\pgfqpoint{5.255392in}{4.368845in}}%
\pgfpathlineto{\pgfqpoint{5.257716in}{4.383446in}}%
\pgfpathlineto{\pgfqpoint{5.260040in}{4.367202in}}%
\pgfpathlineto{\pgfqpoint{5.262364in}{4.370342in}}%
\pgfpathlineto{\pgfqpoint{5.264688in}{4.361309in}}%
\pgfpathlineto{\pgfqpoint{5.267012in}{4.384098in}}%
\pgfpathlineto{\pgfqpoint{5.269336in}{4.342717in}}%
\pgfpathlineto{\pgfqpoint{5.271660in}{4.394388in}}%
\pgfpathlineto{\pgfqpoint{5.273984in}{4.343022in}}%
\pgfpathlineto{\pgfqpoint{5.276308in}{4.393899in}}%
\pgfpathlineto{\pgfqpoint{5.278632in}{4.342013in}}%
\pgfpathlineto{\pgfqpoint{5.280956in}{4.380123in}}%
\pgfpathlineto{\pgfqpoint{5.283280in}{4.381378in}}%
\pgfpathlineto{\pgfqpoint{5.285604in}{4.357197in}}%
\pgfpathlineto{\pgfqpoint{5.290252in}{4.377876in}}%
\pgfpathlineto{\pgfqpoint{5.292576in}{4.349759in}}%
\pgfpathlineto{\pgfqpoint{5.294900in}{4.348888in}}%
\pgfpathlineto{\pgfqpoint{5.297224in}{4.365134in}}%
\pgfpathlineto{\pgfqpoint{5.299548in}{4.366786in}}%
\pgfpathlineto{\pgfqpoint{5.301872in}{4.371068in}}%
\pgfpathlineto{\pgfqpoint{5.304196in}{4.392821in}}%
\pgfpathlineto{\pgfqpoint{5.306519in}{4.350346in}}%
\pgfpathlineto{\pgfqpoint{5.308843in}{4.346584in}}%
\pgfpathlineto{\pgfqpoint{5.311167in}{4.356770in}}%
\pgfpathlineto{\pgfqpoint{5.313491in}{4.327526in}}%
\pgfpathlineto{\pgfqpoint{5.315815in}{4.353471in}}%
\pgfpathlineto{\pgfqpoint{5.320463in}{4.342213in}}%
\pgfpathlineto{\pgfqpoint{5.322787in}{4.343343in}}%
\pgfpathlineto{\pgfqpoint{5.325111in}{4.319814in}}%
\pgfpathlineto{\pgfqpoint{5.327435in}{4.347078in}}%
\pgfpathlineto{\pgfqpoint{5.329759in}{4.329481in}}%
\pgfpathlineto{\pgfqpoint{5.332083in}{4.365842in}}%
\pgfpathlineto{\pgfqpoint{5.334407in}{4.321417in}}%
\pgfpathlineto{\pgfqpoint{5.336731in}{4.324116in}}%
\pgfpathlineto{\pgfqpoint{5.339055in}{4.290961in}}%
\pgfpathlineto{\pgfqpoint{5.341379in}{4.349907in}}%
\pgfpathlineto{\pgfqpoint{5.343703in}{4.331375in}}%
\pgfpathlineto{\pgfqpoint{5.346027in}{4.264093in}}%
\pgfpathlineto{\pgfqpoint{5.348351in}{4.358078in}}%
\pgfpathlineto{\pgfqpoint{5.350675in}{4.324520in}}%
\pgfpathlineto{\pgfqpoint{5.352999in}{4.312388in}}%
\pgfpathlineto{\pgfqpoint{5.355323in}{4.328982in}}%
\pgfpathlineto{\pgfqpoint{5.357647in}{4.274162in}}%
\pgfpathlineto{\pgfqpoint{5.359971in}{4.278874in}}%
\pgfpathlineto{\pgfqpoint{5.362295in}{4.290843in}}%
\pgfpathlineto{\pgfqpoint{5.364619in}{4.324225in}}%
\pgfpathlineto{\pgfqpoint{5.369267in}{4.275953in}}%
\pgfpathlineto{\pgfqpoint{5.371591in}{4.284065in}}%
\pgfpathlineto{\pgfqpoint{5.373915in}{4.311384in}}%
\pgfpathlineto{\pgfqpoint{5.378562in}{4.325977in}}%
\pgfpathlineto{\pgfqpoint{5.380886in}{4.287772in}}%
\pgfpathlineto{\pgfqpoint{5.383210in}{4.311747in}}%
\pgfpathlineto{\pgfqpoint{5.385534in}{4.288658in}}%
\pgfpathlineto{\pgfqpoint{5.387858in}{4.326848in}}%
\pgfpathlineto{\pgfqpoint{5.392506in}{4.257439in}}%
\pgfpathlineto{\pgfqpoint{5.394830in}{4.272090in}}%
\pgfpathlineto{\pgfqpoint{5.397154in}{4.275479in}}%
\pgfpathlineto{\pgfqpoint{5.399478in}{4.304656in}}%
\pgfpathlineto{\pgfqpoint{5.401802in}{4.312248in}}%
\pgfpathlineto{\pgfqpoint{5.404126in}{4.307639in}}%
\pgfpathlineto{\pgfqpoint{5.406450in}{4.287432in}}%
\pgfpathlineto{\pgfqpoint{5.408774in}{4.318922in}}%
\pgfpathlineto{\pgfqpoint{5.411098in}{4.263228in}}%
\pgfpathlineto{\pgfqpoint{5.413422in}{4.259205in}}%
\pgfpathlineto{\pgfqpoint{5.415746in}{4.262942in}}%
\pgfpathlineto{\pgfqpoint{5.418070in}{4.269627in}}%
\pgfpathlineto{\pgfqpoint{5.420394in}{4.251401in}}%
\pgfpathlineto{\pgfqpoint{5.422718in}{4.263525in}}%
\pgfpathlineto{\pgfqpoint{5.425042in}{4.262572in}}%
\pgfpathlineto{\pgfqpoint{5.429690in}{4.294557in}}%
\pgfpathlineto{\pgfqpoint{5.432014in}{4.229182in}}%
\pgfpathlineto{\pgfqpoint{5.434338in}{4.265662in}}%
\pgfpathlineto{\pgfqpoint{5.436662in}{4.209198in}}%
\pgfpathlineto{\pgfqpoint{5.438986in}{4.265770in}}%
\pgfpathlineto{\pgfqpoint{5.441310in}{4.229356in}}%
\pgfpathlineto{\pgfqpoint{5.443634in}{4.280220in}}%
\pgfpathlineto{\pgfqpoint{5.445958in}{4.226730in}}%
\pgfpathlineto{\pgfqpoint{5.448281in}{4.245841in}}%
\pgfpathlineto{\pgfqpoint{5.450605in}{4.242932in}}%
\pgfpathlineto{\pgfqpoint{5.452929in}{4.251253in}}%
\pgfpathlineto{\pgfqpoint{5.455253in}{4.200475in}}%
\pgfpathlineto{\pgfqpoint{5.457577in}{4.177452in}}%
\pgfpathlineto{\pgfqpoint{5.462225in}{4.248818in}}%
\pgfpathlineto{\pgfqpoint{5.464549in}{4.233105in}}%
\pgfpathlineto{\pgfqpoint{5.466873in}{4.237052in}}%
\pgfpathlineto{\pgfqpoint{5.469197in}{4.262118in}}%
\pgfpathlineto{\pgfqpoint{5.471521in}{4.211237in}}%
\pgfpathlineto{\pgfqpoint{5.473845in}{4.185475in}}%
\pgfpathlineto{\pgfqpoint{5.476169in}{4.209369in}}%
\pgfpathlineto{\pgfqpoint{5.480817in}{4.163597in}}%
\pgfpathlineto{\pgfqpoint{5.483141in}{4.198042in}}%
\pgfpathlineto{\pgfqpoint{5.485465in}{4.206837in}}%
\pgfpathlineto{\pgfqpoint{5.487789in}{4.191862in}}%
\pgfpathlineto{\pgfqpoint{5.490113in}{4.230939in}}%
\pgfpathlineto{\pgfqpoint{5.492437in}{4.182533in}}%
\pgfpathlineto{\pgfqpoint{5.494761in}{4.173487in}}%
\pgfpathlineto{\pgfqpoint{5.497085in}{4.139135in}}%
\pgfpathlineto{\pgfqpoint{5.501733in}{4.185048in}}%
\pgfpathlineto{\pgfqpoint{5.504057in}{4.185031in}}%
\pgfpathlineto{\pgfqpoint{5.506381in}{4.190921in}}%
\pgfpathlineto{\pgfqpoint{5.508705in}{4.160418in}}%
\pgfpathlineto{\pgfqpoint{5.511029in}{4.176941in}}%
\pgfpathlineto{\pgfqpoint{5.513353in}{4.171017in}}%
\pgfpathlineto{\pgfqpoint{5.515677in}{4.159437in}}%
\pgfpathlineto{\pgfqpoint{5.518000in}{4.159779in}}%
\pgfpathlineto{\pgfqpoint{5.520324in}{4.175957in}}%
\pgfpathlineto{\pgfqpoint{5.522648in}{4.125995in}}%
\pgfpathlineto{\pgfqpoint{5.524972in}{4.161191in}}%
\pgfpathlineto{\pgfqpoint{5.529620in}{4.140593in}}%
\pgfpathlineto{\pgfqpoint{5.531944in}{4.142786in}}%
\pgfpathlineto{\pgfqpoint{5.534268in}{4.193714in}}%
\pgfpathlineto{\pgfqpoint{5.536592in}{4.159066in}}%
\pgfpathlineto{\pgfqpoint{5.541240in}{4.127082in}}%
\pgfpathlineto{\pgfqpoint{5.543564in}{4.150918in}}%
\pgfpathlineto{\pgfqpoint{5.545888in}{4.155875in}}%
\pgfpathlineto{\pgfqpoint{5.548212in}{4.164051in}}%
\pgfpathlineto{\pgfqpoint{5.552860in}{4.126129in}}%
\pgfpathlineto{\pgfqpoint{5.555184in}{4.131200in}}%
\pgfpathlineto{\pgfqpoint{5.557508in}{4.168965in}}%
\pgfpathlineto{\pgfqpoint{5.559832in}{4.105902in}}%
\pgfpathlineto{\pgfqpoint{5.564480in}{4.138199in}}%
\pgfpathlineto{\pgfqpoint{5.566804in}{4.189641in}}%
\pgfpathlineto{\pgfqpoint{5.569128in}{4.121199in}}%
\pgfpathlineto{\pgfqpoint{5.571452in}{4.123930in}}%
\pgfpathlineto{\pgfqpoint{5.573776in}{4.100897in}}%
\pgfpathlineto{\pgfqpoint{5.576100in}{4.062107in}}%
\pgfpathlineto{\pgfqpoint{5.580748in}{4.146494in}}%
\pgfpathlineto{\pgfqpoint{5.583072in}{4.106440in}}%
\pgfpathlineto{\pgfqpoint{5.585396in}{4.093335in}}%
\pgfpathlineto{\pgfqpoint{5.587720in}{4.087617in}}%
\pgfpathlineto{\pgfqpoint{5.587720in}{4.087617in}}%
\pgfusepath{stroke}%
\end{pgfscope}%
\begin{pgfscope}%
\pgfpathrectangle{\pgfqpoint{0.709829in}{3.729963in}}{\pgfqpoint{5.110171in}{0.887537in}}%
\pgfusepath{clip}%
\pgfsetroundcap%
\pgfsetroundjoin%
\pgfsetlinewidth{1.003750pt}%
\definecolor{currentstroke}{rgb}{0.866667,0.517647,0.321569}%
\pgfsetstrokecolor{currentstroke}%
\pgfsetdash{}{0pt}%
\pgfpathmoveto{\pgfqpoint{0.942110in}{3.865072in}}%
\pgfpathlineto{\pgfqpoint{0.944433in}{3.916184in}}%
\pgfpathlineto{\pgfqpoint{0.946757in}{3.852144in}}%
\pgfpathlineto{\pgfqpoint{0.949081in}{3.864544in}}%
\pgfpathlineto{\pgfqpoint{0.951405in}{3.866449in}}%
\pgfpathlineto{\pgfqpoint{0.953729in}{3.914423in}}%
\pgfpathlineto{\pgfqpoint{0.956053in}{3.839507in}}%
\pgfpathlineto{\pgfqpoint{0.958377in}{3.898913in}}%
\pgfpathlineto{\pgfqpoint{0.960701in}{3.824968in}}%
\pgfpathlineto{\pgfqpoint{0.963025in}{3.926276in}}%
\pgfpathlineto{\pgfqpoint{0.967673in}{3.869445in}}%
\pgfpathlineto{\pgfqpoint{0.969997in}{3.923294in}}%
\pgfpathlineto{\pgfqpoint{0.972321in}{3.851590in}}%
\pgfpathlineto{\pgfqpoint{0.974645in}{3.953301in}}%
\pgfpathlineto{\pgfqpoint{0.976969in}{3.850131in}}%
\pgfpathlineto{\pgfqpoint{0.979293in}{3.887128in}}%
\pgfpathlineto{\pgfqpoint{0.981617in}{3.879927in}}%
\pgfpathlineto{\pgfqpoint{0.983941in}{3.913590in}}%
\pgfpathlineto{\pgfqpoint{0.986265in}{3.881134in}}%
\pgfpathlineto{\pgfqpoint{0.988589in}{3.892853in}}%
\pgfpathlineto{\pgfqpoint{0.993237in}{3.925302in}}%
\pgfpathlineto{\pgfqpoint{0.995561in}{3.936742in}}%
\pgfpathlineto{\pgfqpoint{0.997885in}{3.874518in}}%
\pgfpathlineto{\pgfqpoint{1.000209in}{3.889186in}}%
\pgfpathlineto{\pgfqpoint{1.002533in}{3.922447in}}%
\pgfpathlineto{\pgfqpoint{1.004857in}{3.938616in}}%
\pgfpathlineto{\pgfqpoint{1.007181in}{3.988499in}}%
\pgfpathlineto{\pgfqpoint{1.009505in}{3.893156in}}%
\pgfpathlineto{\pgfqpoint{1.011829in}{3.971543in}}%
\pgfpathlineto{\pgfqpoint{1.014153in}{3.945581in}}%
\pgfpathlineto{\pgfqpoint{1.016476in}{3.995440in}}%
\pgfpathlineto{\pgfqpoint{1.018800in}{4.001297in}}%
\pgfpathlineto{\pgfqpoint{1.021124in}{3.973760in}}%
\pgfpathlineto{\pgfqpoint{1.023448in}{3.930723in}}%
\pgfpathlineto{\pgfqpoint{1.025772in}{3.995929in}}%
\pgfpathlineto{\pgfqpoint{1.028096in}{3.921226in}}%
\pgfpathlineto{\pgfqpoint{1.030420in}{3.978821in}}%
\pgfpathlineto{\pgfqpoint{1.032744in}{3.975242in}}%
\pgfpathlineto{\pgfqpoint{1.035068in}{3.921290in}}%
\pgfpathlineto{\pgfqpoint{1.037392in}{3.937306in}}%
\pgfpathlineto{\pgfqpoint{1.039716in}{4.023971in}}%
\pgfpathlineto{\pgfqpoint{1.042040in}{3.940104in}}%
\pgfpathlineto{\pgfqpoint{1.046688in}{3.989866in}}%
\pgfpathlineto{\pgfqpoint{1.049012in}{3.937362in}}%
\pgfpathlineto{\pgfqpoint{1.053660in}{4.009949in}}%
\pgfpathlineto{\pgfqpoint{1.055984in}{3.937932in}}%
\pgfpathlineto{\pgfqpoint{1.058308in}{4.040631in}}%
\pgfpathlineto{\pgfqpoint{1.060632in}{3.983432in}}%
\pgfpathlineto{\pgfqpoint{1.062956in}{3.953658in}}%
\pgfpathlineto{\pgfqpoint{1.067604in}{3.997507in}}%
\pgfpathlineto{\pgfqpoint{1.069928in}{3.964222in}}%
\pgfpathlineto{\pgfqpoint{1.072252in}{4.010371in}}%
\pgfpathlineto{\pgfqpoint{1.074576in}{4.030847in}}%
\pgfpathlineto{\pgfqpoint{1.076900in}{4.027703in}}%
\pgfpathlineto{\pgfqpoint{1.081548in}{3.965063in}}%
\pgfpathlineto{\pgfqpoint{1.083872in}{4.033772in}}%
\pgfpathlineto{\pgfqpoint{1.086195in}{4.017487in}}%
\pgfpathlineto{\pgfqpoint{1.088519in}{3.975706in}}%
\pgfpathlineto{\pgfqpoint{1.090843in}{3.982470in}}%
\pgfpathlineto{\pgfqpoint{1.093167in}{4.049188in}}%
\pgfpathlineto{\pgfqpoint{1.095491in}{4.070484in}}%
\pgfpathlineto{\pgfqpoint{1.097815in}{4.065304in}}%
\pgfpathlineto{\pgfqpoint{1.100139in}{3.978219in}}%
\pgfpathlineto{\pgfqpoint{1.102463in}{4.043516in}}%
\pgfpathlineto{\pgfqpoint{1.104787in}{4.047392in}}%
\pgfpathlineto{\pgfqpoint{1.107111in}{4.008363in}}%
\pgfpathlineto{\pgfqpoint{1.109435in}{4.054725in}}%
\pgfpathlineto{\pgfqpoint{1.111759in}{4.013271in}}%
\pgfpathlineto{\pgfqpoint{1.114083in}{4.063392in}}%
\pgfpathlineto{\pgfqpoint{1.121055in}{4.037489in}}%
\pgfpathlineto{\pgfqpoint{1.123379in}{4.083866in}}%
\pgfpathlineto{\pgfqpoint{1.125703in}{4.074786in}}%
\pgfpathlineto{\pgfqpoint{1.128027in}{4.050229in}}%
\pgfpathlineto{\pgfqpoint{1.130351in}{4.055788in}}%
\pgfpathlineto{\pgfqpoint{1.132675in}{4.104298in}}%
\pgfpathlineto{\pgfqpoint{1.134999in}{4.088493in}}%
\pgfpathlineto{\pgfqpoint{1.137323in}{4.063378in}}%
\pgfpathlineto{\pgfqpoint{1.139647in}{4.052603in}}%
\pgfpathlineto{\pgfqpoint{1.141971in}{4.099374in}}%
\pgfpathlineto{\pgfqpoint{1.144295in}{4.062161in}}%
\pgfpathlineto{\pgfqpoint{1.146619in}{4.090736in}}%
\pgfpathlineto{\pgfqpoint{1.148943in}{4.076876in}}%
\pgfpathlineto{\pgfqpoint{1.151267in}{3.998210in}}%
\pgfpathlineto{\pgfqpoint{1.153591in}{4.070883in}}%
\pgfpathlineto{\pgfqpoint{1.155914in}{4.091240in}}%
\pgfpathlineto{\pgfqpoint{1.158238in}{4.094068in}}%
\pgfpathlineto{\pgfqpoint{1.160562in}{4.151109in}}%
\pgfpathlineto{\pgfqpoint{1.162886in}{4.068482in}}%
\pgfpathlineto{\pgfqpoint{1.165210in}{4.148650in}}%
\pgfpathlineto{\pgfqpoint{1.167534in}{4.184391in}}%
\pgfpathlineto{\pgfqpoint{1.172182in}{4.100387in}}%
\pgfpathlineto{\pgfqpoint{1.174506in}{4.170992in}}%
\pgfpathlineto{\pgfqpoint{1.176830in}{4.140879in}}%
\pgfpathlineto{\pgfqpoint{1.179154in}{4.077031in}}%
\pgfpathlineto{\pgfqpoint{1.181478in}{4.119740in}}%
\pgfpathlineto{\pgfqpoint{1.183802in}{4.102666in}}%
\pgfpathlineto{\pgfqpoint{1.186126in}{4.128050in}}%
\pgfpathlineto{\pgfqpoint{1.188450in}{4.128543in}}%
\pgfpathlineto{\pgfqpoint{1.190774in}{4.143798in}}%
\pgfpathlineto{\pgfqpoint{1.195422in}{4.116742in}}%
\pgfpathlineto{\pgfqpoint{1.197746in}{4.103391in}}%
\pgfpathlineto{\pgfqpoint{1.200070in}{4.057474in}}%
\pgfpathlineto{\pgfqpoint{1.202394in}{4.162843in}}%
\pgfpathlineto{\pgfqpoint{1.204718in}{4.185885in}}%
\pgfpathlineto{\pgfqpoint{1.207042in}{4.113136in}}%
\pgfpathlineto{\pgfqpoint{1.214014in}{4.206238in}}%
\pgfpathlineto{\pgfqpoint{1.216338in}{4.146597in}}%
\pgfpathlineto{\pgfqpoint{1.218662in}{4.150612in}}%
\pgfpathlineto{\pgfqpoint{1.220986in}{4.169223in}}%
\pgfpathlineto{\pgfqpoint{1.223310in}{4.171224in}}%
\pgfpathlineto{\pgfqpoint{1.225633in}{4.161344in}}%
\pgfpathlineto{\pgfqpoint{1.227957in}{4.186293in}}%
\pgfpathlineto{\pgfqpoint{1.230281in}{4.171434in}}%
\pgfpathlineto{\pgfqpoint{1.232605in}{4.203566in}}%
\pgfpathlineto{\pgfqpoint{1.234929in}{4.214033in}}%
\pgfpathlineto{\pgfqpoint{1.237253in}{4.113104in}}%
\pgfpathlineto{\pgfqpoint{1.239577in}{4.135528in}}%
\pgfpathlineto{\pgfqpoint{1.241901in}{4.247329in}}%
\pgfpathlineto{\pgfqpoint{1.244225in}{4.188181in}}%
\pgfpathlineto{\pgfqpoint{1.246549in}{4.188540in}}%
\pgfpathlineto{\pgfqpoint{1.248873in}{4.238513in}}%
\pgfpathlineto{\pgfqpoint{1.251197in}{4.183693in}}%
\pgfpathlineto{\pgfqpoint{1.253521in}{4.217879in}}%
\pgfpathlineto{\pgfqpoint{1.255845in}{4.235657in}}%
\pgfpathlineto{\pgfqpoint{1.258169in}{4.184721in}}%
\pgfpathlineto{\pgfqpoint{1.260493in}{4.203832in}}%
\pgfpathlineto{\pgfqpoint{1.265141in}{4.269534in}}%
\pgfpathlineto{\pgfqpoint{1.269789in}{4.220319in}}%
\pgfpathlineto{\pgfqpoint{1.272113in}{4.136769in}}%
\pgfpathlineto{\pgfqpoint{1.274437in}{4.214484in}}%
\pgfpathlineto{\pgfqpoint{1.279085in}{4.262328in}}%
\pgfpathlineto{\pgfqpoint{1.281409in}{4.240420in}}%
\pgfpathlineto{\pgfqpoint{1.283733in}{4.237640in}}%
\pgfpathlineto{\pgfqpoint{1.286057in}{4.239320in}}%
\pgfpathlineto{\pgfqpoint{1.288381in}{4.219916in}}%
\pgfpathlineto{\pgfqpoint{1.290705in}{4.275502in}}%
\pgfpathlineto{\pgfqpoint{1.293029in}{4.262247in}}%
\pgfpathlineto{\pgfqpoint{1.295353in}{4.180984in}}%
\pgfpathlineto{\pgfqpoint{1.297676in}{4.239247in}}%
\pgfpathlineto{\pgfqpoint{1.300000in}{4.264974in}}%
\pgfpathlineto{\pgfqpoint{1.302324in}{4.229903in}}%
\pgfpathlineto{\pgfqpoint{1.304648in}{4.272578in}}%
\pgfpathlineto{\pgfqpoint{1.306972in}{4.257810in}}%
\pgfpathlineto{\pgfqpoint{1.309296in}{4.302404in}}%
\pgfpathlineto{\pgfqpoint{1.311620in}{4.309480in}}%
\pgfpathlineto{\pgfqpoint{1.313944in}{4.308848in}}%
\pgfpathlineto{\pgfqpoint{1.316268in}{4.305981in}}%
\pgfpathlineto{\pgfqpoint{1.318592in}{4.298404in}}%
\pgfpathlineto{\pgfqpoint{1.320916in}{4.248759in}}%
\pgfpathlineto{\pgfqpoint{1.323240in}{4.328294in}}%
\pgfpathlineto{\pgfqpoint{1.325564in}{4.266313in}}%
\pgfpathlineto{\pgfqpoint{1.327888in}{4.288137in}}%
\pgfpathlineto{\pgfqpoint{1.330212in}{4.329564in}}%
\pgfpathlineto{\pgfqpoint{1.332536in}{4.276797in}}%
\pgfpathlineto{\pgfqpoint{1.334860in}{4.259693in}}%
\pgfpathlineto{\pgfqpoint{1.337184in}{4.346698in}}%
\pgfpathlineto{\pgfqpoint{1.339508in}{4.282126in}}%
\pgfpathlineto{\pgfqpoint{1.341832in}{4.341185in}}%
\pgfpathlineto{\pgfqpoint{1.344156in}{4.301862in}}%
\pgfpathlineto{\pgfqpoint{1.346480in}{4.282496in}}%
\pgfpathlineto{\pgfqpoint{1.348804in}{4.345442in}}%
\pgfpathlineto{\pgfqpoint{1.351128in}{4.345685in}}%
\pgfpathlineto{\pgfqpoint{1.353452in}{4.360246in}}%
\pgfpathlineto{\pgfqpoint{1.355776in}{4.303096in}}%
\pgfpathlineto{\pgfqpoint{1.358100in}{4.318375in}}%
\pgfpathlineto{\pgfqpoint{1.360424in}{4.287802in}}%
\pgfpathlineto{\pgfqpoint{1.362748in}{4.317659in}}%
\pgfpathlineto{\pgfqpoint{1.365072in}{4.248479in}}%
\pgfpathlineto{\pgfqpoint{1.367395in}{4.341066in}}%
\pgfpathlineto{\pgfqpoint{1.369719in}{4.329457in}}%
\pgfpathlineto{\pgfqpoint{1.372043in}{4.262851in}}%
\pgfpathlineto{\pgfqpoint{1.376691in}{4.352669in}}%
\pgfpathlineto{\pgfqpoint{1.379015in}{4.355362in}}%
\pgfpathlineto{\pgfqpoint{1.381339in}{4.351799in}}%
\pgfpathlineto{\pgfqpoint{1.383663in}{4.293722in}}%
\pgfpathlineto{\pgfqpoint{1.385987in}{4.376285in}}%
\pgfpathlineto{\pgfqpoint{1.388311in}{4.322507in}}%
\pgfpathlineto{\pgfqpoint{1.390635in}{4.297621in}}%
\pgfpathlineto{\pgfqpoint{1.392959in}{4.390659in}}%
\pgfpathlineto{\pgfqpoint{1.395283in}{4.360210in}}%
\pgfpathlineto{\pgfqpoint{1.397607in}{4.269406in}}%
\pgfpathlineto{\pgfqpoint{1.399931in}{4.365412in}}%
\pgfpathlineto{\pgfqpoint{1.402255in}{4.387833in}}%
\pgfpathlineto{\pgfqpoint{1.404579in}{4.347228in}}%
\pgfpathlineto{\pgfqpoint{1.406903in}{4.394378in}}%
\pgfpathlineto{\pgfqpoint{1.409227in}{4.336799in}}%
\pgfpathlineto{\pgfqpoint{1.411551in}{4.331029in}}%
\pgfpathlineto{\pgfqpoint{1.413875in}{4.330976in}}%
\pgfpathlineto{\pgfqpoint{1.416199in}{4.420749in}}%
\pgfpathlineto{\pgfqpoint{1.418523in}{4.401206in}}%
\pgfpathlineto{\pgfqpoint{1.420847in}{4.346681in}}%
\pgfpathlineto{\pgfqpoint{1.423171in}{4.391693in}}%
\pgfpathlineto{\pgfqpoint{1.425495in}{4.361243in}}%
\pgfpathlineto{\pgfqpoint{1.427819in}{4.402533in}}%
\pgfpathlineto{\pgfqpoint{1.430143in}{4.403151in}}%
\pgfpathlineto{\pgfqpoint{1.432467in}{4.394686in}}%
\pgfpathlineto{\pgfqpoint{1.434791in}{4.420658in}}%
\pgfpathlineto{\pgfqpoint{1.437114in}{4.411977in}}%
\pgfpathlineto{\pgfqpoint{1.439438in}{4.384701in}}%
\pgfpathlineto{\pgfqpoint{1.441762in}{4.454235in}}%
\pgfpathlineto{\pgfqpoint{1.444086in}{4.474163in}}%
\pgfpathlineto{\pgfqpoint{1.446410in}{4.436013in}}%
\pgfpathlineto{\pgfqpoint{1.448734in}{4.443035in}}%
\pgfpathlineto{\pgfqpoint{1.453382in}{4.391749in}}%
\pgfpathlineto{\pgfqpoint{1.455706in}{4.421309in}}%
\pgfpathlineto{\pgfqpoint{1.458030in}{4.474401in}}%
\pgfpathlineto{\pgfqpoint{1.460354in}{4.423375in}}%
\pgfpathlineto{\pgfqpoint{1.462678in}{4.413894in}}%
\pgfpathlineto{\pgfqpoint{1.465002in}{4.415672in}}%
\pgfpathlineto{\pgfqpoint{1.467326in}{4.462413in}}%
\pgfpathlineto{\pgfqpoint{1.469650in}{4.476924in}}%
\pgfpathlineto{\pgfqpoint{1.471974in}{4.360993in}}%
\pgfpathlineto{\pgfqpoint{1.474298in}{4.456080in}}%
\pgfpathlineto{\pgfqpoint{1.476622in}{4.491258in}}%
\pgfpathlineto{\pgfqpoint{1.478946in}{4.444954in}}%
\pgfpathlineto{\pgfqpoint{1.481270in}{4.438946in}}%
\pgfpathlineto{\pgfqpoint{1.483594in}{4.420110in}}%
\pgfpathlineto{\pgfqpoint{1.485918in}{4.506635in}}%
\pgfpathlineto{\pgfqpoint{1.488242in}{4.504054in}}%
\pgfpathlineto{\pgfqpoint{1.492890in}{4.455411in}}%
\pgfpathlineto{\pgfqpoint{1.495214in}{4.477709in}}%
\pgfpathlineto{\pgfqpoint{1.497538in}{4.517448in}}%
\pgfpathlineto{\pgfqpoint{1.502186in}{4.436078in}}%
\pgfpathlineto{\pgfqpoint{1.504510in}{4.532696in}}%
\pgfpathlineto{\pgfqpoint{1.506834in}{4.477581in}}%
\pgfpathlineto{\pgfqpoint{1.509157in}{4.462174in}}%
\pgfpathlineto{\pgfqpoint{1.511481in}{4.518189in}}%
\pgfpathlineto{\pgfqpoint{1.513805in}{4.469571in}}%
\pgfpathlineto{\pgfqpoint{1.516129in}{4.562814in}}%
\pgfpathlineto{\pgfqpoint{1.518453in}{4.525849in}}%
\pgfpathlineto{\pgfqpoint{1.520777in}{4.449881in}}%
\pgfpathlineto{\pgfqpoint{1.523101in}{3.797304in}}%
\pgfpathlineto{\pgfqpoint{1.525425in}{3.854339in}}%
\pgfpathlineto{\pgfqpoint{1.527749in}{3.834105in}}%
\pgfpathlineto{\pgfqpoint{1.530073in}{3.845211in}}%
\pgfpathlineto{\pgfqpoint{1.532397in}{3.863574in}}%
\pgfpathlineto{\pgfqpoint{1.537045in}{3.860974in}}%
\pgfpathlineto{\pgfqpoint{1.541693in}{3.814169in}}%
\pgfpathlineto{\pgfqpoint{1.544017in}{3.822061in}}%
\pgfpathlineto{\pgfqpoint{1.546341in}{3.895841in}}%
\pgfpathlineto{\pgfqpoint{1.550989in}{3.897870in}}%
\pgfpathlineto{\pgfqpoint{1.553313in}{3.861522in}}%
\pgfpathlineto{\pgfqpoint{1.555637in}{3.932548in}}%
\pgfpathlineto{\pgfqpoint{1.560285in}{3.867072in}}%
\pgfpathlineto{\pgfqpoint{1.564933in}{3.913156in}}%
\pgfpathlineto{\pgfqpoint{1.567257in}{3.846654in}}%
\pgfpathlineto{\pgfqpoint{1.569581in}{3.926366in}}%
\pgfpathlineto{\pgfqpoint{1.571905in}{3.923174in}}%
\pgfpathlineto{\pgfqpoint{1.574229in}{3.965900in}}%
\pgfpathlineto{\pgfqpoint{1.576553in}{3.912623in}}%
\pgfpathlineto{\pgfqpoint{1.578876in}{3.896108in}}%
\pgfpathlineto{\pgfqpoint{1.581200in}{3.929179in}}%
\pgfpathlineto{\pgfqpoint{1.583524in}{3.904082in}}%
\pgfpathlineto{\pgfqpoint{1.588172in}{3.954855in}}%
\pgfpathlineto{\pgfqpoint{1.590496in}{3.944391in}}%
\pgfpathlineto{\pgfqpoint{1.592820in}{3.919282in}}%
\pgfpathlineto{\pgfqpoint{1.595144in}{3.947564in}}%
\pgfpathlineto{\pgfqpoint{1.597468in}{3.918977in}}%
\pgfpathlineto{\pgfqpoint{1.599792in}{3.854119in}}%
\pgfpathlineto{\pgfqpoint{1.602116in}{3.897370in}}%
\pgfpathlineto{\pgfqpoint{1.604440in}{3.890449in}}%
\pgfpathlineto{\pgfqpoint{1.606764in}{3.948198in}}%
\pgfpathlineto{\pgfqpoint{1.609088in}{3.950698in}}%
\pgfpathlineto{\pgfqpoint{1.611412in}{3.919064in}}%
\pgfpathlineto{\pgfqpoint{1.613736in}{3.910014in}}%
\pgfpathlineto{\pgfqpoint{1.616060in}{3.937080in}}%
\pgfpathlineto{\pgfqpoint{1.618384in}{3.983678in}}%
\pgfpathlineto{\pgfqpoint{1.620708in}{3.934851in}}%
\pgfpathlineto{\pgfqpoint{1.623032in}{4.009677in}}%
\pgfpathlineto{\pgfqpoint{1.625356in}{3.961489in}}%
\pgfpathlineto{\pgfqpoint{1.627680in}{4.002081in}}%
\pgfpathlineto{\pgfqpoint{1.630004in}{3.947323in}}%
\pgfpathlineto{\pgfqpoint{1.634652in}{4.045075in}}%
\pgfpathlineto{\pgfqpoint{1.636976in}{4.067420in}}%
\pgfpathlineto{\pgfqpoint{1.639300in}{3.961797in}}%
\pgfpathlineto{\pgfqpoint{1.641624in}{3.958882in}}%
\pgfpathlineto{\pgfqpoint{1.643948in}{3.950528in}}%
\pgfpathlineto{\pgfqpoint{1.646272in}{3.965300in}}%
\pgfpathlineto{\pgfqpoint{1.648595in}{3.927323in}}%
\pgfpathlineto{\pgfqpoint{1.650919in}{3.995761in}}%
\pgfpathlineto{\pgfqpoint{1.653243in}{3.995187in}}%
\pgfpathlineto{\pgfqpoint{1.657891in}{4.082797in}}%
\pgfpathlineto{\pgfqpoint{1.660215in}{4.020203in}}%
\pgfpathlineto{\pgfqpoint{1.662539in}{4.002267in}}%
\pgfpathlineto{\pgfqpoint{1.664863in}{4.003147in}}%
\pgfpathlineto{\pgfqpoint{1.667187in}{4.071637in}}%
\pgfpathlineto{\pgfqpoint{1.669511in}{4.030822in}}%
\pgfpathlineto{\pgfqpoint{1.671835in}{4.112523in}}%
\pgfpathlineto{\pgfqpoint{1.674159in}{4.033129in}}%
\pgfpathlineto{\pgfqpoint{1.676483in}{4.022831in}}%
\pgfpathlineto{\pgfqpoint{1.678807in}{4.075599in}}%
\pgfpathlineto{\pgfqpoint{1.681131in}{4.040087in}}%
\pgfpathlineto{\pgfqpoint{1.683455in}{4.112110in}}%
\pgfpathlineto{\pgfqpoint{1.688103in}{4.023502in}}%
\pgfpathlineto{\pgfqpoint{1.690427in}{4.094364in}}%
\pgfpathlineto{\pgfqpoint{1.692751in}{4.023444in}}%
\pgfpathlineto{\pgfqpoint{1.695075in}{4.093702in}}%
\pgfpathlineto{\pgfqpoint{1.697399in}{4.037043in}}%
\pgfpathlineto{\pgfqpoint{1.702047in}{4.043807in}}%
\pgfpathlineto{\pgfqpoint{1.706695in}{4.106792in}}%
\pgfpathlineto{\pgfqpoint{1.709019in}{4.069587in}}%
\pgfpathlineto{\pgfqpoint{1.711343in}{4.075875in}}%
\pgfpathlineto{\pgfqpoint{1.713667in}{4.072080in}}%
\pgfpathlineto{\pgfqpoint{1.715991in}{4.089976in}}%
\pgfpathlineto{\pgfqpoint{1.718314in}{4.077715in}}%
\pgfpathlineto{\pgfqpoint{1.720638in}{4.094761in}}%
\pgfpathlineto{\pgfqpoint{1.722962in}{4.087439in}}%
\pgfpathlineto{\pgfqpoint{1.725286in}{4.087524in}}%
\pgfpathlineto{\pgfqpoint{1.727610in}{4.054863in}}%
\pgfpathlineto{\pgfqpoint{1.729934in}{4.094575in}}%
\pgfpathlineto{\pgfqpoint{1.732258in}{4.113274in}}%
\pgfpathlineto{\pgfqpoint{1.734582in}{4.149319in}}%
\pgfpathlineto{\pgfqpoint{1.736906in}{4.059573in}}%
\pgfpathlineto{\pgfqpoint{1.739230in}{4.073155in}}%
\pgfpathlineto{\pgfqpoint{1.741554in}{4.078296in}}%
\pgfpathlineto{\pgfqpoint{1.743878in}{4.135882in}}%
\pgfpathlineto{\pgfqpoint{1.746202in}{4.047324in}}%
\pgfpathlineto{\pgfqpoint{1.748526in}{4.128711in}}%
\pgfpathlineto{\pgfqpoint{1.750850in}{4.125087in}}%
\pgfpathlineto{\pgfqpoint{1.753174in}{4.115005in}}%
\pgfpathlineto{\pgfqpoint{1.755498in}{4.087041in}}%
\pgfpathlineto{\pgfqpoint{1.757822in}{4.223935in}}%
\pgfpathlineto{\pgfqpoint{1.762470in}{4.131418in}}%
\pgfpathlineto{\pgfqpoint{1.764794in}{4.124211in}}%
\pgfpathlineto{\pgfqpoint{1.767118in}{4.132003in}}%
\pgfpathlineto{\pgfqpoint{1.769442in}{4.146138in}}%
\pgfpathlineto{\pgfqpoint{1.771766in}{4.133025in}}%
\pgfpathlineto{\pgfqpoint{1.774090in}{4.168711in}}%
\pgfpathlineto{\pgfqpoint{1.776414in}{4.111934in}}%
\pgfpathlineto{\pgfqpoint{1.778738in}{4.132533in}}%
\pgfpathlineto{\pgfqpoint{1.781062in}{4.132383in}}%
\pgfpathlineto{\pgfqpoint{1.783386in}{4.119971in}}%
\pgfpathlineto{\pgfqpoint{1.785710in}{4.121885in}}%
\pgfpathlineto{\pgfqpoint{1.788034in}{4.129346in}}%
\pgfpathlineto{\pgfqpoint{1.790357in}{4.180377in}}%
\pgfpathlineto{\pgfqpoint{1.792681in}{4.180762in}}%
\pgfpathlineto{\pgfqpoint{1.795005in}{4.213647in}}%
\pgfpathlineto{\pgfqpoint{1.797329in}{4.153979in}}%
\pgfpathlineto{\pgfqpoint{1.799653in}{4.208648in}}%
\pgfpathlineto{\pgfqpoint{1.801977in}{4.211019in}}%
\pgfpathlineto{\pgfqpoint{1.804301in}{4.255518in}}%
\pgfpathlineto{\pgfqpoint{1.806625in}{4.187199in}}%
\pgfpathlineto{\pgfqpoint{1.808949in}{4.201484in}}%
\pgfpathlineto{\pgfqpoint{1.811273in}{4.160109in}}%
\pgfpathlineto{\pgfqpoint{1.813597in}{4.213160in}}%
\pgfpathlineto{\pgfqpoint{1.815921in}{4.119277in}}%
\pgfpathlineto{\pgfqpoint{1.818245in}{4.215170in}}%
\pgfpathlineto{\pgfqpoint{1.820569in}{4.181602in}}%
\pgfpathlineto{\pgfqpoint{1.822893in}{4.239614in}}%
\pgfpathlineto{\pgfqpoint{1.825217in}{4.153834in}}%
\pgfpathlineto{\pgfqpoint{1.827541in}{4.223761in}}%
\pgfpathlineto{\pgfqpoint{1.829865in}{4.196723in}}%
\pgfpathlineto{\pgfqpoint{1.832189in}{4.202697in}}%
\pgfpathlineto{\pgfqpoint{1.834513in}{4.223981in}}%
\pgfpathlineto{\pgfqpoint{1.836837in}{4.210907in}}%
\pgfpathlineto{\pgfqpoint{1.839161in}{4.160532in}}%
\pgfpathlineto{\pgfqpoint{1.841485in}{4.245292in}}%
\pgfpathlineto{\pgfqpoint{1.843809in}{4.227819in}}%
\pgfpathlineto{\pgfqpoint{1.846133in}{4.227582in}}%
\pgfpathlineto{\pgfqpoint{1.848457in}{4.184087in}}%
\pgfpathlineto{\pgfqpoint{1.850781in}{4.285124in}}%
\pgfpathlineto{\pgfqpoint{1.853105in}{4.235736in}}%
\pgfpathlineto{\pgfqpoint{1.855429in}{4.293158in}}%
\pgfpathlineto{\pgfqpoint{1.857753in}{4.221151in}}%
\pgfpathlineto{\pgfqpoint{1.860076in}{4.243812in}}%
\pgfpathlineto{\pgfqpoint{1.862400in}{4.224408in}}%
\pgfpathlineto{\pgfqpoint{1.864724in}{4.235079in}}%
\pgfpathlineto{\pgfqpoint{1.867048in}{4.277724in}}%
\pgfpathlineto{\pgfqpoint{1.869372in}{4.179645in}}%
\pgfpathlineto{\pgfqpoint{1.871696in}{4.183912in}}%
\pgfpathlineto{\pgfqpoint{1.874020in}{4.213763in}}%
\pgfpathlineto{\pgfqpoint{1.876344in}{4.225028in}}%
\pgfpathlineto{\pgfqpoint{1.878668in}{4.274395in}}%
\pgfpathlineto{\pgfqpoint{1.880992in}{4.288398in}}%
\pgfpathlineto{\pgfqpoint{1.883316in}{4.286795in}}%
\pgfpathlineto{\pgfqpoint{1.885640in}{4.248074in}}%
\pgfpathlineto{\pgfqpoint{1.887964in}{4.325862in}}%
\pgfpathlineto{\pgfqpoint{1.890288in}{4.295304in}}%
\pgfpathlineto{\pgfqpoint{1.892612in}{4.247547in}}%
\pgfpathlineto{\pgfqpoint{1.894936in}{4.291001in}}%
\pgfpathlineto{\pgfqpoint{1.897260in}{4.230041in}}%
\pgfpathlineto{\pgfqpoint{1.901908in}{4.359763in}}%
\pgfpathlineto{\pgfqpoint{1.904232in}{4.296251in}}%
\pgfpathlineto{\pgfqpoint{1.906556in}{4.312587in}}%
\pgfpathlineto{\pgfqpoint{1.908880in}{4.233534in}}%
\pgfpathlineto{\pgfqpoint{1.911204in}{4.328200in}}%
\pgfpathlineto{\pgfqpoint{1.913528in}{4.254604in}}%
\pgfpathlineto{\pgfqpoint{1.915852in}{4.279508in}}%
\pgfpathlineto{\pgfqpoint{1.918176in}{4.255189in}}%
\pgfpathlineto{\pgfqpoint{1.922824in}{4.381007in}}%
\pgfpathlineto{\pgfqpoint{1.925148in}{4.239685in}}%
\pgfpathlineto{\pgfqpoint{1.927472in}{4.284959in}}%
\pgfpathlineto{\pgfqpoint{1.929795in}{4.287953in}}%
\pgfpathlineto{\pgfqpoint{1.932119in}{4.335345in}}%
\pgfpathlineto{\pgfqpoint{1.934443in}{4.352052in}}%
\pgfpathlineto{\pgfqpoint{1.936767in}{4.283088in}}%
\pgfpathlineto{\pgfqpoint{1.939091in}{4.328684in}}%
\pgfpathlineto{\pgfqpoint{1.941415in}{4.299570in}}%
\pgfpathlineto{\pgfqpoint{1.943739in}{4.383823in}}%
\pgfpathlineto{\pgfqpoint{1.948387in}{4.288207in}}%
\pgfpathlineto{\pgfqpoint{1.950711in}{4.334075in}}%
\pgfpathlineto{\pgfqpoint{1.953035in}{4.328006in}}%
\pgfpathlineto{\pgfqpoint{1.957683in}{4.338102in}}%
\pgfpathlineto{\pgfqpoint{1.960007in}{4.369848in}}%
\pgfpathlineto{\pgfqpoint{1.964655in}{4.330994in}}%
\pgfpathlineto{\pgfqpoint{1.966979in}{4.330711in}}%
\pgfpathlineto{\pgfqpoint{1.971627in}{4.387989in}}%
\pgfpathlineto{\pgfqpoint{1.973951in}{4.369486in}}%
\pgfpathlineto{\pgfqpoint{1.976275in}{4.318483in}}%
\pgfpathlineto{\pgfqpoint{1.978599in}{4.414815in}}%
\pgfpathlineto{\pgfqpoint{1.983247in}{4.364645in}}%
\pgfpathlineto{\pgfqpoint{1.985571in}{4.405740in}}%
\pgfpathlineto{\pgfqpoint{1.987895in}{4.468220in}}%
\pgfpathlineto{\pgfqpoint{1.990219in}{4.436425in}}%
\pgfpathlineto{\pgfqpoint{1.992543in}{4.372886in}}%
\pgfpathlineto{\pgfqpoint{1.997191in}{4.420633in}}%
\pgfpathlineto{\pgfqpoint{1.999514in}{4.379027in}}%
\pgfpathlineto{\pgfqpoint{2.001838in}{4.428573in}}%
\pgfpathlineto{\pgfqpoint{2.004162in}{4.412324in}}%
\pgfpathlineto{\pgfqpoint{2.006486in}{4.427663in}}%
\pgfpathlineto{\pgfqpoint{2.008810in}{4.388756in}}%
\pgfpathlineto{\pgfqpoint{2.011134in}{4.412800in}}%
\pgfpathlineto{\pgfqpoint{2.013458in}{4.422857in}}%
\pgfpathlineto{\pgfqpoint{2.015782in}{4.412378in}}%
\pgfpathlineto{\pgfqpoint{2.020430in}{4.452339in}}%
\pgfpathlineto{\pgfqpoint{2.027402in}{4.380777in}}%
\pgfpathlineto{\pgfqpoint{2.029726in}{4.411228in}}%
\pgfpathlineto{\pgfqpoint{2.032050in}{4.512118in}}%
\pgfpathlineto{\pgfqpoint{2.034374in}{4.384937in}}%
\pgfpathlineto{\pgfqpoint{2.036698in}{4.421620in}}%
\pgfpathlineto{\pgfqpoint{2.039022in}{4.392296in}}%
\pgfpathlineto{\pgfqpoint{2.043670in}{4.439821in}}%
\pgfpathlineto{\pgfqpoint{2.045994in}{4.456464in}}%
\pgfpathlineto{\pgfqpoint{2.050642in}{4.406883in}}%
\pgfpathlineto{\pgfqpoint{2.052966in}{4.485867in}}%
\pgfpathlineto{\pgfqpoint{2.055290in}{4.471024in}}%
\pgfpathlineto{\pgfqpoint{2.057614in}{4.427950in}}%
\pgfpathlineto{\pgfqpoint{2.059938in}{4.480801in}}%
\pgfpathlineto{\pgfqpoint{2.062262in}{4.494620in}}%
\pgfpathlineto{\pgfqpoint{2.064586in}{4.474532in}}%
\pgfpathlineto{\pgfqpoint{2.066910in}{4.492338in}}%
\pgfpathlineto{\pgfqpoint{2.069234in}{4.440282in}}%
\pgfpathlineto{\pgfqpoint{2.071557in}{4.440831in}}%
\pgfpathlineto{\pgfqpoint{2.073881in}{4.458097in}}%
\pgfpathlineto{\pgfqpoint{2.076205in}{4.459733in}}%
\pgfpathlineto{\pgfqpoint{2.078529in}{4.482848in}}%
\pgfpathlineto{\pgfqpoint{2.080853in}{4.477684in}}%
\pgfpathlineto{\pgfqpoint{2.087825in}{4.511076in}}%
\pgfpathlineto{\pgfqpoint{2.090149in}{4.511022in}}%
\pgfpathlineto{\pgfqpoint{2.092473in}{4.544416in}}%
\pgfpathlineto{\pgfqpoint{2.094797in}{4.495902in}}%
\pgfpathlineto{\pgfqpoint{2.097121in}{4.486500in}}%
\pgfpathlineto{\pgfqpoint{2.099445in}{4.551373in}}%
\pgfpathlineto{\pgfqpoint{2.101769in}{4.474815in}}%
\pgfpathlineto{\pgfqpoint{2.104093in}{3.848489in}}%
\pgfpathlineto{\pgfqpoint{2.106417in}{3.825572in}}%
\pgfpathlineto{\pgfqpoint{2.108741in}{3.864317in}}%
\pgfpathlineto{\pgfqpoint{2.111065in}{3.832199in}}%
\pgfpathlineto{\pgfqpoint{2.113389in}{3.819842in}}%
\pgfpathlineto{\pgfqpoint{2.115713in}{3.869211in}}%
\pgfpathlineto{\pgfqpoint{2.118037in}{3.865336in}}%
\pgfpathlineto{\pgfqpoint{2.120361in}{3.865692in}}%
\pgfpathlineto{\pgfqpoint{2.122685in}{3.947389in}}%
\pgfpathlineto{\pgfqpoint{2.125009in}{3.821634in}}%
\pgfpathlineto{\pgfqpoint{2.127333in}{3.906845in}}%
\pgfpathlineto{\pgfqpoint{2.129657in}{3.872276in}}%
\pgfpathlineto{\pgfqpoint{2.131981in}{3.867361in}}%
\pgfpathlineto{\pgfqpoint{2.134305in}{3.917180in}}%
\pgfpathlineto{\pgfqpoint{2.136629in}{3.829186in}}%
\pgfpathlineto{\pgfqpoint{2.141276in}{3.932779in}}%
\pgfpathlineto{\pgfqpoint{2.143600in}{3.918699in}}%
\pgfpathlineto{\pgfqpoint{2.145924in}{3.951201in}}%
\pgfpathlineto{\pgfqpoint{2.148248in}{3.881510in}}%
\pgfpathlineto{\pgfqpoint{2.150572in}{3.884144in}}%
\pgfpathlineto{\pgfqpoint{2.155220in}{3.913291in}}%
\pgfpathlineto{\pgfqpoint{2.157544in}{3.879903in}}%
\pgfpathlineto{\pgfqpoint{2.159868in}{3.870614in}}%
\pgfpathlineto{\pgfqpoint{2.162192in}{3.918277in}}%
\pgfpathlineto{\pgfqpoint{2.164516in}{3.910828in}}%
\pgfpathlineto{\pgfqpoint{2.166840in}{3.940838in}}%
\pgfpathlineto{\pgfqpoint{2.169164in}{3.942297in}}%
\pgfpathlineto{\pgfqpoint{2.171488in}{3.959787in}}%
\pgfpathlineto{\pgfqpoint{2.173812in}{3.930661in}}%
\pgfpathlineto{\pgfqpoint{2.178460in}{3.983757in}}%
\pgfpathlineto{\pgfqpoint{2.183108in}{3.915266in}}%
\pgfpathlineto{\pgfqpoint{2.185432in}{3.986682in}}%
\pgfpathlineto{\pgfqpoint{2.187756in}{3.899059in}}%
\pgfpathlineto{\pgfqpoint{2.190080in}{3.890964in}}%
\pgfpathlineto{\pgfqpoint{2.194728in}{3.988934in}}%
\pgfpathlineto{\pgfqpoint{2.197052in}{3.921283in}}%
\pgfpathlineto{\pgfqpoint{2.201700in}{3.990788in}}%
\pgfpathlineto{\pgfqpoint{2.204024in}{3.971364in}}%
\pgfpathlineto{\pgfqpoint{2.206348in}{3.976052in}}%
\pgfpathlineto{\pgfqpoint{2.208672in}{3.942784in}}%
\pgfpathlineto{\pgfqpoint{2.210995in}{3.942541in}}%
\pgfpathlineto{\pgfqpoint{2.213319in}{3.975173in}}%
\pgfpathlineto{\pgfqpoint{2.215643in}{4.043383in}}%
\pgfpathlineto{\pgfqpoint{2.217967in}{3.988831in}}%
\pgfpathlineto{\pgfqpoint{2.220291in}{3.978534in}}%
\pgfpathlineto{\pgfqpoint{2.222615in}{3.959759in}}%
\pgfpathlineto{\pgfqpoint{2.224939in}{3.959903in}}%
\pgfpathlineto{\pgfqpoint{2.227263in}{4.003173in}}%
\pgfpathlineto{\pgfqpoint{2.229587in}{3.936602in}}%
\pgfpathlineto{\pgfqpoint{2.231911in}{4.000648in}}%
\pgfpathlineto{\pgfqpoint{2.234235in}{3.997854in}}%
\pgfpathlineto{\pgfqpoint{2.236559in}{3.946057in}}%
\pgfpathlineto{\pgfqpoint{2.238883in}{4.039201in}}%
\pgfpathlineto{\pgfqpoint{2.241207in}{4.080207in}}%
\pgfpathlineto{\pgfqpoint{2.243531in}{4.028946in}}%
\pgfpathlineto{\pgfqpoint{2.245855in}{4.046271in}}%
\pgfpathlineto{\pgfqpoint{2.250503in}{3.966065in}}%
\pgfpathlineto{\pgfqpoint{2.252827in}{4.037181in}}%
\pgfpathlineto{\pgfqpoint{2.255151in}{4.002985in}}%
\pgfpathlineto{\pgfqpoint{2.257475in}{4.047695in}}%
\pgfpathlineto{\pgfqpoint{2.259799in}{4.057554in}}%
\pgfpathlineto{\pgfqpoint{2.262123in}{4.099736in}}%
\pgfpathlineto{\pgfqpoint{2.264447in}{3.979033in}}%
\pgfpathlineto{\pgfqpoint{2.266771in}{4.027707in}}%
\pgfpathlineto{\pgfqpoint{2.269095in}{4.012611in}}%
\pgfpathlineto{\pgfqpoint{2.271419in}{4.008135in}}%
\pgfpathlineto{\pgfqpoint{2.273743in}{4.017773in}}%
\pgfpathlineto{\pgfqpoint{2.276067in}{4.017093in}}%
\pgfpathlineto{\pgfqpoint{2.278391in}{4.070590in}}%
\pgfpathlineto{\pgfqpoint{2.280715in}{4.040193in}}%
\pgfpathlineto{\pgfqpoint{2.283038in}{4.138273in}}%
\pgfpathlineto{\pgfqpoint{2.285362in}{4.119282in}}%
\pgfpathlineto{\pgfqpoint{2.290010in}{4.034341in}}%
\pgfpathlineto{\pgfqpoint{2.292334in}{4.086610in}}%
\pgfpathlineto{\pgfqpoint{2.294658in}{4.008563in}}%
\pgfpathlineto{\pgfqpoint{2.296982in}{4.074803in}}%
\pgfpathlineto{\pgfqpoint{2.299306in}{4.028626in}}%
\pgfpathlineto{\pgfqpoint{2.301630in}{4.065945in}}%
\pgfpathlineto{\pgfqpoint{2.303954in}{4.050611in}}%
\pgfpathlineto{\pgfqpoint{2.306278in}{4.023158in}}%
\pgfpathlineto{\pgfqpoint{2.308602in}{4.128012in}}%
\pgfpathlineto{\pgfqpoint{2.310926in}{4.060877in}}%
\pgfpathlineto{\pgfqpoint{2.313250in}{4.057209in}}%
\pgfpathlineto{\pgfqpoint{2.315574in}{4.093344in}}%
\pgfpathlineto{\pgfqpoint{2.317898in}{4.108966in}}%
\pgfpathlineto{\pgfqpoint{2.320222in}{4.099872in}}%
\pgfpathlineto{\pgfqpoint{2.322546in}{4.096570in}}%
\pgfpathlineto{\pgfqpoint{2.324870in}{4.023155in}}%
\pgfpathlineto{\pgfqpoint{2.327194in}{4.045381in}}%
\pgfpathlineto{\pgfqpoint{2.329518in}{4.056390in}}%
\pgfpathlineto{\pgfqpoint{2.331842in}{4.093234in}}%
\pgfpathlineto{\pgfqpoint{2.334166in}{4.098694in}}%
\pgfpathlineto{\pgfqpoint{2.336490in}{4.107673in}}%
\pgfpathlineto{\pgfqpoint{2.338814in}{4.105370in}}%
\pgfpathlineto{\pgfqpoint{2.341138in}{4.117993in}}%
\pgfpathlineto{\pgfqpoint{2.343462in}{4.074953in}}%
\pgfpathlineto{\pgfqpoint{2.345786in}{4.103337in}}%
\pgfpathlineto{\pgfqpoint{2.348110in}{4.216661in}}%
\pgfpathlineto{\pgfqpoint{2.350434in}{4.106197in}}%
\pgfpathlineto{\pgfqpoint{2.352757in}{4.162653in}}%
\pgfpathlineto{\pgfqpoint{2.355081in}{4.090098in}}%
\pgfpathlineto{\pgfqpoint{2.357405in}{4.191245in}}%
\pgfpathlineto{\pgfqpoint{2.359729in}{4.108620in}}%
\pgfpathlineto{\pgfqpoint{2.362053in}{4.133740in}}%
\pgfpathlineto{\pgfqpoint{2.364377in}{4.143905in}}%
\pgfpathlineto{\pgfqpoint{2.369025in}{4.186601in}}%
\pgfpathlineto{\pgfqpoint{2.371349in}{4.147583in}}%
\pgfpathlineto{\pgfqpoint{2.373673in}{4.215420in}}%
\pgfpathlineto{\pgfqpoint{2.378321in}{4.150417in}}%
\pgfpathlineto{\pgfqpoint{2.382969in}{4.149668in}}%
\pgfpathlineto{\pgfqpoint{2.385293in}{4.198077in}}%
\pgfpathlineto{\pgfqpoint{2.387617in}{4.152913in}}%
\pgfpathlineto{\pgfqpoint{2.389941in}{4.201461in}}%
\pgfpathlineto{\pgfqpoint{2.392265in}{4.126110in}}%
\pgfpathlineto{\pgfqpoint{2.394589in}{4.182457in}}%
\pgfpathlineto{\pgfqpoint{2.396913in}{4.203595in}}%
\pgfpathlineto{\pgfqpoint{2.399237in}{4.203667in}}%
\pgfpathlineto{\pgfqpoint{2.401561in}{4.196601in}}%
\pgfpathlineto{\pgfqpoint{2.403885in}{4.180410in}}%
\pgfpathlineto{\pgfqpoint{2.406209in}{4.255569in}}%
\pgfpathlineto{\pgfqpoint{2.410857in}{4.147050in}}%
\pgfpathlineto{\pgfqpoint{2.413181in}{4.203142in}}%
\pgfpathlineto{\pgfqpoint{2.415505in}{4.166223in}}%
\pgfpathlineto{\pgfqpoint{2.417829in}{4.211681in}}%
\pgfpathlineto{\pgfqpoint{2.420153in}{4.202834in}}%
\pgfpathlineto{\pgfqpoint{2.422476in}{4.216026in}}%
\pgfpathlineto{\pgfqpoint{2.424800in}{4.219179in}}%
\pgfpathlineto{\pgfqpoint{2.427124in}{4.235170in}}%
\pgfpathlineto{\pgfqpoint{2.429448in}{4.168420in}}%
\pgfpathlineto{\pgfqpoint{2.431772in}{4.208901in}}%
\pgfpathlineto{\pgfqpoint{2.434096in}{4.295128in}}%
\pgfpathlineto{\pgfqpoint{2.436420in}{4.249038in}}%
\pgfpathlineto{\pgfqpoint{2.438744in}{4.252126in}}%
\pgfpathlineto{\pgfqpoint{2.441068in}{4.220413in}}%
\pgfpathlineto{\pgfqpoint{2.443392in}{4.211754in}}%
\pgfpathlineto{\pgfqpoint{2.445716in}{4.256206in}}%
\pgfpathlineto{\pgfqpoint{2.448040in}{4.263385in}}%
\pgfpathlineto{\pgfqpoint{2.450364in}{4.213705in}}%
\pgfpathlineto{\pgfqpoint{2.452688in}{4.229597in}}%
\pgfpathlineto{\pgfqpoint{2.455012in}{4.230136in}}%
\pgfpathlineto{\pgfqpoint{2.457336in}{4.213394in}}%
\pgfpathlineto{\pgfqpoint{2.459660in}{4.250122in}}%
\pgfpathlineto{\pgfqpoint{2.461984in}{4.325332in}}%
\pgfpathlineto{\pgfqpoint{2.464308in}{4.262904in}}%
\pgfpathlineto{\pgfqpoint{2.466632in}{4.263714in}}%
\pgfpathlineto{\pgfqpoint{2.468956in}{4.268155in}}%
\pgfpathlineto{\pgfqpoint{2.471280in}{4.269555in}}%
\pgfpathlineto{\pgfqpoint{2.473604in}{4.246152in}}%
\pgfpathlineto{\pgfqpoint{2.475928in}{4.275549in}}%
\pgfpathlineto{\pgfqpoint{2.478252in}{4.258394in}}%
\pgfpathlineto{\pgfqpoint{2.480576in}{4.279473in}}%
\pgfpathlineto{\pgfqpoint{2.482900in}{4.327785in}}%
\pgfpathlineto{\pgfqpoint{2.485224in}{4.275398in}}%
\pgfpathlineto{\pgfqpoint{2.487548in}{4.335018in}}%
\pgfpathlineto{\pgfqpoint{2.489872in}{4.248069in}}%
\pgfpathlineto{\pgfqpoint{2.492195in}{4.295537in}}%
\pgfpathlineto{\pgfqpoint{2.494519in}{4.273409in}}%
\pgfpathlineto{\pgfqpoint{2.496843in}{4.322540in}}%
\pgfpathlineto{\pgfqpoint{2.499167in}{4.269251in}}%
\pgfpathlineto{\pgfqpoint{2.501491in}{4.313660in}}%
\pgfpathlineto{\pgfqpoint{2.503815in}{4.321197in}}%
\pgfpathlineto{\pgfqpoint{2.506139in}{4.246665in}}%
\pgfpathlineto{\pgfqpoint{2.508463in}{4.351944in}}%
\pgfpathlineto{\pgfqpoint{2.510787in}{4.310296in}}%
\pgfpathlineto{\pgfqpoint{2.513111in}{4.314030in}}%
\pgfpathlineto{\pgfqpoint{2.515435in}{4.320254in}}%
\pgfpathlineto{\pgfqpoint{2.520083in}{4.364816in}}%
\pgfpathlineto{\pgfqpoint{2.522407in}{4.333028in}}%
\pgfpathlineto{\pgfqpoint{2.524731in}{4.338954in}}%
\pgfpathlineto{\pgfqpoint{2.527055in}{4.375714in}}%
\pgfpathlineto{\pgfqpoint{2.529379in}{4.274960in}}%
\pgfpathlineto{\pgfqpoint{2.531703in}{4.367966in}}%
\pgfpathlineto{\pgfqpoint{2.534027in}{4.307966in}}%
\pgfpathlineto{\pgfqpoint{2.536351in}{4.363519in}}%
\pgfpathlineto{\pgfqpoint{2.538675in}{4.379125in}}%
\pgfpathlineto{\pgfqpoint{2.540999in}{4.337789in}}%
\pgfpathlineto{\pgfqpoint{2.543323in}{4.338625in}}%
\pgfpathlineto{\pgfqpoint{2.547971in}{4.375246in}}%
\pgfpathlineto{\pgfqpoint{2.550295in}{4.376785in}}%
\pgfpathlineto{\pgfqpoint{2.552619in}{4.373510in}}%
\pgfpathlineto{\pgfqpoint{2.554943in}{4.331462in}}%
\pgfpathlineto{\pgfqpoint{2.557267in}{4.342725in}}%
\pgfpathlineto{\pgfqpoint{2.559591in}{4.315654in}}%
\pgfpathlineto{\pgfqpoint{2.561915in}{4.372828in}}%
\pgfpathlineto{\pgfqpoint{2.564238in}{4.322867in}}%
\pgfpathlineto{\pgfqpoint{2.566562in}{4.397464in}}%
\pgfpathlineto{\pgfqpoint{2.568886in}{4.393613in}}%
\pgfpathlineto{\pgfqpoint{2.571210in}{4.433828in}}%
\pgfpathlineto{\pgfqpoint{2.573534in}{4.388963in}}%
\pgfpathlineto{\pgfqpoint{2.575858in}{4.445695in}}%
\pgfpathlineto{\pgfqpoint{2.578182in}{4.339645in}}%
\pgfpathlineto{\pgfqpoint{2.580506in}{4.390159in}}%
\pgfpathlineto{\pgfqpoint{2.582830in}{4.413791in}}%
\pgfpathlineto{\pgfqpoint{2.585154in}{4.395056in}}%
\pgfpathlineto{\pgfqpoint{2.587478in}{4.416148in}}%
\pgfpathlineto{\pgfqpoint{2.592126in}{4.396608in}}%
\pgfpathlineto{\pgfqpoint{2.594450in}{4.401035in}}%
\pgfpathlineto{\pgfqpoint{2.596774in}{4.446749in}}%
\pgfpathlineto{\pgfqpoint{2.599098in}{4.406409in}}%
\pgfpathlineto{\pgfqpoint{2.601422in}{4.405849in}}%
\pgfpathlineto{\pgfqpoint{2.603746in}{4.417197in}}%
\pgfpathlineto{\pgfqpoint{2.608394in}{4.371606in}}%
\pgfpathlineto{\pgfqpoint{2.610718in}{4.375752in}}%
\pgfpathlineto{\pgfqpoint{2.613042in}{4.405947in}}%
\pgfpathlineto{\pgfqpoint{2.615366in}{4.411523in}}%
\pgfpathlineto{\pgfqpoint{2.617690in}{4.440235in}}%
\pgfpathlineto{\pgfqpoint{2.620014in}{4.410365in}}%
\pgfpathlineto{\pgfqpoint{2.622338in}{4.450393in}}%
\pgfpathlineto{\pgfqpoint{2.624662in}{4.414528in}}%
\pgfpathlineto{\pgfqpoint{2.626986in}{4.483487in}}%
\pgfpathlineto{\pgfqpoint{2.629310in}{4.487375in}}%
\pgfpathlineto{\pgfqpoint{2.631634in}{4.405504in}}%
\pgfpathlineto{\pgfqpoint{2.633957in}{4.496584in}}%
\pgfpathlineto{\pgfqpoint{2.636281in}{4.423717in}}%
\pgfpathlineto{\pgfqpoint{2.638605in}{4.504960in}}%
\pgfpathlineto{\pgfqpoint{2.640929in}{4.442435in}}%
\pgfpathlineto{\pgfqpoint{2.643253in}{4.523146in}}%
\pgfpathlineto{\pgfqpoint{2.645577in}{4.440051in}}%
\pgfpathlineto{\pgfqpoint{2.650225in}{4.498684in}}%
\pgfpathlineto{\pgfqpoint{2.652549in}{4.490441in}}%
\pgfpathlineto{\pgfqpoint{2.657197in}{4.410923in}}%
\pgfpathlineto{\pgfqpoint{2.659521in}{4.469565in}}%
\pgfpathlineto{\pgfqpoint{2.661845in}{4.455126in}}%
\pgfpathlineto{\pgfqpoint{2.664169in}{4.469939in}}%
\pgfpathlineto{\pgfqpoint{2.666493in}{4.475316in}}%
\pgfpathlineto{\pgfqpoint{2.668817in}{4.531382in}}%
\pgfpathlineto{\pgfqpoint{2.671141in}{4.544148in}}%
\pgfpathlineto{\pgfqpoint{2.673465in}{4.506280in}}%
\pgfpathlineto{\pgfqpoint{2.675789in}{4.514940in}}%
\pgfpathlineto{\pgfqpoint{2.678113in}{4.556936in}}%
\pgfpathlineto{\pgfqpoint{2.680437in}{4.475644in}}%
\pgfpathlineto{\pgfqpoint{2.682761in}{4.507273in}}%
\pgfpathlineto{\pgfqpoint{2.685085in}{3.891298in}}%
\pgfpathlineto{\pgfqpoint{2.687409in}{3.770305in}}%
\pgfpathlineto{\pgfqpoint{2.689733in}{3.841997in}}%
\pgfpathlineto{\pgfqpoint{2.692057in}{3.877513in}}%
\pgfpathlineto{\pgfqpoint{2.694381in}{3.881818in}}%
\pgfpathlineto{\pgfqpoint{2.696705in}{3.859841in}}%
\pgfpathlineto{\pgfqpoint{2.699029in}{3.868614in}}%
\pgfpathlineto{\pgfqpoint{2.703676in}{3.907427in}}%
\pgfpathlineto{\pgfqpoint{2.708324in}{3.886081in}}%
\pgfpathlineto{\pgfqpoint{2.710648in}{3.889824in}}%
\pgfpathlineto{\pgfqpoint{2.715296in}{3.874745in}}%
\pgfpathlineto{\pgfqpoint{2.717620in}{3.882479in}}%
\pgfpathlineto{\pgfqpoint{2.719944in}{3.929892in}}%
\pgfpathlineto{\pgfqpoint{2.722268in}{3.868994in}}%
\pgfpathlineto{\pgfqpoint{2.724592in}{3.941930in}}%
\pgfpathlineto{\pgfqpoint{2.726916in}{3.898898in}}%
\pgfpathlineto{\pgfqpoint{2.729240in}{3.959692in}}%
\pgfpathlineto{\pgfqpoint{2.731564in}{3.871832in}}%
\pgfpathlineto{\pgfqpoint{2.733888in}{3.919144in}}%
\pgfpathlineto{\pgfqpoint{2.736212in}{3.883144in}}%
\pgfpathlineto{\pgfqpoint{2.738536in}{3.921535in}}%
\pgfpathlineto{\pgfqpoint{2.740860in}{3.933084in}}%
\pgfpathlineto{\pgfqpoint{2.743184in}{3.978408in}}%
\pgfpathlineto{\pgfqpoint{2.745508in}{3.935102in}}%
\pgfpathlineto{\pgfqpoint{2.747832in}{3.940730in}}%
\pgfpathlineto{\pgfqpoint{2.750156in}{3.896550in}}%
\pgfpathlineto{\pgfqpoint{2.754804in}{3.932187in}}%
\pgfpathlineto{\pgfqpoint{2.757128in}{3.930884in}}%
\pgfpathlineto{\pgfqpoint{2.759452in}{3.984433in}}%
\pgfpathlineto{\pgfqpoint{2.764100in}{3.903596in}}%
\pgfpathlineto{\pgfqpoint{2.766424in}{3.928145in}}%
\pgfpathlineto{\pgfqpoint{2.768748in}{3.894181in}}%
\pgfpathlineto{\pgfqpoint{2.771072in}{3.951208in}}%
\pgfpathlineto{\pgfqpoint{2.773395in}{3.905119in}}%
\pgfpathlineto{\pgfqpoint{2.778043in}{3.977265in}}%
\pgfpathlineto{\pgfqpoint{2.780367in}{3.974686in}}%
\pgfpathlineto{\pgfqpoint{2.782691in}{3.993575in}}%
\pgfpathlineto{\pgfqpoint{2.785015in}{3.952349in}}%
\pgfpathlineto{\pgfqpoint{2.787339in}{4.018697in}}%
\pgfpathlineto{\pgfqpoint{2.789663in}{3.917934in}}%
\pgfpathlineto{\pgfqpoint{2.791987in}{4.002785in}}%
\pgfpathlineto{\pgfqpoint{2.794311in}{4.012675in}}%
\pgfpathlineto{\pgfqpoint{2.796635in}{3.998723in}}%
\pgfpathlineto{\pgfqpoint{2.798959in}{3.962672in}}%
\pgfpathlineto{\pgfqpoint{2.801283in}{3.972674in}}%
\pgfpathlineto{\pgfqpoint{2.805931in}{4.018488in}}%
\pgfpathlineto{\pgfqpoint{2.808255in}{3.968971in}}%
\pgfpathlineto{\pgfqpoint{2.810579in}{3.975497in}}%
\pgfpathlineto{\pgfqpoint{2.812903in}{4.062716in}}%
\pgfpathlineto{\pgfqpoint{2.815227in}{3.967228in}}%
\pgfpathlineto{\pgfqpoint{2.817551in}{3.968573in}}%
\pgfpathlineto{\pgfqpoint{2.822199in}{4.013585in}}%
\pgfpathlineto{\pgfqpoint{2.824523in}{3.989150in}}%
\pgfpathlineto{\pgfqpoint{2.826847in}{3.997365in}}%
\pgfpathlineto{\pgfqpoint{2.829171in}{3.980680in}}%
\pgfpathlineto{\pgfqpoint{2.831495in}{4.035812in}}%
\pgfpathlineto{\pgfqpoint{2.833819in}{4.032601in}}%
\pgfpathlineto{\pgfqpoint{2.836143in}{4.054173in}}%
\pgfpathlineto{\pgfqpoint{2.838467in}{4.061482in}}%
\pgfpathlineto{\pgfqpoint{2.840791in}{4.059728in}}%
\pgfpathlineto{\pgfqpoint{2.843115in}{4.014515in}}%
\pgfpathlineto{\pgfqpoint{2.845438in}{4.048017in}}%
\pgfpathlineto{\pgfqpoint{2.847762in}{4.050517in}}%
\pgfpathlineto{\pgfqpoint{2.850086in}{4.080940in}}%
\pgfpathlineto{\pgfqpoint{2.852410in}{4.075297in}}%
\pgfpathlineto{\pgfqpoint{2.854734in}{4.003018in}}%
\pgfpathlineto{\pgfqpoint{2.859382in}{4.037409in}}%
\pgfpathlineto{\pgfqpoint{2.861706in}{4.024474in}}%
\pgfpathlineto{\pgfqpoint{2.864030in}{4.061005in}}%
\pgfpathlineto{\pgfqpoint{2.866354in}{4.035611in}}%
\pgfpathlineto{\pgfqpoint{2.868678in}{4.092641in}}%
\pgfpathlineto{\pgfqpoint{2.871002in}{4.079981in}}%
\pgfpathlineto{\pgfqpoint{2.873326in}{4.056986in}}%
\pgfpathlineto{\pgfqpoint{2.875650in}{4.107164in}}%
\pgfpathlineto{\pgfqpoint{2.877974in}{4.018776in}}%
\pgfpathlineto{\pgfqpoint{2.880298in}{4.049297in}}%
\pgfpathlineto{\pgfqpoint{2.882622in}{4.116464in}}%
\pgfpathlineto{\pgfqpoint{2.889594in}{4.049261in}}%
\pgfpathlineto{\pgfqpoint{2.891918in}{4.070804in}}%
\pgfpathlineto{\pgfqpoint{2.894242in}{4.107459in}}%
\pgfpathlineto{\pgfqpoint{2.896566in}{4.071633in}}%
\pgfpathlineto{\pgfqpoint{2.898890in}{4.094782in}}%
\pgfpathlineto{\pgfqpoint{2.901214in}{4.096403in}}%
\pgfpathlineto{\pgfqpoint{2.903538in}{4.063356in}}%
\pgfpathlineto{\pgfqpoint{2.905862in}{4.112546in}}%
\pgfpathlineto{\pgfqpoint{2.908186in}{4.068867in}}%
\pgfpathlineto{\pgfqpoint{2.910510in}{4.151677in}}%
\pgfpathlineto{\pgfqpoint{2.912834in}{4.104670in}}%
\pgfpathlineto{\pgfqpoint{2.915157in}{4.096237in}}%
\pgfpathlineto{\pgfqpoint{2.917481in}{4.066060in}}%
\pgfpathlineto{\pgfqpoint{2.919805in}{4.140934in}}%
\pgfpathlineto{\pgfqpoint{2.922129in}{4.103026in}}%
\pgfpathlineto{\pgfqpoint{2.924453in}{4.139453in}}%
\pgfpathlineto{\pgfqpoint{2.926777in}{4.085117in}}%
\pgfpathlineto{\pgfqpoint{2.929101in}{4.157405in}}%
\pgfpathlineto{\pgfqpoint{2.931425in}{4.150056in}}%
\pgfpathlineto{\pgfqpoint{2.933749in}{4.156173in}}%
\pgfpathlineto{\pgfqpoint{2.936073in}{4.154289in}}%
\pgfpathlineto{\pgfqpoint{2.938397in}{4.099183in}}%
\pgfpathlineto{\pgfqpoint{2.940721in}{4.079124in}}%
\pgfpathlineto{\pgfqpoint{2.943045in}{4.154409in}}%
\pgfpathlineto{\pgfqpoint{2.945369in}{4.182172in}}%
\pgfpathlineto{\pgfqpoint{2.947693in}{4.182709in}}%
\pgfpathlineto{\pgfqpoint{2.950017in}{4.108101in}}%
\pgfpathlineto{\pgfqpoint{2.952341in}{4.153765in}}%
\pgfpathlineto{\pgfqpoint{2.954665in}{4.121508in}}%
\pgfpathlineto{\pgfqpoint{2.956989in}{4.167918in}}%
\pgfpathlineto{\pgfqpoint{2.959313in}{4.135382in}}%
\pgfpathlineto{\pgfqpoint{2.961637in}{4.131257in}}%
\pgfpathlineto{\pgfqpoint{2.966285in}{4.220622in}}%
\pgfpathlineto{\pgfqpoint{2.968609in}{4.149658in}}%
\pgfpathlineto{\pgfqpoint{2.970933in}{4.202252in}}%
\pgfpathlineto{\pgfqpoint{2.973257in}{4.224800in}}%
\pgfpathlineto{\pgfqpoint{2.975581in}{4.203198in}}%
\pgfpathlineto{\pgfqpoint{2.977905in}{4.201734in}}%
\pgfpathlineto{\pgfqpoint{2.980229in}{4.198681in}}%
\pgfpathlineto{\pgfqpoint{2.982553in}{4.165479in}}%
\pgfpathlineto{\pgfqpoint{2.984876in}{4.202628in}}%
\pgfpathlineto{\pgfqpoint{2.987200in}{4.216194in}}%
\pgfpathlineto{\pgfqpoint{2.989524in}{4.139521in}}%
\pgfpathlineto{\pgfqpoint{2.994172in}{4.214830in}}%
\pgfpathlineto{\pgfqpoint{2.996496in}{4.205277in}}%
\pgfpathlineto{\pgfqpoint{2.998820in}{4.236791in}}%
\pgfpathlineto{\pgfqpoint{3.001144in}{4.226750in}}%
\pgfpathlineto{\pgfqpoint{3.003468in}{4.276372in}}%
\pgfpathlineto{\pgfqpoint{3.005792in}{4.232719in}}%
\pgfpathlineto{\pgfqpoint{3.008116in}{4.241792in}}%
\pgfpathlineto{\pgfqpoint{3.010440in}{4.270990in}}%
\pgfpathlineto{\pgfqpoint{3.012764in}{4.260961in}}%
\pgfpathlineto{\pgfqpoint{3.015088in}{4.188146in}}%
\pgfpathlineto{\pgfqpoint{3.017412in}{4.252761in}}%
\pgfpathlineto{\pgfqpoint{3.019736in}{4.226387in}}%
\pgfpathlineto{\pgfqpoint{3.022060in}{4.224846in}}%
\pgfpathlineto{\pgfqpoint{3.024384in}{4.253159in}}%
\pgfpathlineto{\pgfqpoint{3.026708in}{4.155811in}}%
\pgfpathlineto{\pgfqpoint{3.031356in}{4.289835in}}%
\pgfpathlineto{\pgfqpoint{3.033680in}{4.173912in}}%
\pgfpathlineto{\pgfqpoint{3.036004in}{4.246849in}}%
\pgfpathlineto{\pgfqpoint{3.038328in}{4.230362in}}%
\pgfpathlineto{\pgfqpoint{3.040652in}{4.248855in}}%
\pgfpathlineto{\pgfqpoint{3.042976in}{4.276634in}}%
\pgfpathlineto{\pgfqpoint{3.045300in}{4.221975in}}%
\pgfpathlineto{\pgfqpoint{3.047624in}{4.283306in}}%
\pgfpathlineto{\pgfqpoint{3.049948in}{4.270742in}}%
\pgfpathlineto{\pgfqpoint{3.054596in}{4.260110in}}%
\pgfpathlineto{\pgfqpoint{3.056919in}{4.284420in}}%
\pgfpathlineto{\pgfqpoint{3.059243in}{4.275731in}}%
\pgfpathlineto{\pgfqpoint{3.061567in}{4.252318in}}%
\pgfpathlineto{\pgfqpoint{3.063891in}{4.339832in}}%
\pgfpathlineto{\pgfqpoint{3.066215in}{4.292181in}}%
\pgfpathlineto{\pgfqpoint{3.068539in}{4.294314in}}%
\pgfpathlineto{\pgfqpoint{3.070863in}{4.225037in}}%
\pgfpathlineto{\pgfqpoint{3.073187in}{4.307192in}}%
\pgfpathlineto{\pgfqpoint{3.075511in}{4.283902in}}%
\pgfpathlineto{\pgfqpoint{3.077835in}{4.327295in}}%
\pgfpathlineto{\pgfqpoint{3.080159in}{4.284865in}}%
\pgfpathlineto{\pgfqpoint{3.082483in}{4.267708in}}%
\pgfpathlineto{\pgfqpoint{3.084807in}{4.265333in}}%
\pgfpathlineto{\pgfqpoint{3.087131in}{4.346614in}}%
\pgfpathlineto{\pgfqpoint{3.089455in}{4.337924in}}%
\pgfpathlineto{\pgfqpoint{3.091779in}{4.266926in}}%
\pgfpathlineto{\pgfqpoint{3.094103in}{4.337194in}}%
\pgfpathlineto{\pgfqpoint{3.096427in}{4.341556in}}%
\pgfpathlineto{\pgfqpoint{3.098751in}{4.356859in}}%
\pgfpathlineto{\pgfqpoint{3.101075in}{4.361775in}}%
\pgfpathlineto{\pgfqpoint{3.103399in}{4.295701in}}%
\pgfpathlineto{\pgfqpoint{3.105723in}{4.332809in}}%
\pgfpathlineto{\pgfqpoint{3.108047in}{4.344602in}}%
\pgfpathlineto{\pgfqpoint{3.110371in}{4.304705in}}%
\pgfpathlineto{\pgfqpoint{3.112695in}{4.353543in}}%
\pgfpathlineto{\pgfqpoint{3.115019in}{4.264190in}}%
\pgfpathlineto{\pgfqpoint{3.117343in}{4.394746in}}%
\pgfpathlineto{\pgfqpoint{3.119667in}{4.329879in}}%
\pgfpathlineto{\pgfqpoint{3.121991in}{4.336506in}}%
\pgfpathlineto{\pgfqpoint{3.124315in}{4.339509in}}%
\pgfpathlineto{\pgfqpoint{3.126638in}{4.373099in}}%
\pgfpathlineto{\pgfqpoint{3.128962in}{4.340532in}}%
\pgfpathlineto{\pgfqpoint{3.131286in}{4.376653in}}%
\pgfpathlineto{\pgfqpoint{3.133610in}{4.336732in}}%
\pgfpathlineto{\pgfqpoint{3.138258in}{4.403796in}}%
\pgfpathlineto{\pgfqpoint{3.140582in}{4.366525in}}%
\pgfpathlineto{\pgfqpoint{3.142906in}{4.358674in}}%
\pgfpathlineto{\pgfqpoint{3.145230in}{4.417476in}}%
\pgfpathlineto{\pgfqpoint{3.147554in}{4.372106in}}%
\pgfpathlineto{\pgfqpoint{3.149878in}{4.380532in}}%
\pgfpathlineto{\pgfqpoint{3.152202in}{4.377696in}}%
\pgfpathlineto{\pgfqpoint{3.154526in}{4.389539in}}%
\pgfpathlineto{\pgfqpoint{3.156850in}{4.392668in}}%
\pgfpathlineto{\pgfqpoint{3.159174in}{4.381053in}}%
\pgfpathlineto{\pgfqpoint{3.161498in}{4.346962in}}%
\pgfpathlineto{\pgfqpoint{3.163822in}{4.419457in}}%
\pgfpathlineto{\pgfqpoint{3.166146in}{4.376088in}}%
\pgfpathlineto{\pgfqpoint{3.168470in}{4.375046in}}%
\pgfpathlineto{\pgfqpoint{3.170794in}{4.394388in}}%
\pgfpathlineto{\pgfqpoint{3.173118in}{4.335175in}}%
\pgfpathlineto{\pgfqpoint{3.175442in}{4.454454in}}%
\pgfpathlineto{\pgfqpoint{3.177766in}{4.416333in}}%
\pgfpathlineto{\pgfqpoint{3.182414in}{4.472318in}}%
\pgfpathlineto{\pgfqpoint{3.184738in}{4.474231in}}%
\pgfpathlineto{\pgfqpoint{3.187062in}{4.349164in}}%
\pgfpathlineto{\pgfqpoint{3.189386in}{4.413920in}}%
\pgfpathlineto{\pgfqpoint{3.191710in}{4.376464in}}%
\pgfpathlineto{\pgfqpoint{3.196357in}{4.421362in}}%
\pgfpathlineto{\pgfqpoint{3.198681in}{4.445063in}}%
\pgfpathlineto{\pgfqpoint{3.201005in}{4.400762in}}%
\pgfpathlineto{\pgfqpoint{3.203329in}{4.483744in}}%
\pgfpathlineto{\pgfqpoint{3.205653in}{4.431771in}}%
\pgfpathlineto{\pgfqpoint{3.207977in}{4.425353in}}%
\pgfpathlineto{\pgfqpoint{3.210301in}{4.473248in}}%
\pgfpathlineto{\pgfqpoint{3.212625in}{4.480118in}}%
\pgfpathlineto{\pgfqpoint{3.214949in}{4.416925in}}%
\pgfpathlineto{\pgfqpoint{3.217273in}{4.401685in}}%
\pgfpathlineto{\pgfqpoint{3.219597in}{4.414458in}}%
\pgfpathlineto{\pgfqpoint{3.221921in}{4.468123in}}%
\pgfpathlineto{\pgfqpoint{3.224245in}{4.456041in}}%
\pgfpathlineto{\pgfqpoint{3.226569in}{4.422229in}}%
\pgfpathlineto{\pgfqpoint{3.228893in}{4.494981in}}%
\pgfpathlineto{\pgfqpoint{3.231217in}{4.428044in}}%
\pgfpathlineto{\pgfqpoint{3.233541in}{4.493136in}}%
\pgfpathlineto{\pgfqpoint{3.235865in}{4.489516in}}%
\pgfpathlineto{\pgfqpoint{3.240513in}{4.460275in}}%
\pgfpathlineto{\pgfqpoint{3.242837in}{4.472008in}}%
\pgfpathlineto{\pgfqpoint{3.247485in}{4.528038in}}%
\pgfpathlineto{\pgfqpoint{3.249809in}{4.482773in}}%
\pgfpathlineto{\pgfqpoint{3.252133in}{4.463783in}}%
\pgfpathlineto{\pgfqpoint{3.254457in}{4.565545in}}%
\pgfpathlineto{\pgfqpoint{3.256781in}{4.451088in}}%
\pgfpathlineto{\pgfqpoint{3.261429in}{4.577157in}}%
\pgfpathlineto{\pgfqpoint{3.263753in}{4.471078in}}%
\pgfpathlineto{\pgfqpoint{3.266076in}{3.871962in}}%
\pgfpathlineto{\pgfqpoint{3.268400in}{3.815827in}}%
\pgfpathlineto{\pgfqpoint{3.270724in}{3.861749in}}%
\pgfpathlineto{\pgfqpoint{3.273048in}{3.870330in}}%
\pgfpathlineto{\pgfqpoint{3.275372in}{3.814753in}}%
\pgfpathlineto{\pgfqpoint{3.280020in}{3.863295in}}%
\pgfpathlineto{\pgfqpoint{3.282344in}{3.855794in}}%
\pgfpathlineto{\pgfqpoint{3.284668in}{3.822189in}}%
\pgfpathlineto{\pgfqpoint{3.289316in}{3.886558in}}%
\pgfpathlineto{\pgfqpoint{3.291640in}{3.906825in}}%
\pgfpathlineto{\pgfqpoint{3.293964in}{3.846043in}}%
\pgfpathlineto{\pgfqpoint{3.296288in}{3.885301in}}%
\pgfpathlineto{\pgfqpoint{3.298612in}{3.904706in}}%
\pgfpathlineto{\pgfqpoint{3.300936in}{3.860924in}}%
\pgfpathlineto{\pgfqpoint{3.303260in}{3.872243in}}%
\pgfpathlineto{\pgfqpoint{3.305584in}{3.864841in}}%
\pgfpathlineto{\pgfqpoint{3.307908in}{3.884042in}}%
\pgfpathlineto{\pgfqpoint{3.310232in}{3.866775in}}%
\pgfpathlineto{\pgfqpoint{3.312556in}{3.878068in}}%
\pgfpathlineto{\pgfqpoint{3.314880in}{3.854731in}}%
\pgfpathlineto{\pgfqpoint{3.317204in}{3.916579in}}%
\pgfpathlineto{\pgfqpoint{3.319528in}{3.882883in}}%
\pgfpathlineto{\pgfqpoint{3.324176in}{3.905205in}}%
\pgfpathlineto{\pgfqpoint{3.326500in}{3.893731in}}%
\pgfpathlineto{\pgfqpoint{3.328824in}{4.033772in}}%
\pgfpathlineto{\pgfqpoint{3.331148in}{3.858353in}}%
\pgfpathlineto{\pgfqpoint{3.333472in}{3.916353in}}%
\pgfpathlineto{\pgfqpoint{3.335796in}{3.918533in}}%
\pgfpathlineto{\pgfqpoint{3.338119in}{3.914453in}}%
\pgfpathlineto{\pgfqpoint{3.340443in}{3.924792in}}%
\pgfpathlineto{\pgfqpoint{3.342767in}{3.977313in}}%
\pgfpathlineto{\pgfqpoint{3.345091in}{3.882652in}}%
\pgfpathlineto{\pgfqpoint{3.349739in}{3.952616in}}%
\pgfpathlineto{\pgfqpoint{3.352063in}{3.969209in}}%
\pgfpathlineto{\pgfqpoint{3.354387in}{3.942794in}}%
\pgfpathlineto{\pgfqpoint{3.356711in}{4.081849in}}%
\pgfpathlineto{\pgfqpoint{3.359035in}{4.016359in}}%
\pgfpathlineto{\pgfqpoint{3.361359in}{3.897175in}}%
\pgfpathlineto{\pgfqpoint{3.363683in}{3.967764in}}%
\pgfpathlineto{\pgfqpoint{3.366007in}{3.947408in}}%
\pgfpathlineto{\pgfqpoint{3.368331in}{3.949265in}}%
\pgfpathlineto{\pgfqpoint{3.370655in}{3.961236in}}%
\pgfpathlineto{\pgfqpoint{3.372979in}{3.966734in}}%
\pgfpathlineto{\pgfqpoint{3.375303in}{3.995827in}}%
\pgfpathlineto{\pgfqpoint{3.377627in}{3.950050in}}%
\pgfpathlineto{\pgfqpoint{3.379951in}{3.987906in}}%
\pgfpathlineto{\pgfqpoint{3.382275in}{3.965836in}}%
\pgfpathlineto{\pgfqpoint{3.384599in}{4.000190in}}%
\pgfpathlineto{\pgfqpoint{3.386923in}{4.009179in}}%
\pgfpathlineto{\pgfqpoint{3.389247in}{4.031001in}}%
\pgfpathlineto{\pgfqpoint{3.391571in}{4.010152in}}%
\pgfpathlineto{\pgfqpoint{3.393895in}{4.035441in}}%
\pgfpathlineto{\pgfqpoint{3.396219in}{4.037761in}}%
\pgfpathlineto{\pgfqpoint{3.398543in}{4.010626in}}%
\pgfpathlineto{\pgfqpoint{3.400867in}{4.007641in}}%
\pgfpathlineto{\pgfqpoint{3.403191in}{3.992440in}}%
\pgfpathlineto{\pgfqpoint{3.407838in}{4.106916in}}%
\pgfpathlineto{\pgfqpoint{3.410162in}{4.046424in}}%
\pgfpathlineto{\pgfqpoint{3.412486in}{4.047720in}}%
\pgfpathlineto{\pgfqpoint{3.414810in}{4.050420in}}%
\pgfpathlineto{\pgfqpoint{3.417134in}{4.007958in}}%
\pgfpathlineto{\pgfqpoint{3.419458in}{3.998625in}}%
\pgfpathlineto{\pgfqpoint{3.421782in}{3.984452in}}%
\pgfpathlineto{\pgfqpoint{3.424106in}{4.043884in}}%
\pgfpathlineto{\pgfqpoint{3.426430in}{4.041661in}}%
\pgfpathlineto{\pgfqpoint{3.428754in}{3.990787in}}%
\pgfpathlineto{\pgfqpoint{3.431078in}{4.010479in}}%
\pgfpathlineto{\pgfqpoint{3.433402in}{3.993891in}}%
\pgfpathlineto{\pgfqpoint{3.435726in}{4.083481in}}%
\pgfpathlineto{\pgfqpoint{3.438050in}{3.997205in}}%
\pgfpathlineto{\pgfqpoint{3.440374in}{4.087158in}}%
\pgfpathlineto{\pgfqpoint{3.442698in}{4.093886in}}%
\pgfpathlineto{\pgfqpoint{3.445022in}{4.031536in}}%
\pgfpathlineto{\pgfqpoint{3.447346in}{4.063095in}}%
\pgfpathlineto{\pgfqpoint{3.449670in}{4.027850in}}%
\pgfpathlineto{\pgfqpoint{3.451994in}{4.017221in}}%
\pgfpathlineto{\pgfqpoint{3.458966in}{4.115879in}}%
\pgfpathlineto{\pgfqpoint{3.461290in}{4.134749in}}%
\pgfpathlineto{\pgfqpoint{3.463614in}{4.068235in}}%
\pgfpathlineto{\pgfqpoint{3.465938in}{4.114886in}}%
\pgfpathlineto{\pgfqpoint{3.468262in}{4.111720in}}%
\pgfpathlineto{\pgfqpoint{3.470586in}{4.027127in}}%
\pgfpathlineto{\pgfqpoint{3.472910in}{4.065948in}}%
\pgfpathlineto{\pgfqpoint{3.475234in}{4.139053in}}%
\pgfpathlineto{\pgfqpoint{3.477557in}{4.062934in}}%
\pgfpathlineto{\pgfqpoint{3.479881in}{4.128538in}}%
\pgfpathlineto{\pgfqpoint{3.482205in}{4.109996in}}%
\pgfpathlineto{\pgfqpoint{3.484529in}{4.159331in}}%
\pgfpathlineto{\pgfqpoint{3.486853in}{4.066954in}}%
\pgfpathlineto{\pgfqpoint{3.489177in}{4.112231in}}%
\pgfpathlineto{\pgfqpoint{3.491501in}{4.123233in}}%
\pgfpathlineto{\pgfqpoint{3.493825in}{4.076173in}}%
\pgfpathlineto{\pgfqpoint{3.496149in}{4.115186in}}%
\pgfpathlineto{\pgfqpoint{3.498473in}{4.093986in}}%
\pgfpathlineto{\pgfqpoint{3.500797in}{4.145992in}}%
\pgfpathlineto{\pgfqpoint{3.503121in}{4.103794in}}%
\pgfpathlineto{\pgfqpoint{3.505445in}{4.110814in}}%
\pgfpathlineto{\pgfqpoint{3.507769in}{4.132002in}}%
\pgfpathlineto{\pgfqpoint{3.510093in}{4.070992in}}%
\pgfpathlineto{\pgfqpoint{3.514741in}{4.173672in}}%
\pgfpathlineto{\pgfqpoint{3.517065in}{4.177541in}}%
\pgfpathlineto{\pgfqpoint{3.519389in}{4.087413in}}%
\pgfpathlineto{\pgfqpoint{3.521713in}{4.173090in}}%
\pgfpathlineto{\pgfqpoint{3.524037in}{4.216767in}}%
\pgfpathlineto{\pgfqpoint{3.526361in}{4.161656in}}%
\pgfpathlineto{\pgfqpoint{3.528685in}{4.154260in}}%
\pgfpathlineto{\pgfqpoint{3.531009in}{4.138695in}}%
\pgfpathlineto{\pgfqpoint{3.533333in}{4.150477in}}%
\pgfpathlineto{\pgfqpoint{3.535657in}{4.199258in}}%
\pgfpathlineto{\pgfqpoint{3.540305in}{4.182676in}}%
\pgfpathlineto{\pgfqpoint{3.542629in}{4.115862in}}%
\pgfpathlineto{\pgfqpoint{3.544953in}{4.172403in}}%
\pgfpathlineto{\pgfqpoint{3.547277in}{4.159357in}}%
\pgfpathlineto{\pgfqpoint{3.549600in}{4.219730in}}%
\pgfpathlineto{\pgfqpoint{3.551924in}{4.101947in}}%
\pgfpathlineto{\pgfqpoint{3.554248in}{4.167642in}}%
\pgfpathlineto{\pgfqpoint{3.556572in}{4.153461in}}%
\pgfpathlineto{\pgfqpoint{3.558896in}{4.216262in}}%
\pgfpathlineto{\pgfqpoint{3.561220in}{4.238939in}}%
\pgfpathlineto{\pgfqpoint{3.565868in}{4.161904in}}%
\pgfpathlineto{\pgfqpoint{3.568192in}{4.196518in}}%
\pgfpathlineto{\pgfqpoint{3.570516in}{4.150777in}}%
\pgfpathlineto{\pgfqpoint{3.572840in}{4.189091in}}%
\pgfpathlineto{\pgfqpoint{3.575164in}{4.141840in}}%
\pgfpathlineto{\pgfqpoint{3.577488in}{4.249179in}}%
\pgfpathlineto{\pgfqpoint{3.582136in}{4.184795in}}%
\pgfpathlineto{\pgfqpoint{3.584460in}{4.246594in}}%
\pgfpathlineto{\pgfqpoint{3.586784in}{4.199131in}}%
\pgfpathlineto{\pgfqpoint{3.589108in}{4.271737in}}%
\pgfpathlineto{\pgfqpoint{3.591432in}{4.259242in}}%
\pgfpathlineto{\pgfqpoint{3.593756in}{4.266332in}}%
\pgfpathlineto{\pgfqpoint{3.596080in}{4.203202in}}%
\pgfpathlineto{\pgfqpoint{3.598404in}{4.205230in}}%
\pgfpathlineto{\pgfqpoint{3.600728in}{4.287707in}}%
\pgfpathlineto{\pgfqpoint{3.603052in}{4.268721in}}%
\pgfpathlineto{\pgfqpoint{3.605376in}{4.236136in}}%
\pgfpathlineto{\pgfqpoint{3.607700in}{4.241192in}}%
\pgfpathlineto{\pgfqpoint{3.610024in}{4.219586in}}%
\pgfpathlineto{\pgfqpoint{3.612348in}{4.267060in}}%
\pgfpathlineto{\pgfqpoint{3.616996in}{4.176177in}}%
\pgfpathlineto{\pgfqpoint{3.619319in}{4.282561in}}%
\pgfpathlineto{\pgfqpoint{3.621643in}{4.254778in}}%
\pgfpathlineto{\pgfqpoint{3.623967in}{4.258748in}}%
\pgfpathlineto{\pgfqpoint{3.626291in}{4.300082in}}%
\pgfpathlineto{\pgfqpoint{3.628615in}{4.306261in}}%
\pgfpathlineto{\pgfqpoint{3.630939in}{4.270334in}}%
\pgfpathlineto{\pgfqpoint{3.633263in}{4.265718in}}%
\pgfpathlineto{\pgfqpoint{3.635587in}{4.267768in}}%
\pgfpathlineto{\pgfqpoint{3.637911in}{4.256091in}}%
\pgfpathlineto{\pgfqpoint{3.642559in}{4.312491in}}%
\pgfpathlineto{\pgfqpoint{3.644883in}{4.274919in}}%
\pgfpathlineto{\pgfqpoint{3.647207in}{4.330021in}}%
\pgfpathlineto{\pgfqpoint{3.649531in}{4.249644in}}%
\pgfpathlineto{\pgfqpoint{3.651855in}{4.293488in}}%
\pgfpathlineto{\pgfqpoint{3.654179in}{4.287589in}}%
\pgfpathlineto{\pgfqpoint{3.656503in}{4.311262in}}%
\pgfpathlineto{\pgfqpoint{3.658827in}{4.308062in}}%
\pgfpathlineto{\pgfqpoint{3.661151in}{4.253049in}}%
\pgfpathlineto{\pgfqpoint{3.663475in}{4.334695in}}%
\pgfpathlineto{\pgfqpoint{3.665799in}{4.310270in}}%
\pgfpathlineto{\pgfqpoint{3.670447in}{4.334451in}}%
\pgfpathlineto{\pgfqpoint{3.672771in}{4.325742in}}%
\pgfpathlineto{\pgfqpoint{3.675095in}{4.368242in}}%
\pgfpathlineto{\pgfqpoint{3.677419in}{4.282082in}}%
\pgfpathlineto{\pgfqpoint{3.679743in}{4.275658in}}%
\pgfpathlineto{\pgfqpoint{3.682067in}{4.374323in}}%
\pgfpathlineto{\pgfqpoint{3.684391in}{4.315376in}}%
\pgfpathlineto{\pgfqpoint{3.686715in}{4.362695in}}%
\pgfpathlineto{\pgfqpoint{3.689038in}{4.354153in}}%
\pgfpathlineto{\pgfqpoint{3.691362in}{4.337912in}}%
\pgfpathlineto{\pgfqpoint{3.693686in}{4.332932in}}%
\pgfpathlineto{\pgfqpoint{3.696010in}{4.368765in}}%
\pgfpathlineto{\pgfqpoint{3.698334in}{4.356011in}}%
\pgfpathlineto{\pgfqpoint{3.700658in}{4.387433in}}%
\pgfpathlineto{\pgfqpoint{3.702982in}{4.305088in}}%
\pgfpathlineto{\pgfqpoint{3.707630in}{4.344648in}}%
\pgfpathlineto{\pgfqpoint{3.712278in}{4.448715in}}%
\pgfpathlineto{\pgfqpoint{3.714602in}{4.369909in}}%
\pgfpathlineto{\pgfqpoint{3.716926in}{4.331143in}}%
\pgfpathlineto{\pgfqpoint{3.719250in}{4.373009in}}%
\pgfpathlineto{\pgfqpoint{3.721574in}{4.345092in}}%
\pgfpathlineto{\pgfqpoint{3.723898in}{4.351498in}}%
\pgfpathlineto{\pgfqpoint{3.726222in}{4.362978in}}%
\pgfpathlineto{\pgfqpoint{3.728546in}{4.385823in}}%
\pgfpathlineto{\pgfqpoint{3.730870in}{4.385254in}}%
\pgfpathlineto{\pgfqpoint{3.733194in}{4.378645in}}%
\pgfpathlineto{\pgfqpoint{3.735518in}{4.405839in}}%
\pgfpathlineto{\pgfqpoint{3.740166in}{4.362642in}}%
\pgfpathlineto{\pgfqpoint{3.742490in}{4.432676in}}%
\pgfpathlineto{\pgfqpoint{3.744814in}{4.428298in}}%
\pgfpathlineto{\pgfqpoint{3.747138in}{4.399240in}}%
\pgfpathlineto{\pgfqpoint{3.749462in}{4.398720in}}%
\pgfpathlineto{\pgfqpoint{3.751786in}{4.439614in}}%
\pgfpathlineto{\pgfqpoint{3.754110in}{4.377100in}}%
\pgfpathlineto{\pgfqpoint{3.758757in}{4.461628in}}%
\pgfpathlineto{\pgfqpoint{3.761081in}{4.417084in}}%
\pgfpathlineto{\pgfqpoint{3.763405in}{4.450136in}}%
\pgfpathlineto{\pgfqpoint{3.765729in}{4.431395in}}%
\pgfpathlineto{\pgfqpoint{3.768053in}{4.440360in}}%
\pgfpathlineto{\pgfqpoint{3.770377in}{4.498916in}}%
\pgfpathlineto{\pgfqpoint{3.772701in}{4.410619in}}%
\pgfpathlineto{\pgfqpoint{3.775025in}{4.516310in}}%
\pgfpathlineto{\pgfqpoint{3.777349in}{4.433875in}}%
\pgfpathlineto{\pgfqpoint{3.781997in}{4.419261in}}%
\pgfpathlineto{\pgfqpoint{3.786645in}{4.429764in}}%
\pgfpathlineto{\pgfqpoint{3.788969in}{4.416525in}}%
\pgfpathlineto{\pgfqpoint{3.791293in}{4.338606in}}%
\pgfpathlineto{\pgfqpoint{3.793617in}{4.470540in}}%
\pgfpathlineto{\pgfqpoint{3.795941in}{4.417136in}}%
\pgfpathlineto{\pgfqpoint{3.798265in}{4.430418in}}%
\pgfpathlineto{\pgfqpoint{3.800589in}{4.516236in}}%
\pgfpathlineto{\pgfqpoint{3.802913in}{4.492366in}}%
\pgfpathlineto{\pgfqpoint{3.805237in}{4.418919in}}%
\pgfpathlineto{\pgfqpoint{3.807561in}{4.455458in}}%
\pgfpathlineto{\pgfqpoint{3.809885in}{4.459183in}}%
\pgfpathlineto{\pgfqpoint{3.812209in}{4.439582in}}%
\pgfpathlineto{\pgfqpoint{3.814533in}{4.497667in}}%
\pgfpathlineto{\pgfqpoint{3.816857in}{4.446287in}}%
\pgfpathlineto{\pgfqpoint{3.819181in}{4.467275in}}%
\pgfpathlineto{\pgfqpoint{3.821505in}{4.510179in}}%
\pgfpathlineto{\pgfqpoint{3.823829in}{4.445466in}}%
\pgfpathlineto{\pgfqpoint{3.826153in}{4.489678in}}%
\pgfpathlineto{\pgfqpoint{3.828477in}{4.509271in}}%
\pgfpathlineto{\pgfqpoint{3.833124in}{4.459053in}}%
\pgfpathlineto{\pgfqpoint{3.837772in}{4.515323in}}%
\pgfpathlineto{\pgfqpoint{3.840096in}{4.512918in}}%
\pgfpathlineto{\pgfqpoint{3.842420in}{4.558750in}}%
\pgfpathlineto{\pgfqpoint{3.844744in}{4.538540in}}%
\pgfpathlineto{\pgfqpoint{3.847068in}{3.875231in}}%
\pgfpathlineto{\pgfqpoint{3.849392in}{3.894966in}}%
\pgfpathlineto{\pgfqpoint{3.854040in}{3.817434in}}%
\pgfpathlineto{\pgfqpoint{3.856364in}{3.867543in}}%
\pgfpathlineto{\pgfqpoint{3.858688in}{3.880354in}}%
\pgfpathlineto{\pgfqpoint{3.861012in}{3.905827in}}%
\pgfpathlineto{\pgfqpoint{3.865660in}{3.887027in}}%
\pgfpathlineto{\pgfqpoint{3.867984in}{3.842473in}}%
\pgfpathlineto{\pgfqpoint{3.870308in}{3.942934in}}%
\pgfpathlineto{\pgfqpoint{3.872632in}{3.891892in}}%
\pgfpathlineto{\pgfqpoint{3.874956in}{3.882156in}}%
\pgfpathlineto{\pgfqpoint{3.877280in}{3.859364in}}%
\pgfpathlineto{\pgfqpoint{3.879604in}{3.908933in}}%
\pgfpathlineto{\pgfqpoint{3.881928in}{3.892803in}}%
\pgfpathlineto{\pgfqpoint{3.884252in}{3.931450in}}%
\pgfpathlineto{\pgfqpoint{3.886576in}{3.832415in}}%
\pgfpathlineto{\pgfqpoint{3.891224in}{3.934747in}}%
\pgfpathlineto{\pgfqpoint{3.893548in}{3.862623in}}%
\pgfpathlineto{\pgfqpoint{3.895872in}{3.895594in}}%
\pgfpathlineto{\pgfqpoint{3.898196in}{3.888014in}}%
\pgfpathlineto{\pgfqpoint{3.900519in}{3.937241in}}%
\pgfpathlineto{\pgfqpoint{3.902843in}{3.897286in}}%
\pgfpathlineto{\pgfqpoint{3.905167in}{4.005499in}}%
\pgfpathlineto{\pgfqpoint{3.907491in}{3.935898in}}%
\pgfpathlineto{\pgfqpoint{3.909815in}{3.930045in}}%
\pgfpathlineto{\pgfqpoint{3.912139in}{3.949043in}}%
\pgfpathlineto{\pgfqpoint{3.914463in}{3.935131in}}%
\pgfpathlineto{\pgfqpoint{3.916787in}{3.971657in}}%
\pgfpathlineto{\pgfqpoint{3.919111in}{3.913596in}}%
\pgfpathlineto{\pgfqpoint{3.921435in}{3.995831in}}%
\pgfpathlineto{\pgfqpoint{3.923759in}{3.902746in}}%
\pgfpathlineto{\pgfqpoint{3.926083in}{3.865812in}}%
\pgfpathlineto{\pgfqpoint{3.930731in}{4.008989in}}%
\pgfpathlineto{\pgfqpoint{3.933055in}{3.995961in}}%
\pgfpathlineto{\pgfqpoint{3.935379in}{3.991024in}}%
\pgfpathlineto{\pgfqpoint{3.937703in}{4.023042in}}%
\pgfpathlineto{\pgfqpoint{3.940027in}{3.977897in}}%
\pgfpathlineto{\pgfqpoint{3.942351in}{3.893079in}}%
\pgfpathlineto{\pgfqpoint{3.944675in}{3.965870in}}%
\pgfpathlineto{\pgfqpoint{3.946999in}{3.978379in}}%
\pgfpathlineto{\pgfqpoint{3.949323in}{3.957086in}}%
\pgfpathlineto{\pgfqpoint{3.951647in}{3.974554in}}%
\pgfpathlineto{\pgfqpoint{3.953971in}{3.965449in}}%
\pgfpathlineto{\pgfqpoint{3.956295in}{3.947309in}}%
\pgfpathlineto{\pgfqpoint{3.958619in}{3.998669in}}%
\pgfpathlineto{\pgfqpoint{3.960943in}{4.005581in}}%
\pgfpathlineto{\pgfqpoint{3.963267in}{3.959172in}}%
\pgfpathlineto{\pgfqpoint{3.965591in}{3.987341in}}%
\pgfpathlineto{\pgfqpoint{3.967915in}{4.037704in}}%
\pgfpathlineto{\pgfqpoint{3.972562in}{3.966996in}}%
\pgfpathlineto{\pgfqpoint{3.974886in}{4.006933in}}%
\pgfpathlineto{\pgfqpoint{3.977210in}{3.986030in}}%
\pgfpathlineto{\pgfqpoint{3.979534in}{3.940219in}}%
\pgfpathlineto{\pgfqpoint{3.981858in}{3.998324in}}%
\pgfpathlineto{\pgfqpoint{3.984182in}{3.993630in}}%
\pgfpathlineto{\pgfqpoint{3.986506in}{3.964546in}}%
\pgfpathlineto{\pgfqpoint{3.988830in}{4.054204in}}%
\pgfpathlineto{\pgfqpoint{3.991154in}{4.035397in}}%
\pgfpathlineto{\pgfqpoint{3.993478in}{4.076631in}}%
\pgfpathlineto{\pgfqpoint{3.995802in}{4.021980in}}%
\pgfpathlineto{\pgfqpoint{3.998126in}{4.005981in}}%
\pgfpathlineto{\pgfqpoint{4.000450in}{4.051027in}}%
\pgfpathlineto{\pgfqpoint{4.002774in}{4.046777in}}%
\pgfpathlineto{\pgfqpoint{4.005098in}{4.002363in}}%
\pgfpathlineto{\pgfqpoint{4.007422in}{4.060711in}}%
\pgfpathlineto{\pgfqpoint{4.009746in}{4.005188in}}%
\pgfpathlineto{\pgfqpoint{4.012070in}{3.985970in}}%
\pgfpathlineto{\pgfqpoint{4.014394in}{3.988001in}}%
\pgfpathlineto{\pgfqpoint{4.016718in}{4.079729in}}%
\pgfpathlineto{\pgfqpoint{4.019042in}{4.118900in}}%
\pgfpathlineto{\pgfqpoint{4.023690in}{4.076105in}}%
\pgfpathlineto{\pgfqpoint{4.026014in}{4.008654in}}%
\pgfpathlineto{\pgfqpoint{4.028338in}{4.091965in}}%
\pgfpathlineto{\pgfqpoint{4.030662in}{4.065885in}}%
\pgfpathlineto{\pgfqpoint{4.032986in}{4.084044in}}%
\pgfpathlineto{\pgfqpoint{4.035310in}{4.080692in}}%
\pgfpathlineto{\pgfqpoint{4.037634in}{4.069281in}}%
\pgfpathlineto{\pgfqpoint{4.042281in}{4.113115in}}%
\pgfpathlineto{\pgfqpoint{4.044605in}{4.079225in}}%
\pgfpathlineto{\pgfqpoint{4.046929in}{4.006342in}}%
\pgfpathlineto{\pgfqpoint{4.049253in}{4.088535in}}%
\pgfpathlineto{\pgfqpoint{4.051577in}{4.119246in}}%
\pgfpathlineto{\pgfqpoint{4.053901in}{4.080433in}}%
\pgfpathlineto{\pgfqpoint{4.056225in}{4.084457in}}%
\pgfpathlineto{\pgfqpoint{4.058549in}{4.110724in}}%
\pgfpathlineto{\pgfqpoint{4.060873in}{4.077323in}}%
\pgfpathlineto{\pgfqpoint{4.063197in}{4.113778in}}%
\pgfpathlineto{\pgfqpoint{4.065521in}{4.128805in}}%
\pgfpathlineto{\pgfqpoint{4.067845in}{4.085258in}}%
\pgfpathlineto{\pgfqpoint{4.070169in}{4.115927in}}%
\pgfpathlineto{\pgfqpoint{4.072493in}{4.130896in}}%
\pgfpathlineto{\pgfqpoint{4.074817in}{4.123953in}}%
\pgfpathlineto{\pgfqpoint{4.077141in}{4.132653in}}%
\pgfpathlineto{\pgfqpoint{4.079465in}{4.105543in}}%
\pgfpathlineto{\pgfqpoint{4.081789in}{4.117580in}}%
\pgfpathlineto{\pgfqpoint{4.084113in}{4.090425in}}%
\pgfpathlineto{\pgfqpoint{4.086437in}{4.130562in}}%
\pgfpathlineto{\pgfqpoint{4.088761in}{4.120983in}}%
\pgfpathlineto{\pgfqpoint{4.091085in}{4.130986in}}%
\pgfpathlineto{\pgfqpoint{4.093409in}{4.146800in}}%
\pgfpathlineto{\pgfqpoint{4.095733in}{4.089508in}}%
\pgfpathlineto{\pgfqpoint{4.098057in}{4.176010in}}%
\pgfpathlineto{\pgfqpoint{4.100381in}{4.109239in}}%
\pgfpathlineto{\pgfqpoint{4.102705in}{4.134946in}}%
\pgfpathlineto{\pgfqpoint{4.105029in}{4.123313in}}%
\pgfpathlineto{\pgfqpoint{4.107353in}{4.147080in}}%
\pgfpathlineto{\pgfqpoint{4.109677in}{4.137766in}}%
\pgfpathlineto{\pgfqpoint{4.114324in}{4.211354in}}%
\pgfpathlineto{\pgfqpoint{4.116648in}{4.112322in}}%
\pgfpathlineto{\pgfqpoint{4.118972in}{4.119901in}}%
\pgfpathlineto{\pgfqpoint{4.121296in}{4.146209in}}%
\pgfpathlineto{\pgfqpoint{4.123620in}{4.193171in}}%
\pgfpathlineto{\pgfqpoint{4.125944in}{4.214452in}}%
\pgfpathlineto{\pgfqpoint{4.130592in}{4.092192in}}%
\pgfpathlineto{\pgfqpoint{4.132916in}{4.121413in}}%
\pgfpathlineto{\pgfqpoint{4.135240in}{4.176352in}}%
\pgfpathlineto{\pgfqpoint{4.139888in}{4.189843in}}%
\pgfpathlineto{\pgfqpoint{4.142212in}{4.183859in}}%
\pgfpathlineto{\pgfqpoint{4.144536in}{4.197871in}}%
\pgfpathlineto{\pgfqpoint{4.146860in}{4.131587in}}%
\pgfpathlineto{\pgfqpoint{4.149184in}{4.213534in}}%
\pgfpathlineto{\pgfqpoint{4.151508in}{4.175410in}}%
\pgfpathlineto{\pgfqpoint{4.153832in}{4.218220in}}%
\pgfpathlineto{\pgfqpoint{4.156156in}{4.209742in}}%
\pgfpathlineto{\pgfqpoint{4.158480in}{4.181896in}}%
\pgfpathlineto{\pgfqpoint{4.160804in}{4.289203in}}%
\pgfpathlineto{\pgfqpoint{4.165452in}{4.200432in}}%
\pgfpathlineto{\pgfqpoint{4.167776in}{4.200667in}}%
\pgfpathlineto{\pgfqpoint{4.170100in}{4.213445in}}%
\pgfpathlineto{\pgfqpoint{4.172424in}{4.181849in}}%
\pgfpathlineto{\pgfqpoint{4.174748in}{4.247439in}}%
\pgfpathlineto{\pgfqpoint{4.177072in}{4.236410in}}%
\pgfpathlineto{\pgfqpoint{4.179396in}{4.205766in}}%
\pgfpathlineto{\pgfqpoint{4.181719in}{4.199919in}}%
\pgfpathlineto{\pgfqpoint{4.184043in}{4.294676in}}%
\pgfpathlineto{\pgfqpoint{4.186367in}{4.208986in}}%
\pgfpathlineto{\pgfqpoint{4.188691in}{4.218867in}}%
\pgfpathlineto{\pgfqpoint{4.191015in}{4.295442in}}%
\pgfpathlineto{\pgfqpoint{4.193339in}{4.209420in}}%
\pgfpathlineto{\pgfqpoint{4.195663in}{4.270939in}}%
\pgfpathlineto{\pgfqpoint{4.197987in}{4.261796in}}%
\pgfpathlineto{\pgfqpoint{4.200311in}{4.288566in}}%
\pgfpathlineto{\pgfqpoint{4.202635in}{4.273683in}}%
\pgfpathlineto{\pgfqpoint{4.204959in}{4.266271in}}%
\pgfpathlineto{\pgfqpoint{4.207283in}{4.207364in}}%
\pgfpathlineto{\pgfqpoint{4.209607in}{4.253058in}}%
\pgfpathlineto{\pgfqpoint{4.211931in}{4.266737in}}%
\pgfpathlineto{\pgfqpoint{4.214255in}{4.253320in}}%
\pgfpathlineto{\pgfqpoint{4.216579in}{4.312900in}}%
\pgfpathlineto{\pgfqpoint{4.218903in}{4.285868in}}%
\pgfpathlineto{\pgfqpoint{4.221227in}{4.272403in}}%
\pgfpathlineto{\pgfqpoint{4.223551in}{4.240291in}}%
\pgfpathlineto{\pgfqpoint{4.225875in}{4.318929in}}%
\pgfpathlineto{\pgfqpoint{4.228199in}{4.358000in}}%
\pgfpathlineto{\pgfqpoint{4.230523in}{4.263767in}}%
\pgfpathlineto{\pgfqpoint{4.232847in}{4.298675in}}%
\pgfpathlineto{\pgfqpoint{4.235171in}{4.267441in}}%
\pgfpathlineto{\pgfqpoint{4.237495in}{4.259858in}}%
\pgfpathlineto{\pgfqpoint{4.239819in}{4.366630in}}%
\pgfpathlineto{\pgfqpoint{4.244467in}{4.244756in}}%
\pgfpathlineto{\pgfqpoint{4.249115in}{4.358361in}}%
\pgfpathlineto{\pgfqpoint{4.251438in}{4.313613in}}%
\pgfpathlineto{\pgfqpoint{4.253762in}{4.298105in}}%
\pgfpathlineto{\pgfqpoint{4.256086in}{4.311869in}}%
\pgfpathlineto{\pgfqpoint{4.258410in}{4.298164in}}%
\pgfpathlineto{\pgfqpoint{4.260734in}{4.271496in}}%
\pgfpathlineto{\pgfqpoint{4.263058in}{4.347525in}}%
\pgfpathlineto{\pgfqpoint{4.265382in}{4.349799in}}%
\pgfpathlineto{\pgfqpoint{4.267706in}{4.360248in}}%
\pgfpathlineto{\pgfqpoint{4.270030in}{4.324524in}}%
\pgfpathlineto{\pgfqpoint{4.272354in}{4.343747in}}%
\pgfpathlineto{\pgfqpoint{4.274678in}{4.332606in}}%
\pgfpathlineto{\pgfqpoint{4.277002in}{4.337982in}}%
\pgfpathlineto{\pgfqpoint{4.279326in}{4.335575in}}%
\pgfpathlineto{\pgfqpoint{4.281650in}{4.404284in}}%
\pgfpathlineto{\pgfqpoint{4.283974in}{4.341336in}}%
\pgfpathlineto{\pgfqpoint{4.286298in}{4.312992in}}%
\pgfpathlineto{\pgfqpoint{4.288622in}{4.407871in}}%
\pgfpathlineto{\pgfqpoint{4.290946in}{4.383077in}}%
\pgfpathlineto{\pgfqpoint{4.293270in}{4.430116in}}%
\pgfpathlineto{\pgfqpoint{4.295594in}{4.358481in}}%
\pgfpathlineto{\pgfqpoint{4.297918in}{4.361744in}}%
\pgfpathlineto{\pgfqpoint{4.300242in}{4.376391in}}%
\pgfpathlineto{\pgfqpoint{4.302566in}{4.361365in}}%
\pgfpathlineto{\pgfqpoint{4.304890in}{4.411324in}}%
\pgfpathlineto{\pgfqpoint{4.307214in}{4.409286in}}%
\pgfpathlineto{\pgfqpoint{4.309538in}{4.371304in}}%
\pgfpathlineto{\pgfqpoint{4.311862in}{4.380192in}}%
\pgfpathlineto{\pgfqpoint{4.314186in}{4.377900in}}%
\pgfpathlineto{\pgfqpoint{4.316510in}{4.403605in}}%
\pgfpathlineto{\pgfqpoint{4.318834in}{4.358935in}}%
\pgfpathlineto{\pgfqpoint{4.321158in}{4.384810in}}%
\pgfpathlineto{\pgfqpoint{4.323481in}{4.429495in}}%
\pgfpathlineto{\pgfqpoint{4.325805in}{4.405891in}}%
\pgfpathlineto{\pgfqpoint{4.330453in}{4.438615in}}%
\pgfpathlineto{\pgfqpoint{4.332777in}{4.431109in}}%
\pgfpathlineto{\pgfqpoint{4.335101in}{4.399209in}}%
\pgfpathlineto{\pgfqpoint{4.337425in}{4.385235in}}%
\pgfpathlineto{\pgfqpoint{4.339749in}{4.335703in}}%
\pgfpathlineto{\pgfqpoint{4.344397in}{4.422802in}}%
\pgfpathlineto{\pgfqpoint{4.346721in}{4.416063in}}%
\pgfpathlineto{\pgfqpoint{4.349045in}{4.460897in}}%
\pgfpathlineto{\pgfqpoint{4.351369in}{4.454044in}}%
\pgfpathlineto{\pgfqpoint{4.353693in}{4.440307in}}%
\pgfpathlineto{\pgfqpoint{4.356017in}{4.402079in}}%
\pgfpathlineto{\pgfqpoint{4.358341in}{4.437295in}}%
\pgfpathlineto{\pgfqpoint{4.360665in}{4.408129in}}%
\pgfpathlineto{\pgfqpoint{4.365313in}{4.486208in}}%
\pgfpathlineto{\pgfqpoint{4.367637in}{4.470444in}}%
\pgfpathlineto{\pgfqpoint{4.369961in}{4.431322in}}%
\pgfpathlineto{\pgfqpoint{4.372285in}{4.441896in}}%
\pgfpathlineto{\pgfqpoint{4.374609in}{4.432647in}}%
\pgfpathlineto{\pgfqpoint{4.376933in}{4.442469in}}%
\pgfpathlineto{\pgfqpoint{4.379257in}{4.446588in}}%
\pgfpathlineto{\pgfqpoint{4.381581in}{4.438343in}}%
\pgfpathlineto{\pgfqpoint{4.383905in}{4.483086in}}%
\pgfpathlineto{\pgfqpoint{4.386229in}{4.406576in}}%
\pgfpathlineto{\pgfqpoint{4.388553in}{4.486069in}}%
\pgfpathlineto{\pgfqpoint{4.390877in}{4.470739in}}%
\pgfpathlineto{\pgfqpoint{4.393200in}{4.446653in}}%
\pgfpathlineto{\pgfqpoint{4.397848in}{4.533159in}}%
\pgfpathlineto{\pgfqpoint{4.400172in}{4.482276in}}%
\pgfpathlineto{\pgfqpoint{4.402496in}{4.515179in}}%
\pgfpathlineto{\pgfqpoint{4.407144in}{4.416604in}}%
\pgfpathlineto{\pgfqpoint{4.409468in}{4.525064in}}%
\pgfpathlineto{\pgfqpoint{4.411792in}{4.518346in}}%
\pgfpathlineto{\pgfqpoint{4.416440in}{4.459691in}}%
\pgfpathlineto{\pgfqpoint{4.421088in}{4.523651in}}%
\pgfpathlineto{\pgfqpoint{4.423412in}{4.456553in}}%
\pgfpathlineto{\pgfqpoint{4.425736in}{4.539859in}}%
\pgfpathlineto{\pgfqpoint{4.428060in}{3.846114in}}%
\pgfpathlineto{\pgfqpoint{4.430384in}{3.879540in}}%
\pgfpathlineto{\pgfqpoint{4.432708in}{3.812590in}}%
\pgfpathlineto{\pgfqpoint{4.437356in}{3.890696in}}%
\pgfpathlineto{\pgfqpoint{4.439680in}{3.872188in}}%
\pgfpathlineto{\pgfqpoint{4.442004in}{3.874002in}}%
\pgfpathlineto{\pgfqpoint{4.444328in}{3.922937in}}%
\pgfpathlineto{\pgfqpoint{4.446652in}{3.856685in}}%
\pgfpathlineto{\pgfqpoint{4.451300in}{3.894114in}}%
\pgfpathlineto{\pgfqpoint{4.455948in}{3.874639in}}%
\pgfpathlineto{\pgfqpoint{4.458272in}{3.954460in}}%
\pgfpathlineto{\pgfqpoint{4.460596in}{3.893954in}}%
\pgfpathlineto{\pgfqpoint{4.462919in}{3.903167in}}%
\pgfpathlineto{\pgfqpoint{4.465243in}{3.922088in}}%
\pgfpathlineto{\pgfqpoint{4.467567in}{3.867659in}}%
\pgfpathlineto{\pgfqpoint{4.469891in}{3.936211in}}%
\pgfpathlineto{\pgfqpoint{4.472215in}{3.849976in}}%
\pgfpathlineto{\pgfqpoint{4.474539in}{3.913255in}}%
\pgfpathlineto{\pgfqpoint{4.479187in}{3.951945in}}%
\pgfpathlineto{\pgfqpoint{4.481511in}{3.912028in}}%
\pgfpathlineto{\pgfqpoint{4.483835in}{3.904103in}}%
\pgfpathlineto{\pgfqpoint{4.488483in}{4.009491in}}%
\pgfpathlineto{\pgfqpoint{4.490807in}{3.853753in}}%
\pgfpathlineto{\pgfqpoint{4.493131in}{3.919526in}}%
\pgfpathlineto{\pgfqpoint{4.495455in}{3.903865in}}%
\pgfpathlineto{\pgfqpoint{4.497779in}{3.980551in}}%
\pgfpathlineto{\pgfqpoint{4.500103in}{3.903132in}}%
\pgfpathlineto{\pgfqpoint{4.502427in}{3.899116in}}%
\pgfpathlineto{\pgfqpoint{4.504751in}{3.999785in}}%
\pgfpathlineto{\pgfqpoint{4.509399in}{3.925662in}}%
\pgfpathlineto{\pgfqpoint{4.511723in}{3.957508in}}%
\pgfpathlineto{\pgfqpoint{4.514047in}{3.937065in}}%
\pgfpathlineto{\pgfqpoint{4.516371in}{3.926989in}}%
\pgfpathlineto{\pgfqpoint{4.518695in}{3.960320in}}%
\pgfpathlineto{\pgfqpoint{4.521019in}{3.914835in}}%
\pgfpathlineto{\pgfqpoint{4.523343in}{3.957776in}}%
\pgfpathlineto{\pgfqpoint{4.525667in}{3.966372in}}%
\pgfpathlineto{\pgfqpoint{4.527991in}{3.959276in}}%
\pgfpathlineto{\pgfqpoint{4.530315in}{4.007501in}}%
\pgfpathlineto{\pgfqpoint{4.532638in}{3.999380in}}%
\pgfpathlineto{\pgfqpoint{4.534962in}{3.978720in}}%
\pgfpathlineto{\pgfqpoint{4.537286in}{3.969162in}}%
\pgfpathlineto{\pgfqpoint{4.539610in}{4.000562in}}%
\pgfpathlineto{\pgfqpoint{4.541934in}{4.015429in}}%
\pgfpathlineto{\pgfqpoint{4.544258in}{3.961530in}}%
\pgfpathlineto{\pgfqpoint{4.546582in}{3.935982in}}%
\pgfpathlineto{\pgfqpoint{4.548906in}{4.000548in}}%
\pgfpathlineto{\pgfqpoint{4.551230in}{3.967898in}}%
\pgfpathlineto{\pgfqpoint{4.553554in}{4.040644in}}%
\pgfpathlineto{\pgfqpoint{4.555878in}{4.037449in}}%
\pgfpathlineto{\pgfqpoint{4.558202in}{3.955846in}}%
\pgfpathlineto{\pgfqpoint{4.562850in}{4.025935in}}%
\pgfpathlineto{\pgfqpoint{4.565174in}{4.029909in}}%
\pgfpathlineto{\pgfqpoint{4.569822in}{3.974124in}}%
\pgfpathlineto{\pgfqpoint{4.572146in}{3.978956in}}%
\pgfpathlineto{\pgfqpoint{4.574470in}{4.044033in}}%
\pgfpathlineto{\pgfqpoint{4.576794in}{3.970444in}}%
\pgfpathlineto{\pgfqpoint{4.579118in}{4.098281in}}%
\pgfpathlineto{\pgfqpoint{4.583766in}{4.025362in}}%
\pgfpathlineto{\pgfqpoint{4.586090in}{4.087339in}}%
\pgfpathlineto{\pgfqpoint{4.588414in}{4.050291in}}%
\pgfpathlineto{\pgfqpoint{4.590738in}{4.031793in}}%
\pgfpathlineto{\pgfqpoint{4.593062in}{4.071330in}}%
\pgfpathlineto{\pgfqpoint{4.595386in}{4.057923in}}%
\pgfpathlineto{\pgfqpoint{4.597710in}{4.061664in}}%
\pgfpathlineto{\pgfqpoint{4.600034in}{4.091672in}}%
\pgfpathlineto{\pgfqpoint{4.602358in}{4.025077in}}%
\pgfpathlineto{\pgfqpoint{4.604681in}{4.051285in}}%
\pgfpathlineto{\pgfqpoint{4.607005in}{4.012065in}}%
\pgfpathlineto{\pgfqpoint{4.609329in}{4.055072in}}%
\pgfpathlineto{\pgfqpoint{4.611653in}{4.065990in}}%
\pgfpathlineto{\pgfqpoint{4.613977in}{4.106468in}}%
\pgfpathlineto{\pgfqpoint{4.616301in}{4.033037in}}%
\pgfpathlineto{\pgfqpoint{4.618625in}{4.057765in}}%
\pgfpathlineto{\pgfqpoint{4.620949in}{4.036984in}}%
\pgfpathlineto{\pgfqpoint{4.623273in}{4.111089in}}%
\pgfpathlineto{\pgfqpoint{4.625597in}{4.078468in}}%
\pgfpathlineto{\pgfqpoint{4.627921in}{4.064962in}}%
\pgfpathlineto{\pgfqpoint{4.630245in}{4.085481in}}%
\pgfpathlineto{\pgfqpoint{4.632569in}{4.145000in}}%
\pgfpathlineto{\pgfqpoint{4.634893in}{4.040673in}}%
\pgfpathlineto{\pgfqpoint{4.637217in}{4.079734in}}%
\pgfpathlineto{\pgfqpoint{4.639541in}{4.048379in}}%
\pgfpathlineto{\pgfqpoint{4.644189in}{4.154638in}}%
\pgfpathlineto{\pgfqpoint{4.646513in}{4.088726in}}%
\pgfpathlineto{\pgfqpoint{4.648837in}{4.095342in}}%
\pgfpathlineto{\pgfqpoint{4.651161in}{4.116884in}}%
\pgfpathlineto{\pgfqpoint{4.653485in}{4.085006in}}%
\pgfpathlineto{\pgfqpoint{4.655809in}{4.119086in}}%
\pgfpathlineto{\pgfqpoint{4.658133in}{4.080568in}}%
\pgfpathlineto{\pgfqpoint{4.660457in}{4.082437in}}%
\pgfpathlineto{\pgfqpoint{4.662781in}{4.147946in}}%
\pgfpathlineto{\pgfqpoint{4.665105in}{4.164072in}}%
\pgfpathlineto{\pgfqpoint{4.667429in}{4.192397in}}%
\pgfpathlineto{\pgfqpoint{4.669753in}{4.138707in}}%
\pgfpathlineto{\pgfqpoint{4.672077in}{4.124179in}}%
\pgfpathlineto{\pgfqpoint{4.674400in}{4.155796in}}%
\pgfpathlineto{\pgfqpoint{4.676724in}{4.139371in}}%
\pgfpathlineto{\pgfqpoint{4.679048in}{4.108753in}}%
\pgfpathlineto{\pgfqpoint{4.681372in}{4.130934in}}%
\pgfpathlineto{\pgfqpoint{4.683696in}{4.123676in}}%
\pgfpathlineto{\pgfqpoint{4.686020in}{4.190130in}}%
\pgfpathlineto{\pgfqpoint{4.688344in}{4.129981in}}%
\pgfpathlineto{\pgfqpoint{4.690668in}{4.166291in}}%
\pgfpathlineto{\pgfqpoint{4.692992in}{4.161062in}}%
\pgfpathlineto{\pgfqpoint{4.695316in}{4.139645in}}%
\pgfpathlineto{\pgfqpoint{4.697640in}{4.090684in}}%
\pgfpathlineto{\pgfqpoint{4.699964in}{4.153632in}}%
\pgfpathlineto{\pgfqpoint{4.702288in}{4.133528in}}%
\pgfpathlineto{\pgfqpoint{4.704612in}{4.179296in}}%
\pgfpathlineto{\pgfqpoint{4.706936in}{4.150594in}}%
\pgfpathlineto{\pgfqpoint{4.709260in}{4.225987in}}%
\pgfpathlineto{\pgfqpoint{4.711584in}{4.167008in}}%
\pgfpathlineto{\pgfqpoint{4.713908in}{4.206159in}}%
\pgfpathlineto{\pgfqpoint{4.716232in}{4.174765in}}%
\pgfpathlineto{\pgfqpoint{4.718556in}{4.216185in}}%
\pgfpathlineto{\pgfqpoint{4.720880in}{4.188475in}}%
\pgfpathlineto{\pgfqpoint{4.723204in}{4.179871in}}%
\pgfpathlineto{\pgfqpoint{4.725528in}{4.190827in}}%
\pgfpathlineto{\pgfqpoint{4.727852in}{4.155964in}}%
\pgfpathlineto{\pgfqpoint{4.730176in}{4.236916in}}%
\pgfpathlineto{\pgfqpoint{4.732500in}{4.235709in}}%
\pgfpathlineto{\pgfqpoint{4.734824in}{4.255403in}}%
\pgfpathlineto{\pgfqpoint{4.737148in}{4.250225in}}%
\pgfpathlineto{\pgfqpoint{4.739472in}{4.198508in}}%
\pgfpathlineto{\pgfqpoint{4.741796in}{4.223592in}}%
\pgfpathlineto{\pgfqpoint{4.744119in}{4.311808in}}%
\pgfpathlineto{\pgfqpoint{4.746443in}{4.199223in}}%
\pgfpathlineto{\pgfqpoint{4.748767in}{4.174130in}}%
\pgfpathlineto{\pgfqpoint{4.753415in}{4.243517in}}%
\pgfpathlineto{\pgfqpoint{4.755739in}{4.183037in}}%
\pgfpathlineto{\pgfqpoint{4.758063in}{4.254792in}}%
\pgfpathlineto{\pgfqpoint{4.760387in}{4.252512in}}%
\pgfpathlineto{\pgfqpoint{4.762711in}{4.287866in}}%
\pgfpathlineto{\pgfqpoint{4.765035in}{4.225624in}}%
\pgfpathlineto{\pgfqpoint{4.767359in}{4.259567in}}%
\pgfpathlineto{\pgfqpoint{4.769683in}{4.227253in}}%
\pgfpathlineto{\pgfqpoint{4.772007in}{4.304506in}}%
\pgfpathlineto{\pgfqpoint{4.774331in}{4.305113in}}%
\pgfpathlineto{\pgfqpoint{4.776655in}{4.273946in}}%
\pgfpathlineto{\pgfqpoint{4.778979in}{4.262910in}}%
\pgfpathlineto{\pgfqpoint{4.781303in}{4.226834in}}%
\pgfpathlineto{\pgfqpoint{4.783627in}{4.272538in}}%
\pgfpathlineto{\pgfqpoint{4.785951in}{4.263607in}}%
\pgfpathlineto{\pgfqpoint{4.788275in}{4.293089in}}%
\pgfpathlineto{\pgfqpoint{4.790599in}{4.249092in}}%
\pgfpathlineto{\pgfqpoint{4.792923in}{4.275929in}}%
\pgfpathlineto{\pgfqpoint{4.795247in}{4.328072in}}%
\pgfpathlineto{\pgfqpoint{4.797571in}{4.247329in}}%
\pgfpathlineto{\pgfqpoint{4.799895in}{4.255552in}}%
\pgfpathlineto{\pgfqpoint{4.802219in}{4.315227in}}%
\pgfpathlineto{\pgfqpoint{4.804543in}{4.281546in}}%
\pgfpathlineto{\pgfqpoint{4.806867in}{4.265822in}}%
\pgfpathlineto{\pgfqpoint{4.809191in}{4.264826in}}%
\pgfpathlineto{\pgfqpoint{4.811515in}{4.314710in}}%
\pgfpathlineto{\pgfqpoint{4.813838in}{4.306379in}}%
\pgfpathlineto{\pgfqpoint{4.816162in}{4.244571in}}%
\pgfpathlineto{\pgfqpoint{4.818486in}{4.325208in}}%
\pgfpathlineto{\pgfqpoint{4.820810in}{4.291333in}}%
\pgfpathlineto{\pgfqpoint{4.823134in}{4.283418in}}%
\pgfpathlineto{\pgfqpoint{4.825458in}{4.269030in}}%
\pgfpathlineto{\pgfqpoint{4.827782in}{4.283198in}}%
\pgfpathlineto{\pgfqpoint{4.830106in}{4.372611in}}%
\pgfpathlineto{\pgfqpoint{4.832430in}{4.233332in}}%
\pgfpathlineto{\pgfqpoint{4.834754in}{4.322606in}}%
\pgfpathlineto{\pgfqpoint{4.837078in}{4.326970in}}%
\pgfpathlineto{\pgfqpoint{4.839402in}{4.337706in}}%
\pgfpathlineto{\pgfqpoint{4.841726in}{4.342023in}}%
\pgfpathlineto{\pgfqpoint{4.844050in}{4.374415in}}%
\pgfpathlineto{\pgfqpoint{4.846374in}{4.353631in}}%
\pgfpathlineto{\pgfqpoint{4.848698in}{4.358422in}}%
\pgfpathlineto{\pgfqpoint{4.851022in}{4.314921in}}%
\pgfpathlineto{\pgfqpoint{4.853346in}{4.366978in}}%
\pgfpathlineto{\pgfqpoint{4.855670in}{4.382409in}}%
\pgfpathlineto{\pgfqpoint{4.857994in}{4.369094in}}%
\pgfpathlineto{\pgfqpoint{4.860318in}{4.379066in}}%
\pgfpathlineto{\pgfqpoint{4.862642in}{4.324159in}}%
\pgfpathlineto{\pgfqpoint{4.864966in}{4.300486in}}%
\pgfpathlineto{\pgfqpoint{4.867290in}{4.296947in}}%
\pgfpathlineto{\pgfqpoint{4.869614in}{4.361073in}}%
\pgfpathlineto{\pgfqpoint{4.871938in}{4.337202in}}%
\pgfpathlineto{\pgfqpoint{4.874262in}{4.397870in}}%
\pgfpathlineto{\pgfqpoint{4.876586in}{4.369971in}}%
\pgfpathlineto{\pgfqpoint{4.878910in}{4.321098in}}%
\pgfpathlineto{\pgfqpoint{4.881234in}{4.387029in}}%
\pgfpathlineto{\pgfqpoint{4.883558in}{4.417820in}}%
\pgfpathlineto{\pgfqpoint{4.888205in}{4.358727in}}%
\pgfpathlineto{\pgfqpoint{4.890529in}{4.427268in}}%
\pgfpathlineto{\pgfqpoint{4.892853in}{4.364980in}}%
\pgfpathlineto{\pgfqpoint{4.895177in}{4.336892in}}%
\pgfpathlineto{\pgfqpoint{4.897501in}{4.419956in}}%
\pgfpathlineto{\pgfqpoint{4.899825in}{4.380041in}}%
\pgfpathlineto{\pgfqpoint{4.902149in}{4.397684in}}%
\pgfpathlineto{\pgfqpoint{4.906797in}{4.363666in}}%
\pgfpathlineto{\pgfqpoint{4.909121in}{4.388768in}}%
\pgfpathlineto{\pgfqpoint{4.911445in}{4.339001in}}%
\pgfpathlineto{\pgfqpoint{4.913769in}{4.454218in}}%
\pgfpathlineto{\pgfqpoint{4.918417in}{4.456185in}}%
\pgfpathlineto{\pgfqpoint{4.923065in}{4.349915in}}%
\pgfpathlineto{\pgfqpoint{4.925389in}{4.413391in}}%
\pgfpathlineto{\pgfqpoint{4.927713in}{4.435733in}}%
\pgfpathlineto{\pgfqpoint{4.930037in}{4.418863in}}%
\pgfpathlineto{\pgfqpoint{4.932361in}{4.412690in}}%
\pgfpathlineto{\pgfqpoint{4.934685in}{4.366668in}}%
\pgfpathlineto{\pgfqpoint{4.937009in}{4.454660in}}%
\pgfpathlineto{\pgfqpoint{4.939333in}{4.480745in}}%
\pgfpathlineto{\pgfqpoint{4.941657in}{4.398462in}}%
\pgfpathlineto{\pgfqpoint{4.946305in}{4.400121in}}%
\pgfpathlineto{\pgfqpoint{4.948629in}{4.426401in}}%
\pgfpathlineto{\pgfqpoint{4.950953in}{4.426716in}}%
\pgfpathlineto{\pgfqpoint{4.953277in}{4.457469in}}%
\pgfpathlineto{\pgfqpoint{4.955600in}{4.454086in}}%
\pgfpathlineto{\pgfqpoint{4.957924in}{4.471133in}}%
\pgfpathlineto{\pgfqpoint{4.960248in}{4.437590in}}%
\pgfpathlineto{\pgfqpoint{4.962572in}{4.387497in}}%
\pgfpathlineto{\pgfqpoint{4.967220in}{4.538948in}}%
\pgfpathlineto{\pgfqpoint{4.971868in}{4.403629in}}%
\pgfpathlineto{\pgfqpoint{4.974192in}{4.449483in}}%
\pgfpathlineto{\pgfqpoint{4.976516in}{4.453034in}}%
\pgfpathlineto{\pgfqpoint{4.978840in}{4.529908in}}%
\pgfpathlineto{\pgfqpoint{4.981164in}{4.484881in}}%
\pgfpathlineto{\pgfqpoint{4.983488in}{4.485490in}}%
\pgfpathlineto{\pgfqpoint{4.985812in}{4.484183in}}%
\pgfpathlineto{\pgfqpoint{4.988136in}{4.490889in}}%
\pgfpathlineto{\pgfqpoint{4.990460in}{4.474237in}}%
\pgfpathlineto{\pgfqpoint{4.992784in}{4.495555in}}%
\pgfpathlineto{\pgfqpoint{4.995108in}{4.535568in}}%
\pgfpathlineto{\pgfqpoint{4.997432in}{4.520378in}}%
\pgfpathlineto{\pgfqpoint{4.999756in}{4.525037in}}%
\pgfpathlineto{\pgfqpoint{5.002080in}{4.532399in}}%
\pgfpathlineto{\pgfqpoint{5.004404in}{4.531597in}}%
\pgfpathlineto{\pgfqpoint{5.006728in}{4.512492in}}%
\pgfpathlineto{\pgfqpoint{5.009052in}{3.854550in}}%
\pgfpathlineto{\pgfqpoint{5.011376in}{3.893332in}}%
\pgfpathlineto{\pgfqpoint{5.013700in}{3.883571in}}%
\pgfpathlineto{\pgfqpoint{5.016024in}{3.897569in}}%
\pgfpathlineto{\pgfqpoint{5.018348in}{3.853833in}}%
\pgfpathlineto{\pgfqpoint{5.020672in}{3.853205in}}%
\pgfpathlineto{\pgfqpoint{5.022996in}{3.896035in}}%
\pgfpathlineto{\pgfqpoint{5.025319in}{3.855793in}}%
\pgfpathlineto{\pgfqpoint{5.029967in}{3.845086in}}%
\pgfpathlineto{\pgfqpoint{5.032291in}{3.862111in}}%
\pgfpathlineto{\pgfqpoint{5.034615in}{3.898520in}}%
\pgfpathlineto{\pgfqpoint{5.036939in}{3.875212in}}%
\pgfpathlineto{\pgfqpoint{5.039263in}{3.876770in}}%
\pgfpathlineto{\pgfqpoint{5.041587in}{3.835156in}}%
\pgfpathlineto{\pgfqpoint{5.043911in}{3.888477in}}%
\pgfpathlineto{\pgfqpoint{5.046235in}{3.867505in}}%
\pgfpathlineto{\pgfqpoint{5.048559in}{3.917197in}}%
\pgfpathlineto{\pgfqpoint{5.050883in}{3.900356in}}%
\pgfpathlineto{\pgfqpoint{5.053207in}{3.935706in}}%
\pgfpathlineto{\pgfqpoint{5.057855in}{3.919832in}}%
\pgfpathlineto{\pgfqpoint{5.060179in}{3.972709in}}%
\pgfpathlineto{\pgfqpoint{5.062503in}{3.919054in}}%
\pgfpathlineto{\pgfqpoint{5.064827in}{3.956232in}}%
\pgfpathlineto{\pgfqpoint{5.067151in}{3.965748in}}%
\pgfpathlineto{\pgfqpoint{5.069475in}{3.966275in}}%
\pgfpathlineto{\pgfqpoint{5.071799in}{3.906190in}}%
\pgfpathlineto{\pgfqpoint{5.074123in}{3.921266in}}%
\pgfpathlineto{\pgfqpoint{5.076447in}{3.892768in}}%
\pgfpathlineto{\pgfqpoint{5.078771in}{3.938082in}}%
\pgfpathlineto{\pgfqpoint{5.081095in}{3.872125in}}%
\pgfpathlineto{\pgfqpoint{5.083419in}{3.954575in}}%
\pgfpathlineto{\pgfqpoint{5.085743in}{3.988187in}}%
\pgfpathlineto{\pgfqpoint{5.088067in}{3.973918in}}%
\pgfpathlineto{\pgfqpoint{5.090391in}{3.974405in}}%
\pgfpathlineto{\pgfqpoint{5.092715in}{3.955145in}}%
\pgfpathlineto{\pgfqpoint{5.095039in}{3.968206in}}%
\pgfpathlineto{\pgfqpoint{5.097362in}{3.988704in}}%
\pgfpathlineto{\pgfqpoint{5.099686in}{3.956868in}}%
\pgfpathlineto{\pgfqpoint{5.102010in}{3.998869in}}%
\pgfpathlineto{\pgfqpoint{5.104334in}{3.970833in}}%
\pgfpathlineto{\pgfqpoint{5.106658in}{3.924526in}}%
\pgfpathlineto{\pgfqpoint{5.108982in}{3.986553in}}%
\pgfpathlineto{\pgfqpoint{5.111306in}{3.973580in}}%
\pgfpathlineto{\pgfqpoint{5.113630in}{4.044617in}}%
\pgfpathlineto{\pgfqpoint{5.115954in}{4.017295in}}%
\pgfpathlineto{\pgfqpoint{5.118278in}{3.965401in}}%
\pgfpathlineto{\pgfqpoint{5.120602in}{3.984832in}}%
\pgfpathlineto{\pgfqpoint{5.122926in}{3.958435in}}%
\pgfpathlineto{\pgfqpoint{5.125250in}{4.004554in}}%
\pgfpathlineto{\pgfqpoint{5.127574in}{3.977074in}}%
\pgfpathlineto{\pgfqpoint{5.129898in}{3.996856in}}%
\pgfpathlineto{\pgfqpoint{5.132222in}{3.968975in}}%
\pgfpathlineto{\pgfqpoint{5.134546in}{4.033436in}}%
\pgfpathlineto{\pgfqpoint{5.136870in}{4.008854in}}%
\pgfpathlineto{\pgfqpoint{5.141518in}{4.031018in}}%
\pgfpathlineto{\pgfqpoint{5.143842in}{3.983923in}}%
\pgfpathlineto{\pgfqpoint{5.146166in}{4.008383in}}%
\pgfpathlineto{\pgfqpoint{5.148490in}{4.020966in}}%
\pgfpathlineto{\pgfqpoint{5.150814in}{4.040800in}}%
\pgfpathlineto{\pgfqpoint{5.153138in}{3.994407in}}%
\pgfpathlineto{\pgfqpoint{5.155462in}{4.010898in}}%
\pgfpathlineto{\pgfqpoint{5.157786in}{4.040702in}}%
\pgfpathlineto{\pgfqpoint{5.160110in}{3.951363in}}%
\pgfpathlineto{\pgfqpoint{5.162434in}{4.005763in}}%
\pgfpathlineto{\pgfqpoint{5.164758in}{4.022216in}}%
\pgfpathlineto{\pgfqpoint{5.167081in}{4.087837in}}%
\pgfpathlineto{\pgfqpoint{5.169405in}{4.010912in}}%
\pgfpathlineto{\pgfqpoint{5.171729in}{4.054510in}}%
\pgfpathlineto{\pgfqpoint{5.174053in}{4.028036in}}%
\pgfpathlineto{\pgfqpoint{5.176377in}{4.082320in}}%
\pgfpathlineto{\pgfqpoint{5.178701in}{4.090741in}}%
\pgfpathlineto{\pgfqpoint{5.181025in}{3.958676in}}%
\pgfpathlineto{\pgfqpoint{5.183349in}{4.037553in}}%
\pgfpathlineto{\pgfqpoint{5.185673in}{4.035049in}}%
\pgfpathlineto{\pgfqpoint{5.187997in}{4.058181in}}%
\pgfpathlineto{\pgfqpoint{5.190321in}{4.007110in}}%
\pgfpathlineto{\pgfqpoint{5.192645in}{4.089288in}}%
\pgfpathlineto{\pgfqpoint{5.194969in}{4.107292in}}%
\pgfpathlineto{\pgfqpoint{5.197293in}{4.079369in}}%
\pgfpathlineto{\pgfqpoint{5.199617in}{4.015752in}}%
\pgfpathlineto{\pgfqpoint{5.201941in}{4.048335in}}%
\pgfpathlineto{\pgfqpoint{5.204265in}{4.099546in}}%
\pgfpathlineto{\pgfqpoint{5.206589in}{4.087391in}}%
\pgfpathlineto{\pgfqpoint{5.208913in}{4.064569in}}%
\pgfpathlineto{\pgfqpoint{5.211237in}{4.085197in}}%
\pgfpathlineto{\pgfqpoint{5.213561in}{4.085309in}}%
\pgfpathlineto{\pgfqpoint{5.215885in}{4.079620in}}%
\pgfpathlineto{\pgfqpoint{5.218209in}{4.097497in}}%
\pgfpathlineto{\pgfqpoint{5.220533in}{4.099244in}}%
\pgfpathlineto{\pgfqpoint{5.225181in}{4.055140in}}%
\pgfpathlineto{\pgfqpoint{5.227505in}{4.073680in}}%
\pgfpathlineto{\pgfqpoint{5.229829in}{4.077334in}}%
\pgfpathlineto{\pgfqpoint{5.232153in}{4.121076in}}%
\pgfpathlineto{\pgfqpoint{5.234477in}{4.131906in}}%
\pgfpathlineto{\pgfqpoint{5.236800in}{4.032720in}}%
\pgfpathlineto{\pgfqpoint{5.241448in}{4.168694in}}%
\pgfpathlineto{\pgfqpoint{5.243772in}{4.152801in}}%
\pgfpathlineto{\pgfqpoint{5.246096in}{4.156146in}}%
\pgfpathlineto{\pgfqpoint{5.248420in}{4.093632in}}%
\pgfpathlineto{\pgfqpoint{5.250744in}{4.097591in}}%
\pgfpathlineto{\pgfqpoint{5.253068in}{4.105281in}}%
\pgfpathlineto{\pgfqpoint{5.255392in}{4.179568in}}%
\pgfpathlineto{\pgfqpoint{5.257716in}{4.140431in}}%
\pgfpathlineto{\pgfqpoint{5.260040in}{4.147842in}}%
\pgfpathlineto{\pgfqpoint{5.262364in}{4.128565in}}%
\pgfpathlineto{\pgfqpoint{5.264688in}{4.199917in}}%
\pgfpathlineto{\pgfqpoint{5.267012in}{4.142704in}}%
\pgfpathlineto{\pgfqpoint{5.269336in}{4.170737in}}%
\pgfpathlineto{\pgfqpoint{5.273984in}{4.131725in}}%
\pgfpathlineto{\pgfqpoint{5.276308in}{4.162755in}}%
\pgfpathlineto{\pgfqpoint{5.278632in}{4.141138in}}%
\pgfpathlineto{\pgfqpoint{5.280956in}{4.199834in}}%
\pgfpathlineto{\pgfqpoint{5.285604in}{4.147467in}}%
\pgfpathlineto{\pgfqpoint{5.287928in}{4.163365in}}%
\pgfpathlineto{\pgfqpoint{5.290252in}{4.157400in}}%
\pgfpathlineto{\pgfqpoint{5.292576in}{4.196153in}}%
\pgfpathlineto{\pgfqpoint{5.294900in}{4.167741in}}%
\pgfpathlineto{\pgfqpoint{5.297224in}{4.188200in}}%
\pgfpathlineto{\pgfqpoint{5.299548in}{4.150604in}}%
\pgfpathlineto{\pgfqpoint{5.304196in}{4.196030in}}%
\pgfpathlineto{\pgfqpoint{5.306519in}{4.203257in}}%
\pgfpathlineto{\pgfqpoint{5.308843in}{4.156751in}}%
\pgfpathlineto{\pgfqpoint{5.311167in}{4.175583in}}%
\pgfpathlineto{\pgfqpoint{5.313491in}{4.220398in}}%
\pgfpathlineto{\pgfqpoint{5.315815in}{4.227070in}}%
\pgfpathlineto{\pgfqpoint{5.318139in}{4.225219in}}%
\pgfpathlineto{\pgfqpoint{5.320463in}{4.238065in}}%
\pgfpathlineto{\pgfqpoint{5.322787in}{4.304930in}}%
\pgfpathlineto{\pgfqpoint{5.325111in}{4.175236in}}%
\pgfpathlineto{\pgfqpoint{5.327435in}{4.191621in}}%
\pgfpathlineto{\pgfqpoint{5.329759in}{4.305868in}}%
\pgfpathlineto{\pgfqpoint{5.332083in}{4.188380in}}%
\pgfpathlineto{\pgfqpoint{5.334407in}{4.181326in}}%
\pgfpathlineto{\pgfqpoint{5.336731in}{4.259360in}}%
\pgfpathlineto{\pgfqpoint{5.339055in}{4.276166in}}%
\pgfpathlineto{\pgfqpoint{5.343703in}{4.208417in}}%
\pgfpathlineto{\pgfqpoint{5.346027in}{4.247027in}}%
\pgfpathlineto{\pgfqpoint{5.348351in}{4.241679in}}%
\pgfpathlineto{\pgfqpoint{5.350675in}{4.199583in}}%
\pgfpathlineto{\pgfqpoint{5.352999in}{4.268191in}}%
\pgfpathlineto{\pgfqpoint{5.355323in}{4.245316in}}%
\pgfpathlineto{\pgfqpoint{5.357647in}{4.237072in}}%
\pgfpathlineto{\pgfqpoint{5.362295in}{4.255593in}}%
\pgfpathlineto{\pgfqpoint{5.364619in}{4.241545in}}%
\pgfpathlineto{\pgfqpoint{5.366943in}{4.329652in}}%
\pgfpathlineto{\pgfqpoint{5.369267in}{4.270669in}}%
\pgfpathlineto{\pgfqpoint{5.371591in}{4.266070in}}%
\pgfpathlineto{\pgfqpoint{5.373915in}{4.283820in}}%
\pgfpathlineto{\pgfqpoint{5.376239in}{4.264603in}}%
\pgfpathlineto{\pgfqpoint{5.378562in}{4.273607in}}%
\pgfpathlineto{\pgfqpoint{5.380886in}{4.294625in}}%
\pgfpathlineto{\pgfqpoint{5.383210in}{4.219480in}}%
\pgfpathlineto{\pgfqpoint{5.385534in}{4.300768in}}%
\pgfpathlineto{\pgfqpoint{5.387858in}{4.239739in}}%
\pgfpathlineto{\pgfqpoint{5.390182in}{4.321805in}}%
\pgfpathlineto{\pgfqpoint{5.392506in}{4.298858in}}%
\pgfpathlineto{\pgfqpoint{5.394830in}{4.339586in}}%
\pgfpathlineto{\pgfqpoint{5.399478in}{4.315886in}}%
\pgfpathlineto{\pgfqpoint{5.401802in}{4.272607in}}%
\pgfpathlineto{\pgfqpoint{5.404126in}{4.292477in}}%
\pgfpathlineto{\pgfqpoint{5.406450in}{4.274433in}}%
\pgfpathlineto{\pgfqpoint{5.408774in}{4.331793in}}%
\pgfpathlineto{\pgfqpoint{5.411098in}{4.352609in}}%
\pgfpathlineto{\pgfqpoint{5.413422in}{4.305298in}}%
\pgfpathlineto{\pgfqpoint{5.415746in}{4.228956in}}%
\pgfpathlineto{\pgfqpoint{5.418070in}{4.329159in}}%
\pgfpathlineto{\pgfqpoint{5.420394in}{4.254944in}}%
\pgfpathlineto{\pgfqpoint{5.422718in}{4.294771in}}%
\pgfpathlineto{\pgfqpoint{5.425042in}{4.309528in}}%
\pgfpathlineto{\pgfqpoint{5.429690in}{4.371043in}}%
\pgfpathlineto{\pgfqpoint{5.432014in}{4.314401in}}%
\pgfpathlineto{\pgfqpoint{5.434338in}{4.357420in}}%
\pgfpathlineto{\pgfqpoint{5.436662in}{4.322081in}}%
\pgfpathlineto{\pgfqpoint{5.438986in}{4.376801in}}%
\pgfpathlineto{\pgfqpoint{5.441310in}{4.356939in}}%
\pgfpathlineto{\pgfqpoint{5.443634in}{4.285371in}}%
\pgfpathlineto{\pgfqpoint{5.445958in}{4.339590in}}%
\pgfpathlineto{\pgfqpoint{5.448281in}{4.325761in}}%
\pgfpathlineto{\pgfqpoint{5.450605in}{4.410978in}}%
\pgfpathlineto{\pgfqpoint{5.452929in}{4.310150in}}%
\pgfpathlineto{\pgfqpoint{5.455253in}{4.360076in}}%
\pgfpathlineto{\pgfqpoint{5.457577in}{4.354195in}}%
\pgfpathlineto{\pgfqpoint{5.459901in}{4.412863in}}%
\pgfpathlineto{\pgfqpoint{5.462225in}{4.359657in}}%
\pgfpathlineto{\pgfqpoint{5.464549in}{4.377055in}}%
\pgfpathlineto{\pgfqpoint{5.466873in}{4.305272in}}%
\pgfpathlineto{\pgfqpoint{5.469197in}{4.389488in}}%
\pgfpathlineto{\pgfqpoint{5.471521in}{4.345999in}}%
\pgfpathlineto{\pgfqpoint{5.473845in}{4.456062in}}%
\pgfpathlineto{\pgfqpoint{5.476169in}{4.385764in}}%
\pgfpathlineto{\pgfqpoint{5.480817in}{4.368299in}}%
\pgfpathlineto{\pgfqpoint{5.485465in}{4.427619in}}%
\pgfpathlineto{\pgfqpoint{5.487789in}{4.406323in}}%
\pgfpathlineto{\pgfqpoint{5.490113in}{4.407408in}}%
\pgfpathlineto{\pgfqpoint{5.492437in}{4.374100in}}%
\pgfpathlineto{\pgfqpoint{5.494761in}{4.384317in}}%
\pgfpathlineto{\pgfqpoint{5.497085in}{4.361011in}}%
\pgfpathlineto{\pgfqpoint{5.499409in}{4.450578in}}%
\pgfpathlineto{\pgfqpoint{5.501733in}{4.403544in}}%
\pgfpathlineto{\pgfqpoint{5.506381in}{4.390594in}}%
\pgfpathlineto{\pgfqpoint{5.508705in}{4.448573in}}%
\pgfpathlineto{\pgfqpoint{5.511029in}{4.430958in}}%
\pgfpathlineto{\pgfqpoint{5.513353in}{4.428642in}}%
\pgfpathlineto{\pgfqpoint{5.515677in}{4.429417in}}%
\pgfpathlineto{\pgfqpoint{5.518000in}{4.413495in}}%
\pgfpathlineto{\pgfqpoint{5.520324in}{4.436182in}}%
\pgfpathlineto{\pgfqpoint{5.522648in}{4.477669in}}%
\pgfpathlineto{\pgfqpoint{5.524972in}{4.378598in}}%
\pgfpathlineto{\pgfqpoint{5.527296in}{4.451351in}}%
\pgfpathlineto{\pgfqpoint{5.529620in}{4.442840in}}%
\pgfpathlineto{\pgfqpoint{5.531944in}{4.385993in}}%
\pgfpathlineto{\pgfqpoint{5.534268in}{4.441814in}}%
\pgfpathlineto{\pgfqpoint{5.536592in}{4.407773in}}%
\pgfpathlineto{\pgfqpoint{5.538916in}{4.448417in}}%
\pgfpathlineto{\pgfqpoint{5.541240in}{4.442526in}}%
\pgfpathlineto{\pgfqpoint{5.545888in}{4.480487in}}%
\pgfpathlineto{\pgfqpoint{5.548212in}{4.466897in}}%
\pgfpathlineto{\pgfqpoint{5.550536in}{4.430940in}}%
\pgfpathlineto{\pgfqpoint{5.552860in}{4.513013in}}%
\pgfpathlineto{\pgfqpoint{5.555184in}{4.482245in}}%
\pgfpathlineto{\pgfqpoint{5.557508in}{4.571415in}}%
\pgfpathlineto{\pgfqpoint{5.559832in}{4.444718in}}%
\pgfpathlineto{\pgfqpoint{5.562156in}{4.468163in}}%
\pgfpathlineto{\pgfqpoint{5.564480in}{4.469829in}}%
\pgfpathlineto{\pgfqpoint{5.566804in}{4.426770in}}%
\pgfpathlineto{\pgfqpoint{5.569128in}{4.491445in}}%
\pgfpathlineto{\pgfqpoint{5.571452in}{4.456315in}}%
\pgfpathlineto{\pgfqpoint{5.573776in}{4.460610in}}%
\pgfpathlineto{\pgfqpoint{5.576100in}{4.413388in}}%
\pgfpathlineto{\pgfqpoint{5.578424in}{4.508712in}}%
\pgfpathlineto{\pgfqpoint{5.583072in}{4.551578in}}%
\pgfpathlineto{\pgfqpoint{5.585396in}{4.465849in}}%
\pgfpathlineto{\pgfqpoint{5.587720in}{3.864851in}}%
\pgfpathlineto{\pgfqpoint{5.587720in}{3.864851in}}%
\pgfusepath{stroke}%
\end{pgfscope}%
\begin{pgfscope}%
\pgfpathrectangle{\pgfqpoint{0.709829in}{3.729963in}}{\pgfqpoint{5.110171in}{0.887537in}}%
\pgfusepath{clip}%
\pgfsetroundcap%
\pgfsetroundjoin%
\pgfsetlinewidth{1.003750pt}%
\definecolor{currentstroke}{rgb}{0.333333,0.658824,0.407843}%
\pgfsetstrokecolor{currentstroke}%
\pgfsetdash{}{0pt}%
\pgfpathmoveto{\pgfqpoint{0.942110in}{4.392026in}}%
\pgfpathlineto{\pgfqpoint{0.944433in}{4.354634in}}%
\pgfpathlineto{\pgfqpoint{0.946757in}{4.371343in}}%
\pgfpathlineto{\pgfqpoint{0.949081in}{4.401117in}}%
\pgfpathlineto{\pgfqpoint{0.953729in}{4.369394in}}%
\pgfpathlineto{\pgfqpoint{0.956053in}{4.324512in}}%
\pgfpathlineto{\pgfqpoint{0.958377in}{4.359129in}}%
\pgfpathlineto{\pgfqpoint{0.960701in}{4.374191in}}%
\pgfpathlineto{\pgfqpoint{0.963025in}{4.401405in}}%
\pgfpathlineto{\pgfqpoint{0.965349in}{4.356345in}}%
\pgfpathlineto{\pgfqpoint{0.967673in}{4.376305in}}%
\pgfpathlineto{\pgfqpoint{0.969997in}{4.365912in}}%
\pgfpathlineto{\pgfqpoint{0.972321in}{4.346171in}}%
\pgfpathlineto{\pgfqpoint{0.976969in}{4.388179in}}%
\pgfpathlineto{\pgfqpoint{0.979293in}{4.356198in}}%
\pgfpathlineto{\pgfqpoint{0.981617in}{4.350747in}}%
\pgfpathlineto{\pgfqpoint{0.983941in}{4.374588in}}%
\pgfpathlineto{\pgfqpoint{0.988589in}{4.357772in}}%
\pgfpathlineto{\pgfqpoint{0.990913in}{4.365636in}}%
\pgfpathlineto{\pgfqpoint{0.993237in}{4.355972in}}%
\pgfpathlineto{\pgfqpoint{0.995561in}{4.375782in}}%
\pgfpathlineto{\pgfqpoint{0.997885in}{4.381008in}}%
\pgfpathlineto{\pgfqpoint{1.002533in}{4.351043in}}%
\pgfpathlineto{\pgfqpoint{1.004857in}{4.343949in}}%
\pgfpathlineto{\pgfqpoint{1.007181in}{4.395850in}}%
\pgfpathlineto{\pgfqpoint{1.009505in}{4.393474in}}%
\pgfpathlineto{\pgfqpoint{1.011829in}{4.377298in}}%
\pgfpathlineto{\pgfqpoint{1.014153in}{4.386823in}}%
\pgfpathlineto{\pgfqpoint{1.016476in}{4.375742in}}%
\pgfpathlineto{\pgfqpoint{1.018800in}{4.347552in}}%
\pgfpathlineto{\pgfqpoint{1.021124in}{4.350890in}}%
\pgfpathlineto{\pgfqpoint{1.023448in}{4.359028in}}%
\pgfpathlineto{\pgfqpoint{1.025772in}{4.409017in}}%
\pgfpathlineto{\pgfqpoint{1.028096in}{4.409830in}}%
\pgfpathlineto{\pgfqpoint{1.030420in}{4.391424in}}%
\pgfpathlineto{\pgfqpoint{1.032744in}{4.390946in}}%
\pgfpathlineto{\pgfqpoint{1.037392in}{4.370456in}}%
\pgfpathlineto{\pgfqpoint{1.039716in}{4.386168in}}%
\pgfpathlineto{\pgfqpoint{1.042040in}{4.383625in}}%
\pgfpathlineto{\pgfqpoint{1.044364in}{4.361161in}}%
\pgfpathlineto{\pgfqpoint{1.046688in}{4.369334in}}%
\pgfpathlineto{\pgfqpoint{1.049012in}{4.344809in}}%
\pgfpathlineto{\pgfqpoint{1.051336in}{4.385456in}}%
\pgfpathlineto{\pgfqpoint{1.053660in}{4.355866in}}%
\pgfpathlineto{\pgfqpoint{1.055984in}{4.364492in}}%
\pgfpathlineto{\pgfqpoint{1.058308in}{4.359079in}}%
\pgfpathlineto{\pgfqpoint{1.060632in}{4.360633in}}%
\pgfpathlineto{\pgfqpoint{1.062956in}{4.351381in}}%
\pgfpathlineto{\pgfqpoint{1.065280in}{4.410202in}}%
\pgfpathlineto{\pgfqpoint{1.067604in}{4.349890in}}%
\pgfpathlineto{\pgfqpoint{1.069928in}{4.343786in}}%
\pgfpathlineto{\pgfqpoint{1.072252in}{4.390920in}}%
\pgfpathlineto{\pgfqpoint{1.074576in}{4.360871in}}%
\pgfpathlineto{\pgfqpoint{1.076900in}{4.357958in}}%
\pgfpathlineto{\pgfqpoint{1.079224in}{4.373650in}}%
\pgfpathlineto{\pgfqpoint{1.083872in}{4.342691in}}%
\pgfpathlineto{\pgfqpoint{1.088519in}{4.378421in}}%
\pgfpathlineto{\pgfqpoint{1.093167in}{4.389059in}}%
\pgfpathlineto{\pgfqpoint{1.095491in}{4.366259in}}%
\pgfpathlineto{\pgfqpoint{1.097815in}{4.385843in}}%
\pgfpathlineto{\pgfqpoint{1.100139in}{4.372851in}}%
\pgfpathlineto{\pgfqpoint{1.102463in}{4.371440in}}%
\pgfpathlineto{\pgfqpoint{1.104787in}{4.388348in}}%
\pgfpathlineto{\pgfqpoint{1.107111in}{4.365478in}}%
\pgfpathlineto{\pgfqpoint{1.109435in}{4.416506in}}%
\pgfpathlineto{\pgfqpoint{1.111759in}{4.366702in}}%
\pgfpathlineto{\pgfqpoint{1.114083in}{4.343862in}}%
\pgfpathlineto{\pgfqpoint{1.116407in}{4.377755in}}%
\pgfpathlineto{\pgfqpoint{1.118731in}{4.346093in}}%
\pgfpathlineto{\pgfqpoint{1.123379in}{4.401886in}}%
\pgfpathlineto{\pgfqpoint{1.128027in}{4.370776in}}%
\pgfpathlineto{\pgfqpoint{1.130351in}{4.341277in}}%
\pgfpathlineto{\pgfqpoint{1.132675in}{4.357768in}}%
\pgfpathlineto{\pgfqpoint{1.134999in}{4.362309in}}%
\pgfpathlineto{\pgfqpoint{1.137323in}{4.386536in}}%
\pgfpathlineto{\pgfqpoint{1.139647in}{4.346871in}}%
\pgfpathlineto{\pgfqpoint{1.141971in}{4.350590in}}%
\pgfpathlineto{\pgfqpoint{1.144295in}{4.372311in}}%
\pgfpathlineto{\pgfqpoint{1.146619in}{4.383331in}}%
\pgfpathlineto{\pgfqpoint{1.148943in}{4.375178in}}%
\pgfpathlineto{\pgfqpoint{1.151267in}{4.410706in}}%
\pgfpathlineto{\pgfqpoint{1.155914in}{4.334497in}}%
\pgfpathlineto{\pgfqpoint{1.158238in}{4.392818in}}%
\pgfpathlineto{\pgfqpoint{1.160562in}{4.378790in}}%
\pgfpathlineto{\pgfqpoint{1.162886in}{4.396469in}}%
\pgfpathlineto{\pgfqpoint{1.165210in}{4.361309in}}%
\pgfpathlineto{\pgfqpoint{1.167534in}{4.353053in}}%
\pgfpathlineto{\pgfqpoint{1.169858in}{4.375184in}}%
\pgfpathlineto{\pgfqpoint{1.172182in}{4.380390in}}%
\pgfpathlineto{\pgfqpoint{1.176830in}{4.364509in}}%
\pgfpathlineto{\pgfqpoint{1.179154in}{4.386579in}}%
\pgfpathlineto{\pgfqpoint{1.181478in}{4.393919in}}%
\pgfpathlineto{\pgfqpoint{1.183802in}{4.365959in}}%
\pgfpathlineto{\pgfqpoint{1.186126in}{4.367832in}}%
\pgfpathlineto{\pgfqpoint{1.188450in}{4.359560in}}%
\pgfpathlineto{\pgfqpoint{1.190774in}{4.356285in}}%
\pgfpathlineto{\pgfqpoint{1.193098in}{4.370302in}}%
\pgfpathlineto{\pgfqpoint{1.195422in}{4.391348in}}%
\pgfpathlineto{\pgfqpoint{1.200070in}{4.384907in}}%
\pgfpathlineto{\pgfqpoint{1.202394in}{4.360240in}}%
\pgfpathlineto{\pgfqpoint{1.204718in}{4.364649in}}%
\pgfpathlineto{\pgfqpoint{1.207042in}{4.364887in}}%
\pgfpathlineto{\pgfqpoint{1.209366in}{4.376312in}}%
\pgfpathlineto{\pgfqpoint{1.211690in}{4.358709in}}%
\pgfpathlineto{\pgfqpoint{1.214014in}{4.360136in}}%
\pgfpathlineto{\pgfqpoint{1.216338in}{4.338083in}}%
\pgfpathlineto{\pgfqpoint{1.218662in}{4.385175in}}%
\pgfpathlineto{\pgfqpoint{1.220986in}{4.391130in}}%
\pgfpathlineto{\pgfqpoint{1.225633in}{4.368198in}}%
\pgfpathlineto{\pgfqpoint{1.227957in}{4.354331in}}%
\pgfpathlineto{\pgfqpoint{1.230281in}{4.374262in}}%
\pgfpathlineto{\pgfqpoint{1.232605in}{3.960963in}}%
\pgfpathlineto{\pgfqpoint{1.234929in}{3.958526in}}%
\pgfpathlineto{\pgfqpoint{1.237253in}{3.971441in}}%
\pgfpathlineto{\pgfqpoint{1.239577in}{3.972148in}}%
\pgfpathlineto{\pgfqpoint{1.241901in}{3.994127in}}%
\pgfpathlineto{\pgfqpoint{1.246549in}{3.976431in}}%
\pgfpathlineto{\pgfqpoint{1.248873in}{4.019023in}}%
\pgfpathlineto{\pgfqpoint{1.251197in}{3.969656in}}%
\pgfpathlineto{\pgfqpoint{1.253521in}{3.957179in}}%
\pgfpathlineto{\pgfqpoint{1.255845in}{3.992114in}}%
\pgfpathlineto{\pgfqpoint{1.258169in}{3.963335in}}%
\pgfpathlineto{\pgfqpoint{1.260493in}{3.986525in}}%
\pgfpathlineto{\pgfqpoint{1.262817in}{3.998076in}}%
\pgfpathlineto{\pgfqpoint{1.265141in}{4.023860in}}%
\pgfpathlineto{\pgfqpoint{1.267465in}{3.950946in}}%
\pgfpathlineto{\pgfqpoint{1.269789in}{3.992784in}}%
\pgfpathlineto{\pgfqpoint{1.274437in}{3.963901in}}%
\pgfpathlineto{\pgfqpoint{1.276761in}{3.972424in}}%
\pgfpathlineto{\pgfqpoint{1.279085in}{3.991089in}}%
\pgfpathlineto{\pgfqpoint{1.281409in}{3.983432in}}%
\pgfpathlineto{\pgfqpoint{1.283733in}{3.994702in}}%
\pgfpathlineto{\pgfqpoint{1.286057in}{3.999732in}}%
\pgfpathlineto{\pgfqpoint{1.288381in}{3.988877in}}%
\pgfpathlineto{\pgfqpoint{1.290705in}{3.999841in}}%
\pgfpathlineto{\pgfqpoint{1.293029in}{4.029623in}}%
\pgfpathlineto{\pgfqpoint{1.297676in}{3.972276in}}%
\pgfpathlineto{\pgfqpoint{1.300000in}{3.981376in}}%
\pgfpathlineto{\pgfqpoint{1.302324in}{3.975826in}}%
\pgfpathlineto{\pgfqpoint{1.304648in}{3.952441in}}%
\pgfpathlineto{\pgfqpoint{1.306972in}{3.977420in}}%
\pgfpathlineto{\pgfqpoint{1.309296in}{4.016205in}}%
\pgfpathlineto{\pgfqpoint{1.311620in}{4.000469in}}%
\pgfpathlineto{\pgfqpoint{1.316268in}{3.940222in}}%
\pgfpathlineto{\pgfqpoint{1.318592in}{3.990114in}}%
\pgfpathlineto{\pgfqpoint{1.320916in}{3.984298in}}%
\pgfpathlineto{\pgfqpoint{1.323240in}{4.042530in}}%
\pgfpathlineto{\pgfqpoint{1.325564in}{3.981157in}}%
\pgfpathlineto{\pgfqpoint{1.327888in}{3.998304in}}%
\pgfpathlineto{\pgfqpoint{1.330212in}{3.986510in}}%
\pgfpathlineto{\pgfqpoint{1.332536in}{3.959777in}}%
\pgfpathlineto{\pgfqpoint{1.334860in}{4.015866in}}%
\pgfpathlineto{\pgfqpoint{1.337184in}{3.965520in}}%
\pgfpathlineto{\pgfqpoint{1.339508in}{3.994349in}}%
\pgfpathlineto{\pgfqpoint{1.341832in}{3.996422in}}%
\pgfpathlineto{\pgfqpoint{1.344156in}{3.973533in}}%
\pgfpathlineto{\pgfqpoint{1.348804in}{4.017986in}}%
\pgfpathlineto{\pgfqpoint{1.351128in}{3.948174in}}%
\pgfpathlineto{\pgfqpoint{1.353452in}{3.992227in}}%
\pgfpathlineto{\pgfqpoint{1.355776in}{3.985986in}}%
\pgfpathlineto{\pgfqpoint{1.358100in}{3.988887in}}%
\pgfpathlineto{\pgfqpoint{1.362748in}{3.982503in}}%
\pgfpathlineto{\pgfqpoint{1.365072in}{4.013571in}}%
\pgfpathlineto{\pgfqpoint{1.369719in}{3.978026in}}%
\pgfpathlineto{\pgfqpoint{1.372043in}{3.991376in}}%
\pgfpathlineto{\pgfqpoint{1.374367in}{4.029436in}}%
\pgfpathlineto{\pgfqpoint{1.376691in}{3.956930in}}%
\pgfpathlineto{\pgfqpoint{1.379015in}{3.997769in}}%
\pgfpathlineto{\pgfqpoint{1.381339in}{4.001062in}}%
\pgfpathlineto{\pgfqpoint{1.385987in}{3.984032in}}%
\pgfpathlineto{\pgfqpoint{1.388311in}{3.988482in}}%
\pgfpathlineto{\pgfqpoint{1.390635in}{3.978373in}}%
\pgfpathlineto{\pgfqpoint{1.392959in}{3.980391in}}%
\pgfpathlineto{\pgfqpoint{1.395283in}{3.999502in}}%
\pgfpathlineto{\pgfqpoint{1.397607in}{4.001374in}}%
\pgfpathlineto{\pgfqpoint{1.399931in}{3.970265in}}%
\pgfpathlineto{\pgfqpoint{1.402255in}{3.982002in}}%
\pgfpathlineto{\pgfqpoint{1.404579in}{3.987574in}}%
\pgfpathlineto{\pgfqpoint{1.406903in}{3.972935in}}%
\pgfpathlineto{\pgfqpoint{1.409227in}{3.979708in}}%
\pgfpathlineto{\pgfqpoint{1.411551in}{3.993189in}}%
\pgfpathlineto{\pgfqpoint{1.413875in}{3.966074in}}%
\pgfpathlineto{\pgfqpoint{1.416199in}{3.988466in}}%
\pgfpathlineto{\pgfqpoint{1.420847in}{4.003919in}}%
\pgfpathlineto{\pgfqpoint{1.423171in}{3.985210in}}%
\pgfpathlineto{\pgfqpoint{1.425495in}{3.994492in}}%
\pgfpathlineto{\pgfqpoint{1.427819in}{3.990961in}}%
\pgfpathlineto{\pgfqpoint{1.430143in}{3.999899in}}%
\pgfpathlineto{\pgfqpoint{1.432467in}{3.998697in}}%
\pgfpathlineto{\pgfqpoint{1.437114in}{4.017131in}}%
\pgfpathlineto{\pgfqpoint{1.439438in}{3.967705in}}%
\pgfpathlineto{\pgfqpoint{1.441762in}{3.989606in}}%
\pgfpathlineto{\pgfqpoint{1.446410in}{4.000929in}}%
\pgfpathlineto{\pgfqpoint{1.448734in}{3.989702in}}%
\pgfpathlineto{\pgfqpoint{1.451058in}{3.972911in}}%
\pgfpathlineto{\pgfqpoint{1.453382in}{3.999638in}}%
\pgfpathlineto{\pgfqpoint{1.458030in}{3.952674in}}%
\pgfpathlineto{\pgfqpoint{1.460354in}{3.954961in}}%
\pgfpathlineto{\pgfqpoint{1.462678in}{3.995184in}}%
\pgfpathlineto{\pgfqpoint{1.465002in}{4.010429in}}%
\pgfpathlineto{\pgfqpoint{1.467326in}{3.979682in}}%
\pgfpathlineto{\pgfqpoint{1.471974in}{4.031453in}}%
\pgfpathlineto{\pgfqpoint{1.474298in}{4.036499in}}%
\pgfpathlineto{\pgfqpoint{1.476622in}{3.962267in}}%
\pgfpathlineto{\pgfqpoint{1.478946in}{4.011839in}}%
\pgfpathlineto{\pgfqpoint{1.481270in}{3.994772in}}%
\pgfpathlineto{\pgfqpoint{1.483594in}{3.967846in}}%
\pgfpathlineto{\pgfqpoint{1.485918in}{3.995477in}}%
\pgfpathlineto{\pgfqpoint{1.488242in}{3.990317in}}%
\pgfpathlineto{\pgfqpoint{1.490566in}{3.977512in}}%
\pgfpathlineto{\pgfqpoint{1.492890in}{3.985436in}}%
\pgfpathlineto{\pgfqpoint{1.495214in}{3.975634in}}%
\pgfpathlineto{\pgfqpoint{1.497538in}{3.980425in}}%
\pgfpathlineto{\pgfqpoint{1.499862in}{3.994504in}}%
\pgfpathlineto{\pgfqpoint{1.502186in}{3.974028in}}%
\pgfpathlineto{\pgfqpoint{1.504510in}{4.006523in}}%
\pgfpathlineto{\pgfqpoint{1.506834in}{3.975393in}}%
\pgfpathlineto{\pgfqpoint{1.509157in}{4.038725in}}%
\pgfpathlineto{\pgfqpoint{1.511481in}{3.966432in}}%
\pgfpathlineto{\pgfqpoint{1.513805in}{3.968476in}}%
\pgfpathlineto{\pgfqpoint{1.516129in}{3.988445in}}%
\pgfpathlineto{\pgfqpoint{1.518453in}{3.991185in}}%
\pgfpathlineto{\pgfqpoint{1.520777in}{3.996791in}}%
\pgfpathlineto{\pgfqpoint{1.523101in}{4.381884in}}%
\pgfpathlineto{\pgfqpoint{1.525425in}{4.389019in}}%
\pgfpathlineto{\pgfqpoint{1.527749in}{4.374076in}}%
\pgfpathlineto{\pgfqpoint{1.530073in}{4.378395in}}%
\pgfpathlineto{\pgfqpoint{1.532397in}{4.387080in}}%
\pgfpathlineto{\pgfqpoint{1.537045in}{4.363777in}}%
\pgfpathlineto{\pgfqpoint{1.539369in}{4.377001in}}%
\pgfpathlineto{\pgfqpoint{1.541693in}{4.372196in}}%
\pgfpathlineto{\pgfqpoint{1.544017in}{4.372033in}}%
\pgfpathlineto{\pgfqpoint{1.546341in}{4.385014in}}%
\pgfpathlineto{\pgfqpoint{1.548665in}{4.376154in}}%
\pgfpathlineto{\pgfqpoint{1.550989in}{4.347322in}}%
\pgfpathlineto{\pgfqpoint{1.553313in}{4.372359in}}%
\pgfpathlineto{\pgfqpoint{1.555637in}{4.356896in}}%
\pgfpathlineto{\pgfqpoint{1.557961in}{4.352726in}}%
\pgfpathlineto{\pgfqpoint{1.560285in}{4.400293in}}%
\pgfpathlineto{\pgfqpoint{1.562609in}{4.365838in}}%
\pgfpathlineto{\pgfqpoint{1.564933in}{4.382837in}}%
\pgfpathlineto{\pgfqpoint{1.567257in}{4.371166in}}%
\pgfpathlineto{\pgfqpoint{1.571905in}{4.406500in}}%
\pgfpathlineto{\pgfqpoint{1.574229in}{4.357018in}}%
\pgfpathlineto{\pgfqpoint{1.576553in}{4.373536in}}%
\pgfpathlineto{\pgfqpoint{1.578876in}{4.360552in}}%
\pgfpathlineto{\pgfqpoint{1.581200in}{4.415868in}}%
\pgfpathlineto{\pgfqpoint{1.583524in}{4.379527in}}%
\pgfpathlineto{\pgfqpoint{1.585848in}{4.371985in}}%
\pgfpathlineto{\pgfqpoint{1.588172in}{4.357099in}}%
\pgfpathlineto{\pgfqpoint{1.592820in}{4.391908in}}%
\pgfpathlineto{\pgfqpoint{1.595144in}{4.369623in}}%
\pgfpathlineto{\pgfqpoint{1.599792in}{4.351257in}}%
\pgfpathlineto{\pgfqpoint{1.602116in}{4.353576in}}%
\pgfpathlineto{\pgfqpoint{1.604440in}{4.365554in}}%
\pgfpathlineto{\pgfqpoint{1.606764in}{4.395380in}}%
\pgfpathlineto{\pgfqpoint{1.609088in}{4.375218in}}%
\pgfpathlineto{\pgfqpoint{1.611412in}{4.371122in}}%
\pgfpathlineto{\pgfqpoint{1.613736in}{4.389864in}}%
\pgfpathlineto{\pgfqpoint{1.618384in}{4.353030in}}%
\pgfpathlineto{\pgfqpoint{1.620708in}{4.374290in}}%
\pgfpathlineto{\pgfqpoint{1.623032in}{4.354876in}}%
\pgfpathlineto{\pgfqpoint{1.627680in}{4.363268in}}%
\pgfpathlineto{\pgfqpoint{1.630004in}{4.360944in}}%
\pgfpathlineto{\pgfqpoint{1.632328in}{4.369965in}}%
\pgfpathlineto{\pgfqpoint{1.634652in}{4.356942in}}%
\pgfpathlineto{\pgfqpoint{1.636976in}{4.365593in}}%
\pgfpathlineto{\pgfqpoint{1.639300in}{4.380342in}}%
\pgfpathlineto{\pgfqpoint{1.641624in}{4.408160in}}%
\pgfpathlineto{\pgfqpoint{1.643948in}{4.391681in}}%
\pgfpathlineto{\pgfqpoint{1.646272in}{4.346528in}}%
\pgfpathlineto{\pgfqpoint{1.648595in}{4.349868in}}%
\pgfpathlineto{\pgfqpoint{1.650919in}{4.368271in}}%
\pgfpathlineto{\pgfqpoint{1.653243in}{4.348490in}}%
\pgfpathlineto{\pgfqpoint{1.655567in}{4.377242in}}%
\pgfpathlineto{\pgfqpoint{1.657891in}{4.368032in}}%
\pgfpathlineto{\pgfqpoint{1.660215in}{4.401443in}}%
\pgfpathlineto{\pgfqpoint{1.664863in}{4.357253in}}%
\pgfpathlineto{\pgfqpoint{1.667187in}{4.358171in}}%
\pgfpathlineto{\pgfqpoint{1.669511in}{4.398126in}}%
\pgfpathlineto{\pgfqpoint{1.671835in}{4.371889in}}%
\pgfpathlineto{\pgfqpoint{1.674159in}{4.400749in}}%
\pgfpathlineto{\pgfqpoint{1.676483in}{4.355204in}}%
\pgfpathlineto{\pgfqpoint{1.678807in}{4.365471in}}%
\pgfpathlineto{\pgfqpoint{1.681131in}{4.389824in}}%
\pgfpathlineto{\pgfqpoint{1.683455in}{4.348863in}}%
\pgfpathlineto{\pgfqpoint{1.685779in}{4.378989in}}%
\pgfpathlineto{\pgfqpoint{1.688103in}{4.371886in}}%
\pgfpathlineto{\pgfqpoint{1.690427in}{4.386515in}}%
\pgfpathlineto{\pgfqpoint{1.692751in}{4.391838in}}%
\pgfpathlineto{\pgfqpoint{1.695075in}{4.411818in}}%
\pgfpathlineto{\pgfqpoint{1.697399in}{4.356799in}}%
\pgfpathlineto{\pgfqpoint{1.702047in}{4.393667in}}%
\pgfpathlineto{\pgfqpoint{1.704371in}{4.365556in}}%
\pgfpathlineto{\pgfqpoint{1.706695in}{4.367522in}}%
\pgfpathlineto{\pgfqpoint{1.709019in}{4.358034in}}%
\pgfpathlineto{\pgfqpoint{1.711343in}{4.381209in}}%
\pgfpathlineto{\pgfqpoint{1.713667in}{4.351378in}}%
\pgfpathlineto{\pgfqpoint{1.715991in}{4.390377in}}%
\pgfpathlineto{\pgfqpoint{1.718314in}{4.377676in}}%
\pgfpathlineto{\pgfqpoint{1.720638in}{4.338561in}}%
\pgfpathlineto{\pgfqpoint{1.722962in}{4.354194in}}%
\pgfpathlineto{\pgfqpoint{1.725286in}{4.403203in}}%
\pgfpathlineto{\pgfqpoint{1.727610in}{4.377603in}}%
\pgfpathlineto{\pgfqpoint{1.729934in}{4.375175in}}%
\pgfpathlineto{\pgfqpoint{1.734582in}{4.358174in}}%
\pgfpathlineto{\pgfqpoint{1.736906in}{4.394675in}}%
\pgfpathlineto{\pgfqpoint{1.739230in}{4.375948in}}%
\pgfpathlineto{\pgfqpoint{1.741554in}{4.387730in}}%
\pgfpathlineto{\pgfqpoint{1.743878in}{4.387977in}}%
\pgfpathlineto{\pgfqpoint{1.746202in}{4.405226in}}%
\pgfpathlineto{\pgfqpoint{1.750850in}{4.381642in}}%
\pgfpathlineto{\pgfqpoint{1.753174in}{4.356988in}}%
\pgfpathlineto{\pgfqpoint{1.755498in}{4.370334in}}%
\pgfpathlineto{\pgfqpoint{1.757822in}{4.376939in}}%
\pgfpathlineto{\pgfqpoint{1.760146in}{4.367566in}}%
\pgfpathlineto{\pgfqpoint{1.762470in}{4.349600in}}%
\pgfpathlineto{\pgfqpoint{1.764794in}{4.375197in}}%
\pgfpathlineto{\pgfqpoint{1.767118in}{4.373629in}}%
\pgfpathlineto{\pgfqpoint{1.769442in}{4.353598in}}%
\pgfpathlineto{\pgfqpoint{1.771766in}{4.345927in}}%
\pgfpathlineto{\pgfqpoint{1.774090in}{4.384922in}}%
\pgfpathlineto{\pgfqpoint{1.776414in}{4.379592in}}%
\pgfpathlineto{\pgfqpoint{1.778738in}{4.384727in}}%
\pgfpathlineto{\pgfqpoint{1.781062in}{4.358022in}}%
\pgfpathlineto{\pgfqpoint{1.783386in}{4.386267in}}%
\pgfpathlineto{\pgfqpoint{1.785710in}{4.368898in}}%
\pgfpathlineto{\pgfqpoint{1.788034in}{4.414408in}}%
\pgfpathlineto{\pgfqpoint{1.790357in}{4.372064in}}%
\pgfpathlineto{\pgfqpoint{1.792681in}{4.352200in}}%
\pgfpathlineto{\pgfqpoint{1.795005in}{4.358109in}}%
\pgfpathlineto{\pgfqpoint{1.797329in}{4.378459in}}%
\pgfpathlineto{\pgfqpoint{1.801977in}{4.386776in}}%
\pgfpathlineto{\pgfqpoint{1.804301in}{4.407151in}}%
\pgfpathlineto{\pgfqpoint{1.806625in}{4.398633in}}%
\pgfpathlineto{\pgfqpoint{1.811273in}{4.390930in}}%
\pgfpathlineto{\pgfqpoint{1.813597in}{3.988453in}}%
\pgfpathlineto{\pgfqpoint{1.815921in}{3.969080in}}%
\pgfpathlineto{\pgfqpoint{1.818245in}{4.001142in}}%
\pgfpathlineto{\pgfqpoint{1.820569in}{3.995139in}}%
\pgfpathlineto{\pgfqpoint{1.822893in}{3.993787in}}%
\pgfpathlineto{\pgfqpoint{1.825217in}{4.020021in}}%
\pgfpathlineto{\pgfqpoint{1.827541in}{3.985857in}}%
\pgfpathlineto{\pgfqpoint{1.829865in}{4.014336in}}%
\pgfpathlineto{\pgfqpoint{1.832189in}{3.961724in}}%
\pgfpathlineto{\pgfqpoint{1.834513in}{3.995559in}}%
\pgfpathlineto{\pgfqpoint{1.836837in}{3.996703in}}%
\pgfpathlineto{\pgfqpoint{1.839161in}{3.982274in}}%
\pgfpathlineto{\pgfqpoint{1.843809in}{4.022178in}}%
\pgfpathlineto{\pgfqpoint{1.846133in}{3.994026in}}%
\pgfpathlineto{\pgfqpoint{1.848457in}{4.005720in}}%
\pgfpathlineto{\pgfqpoint{1.853105in}{3.979029in}}%
\pgfpathlineto{\pgfqpoint{1.855429in}{3.990705in}}%
\pgfpathlineto{\pgfqpoint{1.857753in}{3.954812in}}%
\pgfpathlineto{\pgfqpoint{1.860076in}{4.011151in}}%
\pgfpathlineto{\pgfqpoint{1.862400in}{3.985696in}}%
\pgfpathlineto{\pgfqpoint{1.864724in}{3.981238in}}%
\pgfpathlineto{\pgfqpoint{1.869372in}{4.002261in}}%
\pgfpathlineto{\pgfqpoint{1.871696in}{4.021284in}}%
\pgfpathlineto{\pgfqpoint{1.874020in}{3.967773in}}%
\pgfpathlineto{\pgfqpoint{1.876344in}{4.015765in}}%
\pgfpathlineto{\pgfqpoint{1.880992in}{3.982307in}}%
\pgfpathlineto{\pgfqpoint{1.883316in}{4.001750in}}%
\pgfpathlineto{\pgfqpoint{1.885640in}{3.995820in}}%
\pgfpathlineto{\pgfqpoint{1.887964in}{3.983472in}}%
\pgfpathlineto{\pgfqpoint{1.890288in}{3.997306in}}%
\pgfpathlineto{\pgfqpoint{1.892612in}{3.996751in}}%
\pgfpathlineto{\pgfqpoint{1.897260in}{3.960732in}}%
\pgfpathlineto{\pgfqpoint{1.901908in}{3.993226in}}%
\pgfpathlineto{\pgfqpoint{1.904232in}{3.999300in}}%
\pgfpathlineto{\pgfqpoint{1.906556in}{3.983815in}}%
\pgfpathlineto{\pgfqpoint{1.908880in}{3.976071in}}%
\pgfpathlineto{\pgfqpoint{1.911204in}{4.002439in}}%
\pgfpathlineto{\pgfqpoint{1.913528in}{3.937894in}}%
\pgfpathlineto{\pgfqpoint{1.915852in}{3.982415in}}%
\pgfpathlineto{\pgfqpoint{1.918176in}{3.984301in}}%
\pgfpathlineto{\pgfqpoint{1.920500in}{3.997855in}}%
\pgfpathlineto{\pgfqpoint{1.922824in}{3.963789in}}%
\pgfpathlineto{\pgfqpoint{1.925148in}{4.017853in}}%
\pgfpathlineto{\pgfqpoint{1.927472in}{3.999066in}}%
\pgfpathlineto{\pgfqpoint{1.929795in}{3.963286in}}%
\pgfpathlineto{\pgfqpoint{1.932119in}{3.989361in}}%
\pgfpathlineto{\pgfqpoint{1.934443in}{4.000700in}}%
\pgfpathlineto{\pgfqpoint{1.936767in}{4.000991in}}%
\pgfpathlineto{\pgfqpoint{1.939091in}{4.021121in}}%
\pgfpathlineto{\pgfqpoint{1.941415in}{3.994059in}}%
\pgfpathlineto{\pgfqpoint{1.943739in}{3.980252in}}%
\pgfpathlineto{\pgfqpoint{1.946063in}{3.994032in}}%
\pgfpathlineto{\pgfqpoint{1.950711in}{3.972690in}}%
\pgfpathlineto{\pgfqpoint{1.953035in}{3.982032in}}%
\pgfpathlineto{\pgfqpoint{1.955359in}{4.002509in}}%
\pgfpathlineto{\pgfqpoint{1.957683in}{3.984334in}}%
\pgfpathlineto{\pgfqpoint{1.962331in}{3.966186in}}%
\pgfpathlineto{\pgfqpoint{1.964655in}{3.983567in}}%
\pgfpathlineto{\pgfqpoint{1.966979in}{3.986781in}}%
\pgfpathlineto{\pgfqpoint{1.969303in}{4.012881in}}%
\pgfpathlineto{\pgfqpoint{1.971627in}{3.976183in}}%
\pgfpathlineto{\pgfqpoint{1.973951in}{3.979276in}}%
\pgfpathlineto{\pgfqpoint{1.976275in}{4.019588in}}%
\pgfpathlineto{\pgfqpoint{1.978599in}{3.982051in}}%
\pgfpathlineto{\pgfqpoint{1.980923in}{3.978921in}}%
\pgfpathlineto{\pgfqpoint{1.983247in}{4.000172in}}%
\pgfpathlineto{\pgfqpoint{1.987895in}{4.001313in}}%
\pgfpathlineto{\pgfqpoint{1.990219in}{3.987378in}}%
\pgfpathlineto{\pgfqpoint{1.992543in}{3.956004in}}%
\pgfpathlineto{\pgfqpoint{1.994867in}{4.020158in}}%
\pgfpathlineto{\pgfqpoint{1.997191in}{3.967043in}}%
\pgfpathlineto{\pgfqpoint{1.999514in}{3.998095in}}%
\pgfpathlineto{\pgfqpoint{2.001838in}{4.005723in}}%
\pgfpathlineto{\pgfqpoint{2.004162in}{4.021355in}}%
\pgfpathlineto{\pgfqpoint{2.006486in}{3.956422in}}%
\pgfpathlineto{\pgfqpoint{2.008810in}{3.991037in}}%
\pgfpathlineto{\pgfqpoint{2.011134in}{4.007909in}}%
\pgfpathlineto{\pgfqpoint{2.013458in}{4.013374in}}%
\pgfpathlineto{\pgfqpoint{2.015782in}{3.980997in}}%
\pgfpathlineto{\pgfqpoint{2.018106in}{3.981291in}}%
\pgfpathlineto{\pgfqpoint{2.020430in}{3.990237in}}%
\pgfpathlineto{\pgfqpoint{2.022754in}{3.953126in}}%
\pgfpathlineto{\pgfqpoint{2.025078in}{3.982016in}}%
\pgfpathlineto{\pgfqpoint{2.027402in}{4.028209in}}%
\pgfpathlineto{\pgfqpoint{2.032050in}{3.986950in}}%
\pgfpathlineto{\pgfqpoint{2.034374in}{3.984639in}}%
\pgfpathlineto{\pgfqpoint{2.039022in}{3.937419in}}%
\pgfpathlineto{\pgfqpoint{2.043670in}{3.991473in}}%
\pgfpathlineto{\pgfqpoint{2.045994in}{4.009694in}}%
\pgfpathlineto{\pgfqpoint{2.050642in}{4.000565in}}%
\pgfpathlineto{\pgfqpoint{2.055290in}{4.013960in}}%
\pgfpathlineto{\pgfqpoint{2.059938in}{4.004674in}}%
\pgfpathlineto{\pgfqpoint{2.062262in}{4.025553in}}%
\pgfpathlineto{\pgfqpoint{2.069234in}{3.971906in}}%
\pgfpathlineto{\pgfqpoint{2.071557in}{4.024657in}}%
\pgfpathlineto{\pgfqpoint{2.073881in}{3.983525in}}%
\pgfpathlineto{\pgfqpoint{2.078529in}{3.997411in}}%
\pgfpathlineto{\pgfqpoint{2.080853in}{3.996432in}}%
\pgfpathlineto{\pgfqpoint{2.083177in}{3.984780in}}%
\pgfpathlineto{\pgfqpoint{2.085501in}{4.001818in}}%
\pgfpathlineto{\pgfqpoint{2.087825in}{3.976881in}}%
\pgfpathlineto{\pgfqpoint{2.090149in}{4.004137in}}%
\pgfpathlineto{\pgfqpoint{2.092473in}{4.008340in}}%
\pgfpathlineto{\pgfqpoint{2.094797in}{3.971893in}}%
\pgfpathlineto{\pgfqpoint{2.097121in}{4.000088in}}%
\pgfpathlineto{\pgfqpoint{2.099445in}{3.986413in}}%
\pgfpathlineto{\pgfqpoint{2.101769in}{3.988804in}}%
\pgfpathlineto{\pgfqpoint{2.104093in}{4.379798in}}%
\pgfpathlineto{\pgfqpoint{2.106417in}{4.383798in}}%
\pgfpathlineto{\pgfqpoint{2.108741in}{4.393561in}}%
\pgfpathlineto{\pgfqpoint{2.111065in}{4.397350in}}%
\pgfpathlineto{\pgfqpoint{2.113389in}{4.388951in}}%
\pgfpathlineto{\pgfqpoint{2.115713in}{4.371564in}}%
\pgfpathlineto{\pgfqpoint{2.118037in}{4.399228in}}%
\pgfpathlineto{\pgfqpoint{2.120361in}{4.407620in}}%
\pgfpathlineto{\pgfqpoint{2.122685in}{4.371309in}}%
\pgfpathlineto{\pgfqpoint{2.125009in}{4.376253in}}%
\pgfpathlineto{\pgfqpoint{2.127333in}{4.350131in}}%
\pgfpathlineto{\pgfqpoint{2.129657in}{4.351176in}}%
\pgfpathlineto{\pgfqpoint{2.131981in}{4.406591in}}%
\pgfpathlineto{\pgfqpoint{2.134305in}{4.372226in}}%
\pgfpathlineto{\pgfqpoint{2.136629in}{4.365609in}}%
\pgfpathlineto{\pgfqpoint{2.143600in}{4.384561in}}%
\pgfpathlineto{\pgfqpoint{2.145924in}{4.372164in}}%
\pgfpathlineto{\pgfqpoint{2.148248in}{4.376186in}}%
\pgfpathlineto{\pgfqpoint{2.150572in}{4.351646in}}%
\pgfpathlineto{\pgfqpoint{2.155220in}{4.389575in}}%
\pgfpathlineto{\pgfqpoint{2.157544in}{4.374941in}}%
\pgfpathlineto{\pgfqpoint{2.159868in}{4.409001in}}%
\pgfpathlineto{\pgfqpoint{2.164516in}{4.385234in}}%
\pgfpathlineto{\pgfqpoint{2.166840in}{4.324737in}}%
\pgfpathlineto{\pgfqpoint{2.169164in}{4.366512in}}%
\pgfpathlineto{\pgfqpoint{2.171488in}{4.383985in}}%
\pgfpathlineto{\pgfqpoint{2.173812in}{4.377927in}}%
\pgfpathlineto{\pgfqpoint{2.176136in}{4.393063in}}%
\pgfpathlineto{\pgfqpoint{2.178460in}{4.346927in}}%
\pgfpathlineto{\pgfqpoint{2.180784in}{4.370451in}}%
\pgfpathlineto{\pgfqpoint{2.183108in}{4.343719in}}%
\pgfpathlineto{\pgfqpoint{2.187756in}{4.390773in}}%
\pgfpathlineto{\pgfqpoint{2.190080in}{4.352987in}}%
\pgfpathlineto{\pgfqpoint{2.192404in}{4.377662in}}%
\pgfpathlineto{\pgfqpoint{2.194728in}{4.383402in}}%
\pgfpathlineto{\pgfqpoint{2.197052in}{4.358085in}}%
\pgfpathlineto{\pgfqpoint{2.199376in}{4.364774in}}%
\pgfpathlineto{\pgfqpoint{2.201700in}{4.348219in}}%
\pgfpathlineto{\pgfqpoint{2.204024in}{4.369181in}}%
\pgfpathlineto{\pgfqpoint{2.206348in}{4.366736in}}%
\pgfpathlineto{\pgfqpoint{2.208672in}{4.385864in}}%
\pgfpathlineto{\pgfqpoint{2.210995in}{4.376487in}}%
\pgfpathlineto{\pgfqpoint{2.215643in}{4.345561in}}%
\pgfpathlineto{\pgfqpoint{2.217967in}{4.395521in}}%
\pgfpathlineto{\pgfqpoint{2.220291in}{4.341114in}}%
\pgfpathlineto{\pgfqpoint{2.222615in}{4.355218in}}%
\pgfpathlineto{\pgfqpoint{2.224939in}{4.400134in}}%
\pgfpathlineto{\pgfqpoint{2.227263in}{4.379190in}}%
\pgfpathlineto{\pgfqpoint{2.229587in}{4.377437in}}%
\pgfpathlineto{\pgfqpoint{2.231911in}{4.357611in}}%
\pgfpathlineto{\pgfqpoint{2.234235in}{4.355019in}}%
\pgfpathlineto{\pgfqpoint{2.236559in}{4.384571in}}%
\pgfpathlineto{\pgfqpoint{2.238883in}{4.354098in}}%
\pgfpathlineto{\pgfqpoint{2.241207in}{4.385973in}}%
\pgfpathlineto{\pgfqpoint{2.243531in}{4.374964in}}%
\pgfpathlineto{\pgfqpoint{2.245855in}{4.371680in}}%
\pgfpathlineto{\pgfqpoint{2.248179in}{4.396444in}}%
\pgfpathlineto{\pgfqpoint{2.250503in}{4.401202in}}%
\pgfpathlineto{\pgfqpoint{2.252827in}{4.375471in}}%
\pgfpathlineto{\pgfqpoint{2.255151in}{4.376590in}}%
\pgfpathlineto{\pgfqpoint{2.257475in}{4.343533in}}%
\pgfpathlineto{\pgfqpoint{2.264447in}{4.387259in}}%
\pgfpathlineto{\pgfqpoint{2.266771in}{4.352246in}}%
\pgfpathlineto{\pgfqpoint{2.269095in}{4.355307in}}%
\pgfpathlineto{\pgfqpoint{2.271419in}{4.378031in}}%
\pgfpathlineto{\pgfqpoint{2.273743in}{4.347160in}}%
\pgfpathlineto{\pgfqpoint{2.278391in}{4.368416in}}%
\pgfpathlineto{\pgfqpoint{2.280715in}{4.354751in}}%
\pgfpathlineto{\pgfqpoint{2.283038in}{4.405737in}}%
\pgfpathlineto{\pgfqpoint{2.285362in}{4.380116in}}%
\pgfpathlineto{\pgfqpoint{2.287686in}{4.371512in}}%
\pgfpathlineto{\pgfqpoint{2.290010in}{4.388649in}}%
\pgfpathlineto{\pgfqpoint{2.292334in}{4.342164in}}%
\pgfpathlineto{\pgfqpoint{2.294658in}{4.410914in}}%
\pgfpathlineto{\pgfqpoint{2.299306in}{4.389929in}}%
\pgfpathlineto{\pgfqpoint{2.301630in}{4.373781in}}%
\pgfpathlineto{\pgfqpoint{2.303954in}{4.380980in}}%
\pgfpathlineto{\pgfqpoint{2.306278in}{4.357980in}}%
\pgfpathlineto{\pgfqpoint{2.308602in}{4.392458in}}%
\pgfpathlineto{\pgfqpoint{2.310926in}{4.407749in}}%
\pgfpathlineto{\pgfqpoint{2.313250in}{4.369820in}}%
\pgfpathlineto{\pgfqpoint{2.315574in}{4.390270in}}%
\pgfpathlineto{\pgfqpoint{2.317898in}{4.356920in}}%
\pgfpathlineto{\pgfqpoint{2.320222in}{4.408860in}}%
\pgfpathlineto{\pgfqpoint{2.322546in}{4.362151in}}%
\pgfpathlineto{\pgfqpoint{2.324870in}{4.394630in}}%
\pgfpathlineto{\pgfqpoint{2.327194in}{4.366413in}}%
\pgfpathlineto{\pgfqpoint{2.329518in}{4.373378in}}%
\pgfpathlineto{\pgfqpoint{2.331842in}{4.373827in}}%
\pgfpathlineto{\pgfqpoint{2.334166in}{4.378531in}}%
\pgfpathlineto{\pgfqpoint{2.336490in}{4.400683in}}%
\pgfpathlineto{\pgfqpoint{2.338814in}{4.361123in}}%
\pgfpathlineto{\pgfqpoint{2.341138in}{4.366369in}}%
\pgfpathlineto{\pgfqpoint{2.343462in}{4.368014in}}%
\pgfpathlineto{\pgfqpoint{2.345786in}{4.358743in}}%
\pgfpathlineto{\pgfqpoint{2.350434in}{4.358515in}}%
\pgfpathlineto{\pgfqpoint{2.355081in}{4.373734in}}%
\pgfpathlineto{\pgfqpoint{2.357405in}{4.375084in}}%
\pgfpathlineto{\pgfqpoint{2.359729in}{4.367468in}}%
\pgfpathlineto{\pgfqpoint{2.362053in}{4.354979in}}%
\pgfpathlineto{\pgfqpoint{2.366701in}{4.382626in}}%
\pgfpathlineto{\pgfqpoint{2.369025in}{4.347227in}}%
\pgfpathlineto{\pgfqpoint{2.371349in}{4.348137in}}%
\pgfpathlineto{\pgfqpoint{2.373673in}{4.374609in}}%
\pgfpathlineto{\pgfqpoint{2.375997in}{4.375337in}}%
\pgfpathlineto{\pgfqpoint{2.378321in}{4.400499in}}%
\pgfpathlineto{\pgfqpoint{2.380645in}{4.371252in}}%
\pgfpathlineto{\pgfqpoint{2.382969in}{4.361068in}}%
\pgfpathlineto{\pgfqpoint{2.387617in}{4.372162in}}%
\pgfpathlineto{\pgfqpoint{2.389941in}{4.368567in}}%
\pgfpathlineto{\pgfqpoint{2.392265in}{4.378586in}}%
\pgfpathlineto{\pgfqpoint{2.394589in}{3.981587in}}%
\pgfpathlineto{\pgfqpoint{2.399237in}{4.023769in}}%
\pgfpathlineto{\pgfqpoint{2.401561in}{3.959460in}}%
\pgfpathlineto{\pgfqpoint{2.403885in}{4.005948in}}%
\pgfpathlineto{\pgfqpoint{2.406209in}{3.984166in}}%
\pgfpathlineto{\pgfqpoint{2.408533in}{3.986104in}}%
\pgfpathlineto{\pgfqpoint{2.410857in}{3.996698in}}%
\pgfpathlineto{\pgfqpoint{2.413181in}{3.988729in}}%
\pgfpathlineto{\pgfqpoint{2.415505in}{3.991716in}}%
\pgfpathlineto{\pgfqpoint{2.417829in}{3.985783in}}%
\pgfpathlineto{\pgfqpoint{2.420153in}{4.015031in}}%
\pgfpathlineto{\pgfqpoint{2.422476in}{3.984722in}}%
\pgfpathlineto{\pgfqpoint{2.424800in}{4.019374in}}%
\pgfpathlineto{\pgfqpoint{2.427124in}{3.996341in}}%
\pgfpathlineto{\pgfqpoint{2.429448in}{3.988024in}}%
\pgfpathlineto{\pgfqpoint{2.431772in}{4.014244in}}%
\pgfpathlineto{\pgfqpoint{2.434096in}{4.011140in}}%
\pgfpathlineto{\pgfqpoint{2.436420in}{3.970654in}}%
\pgfpathlineto{\pgfqpoint{2.438744in}{4.003940in}}%
\pgfpathlineto{\pgfqpoint{2.441068in}{3.974545in}}%
\pgfpathlineto{\pgfqpoint{2.443392in}{4.007682in}}%
\pgfpathlineto{\pgfqpoint{2.445716in}{3.984152in}}%
\pgfpathlineto{\pgfqpoint{2.448040in}{3.991182in}}%
\pgfpathlineto{\pgfqpoint{2.452688in}{3.949334in}}%
\pgfpathlineto{\pgfqpoint{2.455012in}{3.992489in}}%
\pgfpathlineto{\pgfqpoint{2.459660in}{4.021084in}}%
\pgfpathlineto{\pgfqpoint{2.461984in}{4.012346in}}%
\pgfpathlineto{\pgfqpoint{2.464308in}{4.011143in}}%
\pgfpathlineto{\pgfqpoint{2.466632in}{4.015261in}}%
\pgfpathlineto{\pgfqpoint{2.468956in}{3.997174in}}%
\pgfpathlineto{\pgfqpoint{2.471280in}{4.005803in}}%
\pgfpathlineto{\pgfqpoint{2.473604in}{3.969881in}}%
\pgfpathlineto{\pgfqpoint{2.475928in}{3.987930in}}%
\pgfpathlineto{\pgfqpoint{2.478252in}{3.961930in}}%
\pgfpathlineto{\pgfqpoint{2.480576in}{3.966905in}}%
\pgfpathlineto{\pgfqpoint{2.482900in}{3.976843in}}%
\pgfpathlineto{\pgfqpoint{2.485224in}{3.994769in}}%
\pgfpathlineto{\pgfqpoint{2.487548in}{3.994016in}}%
\pgfpathlineto{\pgfqpoint{2.489872in}{3.961684in}}%
\pgfpathlineto{\pgfqpoint{2.492195in}{3.972667in}}%
\pgfpathlineto{\pgfqpoint{2.494519in}{3.976253in}}%
\pgfpathlineto{\pgfqpoint{2.496843in}{4.008750in}}%
\pgfpathlineto{\pgfqpoint{2.499167in}{4.003292in}}%
\pgfpathlineto{\pgfqpoint{2.501491in}{3.979449in}}%
\pgfpathlineto{\pgfqpoint{2.503815in}{3.931404in}}%
\pgfpathlineto{\pgfqpoint{2.506139in}{4.003673in}}%
\pgfpathlineto{\pgfqpoint{2.508463in}{3.989111in}}%
\pgfpathlineto{\pgfqpoint{2.510787in}{4.011247in}}%
\pgfpathlineto{\pgfqpoint{2.513111in}{4.005202in}}%
\pgfpathlineto{\pgfqpoint{2.515435in}{4.017102in}}%
\pgfpathlineto{\pgfqpoint{2.517759in}{3.985521in}}%
\pgfpathlineto{\pgfqpoint{2.520083in}{4.004730in}}%
\pgfpathlineto{\pgfqpoint{2.522407in}{3.976771in}}%
\pgfpathlineto{\pgfqpoint{2.524731in}{3.992902in}}%
\pgfpathlineto{\pgfqpoint{2.527055in}{3.958564in}}%
\pgfpathlineto{\pgfqpoint{2.529379in}{3.998302in}}%
\pgfpathlineto{\pgfqpoint{2.531703in}{3.992640in}}%
\pgfpathlineto{\pgfqpoint{2.534027in}{3.978475in}}%
\pgfpathlineto{\pgfqpoint{2.536351in}{4.046981in}}%
\pgfpathlineto{\pgfqpoint{2.538675in}{3.961225in}}%
\pgfpathlineto{\pgfqpoint{2.540999in}{3.968097in}}%
\pgfpathlineto{\pgfqpoint{2.543323in}{4.017410in}}%
\pgfpathlineto{\pgfqpoint{2.545647in}{4.000580in}}%
\pgfpathlineto{\pgfqpoint{2.547971in}{4.000227in}}%
\pgfpathlineto{\pgfqpoint{2.550295in}{4.023939in}}%
\pgfpathlineto{\pgfqpoint{2.552619in}{3.993373in}}%
\pgfpathlineto{\pgfqpoint{2.554943in}{3.997169in}}%
\pgfpathlineto{\pgfqpoint{2.557267in}{4.015486in}}%
\pgfpathlineto{\pgfqpoint{2.559591in}{4.022083in}}%
\pgfpathlineto{\pgfqpoint{2.561915in}{3.988820in}}%
\pgfpathlineto{\pgfqpoint{2.564238in}{3.999708in}}%
\pgfpathlineto{\pgfqpoint{2.566562in}{4.020480in}}%
\pgfpathlineto{\pgfqpoint{2.568886in}{3.998817in}}%
\pgfpathlineto{\pgfqpoint{2.571210in}{4.023609in}}%
\pgfpathlineto{\pgfqpoint{2.573534in}{3.982158in}}%
\pgfpathlineto{\pgfqpoint{2.575858in}{4.018026in}}%
\pgfpathlineto{\pgfqpoint{2.578182in}{3.936850in}}%
\pgfpathlineto{\pgfqpoint{2.580506in}{4.017764in}}%
\pgfpathlineto{\pgfqpoint{2.582830in}{4.015900in}}%
\pgfpathlineto{\pgfqpoint{2.585154in}{3.965969in}}%
\pgfpathlineto{\pgfqpoint{2.592126in}{4.027236in}}%
\pgfpathlineto{\pgfqpoint{2.594450in}{3.979721in}}%
\pgfpathlineto{\pgfqpoint{2.596774in}{3.981464in}}%
\pgfpathlineto{\pgfqpoint{2.599098in}{4.001116in}}%
\pgfpathlineto{\pgfqpoint{2.601422in}{3.945640in}}%
\pgfpathlineto{\pgfqpoint{2.603746in}{4.008152in}}%
\pgfpathlineto{\pgfqpoint{2.606070in}{3.964220in}}%
\pgfpathlineto{\pgfqpoint{2.608394in}{4.011364in}}%
\pgfpathlineto{\pgfqpoint{2.610718in}{3.972522in}}%
\pgfpathlineto{\pgfqpoint{2.613042in}{3.962505in}}%
\pgfpathlineto{\pgfqpoint{2.615366in}{3.977390in}}%
\pgfpathlineto{\pgfqpoint{2.617690in}{3.972000in}}%
\pgfpathlineto{\pgfqpoint{2.620014in}{3.985253in}}%
\pgfpathlineto{\pgfqpoint{2.622338in}{4.034680in}}%
\pgfpathlineto{\pgfqpoint{2.629310in}{3.958782in}}%
\pgfpathlineto{\pgfqpoint{2.633957in}{3.986504in}}%
\pgfpathlineto{\pgfqpoint{2.640929in}{4.011444in}}%
\pgfpathlineto{\pgfqpoint{2.643253in}{3.969992in}}%
\pgfpathlineto{\pgfqpoint{2.645577in}{3.981246in}}%
\pgfpathlineto{\pgfqpoint{2.647901in}{4.006538in}}%
\pgfpathlineto{\pgfqpoint{2.650225in}{3.974009in}}%
\pgfpathlineto{\pgfqpoint{2.652549in}{3.974756in}}%
\pgfpathlineto{\pgfqpoint{2.654873in}{3.985253in}}%
\pgfpathlineto{\pgfqpoint{2.657197in}{4.025338in}}%
\pgfpathlineto{\pgfqpoint{2.659521in}{3.987189in}}%
\pgfpathlineto{\pgfqpoint{2.661845in}{4.003979in}}%
\pgfpathlineto{\pgfqpoint{2.664169in}{4.001828in}}%
\pgfpathlineto{\pgfqpoint{2.666493in}{3.976290in}}%
\pgfpathlineto{\pgfqpoint{2.668817in}{3.979991in}}%
\pgfpathlineto{\pgfqpoint{2.671141in}{3.964801in}}%
\pgfpathlineto{\pgfqpoint{2.673465in}{3.996492in}}%
\pgfpathlineto{\pgfqpoint{2.675789in}{3.959106in}}%
\pgfpathlineto{\pgfqpoint{2.678113in}{4.007887in}}%
\pgfpathlineto{\pgfqpoint{2.680437in}{3.987398in}}%
\pgfpathlineto{\pgfqpoint{2.682761in}{3.981559in}}%
\pgfpathlineto{\pgfqpoint{2.685085in}{4.368876in}}%
\pgfpathlineto{\pgfqpoint{2.687409in}{4.355394in}}%
\pgfpathlineto{\pgfqpoint{2.689733in}{4.379678in}}%
\pgfpathlineto{\pgfqpoint{2.694381in}{4.338177in}}%
\pgfpathlineto{\pgfqpoint{2.696705in}{4.350839in}}%
\pgfpathlineto{\pgfqpoint{2.701353in}{4.390988in}}%
\pgfpathlineto{\pgfqpoint{2.703676in}{4.372472in}}%
\pgfpathlineto{\pgfqpoint{2.706000in}{4.362923in}}%
\pgfpathlineto{\pgfqpoint{2.708324in}{4.348340in}}%
\pgfpathlineto{\pgfqpoint{2.715296in}{4.389654in}}%
\pgfpathlineto{\pgfqpoint{2.717620in}{4.420425in}}%
\pgfpathlineto{\pgfqpoint{2.719944in}{4.362592in}}%
\pgfpathlineto{\pgfqpoint{2.724592in}{4.371022in}}%
\pgfpathlineto{\pgfqpoint{2.729240in}{4.358636in}}%
\pgfpathlineto{\pgfqpoint{2.731564in}{4.364530in}}%
\pgfpathlineto{\pgfqpoint{2.733888in}{4.401668in}}%
\pgfpathlineto{\pgfqpoint{2.738536in}{4.371095in}}%
\pgfpathlineto{\pgfqpoint{2.740860in}{4.384468in}}%
\pgfpathlineto{\pgfqpoint{2.743184in}{4.350470in}}%
\pgfpathlineto{\pgfqpoint{2.745508in}{4.365305in}}%
\pgfpathlineto{\pgfqpoint{2.747832in}{4.350019in}}%
\pgfpathlineto{\pgfqpoint{2.752480in}{4.369425in}}%
\pgfpathlineto{\pgfqpoint{2.754804in}{4.343833in}}%
\pgfpathlineto{\pgfqpoint{2.759452in}{4.410808in}}%
\pgfpathlineto{\pgfqpoint{2.761776in}{4.352702in}}%
\pgfpathlineto{\pgfqpoint{2.764100in}{4.338806in}}%
\pgfpathlineto{\pgfqpoint{2.766424in}{4.388181in}}%
\pgfpathlineto{\pgfqpoint{2.768748in}{4.321326in}}%
\pgfpathlineto{\pgfqpoint{2.775719in}{4.400023in}}%
\pgfpathlineto{\pgfqpoint{2.778043in}{4.338911in}}%
\pgfpathlineto{\pgfqpoint{2.785015in}{4.399606in}}%
\pgfpathlineto{\pgfqpoint{2.787339in}{4.375339in}}%
\pgfpathlineto{\pgfqpoint{2.789663in}{4.367553in}}%
\pgfpathlineto{\pgfqpoint{2.791987in}{4.371202in}}%
\pgfpathlineto{\pgfqpoint{2.794311in}{4.369018in}}%
\pgfpathlineto{\pgfqpoint{2.796635in}{4.363953in}}%
\pgfpathlineto{\pgfqpoint{2.801283in}{4.373059in}}%
\pgfpathlineto{\pgfqpoint{2.803607in}{4.382511in}}%
\pgfpathlineto{\pgfqpoint{2.805931in}{4.386116in}}%
\pgfpathlineto{\pgfqpoint{2.808255in}{4.370019in}}%
\pgfpathlineto{\pgfqpoint{2.810579in}{4.368994in}}%
\pgfpathlineto{\pgfqpoint{2.812903in}{4.397020in}}%
\pgfpathlineto{\pgfqpoint{2.817551in}{4.363586in}}%
\pgfpathlineto{\pgfqpoint{2.819875in}{4.362787in}}%
\pgfpathlineto{\pgfqpoint{2.822199in}{4.377779in}}%
\pgfpathlineto{\pgfqpoint{2.824523in}{4.347116in}}%
\pgfpathlineto{\pgfqpoint{2.826847in}{4.334877in}}%
\pgfpathlineto{\pgfqpoint{2.829171in}{4.386672in}}%
\pgfpathlineto{\pgfqpoint{2.831495in}{4.360603in}}%
\pgfpathlineto{\pgfqpoint{2.833819in}{4.365481in}}%
\pgfpathlineto{\pgfqpoint{2.836143in}{4.364263in}}%
\pgfpathlineto{\pgfqpoint{2.838467in}{4.409816in}}%
\pgfpathlineto{\pgfqpoint{2.840791in}{4.344529in}}%
\pgfpathlineto{\pgfqpoint{2.843115in}{4.404919in}}%
\pgfpathlineto{\pgfqpoint{2.845438in}{4.375245in}}%
\pgfpathlineto{\pgfqpoint{2.847762in}{4.380930in}}%
\pgfpathlineto{\pgfqpoint{2.850086in}{4.333430in}}%
\pgfpathlineto{\pgfqpoint{2.854734in}{4.371538in}}%
\pgfpathlineto{\pgfqpoint{2.857058in}{4.343821in}}%
\pgfpathlineto{\pgfqpoint{2.859382in}{4.360438in}}%
\pgfpathlineto{\pgfqpoint{2.861706in}{4.385443in}}%
\pgfpathlineto{\pgfqpoint{2.864030in}{4.356921in}}%
\pgfpathlineto{\pgfqpoint{2.866354in}{4.386787in}}%
\pgfpathlineto{\pgfqpoint{2.868678in}{4.388042in}}%
\pgfpathlineto{\pgfqpoint{2.871002in}{4.408782in}}%
\pgfpathlineto{\pgfqpoint{2.873326in}{4.392235in}}%
\pgfpathlineto{\pgfqpoint{2.875650in}{4.386155in}}%
\pgfpathlineto{\pgfqpoint{2.877974in}{4.384263in}}%
\pgfpathlineto{\pgfqpoint{2.880298in}{4.369721in}}%
\pgfpathlineto{\pgfqpoint{2.882622in}{4.329250in}}%
\pgfpathlineto{\pgfqpoint{2.884946in}{4.371742in}}%
\pgfpathlineto{\pgfqpoint{2.887270in}{4.383582in}}%
\pgfpathlineto{\pgfqpoint{2.889594in}{4.355588in}}%
\pgfpathlineto{\pgfqpoint{2.891918in}{4.346835in}}%
\pgfpathlineto{\pgfqpoint{2.894242in}{4.398721in}}%
\pgfpathlineto{\pgfqpoint{2.898890in}{4.372464in}}%
\pgfpathlineto{\pgfqpoint{2.901214in}{4.378098in}}%
\pgfpathlineto{\pgfqpoint{2.903538in}{4.357333in}}%
\pgfpathlineto{\pgfqpoint{2.905862in}{4.393202in}}%
\pgfpathlineto{\pgfqpoint{2.908186in}{4.395196in}}%
\pgfpathlineto{\pgfqpoint{2.910510in}{4.384747in}}%
\pgfpathlineto{\pgfqpoint{2.912834in}{4.390535in}}%
\pgfpathlineto{\pgfqpoint{2.915157in}{4.370019in}}%
\pgfpathlineto{\pgfqpoint{2.919805in}{4.379816in}}%
\pgfpathlineto{\pgfqpoint{2.922129in}{4.368106in}}%
\pgfpathlineto{\pgfqpoint{2.924453in}{4.369298in}}%
\pgfpathlineto{\pgfqpoint{2.926777in}{4.363401in}}%
\pgfpathlineto{\pgfqpoint{2.929101in}{4.373362in}}%
\pgfpathlineto{\pgfqpoint{2.931425in}{4.394106in}}%
\pgfpathlineto{\pgfqpoint{2.933749in}{4.337486in}}%
\pgfpathlineto{\pgfqpoint{2.936073in}{4.372267in}}%
\pgfpathlineto{\pgfqpoint{2.938397in}{4.372141in}}%
\pgfpathlineto{\pgfqpoint{2.940721in}{4.378971in}}%
\pgfpathlineto{\pgfqpoint{2.943045in}{4.372686in}}%
\pgfpathlineto{\pgfqpoint{2.945369in}{4.372474in}}%
\pgfpathlineto{\pgfqpoint{2.947693in}{4.334985in}}%
\pgfpathlineto{\pgfqpoint{2.952341in}{4.380916in}}%
\pgfpathlineto{\pgfqpoint{2.954665in}{4.384798in}}%
\pgfpathlineto{\pgfqpoint{2.956989in}{4.331381in}}%
\pgfpathlineto{\pgfqpoint{2.959313in}{4.355717in}}%
\pgfpathlineto{\pgfqpoint{2.961637in}{4.352424in}}%
\pgfpathlineto{\pgfqpoint{2.963961in}{4.356417in}}%
\pgfpathlineto{\pgfqpoint{2.966285in}{4.356466in}}%
\pgfpathlineto{\pgfqpoint{2.968609in}{4.341098in}}%
\pgfpathlineto{\pgfqpoint{2.970933in}{4.411776in}}%
\pgfpathlineto{\pgfqpoint{2.973257in}{4.398977in}}%
\pgfpathlineto{\pgfqpoint{2.975581in}{4.007681in}}%
\pgfpathlineto{\pgfqpoint{2.977905in}{3.989838in}}%
\pgfpathlineto{\pgfqpoint{2.980229in}{4.010566in}}%
\pgfpathlineto{\pgfqpoint{2.982553in}{3.992690in}}%
\pgfpathlineto{\pgfqpoint{2.984876in}{3.959219in}}%
\pgfpathlineto{\pgfqpoint{2.987200in}{3.999492in}}%
\pgfpathlineto{\pgfqpoint{2.989524in}{3.990618in}}%
\pgfpathlineto{\pgfqpoint{2.991848in}{3.952960in}}%
\pgfpathlineto{\pgfqpoint{2.994172in}{3.998116in}}%
\pgfpathlineto{\pgfqpoint{2.996496in}{3.962307in}}%
\pgfpathlineto{\pgfqpoint{2.998820in}{3.993673in}}%
\pgfpathlineto{\pgfqpoint{3.001144in}{3.987070in}}%
\pgfpathlineto{\pgfqpoint{3.003468in}{3.998915in}}%
\pgfpathlineto{\pgfqpoint{3.005792in}{3.995249in}}%
\pgfpathlineto{\pgfqpoint{3.008116in}{4.025797in}}%
\pgfpathlineto{\pgfqpoint{3.010440in}{3.986077in}}%
\pgfpathlineto{\pgfqpoint{3.012764in}{3.977224in}}%
\pgfpathlineto{\pgfqpoint{3.015088in}{3.974258in}}%
\pgfpathlineto{\pgfqpoint{3.017412in}{3.984471in}}%
\pgfpathlineto{\pgfqpoint{3.019736in}{3.975996in}}%
\pgfpathlineto{\pgfqpoint{3.022060in}{3.992341in}}%
\pgfpathlineto{\pgfqpoint{3.024384in}{3.987478in}}%
\pgfpathlineto{\pgfqpoint{3.026708in}{3.992847in}}%
\pgfpathlineto{\pgfqpoint{3.029032in}{4.003784in}}%
\pgfpathlineto{\pgfqpoint{3.031356in}{3.992991in}}%
\pgfpathlineto{\pgfqpoint{3.033680in}{3.960366in}}%
\pgfpathlineto{\pgfqpoint{3.036004in}{3.996268in}}%
\pgfpathlineto{\pgfqpoint{3.038328in}{4.001598in}}%
\pgfpathlineto{\pgfqpoint{3.040652in}{3.990252in}}%
\pgfpathlineto{\pgfqpoint{3.042976in}{3.969119in}}%
\pgfpathlineto{\pgfqpoint{3.045300in}{3.981297in}}%
\pgfpathlineto{\pgfqpoint{3.047624in}{3.969996in}}%
\pgfpathlineto{\pgfqpoint{3.049948in}{3.981546in}}%
\pgfpathlineto{\pgfqpoint{3.054596in}{4.034081in}}%
\pgfpathlineto{\pgfqpoint{3.056919in}{3.976895in}}%
\pgfpathlineto{\pgfqpoint{3.059243in}{4.021629in}}%
\pgfpathlineto{\pgfqpoint{3.061567in}{4.013484in}}%
\pgfpathlineto{\pgfqpoint{3.063891in}{3.980156in}}%
\pgfpathlineto{\pgfqpoint{3.066215in}{4.009766in}}%
\pgfpathlineto{\pgfqpoint{3.068539in}{3.972099in}}%
\pgfpathlineto{\pgfqpoint{3.070863in}{3.973532in}}%
\pgfpathlineto{\pgfqpoint{3.073187in}{3.984498in}}%
\pgfpathlineto{\pgfqpoint{3.077835in}{4.028473in}}%
\pgfpathlineto{\pgfqpoint{3.080159in}{3.980736in}}%
\pgfpathlineto{\pgfqpoint{3.082483in}{3.972754in}}%
\pgfpathlineto{\pgfqpoint{3.084807in}{3.971014in}}%
\pgfpathlineto{\pgfqpoint{3.087131in}{3.977662in}}%
\pgfpathlineto{\pgfqpoint{3.091779in}{4.010303in}}%
\pgfpathlineto{\pgfqpoint{3.094103in}{3.969813in}}%
\pgfpathlineto{\pgfqpoint{3.096427in}{3.996991in}}%
\pgfpathlineto{\pgfqpoint{3.098751in}{3.961785in}}%
\pgfpathlineto{\pgfqpoint{3.101075in}{4.021045in}}%
\pgfpathlineto{\pgfqpoint{3.105723in}{3.984919in}}%
\pgfpathlineto{\pgfqpoint{3.108047in}{3.964377in}}%
\pgfpathlineto{\pgfqpoint{3.110371in}{3.957883in}}%
\pgfpathlineto{\pgfqpoint{3.112695in}{4.002017in}}%
\pgfpathlineto{\pgfqpoint{3.115019in}{3.971773in}}%
\pgfpathlineto{\pgfqpoint{3.117343in}{4.005970in}}%
\pgfpathlineto{\pgfqpoint{3.119667in}{3.996233in}}%
\pgfpathlineto{\pgfqpoint{3.121991in}{3.976931in}}%
\pgfpathlineto{\pgfqpoint{3.124315in}{3.991040in}}%
\pgfpathlineto{\pgfqpoint{3.126638in}{3.993664in}}%
\pgfpathlineto{\pgfqpoint{3.128962in}{3.985116in}}%
\pgfpathlineto{\pgfqpoint{3.131286in}{4.007533in}}%
\pgfpathlineto{\pgfqpoint{3.133610in}{3.999960in}}%
\pgfpathlineto{\pgfqpoint{3.135934in}{4.011684in}}%
\pgfpathlineto{\pgfqpoint{3.138258in}{3.969222in}}%
\pgfpathlineto{\pgfqpoint{3.140582in}{4.010529in}}%
\pgfpathlineto{\pgfqpoint{3.142906in}{3.972031in}}%
\pgfpathlineto{\pgfqpoint{3.145230in}{4.004382in}}%
\pgfpathlineto{\pgfqpoint{3.147554in}{3.980219in}}%
\pgfpathlineto{\pgfqpoint{3.149878in}{3.987669in}}%
\pgfpathlineto{\pgfqpoint{3.152202in}{3.974972in}}%
\pgfpathlineto{\pgfqpoint{3.154526in}{3.971163in}}%
\pgfpathlineto{\pgfqpoint{3.156850in}{4.035671in}}%
\pgfpathlineto{\pgfqpoint{3.159174in}{4.020117in}}%
\pgfpathlineto{\pgfqpoint{3.161498in}{3.960921in}}%
\pgfpathlineto{\pgfqpoint{3.163822in}{4.006780in}}%
\pgfpathlineto{\pgfqpoint{3.166146in}{3.986836in}}%
\pgfpathlineto{\pgfqpoint{3.168470in}{4.001721in}}%
\pgfpathlineto{\pgfqpoint{3.170794in}{3.987088in}}%
\pgfpathlineto{\pgfqpoint{3.173118in}{3.999549in}}%
\pgfpathlineto{\pgfqpoint{3.175442in}{3.984450in}}%
\pgfpathlineto{\pgfqpoint{3.177766in}{4.011126in}}%
\pgfpathlineto{\pgfqpoint{3.180090in}{4.018521in}}%
\pgfpathlineto{\pgfqpoint{3.182414in}{4.015698in}}%
\pgfpathlineto{\pgfqpoint{3.187062in}{3.965702in}}%
\pgfpathlineto{\pgfqpoint{3.189386in}{3.988067in}}%
\pgfpathlineto{\pgfqpoint{3.191710in}{3.988498in}}%
\pgfpathlineto{\pgfqpoint{3.194034in}{3.976800in}}%
\pgfpathlineto{\pgfqpoint{3.198681in}{3.972400in}}%
\pgfpathlineto{\pgfqpoint{3.201005in}{4.010337in}}%
\pgfpathlineto{\pgfqpoint{3.203329in}{3.969105in}}%
\pgfpathlineto{\pgfqpoint{3.205653in}{4.001731in}}%
\pgfpathlineto{\pgfqpoint{3.207977in}{3.983088in}}%
\pgfpathlineto{\pgfqpoint{3.212625in}{3.986912in}}%
\pgfpathlineto{\pgfqpoint{3.214949in}{4.002299in}}%
\pgfpathlineto{\pgfqpoint{3.217273in}{4.002705in}}%
\pgfpathlineto{\pgfqpoint{3.219597in}{4.013534in}}%
\pgfpathlineto{\pgfqpoint{3.221921in}{3.966627in}}%
\pgfpathlineto{\pgfqpoint{3.224245in}{3.958224in}}%
\pgfpathlineto{\pgfqpoint{3.226569in}{3.994140in}}%
\pgfpathlineto{\pgfqpoint{3.228893in}{3.973057in}}%
\pgfpathlineto{\pgfqpoint{3.231217in}{4.006196in}}%
\pgfpathlineto{\pgfqpoint{3.233541in}{3.965744in}}%
\pgfpathlineto{\pgfqpoint{3.238189in}{4.014717in}}%
\pgfpathlineto{\pgfqpoint{3.240513in}{3.989865in}}%
\pgfpathlineto{\pgfqpoint{3.242837in}{3.987599in}}%
\pgfpathlineto{\pgfqpoint{3.245161in}{4.028184in}}%
\pgfpathlineto{\pgfqpoint{3.247485in}{3.981358in}}%
\pgfpathlineto{\pgfqpoint{3.249809in}{4.007292in}}%
\pgfpathlineto{\pgfqpoint{3.252133in}{3.992944in}}%
\pgfpathlineto{\pgfqpoint{3.254457in}{3.999546in}}%
\pgfpathlineto{\pgfqpoint{3.256781in}{3.984477in}}%
\pgfpathlineto{\pgfqpoint{3.259105in}{3.994067in}}%
\pgfpathlineto{\pgfqpoint{3.261429in}{3.973640in}}%
\pgfpathlineto{\pgfqpoint{3.263753in}{3.989447in}}%
\pgfpathlineto{\pgfqpoint{3.266076in}{4.371128in}}%
\pgfpathlineto{\pgfqpoint{3.268400in}{4.365790in}}%
\pgfpathlineto{\pgfqpoint{3.270724in}{4.344056in}}%
\pgfpathlineto{\pgfqpoint{3.273048in}{4.366108in}}%
\pgfpathlineto{\pgfqpoint{3.275372in}{4.345028in}}%
\pgfpathlineto{\pgfqpoint{3.277696in}{4.373095in}}%
\pgfpathlineto{\pgfqpoint{3.280020in}{4.319586in}}%
\pgfpathlineto{\pgfqpoint{3.282344in}{4.368830in}}%
\pgfpathlineto{\pgfqpoint{3.284668in}{4.377980in}}%
\pgfpathlineto{\pgfqpoint{3.286992in}{4.382346in}}%
\pgfpathlineto{\pgfqpoint{3.289316in}{4.360220in}}%
\pgfpathlineto{\pgfqpoint{3.291640in}{4.384636in}}%
\pgfpathlineto{\pgfqpoint{3.293964in}{4.348832in}}%
\pgfpathlineto{\pgfqpoint{3.296288in}{4.418435in}}%
\pgfpathlineto{\pgfqpoint{3.298612in}{4.379171in}}%
\pgfpathlineto{\pgfqpoint{3.300936in}{4.414608in}}%
\pgfpathlineto{\pgfqpoint{3.303260in}{4.379100in}}%
\pgfpathlineto{\pgfqpoint{3.305584in}{4.384153in}}%
\pgfpathlineto{\pgfqpoint{3.307908in}{4.363959in}}%
\pgfpathlineto{\pgfqpoint{3.310232in}{4.376998in}}%
\pgfpathlineto{\pgfqpoint{3.312556in}{4.350118in}}%
\pgfpathlineto{\pgfqpoint{3.314880in}{4.390448in}}%
\pgfpathlineto{\pgfqpoint{3.317204in}{4.395839in}}%
\pgfpathlineto{\pgfqpoint{3.321852in}{4.355636in}}%
\pgfpathlineto{\pgfqpoint{3.324176in}{4.411705in}}%
\pgfpathlineto{\pgfqpoint{3.326500in}{4.365912in}}%
\pgfpathlineto{\pgfqpoint{3.328824in}{4.344447in}}%
\pgfpathlineto{\pgfqpoint{3.333472in}{4.394431in}}%
\pgfpathlineto{\pgfqpoint{3.338119in}{4.347888in}}%
\pgfpathlineto{\pgfqpoint{3.340443in}{4.378932in}}%
\pgfpathlineto{\pgfqpoint{3.342767in}{4.368087in}}%
\pgfpathlineto{\pgfqpoint{3.345091in}{4.380886in}}%
\pgfpathlineto{\pgfqpoint{3.347415in}{4.364557in}}%
\pgfpathlineto{\pgfqpoint{3.349739in}{4.374540in}}%
\pgfpathlineto{\pgfqpoint{3.352063in}{4.357784in}}%
\pgfpathlineto{\pgfqpoint{3.354387in}{4.352460in}}%
\pgfpathlineto{\pgfqpoint{3.356711in}{4.417574in}}%
\pgfpathlineto{\pgfqpoint{3.359035in}{4.372449in}}%
\pgfpathlineto{\pgfqpoint{3.361359in}{4.354962in}}%
\pgfpathlineto{\pgfqpoint{3.363683in}{4.373584in}}%
\pgfpathlineto{\pgfqpoint{3.368331in}{4.366027in}}%
\pgfpathlineto{\pgfqpoint{3.370655in}{4.357358in}}%
\pgfpathlineto{\pgfqpoint{3.372979in}{4.369693in}}%
\pgfpathlineto{\pgfqpoint{3.375303in}{4.340126in}}%
\pgfpathlineto{\pgfqpoint{3.377627in}{4.370541in}}%
\pgfpathlineto{\pgfqpoint{3.379951in}{4.354573in}}%
\pgfpathlineto{\pgfqpoint{3.382275in}{4.394012in}}%
\pgfpathlineto{\pgfqpoint{3.384599in}{4.367843in}}%
\pgfpathlineto{\pgfqpoint{3.386923in}{4.388178in}}%
\pgfpathlineto{\pgfqpoint{3.389247in}{4.397674in}}%
\pgfpathlineto{\pgfqpoint{3.393895in}{4.355719in}}%
\pgfpathlineto{\pgfqpoint{3.396219in}{4.354229in}}%
\pgfpathlineto{\pgfqpoint{3.398543in}{4.360200in}}%
\pgfpathlineto{\pgfqpoint{3.400867in}{4.358377in}}%
\pgfpathlineto{\pgfqpoint{3.403191in}{4.360156in}}%
\pgfpathlineto{\pgfqpoint{3.405515in}{4.390672in}}%
\pgfpathlineto{\pgfqpoint{3.407838in}{4.367089in}}%
\pgfpathlineto{\pgfqpoint{3.410162in}{4.366402in}}%
\pgfpathlineto{\pgfqpoint{3.412486in}{4.369517in}}%
\pgfpathlineto{\pgfqpoint{3.414810in}{4.349824in}}%
\pgfpathlineto{\pgfqpoint{3.417134in}{4.371723in}}%
\pgfpathlineto{\pgfqpoint{3.421782in}{4.324088in}}%
\pgfpathlineto{\pgfqpoint{3.424106in}{4.368157in}}%
\pgfpathlineto{\pgfqpoint{3.426430in}{4.358667in}}%
\pgfpathlineto{\pgfqpoint{3.428754in}{4.394073in}}%
\pgfpathlineto{\pgfqpoint{3.433402in}{4.356006in}}%
\pgfpathlineto{\pgfqpoint{3.435726in}{4.379988in}}%
\pgfpathlineto{\pgfqpoint{3.438050in}{4.361515in}}%
\pgfpathlineto{\pgfqpoint{3.442698in}{4.428400in}}%
\pgfpathlineto{\pgfqpoint{3.449670in}{4.358460in}}%
\pgfpathlineto{\pgfqpoint{3.451994in}{4.382849in}}%
\pgfpathlineto{\pgfqpoint{3.454318in}{4.387519in}}%
\pgfpathlineto{\pgfqpoint{3.456642in}{4.337509in}}%
\pgfpathlineto{\pgfqpoint{3.458966in}{4.375064in}}%
\pgfpathlineto{\pgfqpoint{3.461290in}{4.391424in}}%
\pgfpathlineto{\pgfqpoint{3.465938in}{4.342473in}}%
\pgfpathlineto{\pgfqpoint{3.468262in}{4.344518in}}%
\pgfpathlineto{\pgfqpoint{3.470586in}{4.375792in}}%
\pgfpathlineto{\pgfqpoint{3.472910in}{4.361635in}}%
\pgfpathlineto{\pgfqpoint{3.475234in}{4.383553in}}%
\pgfpathlineto{\pgfqpoint{3.477557in}{4.368752in}}%
\pgfpathlineto{\pgfqpoint{3.479881in}{4.411487in}}%
\pgfpathlineto{\pgfqpoint{3.482205in}{4.370150in}}%
\pgfpathlineto{\pgfqpoint{3.484529in}{4.362306in}}%
\pgfpathlineto{\pgfqpoint{3.486853in}{4.367653in}}%
\pgfpathlineto{\pgfqpoint{3.489177in}{4.349158in}}%
\pgfpathlineto{\pgfqpoint{3.491501in}{4.383589in}}%
\pgfpathlineto{\pgfqpoint{3.493825in}{4.375689in}}%
\pgfpathlineto{\pgfqpoint{3.498473in}{4.339629in}}%
\pgfpathlineto{\pgfqpoint{3.503121in}{4.372444in}}%
\pgfpathlineto{\pgfqpoint{3.505445in}{4.381311in}}%
\pgfpathlineto{\pgfqpoint{3.507769in}{4.372648in}}%
\pgfpathlineto{\pgfqpoint{3.510093in}{4.337769in}}%
\pgfpathlineto{\pgfqpoint{3.512417in}{4.378553in}}%
\pgfpathlineto{\pgfqpoint{3.514741in}{4.392105in}}%
\pgfpathlineto{\pgfqpoint{3.517065in}{4.368520in}}%
\pgfpathlineto{\pgfqpoint{3.519389in}{4.387342in}}%
\pgfpathlineto{\pgfqpoint{3.521713in}{4.371627in}}%
\pgfpathlineto{\pgfqpoint{3.524037in}{4.400019in}}%
\pgfpathlineto{\pgfqpoint{3.528685in}{4.363882in}}%
\pgfpathlineto{\pgfqpoint{3.531009in}{4.405496in}}%
\pgfpathlineto{\pgfqpoint{3.533333in}{4.381504in}}%
\pgfpathlineto{\pgfqpoint{3.535657in}{4.393367in}}%
\pgfpathlineto{\pgfqpoint{3.542629in}{4.372784in}}%
\pgfpathlineto{\pgfqpoint{3.544953in}{4.395118in}}%
\pgfpathlineto{\pgfqpoint{3.547277in}{4.357008in}}%
\pgfpathlineto{\pgfqpoint{3.549600in}{4.339352in}}%
\pgfpathlineto{\pgfqpoint{3.551924in}{4.367273in}}%
\pgfpathlineto{\pgfqpoint{3.554248in}{4.372109in}}%
\pgfpathlineto{\pgfqpoint{3.556572in}{4.025738in}}%
\pgfpathlineto{\pgfqpoint{3.558896in}{3.974992in}}%
\pgfpathlineto{\pgfqpoint{3.561220in}{3.995404in}}%
\pgfpathlineto{\pgfqpoint{3.563544in}{3.991892in}}%
\pgfpathlineto{\pgfqpoint{3.565868in}{3.961873in}}%
\pgfpathlineto{\pgfqpoint{3.570516in}{4.010308in}}%
\pgfpathlineto{\pgfqpoint{3.572840in}{3.975202in}}%
\pgfpathlineto{\pgfqpoint{3.575164in}{3.980128in}}%
\pgfpathlineto{\pgfqpoint{3.577488in}{3.993868in}}%
\pgfpathlineto{\pgfqpoint{3.579812in}{3.970936in}}%
\pgfpathlineto{\pgfqpoint{3.586784in}{4.003711in}}%
\pgfpathlineto{\pgfqpoint{3.591432in}{3.989429in}}%
\pgfpathlineto{\pgfqpoint{3.593756in}{3.984654in}}%
\pgfpathlineto{\pgfqpoint{3.596080in}{3.975004in}}%
\pgfpathlineto{\pgfqpoint{3.598404in}{3.979809in}}%
\pgfpathlineto{\pgfqpoint{3.600728in}{3.998596in}}%
\pgfpathlineto{\pgfqpoint{3.603052in}{3.965575in}}%
\pgfpathlineto{\pgfqpoint{3.605376in}{3.961386in}}%
\pgfpathlineto{\pgfqpoint{3.607700in}{3.984807in}}%
\pgfpathlineto{\pgfqpoint{3.610024in}{3.991850in}}%
\pgfpathlineto{\pgfqpoint{3.612348in}{4.022880in}}%
\pgfpathlineto{\pgfqpoint{3.614672in}{3.996908in}}%
\pgfpathlineto{\pgfqpoint{3.616996in}{4.004973in}}%
\pgfpathlineto{\pgfqpoint{3.621643in}{3.980398in}}%
\pgfpathlineto{\pgfqpoint{3.626291in}{4.034858in}}%
\pgfpathlineto{\pgfqpoint{3.628615in}{3.960672in}}%
\pgfpathlineto{\pgfqpoint{3.630939in}{3.986755in}}%
\pgfpathlineto{\pgfqpoint{3.633263in}{3.985084in}}%
\pgfpathlineto{\pgfqpoint{3.635587in}{3.968893in}}%
\pgfpathlineto{\pgfqpoint{3.637911in}{4.011021in}}%
\pgfpathlineto{\pgfqpoint{3.640235in}{3.951814in}}%
\pgfpathlineto{\pgfqpoint{3.642559in}{3.976960in}}%
\pgfpathlineto{\pgfqpoint{3.644883in}{4.017058in}}%
\pgfpathlineto{\pgfqpoint{3.647207in}{3.985869in}}%
\pgfpathlineto{\pgfqpoint{3.649531in}{4.013507in}}%
\pgfpathlineto{\pgfqpoint{3.651855in}{3.997455in}}%
\pgfpathlineto{\pgfqpoint{3.654179in}{4.022779in}}%
\pgfpathlineto{\pgfqpoint{3.656503in}{4.029397in}}%
\pgfpathlineto{\pgfqpoint{3.658827in}{4.007762in}}%
\pgfpathlineto{\pgfqpoint{3.661151in}{3.970801in}}%
\pgfpathlineto{\pgfqpoint{3.663475in}{4.005986in}}%
\pgfpathlineto{\pgfqpoint{3.665799in}{3.994968in}}%
\pgfpathlineto{\pgfqpoint{3.668123in}{4.002700in}}%
\pgfpathlineto{\pgfqpoint{3.670447in}{3.956274in}}%
\pgfpathlineto{\pgfqpoint{3.672771in}{4.031335in}}%
\pgfpathlineto{\pgfqpoint{3.677419in}{3.984582in}}%
\pgfpathlineto{\pgfqpoint{3.682067in}{3.980172in}}%
\pgfpathlineto{\pgfqpoint{3.684391in}{3.981350in}}%
\pgfpathlineto{\pgfqpoint{3.686715in}{3.939062in}}%
\pgfpathlineto{\pgfqpoint{3.689038in}{3.994998in}}%
\pgfpathlineto{\pgfqpoint{3.691362in}{3.983378in}}%
\pgfpathlineto{\pgfqpoint{3.693686in}{3.979099in}}%
\pgfpathlineto{\pgfqpoint{3.698334in}{4.011438in}}%
\pgfpathlineto{\pgfqpoint{3.702982in}{3.987119in}}%
\pgfpathlineto{\pgfqpoint{3.705306in}{3.987735in}}%
\pgfpathlineto{\pgfqpoint{3.707630in}{3.965601in}}%
\pgfpathlineto{\pgfqpoint{3.714602in}{4.009523in}}%
\pgfpathlineto{\pgfqpoint{3.716926in}{3.993730in}}%
\pgfpathlineto{\pgfqpoint{3.719250in}{3.959167in}}%
\pgfpathlineto{\pgfqpoint{3.721574in}{3.998424in}}%
\pgfpathlineto{\pgfqpoint{3.723898in}{4.011724in}}%
\pgfpathlineto{\pgfqpoint{3.726222in}{3.999110in}}%
\pgfpathlineto{\pgfqpoint{3.728546in}{4.013736in}}%
\pgfpathlineto{\pgfqpoint{3.733194in}{3.961330in}}%
\pgfpathlineto{\pgfqpoint{3.737842in}{4.006239in}}%
\pgfpathlineto{\pgfqpoint{3.740166in}{3.972377in}}%
\pgfpathlineto{\pgfqpoint{3.742490in}{4.004532in}}%
\pgfpathlineto{\pgfqpoint{3.744814in}{3.966648in}}%
\pgfpathlineto{\pgfqpoint{3.747138in}{4.006106in}}%
\pgfpathlineto{\pgfqpoint{3.749462in}{3.987635in}}%
\pgfpathlineto{\pgfqpoint{3.751786in}{3.957109in}}%
\pgfpathlineto{\pgfqpoint{3.754110in}{3.973906in}}%
\pgfpathlineto{\pgfqpoint{3.758757in}{3.990439in}}%
\pgfpathlineto{\pgfqpoint{3.761081in}{3.982676in}}%
\pgfpathlineto{\pgfqpoint{3.763405in}{3.960125in}}%
\pgfpathlineto{\pgfqpoint{3.765729in}{3.997929in}}%
\pgfpathlineto{\pgfqpoint{3.768053in}{3.996572in}}%
\pgfpathlineto{\pgfqpoint{3.770377in}{3.999390in}}%
\pgfpathlineto{\pgfqpoint{3.772701in}{4.010666in}}%
\pgfpathlineto{\pgfqpoint{3.775025in}{3.982654in}}%
\pgfpathlineto{\pgfqpoint{3.777349in}{3.994334in}}%
\pgfpathlineto{\pgfqpoint{3.779673in}{3.983945in}}%
\pgfpathlineto{\pgfqpoint{3.781997in}{3.965311in}}%
\pgfpathlineto{\pgfqpoint{3.784321in}{3.971418in}}%
\pgfpathlineto{\pgfqpoint{3.786645in}{3.993065in}}%
\pgfpathlineto{\pgfqpoint{3.788969in}{3.981825in}}%
\pgfpathlineto{\pgfqpoint{3.791293in}{3.993966in}}%
\pgfpathlineto{\pgfqpoint{3.793617in}{3.981294in}}%
\pgfpathlineto{\pgfqpoint{3.798265in}{3.998742in}}%
\pgfpathlineto{\pgfqpoint{3.805237in}{3.975593in}}%
\pgfpathlineto{\pgfqpoint{3.807561in}{4.024565in}}%
\pgfpathlineto{\pgfqpoint{3.809885in}{3.981992in}}%
\pgfpathlineto{\pgfqpoint{3.812209in}{4.002374in}}%
\pgfpathlineto{\pgfqpoint{3.814533in}{3.987841in}}%
\pgfpathlineto{\pgfqpoint{3.816857in}{3.995671in}}%
\pgfpathlineto{\pgfqpoint{3.821505in}{3.999244in}}%
\pgfpathlineto{\pgfqpoint{3.823829in}{3.992009in}}%
\pgfpathlineto{\pgfqpoint{3.826153in}{3.994919in}}%
\pgfpathlineto{\pgfqpoint{3.828477in}{4.026517in}}%
\pgfpathlineto{\pgfqpoint{3.833124in}{3.970626in}}%
\pgfpathlineto{\pgfqpoint{3.835448in}{4.003990in}}%
\pgfpathlineto{\pgfqpoint{3.840096in}{3.954271in}}%
\pgfpathlineto{\pgfqpoint{3.842420in}{3.997147in}}%
\pgfpathlineto{\pgfqpoint{3.844744in}{4.011278in}}%
\pgfpathlineto{\pgfqpoint{3.847068in}{4.301819in}}%
\pgfpathlineto{\pgfqpoint{3.849392in}{4.351816in}}%
\pgfpathlineto{\pgfqpoint{3.851716in}{4.371356in}}%
\pgfpathlineto{\pgfqpoint{3.854040in}{4.409454in}}%
\pgfpathlineto{\pgfqpoint{3.856364in}{4.354020in}}%
\pgfpathlineto{\pgfqpoint{3.858688in}{4.383379in}}%
\pgfpathlineto{\pgfqpoint{3.863336in}{4.361131in}}%
\pgfpathlineto{\pgfqpoint{3.865660in}{4.395396in}}%
\pgfpathlineto{\pgfqpoint{3.867984in}{4.356768in}}%
\pgfpathlineto{\pgfqpoint{3.870308in}{4.348029in}}%
\pgfpathlineto{\pgfqpoint{3.872632in}{4.344149in}}%
\pgfpathlineto{\pgfqpoint{3.874956in}{4.333238in}}%
\pgfpathlineto{\pgfqpoint{3.877280in}{4.348799in}}%
\pgfpathlineto{\pgfqpoint{3.879604in}{4.387394in}}%
\pgfpathlineto{\pgfqpoint{3.881928in}{4.359346in}}%
\pgfpathlineto{\pgfqpoint{3.884252in}{4.371352in}}%
\pgfpathlineto{\pgfqpoint{3.888900in}{4.369125in}}%
\pgfpathlineto{\pgfqpoint{3.891224in}{4.381862in}}%
\pgfpathlineto{\pgfqpoint{3.893548in}{4.360911in}}%
\pgfpathlineto{\pgfqpoint{3.895872in}{4.371801in}}%
\pgfpathlineto{\pgfqpoint{3.898196in}{4.341936in}}%
\pgfpathlineto{\pgfqpoint{3.900519in}{4.369221in}}%
\pgfpathlineto{\pgfqpoint{3.902843in}{4.375593in}}%
\pgfpathlineto{\pgfqpoint{3.907491in}{4.362155in}}%
\pgfpathlineto{\pgfqpoint{3.909815in}{4.362672in}}%
\pgfpathlineto{\pgfqpoint{3.912139in}{4.374812in}}%
\pgfpathlineto{\pgfqpoint{3.914463in}{4.345268in}}%
\pgfpathlineto{\pgfqpoint{3.919111in}{4.359038in}}%
\pgfpathlineto{\pgfqpoint{3.923759in}{4.422148in}}%
\pgfpathlineto{\pgfqpoint{3.926083in}{4.380953in}}%
\pgfpathlineto{\pgfqpoint{3.928407in}{4.380975in}}%
\pgfpathlineto{\pgfqpoint{3.930731in}{4.377135in}}%
\pgfpathlineto{\pgfqpoint{3.933055in}{4.359205in}}%
\pgfpathlineto{\pgfqpoint{3.935379in}{4.399942in}}%
\pgfpathlineto{\pgfqpoint{3.937703in}{4.368814in}}%
\pgfpathlineto{\pgfqpoint{3.940027in}{4.354748in}}%
\pgfpathlineto{\pgfqpoint{3.942351in}{4.351310in}}%
\pgfpathlineto{\pgfqpoint{3.944675in}{4.399494in}}%
\pgfpathlineto{\pgfqpoint{3.946999in}{4.388189in}}%
\pgfpathlineto{\pgfqpoint{3.949323in}{4.345342in}}%
\pgfpathlineto{\pgfqpoint{3.951647in}{4.356654in}}%
\pgfpathlineto{\pgfqpoint{3.956295in}{4.407230in}}%
\pgfpathlineto{\pgfqpoint{3.958619in}{4.352318in}}%
\pgfpathlineto{\pgfqpoint{3.960943in}{4.403245in}}%
\pgfpathlineto{\pgfqpoint{3.963267in}{4.372883in}}%
\pgfpathlineto{\pgfqpoint{3.965591in}{4.372079in}}%
\pgfpathlineto{\pgfqpoint{3.967915in}{4.384210in}}%
\pgfpathlineto{\pgfqpoint{3.970238in}{4.366439in}}%
\pgfpathlineto{\pgfqpoint{3.972562in}{4.377470in}}%
\pgfpathlineto{\pgfqpoint{3.974886in}{4.365654in}}%
\pgfpathlineto{\pgfqpoint{3.977210in}{4.387248in}}%
\pgfpathlineto{\pgfqpoint{3.979534in}{4.370459in}}%
\pgfpathlineto{\pgfqpoint{3.981858in}{4.375396in}}%
\pgfpathlineto{\pgfqpoint{3.984182in}{4.408276in}}%
\pgfpathlineto{\pgfqpoint{3.986506in}{4.376174in}}%
\pgfpathlineto{\pgfqpoint{3.988830in}{4.363419in}}%
\pgfpathlineto{\pgfqpoint{3.991154in}{4.385727in}}%
\pgfpathlineto{\pgfqpoint{3.993478in}{4.360178in}}%
\pgfpathlineto{\pgfqpoint{3.998126in}{4.364774in}}%
\pgfpathlineto{\pgfqpoint{4.000450in}{4.359868in}}%
\pgfpathlineto{\pgfqpoint{4.002774in}{4.371252in}}%
\pgfpathlineto{\pgfqpoint{4.005098in}{4.369559in}}%
\pgfpathlineto{\pgfqpoint{4.007422in}{4.358704in}}%
\pgfpathlineto{\pgfqpoint{4.009746in}{4.381090in}}%
\pgfpathlineto{\pgfqpoint{4.012070in}{4.385208in}}%
\pgfpathlineto{\pgfqpoint{4.014394in}{4.360118in}}%
\pgfpathlineto{\pgfqpoint{4.019042in}{4.382928in}}%
\pgfpathlineto{\pgfqpoint{4.021366in}{4.356667in}}%
\pgfpathlineto{\pgfqpoint{4.023690in}{4.390321in}}%
\pgfpathlineto{\pgfqpoint{4.026014in}{4.371986in}}%
\pgfpathlineto{\pgfqpoint{4.028338in}{4.376703in}}%
\pgfpathlineto{\pgfqpoint{4.030662in}{4.355884in}}%
\pgfpathlineto{\pgfqpoint{4.032986in}{4.374565in}}%
\pgfpathlineto{\pgfqpoint{4.035310in}{4.363684in}}%
\pgfpathlineto{\pgfqpoint{4.037634in}{4.403465in}}%
\pgfpathlineto{\pgfqpoint{4.039957in}{4.408561in}}%
\pgfpathlineto{\pgfqpoint{4.042281in}{4.340086in}}%
\pgfpathlineto{\pgfqpoint{4.044605in}{4.387940in}}%
\pgfpathlineto{\pgfqpoint{4.049253in}{4.336220in}}%
\pgfpathlineto{\pgfqpoint{4.051577in}{4.376698in}}%
\pgfpathlineto{\pgfqpoint{4.053901in}{4.394311in}}%
\pgfpathlineto{\pgfqpoint{4.056225in}{4.368926in}}%
\pgfpathlineto{\pgfqpoint{4.058549in}{4.409835in}}%
\pgfpathlineto{\pgfqpoint{4.060873in}{4.397364in}}%
\pgfpathlineto{\pgfqpoint{4.063197in}{4.403357in}}%
\pgfpathlineto{\pgfqpoint{4.065521in}{4.330657in}}%
\pgfpathlineto{\pgfqpoint{4.067845in}{4.398643in}}%
\pgfpathlineto{\pgfqpoint{4.072493in}{4.375132in}}%
\pgfpathlineto{\pgfqpoint{4.074817in}{4.386835in}}%
\pgfpathlineto{\pgfqpoint{4.077141in}{4.392561in}}%
\pgfpathlineto{\pgfqpoint{4.079465in}{4.370592in}}%
\pgfpathlineto{\pgfqpoint{4.081789in}{4.362279in}}%
\pgfpathlineto{\pgfqpoint{4.084113in}{4.375709in}}%
\pgfpathlineto{\pgfqpoint{4.086437in}{4.355199in}}%
\pgfpathlineto{\pgfqpoint{4.088761in}{4.345811in}}%
\pgfpathlineto{\pgfqpoint{4.091085in}{4.371463in}}%
\pgfpathlineto{\pgfqpoint{4.093409in}{4.372298in}}%
\pgfpathlineto{\pgfqpoint{4.095733in}{4.379117in}}%
\pgfpathlineto{\pgfqpoint{4.098057in}{4.355082in}}%
\pgfpathlineto{\pgfqpoint{4.100381in}{4.347259in}}%
\pgfpathlineto{\pgfqpoint{4.102705in}{4.328821in}}%
\pgfpathlineto{\pgfqpoint{4.105029in}{4.390204in}}%
\pgfpathlineto{\pgfqpoint{4.107353in}{4.400663in}}%
\pgfpathlineto{\pgfqpoint{4.109677in}{4.359110in}}%
\pgfpathlineto{\pgfqpoint{4.112000in}{4.385124in}}%
\pgfpathlineto{\pgfqpoint{4.114324in}{4.371423in}}%
\pgfpathlineto{\pgfqpoint{4.116648in}{4.340548in}}%
\pgfpathlineto{\pgfqpoint{4.118972in}{4.385271in}}%
\pgfpathlineto{\pgfqpoint{4.121296in}{4.343635in}}%
\pgfpathlineto{\pgfqpoint{4.123620in}{4.359745in}}%
\pgfpathlineto{\pgfqpoint{4.125944in}{4.355580in}}%
\pgfpathlineto{\pgfqpoint{4.128268in}{4.369926in}}%
\pgfpathlineto{\pgfqpoint{4.130592in}{4.392626in}}%
\pgfpathlineto{\pgfqpoint{4.132916in}{4.341954in}}%
\pgfpathlineto{\pgfqpoint{4.135240in}{4.400680in}}%
\pgfpathlineto{\pgfqpoint{4.137564in}{3.991606in}}%
\pgfpathlineto{\pgfqpoint{4.139888in}{3.989822in}}%
\pgfpathlineto{\pgfqpoint{4.142212in}{4.006731in}}%
\pgfpathlineto{\pgfqpoint{4.144536in}{3.965888in}}%
\pgfpathlineto{\pgfqpoint{4.146860in}{3.967921in}}%
\pgfpathlineto{\pgfqpoint{4.149184in}{3.990926in}}%
\pgfpathlineto{\pgfqpoint{4.151508in}{3.981184in}}%
\pgfpathlineto{\pgfqpoint{4.153832in}{3.964775in}}%
\pgfpathlineto{\pgfqpoint{4.156156in}{4.022504in}}%
\pgfpathlineto{\pgfqpoint{4.158480in}{4.009600in}}%
\pgfpathlineto{\pgfqpoint{4.160804in}{3.968450in}}%
\pgfpathlineto{\pgfqpoint{4.163128in}{4.007830in}}%
\pgfpathlineto{\pgfqpoint{4.165452in}{4.008445in}}%
\pgfpathlineto{\pgfqpoint{4.167776in}{3.999717in}}%
\pgfpathlineto{\pgfqpoint{4.170100in}{3.973967in}}%
\pgfpathlineto{\pgfqpoint{4.172424in}{4.026127in}}%
\pgfpathlineto{\pgfqpoint{4.177072in}{3.991934in}}%
\pgfpathlineto{\pgfqpoint{4.179396in}{4.004840in}}%
\pgfpathlineto{\pgfqpoint{4.181719in}{3.997621in}}%
\pgfpathlineto{\pgfqpoint{4.186367in}{3.971885in}}%
\pgfpathlineto{\pgfqpoint{4.188691in}{4.011532in}}%
\pgfpathlineto{\pgfqpoint{4.191015in}{3.998973in}}%
\pgfpathlineto{\pgfqpoint{4.193339in}{3.997279in}}%
\pgfpathlineto{\pgfqpoint{4.195663in}{3.981222in}}%
\pgfpathlineto{\pgfqpoint{4.197987in}{3.993660in}}%
\pgfpathlineto{\pgfqpoint{4.200311in}{3.973025in}}%
\pgfpathlineto{\pgfqpoint{4.202635in}{3.998747in}}%
\pgfpathlineto{\pgfqpoint{4.204959in}{3.970958in}}%
\pgfpathlineto{\pgfqpoint{4.209607in}{4.006547in}}%
\pgfpathlineto{\pgfqpoint{4.211931in}{4.006215in}}%
\pgfpathlineto{\pgfqpoint{4.214255in}{3.974446in}}%
\pgfpathlineto{\pgfqpoint{4.216579in}{3.960962in}}%
\pgfpathlineto{\pgfqpoint{4.218903in}{3.972462in}}%
\pgfpathlineto{\pgfqpoint{4.221227in}{3.972870in}}%
\pgfpathlineto{\pgfqpoint{4.223551in}{3.951387in}}%
\pgfpathlineto{\pgfqpoint{4.225875in}{4.022211in}}%
\pgfpathlineto{\pgfqpoint{4.228199in}{4.020165in}}%
\pgfpathlineto{\pgfqpoint{4.230523in}{3.980239in}}%
\pgfpathlineto{\pgfqpoint{4.232847in}{3.983463in}}%
\pgfpathlineto{\pgfqpoint{4.235171in}{4.003851in}}%
\pgfpathlineto{\pgfqpoint{4.237495in}{3.990110in}}%
\pgfpathlineto{\pgfqpoint{4.239819in}{3.992819in}}%
\pgfpathlineto{\pgfqpoint{4.242143in}{3.990391in}}%
\pgfpathlineto{\pgfqpoint{4.244467in}{3.984649in}}%
\pgfpathlineto{\pgfqpoint{4.246791in}{4.031533in}}%
\pgfpathlineto{\pgfqpoint{4.253762in}{3.976691in}}%
\pgfpathlineto{\pgfqpoint{4.256086in}{3.986408in}}%
\pgfpathlineto{\pgfqpoint{4.258410in}{4.005045in}}%
\pgfpathlineto{\pgfqpoint{4.260734in}{3.963729in}}%
\pgfpathlineto{\pgfqpoint{4.263058in}{4.003256in}}%
\pgfpathlineto{\pgfqpoint{4.265382in}{3.992445in}}%
\pgfpathlineto{\pgfqpoint{4.267706in}{4.013696in}}%
\pgfpathlineto{\pgfqpoint{4.270030in}{4.001840in}}%
\pgfpathlineto{\pgfqpoint{4.272354in}{3.997889in}}%
\pgfpathlineto{\pgfqpoint{4.274678in}{4.015930in}}%
\pgfpathlineto{\pgfqpoint{4.277002in}{3.986382in}}%
\pgfpathlineto{\pgfqpoint{4.279326in}{3.975038in}}%
\pgfpathlineto{\pgfqpoint{4.281650in}{3.985240in}}%
\pgfpathlineto{\pgfqpoint{4.283974in}{3.985697in}}%
\pgfpathlineto{\pgfqpoint{4.286298in}{3.988889in}}%
\pgfpathlineto{\pgfqpoint{4.288622in}{3.999435in}}%
\pgfpathlineto{\pgfqpoint{4.290946in}{3.999667in}}%
\pgfpathlineto{\pgfqpoint{4.293270in}{4.014207in}}%
\pgfpathlineto{\pgfqpoint{4.295594in}{4.013123in}}%
\pgfpathlineto{\pgfqpoint{4.297918in}{4.003114in}}%
\pgfpathlineto{\pgfqpoint{4.300242in}{3.935693in}}%
\pgfpathlineto{\pgfqpoint{4.302566in}{3.972331in}}%
\pgfpathlineto{\pgfqpoint{4.304890in}{3.969388in}}%
\pgfpathlineto{\pgfqpoint{4.307214in}{4.008845in}}%
\pgfpathlineto{\pgfqpoint{4.309538in}{3.983669in}}%
\pgfpathlineto{\pgfqpoint{4.311862in}{3.984921in}}%
\pgfpathlineto{\pgfqpoint{4.314186in}{3.996101in}}%
\pgfpathlineto{\pgfqpoint{4.316510in}{4.023433in}}%
\pgfpathlineto{\pgfqpoint{4.318834in}{4.010957in}}%
\pgfpathlineto{\pgfqpoint{4.321158in}{4.004771in}}%
\pgfpathlineto{\pgfqpoint{4.323481in}{4.020632in}}%
\pgfpathlineto{\pgfqpoint{4.325805in}{3.998481in}}%
\pgfpathlineto{\pgfqpoint{4.328129in}{3.993973in}}%
\pgfpathlineto{\pgfqpoint{4.332777in}{4.001559in}}%
\pgfpathlineto{\pgfqpoint{4.335101in}{4.022232in}}%
\pgfpathlineto{\pgfqpoint{4.337425in}{3.961496in}}%
\pgfpathlineto{\pgfqpoint{4.339749in}{3.986007in}}%
\pgfpathlineto{\pgfqpoint{4.342073in}{3.980561in}}%
\pgfpathlineto{\pgfqpoint{4.344397in}{4.020338in}}%
\pgfpathlineto{\pgfqpoint{4.346721in}{3.948946in}}%
\pgfpathlineto{\pgfqpoint{4.349045in}{3.984443in}}%
\pgfpathlineto{\pgfqpoint{4.351369in}{3.976912in}}%
\pgfpathlineto{\pgfqpoint{4.353693in}{3.979398in}}%
\pgfpathlineto{\pgfqpoint{4.356017in}{4.019844in}}%
\pgfpathlineto{\pgfqpoint{4.358341in}{3.974287in}}%
\pgfpathlineto{\pgfqpoint{4.360665in}{3.963609in}}%
\pgfpathlineto{\pgfqpoint{4.362989in}{4.011454in}}%
\pgfpathlineto{\pgfqpoint{4.365313in}{3.999583in}}%
\pgfpathlineto{\pgfqpoint{4.367637in}{3.995024in}}%
\pgfpathlineto{\pgfqpoint{4.369961in}{4.011618in}}%
\pgfpathlineto{\pgfqpoint{4.372285in}{3.982634in}}%
\pgfpathlineto{\pgfqpoint{4.374609in}{4.002528in}}%
\pgfpathlineto{\pgfqpoint{4.376933in}{3.991456in}}%
\pgfpathlineto{\pgfqpoint{4.379257in}{3.993684in}}%
\pgfpathlineto{\pgfqpoint{4.381581in}{3.964629in}}%
\pgfpathlineto{\pgfqpoint{4.383905in}{3.990111in}}%
\pgfpathlineto{\pgfqpoint{4.386229in}{3.995609in}}%
\pgfpathlineto{\pgfqpoint{4.388553in}{4.004222in}}%
\pgfpathlineto{\pgfqpoint{4.390877in}{4.002878in}}%
\pgfpathlineto{\pgfqpoint{4.393200in}{3.984523in}}%
\pgfpathlineto{\pgfqpoint{4.395524in}{3.990746in}}%
\pgfpathlineto{\pgfqpoint{4.397848in}{3.956532in}}%
\pgfpathlineto{\pgfqpoint{4.400172in}{4.004376in}}%
\pgfpathlineto{\pgfqpoint{4.402496in}{3.933496in}}%
\pgfpathlineto{\pgfqpoint{4.404820in}{4.041977in}}%
\pgfpathlineto{\pgfqpoint{4.407144in}{3.984105in}}%
\pgfpathlineto{\pgfqpoint{4.411792in}{4.013041in}}%
\pgfpathlineto{\pgfqpoint{4.414116in}{3.988386in}}%
\pgfpathlineto{\pgfqpoint{4.416440in}{3.946664in}}%
\pgfpathlineto{\pgfqpoint{4.418764in}{3.984507in}}%
\pgfpathlineto{\pgfqpoint{4.421088in}{3.978130in}}%
\pgfpathlineto{\pgfqpoint{4.423412in}{3.990727in}}%
\pgfpathlineto{\pgfqpoint{4.425736in}{3.981525in}}%
\pgfpathlineto{\pgfqpoint{4.428060in}{4.371370in}}%
\pgfpathlineto{\pgfqpoint{4.430384in}{4.389625in}}%
\pgfpathlineto{\pgfqpoint{4.432708in}{4.354561in}}%
\pgfpathlineto{\pgfqpoint{4.435032in}{4.376010in}}%
\pgfpathlineto{\pgfqpoint{4.437356in}{4.369943in}}%
\pgfpathlineto{\pgfqpoint{4.439680in}{4.383781in}}%
\pgfpathlineto{\pgfqpoint{4.442004in}{4.377659in}}%
\pgfpathlineto{\pgfqpoint{4.444328in}{4.367075in}}%
\pgfpathlineto{\pgfqpoint{4.446652in}{4.391796in}}%
\pgfpathlineto{\pgfqpoint{4.448976in}{4.355370in}}%
\pgfpathlineto{\pgfqpoint{4.451300in}{4.371591in}}%
\pgfpathlineto{\pgfqpoint{4.453624in}{4.415560in}}%
\pgfpathlineto{\pgfqpoint{4.455948in}{4.399284in}}%
\pgfpathlineto{\pgfqpoint{4.458272in}{4.374877in}}%
\pgfpathlineto{\pgfqpoint{4.460596in}{4.380953in}}%
\pgfpathlineto{\pgfqpoint{4.465243in}{4.362719in}}%
\pgfpathlineto{\pgfqpoint{4.467567in}{4.364429in}}%
\pgfpathlineto{\pgfqpoint{4.469891in}{4.362281in}}%
\pgfpathlineto{\pgfqpoint{4.472215in}{4.423680in}}%
\pgfpathlineto{\pgfqpoint{4.474539in}{4.369157in}}%
\pgfpathlineto{\pgfqpoint{4.476863in}{4.393526in}}%
\pgfpathlineto{\pgfqpoint{4.481511in}{4.363232in}}%
\pgfpathlineto{\pgfqpoint{4.483835in}{4.371898in}}%
\pgfpathlineto{\pgfqpoint{4.486159in}{4.366518in}}%
\pgfpathlineto{\pgfqpoint{4.488483in}{4.346305in}}%
\pgfpathlineto{\pgfqpoint{4.490807in}{4.379214in}}%
\pgfpathlineto{\pgfqpoint{4.493131in}{4.347250in}}%
\pgfpathlineto{\pgfqpoint{4.495455in}{4.350871in}}%
\pgfpathlineto{\pgfqpoint{4.497779in}{4.380795in}}%
\pgfpathlineto{\pgfqpoint{4.500103in}{4.365071in}}%
\pgfpathlineto{\pgfqpoint{4.502427in}{4.367751in}}%
\pgfpathlineto{\pgfqpoint{4.504751in}{4.375382in}}%
\pgfpathlineto{\pgfqpoint{4.507075in}{4.373830in}}%
\pgfpathlineto{\pgfqpoint{4.509399in}{4.400059in}}%
\pgfpathlineto{\pgfqpoint{4.511723in}{4.392805in}}%
\pgfpathlineto{\pgfqpoint{4.514047in}{4.376840in}}%
\pgfpathlineto{\pgfqpoint{4.516371in}{4.381850in}}%
\pgfpathlineto{\pgfqpoint{4.518695in}{4.368176in}}%
\pgfpathlineto{\pgfqpoint{4.521019in}{4.363355in}}%
\pgfpathlineto{\pgfqpoint{4.525667in}{4.380576in}}%
\pgfpathlineto{\pgfqpoint{4.530315in}{4.360521in}}%
\pgfpathlineto{\pgfqpoint{4.532638in}{4.378188in}}%
\pgfpathlineto{\pgfqpoint{4.534962in}{4.375377in}}%
\pgfpathlineto{\pgfqpoint{4.537286in}{4.369292in}}%
\pgfpathlineto{\pgfqpoint{4.539610in}{4.367156in}}%
\pgfpathlineto{\pgfqpoint{4.541934in}{4.354667in}}%
\pgfpathlineto{\pgfqpoint{4.544258in}{4.391647in}}%
\pgfpathlineto{\pgfqpoint{4.546582in}{4.368442in}}%
\pgfpathlineto{\pgfqpoint{4.548906in}{4.378701in}}%
\pgfpathlineto{\pgfqpoint{4.551230in}{4.348789in}}%
\pgfpathlineto{\pgfqpoint{4.553554in}{4.358713in}}%
\pgfpathlineto{\pgfqpoint{4.555878in}{4.363525in}}%
\pgfpathlineto{\pgfqpoint{4.558202in}{4.383455in}}%
\pgfpathlineto{\pgfqpoint{4.560526in}{4.350496in}}%
\pgfpathlineto{\pgfqpoint{4.562850in}{4.422952in}}%
\pgfpathlineto{\pgfqpoint{4.565174in}{4.376118in}}%
\pgfpathlineto{\pgfqpoint{4.567498in}{4.378940in}}%
\pgfpathlineto{\pgfqpoint{4.569822in}{4.375443in}}%
\pgfpathlineto{\pgfqpoint{4.572146in}{4.362775in}}%
\pgfpathlineto{\pgfqpoint{4.574470in}{4.376349in}}%
\pgfpathlineto{\pgfqpoint{4.579118in}{4.381849in}}%
\pgfpathlineto{\pgfqpoint{4.581442in}{4.338664in}}%
\pgfpathlineto{\pgfqpoint{4.583766in}{4.364523in}}%
\pgfpathlineto{\pgfqpoint{4.586090in}{4.355644in}}%
\pgfpathlineto{\pgfqpoint{4.588414in}{4.358414in}}%
\pgfpathlineto{\pgfqpoint{4.590738in}{4.352210in}}%
\pgfpathlineto{\pgfqpoint{4.593062in}{4.404581in}}%
\pgfpathlineto{\pgfqpoint{4.595386in}{4.381614in}}%
\pgfpathlineto{\pgfqpoint{4.597710in}{4.370806in}}%
\pgfpathlineto{\pgfqpoint{4.600034in}{4.387630in}}%
\pgfpathlineto{\pgfqpoint{4.602358in}{4.345690in}}%
\pgfpathlineto{\pgfqpoint{4.604681in}{4.399253in}}%
\pgfpathlineto{\pgfqpoint{4.607005in}{4.356925in}}%
\pgfpathlineto{\pgfqpoint{4.609329in}{4.359532in}}%
\pgfpathlineto{\pgfqpoint{4.611653in}{4.397820in}}%
\pgfpathlineto{\pgfqpoint{4.613977in}{4.386638in}}%
\pgfpathlineto{\pgfqpoint{4.616301in}{4.364346in}}%
\pgfpathlineto{\pgfqpoint{4.618625in}{4.384205in}}%
\pgfpathlineto{\pgfqpoint{4.620949in}{4.386501in}}%
\pgfpathlineto{\pgfqpoint{4.623273in}{4.404075in}}%
\pgfpathlineto{\pgfqpoint{4.625597in}{4.383404in}}%
\pgfpathlineto{\pgfqpoint{4.627921in}{4.342848in}}%
\pgfpathlineto{\pgfqpoint{4.630245in}{4.342695in}}%
\pgfpathlineto{\pgfqpoint{4.632569in}{4.393878in}}%
\pgfpathlineto{\pgfqpoint{4.634893in}{4.368255in}}%
\pgfpathlineto{\pgfqpoint{4.637217in}{4.369170in}}%
\pgfpathlineto{\pgfqpoint{4.639541in}{4.387438in}}%
\pgfpathlineto{\pgfqpoint{4.641865in}{4.395011in}}%
\pgfpathlineto{\pgfqpoint{4.644189in}{4.391016in}}%
\pgfpathlineto{\pgfqpoint{4.646513in}{4.348661in}}%
\pgfpathlineto{\pgfqpoint{4.648837in}{4.411504in}}%
\pgfpathlineto{\pgfqpoint{4.653485in}{4.335338in}}%
\pgfpathlineto{\pgfqpoint{4.655809in}{4.387534in}}%
\pgfpathlineto{\pgfqpoint{4.658133in}{4.367162in}}%
\pgfpathlineto{\pgfqpoint{4.660457in}{4.380616in}}%
\pgfpathlineto{\pgfqpoint{4.662781in}{4.366483in}}%
\pgfpathlineto{\pgfqpoint{4.665105in}{4.359935in}}%
\pgfpathlineto{\pgfqpoint{4.669753in}{4.392565in}}%
\pgfpathlineto{\pgfqpoint{4.672077in}{4.368723in}}%
\pgfpathlineto{\pgfqpoint{4.676724in}{4.378564in}}%
\pgfpathlineto{\pgfqpoint{4.679048in}{4.393935in}}%
\pgfpathlineto{\pgfqpoint{4.681372in}{4.370481in}}%
\pgfpathlineto{\pgfqpoint{4.683696in}{4.370719in}}%
\pgfpathlineto{\pgfqpoint{4.686020in}{4.357562in}}%
\pgfpathlineto{\pgfqpoint{4.688344in}{4.397237in}}%
\pgfpathlineto{\pgfqpoint{4.690668in}{4.396196in}}%
\pgfpathlineto{\pgfqpoint{4.692992in}{4.406356in}}%
\pgfpathlineto{\pgfqpoint{4.695316in}{4.363010in}}%
\pgfpathlineto{\pgfqpoint{4.697640in}{4.391449in}}%
\pgfpathlineto{\pgfqpoint{4.699964in}{4.397369in}}%
\pgfpathlineto{\pgfqpoint{4.702288in}{4.360106in}}%
\pgfpathlineto{\pgfqpoint{4.704612in}{4.391297in}}%
\pgfpathlineto{\pgfqpoint{4.711584in}{4.350386in}}%
\pgfpathlineto{\pgfqpoint{4.713908in}{4.370052in}}%
\pgfpathlineto{\pgfqpoint{4.716232in}{4.359726in}}%
\pgfpathlineto{\pgfqpoint{4.718556in}{3.977602in}}%
\pgfpathlineto{\pgfqpoint{4.720880in}{3.963696in}}%
\pgfpathlineto{\pgfqpoint{4.723204in}{3.972706in}}%
\pgfpathlineto{\pgfqpoint{4.725528in}{4.006673in}}%
\pgfpathlineto{\pgfqpoint{4.727852in}{3.984467in}}%
\pgfpathlineto{\pgfqpoint{4.730176in}{4.019808in}}%
\pgfpathlineto{\pgfqpoint{4.732500in}{3.964237in}}%
\pgfpathlineto{\pgfqpoint{4.734824in}{3.990195in}}%
\pgfpathlineto{\pgfqpoint{4.737148in}{4.002942in}}%
\pgfpathlineto{\pgfqpoint{4.741796in}{3.967227in}}%
\pgfpathlineto{\pgfqpoint{4.744119in}{3.945705in}}%
\pgfpathlineto{\pgfqpoint{4.746443in}{4.001089in}}%
\pgfpathlineto{\pgfqpoint{4.748767in}{3.961202in}}%
\pgfpathlineto{\pgfqpoint{4.751091in}{3.958824in}}%
\pgfpathlineto{\pgfqpoint{4.753415in}{3.996691in}}%
\pgfpathlineto{\pgfqpoint{4.755739in}{3.984949in}}%
\pgfpathlineto{\pgfqpoint{4.758063in}{3.993492in}}%
\pgfpathlineto{\pgfqpoint{4.760387in}{3.992031in}}%
\pgfpathlineto{\pgfqpoint{4.762711in}{3.969446in}}%
\pgfpathlineto{\pgfqpoint{4.765035in}{3.978165in}}%
\pgfpathlineto{\pgfqpoint{4.767359in}{3.982256in}}%
\pgfpathlineto{\pgfqpoint{4.769683in}{3.966862in}}%
\pgfpathlineto{\pgfqpoint{4.772007in}{3.995283in}}%
\pgfpathlineto{\pgfqpoint{4.774331in}{4.000711in}}%
\pgfpathlineto{\pgfqpoint{4.776655in}{4.016221in}}%
\pgfpathlineto{\pgfqpoint{4.778979in}{4.002623in}}%
\pgfpathlineto{\pgfqpoint{4.781303in}{4.017006in}}%
\pgfpathlineto{\pgfqpoint{4.783627in}{3.972789in}}%
\pgfpathlineto{\pgfqpoint{4.785951in}{3.968616in}}%
\pgfpathlineto{\pgfqpoint{4.788275in}{4.007736in}}%
\pgfpathlineto{\pgfqpoint{4.790599in}{3.966005in}}%
\pgfpathlineto{\pgfqpoint{4.792923in}{4.020933in}}%
\pgfpathlineto{\pgfqpoint{4.795247in}{3.956180in}}%
\pgfpathlineto{\pgfqpoint{4.797571in}{3.979415in}}%
\pgfpathlineto{\pgfqpoint{4.799895in}{4.027479in}}%
\pgfpathlineto{\pgfqpoint{4.802219in}{3.953674in}}%
\pgfpathlineto{\pgfqpoint{4.804543in}{4.008427in}}%
\pgfpathlineto{\pgfqpoint{4.806867in}{3.999110in}}%
\pgfpathlineto{\pgfqpoint{4.809191in}{3.961597in}}%
\pgfpathlineto{\pgfqpoint{4.811515in}{4.008750in}}%
\pgfpathlineto{\pgfqpoint{4.816162in}{3.980239in}}%
\pgfpathlineto{\pgfqpoint{4.818486in}{4.038616in}}%
\pgfpathlineto{\pgfqpoint{4.820810in}{3.991307in}}%
\pgfpathlineto{\pgfqpoint{4.823134in}{3.993529in}}%
\pgfpathlineto{\pgfqpoint{4.825458in}{3.980103in}}%
\pgfpathlineto{\pgfqpoint{4.827782in}{4.002811in}}%
\pgfpathlineto{\pgfqpoint{4.830106in}{3.970428in}}%
\pgfpathlineto{\pgfqpoint{4.834754in}{4.003575in}}%
\pgfpathlineto{\pgfqpoint{4.837078in}{3.962092in}}%
\pgfpathlineto{\pgfqpoint{4.839402in}{3.965835in}}%
\pgfpathlineto{\pgfqpoint{4.841726in}{3.988898in}}%
\pgfpathlineto{\pgfqpoint{4.844050in}{3.980118in}}%
\pgfpathlineto{\pgfqpoint{4.848698in}{4.028473in}}%
\pgfpathlineto{\pgfqpoint{4.851022in}{3.977085in}}%
\pgfpathlineto{\pgfqpoint{4.853346in}{3.969743in}}%
\pgfpathlineto{\pgfqpoint{4.855670in}{3.996619in}}%
\pgfpathlineto{\pgfqpoint{4.857994in}{3.959923in}}%
\pgfpathlineto{\pgfqpoint{4.860318in}{4.009305in}}%
\pgfpathlineto{\pgfqpoint{4.862642in}{4.005182in}}%
\pgfpathlineto{\pgfqpoint{4.864966in}{3.982611in}}%
\pgfpathlineto{\pgfqpoint{4.867290in}{4.023592in}}%
\pgfpathlineto{\pgfqpoint{4.869614in}{3.976928in}}%
\pgfpathlineto{\pgfqpoint{4.871938in}{3.972039in}}%
\pgfpathlineto{\pgfqpoint{4.876586in}{4.005931in}}%
\pgfpathlineto{\pgfqpoint{4.878910in}{4.009062in}}%
\pgfpathlineto{\pgfqpoint{4.881234in}{3.989153in}}%
\pgfpathlineto{\pgfqpoint{4.883558in}{4.010054in}}%
\pgfpathlineto{\pgfqpoint{4.885881in}{4.000077in}}%
\pgfpathlineto{\pgfqpoint{4.888205in}{3.959617in}}%
\pgfpathlineto{\pgfqpoint{4.890529in}{3.983652in}}%
\pgfpathlineto{\pgfqpoint{4.892853in}{3.981980in}}%
\pgfpathlineto{\pgfqpoint{4.895177in}{4.012754in}}%
\pgfpathlineto{\pgfqpoint{4.897501in}{3.996767in}}%
\pgfpathlineto{\pgfqpoint{4.899825in}{3.960652in}}%
\pgfpathlineto{\pgfqpoint{4.902149in}{3.975978in}}%
\pgfpathlineto{\pgfqpoint{4.904473in}{3.969820in}}%
\pgfpathlineto{\pgfqpoint{4.906797in}{4.030220in}}%
\pgfpathlineto{\pgfqpoint{4.909121in}{3.962576in}}%
\pgfpathlineto{\pgfqpoint{4.911445in}{4.011264in}}%
\pgfpathlineto{\pgfqpoint{4.913769in}{3.980346in}}%
\pgfpathlineto{\pgfqpoint{4.916093in}{3.982659in}}%
\pgfpathlineto{\pgfqpoint{4.918417in}{3.993769in}}%
\pgfpathlineto{\pgfqpoint{4.920741in}{3.967121in}}%
\pgfpathlineto{\pgfqpoint{4.923065in}{4.010065in}}%
\pgfpathlineto{\pgfqpoint{4.925389in}{3.980807in}}%
\pgfpathlineto{\pgfqpoint{4.927713in}{4.021794in}}%
\pgfpathlineto{\pgfqpoint{4.930037in}{3.984224in}}%
\pgfpathlineto{\pgfqpoint{4.932361in}{4.013072in}}%
\pgfpathlineto{\pgfqpoint{4.934685in}{3.966349in}}%
\pgfpathlineto{\pgfqpoint{4.937009in}{3.973935in}}%
\pgfpathlineto{\pgfqpoint{4.939333in}{4.005724in}}%
\pgfpathlineto{\pgfqpoint{4.941657in}{3.939984in}}%
\pgfpathlineto{\pgfqpoint{4.943981in}{3.991798in}}%
\pgfpathlineto{\pgfqpoint{4.946305in}{4.009022in}}%
\pgfpathlineto{\pgfqpoint{4.948629in}{3.968387in}}%
\pgfpathlineto{\pgfqpoint{4.950953in}{4.034383in}}%
\pgfpathlineto{\pgfqpoint{4.953277in}{3.985629in}}%
\pgfpathlineto{\pgfqpoint{4.955600in}{3.963044in}}%
\pgfpathlineto{\pgfqpoint{4.957924in}{3.966535in}}%
\pgfpathlineto{\pgfqpoint{4.960248in}{3.995903in}}%
\pgfpathlineto{\pgfqpoint{4.962572in}{3.976454in}}%
\pgfpathlineto{\pgfqpoint{4.964896in}{4.003487in}}%
\pgfpathlineto{\pgfqpoint{4.967220in}{3.989674in}}%
\pgfpathlineto{\pgfqpoint{4.969544in}{3.983366in}}%
\pgfpathlineto{\pgfqpoint{4.971868in}{3.968330in}}%
\pgfpathlineto{\pgfqpoint{4.974192in}{3.969432in}}%
\pgfpathlineto{\pgfqpoint{4.976516in}{3.996586in}}%
\pgfpathlineto{\pgfqpoint{4.978840in}{3.965927in}}%
\pgfpathlineto{\pgfqpoint{4.981164in}{3.995698in}}%
\pgfpathlineto{\pgfqpoint{4.983488in}{3.989649in}}%
\pgfpathlineto{\pgfqpoint{4.985812in}{4.009407in}}%
\pgfpathlineto{\pgfqpoint{4.988136in}{3.988499in}}%
\pgfpathlineto{\pgfqpoint{4.990460in}{3.977764in}}%
\pgfpathlineto{\pgfqpoint{4.992784in}{3.950798in}}%
\pgfpathlineto{\pgfqpoint{4.995108in}{3.986121in}}%
\pgfpathlineto{\pgfqpoint{4.997432in}{4.001321in}}%
\pgfpathlineto{\pgfqpoint{4.999756in}{3.982571in}}%
\pgfpathlineto{\pgfqpoint{5.004404in}{4.024769in}}%
\pgfpathlineto{\pgfqpoint{5.006728in}{3.973296in}}%
\pgfpathlineto{\pgfqpoint{5.009052in}{4.388382in}}%
\pgfpathlineto{\pgfqpoint{5.011376in}{4.390670in}}%
\pgfpathlineto{\pgfqpoint{5.013700in}{4.370844in}}%
\pgfpathlineto{\pgfqpoint{5.016024in}{4.419935in}}%
\pgfpathlineto{\pgfqpoint{5.018348in}{4.395496in}}%
\pgfpathlineto{\pgfqpoint{5.020672in}{4.405197in}}%
\pgfpathlineto{\pgfqpoint{5.022996in}{4.367403in}}%
\pgfpathlineto{\pgfqpoint{5.027643in}{4.412770in}}%
\pgfpathlineto{\pgfqpoint{5.034615in}{4.375799in}}%
\pgfpathlineto{\pgfqpoint{5.036939in}{4.372305in}}%
\pgfpathlineto{\pgfqpoint{5.039263in}{4.386727in}}%
\pgfpathlineto{\pgfqpoint{5.041587in}{4.361353in}}%
\pgfpathlineto{\pgfqpoint{5.043911in}{4.349865in}}%
\pgfpathlineto{\pgfqpoint{5.046235in}{4.388218in}}%
\pgfpathlineto{\pgfqpoint{5.048559in}{4.378608in}}%
\pgfpathlineto{\pgfqpoint{5.050883in}{4.405239in}}%
\pgfpathlineto{\pgfqpoint{5.053207in}{4.375202in}}%
\pgfpathlineto{\pgfqpoint{5.057855in}{4.408154in}}%
\pgfpathlineto{\pgfqpoint{5.060179in}{4.404701in}}%
\pgfpathlineto{\pgfqpoint{5.062503in}{4.374596in}}%
\pgfpathlineto{\pgfqpoint{5.064827in}{4.402185in}}%
\pgfpathlineto{\pgfqpoint{5.067151in}{4.384173in}}%
\pgfpathlineto{\pgfqpoint{5.069475in}{4.352153in}}%
\pgfpathlineto{\pgfqpoint{5.071799in}{4.369381in}}%
\pgfpathlineto{\pgfqpoint{5.074123in}{4.345556in}}%
\pgfpathlineto{\pgfqpoint{5.076447in}{4.383722in}}%
\pgfpathlineto{\pgfqpoint{5.078771in}{4.380279in}}%
\pgfpathlineto{\pgfqpoint{5.081095in}{4.379360in}}%
\pgfpathlineto{\pgfqpoint{5.085743in}{4.366865in}}%
\pgfpathlineto{\pgfqpoint{5.088067in}{4.363201in}}%
\pgfpathlineto{\pgfqpoint{5.090391in}{4.352743in}}%
\pgfpathlineto{\pgfqpoint{5.092715in}{4.390076in}}%
\pgfpathlineto{\pgfqpoint{5.095039in}{4.366331in}}%
\pgfpathlineto{\pgfqpoint{5.097362in}{4.376002in}}%
\pgfpathlineto{\pgfqpoint{5.102010in}{4.360801in}}%
\pgfpathlineto{\pgfqpoint{5.104334in}{4.385530in}}%
\pgfpathlineto{\pgfqpoint{5.106658in}{4.375395in}}%
\pgfpathlineto{\pgfqpoint{5.108982in}{4.341541in}}%
\pgfpathlineto{\pgfqpoint{5.111306in}{4.383674in}}%
\pgfpathlineto{\pgfqpoint{5.113630in}{4.383598in}}%
\pgfpathlineto{\pgfqpoint{5.115954in}{4.378012in}}%
\pgfpathlineto{\pgfqpoint{5.118278in}{4.361095in}}%
\pgfpathlineto{\pgfqpoint{5.120602in}{4.386555in}}%
\pgfpathlineto{\pgfqpoint{5.122926in}{4.371832in}}%
\pgfpathlineto{\pgfqpoint{5.129898in}{4.407753in}}%
\pgfpathlineto{\pgfqpoint{5.132222in}{4.401325in}}%
\pgfpathlineto{\pgfqpoint{5.134546in}{4.380934in}}%
\pgfpathlineto{\pgfqpoint{5.136870in}{4.385257in}}%
\pgfpathlineto{\pgfqpoint{5.139194in}{4.396106in}}%
\pgfpathlineto{\pgfqpoint{5.141518in}{4.357168in}}%
\pgfpathlineto{\pgfqpoint{5.143842in}{4.364515in}}%
\pgfpathlineto{\pgfqpoint{5.146166in}{4.377558in}}%
\pgfpathlineto{\pgfqpoint{5.148490in}{4.407396in}}%
\pgfpathlineto{\pgfqpoint{5.150814in}{4.366521in}}%
\pgfpathlineto{\pgfqpoint{5.153138in}{4.386559in}}%
\pgfpathlineto{\pgfqpoint{5.160110in}{4.356236in}}%
\pgfpathlineto{\pgfqpoint{5.162434in}{4.357227in}}%
\pgfpathlineto{\pgfqpoint{5.164758in}{4.370694in}}%
\pgfpathlineto{\pgfqpoint{5.167081in}{4.368040in}}%
\pgfpathlineto{\pgfqpoint{5.169405in}{4.347922in}}%
\pgfpathlineto{\pgfqpoint{5.171729in}{4.418846in}}%
\pgfpathlineto{\pgfqpoint{5.174053in}{4.358304in}}%
\pgfpathlineto{\pgfqpoint{5.176377in}{4.355146in}}%
\pgfpathlineto{\pgfqpoint{5.181025in}{4.385928in}}%
\pgfpathlineto{\pgfqpoint{5.183349in}{4.382207in}}%
\pgfpathlineto{\pgfqpoint{5.185673in}{4.402690in}}%
\pgfpathlineto{\pgfqpoint{5.187997in}{4.375773in}}%
\pgfpathlineto{\pgfqpoint{5.192645in}{4.362569in}}%
\pgfpathlineto{\pgfqpoint{5.194969in}{4.379675in}}%
\pgfpathlineto{\pgfqpoint{5.197293in}{4.377982in}}%
\pgfpathlineto{\pgfqpoint{5.199617in}{4.392996in}}%
\pgfpathlineto{\pgfqpoint{5.204265in}{4.377925in}}%
\pgfpathlineto{\pgfqpoint{5.206589in}{4.345091in}}%
\pgfpathlineto{\pgfqpoint{5.208913in}{4.372611in}}%
\pgfpathlineto{\pgfqpoint{5.211237in}{4.360802in}}%
\pgfpathlineto{\pgfqpoint{5.213561in}{4.396319in}}%
\pgfpathlineto{\pgfqpoint{5.215885in}{4.354539in}}%
\pgfpathlineto{\pgfqpoint{5.218209in}{4.361369in}}%
\pgfpathlineto{\pgfqpoint{5.220533in}{4.353063in}}%
\pgfpathlineto{\pgfqpoint{5.222857in}{4.408457in}}%
\pgfpathlineto{\pgfqpoint{5.225181in}{4.362735in}}%
\pgfpathlineto{\pgfqpoint{5.227505in}{4.362716in}}%
\pgfpathlineto{\pgfqpoint{5.232153in}{4.391725in}}%
\pgfpathlineto{\pgfqpoint{5.236800in}{4.370148in}}%
\pgfpathlineto{\pgfqpoint{5.239124in}{4.408063in}}%
\pgfpathlineto{\pgfqpoint{5.241448in}{4.407027in}}%
\pgfpathlineto{\pgfqpoint{5.246096in}{4.362359in}}%
\pgfpathlineto{\pgfqpoint{5.248420in}{4.376553in}}%
\pgfpathlineto{\pgfqpoint{5.250744in}{4.361605in}}%
\pgfpathlineto{\pgfqpoint{5.253068in}{4.393361in}}%
\pgfpathlineto{\pgfqpoint{5.255392in}{4.353790in}}%
\pgfpathlineto{\pgfqpoint{5.257716in}{4.354896in}}%
\pgfpathlineto{\pgfqpoint{5.262364in}{4.376811in}}%
\pgfpathlineto{\pgfqpoint{5.264688in}{4.337308in}}%
\pgfpathlineto{\pgfqpoint{5.267012in}{4.366507in}}%
\pgfpathlineto{\pgfqpoint{5.269336in}{4.341362in}}%
\pgfpathlineto{\pgfqpoint{5.271660in}{4.363013in}}%
\pgfpathlineto{\pgfqpoint{5.273984in}{4.328368in}}%
\pgfpathlineto{\pgfqpoint{5.276308in}{4.383114in}}%
\pgfpathlineto{\pgfqpoint{5.278632in}{4.384693in}}%
\pgfpathlineto{\pgfqpoint{5.280956in}{4.381436in}}%
\pgfpathlineto{\pgfqpoint{5.283280in}{4.400193in}}%
\pgfpathlineto{\pgfqpoint{5.285604in}{4.373568in}}%
\pgfpathlineto{\pgfqpoint{5.290252in}{4.414564in}}%
\pgfpathlineto{\pgfqpoint{5.292576in}{4.357031in}}%
\pgfpathlineto{\pgfqpoint{5.294900in}{4.408858in}}%
\pgfpathlineto{\pgfqpoint{5.297224in}{4.345812in}}%
\pgfpathlineto{\pgfqpoint{5.299548in}{3.978061in}}%
\pgfpathlineto{\pgfqpoint{5.301872in}{4.019919in}}%
\pgfpathlineto{\pgfqpoint{5.304196in}{3.992372in}}%
\pgfpathlineto{\pgfqpoint{5.306519in}{3.999973in}}%
\pgfpathlineto{\pgfqpoint{5.311167in}{3.989948in}}%
\pgfpathlineto{\pgfqpoint{5.313491in}{3.993992in}}%
\pgfpathlineto{\pgfqpoint{5.315815in}{4.032884in}}%
\pgfpathlineto{\pgfqpoint{5.318139in}{4.011753in}}%
\pgfpathlineto{\pgfqpoint{5.320463in}{4.014971in}}%
\pgfpathlineto{\pgfqpoint{5.322787in}{3.957586in}}%
\pgfpathlineto{\pgfqpoint{5.325111in}{4.012950in}}%
\pgfpathlineto{\pgfqpoint{5.327435in}{4.006257in}}%
\pgfpathlineto{\pgfqpoint{5.329759in}{3.981351in}}%
\pgfpathlineto{\pgfqpoint{5.332083in}{4.001291in}}%
\pgfpathlineto{\pgfqpoint{5.334407in}{3.951798in}}%
\pgfpathlineto{\pgfqpoint{5.336731in}{3.991737in}}%
\pgfpathlineto{\pgfqpoint{5.339055in}{3.999107in}}%
\pgfpathlineto{\pgfqpoint{5.341379in}{4.039484in}}%
\pgfpathlineto{\pgfqpoint{5.346027in}{3.983748in}}%
\pgfpathlineto{\pgfqpoint{5.348351in}{3.976550in}}%
\pgfpathlineto{\pgfqpoint{5.350675in}{4.026304in}}%
\pgfpathlineto{\pgfqpoint{5.352999in}{3.978959in}}%
\pgfpathlineto{\pgfqpoint{5.355323in}{4.027816in}}%
\pgfpathlineto{\pgfqpoint{5.357647in}{3.981258in}}%
\pgfpathlineto{\pgfqpoint{5.359971in}{4.007306in}}%
\pgfpathlineto{\pgfqpoint{5.362295in}{3.999674in}}%
\pgfpathlineto{\pgfqpoint{5.364619in}{3.944872in}}%
\pgfpathlineto{\pgfqpoint{5.369267in}{3.991124in}}%
\pgfpathlineto{\pgfqpoint{5.371591in}{3.987838in}}%
\pgfpathlineto{\pgfqpoint{5.373915in}{3.970285in}}%
\pgfpathlineto{\pgfqpoint{5.380886in}{3.994484in}}%
\pgfpathlineto{\pgfqpoint{5.383210in}{4.012596in}}%
\pgfpathlineto{\pgfqpoint{5.385534in}{4.017700in}}%
\pgfpathlineto{\pgfqpoint{5.387858in}{4.010393in}}%
\pgfpathlineto{\pgfqpoint{5.390182in}{3.960813in}}%
\pgfpathlineto{\pgfqpoint{5.392506in}{3.973790in}}%
\pgfpathlineto{\pgfqpoint{5.394830in}{4.033617in}}%
\pgfpathlineto{\pgfqpoint{5.399478in}{3.967307in}}%
\pgfpathlineto{\pgfqpoint{5.401802in}{3.966121in}}%
\pgfpathlineto{\pgfqpoint{5.404126in}{4.007083in}}%
\pgfpathlineto{\pgfqpoint{5.406450in}{3.960383in}}%
\pgfpathlineto{\pgfqpoint{5.408774in}{4.019002in}}%
\pgfpathlineto{\pgfqpoint{5.411098in}{4.023726in}}%
\pgfpathlineto{\pgfqpoint{5.415746in}{3.972810in}}%
\pgfpathlineto{\pgfqpoint{5.418070in}{3.995895in}}%
\pgfpathlineto{\pgfqpoint{5.420394in}{3.993340in}}%
\pgfpathlineto{\pgfqpoint{5.422718in}{3.997730in}}%
\pgfpathlineto{\pgfqpoint{5.425042in}{3.988623in}}%
\pgfpathlineto{\pgfqpoint{5.427366in}{3.973183in}}%
\pgfpathlineto{\pgfqpoint{5.429690in}{3.975374in}}%
\pgfpathlineto{\pgfqpoint{5.432014in}{4.002183in}}%
\pgfpathlineto{\pgfqpoint{5.434338in}{4.004605in}}%
\pgfpathlineto{\pgfqpoint{5.436662in}{3.954448in}}%
\pgfpathlineto{\pgfqpoint{5.438986in}{4.011926in}}%
\pgfpathlineto{\pgfqpoint{5.441310in}{3.998373in}}%
\pgfpathlineto{\pgfqpoint{5.443634in}{4.005180in}}%
\pgfpathlineto{\pgfqpoint{5.448281in}{4.000969in}}%
\pgfpathlineto{\pgfqpoint{5.450605in}{3.995752in}}%
\pgfpathlineto{\pgfqpoint{5.452929in}{3.982298in}}%
\pgfpathlineto{\pgfqpoint{5.455253in}{4.023628in}}%
\pgfpathlineto{\pgfqpoint{5.457577in}{3.990120in}}%
\pgfpathlineto{\pgfqpoint{5.459901in}{4.012413in}}%
\pgfpathlineto{\pgfqpoint{5.464549in}{3.975336in}}%
\pgfpathlineto{\pgfqpoint{5.466873in}{4.011496in}}%
\pgfpathlineto{\pgfqpoint{5.469197in}{4.007745in}}%
\pgfpathlineto{\pgfqpoint{5.471521in}{4.016786in}}%
\pgfpathlineto{\pgfqpoint{5.473845in}{3.984229in}}%
\pgfpathlineto{\pgfqpoint{5.478493in}{4.014164in}}%
\pgfpathlineto{\pgfqpoint{5.480817in}{3.974042in}}%
\pgfpathlineto{\pgfqpoint{5.485465in}{4.006240in}}%
\pgfpathlineto{\pgfqpoint{5.490113in}{3.944355in}}%
\pgfpathlineto{\pgfqpoint{5.492437in}{4.009878in}}%
\pgfpathlineto{\pgfqpoint{5.494761in}{3.981325in}}%
\pgfpathlineto{\pgfqpoint{5.497085in}{4.001164in}}%
\pgfpathlineto{\pgfqpoint{5.499409in}{3.971015in}}%
\pgfpathlineto{\pgfqpoint{5.501733in}{3.969049in}}%
\pgfpathlineto{\pgfqpoint{5.504057in}{3.991126in}}%
\pgfpathlineto{\pgfqpoint{5.506381in}{4.038117in}}%
\pgfpathlineto{\pgfqpoint{5.508705in}{3.989511in}}%
\pgfpathlineto{\pgfqpoint{5.511029in}{3.977320in}}%
\pgfpathlineto{\pgfqpoint{5.513353in}{3.994188in}}%
\pgfpathlineto{\pgfqpoint{5.515677in}{3.966487in}}%
\pgfpathlineto{\pgfqpoint{5.520324in}{4.024699in}}%
\pgfpathlineto{\pgfqpoint{5.522648in}{4.000384in}}%
\pgfpathlineto{\pgfqpoint{5.524972in}{3.999660in}}%
\pgfpathlineto{\pgfqpoint{5.527296in}{3.963963in}}%
\pgfpathlineto{\pgfqpoint{5.529620in}{4.017296in}}%
\pgfpathlineto{\pgfqpoint{5.534268in}{3.976725in}}%
\pgfpathlineto{\pgfqpoint{5.536592in}{3.993779in}}%
\pgfpathlineto{\pgfqpoint{5.538916in}{3.987016in}}%
\pgfpathlineto{\pgfqpoint{5.541240in}{3.976789in}}%
\pgfpathlineto{\pgfqpoint{5.543564in}{3.981696in}}%
\pgfpathlineto{\pgfqpoint{5.545888in}{4.024677in}}%
\pgfpathlineto{\pgfqpoint{5.548212in}{3.970908in}}%
\pgfpathlineto{\pgfqpoint{5.550536in}{4.003592in}}%
\pgfpathlineto{\pgfqpoint{5.552860in}{3.970199in}}%
\pgfpathlineto{\pgfqpoint{5.555184in}{3.979011in}}%
\pgfpathlineto{\pgfqpoint{5.557508in}{4.010746in}}%
\pgfpathlineto{\pgfqpoint{5.562156in}{3.965484in}}%
\pgfpathlineto{\pgfqpoint{5.564480in}{3.996785in}}%
\pgfpathlineto{\pgfqpoint{5.566804in}{3.989686in}}%
\pgfpathlineto{\pgfqpoint{5.571452in}{3.990789in}}%
\pgfpathlineto{\pgfqpoint{5.573776in}{3.984369in}}%
\pgfpathlineto{\pgfqpoint{5.578424in}{4.005420in}}%
\pgfpathlineto{\pgfqpoint{5.580748in}{4.004029in}}%
\pgfpathlineto{\pgfqpoint{5.583072in}{3.975092in}}%
\pgfpathlineto{\pgfqpoint{5.585396in}{3.997728in}}%
\pgfpathlineto{\pgfqpoint{5.587720in}{4.378793in}}%
\pgfpathlineto{\pgfqpoint{5.587720in}{4.378793in}}%
\pgfusepath{stroke}%
\end{pgfscope}%
\begin{pgfscope}%
\pgfsetrectcap%
\pgfsetmiterjoin%
\pgfsetlinewidth{1.254687pt}%
\definecolor{currentstroke}{rgb}{0.800000,0.800000,0.800000}%
\pgfsetstrokecolor{currentstroke}%
\pgfsetdash{}{0pt}%
\pgfpathmoveto{\pgfqpoint{0.709829in}{3.729963in}}%
\pgfpathlineto{\pgfqpoint{0.709829in}{4.617500in}}%
\pgfusepath{stroke}%
\end{pgfscope}%
\begin{pgfscope}%
\pgfsetrectcap%
\pgfsetmiterjoin%
\pgfsetlinewidth{1.254687pt}%
\definecolor{currentstroke}{rgb}{0.800000,0.800000,0.800000}%
\pgfsetstrokecolor{currentstroke}%
\pgfsetdash{}{0pt}%
\pgfpathmoveto{\pgfqpoint{5.820000in}{3.729963in}}%
\pgfpathlineto{\pgfqpoint{5.820000in}{4.617500in}}%
\pgfusepath{stroke}%
\end{pgfscope}%
\begin{pgfscope}%
\pgfsetrectcap%
\pgfsetmiterjoin%
\pgfsetlinewidth{1.254687pt}%
\definecolor{currentstroke}{rgb}{0.800000,0.800000,0.800000}%
\pgfsetstrokecolor{currentstroke}%
\pgfsetdash{}{0pt}%
\pgfpathmoveto{\pgfqpoint{0.709829in}{3.729963in}}%
\pgfpathlineto{\pgfqpoint{5.820000in}{3.729963in}}%
\pgfusepath{stroke}%
\end{pgfscope}%
\begin{pgfscope}%
\pgfsetrectcap%
\pgfsetmiterjoin%
\pgfsetlinewidth{1.254687pt}%
\definecolor{currentstroke}{rgb}{0.800000,0.800000,0.800000}%
\pgfsetstrokecolor{currentstroke}%
\pgfsetdash{}{0pt}%
\pgfpathmoveto{\pgfqpoint{0.709829in}{4.617500in}}%
\pgfpathlineto{\pgfqpoint{5.820000in}{4.617500in}}%
\pgfusepath{stroke}%
\end{pgfscope}%
\begin{pgfscope}%
\definecolor{textcolor}{rgb}{0.150000,0.150000,0.150000}%
\pgfsetstrokecolor{textcolor}%
\pgfsetfillcolor{textcolor}%
\pgftext[x=3.264915in,y=4.700833in,,base]{\color{textcolor}{\sffamily\fontsize{12.000000}{14.400000}\selectfont\catcode`\^=\active\def^{\ifmmode\sp\else\^{}\fi}\catcode`\%=\active\def%{\%}Original Signals}}%
\end{pgfscope}%
\begin{pgfscope}%
\pgfsetbuttcap%
\pgfsetmiterjoin%
\definecolor{currentfill}{rgb}{1.000000,1.000000,1.000000}%
\pgfsetfillcolor{currentfill}%
\pgfsetlinewidth{0.000000pt}%
\definecolor{currentstroke}{rgb}{0.000000,0.000000,0.000000}%
\pgfsetstrokecolor{currentstroke}%
\pgfsetstrokeopacity{0.000000}%
\pgfsetdash{}{0pt}%
\pgfpathmoveto{\pgfqpoint{0.709829in}{2.192315in}}%
\pgfpathlineto{\pgfqpoint{5.820000in}{2.192315in}}%
\pgfpathlineto{\pgfqpoint{5.820000in}{3.079852in}}%
\pgfpathlineto{\pgfqpoint{0.709829in}{3.079852in}}%
\pgfpathlineto{\pgfqpoint{0.709829in}{2.192315in}}%
\pgfpathclose%
\pgfusepath{fill}%
\end{pgfscope}%
\begin{pgfscope}%
\pgfpathrectangle{\pgfqpoint{0.709829in}{2.192315in}}{\pgfqpoint{5.110171in}{0.887537in}}%
\pgfusepath{clip}%
\pgfsetroundcap%
\pgfsetroundjoin%
\pgfsetlinewidth{1.003750pt}%
\definecolor{currentstroke}{rgb}{0.800000,0.800000,0.800000}%
\pgfsetstrokecolor{currentstroke}%
\pgfsetdash{}{0pt}%
\pgfpathmoveto{\pgfqpoint{0.942110in}{2.192315in}}%
\pgfpathlineto{\pgfqpoint{0.942110in}{3.079852in}}%
\pgfusepath{stroke}%
\end{pgfscope}%
\begin{pgfscope}%
\definecolor{textcolor}{rgb}{0.150000,0.150000,0.150000}%
\pgfsetstrokecolor{textcolor}%
\pgfsetfillcolor{textcolor}%
\pgftext[x=0.942110in,y=2.060370in,,top]{\color{textcolor}{\sffamily\fontsize{11.000000}{13.200000}\selectfont\catcode`\^=\active\def^{\ifmmode\sp\else\^{}\fi}\catcode`\%=\active\def%{\%}$\mathdefault{0}$}}%
\end{pgfscope}%
\begin{pgfscope}%
\pgfpathrectangle{\pgfqpoint{0.709829in}{2.192315in}}{\pgfqpoint{5.110171in}{0.887537in}}%
\pgfusepath{clip}%
\pgfsetroundcap%
\pgfsetroundjoin%
\pgfsetlinewidth{1.003750pt}%
\definecolor{currentstroke}{rgb}{0.800000,0.800000,0.800000}%
\pgfsetstrokecolor{currentstroke}%
\pgfsetdash{}{0pt}%
\pgfpathmoveto{\pgfqpoint{1.523101in}{2.192315in}}%
\pgfpathlineto{\pgfqpoint{1.523101in}{3.079852in}}%
\pgfusepath{stroke}%
\end{pgfscope}%
\begin{pgfscope}%
\definecolor{textcolor}{rgb}{0.150000,0.150000,0.150000}%
\pgfsetstrokecolor{textcolor}%
\pgfsetfillcolor{textcolor}%
\pgftext[x=1.523101in,y=2.060370in,,top]{\color{textcolor}{\sffamily\fontsize{11.000000}{13.200000}\selectfont\catcode`\^=\active\def^{\ifmmode\sp\else\^{}\fi}\catcode`\%=\active\def%{\%}$\mathdefault{250}$}}%
\end{pgfscope}%
\begin{pgfscope}%
\pgfpathrectangle{\pgfqpoint{0.709829in}{2.192315in}}{\pgfqpoint{5.110171in}{0.887537in}}%
\pgfusepath{clip}%
\pgfsetroundcap%
\pgfsetroundjoin%
\pgfsetlinewidth{1.003750pt}%
\definecolor{currentstroke}{rgb}{0.800000,0.800000,0.800000}%
\pgfsetstrokecolor{currentstroke}%
\pgfsetdash{}{0pt}%
\pgfpathmoveto{\pgfqpoint{2.104093in}{2.192315in}}%
\pgfpathlineto{\pgfqpoint{2.104093in}{3.079852in}}%
\pgfusepath{stroke}%
\end{pgfscope}%
\begin{pgfscope}%
\definecolor{textcolor}{rgb}{0.150000,0.150000,0.150000}%
\pgfsetstrokecolor{textcolor}%
\pgfsetfillcolor{textcolor}%
\pgftext[x=2.104093in,y=2.060370in,,top]{\color{textcolor}{\sffamily\fontsize{11.000000}{13.200000}\selectfont\catcode`\^=\active\def^{\ifmmode\sp\else\^{}\fi}\catcode`\%=\active\def%{\%}$\mathdefault{500}$}}%
\end{pgfscope}%
\begin{pgfscope}%
\pgfpathrectangle{\pgfqpoint{0.709829in}{2.192315in}}{\pgfqpoint{5.110171in}{0.887537in}}%
\pgfusepath{clip}%
\pgfsetroundcap%
\pgfsetroundjoin%
\pgfsetlinewidth{1.003750pt}%
\definecolor{currentstroke}{rgb}{0.800000,0.800000,0.800000}%
\pgfsetstrokecolor{currentstroke}%
\pgfsetdash{}{0pt}%
\pgfpathmoveto{\pgfqpoint{2.685085in}{2.192315in}}%
\pgfpathlineto{\pgfqpoint{2.685085in}{3.079852in}}%
\pgfusepath{stroke}%
\end{pgfscope}%
\begin{pgfscope}%
\definecolor{textcolor}{rgb}{0.150000,0.150000,0.150000}%
\pgfsetstrokecolor{textcolor}%
\pgfsetfillcolor{textcolor}%
\pgftext[x=2.685085in,y=2.060370in,,top]{\color{textcolor}{\sffamily\fontsize{11.000000}{13.200000}\selectfont\catcode`\^=\active\def^{\ifmmode\sp\else\^{}\fi}\catcode`\%=\active\def%{\%}$\mathdefault{750}$}}%
\end{pgfscope}%
\begin{pgfscope}%
\pgfpathrectangle{\pgfqpoint{0.709829in}{2.192315in}}{\pgfqpoint{5.110171in}{0.887537in}}%
\pgfusepath{clip}%
\pgfsetroundcap%
\pgfsetroundjoin%
\pgfsetlinewidth{1.003750pt}%
\definecolor{currentstroke}{rgb}{0.800000,0.800000,0.800000}%
\pgfsetstrokecolor{currentstroke}%
\pgfsetdash{}{0pt}%
\pgfpathmoveto{\pgfqpoint{3.266076in}{2.192315in}}%
\pgfpathlineto{\pgfqpoint{3.266076in}{3.079852in}}%
\pgfusepath{stroke}%
\end{pgfscope}%
\begin{pgfscope}%
\definecolor{textcolor}{rgb}{0.150000,0.150000,0.150000}%
\pgfsetstrokecolor{textcolor}%
\pgfsetfillcolor{textcolor}%
\pgftext[x=3.266076in,y=2.060370in,,top]{\color{textcolor}{\sffamily\fontsize{11.000000}{13.200000}\selectfont\catcode`\^=\active\def^{\ifmmode\sp\else\^{}\fi}\catcode`\%=\active\def%{\%}$\mathdefault{1000}$}}%
\end{pgfscope}%
\begin{pgfscope}%
\pgfpathrectangle{\pgfqpoint{0.709829in}{2.192315in}}{\pgfqpoint{5.110171in}{0.887537in}}%
\pgfusepath{clip}%
\pgfsetroundcap%
\pgfsetroundjoin%
\pgfsetlinewidth{1.003750pt}%
\definecolor{currentstroke}{rgb}{0.800000,0.800000,0.800000}%
\pgfsetstrokecolor{currentstroke}%
\pgfsetdash{}{0pt}%
\pgfpathmoveto{\pgfqpoint{3.847068in}{2.192315in}}%
\pgfpathlineto{\pgfqpoint{3.847068in}{3.079852in}}%
\pgfusepath{stroke}%
\end{pgfscope}%
\begin{pgfscope}%
\definecolor{textcolor}{rgb}{0.150000,0.150000,0.150000}%
\pgfsetstrokecolor{textcolor}%
\pgfsetfillcolor{textcolor}%
\pgftext[x=3.847068in,y=2.060370in,,top]{\color{textcolor}{\sffamily\fontsize{11.000000}{13.200000}\selectfont\catcode`\^=\active\def^{\ifmmode\sp\else\^{}\fi}\catcode`\%=\active\def%{\%}$\mathdefault{1250}$}}%
\end{pgfscope}%
\begin{pgfscope}%
\pgfpathrectangle{\pgfqpoint{0.709829in}{2.192315in}}{\pgfqpoint{5.110171in}{0.887537in}}%
\pgfusepath{clip}%
\pgfsetroundcap%
\pgfsetroundjoin%
\pgfsetlinewidth{1.003750pt}%
\definecolor{currentstroke}{rgb}{0.800000,0.800000,0.800000}%
\pgfsetstrokecolor{currentstroke}%
\pgfsetdash{}{0pt}%
\pgfpathmoveto{\pgfqpoint{4.428060in}{2.192315in}}%
\pgfpathlineto{\pgfqpoint{4.428060in}{3.079852in}}%
\pgfusepath{stroke}%
\end{pgfscope}%
\begin{pgfscope}%
\definecolor{textcolor}{rgb}{0.150000,0.150000,0.150000}%
\pgfsetstrokecolor{textcolor}%
\pgfsetfillcolor{textcolor}%
\pgftext[x=4.428060in,y=2.060370in,,top]{\color{textcolor}{\sffamily\fontsize{11.000000}{13.200000}\selectfont\catcode`\^=\active\def^{\ifmmode\sp\else\^{}\fi}\catcode`\%=\active\def%{\%}$\mathdefault{1500}$}}%
\end{pgfscope}%
\begin{pgfscope}%
\pgfpathrectangle{\pgfqpoint{0.709829in}{2.192315in}}{\pgfqpoint{5.110171in}{0.887537in}}%
\pgfusepath{clip}%
\pgfsetroundcap%
\pgfsetroundjoin%
\pgfsetlinewidth{1.003750pt}%
\definecolor{currentstroke}{rgb}{0.800000,0.800000,0.800000}%
\pgfsetstrokecolor{currentstroke}%
\pgfsetdash{}{0pt}%
\pgfpathmoveto{\pgfqpoint{5.009052in}{2.192315in}}%
\pgfpathlineto{\pgfqpoint{5.009052in}{3.079852in}}%
\pgfusepath{stroke}%
\end{pgfscope}%
\begin{pgfscope}%
\definecolor{textcolor}{rgb}{0.150000,0.150000,0.150000}%
\pgfsetstrokecolor{textcolor}%
\pgfsetfillcolor{textcolor}%
\pgftext[x=5.009052in,y=2.060370in,,top]{\color{textcolor}{\sffamily\fontsize{11.000000}{13.200000}\selectfont\catcode`\^=\active\def^{\ifmmode\sp\else\^{}\fi}\catcode`\%=\active\def%{\%}$\mathdefault{1750}$}}%
\end{pgfscope}%
\begin{pgfscope}%
\pgfpathrectangle{\pgfqpoint{0.709829in}{2.192315in}}{\pgfqpoint{5.110171in}{0.887537in}}%
\pgfusepath{clip}%
\pgfsetroundcap%
\pgfsetroundjoin%
\pgfsetlinewidth{1.003750pt}%
\definecolor{currentstroke}{rgb}{0.800000,0.800000,0.800000}%
\pgfsetstrokecolor{currentstroke}%
\pgfsetdash{}{0pt}%
\pgfpathmoveto{\pgfqpoint{5.590043in}{2.192315in}}%
\pgfpathlineto{\pgfqpoint{5.590043in}{3.079852in}}%
\pgfusepath{stroke}%
\end{pgfscope}%
\begin{pgfscope}%
\definecolor{textcolor}{rgb}{0.150000,0.150000,0.150000}%
\pgfsetstrokecolor{textcolor}%
\pgfsetfillcolor{textcolor}%
\pgftext[x=5.590043in,y=2.060370in,,top]{\color{textcolor}{\sffamily\fontsize{11.000000}{13.200000}\selectfont\catcode`\^=\active\def^{\ifmmode\sp\else\^{}\fi}\catcode`\%=\active\def%{\%}$\mathdefault{2000}$}}%
\end{pgfscope}%
\begin{pgfscope}%
\pgfpathrectangle{\pgfqpoint{0.709829in}{2.192315in}}{\pgfqpoint{5.110171in}{0.887537in}}%
\pgfusepath{clip}%
\pgfsetroundcap%
\pgfsetroundjoin%
\pgfsetlinewidth{1.003750pt}%
\definecolor{currentstroke}{rgb}{0.800000,0.800000,0.800000}%
\pgfsetstrokecolor{currentstroke}%
\pgfsetdash{}{0pt}%
\pgfpathmoveto{\pgfqpoint{0.709829in}{2.265115in}}%
\pgfpathlineto{\pgfqpoint{5.820000in}{2.265115in}}%
\pgfusepath{stroke}%
\end{pgfscope}%
\begin{pgfscope}%
\definecolor{textcolor}{rgb}{0.150000,0.150000,0.150000}%
\pgfsetstrokecolor{textcolor}%
\pgfsetfillcolor{textcolor}%
\pgftext[x=0.383555in, y=2.212101in, left, base]{\color{textcolor}{\sffamily\fontsize{11.000000}{13.200000}\selectfont\catcode`\^=\active\def^{\ifmmode\sp\else\^{}\fi}\catcode`\%=\active\def%{\%}$\mathdefault{-2}$}}%
\end{pgfscope}%
\begin{pgfscope}%
\pgfpathrectangle{\pgfqpoint{0.709829in}{2.192315in}}{\pgfqpoint{5.110171in}{0.887537in}}%
\pgfusepath{clip}%
\pgfsetroundcap%
\pgfsetroundjoin%
\pgfsetlinewidth{1.003750pt}%
\definecolor{currentstroke}{rgb}{0.800000,0.800000,0.800000}%
\pgfsetstrokecolor{currentstroke}%
\pgfsetdash{}{0pt}%
\pgfpathmoveto{\pgfqpoint{0.709829in}{2.631526in}}%
\pgfpathlineto{\pgfqpoint{5.820000in}{2.631526in}}%
\pgfusepath{stroke}%
\end{pgfscope}%
\begin{pgfscope}%
\definecolor{textcolor}{rgb}{0.150000,0.150000,0.150000}%
\pgfsetstrokecolor{textcolor}%
\pgfsetfillcolor{textcolor}%
\pgftext[x=0.501843in, y=2.578512in, left, base]{\color{textcolor}{\sffamily\fontsize{11.000000}{13.200000}\selectfont\catcode`\^=\active\def^{\ifmmode\sp\else\^{}\fi}\catcode`\%=\active\def%{\%}$\mathdefault{0}$}}%
\end{pgfscope}%
\begin{pgfscope}%
\pgfpathrectangle{\pgfqpoint{0.709829in}{2.192315in}}{\pgfqpoint{5.110171in}{0.887537in}}%
\pgfusepath{clip}%
\pgfsetroundcap%
\pgfsetroundjoin%
\pgfsetlinewidth{1.003750pt}%
\definecolor{currentstroke}{rgb}{0.800000,0.800000,0.800000}%
\pgfsetstrokecolor{currentstroke}%
\pgfsetdash{}{0pt}%
\pgfpathmoveto{\pgfqpoint{0.709829in}{2.997938in}}%
\pgfpathlineto{\pgfqpoint{5.820000in}{2.997938in}}%
\pgfusepath{stroke}%
\end{pgfscope}%
\begin{pgfscope}%
\definecolor{textcolor}{rgb}{0.150000,0.150000,0.150000}%
\pgfsetstrokecolor{textcolor}%
\pgfsetfillcolor{textcolor}%
\pgftext[x=0.501843in, y=2.944924in, left, base]{\color{textcolor}{\sffamily\fontsize{11.000000}{13.200000}\selectfont\catcode`\^=\active\def^{\ifmmode\sp\else\^{}\fi}\catcode`\%=\active\def%{\%}$\mathdefault{2}$}}%
\end{pgfscope}%
\begin{pgfscope}%
\definecolor{textcolor}{rgb}{0.150000,0.150000,0.150000}%
\pgfsetstrokecolor{textcolor}%
\pgfsetfillcolor{textcolor}%
\pgftext[x=0.328000in,y=2.636083in,,bottom,rotate=90.000000]{\color{textcolor}{\sffamily\fontsize{12.000000}{14.400000}\selectfont\catcode`\^=\active\def^{\ifmmode\sp\else\^{}\fi}\catcode`\%=\active\def%{\%}Amplitude}}%
\end{pgfscope}%
\begin{pgfscope}%
\pgfpathrectangle{\pgfqpoint{0.709829in}{2.192315in}}{\pgfqpoint{5.110171in}{0.887537in}}%
\pgfusepath{clip}%
\pgfsetroundcap%
\pgfsetroundjoin%
\pgfsetlinewidth{1.003750pt}%
\definecolor{currentstroke}{rgb}{0.298039,0.447059,0.690196}%
\pgfsetstrokecolor{currentstroke}%
\pgfsetdash{}{0pt}%
\pgfpathmoveto{\pgfqpoint{0.942110in}{2.616948in}}%
\pgfpathlineto{\pgfqpoint{0.944433in}{2.624058in}}%
\pgfpathlineto{\pgfqpoint{0.946757in}{2.584935in}}%
\pgfpathlineto{\pgfqpoint{0.949081in}{2.610979in}}%
\pgfpathlineto{\pgfqpoint{0.951405in}{2.601600in}}%
\pgfpathlineto{\pgfqpoint{0.953729in}{2.616392in}}%
\pgfpathlineto{\pgfqpoint{0.956053in}{2.535991in}}%
\pgfpathlineto{\pgfqpoint{0.958377in}{2.604407in}}%
\pgfpathlineto{\pgfqpoint{0.960701in}{2.591854in}}%
\pgfpathlineto{\pgfqpoint{0.963025in}{2.646913in}}%
\pgfpathlineto{\pgfqpoint{0.965349in}{2.594269in}}%
\pgfpathlineto{\pgfqpoint{0.967673in}{2.573567in}}%
\pgfpathlineto{\pgfqpoint{0.969997in}{2.635866in}}%
\pgfpathlineto{\pgfqpoint{0.972321in}{2.560006in}}%
\pgfpathlineto{\pgfqpoint{0.974645in}{2.619672in}}%
\pgfpathlineto{\pgfqpoint{0.979293in}{2.576921in}}%
\pgfpathlineto{\pgfqpoint{0.981617in}{2.592431in}}%
\pgfpathlineto{\pgfqpoint{0.983941in}{2.627913in}}%
\pgfpathlineto{\pgfqpoint{0.986265in}{2.606592in}}%
\pgfpathlineto{\pgfqpoint{0.988589in}{2.596276in}}%
\pgfpathlineto{\pgfqpoint{0.990913in}{2.603491in}}%
\pgfpathlineto{\pgfqpoint{0.993237in}{2.605004in}}%
\pgfpathlineto{\pgfqpoint{0.995561in}{2.648309in}}%
\pgfpathlineto{\pgfqpoint{1.000209in}{2.602115in}}%
\pgfpathlineto{\pgfqpoint{1.004857in}{2.637370in}}%
\pgfpathlineto{\pgfqpoint{1.007181in}{2.716821in}}%
\pgfpathlineto{\pgfqpoint{1.009505in}{2.639129in}}%
\pgfpathlineto{\pgfqpoint{1.011829in}{2.671508in}}%
\pgfpathlineto{\pgfqpoint{1.016476in}{2.691233in}}%
\pgfpathlineto{\pgfqpoint{1.018800in}{2.677538in}}%
\pgfpathlineto{\pgfqpoint{1.021124in}{2.645109in}}%
\pgfpathlineto{\pgfqpoint{1.023448in}{2.664123in}}%
\pgfpathlineto{\pgfqpoint{1.025772in}{2.744144in}}%
\pgfpathlineto{\pgfqpoint{1.028096in}{2.688982in}}%
\pgfpathlineto{\pgfqpoint{1.030420in}{2.695993in}}%
\pgfpathlineto{\pgfqpoint{1.032744in}{2.695663in}}%
\pgfpathlineto{\pgfqpoint{1.035068in}{2.660738in}}%
\pgfpathlineto{\pgfqpoint{1.037392in}{2.675415in}}%
\pgfpathlineto{\pgfqpoint{1.039716in}{2.720477in}}%
\pgfpathlineto{\pgfqpoint{1.042040in}{2.678748in}}%
\pgfpathlineto{\pgfqpoint{1.044364in}{2.666456in}}%
\pgfpathlineto{\pgfqpoint{1.046688in}{2.702024in}}%
\pgfpathlineto{\pgfqpoint{1.049012in}{2.648370in}}%
\pgfpathlineto{\pgfqpoint{1.051336in}{2.709012in}}%
\pgfpathlineto{\pgfqpoint{1.053660in}{2.725730in}}%
\pgfpathlineto{\pgfqpoint{1.055984in}{2.675534in}}%
\pgfpathlineto{\pgfqpoint{1.058308in}{2.719924in}}%
\pgfpathlineto{\pgfqpoint{1.060632in}{2.678847in}}%
\pgfpathlineto{\pgfqpoint{1.062956in}{2.677116in}}%
\pgfpathlineto{\pgfqpoint{1.065280in}{2.720190in}}%
\pgfpathlineto{\pgfqpoint{1.069928in}{2.676610in}}%
\pgfpathlineto{\pgfqpoint{1.072252in}{2.747077in}}%
\pgfpathlineto{\pgfqpoint{1.076900in}{2.706778in}}%
\pgfpathlineto{\pgfqpoint{1.079224in}{2.703272in}}%
\pgfpathlineto{\pgfqpoint{1.081548in}{2.681191in}}%
\pgfpathlineto{\pgfqpoint{1.083872in}{2.686784in}}%
\pgfpathlineto{\pgfqpoint{1.088519in}{2.726135in}}%
\pgfpathlineto{\pgfqpoint{1.090843in}{2.716528in}}%
\pgfpathlineto{\pgfqpoint{1.093167in}{2.760487in}}%
\pgfpathlineto{\pgfqpoint{1.095491in}{2.782932in}}%
\pgfpathlineto{\pgfqpoint{1.097815in}{2.768400in}}%
\pgfpathlineto{\pgfqpoint{1.100139in}{2.716686in}}%
\pgfpathlineto{\pgfqpoint{1.102463in}{2.739520in}}%
\pgfpathlineto{\pgfqpoint{1.104787in}{2.775427in}}%
\pgfpathlineto{\pgfqpoint{1.107111in}{2.718352in}}%
\pgfpathlineto{\pgfqpoint{1.109435in}{2.793252in}}%
\pgfpathlineto{\pgfqpoint{1.111759in}{2.728786in}}%
\pgfpathlineto{\pgfqpoint{1.114083in}{2.736390in}}%
\pgfpathlineto{\pgfqpoint{1.116407in}{2.765694in}}%
\pgfpathlineto{\pgfqpoint{1.118731in}{2.727070in}}%
\pgfpathlineto{\pgfqpoint{1.123379in}{2.790347in}}%
\pgfpathlineto{\pgfqpoint{1.125703in}{2.762439in}}%
\pgfpathlineto{\pgfqpoint{1.128027in}{2.752668in}}%
\pgfpathlineto{\pgfqpoint{1.130351in}{2.736982in}}%
\pgfpathlineto{\pgfqpoint{1.132675in}{2.797005in}}%
\pgfpathlineto{\pgfqpoint{1.134999in}{2.761410in}}%
\pgfpathlineto{\pgfqpoint{1.137323in}{2.785268in}}%
\pgfpathlineto{\pgfqpoint{1.139647in}{2.754059in}}%
\pgfpathlineto{\pgfqpoint{1.141971in}{2.759344in}}%
\pgfpathlineto{\pgfqpoint{1.148943in}{2.787203in}}%
\pgfpathlineto{\pgfqpoint{1.151267in}{2.772847in}}%
\pgfpathlineto{\pgfqpoint{1.153591in}{2.781009in}}%
\pgfpathlineto{\pgfqpoint{1.155914in}{2.769927in}}%
\pgfpathlineto{\pgfqpoint{1.158238in}{2.837505in}}%
\pgfpathlineto{\pgfqpoint{1.160562in}{2.829072in}}%
\pgfpathlineto{\pgfqpoint{1.162886in}{2.809486in}}%
\pgfpathlineto{\pgfqpoint{1.165210in}{2.830976in}}%
\pgfpathlineto{\pgfqpoint{1.169858in}{2.833753in}}%
\pgfpathlineto{\pgfqpoint{1.172182in}{2.818076in}}%
\pgfpathlineto{\pgfqpoint{1.174506in}{2.831630in}}%
\pgfpathlineto{\pgfqpoint{1.176830in}{2.816216in}}%
\pgfpathlineto{\pgfqpoint{1.179154in}{2.788641in}}%
\pgfpathlineto{\pgfqpoint{1.181478in}{2.834509in}}%
\pgfpathlineto{\pgfqpoint{1.183802in}{2.791136in}}%
\pgfpathlineto{\pgfqpoint{1.186126in}{2.822732in}}%
\pgfpathlineto{\pgfqpoint{1.188450in}{2.815178in}}%
\pgfpathlineto{\pgfqpoint{1.193098in}{2.811390in}}%
\pgfpathlineto{\pgfqpoint{1.195422in}{2.862470in}}%
\pgfpathlineto{\pgfqpoint{1.197746in}{2.834859in}}%
\pgfpathlineto{\pgfqpoint{1.200070in}{2.782820in}}%
\pgfpathlineto{\pgfqpoint{1.202394in}{2.820195in}}%
\pgfpathlineto{\pgfqpoint{1.204718in}{2.877925in}}%
\pgfpathlineto{\pgfqpoint{1.207042in}{2.807664in}}%
\pgfpathlineto{\pgfqpoint{1.209366in}{2.840602in}}%
\pgfpathlineto{\pgfqpoint{1.211690in}{2.852765in}}%
\pgfpathlineto{\pgfqpoint{1.214014in}{2.852715in}}%
\pgfpathlineto{\pgfqpoint{1.216338in}{2.808852in}}%
\pgfpathlineto{\pgfqpoint{1.220986in}{2.881910in}}%
\pgfpathlineto{\pgfqpoint{1.227957in}{2.842877in}}%
\pgfpathlineto{\pgfqpoint{1.230281in}{2.873427in}}%
\pgfpathlineto{\pgfqpoint{1.232605in}{2.556794in}}%
\pgfpathlineto{\pgfqpoint{1.234929in}{2.553462in}}%
\pgfpathlineto{\pgfqpoint{1.237253in}{2.510640in}}%
\pgfpathlineto{\pgfqpoint{1.239577in}{2.521105in}}%
\pgfpathlineto{\pgfqpoint{1.241901in}{2.588827in}}%
\pgfpathlineto{\pgfqpoint{1.244225in}{2.569558in}}%
\pgfpathlineto{\pgfqpoint{1.246549in}{2.538543in}}%
\pgfpathlineto{\pgfqpoint{1.248873in}{2.632084in}}%
\pgfpathlineto{\pgfqpoint{1.251197in}{2.559847in}}%
\pgfpathlineto{\pgfqpoint{1.253521in}{2.569795in}}%
\pgfpathlineto{\pgfqpoint{1.255845in}{2.604223in}}%
\pgfpathlineto{\pgfqpoint{1.258169in}{2.544627in}}%
\pgfpathlineto{\pgfqpoint{1.262817in}{2.628524in}}%
\pgfpathlineto{\pgfqpoint{1.265141in}{2.649319in}}%
\pgfpathlineto{\pgfqpoint{1.267465in}{2.583240in}}%
\pgfpathlineto{\pgfqpoint{1.269789in}{2.595364in}}%
\pgfpathlineto{\pgfqpoint{1.272113in}{2.544920in}}%
\pgfpathlineto{\pgfqpoint{1.279085in}{2.622998in}}%
\pgfpathlineto{\pgfqpoint{1.283733in}{2.594641in}}%
\pgfpathlineto{\pgfqpoint{1.286057in}{2.623746in}}%
\pgfpathlineto{\pgfqpoint{1.288381in}{2.582282in}}%
\pgfpathlineto{\pgfqpoint{1.293029in}{2.655630in}}%
\pgfpathlineto{\pgfqpoint{1.295353in}{2.570432in}}%
\pgfpathlineto{\pgfqpoint{1.300000in}{2.611062in}}%
\pgfpathlineto{\pgfqpoint{1.302324in}{2.619042in}}%
\pgfpathlineto{\pgfqpoint{1.304648in}{2.571431in}}%
\pgfpathlineto{\pgfqpoint{1.306972in}{2.603307in}}%
\pgfpathlineto{\pgfqpoint{1.309296in}{2.688093in}}%
\pgfpathlineto{\pgfqpoint{1.311620in}{2.649447in}}%
\pgfpathlineto{\pgfqpoint{1.313944in}{2.646996in}}%
\pgfpathlineto{\pgfqpoint{1.316268in}{2.604759in}}%
\pgfpathlineto{\pgfqpoint{1.318592in}{2.663148in}}%
\pgfpathlineto{\pgfqpoint{1.320916in}{2.613850in}}%
\pgfpathlineto{\pgfqpoint{1.323240in}{2.718937in}}%
\pgfpathlineto{\pgfqpoint{1.325564in}{2.618915in}}%
\pgfpathlineto{\pgfqpoint{1.327888in}{2.635048in}}%
\pgfpathlineto{\pgfqpoint{1.330212in}{2.662119in}}%
\pgfpathlineto{\pgfqpoint{1.332536in}{2.605492in}}%
\pgfpathlineto{\pgfqpoint{1.334860in}{2.626325in}}%
\pgfpathlineto{\pgfqpoint{1.337184in}{2.666967in}}%
\pgfpathlineto{\pgfqpoint{1.339508in}{2.618215in}}%
\pgfpathlineto{\pgfqpoint{1.341832in}{2.684265in}}%
\pgfpathlineto{\pgfqpoint{1.346480in}{2.623331in}}%
\pgfpathlineto{\pgfqpoint{1.348804in}{2.688517in}}%
\pgfpathlineto{\pgfqpoint{1.351128in}{2.615431in}}%
\pgfpathlineto{\pgfqpoint{1.353452in}{2.692016in}}%
\pgfpathlineto{\pgfqpoint{1.355776in}{2.633236in}}%
\pgfpathlineto{\pgfqpoint{1.358100in}{2.657655in}}%
\pgfpathlineto{\pgfqpoint{1.360424in}{2.641992in}}%
\pgfpathlineto{\pgfqpoint{1.362748in}{2.672132in}}%
\pgfpathlineto{\pgfqpoint{1.365072in}{2.650552in}}%
\pgfpathlineto{\pgfqpoint{1.367395in}{2.669673in}}%
\pgfpathlineto{\pgfqpoint{1.369719in}{2.669220in}}%
\pgfpathlineto{\pgfqpoint{1.372043in}{2.644177in}}%
\pgfpathlineto{\pgfqpoint{1.374367in}{2.677442in}}%
\pgfpathlineto{\pgfqpoint{1.376691in}{2.654944in}}%
\pgfpathlineto{\pgfqpoint{1.381339in}{2.697738in}}%
\pgfpathlineto{\pgfqpoint{1.383663in}{2.653474in}}%
\pgfpathlineto{\pgfqpoint{1.385987in}{2.700600in}}%
\pgfpathlineto{\pgfqpoint{1.388311in}{2.653594in}}%
\pgfpathlineto{\pgfqpoint{1.390635in}{2.653117in}}%
\pgfpathlineto{\pgfqpoint{1.392959in}{2.684179in}}%
\pgfpathlineto{\pgfqpoint{1.395283in}{2.689064in}}%
\pgfpathlineto{\pgfqpoint{1.397607in}{2.636947in}}%
\pgfpathlineto{\pgfqpoint{1.399931in}{2.679270in}}%
\pgfpathlineto{\pgfqpoint{1.402255in}{2.701151in}}%
\pgfpathlineto{\pgfqpoint{1.404579in}{2.674607in}}%
\pgfpathlineto{\pgfqpoint{1.406903in}{2.675601in}}%
\pgfpathlineto{\pgfqpoint{1.409227in}{2.667257in}}%
\pgfpathlineto{\pgfqpoint{1.411551in}{2.670536in}}%
\pgfpathlineto{\pgfqpoint{1.413875in}{2.645523in}}%
\pgfpathlineto{\pgfqpoint{1.416199in}{2.728856in}}%
\pgfpathlineto{\pgfqpoint{1.418523in}{2.721820in}}%
\pgfpathlineto{\pgfqpoint{1.420847in}{2.704594in}}%
\pgfpathlineto{\pgfqpoint{1.425495in}{2.686259in}}%
\pgfpathlineto{\pgfqpoint{1.427819in}{2.731950in}}%
\pgfpathlineto{\pgfqpoint{1.430143in}{2.711954in}}%
\pgfpathlineto{\pgfqpoint{1.432467in}{2.709488in}}%
\pgfpathlineto{\pgfqpoint{1.434791in}{2.725008in}}%
\pgfpathlineto{\pgfqpoint{1.437114in}{2.754609in}}%
\pgfpathlineto{\pgfqpoint{1.439438in}{2.694558in}}%
\pgfpathlineto{\pgfqpoint{1.441762in}{2.738205in}}%
\pgfpathlineto{\pgfqpoint{1.444086in}{2.752111in}}%
\pgfpathlineto{\pgfqpoint{1.446410in}{2.712606in}}%
\pgfpathlineto{\pgfqpoint{1.448734in}{2.737281in}}%
\pgfpathlineto{\pgfqpoint{1.451058in}{2.690948in}}%
\pgfpathlineto{\pgfqpoint{1.458030in}{2.722738in}}%
\pgfpathlineto{\pgfqpoint{1.460354in}{2.698396in}}%
\pgfpathlineto{\pgfqpoint{1.462678in}{2.697807in}}%
\pgfpathlineto{\pgfqpoint{1.465002in}{2.738157in}}%
\pgfpathlineto{\pgfqpoint{1.467326in}{2.753278in}}%
\pgfpathlineto{\pgfqpoint{1.469650in}{2.780722in}}%
\pgfpathlineto{\pgfqpoint{1.471974in}{2.709334in}}%
\pgfpathlineto{\pgfqpoint{1.474298in}{2.759153in}}%
\pgfpathlineto{\pgfqpoint{1.478946in}{2.730199in}}%
\pgfpathlineto{\pgfqpoint{1.481270in}{2.729271in}}%
\pgfpathlineto{\pgfqpoint{1.483594in}{2.712665in}}%
\pgfpathlineto{\pgfqpoint{1.488242in}{2.792828in}}%
\pgfpathlineto{\pgfqpoint{1.490566in}{2.759353in}}%
\pgfpathlineto{\pgfqpoint{1.492890in}{2.742196in}}%
\pgfpathlineto{\pgfqpoint{1.495214in}{2.754946in}}%
\pgfpathlineto{\pgfqpoint{1.497538in}{2.761492in}}%
\pgfpathlineto{\pgfqpoint{1.499862in}{2.744594in}}%
\pgfpathlineto{\pgfqpoint{1.502186in}{2.707753in}}%
\pgfpathlineto{\pgfqpoint{1.504510in}{2.785216in}}%
\pgfpathlineto{\pgfqpoint{1.506834in}{2.744437in}}%
\pgfpathlineto{\pgfqpoint{1.509157in}{2.773027in}}%
\pgfpathlineto{\pgfqpoint{1.511481in}{2.727946in}}%
\pgfpathlineto{\pgfqpoint{1.513805in}{2.724958in}}%
\pgfpathlineto{\pgfqpoint{1.516129in}{2.784598in}}%
\pgfpathlineto{\pgfqpoint{1.518453in}{2.768588in}}%
\pgfpathlineto{\pgfqpoint{1.520777in}{2.744387in}}%
\pgfpathlineto{\pgfqpoint{1.523101in}{2.662365in}}%
\pgfpathlineto{\pgfqpoint{1.525425in}{2.712720in}}%
\pgfpathlineto{\pgfqpoint{1.527749in}{2.676210in}}%
\pgfpathlineto{\pgfqpoint{1.530073in}{2.659896in}}%
\pgfpathlineto{\pgfqpoint{1.532397in}{2.704161in}}%
\pgfpathlineto{\pgfqpoint{1.534721in}{2.670792in}}%
\pgfpathlineto{\pgfqpoint{1.537045in}{2.696130in}}%
\pgfpathlineto{\pgfqpoint{1.541693in}{2.648133in}}%
\pgfpathlineto{\pgfqpoint{1.544017in}{2.652849in}}%
\pgfpathlineto{\pgfqpoint{1.546341in}{2.728189in}}%
\pgfpathlineto{\pgfqpoint{1.548665in}{2.688353in}}%
\pgfpathlineto{\pgfqpoint{1.550989in}{2.686990in}}%
\pgfpathlineto{\pgfqpoint{1.553313in}{2.678181in}}%
\pgfpathlineto{\pgfqpoint{1.555637in}{2.697962in}}%
\pgfpathlineto{\pgfqpoint{1.557961in}{2.671136in}}%
\pgfpathlineto{\pgfqpoint{1.560285in}{2.698066in}}%
\pgfpathlineto{\pgfqpoint{1.562609in}{2.700865in}}%
\pgfpathlineto{\pgfqpoint{1.564933in}{2.726606in}}%
\pgfpathlineto{\pgfqpoint{1.567257in}{2.665508in}}%
\pgfpathlineto{\pgfqpoint{1.569581in}{2.724260in}}%
\pgfpathlineto{\pgfqpoint{1.571905in}{2.731492in}}%
\pgfpathlineto{\pgfqpoint{1.576553in}{2.705509in}}%
\pgfpathlineto{\pgfqpoint{1.578876in}{2.674587in}}%
\pgfpathlineto{\pgfqpoint{1.581200in}{2.751951in}}%
\pgfpathlineto{\pgfqpoint{1.583524in}{2.719253in}}%
\pgfpathlineto{\pgfqpoint{1.585848in}{2.720586in}}%
\pgfpathlineto{\pgfqpoint{1.588172in}{2.715228in}}%
\pgfpathlineto{\pgfqpoint{1.590496in}{2.721811in}}%
\pgfpathlineto{\pgfqpoint{1.592820in}{2.722899in}}%
\pgfpathlineto{\pgfqpoint{1.595144in}{2.743437in}}%
\pgfpathlineto{\pgfqpoint{1.599792in}{2.646897in}}%
\pgfpathlineto{\pgfqpoint{1.602116in}{2.669707in}}%
\pgfpathlineto{\pgfqpoint{1.604440in}{2.673545in}}%
\pgfpathlineto{\pgfqpoint{1.606764in}{2.730460in}}%
\pgfpathlineto{\pgfqpoint{1.609088in}{2.723371in}}%
\pgfpathlineto{\pgfqpoint{1.611412in}{2.701094in}}%
\pgfpathlineto{\pgfqpoint{1.616060in}{2.718652in}}%
\pgfpathlineto{\pgfqpoint{1.620708in}{2.705345in}}%
\pgfpathlineto{\pgfqpoint{1.623032in}{2.715725in}}%
\pgfpathlineto{\pgfqpoint{1.625356in}{2.710828in}}%
\pgfpathlineto{\pgfqpoint{1.627680in}{2.727151in}}%
\pgfpathlineto{\pgfqpoint{1.630004in}{2.705222in}}%
\pgfpathlineto{\pgfqpoint{1.632328in}{2.750094in}}%
\pgfpathlineto{\pgfqpoint{1.634652in}{2.739658in}}%
\pgfpathlineto{\pgfqpoint{1.636976in}{2.757962in}}%
\pgfpathlineto{\pgfqpoint{1.639300in}{2.706629in}}%
\pgfpathlineto{\pgfqpoint{1.641624in}{2.745795in}}%
\pgfpathlineto{\pgfqpoint{1.643948in}{2.718540in}}%
\pgfpathlineto{\pgfqpoint{1.646272in}{2.711945in}}%
\pgfpathlineto{\pgfqpoint{1.648595in}{2.687639in}}%
\pgfpathlineto{\pgfqpoint{1.650919in}{2.742101in}}%
\pgfpathlineto{\pgfqpoint{1.653243in}{2.717977in}}%
\pgfpathlineto{\pgfqpoint{1.655567in}{2.782035in}}%
\pgfpathlineto{\pgfqpoint{1.657891in}{2.781288in}}%
\pgfpathlineto{\pgfqpoint{1.662539in}{2.758466in}}%
\pgfpathlineto{\pgfqpoint{1.664863in}{2.706008in}}%
\pgfpathlineto{\pgfqpoint{1.667187in}{2.751462in}}%
\pgfpathlineto{\pgfqpoint{1.669511in}{2.763868in}}%
\pgfpathlineto{\pgfqpoint{1.671835in}{2.802877in}}%
\pgfpathlineto{\pgfqpoint{1.676483in}{2.734375in}}%
\pgfpathlineto{\pgfqpoint{1.678807in}{2.770502in}}%
\pgfpathlineto{\pgfqpoint{1.681131in}{2.771301in}}%
\pgfpathlineto{\pgfqpoint{1.683455in}{2.765317in}}%
\pgfpathlineto{\pgfqpoint{1.685779in}{2.767525in}}%
\pgfpathlineto{\pgfqpoint{1.688103in}{2.748087in}}%
\pgfpathlineto{\pgfqpoint{1.690427in}{2.792107in}}%
\pgfpathlineto{\pgfqpoint{1.692751in}{2.762135in}}%
\pgfpathlineto{\pgfqpoint{1.695075in}{2.812960in}}%
\pgfpathlineto{\pgfqpoint{1.697399in}{2.727686in}}%
\pgfpathlineto{\pgfqpoint{1.702047in}{2.761682in}}%
\pgfpathlineto{\pgfqpoint{1.704371in}{2.746382in}}%
\pgfpathlineto{\pgfqpoint{1.706695in}{2.786982in}}%
\pgfpathlineto{\pgfqpoint{1.709019in}{2.752890in}}%
\pgfpathlineto{\pgfqpoint{1.711343in}{2.772153in}}%
\pgfpathlineto{\pgfqpoint{1.713667in}{2.739947in}}%
\pgfpathlineto{\pgfqpoint{1.715991in}{2.779374in}}%
\pgfpathlineto{\pgfqpoint{1.720638in}{2.745428in}}%
\pgfpathlineto{\pgfqpoint{1.722962in}{2.749845in}}%
\pgfpathlineto{\pgfqpoint{1.725286in}{2.808536in}}%
\pgfpathlineto{\pgfqpoint{1.727610in}{2.738794in}}%
\pgfpathlineto{\pgfqpoint{1.729934in}{2.749506in}}%
\pgfpathlineto{\pgfqpoint{1.732258in}{2.783344in}}%
\pgfpathlineto{\pgfqpoint{1.734582in}{2.773779in}}%
\pgfpathlineto{\pgfqpoint{1.736906in}{2.748981in}}%
\pgfpathlineto{\pgfqpoint{1.739230in}{2.747351in}}%
\pgfpathlineto{\pgfqpoint{1.741554in}{2.762685in}}%
\pgfpathlineto{\pgfqpoint{1.743878in}{2.811419in}}%
\pgfpathlineto{\pgfqpoint{1.746202in}{2.763519in}}%
\pgfpathlineto{\pgfqpoint{1.748526in}{2.801840in}}%
\pgfpathlineto{\pgfqpoint{1.750850in}{2.801518in}}%
\pgfpathlineto{\pgfqpoint{1.753174in}{2.751095in}}%
\pgfpathlineto{\pgfqpoint{1.755498in}{2.742749in}}%
\pgfpathlineto{\pgfqpoint{1.757822in}{2.850001in}}%
\pgfpathlineto{\pgfqpoint{1.760146in}{2.777443in}}%
\pgfpathlineto{\pgfqpoint{1.762470in}{2.750773in}}%
\pgfpathlineto{\pgfqpoint{1.764794in}{2.776425in}}%
\pgfpathlineto{\pgfqpoint{1.767118in}{2.780805in}}%
\pgfpathlineto{\pgfqpoint{1.769442in}{2.771164in}}%
\pgfpathlineto{\pgfqpoint{1.771766in}{2.736266in}}%
\pgfpathlineto{\pgfqpoint{1.774090in}{2.805834in}}%
\pgfpathlineto{\pgfqpoint{1.776414in}{2.753325in}}%
\pgfpathlineto{\pgfqpoint{1.778738in}{2.776459in}}%
\pgfpathlineto{\pgfqpoint{1.781062in}{2.747167in}}%
\pgfpathlineto{\pgfqpoint{1.783386in}{2.760874in}}%
\pgfpathlineto{\pgfqpoint{1.785710in}{2.759899in}}%
\pgfpathlineto{\pgfqpoint{1.790357in}{2.794108in}}%
\pgfpathlineto{\pgfqpoint{1.792681in}{2.782557in}}%
\pgfpathlineto{\pgfqpoint{1.795005in}{2.801736in}}%
\pgfpathlineto{\pgfqpoint{1.797329in}{2.775115in}}%
\pgfpathlineto{\pgfqpoint{1.799653in}{2.823286in}}%
\pgfpathlineto{\pgfqpoint{1.801977in}{2.820106in}}%
\pgfpathlineto{\pgfqpoint{1.804301in}{2.845610in}}%
\pgfpathlineto{\pgfqpoint{1.806625in}{2.812262in}}%
\pgfpathlineto{\pgfqpoint{1.808949in}{2.816497in}}%
\pgfpathlineto{\pgfqpoint{1.811273in}{2.778098in}}%
\pgfpathlineto{\pgfqpoint{1.813597in}{2.502536in}}%
\pgfpathlineto{\pgfqpoint{1.815921in}{2.420913in}}%
\pgfpathlineto{\pgfqpoint{1.818245in}{2.507282in}}%
\pgfpathlineto{\pgfqpoint{1.820569in}{2.479327in}}%
\pgfpathlineto{\pgfqpoint{1.822893in}{2.514561in}}%
\pgfpathlineto{\pgfqpoint{1.825217in}{2.465054in}}%
\pgfpathlineto{\pgfqpoint{1.829865in}{2.499656in}}%
\pgfpathlineto{\pgfqpoint{1.832189in}{2.459734in}}%
\pgfpathlineto{\pgfqpoint{1.834513in}{2.503451in}}%
\pgfpathlineto{\pgfqpoint{1.836837in}{2.509457in}}%
\pgfpathlineto{\pgfqpoint{1.839161in}{2.469584in}}%
\pgfpathlineto{\pgfqpoint{1.841485in}{2.534411in}}%
\pgfpathlineto{\pgfqpoint{1.843809in}{2.530729in}}%
\pgfpathlineto{\pgfqpoint{1.848457in}{2.481915in}}%
\pgfpathlineto{\pgfqpoint{1.850781in}{2.524893in}}%
\pgfpathlineto{\pgfqpoint{1.853105in}{2.492262in}}%
\pgfpathlineto{\pgfqpoint{1.855429in}{2.541860in}}%
\pgfpathlineto{\pgfqpoint{1.857753in}{2.453764in}}%
\pgfpathlineto{\pgfqpoint{1.860076in}{2.513180in}}%
\pgfpathlineto{\pgfqpoint{1.862400in}{2.493370in}}%
\pgfpathlineto{\pgfqpoint{1.864724in}{2.497171in}}%
\pgfpathlineto{\pgfqpoint{1.867048in}{2.533110in}}%
\pgfpathlineto{\pgfqpoint{1.869372in}{2.458922in}}%
\pgfpathlineto{\pgfqpoint{1.871696in}{2.472330in}}%
\pgfpathlineto{\pgfqpoint{1.874020in}{2.464274in}}%
\pgfpathlineto{\pgfqpoint{1.876344in}{2.511305in}}%
\pgfpathlineto{\pgfqpoint{1.878668in}{2.523358in}}%
\pgfpathlineto{\pgfqpoint{1.880992in}{2.512690in}}%
\pgfpathlineto{\pgfqpoint{1.883316in}{2.525207in}}%
\pgfpathlineto{\pgfqpoint{1.885640in}{2.481341in}}%
\pgfpathlineto{\pgfqpoint{1.887964in}{2.526368in}}%
\pgfpathlineto{\pgfqpoint{1.890288in}{2.521077in}}%
\pgfpathlineto{\pgfqpoint{1.892612in}{2.491219in}}%
\pgfpathlineto{\pgfqpoint{1.894936in}{2.513586in}}%
\pgfpathlineto{\pgfqpoint{1.897260in}{2.471804in}}%
\pgfpathlineto{\pgfqpoint{1.901908in}{2.540999in}}%
\pgfpathlineto{\pgfqpoint{1.904232in}{2.525828in}}%
\pgfpathlineto{\pgfqpoint{1.906556in}{2.522337in}}%
\pgfpathlineto{\pgfqpoint{1.908880in}{2.458028in}}%
\pgfpathlineto{\pgfqpoint{1.911204in}{2.541730in}}%
\pgfpathlineto{\pgfqpoint{1.913528in}{2.437581in}}%
\pgfpathlineto{\pgfqpoint{1.915852in}{2.504808in}}%
\pgfpathlineto{\pgfqpoint{1.918176in}{2.478931in}}%
\pgfpathlineto{\pgfqpoint{1.920500in}{2.550448in}}%
\pgfpathlineto{\pgfqpoint{1.922824in}{2.516231in}}%
\pgfpathlineto{\pgfqpoint{1.925148in}{2.514732in}}%
\pgfpathlineto{\pgfqpoint{1.927472in}{2.522368in}}%
\pgfpathlineto{\pgfqpoint{1.929795in}{2.482025in}}%
\pgfpathlineto{\pgfqpoint{1.934443in}{2.535873in}}%
\pgfpathlineto{\pgfqpoint{1.936767in}{2.491564in}}%
\pgfpathlineto{\pgfqpoint{1.939091in}{2.536783in}}%
\pgfpathlineto{\pgfqpoint{1.941415in}{2.508749in}}%
\pgfpathlineto{\pgfqpoint{1.943739in}{2.530420in}}%
\pgfpathlineto{\pgfqpoint{1.946063in}{2.529902in}}%
\pgfpathlineto{\pgfqpoint{1.948387in}{2.461187in}}%
\pgfpathlineto{\pgfqpoint{1.950711in}{2.506615in}}%
\pgfpathlineto{\pgfqpoint{1.953035in}{2.490185in}}%
\pgfpathlineto{\pgfqpoint{1.955359in}{2.530527in}}%
\pgfpathlineto{\pgfqpoint{1.957683in}{2.505125in}}%
\pgfpathlineto{\pgfqpoint{1.960007in}{2.513977in}}%
\pgfpathlineto{\pgfqpoint{1.962331in}{2.496441in}}%
\pgfpathlineto{\pgfqpoint{1.966979in}{2.505406in}}%
\pgfpathlineto{\pgfqpoint{1.969303in}{2.558415in}}%
\pgfpathlineto{\pgfqpoint{1.971627in}{2.534298in}}%
\pgfpathlineto{\pgfqpoint{1.976275in}{2.512137in}}%
\pgfpathlineto{\pgfqpoint{1.978599in}{2.537217in}}%
\pgfpathlineto{\pgfqpoint{1.980923in}{2.522365in}}%
\pgfpathlineto{\pgfqpoint{1.983247in}{2.537908in}}%
\pgfpathlineto{\pgfqpoint{1.985571in}{2.543976in}}%
\pgfpathlineto{\pgfqpoint{1.987895in}{2.582033in}}%
\pgfpathlineto{\pgfqpoint{1.990219in}{2.588271in}}%
\pgfpathlineto{\pgfqpoint{1.992543in}{2.492578in}}%
\pgfpathlineto{\pgfqpoint{1.994867in}{2.554533in}}%
\pgfpathlineto{\pgfqpoint{1.997191in}{2.524287in}}%
\pgfpathlineto{\pgfqpoint{1.999514in}{2.525623in}}%
\pgfpathlineto{\pgfqpoint{2.001838in}{2.572850in}}%
\pgfpathlineto{\pgfqpoint{2.004162in}{2.572357in}}%
\pgfpathlineto{\pgfqpoint{2.006486in}{2.519761in}}%
\pgfpathlineto{\pgfqpoint{2.008810in}{2.534163in}}%
\pgfpathlineto{\pgfqpoint{2.011134in}{2.568693in}}%
\pgfpathlineto{\pgfqpoint{2.013458in}{2.583381in}}%
\pgfpathlineto{\pgfqpoint{2.015782in}{2.529294in}}%
\pgfpathlineto{\pgfqpoint{2.018106in}{2.554652in}}%
\pgfpathlineto{\pgfqpoint{2.020430in}{2.564755in}}%
\pgfpathlineto{\pgfqpoint{2.022754in}{2.504820in}}%
\pgfpathlineto{\pgfqpoint{2.027402in}{2.558374in}}%
\pgfpathlineto{\pgfqpoint{2.029726in}{2.545874in}}%
\pgfpathlineto{\pgfqpoint{2.032050in}{2.600601in}}%
\pgfpathlineto{\pgfqpoint{2.034374in}{2.511941in}}%
\pgfpathlineto{\pgfqpoint{2.039022in}{2.497613in}}%
\pgfpathlineto{\pgfqpoint{2.041346in}{2.506815in}}%
\pgfpathlineto{\pgfqpoint{2.043670in}{2.561592in}}%
\pgfpathlineto{\pgfqpoint{2.045994in}{2.579824in}}%
\pgfpathlineto{\pgfqpoint{2.050642in}{2.504623in}}%
\pgfpathlineto{\pgfqpoint{2.052966in}{2.579995in}}%
\pgfpathlineto{\pgfqpoint{2.057614in}{2.552161in}}%
\pgfpathlineto{\pgfqpoint{2.059938in}{2.586363in}}%
\pgfpathlineto{\pgfqpoint{2.062262in}{2.568485in}}%
\pgfpathlineto{\pgfqpoint{2.064586in}{2.578578in}}%
\pgfpathlineto{\pgfqpoint{2.066910in}{2.576747in}}%
\pgfpathlineto{\pgfqpoint{2.069234in}{2.516557in}}%
\pgfpathlineto{\pgfqpoint{2.071557in}{2.551565in}}%
\pgfpathlineto{\pgfqpoint{2.073881in}{2.552942in}}%
\pgfpathlineto{\pgfqpoint{2.076205in}{2.552820in}}%
\pgfpathlineto{\pgfqpoint{2.078529in}{2.577904in}}%
\pgfpathlineto{\pgfqpoint{2.080853in}{2.574618in}}%
\pgfpathlineto{\pgfqpoint{2.083177in}{2.557085in}}%
\pgfpathlineto{\pgfqpoint{2.085501in}{2.573178in}}%
\pgfpathlineto{\pgfqpoint{2.087825in}{2.568370in}}%
\pgfpathlineto{\pgfqpoint{2.090149in}{2.573896in}}%
\pgfpathlineto{\pgfqpoint{2.092473in}{2.611290in}}%
\pgfpathlineto{\pgfqpoint{2.097121in}{2.559410in}}%
\pgfpathlineto{\pgfqpoint{2.099445in}{2.570246in}}%
\pgfpathlineto{\pgfqpoint{2.101769in}{2.547823in}}%
\pgfpathlineto{\pgfqpoint{2.104093in}{2.488503in}}%
\pgfpathlineto{\pgfqpoint{2.106417in}{2.470331in}}%
\pgfpathlineto{\pgfqpoint{2.108741in}{2.508869in}}%
\pgfpathlineto{\pgfqpoint{2.113389in}{2.458853in}}%
\pgfpathlineto{\pgfqpoint{2.115713in}{2.493979in}}%
\pgfpathlineto{\pgfqpoint{2.118037in}{2.510914in}}%
\pgfpathlineto{\pgfqpoint{2.120361in}{2.512308in}}%
\pgfpathlineto{\pgfqpoint{2.122685in}{2.536566in}}%
\pgfpathlineto{\pgfqpoint{2.125009in}{2.480053in}}%
\pgfpathlineto{\pgfqpoint{2.127333in}{2.475304in}}%
\pgfpathlineto{\pgfqpoint{2.129657in}{2.476586in}}%
\pgfpathlineto{\pgfqpoint{2.134305in}{2.512691in}}%
\pgfpathlineto{\pgfqpoint{2.136629in}{2.467614in}}%
\pgfpathlineto{\pgfqpoint{2.138953in}{2.507400in}}%
\pgfpathlineto{\pgfqpoint{2.141276in}{2.524351in}}%
\pgfpathlineto{\pgfqpoint{2.143600in}{2.525378in}}%
\pgfpathlineto{\pgfqpoint{2.145924in}{2.536389in}}%
\pgfpathlineto{\pgfqpoint{2.148248in}{2.498031in}}%
\pgfpathlineto{\pgfqpoint{2.150572in}{2.494635in}}%
\pgfpathlineto{\pgfqpoint{2.152896in}{2.517572in}}%
\pgfpathlineto{\pgfqpoint{2.155220in}{2.522673in}}%
\pgfpathlineto{\pgfqpoint{2.157544in}{2.481133in}}%
\pgfpathlineto{\pgfqpoint{2.159868in}{2.493547in}}%
\pgfpathlineto{\pgfqpoint{2.162192in}{2.539574in}}%
\pgfpathlineto{\pgfqpoint{2.164516in}{2.504265in}}%
\pgfpathlineto{\pgfqpoint{2.166840in}{2.495676in}}%
\pgfpathlineto{\pgfqpoint{2.169164in}{2.508069in}}%
\pgfpathlineto{\pgfqpoint{2.171488in}{2.573041in}}%
\pgfpathlineto{\pgfqpoint{2.173812in}{2.507670in}}%
\pgfpathlineto{\pgfqpoint{2.176136in}{2.565328in}}%
\pgfpathlineto{\pgfqpoint{2.178460in}{2.516046in}}%
\pgfpathlineto{\pgfqpoint{2.180784in}{2.521195in}}%
\pgfpathlineto{\pgfqpoint{2.183108in}{2.470568in}}%
\pgfpathlineto{\pgfqpoint{2.185432in}{2.553236in}}%
\pgfpathlineto{\pgfqpoint{2.187756in}{2.504951in}}%
\pgfpathlineto{\pgfqpoint{2.190080in}{2.499728in}}%
\pgfpathlineto{\pgfqpoint{2.192404in}{2.511594in}}%
\pgfpathlineto{\pgfqpoint{2.194728in}{2.575344in}}%
\pgfpathlineto{\pgfqpoint{2.197052in}{2.486769in}}%
\pgfpathlineto{\pgfqpoint{2.199376in}{2.540826in}}%
\pgfpathlineto{\pgfqpoint{2.201700in}{2.526176in}}%
\pgfpathlineto{\pgfqpoint{2.204024in}{2.549642in}}%
\pgfpathlineto{\pgfqpoint{2.210995in}{2.514570in}}%
\pgfpathlineto{\pgfqpoint{2.215643in}{2.560551in}}%
\pgfpathlineto{\pgfqpoint{2.217967in}{2.567418in}}%
\pgfpathlineto{\pgfqpoint{2.220291in}{2.524277in}}%
\pgfpathlineto{\pgfqpoint{2.222615in}{2.521915in}}%
\pgfpathlineto{\pgfqpoint{2.224939in}{2.561830in}}%
\pgfpathlineto{\pgfqpoint{2.227263in}{2.561249in}}%
\pgfpathlineto{\pgfqpoint{2.229587in}{2.496610in}}%
\pgfpathlineto{\pgfqpoint{2.231911in}{2.544570in}}%
\pgfpathlineto{\pgfqpoint{2.234235in}{2.534168in}}%
\pgfpathlineto{\pgfqpoint{2.236559in}{2.538172in}}%
\pgfpathlineto{\pgfqpoint{2.238883in}{2.562048in}}%
\pgfpathlineto{\pgfqpoint{2.241207in}{2.608794in}}%
\pgfpathlineto{\pgfqpoint{2.243531in}{2.568926in}}%
\pgfpathlineto{\pgfqpoint{2.245855in}{2.600819in}}%
\pgfpathlineto{\pgfqpoint{2.250503in}{2.534112in}}%
\pgfpathlineto{\pgfqpoint{2.252827in}{2.597385in}}%
\pgfpathlineto{\pgfqpoint{2.255151in}{2.560150in}}%
\pgfpathlineto{\pgfqpoint{2.257475in}{2.553047in}}%
\pgfpathlineto{\pgfqpoint{2.259799in}{2.566013in}}%
\pgfpathlineto{\pgfqpoint{2.262123in}{2.596502in}}%
\pgfpathlineto{\pgfqpoint{2.264447in}{2.539306in}}%
\pgfpathlineto{\pgfqpoint{2.266771in}{2.555119in}}%
\pgfpathlineto{\pgfqpoint{2.269095in}{2.539952in}}%
\pgfpathlineto{\pgfqpoint{2.271419in}{2.554496in}}%
\pgfpathlineto{\pgfqpoint{2.273743in}{2.543617in}}%
\pgfpathlineto{\pgfqpoint{2.278391in}{2.587617in}}%
\pgfpathlineto{\pgfqpoint{2.280715in}{2.561155in}}%
\pgfpathlineto{\pgfqpoint{2.283038in}{2.672220in}}%
\pgfpathlineto{\pgfqpoint{2.287686in}{2.585432in}}%
\pgfpathlineto{\pgfqpoint{2.292334in}{2.551666in}}%
\pgfpathlineto{\pgfqpoint{2.294658in}{2.556828in}}%
\pgfpathlineto{\pgfqpoint{2.296982in}{2.610561in}}%
\pgfpathlineto{\pgfqpoint{2.299306in}{2.587488in}}%
\pgfpathlineto{\pgfqpoint{2.301630in}{2.588153in}}%
\pgfpathlineto{\pgfqpoint{2.303954in}{2.597465in}}%
\pgfpathlineto{\pgfqpoint{2.306278in}{2.547449in}}%
\pgfpathlineto{\pgfqpoint{2.308602in}{2.640508in}}%
\pgfpathlineto{\pgfqpoint{2.310926in}{2.617184in}}%
\pgfpathlineto{\pgfqpoint{2.313250in}{2.580035in}}%
\pgfpathlineto{\pgfqpoint{2.315574in}{2.626127in}}%
\pgfpathlineto{\pgfqpoint{2.317898in}{2.611125in}}%
\pgfpathlineto{\pgfqpoint{2.320222in}{2.636190in}}%
\pgfpathlineto{\pgfqpoint{2.322546in}{2.576165in}}%
\pgfpathlineto{\pgfqpoint{2.324870in}{2.581070in}}%
\pgfpathlineto{\pgfqpoint{2.327194in}{2.568035in}}%
\pgfpathlineto{\pgfqpoint{2.329518in}{2.587236in}}%
\pgfpathlineto{\pgfqpoint{2.331842in}{2.590391in}}%
\pgfpathlineto{\pgfqpoint{2.336490in}{2.642649in}}%
\pgfpathlineto{\pgfqpoint{2.341138in}{2.592534in}}%
\pgfpathlineto{\pgfqpoint{2.343462in}{2.596555in}}%
\pgfpathlineto{\pgfqpoint{2.345786in}{2.608465in}}%
\pgfpathlineto{\pgfqpoint{2.348110in}{2.648751in}}%
\pgfpathlineto{\pgfqpoint{2.350434in}{2.583822in}}%
\pgfpathlineto{\pgfqpoint{2.352757in}{2.637184in}}%
\pgfpathlineto{\pgfqpoint{2.355081in}{2.591060in}}%
\pgfpathlineto{\pgfqpoint{2.357405in}{2.664283in}}%
\pgfpathlineto{\pgfqpoint{2.359729in}{2.607037in}}%
\pgfpathlineto{\pgfqpoint{2.362053in}{2.608290in}}%
\pgfpathlineto{\pgfqpoint{2.364377in}{2.637604in}}%
\pgfpathlineto{\pgfqpoint{2.366701in}{2.636331in}}%
\pgfpathlineto{\pgfqpoint{2.369025in}{2.629669in}}%
\pgfpathlineto{\pgfqpoint{2.371349in}{2.616671in}}%
\pgfpathlineto{\pgfqpoint{2.373673in}{2.681367in}}%
\pgfpathlineto{\pgfqpoint{2.375997in}{2.651238in}}%
\pgfpathlineto{\pgfqpoint{2.378321in}{2.661678in}}%
\pgfpathlineto{\pgfqpoint{2.382969in}{2.621014in}}%
\pgfpathlineto{\pgfqpoint{2.385293in}{2.651119in}}%
\pgfpathlineto{\pgfqpoint{2.387617in}{2.647439in}}%
\pgfpathlineto{\pgfqpoint{2.389941in}{2.659226in}}%
\pgfpathlineto{\pgfqpoint{2.392265in}{2.622050in}}%
\pgfpathlineto{\pgfqpoint{2.394589in}{2.337139in}}%
\pgfpathlineto{\pgfqpoint{2.396913in}{2.383898in}}%
\pgfpathlineto{\pgfqpoint{2.399237in}{2.385424in}}%
\pgfpathlineto{\pgfqpoint{2.401561in}{2.327230in}}%
\pgfpathlineto{\pgfqpoint{2.403885in}{2.381155in}}%
\pgfpathlineto{\pgfqpoint{2.406209in}{2.372731in}}%
\pgfpathlineto{\pgfqpoint{2.408533in}{2.369932in}}%
\pgfpathlineto{\pgfqpoint{2.410857in}{2.357940in}}%
\pgfpathlineto{\pgfqpoint{2.413181in}{2.376516in}}%
\pgfpathlineto{\pgfqpoint{2.415505in}{2.344807in}}%
\pgfpathlineto{\pgfqpoint{2.417829in}{2.375303in}}%
\pgfpathlineto{\pgfqpoint{2.420153in}{2.382181in}}%
\pgfpathlineto{\pgfqpoint{2.422476in}{2.347484in}}%
\pgfpathlineto{\pgfqpoint{2.424800in}{2.418845in}}%
\pgfpathlineto{\pgfqpoint{2.427124in}{2.368609in}}%
\pgfpathlineto{\pgfqpoint{2.429448in}{2.363114in}}%
\pgfpathlineto{\pgfqpoint{2.431772in}{2.393470in}}%
\pgfpathlineto{\pgfqpoint{2.434096in}{2.438433in}}%
\pgfpathlineto{\pgfqpoint{2.436420in}{2.393592in}}%
\pgfpathlineto{\pgfqpoint{2.438744in}{2.406522in}}%
\pgfpathlineto{\pgfqpoint{2.441068in}{2.384474in}}%
\pgfpathlineto{\pgfqpoint{2.445716in}{2.413273in}}%
\pgfpathlineto{\pgfqpoint{2.448040in}{2.410171in}}%
\pgfpathlineto{\pgfqpoint{2.450364in}{2.390305in}}%
\pgfpathlineto{\pgfqpoint{2.452688in}{2.345233in}}%
\pgfpathlineto{\pgfqpoint{2.455012in}{2.384748in}}%
\pgfpathlineto{\pgfqpoint{2.457336in}{2.399967in}}%
\pgfpathlineto{\pgfqpoint{2.459660in}{2.448968in}}%
\pgfpathlineto{\pgfqpoint{2.461984in}{2.468909in}}%
\pgfpathlineto{\pgfqpoint{2.464308in}{2.410388in}}%
\pgfpathlineto{\pgfqpoint{2.466632in}{2.433225in}}%
\pgfpathlineto{\pgfqpoint{2.468956in}{2.413207in}}%
\pgfpathlineto{\pgfqpoint{2.471280in}{2.425853in}}%
\pgfpathlineto{\pgfqpoint{2.473604in}{2.392326in}}%
\pgfpathlineto{\pgfqpoint{2.475928in}{2.410771in}}%
\pgfpathlineto{\pgfqpoint{2.478252in}{2.379565in}}%
\pgfpathlineto{\pgfqpoint{2.482900in}{2.444287in}}%
\pgfpathlineto{\pgfqpoint{2.485224in}{2.442743in}}%
\pgfpathlineto{\pgfqpoint{2.487548in}{2.464303in}}%
\pgfpathlineto{\pgfqpoint{2.489872in}{2.372181in}}%
\pgfpathlineto{\pgfqpoint{2.492195in}{2.403430in}}%
\pgfpathlineto{\pgfqpoint{2.494519in}{2.412564in}}%
\pgfpathlineto{\pgfqpoint{2.496843in}{2.464106in}}%
\pgfpathlineto{\pgfqpoint{2.499167in}{2.438191in}}%
\pgfpathlineto{\pgfqpoint{2.501491in}{2.449955in}}%
\pgfpathlineto{\pgfqpoint{2.503815in}{2.416649in}}%
\pgfpathlineto{\pgfqpoint{2.506139in}{2.429833in}}%
\pgfpathlineto{\pgfqpoint{2.508463in}{2.472642in}}%
\pgfpathlineto{\pgfqpoint{2.510787in}{2.457492in}}%
\pgfpathlineto{\pgfqpoint{2.513111in}{2.466097in}}%
\pgfpathlineto{\pgfqpoint{2.515435in}{2.482152in}}%
\pgfpathlineto{\pgfqpoint{2.517759in}{2.486891in}}%
\pgfpathlineto{\pgfqpoint{2.520083in}{2.515167in}}%
\pgfpathlineto{\pgfqpoint{2.522407in}{2.442621in}}%
\pgfpathlineto{\pgfqpoint{2.524731in}{2.472085in}}%
\pgfpathlineto{\pgfqpoint{2.527055in}{2.463497in}}%
\pgfpathlineto{\pgfqpoint{2.529379in}{2.449609in}}%
\pgfpathlineto{\pgfqpoint{2.531703in}{2.482731in}}%
\pgfpathlineto{\pgfqpoint{2.534027in}{2.449158in}}%
\pgfpathlineto{\pgfqpoint{2.536351in}{2.537800in}}%
\pgfpathlineto{\pgfqpoint{2.540999in}{2.442327in}}%
\pgfpathlineto{\pgfqpoint{2.543323in}{2.515946in}}%
\pgfpathlineto{\pgfqpoint{2.545647in}{2.513662in}}%
\pgfpathlineto{\pgfqpoint{2.547971in}{2.507514in}}%
\pgfpathlineto{\pgfqpoint{2.550295in}{2.527403in}}%
\pgfpathlineto{\pgfqpoint{2.552619in}{2.511284in}}%
\pgfpathlineto{\pgfqpoint{2.554943in}{2.469682in}}%
\pgfpathlineto{\pgfqpoint{2.557267in}{2.487723in}}%
\pgfpathlineto{\pgfqpoint{2.559591in}{2.515703in}}%
\pgfpathlineto{\pgfqpoint{2.561915in}{2.499250in}}%
\pgfpathlineto{\pgfqpoint{2.564238in}{2.493754in}}%
\pgfpathlineto{\pgfqpoint{2.566562in}{2.532401in}}%
\pgfpathlineto{\pgfqpoint{2.568886in}{2.524383in}}%
\pgfpathlineto{\pgfqpoint{2.571210in}{2.559898in}}%
\pgfpathlineto{\pgfqpoint{2.573534in}{2.517826in}}%
\pgfpathlineto{\pgfqpoint{2.575858in}{2.566018in}}%
\pgfpathlineto{\pgfqpoint{2.578182in}{2.444331in}}%
\pgfpathlineto{\pgfqpoint{2.580506in}{2.528617in}}%
\pgfpathlineto{\pgfqpoint{2.582830in}{2.547088in}}%
\pgfpathlineto{\pgfqpoint{2.585154in}{2.505517in}}%
\pgfpathlineto{\pgfqpoint{2.592126in}{2.547377in}}%
\pgfpathlineto{\pgfqpoint{2.594450in}{2.512418in}}%
\pgfpathlineto{\pgfqpoint{2.596774in}{2.556542in}}%
\pgfpathlineto{\pgfqpoint{2.599098in}{2.538696in}}%
\pgfpathlineto{\pgfqpoint{2.601422in}{2.490941in}}%
\pgfpathlineto{\pgfqpoint{2.603746in}{2.541340in}}%
\pgfpathlineto{\pgfqpoint{2.606070in}{2.508343in}}%
\pgfpathlineto{\pgfqpoint{2.608394in}{2.536249in}}%
\pgfpathlineto{\pgfqpoint{2.610718in}{2.507010in}}%
\pgfpathlineto{\pgfqpoint{2.613042in}{2.508716in}}%
\pgfpathlineto{\pgfqpoint{2.615366in}{2.523092in}}%
\pgfpathlineto{\pgfqpoint{2.617690in}{2.560129in}}%
\pgfpathlineto{\pgfqpoint{2.620014in}{2.536632in}}%
\pgfpathlineto{\pgfqpoint{2.622338in}{2.615332in}}%
\pgfpathlineto{\pgfqpoint{2.624662in}{2.566105in}}%
\pgfpathlineto{\pgfqpoint{2.626986in}{2.587273in}}%
\pgfpathlineto{\pgfqpoint{2.629310in}{2.567066in}}%
\pgfpathlineto{\pgfqpoint{2.631634in}{2.526192in}}%
\pgfpathlineto{\pgfqpoint{2.633957in}{2.586490in}}%
\pgfpathlineto{\pgfqpoint{2.636281in}{2.551082in}}%
\pgfpathlineto{\pgfqpoint{2.638605in}{2.616152in}}%
\pgfpathlineto{\pgfqpoint{2.640929in}{2.571666in}}%
\pgfpathlineto{\pgfqpoint{2.643253in}{2.597093in}}%
\pgfpathlineto{\pgfqpoint{2.645577in}{2.570984in}}%
\pgfpathlineto{\pgfqpoint{2.647901in}{2.614454in}}%
\pgfpathlineto{\pgfqpoint{2.652549in}{2.589354in}}%
\pgfpathlineto{\pgfqpoint{2.654873in}{2.597962in}}%
\pgfpathlineto{\pgfqpoint{2.657197in}{2.574861in}}%
\pgfpathlineto{\pgfqpoint{2.659521in}{2.589610in}}%
\pgfpathlineto{\pgfqpoint{2.661845in}{2.614222in}}%
\pgfpathlineto{\pgfqpoint{2.664169in}{2.600797in}}%
\pgfpathlineto{\pgfqpoint{2.666493in}{2.600804in}}%
\pgfpathlineto{\pgfqpoint{2.668817in}{2.613216in}}%
\pgfpathlineto{\pgfqpoint{2.671141in}{2.601631in}}%
\pgfpathlineto{\pgfqpoint{2.673465in}{2.635861in}}%
\pgfpathlineto{\pgfqpoint{2.675789in}{2.596002in}}%
\pgfpathlineto{\pgfqpoint{2.678113in}{2.674996in}}%
\pgfpathlineto{\pgfqpoint{2.680437in}{2.580222in}}%
\pgfpathlineto{\pgfqpoint{2.682761in}{2.624599in}}%
\pgfpathlineto{\pgfqpoint{2.687409in}{2.487216in}}%
\pgfpathlineto{\pgfqpoint{2.689733in}{2.538477in}}%
\pgfpathlineto{\pgfqpoint{2.692057in}{2.561570in}}%
\pgfpathlineto{\pgfqpoint{2.694381in}{2.516832in}}%
\pgfpathlineto{\pgfqpoint{2.701353in}{2.580808in}}%
\pgfpathlineto{\pgfqpoint{2.703676in}{2.586934in}}%
\pgfpathlineto{\pgfqpoint{2.706000in}{2.586065in}}%
\pgfpathlineto{\pgfqpoint{2.708324in}{2.538011in}}%
\pgfpathlineto{\pgfqpoint{2.715296in}{2.577513in}}%
\pgfpathlineto{\pgfqpoint{2.717620in}{2.614207in}}%
\pgfpathlineto{\pgfqpoint{2.719944in}{2.595363in}}%
\pgfpathlineto{\pgfqpoint{2.722268in}{2.546086in}}%
\pgfpathlineto{\pgfqpoint{2.724592in}{2.603176in}}%
\pgfpathlineto{\pgfqpoint{2.726916in}{2.584587in}}%
\pgfpathlineto{\pgfqpoint{2.729240in}{2.611954in}}%
\pgfpathlineto{\pgfqpoint{2.731564in}{2.573336in}}%
\pgfpathlineto{\pgfqpoint{2.733888in}{2.623658in}}%
\pgfpathlineto{\pgfqpoint{2.736212in}{2.589529in}}%
\pgfpathlineto{\pgfqpoint{2.738536in}{2.598078in}}%
\pgfpathlineto{\pgfqpoint{2.740860in}{2.598310in}}%
\pgfpathlineto{\pgfqpoint{2.743184in}{2.632535in}}%
\pgfpathlineto{\pgfqpoint{2.750156in}{2.586588in}}%
\pgfpathlineto{\pgfqpoint{2.752480in}{2.581289in}}%
\pgfpathlineto{\pgfqpoint{2.754804in}{2.599700in}}%
\pgfpathlineto{\pgfqpoint{2.757128in}{2.634704in}}%
\pgfpathlineto{\pgfqpoint{2.759452in}{2.691651in}}%
\pgfpathlineto{\pgfqpoint{2.761776in}{2.613302in}}%
\pgfpathlineto{\pgfqpoint{2.764100in}{2.588169in}}%
\pgfpathlineto{\pgfqpoint{2.766424in}{2.640863in}}%
\pgfpathlineto{\pgfqpoint{2.768748in}{2.562711in}}%
\pgfpathlineto{\pgfqpoint{2.771072in}{2.603405in}}%
\pgfpathlineto{\pgfqpoint{2.773395in}{2.603217in}}%
\pgfpathlineto{\pgfqpoint{2.775719in}{2.652654in}}%
\pgfpathlineto{\pgfqpoint{2.780367in}{2.630257in}}%
\pgfpathlineto{\pgfqpoint{2.782691in}{2.667241in}}%
\pgfpathlineto{\pgfqpoint{2.785015in}{2.677671in}}%
\pgfpathlineto{\pgfqpoint{2.787339in}{2.677183in}}%
\pgfpathlineto{\pgfqpoint{2.789663in}{2.607763in}}%
\pgfpathlineto{\pgfqpoint{2.791987in}{2.669625in}}%
\pgfpathlineto{\pgfqpoint{2.794311in}{2.672685in}}%
\pgfpathlineto{\pgfqpoint{2.796635in}{2.659484in}}%
\pgfpathlineto{\pgfqpoint{2.798959in}{2.657239in}}%
\pgfpathlineto{\pgfqpoint{2.801283in}{2.661495in}}%
\pgfpathlineto{\pgfqpoint{2.803607in}{2.694899in}}%
\pgfpathlineto{\pgfqpoint{2.805931in}{2.692191in}}%
\pgfpathlineto{\pgfqpoint{2.808255in}{2.669520in}}%
\pgfpathlineto{\pgfqpoint{2.810579in}{2.669219in}}%
\pgfpathlineto{\pgfqpoint{2.812903in}{2.760110in}}%
\pgfpathlineto{\pgfqpoint{2.815227in}{2.681000in}}%
\pgfpathlineto{\pgfqpoint{2.817551in}{2.651911in}}%
\pgfpathlineto{\pgfqpoint{2.819875in}{2.669443in}}%
\pgfpathlineto{\pgfqpoint{2.822199in}{2.705596in}}%
\pgfpathlineto{\pgfqpoint{2.824523in}{2.659255in}}%
\pgfpathlineto{\pgfqpoint{2.826847in}{2.657572in}}%
\pgfpathlineto{\pgfqpoint{2.831495in}{2.705631in}}%
\pgfpathlineto{\pgfqpoint{2.833819in}{2.710978in}}%
\pgfpathlineto{\pgfqpoint{2.836143in}{2.710065in}}%
\pgfpathlineto{\pgfqpoint{2.838467in}{2.760809in}}%
\pgfpathlineto{\pgfqpoint{2.840791in}{2.723988in}}%
\pgfpathlineto{\pgfqpoint{2.845438in}{2.715941in}}%
\pgfpathlineto{\pgfqpoint{2.847762in}{2.720167in}}%
\pgfpathlineto{\pgfqpoint{2.850086in}{2.697400in}}%
\pgfpathlineto{\pgfqpoint{2.852410in}{2.716378in}}%
\pgfpathlineto{\pgfqpoint{2.854734in}{2.701727in}}%
\pgfpathlineto{\pgfqpoint{2.857058in}{2.659898in}}%
\pgfpathlineto{\pgfqpoint{2.861706in}{2.733667in}}%
\pgfpathlineto{\pgfqpoint{2.864030in}{2.743229in}}%
\pgfpathlineto{\pgfqpoint{2.866354in}{2.719359in}}%
\pgfpathlineto{\pgfqpoint{2.868678in}{2.771856in}}%
\pgfpathlineto{\pgfqpoint{2.871002in}{2.781634in}}%
\pgfpathlineto{\pgfqpoint{2.873326in}{2.770229in}}%
\pgfpathlineto{\pgfqpoint{2.875650in}{2.786683in}}%
\pgfpathlineto{\pgfqpoint{2.877974in}{2.714776in}}%
\pgfpathlineto{\pgfqpoint{2.880298in}{2.755769in}}%
\pgfpathlineto{\pgfqpoint{2.882622in}{2.730212in}}%
\pgfpathlineto{\pgfqpoint{2.884946in}{2.736607in}}%
\pgfpathlineto{\pgfqpoint{2.887270in}{2.755929in}}%
\pgfpathlineto{\pgfqpoint{2.889594in}{2.720833in}}%
\pgfpathlineto{\pgfqpoint{2.891918in}{2.742740in}}%
\pgfpathlineto{\pgfqpoint{2.894242in}{2.798177in}}%
\pgfpathlineto{\pgfqpoint{2.896566in}{2.765704in}}%
\pgfpathlineto{\pgfqpoint{2.898890in}{2.778402in}}%
\pgfpathlineto{\pgfqpoint{2.901214in}{2.779304in}}%
\pgfpathlineto{\pgfqpoint{2.903538in}{2.728660in}}%
\pgfpathlineto{\pgfqpoint{2.905862in}{2.807255in}}%
\pgfpathlineto{\pgfqpoint{2.908186in}{2.773799in}}%
\pgfpathlineto{\pgfqpoint{2.910510in}{2.817909in}}%
\pgfpathlineto{\pgfqpoint{2.912834in}{2.820848in}}%
\pgfpathlineto{\pgfqpoint{2.917481in}{2.744880in}}%
\pgfpathlineto{\pgfqpoint{2.919805in}{2.820191in}}%
\pgfpathlineto{\pgfqpoint{2.922129in}{2.789624in}}%
\pgfpathlineto{\pgfqpoint{2.924453in}{2.811799in}}%
\pgfpathlineto{\pgfqpoint{2.926777in}{2.782007in}}%
\pgfpathlineto{\pgfqpoint{2.929101in}{2.815550in}}%
\pgfpathlineto{\pgfqpoint{2.931425in}{2.817219in}}%
\pgfpathlineto{\pgfqpoint{2.933749in}{2.799435in}}%
\pgfpathlineto{\pgfqpoint{2.936073in}{2.807970in}}%
\pgfpathlineto{\pgfqpoint{2.938397in}{2.792107in}}%
\pgfpathlineto{\pgfqpoint{2.940721in}{2.764058in}}%
\pgfpathlineto{\pgfqpoint{2.943045in}{2.830980in}}%
\pgfpathlineto{\pgfqpoint{2.945369in}{2.820251in}}%
\pgfpathlineto{\pgfqpoint{2.947693in}{2.832183in}}%
\pgfpathlineto{\pgfqpoint{2.950017in}{2.793252in}}%
\pgfpathlineto{\pgfqpoint{2.952341in}{2.827507in}}%
\pgfpathlineto{\pgfqpoint{2.954665in}{2.803010in}}%
\pgfpathlineto{\pgfqpoint{2.959313in}{2.799279in}}%
\pgfpathlineto{\pgfqpoint{2.961637in}{2.807680in}}%
\pgfpathlineto{\pgfqpoint{2.963961in}{2.829288in}}%
\pgfpathlineto{\pgfqpoint{2.966285in}{2.869690in}}%
\pgfpathlineto{\pgfqpoint{2.968609in}{2.798203in}}%
\pgfpathlineto{\pgfqpoint{2.970933in}{2.893727in}}%
\pgfpathlineto{\pgfqpoint{2.973257in}{2.883709in}}%
\pgfpathlineto{\pgfqpoint{2.975581in}{2.569496in}}%
\pgfpathlineto{\pgfqpoint{2.977905in}{2.575645in}}%
\pgfpathlineto{\pgfqpoint{2.984876in}{2.529133in}}%
\pgfpathlineto{\pgfqpoint{2.987200in}{2.578264in}}%
\pgfpathlineto{\pgfqpoint{2.989524in}{2.519292in}}%
\pgfpathlineto{\pgfqpoint{2.991848in}{2.522418in}}%
\pgfpathlineto{\pgfqpoint{2.994172in}{2.553991in}}%
\pgfpathlineto{\pgfqpoint{2.996496in}{2.548367in}}%
\pgfpathlineto{\pgfqpoint{2.998820in}{2.573325in}}%
\pgfpathlineto{\pgfqpoint{3.001144in}{2.579725in}}%
\pgfpathlineto{\pgfqpoint{3.003468in}{2.610104in}}%
\pgfpathlineto{\pgfqpoint{3.005792in}{2.582101in}}%
\pgfpathlineto{\pgfqpoint{3.008116in}{2.589128in}}%
\pgfpathlineto{\pgfqpoint{3.010440in}{2.591379in}}%
\pgfpathlineto{\pgfqpoint{3.012764in}{2.591355in}}%
\pgfpathlineto{\pgfqpoint{3.015088in}{2.556898in}}%
\pgfpathlineto{\pgfqpoint{3.017412in}{2.583153in}}%
\pgfpathlineto{\pgfqpoint{3.019736in}{2.550311in}}%
\pgfpathlineto{\pgfqpoint{3.022060in}{2.589338in}}%
\pgfpathlineto{\pgfqpoint{3.024384in}{2.589788in}}%
\pgfpathlineto{\pgfqpoint{3.026708in}{2.544728in}}%
\pgfpathlineto{\pgfqpoint{3.029032in}{2.604219in}}%
\pgfpathlineto{\pgfqpoint{3.031356in}{2.631308in}}%
\pgfpathlineto{\pgfqpoint{3.033680in}{2.505129in}}%
\pgfpathlineto{\pgfqpoint{3.036004in}{2.600213in}}%
\pgfpathlineto{\pgfqpoint{3.038328in}{2.588314in}}%
\pgfpathlineto{\pgfqpoint{3.040652in}{2.605762in}}%
\pgfpathlineto{\pgfqpoint{3.042976in}{2.603064in}}%
\pgfpathlineto{\pgfqpoint{3.045300in}{2.567481in}}%
\pgfpathlineto{\pgfqpoint{3.047624in}{2.613596in}}%
\pgfpathlineto{\pgfqpoint{3.049948in}{2.612369in}}%
\pgfpathlineto{\pgfqpoint{3.052272in}{2.605013in}}%
\pgfpathlineto{\pgfqpoint{3.054596in}{2.618117in}}%
\pgfpathlineto{\pgfqpoint{3.056919in}{2.620250in}}%
\pgfpathlineto{\pgfqpoint{3.059243in}{2.635777in}}%
\pgfpathlineto{\pgfqpoint{3.061567in}{2.628977in}}%
\pgfpathlineto{\pgfqpoint{3.063891in}{2.646397in}}%
\pgfpathlineto{\pgfqpoint{3.066215in}{2.633796in}}%
\pgfpathlineto{\pgfqpoint{3.068539in}{2.629019in}}%
\pgfpathlineto{\pgfqpoint{3.070863in}{2.571067in}}%
\pgfpathlineto{\pgfqpoint{3.073187in}{2.633499in}}%
\pgfpathlineto{\pgfqpoint{3.075511in}{2.638145in}}%
\pgfpathlineto{\pgfqpoint{3.077835in}{2.673452in}}%
\pgfpathlineto{\pgfqpoint{3.080159in}{2.625991in}}%
\pgfpathlineto{\pgfqpoint{3.082483in}{2.618284in}}%
\pgfpathlineto{\pgfqpoint{3.084807in}{2.588244in}}%
\pgfpathlineto{\pgfqpoint{3.087131in}{2.665586in}}%
\pgfpathlineto{\pgfqpoint{3.089455in}{2.666262in}}%
\pgfpathlineto{\pgfqpoint{3.091779in}{2.624484in}}%
\pgfpathlineto{\pgfqpoint{3.096427in}{2.673254in}}%
\pgfpathlineto{\pgfqpoint{3.098751in}{2.643110in}}%
\pgfpathlineto{\pgfqpoint{3.101075in}{2.713591in}}%
\pgfpathlineto{\pgfqpoint{3.103399in}{2.647512in}}%
\pgfpathlineto{\pgfqpoint{3.105723in}{2.648929in}}%
\pgfpathlineto{\pgfqpoint{3.110371in}{2.626058in}}%
\pgfpathlineto{\pgfqpoint{3.112695in}{2.683754in}}%
\pgfpathlineto{\pgfqpoint{3.115019in}{2.616216in}}%
\pgfpathlineto{\pgfqpoint{3.117343in}{2.714181in}}%
\pgfpathlineto{\pgfqpoint{3.119667in}{2.688611in}}%
\pgfpathlineto{\pgfqpoint{3.121991in}{2.642444in}}%
\pgfpathlineto{\pgfqpoint{3.124315in}{2.661333in}}%
\pgfpathlineto{\pgfqpoint{3.126638in}{2.691563in}}%
\pgfpathlineto{\pgfqpoint{3.128962in}{2.661163in}}%
\pgfpathlineto{\pgfqpoint{3.131286in}{2.699956in}}%
\pgfpathlineto{\pgfqpoint{3.133610in}{2.690234in}}%
\pgfpathlineto{\pgfqpoint{3.135934in}{2.693685in}}%
\pgfpathlineto{\pgfqpoint{3.138258in}{2.689766in}}%
\pgfpathlineto{\pgfqpoint{3.140582in}{2.692259in}}%
\pgfpathlineto{\pgfqpoint{3.142906in}{2.662324in}}%
\pgfpathlineto{\pgfqpoint{3.145230in}{2.731131in}}%
\pgfpathlineto{\pgfqpoint{3.147554in}{2.690438in}}%
\pgfpathlineto{\pgfqpoint{3.149878in}{2.701499in}}%
\pgfpathlineto{\pgfqpoint{3.152202in}{2.692385in}}%
\pgfpathlineto{\pgfqpoint{3.154526in}{2.696782in}}%
\pgfpathlineto{\pgfqpoint{3.156850in}{2.727210in}}%
\pgfpathlineto{\pgfqpoint{3.159174in}{2.716314in}}%
\pgfpathlineto{\pgfqpoint{3.161498in}{2.652336in}}%
\pgfpathlineto{\pgfqpoint{3.163822in}{2.747899in}}%
\pgfpathlineto{\pgfqpoint{3.166146in}{2.691040in}}%
\pgfpathlineto{\pgfqpoint{3.168470in}{2.704817in}}%
\pgfpathlineto{\pgfqpoint{3.173118in}{2.669824in}}%
\pgfpathlineto{\pgfqpoint{3.175442in}{2.728723in}}%
\pgfpathlineto{\pgfqpoint{3.177766in}{2.729923in}}%
\pgfpathlineto{\pgfqpoint{3.182414in}{2.785331in}}%
\pgfpathlineto{\pgfqpoint{3.184738in}{2.776038in}}%
\pgfpathlineto{\pgfqpoint{3.187062in}{2.654867in}}%
\pgfpathlineto{\pgfqpoint{3.189386in}{2.711636in}}%
\pgfpathlineto{\pgfqpoint{3.191710in}{2.683156in}}%
\pgfpathlineto{\pgfqpoint{3.194034in}{2.700514in}}%
\pgfpathlineto{\pgfqpoint{3.196357in}{2.699848in}}%
\pgfpathlineto{\pgfqpoint{3.198681in}{2.737995in}}%
\pgfpathlineto{\pgfqpoint{3.201005in}{2.724035in}}%
\pgfpathlineto{\pgfqpoint{3.203329in}{2.760162in}}%
\pgfpathlineto{\pgfqpoint{3.207977in}{2.708595in}}%
\pgfpathlineto{\pgfqpoint{3.210301in}{2.752223in}}%
\pgfpathlineto{\pgfqpoint{3.212625in}{2.764106in}}%
\pgfpathlineto{\pgfqpoint{3.214949in}{2.716321in}}%
\pgfpathlineto{\pgfqpoint{3.217273in}{2.716473in}}%
\pgfpathlineto{\pgfqpoint{3.219597in}{2.739919in}}%
\pgfpathlineto{\pgfqpoint{3.221921in}{2.742666in}}%
\pgfpathlineto{\pgfqpoint{3.224245in}{2.702581in}}%
\pgfpathlineto{\pgfqpoint{3.228893in}{2.750846in}}%
\pgfpathlineto{\pgfqpoint{3.231217in}{2.743810in}}%
\pgfpathlineto{\pgfqpoint{3.233541in}{2.745816in}}%
\pgfpathlineto{\pgfqpoint{3.235865in}{2.776645in}}%
\pgfpathlineto{\pgfqpoint{3.238189in}{2.779596in}}%
\pgfpathlineto{\pgfqpoint{3.240513in}{2.744862in}}%
\pgfpathlineto{\pgfqpoint{3.242837in}{2.740773in}}%
\pgfpathlineto{\pgfqpoint{3.245161in}{2.794740in}}%
\pgfpathlineto{\pgfqpoint{3.247485in}{2.786977in}}%
\pgfpathlineto{\pgfqpoint{3.249809in}{2.767080in}}%
\pgfpathlineto{\pgfqpoint{3.252133in}{2.725868in}}%
\pgfpathlineto{\pgfqpoint{3.254457in}{2.814601in}}%
\pgfpathlineto{\pgfqpoint{3.256781in}{2.728530in}}%
\pgfpathlineto{\pgfqpoint{3.261429in}{2.806736in}}%
\pgfpathlineto{\pgfqpoint{3.266076in}{2.717411in}}%
\pgfpathlineto{\pgfqpoint{3.268400in}{2.663638in}}%
\pgfpathlineto{\pgfqpoint{3.270724in}{2.671654in}}%
\pgfpathlineto{\pgfqpoint{3.273048in}{2.697208in}}%
\pgfpathlineto{\pgfqpoint{3.275372in}{2.642942in}}%
\pgfpathlineto{\pgfqpoint{3.277696in}{2.684633in}}%
\pgfpathlineto{\pgfqpoint{3.280020in}{2.661010in}}%
\pgfpathlineto{\pgfqpoint{3.282344in}{2.684393in}}%
\pgfpathlineto{\pgfqpoint{3.284668in}{2.683603in}}%
\pgfpathlineto{\pgfqpoint{3.286992in}{2.715205in}}%
\pgfpathlineto{\pgfqpoint{3.289316in}{2.708346in}}%
\pgfpathlineto{\pgfqpoint{3.291640in}{2.720428in}}%
\pgfpathlineto{\pgfqpoint{3.293964in}{2.654006in}}%
\pgfpathlineto{\pgfqpoint{3.296288in}{2.761836in}}%
\pgfpathlineto{\pgfqpoint{3.298612in}{2.728797in}}%
\pgfpathlineto{\pgfqpoint{3.300936in}{2.735920in}}%
\pgfpathlineto{\pgfqpoint{3.303260in}{2.701109in}}%
\pgfpathlineto{\pgfqpoint{3.305584in}{2.714352in}}%
\pgfpathlineto{\pgfqpoint{3.307908in}{2.688319in}}%
\pgfpathlineto{\pgfqpoint{3.310232in}{2.705392in}}%
\pgfpathlineto{\pgfqpoint{3.312556in}{2.677388in}}%
\pgfpathlineto{\pgfqpoint{3.314880in}{2.685511in}}%
\pgfpathlineto{\pgfqpoint{3.317204in}{2.737891in}}%
\pgfpathlineto{\pgfqpoint{3.319528in}{2.691760in}}%
\pgfpathlineto{\pgfqpoint{3.321852in}{2.675070in}}%
\pgfpathlineto{\pgfqpoint{3.324176in}{2.749542in}}%
\pgfpathlineto{\pgfqpoint{3.326500in}{2.712582in}}%
\pgfpathlineto{\pgfqpoint{3.328824in}{2.764134in}}%
\pgfpathlineto{\pgfqpoint{3.331148in}{2.687964in}}%
\pgfpathlineto{\pgfqpoint{3.333472in}{2.759411in}}%
\pgfpathlineto{\pgfqpoint{3.338119in}{2.691060in}}%
\pgfpathlineto{\pgfqpoint{3.340443in}{2.746248in}}%
\pgfpathlineto{\pgfqpoint{3.342767in}{2.740951in}}%
\pgfpathlineto{\pgfqpoint{3.345091in}{2.703156in}}%
\pgfpathlineto{\pgfqpoint{3.347415in}{2.732420in}}%
\pgfpathlineto{\pgfqpoint{3.349739in}{2.727104in}}%
\pgfpathlineto{\pgfqpoint{3.352063in}{2.744658in}}%
\pgfpathlineto{\pgfqpoint{3.354387in}{2.718515in}}%
\pgfpathlineto{\pgfqpoint{3.356711in}{2.849893in}}%
\pgfpathlineto{\pgfqpoint{3.361359in}{2.688633in}}%
\pgfpathlineto{\pgfqpoint{3.363683in}{2.737071in}}%
\pgfpathlineto{\pgfqpoint{3.368331in}{2.712037in}}%
\pgfpathlineto{\pgfqpoint{3.370655in}{2.713346in}}%
\pgfpathlineto{\pgfqpoint{3.372979in}{2.755035in}}%
\pgfpathlineto{\pgfqpoint{3.377627in}{2.725989in}}%
\pgfpathlineto{\pgfqpoint{3.379951in}{2.733550in}}%
\pgfpathlineto{\pgfqpoint{3.384599in}{2.761154in}}%
\pgfpathlineto{\pgfqpoint{3.386923in}{2.773647in}}%
\pgfpathlineto{\pgfqpoint{3.389247in}{2.798543in}}%
\pgfpathlineto{\pgfqpoint{3.391571in}{2.745709in}}%
\pgfpathlineto{\pgfqpoint{3.393895in}{2.770788in}}%
\pgfpathlineto{\pgfqpoint{3.396219in}{2.747076in}}%
\pgfpathlineto{\pgfqpoint{3.398543in}{2.745928in}}%
\pgfpathlineto{\pgfqpoint{3.400867in}{2.731586in}}%
\pgfpathlineto{\pgfqpoint{3.403191in}{2.748845in}}%
\pgfpathlineto{\pgfqpoint{3.405515in}{2.795936in}}%
\pgfpathlineto{\pgfqpoint{3.407838in}{2.802390in}}%
\pgfpathlineto{\pgfqpoint{3.410162in}{2.764258in}}%
\pgfpathlineto{\pgfqpoint{3.412486in}{2.805470in}}%
\pgfpathlineto{\pgfqpoint{3.419458in}{2.733261in}}%
\pgfpathlineto{\pgfqpoint{3.421782in}{2.707825in}}%
\pgfpathlineto{\pgfqpoint{3.424106in}{2.764353in}}%
\pgfpathlineto{\pgfqpoint{3.426430in}{2.779275in}}%
\pgfpathlineto{\pgfqpoint{3.431078in}{2.737814in}}%
\pgfpathlineto{\pgfqpoint{3.433402in}{2.731885in}}%
\pgfpathlineto{\pgfqpoint{3.435726in}{2.807419in}}%
\pgfpathlineto{\pgfqpoint{3.438050in}{2.736729in}}%
\pgfpathlineto{\pgfqpoint{3.440374in}{2.819661in}}%
\pgfpathlineto{\pgfqpoint{3.442698in}{2.849226in}}%
\pgfpathlineto{\pgfqpoint{3.445022in}{2.781658in}}%
\pgfpathlineto{\pgfqpoint{3.447346in}{2.792307in}}%
\pgfpathlineto{\pgfqpoint{3.449670in}{2.742142in}}%
\pgfpathlineto{\pgfqpoint{3.451994in}{2.776839in}}%
\pgfpathlineto{\pgfqpoint{3.454318in}{2.782009in}}%
\pgfpathlineto{\pgfqpoint{3.456642in}{2.772841in}}%
\pgfpathlineto{\pgfqpoint{3.458966in}{2.831609in}}%
\pgfpathlineto{\pgfqpoint{3.461290in}{2.837248in}}%
\pgfpathlineto{\pgfqpoint{3.463614in}{2.778614in}}%
\pgfpathlineto{\pgfqpoint{3.465938in}{2.798161in}}%
\pgfpathlineto{\pgfqpoint{3.468262in}{2.788912in}}%
\pgfpathlineto{\pgfqpoint{3.470586in}{2.757110in}}%
\pgfpathlineto{\pgfqpoint{3.472910in}{2.744116in}}%
\pgfpathlineto{\pgfqpoint{3.475234in}{2.820944in}}%
\pgfpathlineto{\pgfqpoint{3.477557in}{2.766207in}}%
\pgfpathlineto{\pgfqpoint{3.479881in}{2.831930in}}%
\pgfpathlineto{\pgfqpoint{3.482205in}{2.811069in}}%
\pgfpathlineto{\pgfqpoint{3.484529in}{2.816684in}}%
\pgfpathlineto{\pgfqpoint{3.489177in}{2.761109in}}%
\pgfpathlineto{\pgfqpoint{3.491501in}{2.834007in}}%
\pgfpathlineto{\pgfqpoint{3.493825in}{2.782546in}}%
\pgfpathlineto{\pgfqpoint{3.496149in}{2.805780in}}%
\pgfpathlineto{\pgfqpoint{3.498473in}{2.737879in}}%
\pgfpathlineto{\pgfqpoint{3.500797in}{2.813221in}}%
\pgfpathlineto{\pgfqpoint{3.503121in}{2.797351in}}%
\pgfpathlineto{\pgfqpoint{3.505445in}{2.805111in}}%
\pgfpathlineto{\pgfqpoint{3.507769in}{2.786140in}}%
\pgfpathlineto{\pgfqpoint{3.510093in}{2.716468in}}%
\pgfpathlineto{\pgfqpoint{3.512417in}{2.817085in}}%
\pgfpathlineto{\pgfqpoint{3.514741in}{2.828656in}}%
\pgfpathlineto{\pgfqpoint{3.517065in}{2.817735in}}%
\pgfpathlineto{\pgfqpoint{3.519389in}{2.766196in}}%
\pgfpathlineto{\pgfqpoint{3.524037in}{2.876283in}}%
\pgfpathlineto{\pgfqpoint{3.526361in}{2.820348in}}%
\pgfpathlineto{\pgfqpoint{3.528685in}{2.813812in}}%
\pgfpathlineto{\pgfqpoint{3.531009in}{2.828211in}}%
\pgfpathlineto{\pgfqpoint{3.533333in}{2.829664in}}%
\pgfpathlineto{\pgfqpoint{3.535657in}{2.851313in}}%
\pgfpathlineto{\pgfqpoint{3.537981in}{2.848267in}}%
\pgfpathlineto{\pgfqpoint{3.540305in}{2.841510in}}%
\pgfpathlineto{\pgfqpoint{3.542629in}{2.777397in}}%
\pgfpathlineto{\pgfqpoint{3.544953in}{2.826577in}}%
\pgfpathlineto{\pgfqpoint{3.547277in}{2.797895in}}%
\pgfpathlineto{\pgfqpoint{3.549600in}{2.800455in}}%
\pgfpathlineto{\pgfqpoint{3.551924in}{2.757091in}}%
\pgfpathlineto{\pgfqpoint{3.554248in}{2.806491in}}%
\pgfpathlineto{\pgfqpoint{3.556572in}{2.512984in}}%
\pgfpathlineto{\pgfqpoint{3.558896in}{2.514712in}}%
\pgfpathlineto{\pgfqpoint{3.561220in}{2.566560in}}%
\pgfpathlineto{\pgfqpoint{3.563544in}{2.504574in}}%
\pgfpathlineto{\pgfqpoint{3.565868in}{2.475618in}}%
\pgfpathlineto{\pgfqpoint{3.568192in}{2.518091in}}%
\pgfpathlineto{\pgfqpoint{3.570516in}{2.515322in}}%
\pgfpathlineto{\pgfqpoint{3.575164in}{2.473029in}}%
\pgfpathlineto{\pgfqpoint{3.577488in}{2.540628in}}%
\pgfpathlineto{\pgfqpoint{3.579812in}{2.512042in}}%
\pgfpathlineto{\pgfqpoint{3.582136in}{2.506360in}}%
\pgfpathlineto{\pgfqpoint{3.584460in}{2.539148in}}%
\pgfpathlineto{\pgfqpoint{3.586784in}{2.538552in}}%
\pgfpathlineto{\pgfqpoint{3.589108in}{2.556792in}}%
\pgfpathlineto{\pgfqpoint{3.591432in}{2.521949in}}%
\pgfpathlineto{\pgfqpoint{3.593756in}{2.536895in}}%
\pgfpathlineto{\pgfqpoint{3.596080in}{2.503747in}}%
\pgfpathlineto{\pgfqpoint{3.600728in}{2.544808in}}%
\pgfpathlineto{\pgfqpoint{3.603052in}{2.528607in}}%
\pgfpathlineto{\pgfqpoint{3.605376in}{2.495698in}}%
\pgfpathlineto{\pgfqpoint{3.607700in}{2.538235in}}%
\pgfpathlineto{\pgfqpoint{3.610024in}{2.517498in}}%
\pgfpathlineto{\pgfqpoint{3.612348in}{2.607217in}}%
\pgfpathlineto{\pgfqpoint{3.614672in}{2.516062in}}%
\pgfpathlineto{\pgfqpoint{3.616996in}{2.480882in}}%
\pgfpathlineto{\pgfqpoint{3.619319in}{2.556141in}}%
\pgfpathlineto{\pgfqpoint{3.621643in}{2.545158in}}%
\pgfpathlineto{\pgfqpoint{3.623967in}{2.509896in}}%
\pgfpathlineto{\pgfqpoint{3.626291in}{2.591382in}}%
\pgfpathlineto{\pgfqpoint{3.628615in}{2.532415in}}%
\pgfpathlineto{\pgfqpoint{3.630939in}{2.544921in}}%
\pgfpathlineto{\pgfqpoint{3.633263in}{2.525975in}}%
\pgfpathlineto{\pgfqpoint{3.635587in}{2.517478in}}%
\pgfpathlineto{\pgfqpoint{3.637911in}{2.545966in}}%
\pgfpathlineto{\pgfqpoint{3.640235in}{2.508344in}}%
\pgfpathlineto{\pgfqpoint{3.647207in}{2.565944in}}%
\pgfpathlineto{\pgfqpoint{3.649531in}{2.537094in}}%
\pgfpathlineto{\pgfqpoint{3.651855in}{2.548374in}}%
\pgfpathlineto{\pgfqpoint{3.654179in}{2.579849in}}%
\pgfpathlineto{\pgfqpoint{3.656503in}{2.571702in}}%
\pgfpathlineto{\pgfqpoint{3.658827in}{2.555662in}}%
\pgfpathlineto{\pgfqpoint{3.661151in}{2.496062in}}%
\pgfpathlineto{\pgfqpoint{3.663475in}{2.580836in}}%
\pgfpathlineto{\pgfqpoint{3.670447in}{2.524363in}}%
\pgfpathlineto{\pgfqpoint{3.672771in}{2.594101in}}%
\pgfpathlineto{\pgfqpoint{3.675095in}{2.606522in}}%
\pgfpathlineto{\pgfqpoint{3.677419in}{2.527198in}}%
\pgfpathlineto{\pgfqpoint{3.679743in}{2.534376in}}%
\pgfpathlineto{\pgfqpoint{3.682067in}{2.553137in}}%
\pgfpathlineto{\pgfqpoint{3.684391in}{2.552255in}}%
\pgfpathlineto{\pgfqpoint{3.686715in}{2.542638in}}%
\pgfpathlineto{\pgfqpoint{3.689038in}{2.571928in}}%
\pgfpathlineto{\pgfqpoint{3.691362in}{2.538987in}}%
\pgfpathlineto{\pgfqpoint{3.693686in}{2.541741in}}%
\pgfpathlineto{\pgfqpoint{3.696010in}{2.575784in}}%
\pgfpathlineto{\pgfqpoint{3.698334in}{2.570844in}}%
\pgfpathlineto{\pgfqpoint{3.700658in}{2.605624in}}%
\pgfpathlineto{\pgfqpoint{3.702982in}{2.513539in}}%
\pgfpathlineto{\pgfqpoint{3.705306in}{2.521558in}}%
\pgfpathlineto{\pgfqpoint{3.707630in}{2.524250in}}%
\pgfpathlineto{\pgfqpoint{3.712278in}{2.617995in}}%
\pgfpathlineto{\pgfqpoint{3.714602in}{2.585188in}}%
\pgfpathlineto{\pgfqpoint{3.716926in}{2.520686in}}%
\pgfpathlineto{\pgfqpoint{3.719250in}{2.547448in}}%
\pgfpathlineto{\pgfqpoint{3.721574in}{2.541415in}}%
\pgfpathlineto{\pgfqpoint{3.726222in}{2.576004in}}%
\pgfpathlineto{\pgfqpoint{3.728546in}{2.563116in}}%
\pgfpathlineto{\pgfqpoint{3.730870in}{2.565626in}}%
\pgfpathlineto{\pgfqpoint{3.733194in}{2.529200in}}%
\pgfpathlineto{\pgfqpoint{3.735518in}{2.593177in}}%
\pgfpathlineto{\pgfqpoint{3.740166in}{2.545040in}}%
\pgfpathlineto{\pgfqpoint{3.742490in}{2.624474in}}%
\pgfpathlineto{\pgfqpoint{3.744814in}{2.564658in}}%
\pgfpathlineto{\pgfqpoint{3.747138in}{2.597480in}}%
\pgfpathlineto{\pgfqpoint{3.749462in}{2.571097in}}%
\pgfpathlineto{\pgfqpoint{3.751786in}{2.576904in}}%
\pgfpathlineto{\pgfqpoint{3.754110in}{2.518037in}}%
\pgfpathlineto{\pgfqpoint{3.756434in}{2.556185in}}%
\pgfpathlineto{\pgfqpoint{3.758757in}{2.614053in}}%
\pgfpathlineto{\pgfqpoint{3.763405in}{2.567693in}}%
\pgfpathlineto{\pgfqpoint{3.765729in}{2.582765in}}%
\pgfpathlineto{\pgfqpoint{3.768053in}{2.584309in}}%
\pgfpathlineto{\pgfqpoint{3.770377in}{2.618257in}}%
\pgfpathlineto{\pgfqpoint{3.772701in}{2.576443in}}%
\pgfpathlineto{\pgfqpoint{3.775025in}{2.621557in}}%
\pgfpathlineto{\pgfqpoint{3.777349in}{2.564903in}}%
\pgfpathlineto{\pgfqpoint{3.779673in}{2.541477in}}%
\pgfpathlineto{\pgfqpoint{3.781997in}{2.541206in}}%
\pgfpathlineto{\pgfqpoint{3.784321in}{2.560873in}}%
\pgfpathlineto{\pgfqpoint{3.786645in}{2.563539in}}%
\pgfpathlineto{\pgfqpoint{3.788969in}{2.562311in}}%
\pgfpathlineto{\pgfqpoint{3.791293in}{2.500963in}}%
\pgfpathlineto{\pgfqpoint{3.793617in}{2.602517in}}%
\pgfpathlineto{\pgfqpoint{3.795941in}{2.558924in}}%
\pgfpathlineto{\pgfqpoint{3.798265in}{2.568061in}}%
\pgfpathlineto{\pgfqpoint{3.800589in}{2.646658in}}%
\pgfpathlineto{\pgfqpoint{3.805237in}{2.552465in}}%
\pgfpathlineto{\pgfqpoint{3.807561in}{2.608236in}}%
\pgfpathlineto{\pgfqpoint{3.809885in}{2.580160in}}%
\pgfpathlineto{\pgfqpoint{3.812209in}{2.578574in}}%
\pgfpathlineto{\pgfqpoint{3.814533in}{2.582751in}}%
\pgfpathlineto{\pgfqpoint{3.816857in}{2.578198in}}%
\pgfpathlineto{\pgfqpoint{3.819181in}{2.614350in}}%
\pgfpathlineto{\pgfqpoint{3.821505in}{2.630499in}}%
\pgfpathlineto{\pgfqpoint{3.823829in}{2.555661in}}%
\pgfpathlineto{\pgfqpoint{3.828477in}{2.624035in}}%
\pgfpathlineto{\pgfqpoint{3.830800in}{2.572485in}}%
\pgfpathlineto{\pgfqpoint{3.833124in}{2.568464in}}%
\pgfpathlineto{\pgfqpoint{3.835448in}{2.572413in}}%
\pgfpathlineto{\pgfqpoint{3.837772in}{2.606923in}}%
\pgfpathlineto{\pgfqpoint{3.840096in}{2.580218in}}%
\pgfpathlineto{\pgfqpoint{3.842420in}{2.618915in}}%
\pgfpathlineto{\pgfqpoint{3.844744in}{2.615648in}}%
\pgfpathlineto{\pgfqpoint{3.847068in}{2.461070in}}%
\pgfpathlineto{\pgfqpoint{3.849392in}{2.528756in}}%
\pgfpathlineto{\pgfqpoint{3.851716in}{2.507214in}}%
\pgfpathlineto{\pgfqpoint{3.854040in}{2.510718in}}%
\pgfpathlineto{\pgfqpoint{3.856364in}{2.496802in}}%
\pgfpathlineto{\pgfqpoint{3.861012in}{2.544402in}}%
\pgfpathlineto{\pgfqpoint{3.863336in}{2.516195in}}%
\pgfpathlineto{\pgfqpoint{3.865660in}{2.552528in}}%
\pgfpathlineto{\pgfqpoint{3.867984in}{2.491615in}}%
\pgfpathlineto{\pgfqpoint{3.870308in}{2.544910in}}%
\pgfpathlineto{\pgfqpoint{3.872632in}{2.502141in}}%
\pgfpathlineto{\pgfqpoint{3.874956in}{2.492086in}}%
\pgfpathlineto{\pgfqpoint{3.877280in}{2.503502in}}%
\pgfpathlineto{\pgfqpoint{3.879604in}{2.549603in}}%
\pgfpathlineto{\pgfqpoint{3.881928in}{2.507257in}}%
\pgfpathlineto{\pgfqpoint{3.884252in}{2.528655in}}%
\pgfpathlineto{\pgfqpoint{3.886576in}{2.480436in}}%
\pgfpathlineto{\pgfqpoint{3.891224in}{2.547043in}}%
\pgfpathlineto{\pgfqpoint{3.893548in}{2.504504in}}%
\pgfpathlineto{\pgfqpoint{3.895872in}{2.518789in}}%
\pgfpathlineto{\pgfqpoint{3.898196in}{2.476242in}}%
\pgfpathlineto{\pgfqpoint{3.900519in}{2.530479in}}%
\pgfpathlineto{\pgfqpoint{3.902843in}{2.538828in}}%
\pgfpathlineto{\pgfqpoint{3.905167in}{2.566925in}}%
\pgfpathlineto{\pgfqpoint{3.909815in}{2.513108in}}%
\pgfpathlineto{\pgfqpoint{3.912139in}{2.547318in}}%
\pgfpathlineto{\pgfqpoint{3.914463in}{2.520759in}}%
\pgfpathlineto{\pgfqpoint{3.916787in}{2.559012in}}%
\pgfpathlineto{\pgfqpoint{3.919111in}{2.519120in}}%
\pgfpathlineto{\pgfqpoint{3.921435in}{2.589694in}}%
\pgfpathlineto{\pgfqpoint{3.923759in}{2.563273in}}%
\pgfpathlineto{\pgfqpoint{3.926083in}{2.510469in}}%
\pgfpathlineto{\pgfqpoint{3.930731in}{2.571312in}}%
\pgfpathlineto{\pgfqpoint{3.933055in}{2.570122in}}%
\pgfpathlineto{\pgfqpoint{3.935379in}{2.577750in}}%
\pgfpathlineto{\pgfqpoint{3.940027in}{2.551019in}}%
\pgfpathlineto{\pgfqpoint{3.942351in}{2.478620in}}%
\pgfpathlineto{\pgfqpoint{3.944675in}{2.553321in}}%
\pgfpathlineto{\pgfqpoint{3.946999in}{2.568122in}}%
\pgfpathlineto{\pgfqpoint{3.949323in}{2.528546in}}%
\pgfpathlineto{\pgfqpoint{3.951647in}{2.534663in}}%
\pgfpathlineto{\pgfqpoint{3.953971in}{2.560828in}}%
\pgfpathlineto{\pgfqpoint{3.958619in}{2.549380in}}%
\pgfpathlineto{\pgfqpoint{3.960943in}{2.589233in}}%
\pgfpathlineto{\pgfqpoint{3.963267in}{2.532376in}}%
\pgfpathlineto{\pgfqpoint{3.965591in}{2.538819in}}%
\pgfpathlineto{\pgfqpoint{3.967915in}{2.604501in}}%
\pgfpathlineto{\pgfqpoint{3.972562in}{2.523935in}}%
\pgfpathlineto{\pgfqpoint{3.977210in}{2.553601in}}%
\pgfpathlineto{\pgfqpoint{3.979534in}{2.534310in}}%
\pgfpathlineto{\pgfqpoint{3.981858in}{2.573039in}}%
\pgfpathlineto{\pgfqpoint{3.984182in}{2.568147in}}%
\pgfpathlineto{\pgfqpoint{3.986506in}{2.532637in}}%
\pgfpathlineto{\pgfqpoint{3.988830in}{2.594136in}}%
\pgfpathlineto{\pgfqpoint{3.991154in}{2.593743in}}%
\pgfpathlineto{\pgfqpoint{3.993478in}{2.588562in}}%
\pgfpathlineto{\pgfqpoint{3.998126in}{2.547853in}}%
\pgfpathlineto{\pgfqpoint{4.002774in}{2.577059in}}%
\pgfpathlineto{\pgfqpoint{4.005098in}{2.539221in}}%
\pgfpathlineto{\pgfqpoint{4.007422in}{2.574997in}}%
\pgfpathlineto{\pgfqpoint{4.009746in}{2.554695in}}%
\pgfpathlineto{\pgfqpoint{4.012070in}{2.562968in}}%
\pgfpathlineto{\pgfqpoint{4.014394in}{2.548026in}}%
\pgfpathlineto{\pgfqpoint{4.016718in}{2.620356in}}%
\pgfpathlineto{\pgfqpoint{4.019042in}{2.640068in}}%
\pgfpathlineto{\pgfqpoint{4.021366in}{2.585307in}}%
\pgfpathlineto{\pgfqpoint{4.023690in}{2.625336in}}%
\pgfpathlineto{\pgfqpoint{4.026014in}{2.576102in}}%
\pgfpathlineto{\pgfqpoint{4.028338in}{2.634315in}}%
\pgfpathlineto{\pgfqpoint{4.030662in}{2.567006in}}%
\pgfpathlineto{\pgfqpoint{4.032986in}{2.587430in}}%
\pgfpathlineto{\pgfqpoint{4.035310in}{2.584440in}}%
\pgfpathlineto{\pgfqpoint{4.037634in}{2.638995in}}%
\pgfpathlineto{\pgfqpoint{4.039957in}{2.626971in}}%
\pgfpathlineto{\pgfqpoint{4.042281in}{2.596191in}}%
\pgfpathlineto{\pgfqpoint{4.044605in}{2.619111in}}%
\pgfpathlineto{\pgfqpoint{4.046929in}{2.558343in}}%
\pgfpathlineto{\pgfqpoint{4.049253in}{2.566519in}}%
\pgfpathlineto{\pgfqpoint{4.051577in}{2.612356in}}%
\pgfpathlineto{\pgfqpoint{4.053901in}{2.602979in}}%
\pgfpathlineto{\pgfqpoint{4.056225in}{2.575977in}}%
\pgfpathlineto{\pgfqpoint{4.058549in}{2.648004in}}%
\pgfpathlineto{\pgfqpoint{4.060873in}{2.621945in}}%
\pgfpathlineto{\pgfqpoint{4.063197in}{2.631986in}}%
\pgfpathlineto{\pgfqpoint{4.065521in}{2.604860in}}%
\pgfpathlineto{\pgfqpoint{4.070169in}{2.640466in}}%
\pgfpathlineto{\pgfqpoint{4.072493in}{2.621151in}}%
\pgfpathlineto{\pgfqpoint{4.074817in}{2.614164in}}%
\pgfpathlineto{\pgfqpoint{4.077141in}{2.657300in}}%
\pgfpathlineto{\pgfqpoint{4.079465in}{2.589105in}}%
\pgfpathlineto{\pgfqpoint{4.081789in}{2.605806in}}%
\pgfpathlineto{\pgfqpoint{4.084113in}{2.612543in}}%
\pgfpathlineto{\pgfqpoint{4.086437in}{2.612278in}}%
\pgfpathlineto{\pgfqpoint{4.088761in}{2.614263in}}%
\pgfpathlineto{\pgfqpoint{4.091085in}{2.633796in}}%
\pgfpathlineto{\pgfqpoint{4.093409in}{2.633386in}}%
\pgfpathlineto{\pgfqpoint{4.095733in}{2.596440in}}%
\pgfpathlineto{\pgfqpoint{4.098057in}{2.620274in}}%
\pgfpathlineto{\pgfqpoint{4.100381in}{2.581549in}}%
\pgfpathlineto{\pgfqpoint{4.102705in}{2.582952in}}%
\pgfpathlineto{\pgfqpoint{4.105029in}{2.605758in}}%
\pgfpathlineto{\pgfqpoint{4.107353in}{2.641682in}}%
\pgfpathlineto{\pgfqpoint{4.109677in}{2.602318in}}%
\pgfpathlineto{\pgfqpoint{4.112000in}{2.668443in}}%
\pgfpathlineto{\pgfqpoint{4.114324in}{2.700747in}}%
\pgfpathlineto{\pgfqpoint{4.116648in}{2.584957in}}%
\pgfpathlineto{\pgfqpoint{4.118972in}{2.627197in}}%
\pgfpathlineto{\pgfqpoint{4.121296in}{2.600826in}}%
\pgfpathlineto{\pgfqpoint{4.123620in}{2.675736in}}%
\pgfpathlineto{\pgfqpoint{4.128268in}{2.649143in}}%
\pgfpathlineto{\pgfqpoint{4.130592in}{2.597673in}}%
\pgfpathlineto{\pgfqpoint{4.132916in}{2.594675in}}%
\pgfpathlineto{\pgfqpoint{4.135240in}{2.687238in}}%
\pgfpathlineto{\pgfqpoint{4.137564in}{2.355584in}}%
\pgfpathlineto{\pgfqpoint{4.139888in}{2.345136in}}%
\pgfpathlineto{\pgfqpoint{4.142212in}{2.371375in}}%
\pgfpathlineto{\pgfqpoint{4.144536in}{2.347940in}}%
\pgfpathlineto{\pgfqpoint{4.146860in}{2.297211in}}%
\pgfpathlineto{\pgfqpoint{4.149184in}{2.375347in}}%
\pgfpathlineto{\pgfqpoint{4.151508in}{2.332639in}}%
\pgfpathlineto{\pgfqpoint{4.156156in}{2.392450in}}%
\pgfpathlineto{\pgfqpoint{4.158480in}{2.342641in}}%
\pgfpathlineto{\pgfqpoint{4.160804in}{2.381607in}}%
\pgfpathlineto{\pgfqpoint{4.163128in}{2.391999in}}%
\pgfpathlineto{\pgfqpoint{4.165452in}{2.381897in}}%
\pgfpathlineto{\pgfqpoint{4.167776in}{2.358009in}}%
\pgfpathlineto{\pgfqpoint{4.172424in}{2.391044in}}%
\pgfpathlineto{\pgfqpoint{4.177072in}{2.391627in}}%
\pgfpathlineto{\pgfqpoint{4.179396in}{2.355352in}}%
\pgfpathlineto{\pgfqpoint{4.181719in}{2.373152in}}%
\pgfpathlineto{\pgfqpoint{4.184043in}{2.413663in}}%
\pgfpathlineto{\pgfqpoint{4.186367in}{2.344440in}}%
\pgfpathlineto{\pgfqpoint{4.188691in}{2.392846in}}%
\pgfpathlineto{\pgfqpoint{4.191015in}{2.417246in}}%
\pgfpathlineto{\pgfqpoint{4.193339in}{2.374712in}}%
\pgfpathlineto{\pgfqpoint{4.195663in}{2.396461in}}%
\pgfpathlineto{\pgfqpoint{4.197987in}{2.383715in}}%
\pgfpathlineto{\pgfqpoint{4.200311in}{2.403168in}}%
\pgfpathlineto{\pgfqpoint{4.202635in}{2.384558in}}%
\pgfpathlineto{\pgfqpoint{4.204959in}{2.380153in}}%
\pgfpathlineto{\pgfqpoint{4.207283in}{2.342791in}}%
\pgfpathlineto{\pgfqpoint{4.209607in}{2.422954in}}%
\pgfpathlineto{\pgfqpoint{4.211931in}{2.408125in}}%
\pgfpathlineto{\pgfqpoint{4.214255in}{2.383236in}}%
\pgfpathlineto{\pgfqpoint{4.216579in}{2.405805in}}%
\pgfpathlineto{\pgfqpoint{4.218903in}{2.410162in}}%
\pgfpathlineto{\pgfqpoint{4.221227in}{2.392455in}}%
\pgfpathlineto{\pgfqpoint{4.223551in}{2.350101in}}%
\pgfpathlineto{\pgfqpoint{4.225875in}{2.462617in}}%
\pgfpathlineto{\pgfqpoint{4.228199in}{2.470030in}}%
\pgfpathlineto{\pgfqpoint{4.230523in}{2.411133in}}%
\pgfpathlineto{\pgfqpoint{4.232847in}{2.395857in}}%
\pgfpathlineto{\pgfqpoint{4.235171in}{2.418257in}}%
\pgfpathlineto{\pgfqpoint{4.237495in}{2.405084in}}%
\pgfpathlineto{\pgfqpoint{4.239819in}{2.471037in}}%
\pgfpathlineto{\pgfqpoint{4.242143in}{2.410624in}}%
\pgfpathlineto{\pgfqpoint{4.244467in}{2.387071in}}%
\pgfpathlineto{\pgfqpoint{4.246791in}{2.451559in}}%
\pgfpathlineto{\pgfqpoint{4.249115in}{2.470813in}}%
\pgfpathlineto{\pgfqpoint{4.251438in}{2.454163in}}%
\pgfpathlineto{\pgfqpoint{4.253762in}{2.410462in}}%
\pgfpathlineto{\pgfqpoint{4.258410in}{2.455831in}}%
\pgfpathlineto{\pgfqpoint{4.260734in}{2.384362in}}%
\pgfpathlineto{\pgfqpoint{4.263058in}{2.462565in}}%
\pgfpathlineto{\pgfqpoint{4.265382in}{2.458560in}}%
\pgfpathlineto{\pgfqpoint{4.267706in}{2.482092in}}%
\pgfpathlineto{\pgfqpoint{4.270030in}{2.466391in}}%
\pgfpathlineto{\pgfqpoint{4.272354in}{2.438560in}}%
\pgfpathlineto{\pgfqpoint{4.274678in}{2.473367in}}%
\pgfpathlineto{\pgfqpoint{4.277002in}{2.450413in}}%
\pgfpathlineto{\pgfqpoint{4.279326in}{2.453412in}}%
\pgfpathlineto{\pgfqpoint{4.281650in}{2.490027in}}%
\pgfpathlineto{\pgfqpoint{4.286298in}{2.433783in}}%
\pgfpathlineto{\pgfqpoint{4.288622in}{2.514613in}}%
\pgfpathlineto{\pgfqpoint{4.290946in}{2.482507in}}%
\pgfpathlineto{\pgfqpoint{4.293270in}{2.548726in}}%
\pgfpathlineto{\pgfqpoint{4.295594in}{2.484462in}}%
\pgfpathlineto{\pgfqpoint{4.297918in}{2.477528in}}%
\pgfpathlineto{\pgfqpoint{4.300242in}{2.450978in}}%
\pgfpathlineto{\pgfqpoint{4.302566in}{2.463412in}}%
\pgfpathlineto{\pgfqpoint{4.304890in}{2.487894in}}%
\pgfpathlineto{\pgfqpoint{4.307214in}{2.537151in}}%
\pgfpathlineto{\pgfqpoint{4.309538in}{2.466619in}}%
\pgfpathlineto{\pgfqpoint{4.311862in}{2.482984in}}%
\pgfpathlineto{\pgfqpoint{4.314186in}{2.489061in}}%
\pgfpathlineto{\pgfqpoint{4.316510in}{2.509864in}}%
\pgfpathlineto{\pgfqpoint{4.318834in}{2.505268in}}%
\pgfpathlineto{\pgfqpoint{4.321158in}{2.530561in}}%
\pgfpathlineto{\pgfqpoint{4.323481in}{2.570830in}}%
\pgfpathlineto{\pgfqpoint{4.325805in}{2.499882in}}%
\pgfpathlineto{\pgfqpoint{4.328129in}{2.528728in}}%
\pgfpathlineto{\pgfqpoint{4.330453in}{2.517959in}}%
\pgfpathlineto{\pgfqpoint{4.332777in}{2.518270in}}%
\pgfpathlineto{\pgfqpoint{4.335101in}{2.536302in}}%
\pgfpathlineto{\pgfqpoint{4.337425in}{2.480057in}}%
\pgfpathlineto{\pgfqpoint{4.339749in}{2.455037in}}%
\pgfpathlineto{\pgfqpoint{4.342073in}{2.486339in}}%
\pgfpathlineto{\pgfqpoint{4.344397in}{2.563200in}}%
\pgfpathlineto{\pgfqpoint{4.346721in}{2.482118in}}%
\pgfpathlineto{\pgfqpoint{4.349045in}{2.548862in}}%
\pgfpathlineto{\pgfqpoint{4.351369in}{2.527216in}}%
\pgfpathlineto{\pgfqpoint{4.353693in}{2.516105in}}%
\pgfpathlineto{\pgfqpoint{4.356017in}{2.542536in}}%
\pgfpathlineto{\pgfqpoint{4.358341in}{2.528224in}}%
\pgfpathlineto{\pgfqpoint{4.360665in}{2.482811in}}%
\pgfpathlineto{\pgfqpoint{4.362989in}{2.576042in}}%
\pgfpathlineto{\pgfqpoint{4.365313in}{2.592521in}}%
\pgfpathlineto{\pgfqpoint{4.367637in}{2.555020in}}%
\pgfpathlineto{\pgfqpoint{4.369961in}{2.547292in}}%
\pgfpathlineto{\pgfqpoint{4.374609in}{2.553301in}}%
\pgfpathlineto{\pgfqpoint{4.379257in}{2.570979in}}%
\pgfpathlineto{\pgfqpoint{4.381581in}{2.509227in}}%
\pgfpathlineto{\pgfqpoint{4.383905in}{2.602321in}}%
\pgfpathlineto{\pgfqpoint{4.386229in}{2.515131in}}%
\pgfpathlineto{\pgfqpoint{4.388553in}{2.581133in}}%
\pgfpathlineto{\pgfqpoint{4.390877in}{2.585038in}}%
\pgfpathlineto{\pgfqpoint{4.393200in}{2.540000in}}%
\pgfpathlineto{\pgfqpoint{4.395524in}{2.589941in}}%
\pgfpathlineto{\pgfqpoint{4.400172in}{2.569690in}}%
\pgfpathlineto{\pgfqpoint{4.402496in}{2.545295in}}%
\pgfpathlineto{\pgfqpoint{4.404820in}{2.581065in}}%
\pgfpathlineto{\pgfqpoint{4.407144in}{2.530789in}}%
\pgfpathlineto{\pgfqpoint{4.409468in}{2.616646in}}%
\pgfpathlineto{\pgfqpoint{4.411792in}{2.613936in}}%
\pgfpathlineto{\pgfqpoint{4.414116in}{2.576827in}}%
\pgfpathlineto{\pgfqpoint{4.416440in}{2.509742in}}%
\pgfpathlineto{\pgfqpoint{4.418764in}{2.582408in}}%
\pgfpathlineto{\pgfqpoint{4.421088in}{2.594467in}}%
\pgfpathlineto{\pgfqpoint{4.423412in}{2.579141in}}%
\pgfpathlineto{\pgfqpoint{4.425736in}{2.607751in}}%
\pgfpathlineto{\pgfqpoint{4.428060in}{2.518971in}}%
\pgfpathlineto{\pgfqpoint{4.430384in}{2.540342in}}%
\pgfpathlineto{\pgfqpoint{4.432708in}{2.467421in}}%
\pgfpathlineto{\pgfqpoint{4.435032in}{2.523642in}}%
\pgfpathlineto{\pgfqpoint{4.437356in}{2.526904in}}%
\pgfpathlineto{\pgfqpoint{4.439680in}{2.517148in}}%
\pgfpathlineto{\pgfqpoint{4.442004in}{2.534816in}}%
\pgfpathlineto{\pgfqpoint{4.446652in}{2.552820in}}%
\pgfpathlineto{\pgfqpoint{4.448976in}{2.524636in}}%
\pgfpathlineto{\pgfqpoint{4.453624in}{2.588899in}}%
\pgfpathlineto{\pgfqpoint{4.455948in}{2.567691in}}%
\pgfpathlineto{\pgfqpoint{4.458272in}{2.585711in}}%
\pgfpathlineto{\pgfqpoint{4.460596in}{2.553463in}}%
\pgfpathlineto{\pgfqpoint{4.462919in}{2.571016in}}%
\pgfpathlineto{\pgfqpoint{4.465243in}{2.552560in}}%
\pgfpathlineto{\pgfqpoint{4.467567in}{2.514228in}}%
\pgfpathlineto{\pgfqpoint{4.469891in}{2.566685in}}%
\pgfpathlineto{\pgfqpoint{4.472215in}{2.580510in}}%
\pgfpathlineto{\pgfqpoint{4.474539in}{2.543116in}}%
\pgfpathlineto{\pgfqpoint{4.476863in}{2.583978in}}%
\pgfpathlineto{\pgfqpoint{4.479187in}{2.588872in}}%
\pgfpathlineto{\pgfqpoint{4.481511in}{2.546287in}}%
\pgfpathlineto{\pgfqpoint{4.483835in}{2.567107in}}%
\pgfpathlineto{\pgfqpoint{4.486159in}{2.576377in}}%
\pgfpathlineto{\pgfqpoint{4.488483in}{2.593758in}}%
\pgfpathlineto{\pgfqpoint{4.490807in}{2.545920in}}%
\pgfpathlineto{\pgfqpoint{4.493131in}{2.556879in}}%
\pgfpathlineto{\pgfqpoint{4.495455in}{2.539080in}}%
\pgfpathlineto{\pgfqpoint{4.497779in}{2.609046in}}%
\pgfpathlineto{\pgfqpoint{4.502427in}{2.542231in}}%
\pgfpathlineto{\pgfqpoint{4.504751in}{2.625050in}}%
\pgfpathlineto{\pgfqpoint{4.507075in}{2.601948in}}%
\pgfpathlineto{\pgfqpoint{4.509399in}{2.594419in}}%
\pgfpathlineto{\pgfqpoint{4.511723in}{2.636430in}}%
\pgfpathlineto{\pgfqpoint{4.516371in}{2.598483in}}%
\pgfpathlineto{\pgfqpoint{4.518695in}{2.592517in}}%
\pgfpathlineto{\pgfqpoint{4.521019in}{2.580542in}}%
\pgfpathlineto{\pgfqpoint{4.523343in}{2.633222in}}%
\pgfpathlineto{\pgfqpoint{4.525667in}{2.614363in}}%
\pgfpathlineto{\pgfqpoint{4.527991in}{2.604871in}}%
\pgfpathlineto{\pgfqpoint{4.530315in}{2.635997in}}%
\pgfpathlineto{\pgfqpoint{4.532638in}{2.652180in}}%
\pgfpathlineto{\pgfqpoint{4.534962in}{2.641635in}}%
\pgfpathlineto{\pgfqpoint{4.537286in}{2.617301in}}%
\pgfpathlineto{\pgfqpoint{4.539610in}{2.615715in}}%
\pgfpathlineto{\pgfqpoint{4.541934in}{2.620082in}}%
\pgfpathlineto{\pgfqpoint{4.544258in}{2.645477in}}%
\pgfpathlineto{\pgfqpoint{4.546582in}{2.624493in}}%
\pgfpathlineto{\pgfqpoint{4.548906in}{2.661477in}}%
\pgfpathlineto{\pgfqpoint{4.551230in}{2.618253in}}%
\pgfpathlineto{\pgfqpoint{4.553554in}{2.651880in}}%
\pgfpathlineto{\pgfqpoint{4.555878in}{2.645195in}}%
\pgfpathlineto{\pgfqpoint{4.558202in}{2.625780in}}%
\pgfpathlineto{\pgfqpoint{4.560526in}{2.630360in}}%
\pgfpathlineto{\pgfqpoint{4.562850in}{2.684694in}}%
\pgfpathlineto{\pgfqpoint{4.565174in}{2.677080in}}%
\pgfpathlineto{\pgfqpoint{4.567498in}{2.664538in}}%
\pgfpathlineto{\pgfqpoint{4.569822in}{2.630855in}}%
\pgfpathlineto{\pgfqpoint{4.572146in}{2.634969in}}%
\pgfpathlineto{\pgfqpoint{4.574470in}{2.680684in}}%
\pgfpathlineto{\pgfqpoint{4.576794in}{2.641257in}}%
\pgfpathlineto{\pgfqpoint{4.579118in}{2.729788in}}%
\pgfpathlineto{\pgfqpoint{4.581442in}{2.677090in}}%
\pgfpathlineto{\pgfqpoint{4.583766in}{2.666308in}}%
\pgfpathlineto{\pgfqpoint{4.586090in}{2.708747in}}%
\pgfpathlineto{\pgfqpoint{4.590738in}{2.639872in}}%
\pgfpathlineto{\pgfqpoint{4.593062in}{2.708742in}}%
\pgfpathlineto{\pgfqpoint{4.595386in}{2.711878in}}%
\pgfpathlineto{\pgfqpoint{4.597710in}{2.699303in}}%
\pgfpathlineto{\pgfqpoint{4.600034in}{2.723949in}}%
\pgfpathlineto{\pgfqpoint{4.602358in}{2.690833in}}%
\pgfpathlineto{\pgfqpoint{4.604681in}{2.735266in}}%
\pgfpathlineto{\pgfqpoint{4.607005in}{2.686426in}}%
\pgfpathlineto{\pgfqpoint{4.609329in}{2.672260in}}%
\pgfpathlineto{\pgfqpoint{4.613977in}{2.735068in}}%
\pgfpathlineto{\pgfqpoint{4.616301in}{2.684508in}}%
\pgfpathlineto{\pgfqpoint{4.618625in}{2.741791in}}%
\pgfpathlineto{\pgfqpoint{4.620949in}{2.704110in}}%
\pgfpathlineto{\pgfqpoint{4.623273in}{2.786075in}}%
\pgfpathlineto{\pgfqpoint{4.627921in}{2.689185in}}%
\pgfpathlineto{\pgfqpoint{4.630245in}{2.716758in}}%
\pgfpathlineto{\pgfqpoint{4.632569in}{2.761565in}}%
\pgfpathlineto{\pgfqpoint{4.634893in}{2.704853in}}%
\pgfpathlineto{\pgfqpoint{4.637217in}{2.713491in}}%
\pgfpathlineto{\pgfqpoint{4.639541in}{2.713841in}}%
\pgfpathlineto{\pgfqpoint{4.644189in}{2.803661in}}%
\pgfpathlineto{\pgfqpoint{4.646513in}{2.723429in}}%
\pgfpathlineto{\pgfqpoint{4.648837in}{2.780384in}}%
\pgfpathlineto{\pgfqpoint{4.651161in}{2.755153in}}%
\pgfpathlineto{\pgfqpoint{4.653485in}{2.690720in}}%
\pgfpathlineto{\pgfqpoint{4.655809in}{2.774425in}}%
\pgfpathlineto{\pgfqpoint{4.658133in}{2.724174in}}%
\pgfpathlineto{\pgfqpoint{4.662781in}{2.780059in}}%
\pgfpathlineto{\pgfqpoint{4.665105in}{2.780688in}}%
\pgfpathlineto{\pgfqpoint{4.667429in}{2.812577in}}%
\pgfpathlineto{\pgfqpoint{4.669753in}{2.803462in}}%
\pgfpathlineto{\pgfqpoint{4.672077in}{2.771118in}}%
\pgfpathlineto{\pgfqpoint{4.674400in}{2.786769in}}%
\pgfpathlineto{\pgfqpoint{4.676724in}{2.784106in}}%
\pgfpathlineto{\pgfqpoint{4.679048in}{2.797364in}}%
\pgfpathlineto{\pgfqpoint{4.681372in}{2.769550in}}%
\pgfpathlineto{\pgfqpoint{4.683696in}{2.755458in}}%
\pgfpathlineto{\pgfqpoint{4.686020in}{2.794497in}}%
\pgfpathlineto{\pgfqpoint{4.688344in}{2.796160in}}%
\pgfpathlineto{\pgfqpoint{4.690668in}{2.804323in}}%
\pgfpathlineto{\pgfqpoint{4.692992in}{2.842253in}}%
\pgfpathlineto{\pgfqpoint{4.695316in}{2.767933in}}%
\pgfpathlineto{\pgfqpoint{4.697640in}{2.776344in}}%
\pgfpathlineto{\pgfqpoint{4.699964in}{2.831121in}}%
\pgfpathlineto{\pgfqpoint{4.702288in}{2.782289in}}%
\pgfpathlineto{\pgfqpoint{4.704612in}{2.823139in}}%
\pgfpathlineto{\pgfqpoint{4.706936in}{2.807764in}}%
\pgfpathlineto{\pgfqpoint{4.709260in}{2.831295in}}%
\pgfpathlineto{\pgfqpoint{4.711584in}{2.790548in}}%
\pgfpathlineto{\pgfqpoint{4.713908in}{2.830595in}}%
\pgfpathlineto{\pgfqpoint{4.716232in}{2.785979in}}%
\pgfpathlineto{\pgfqpoint{4.718556in}{2.524218in}}%
\pgfpathlineto{\pgfqpoint{4.720880in}{2.494812in}}%
\pgfpathlineto{\pgfqpoint{4.723204in}{2.503649in}}%
\pgfpathlineto{\pgfqpoint{4.725528in}{2.518604in}}%
\pgfpathlineto{\pgfqpoint{4.727852in}{2.502023in}}%
\pgfpathlineto{\pgfqpoint{4.730176in}{2.579518in}}%
\pgfpathlineto{\pgfqpoint{4.732500in}{2.532099in}}%
\pgfpathlineto{\pgfqpoint{4.734824in}{2.569034in}}%
\pgfpathlineto{\pgfqpoint{4.737148in}{2.587091in}}%
\pgfpathlineto{\pgfqpoint{4.739472in}{2.547196in}}%
\pgfpathlineto{\pgfqpoint{4.741796in}{2.545727in}}%
\pgfpathlineto{\pgfqpoint{4.744119in}{2.564774in}}%
\pgfpathlineto{\pgfqpoint{4.746443in}{2.557577in}}%
\pgfpathlineto{\pgfqpoint{4.748767in}{2.505439in}}%
\pgfpathlineto{\pgfqpoint{4.751091in}{2.510984in}}%
\pgfpathlineto{\pgfqpoint{4.753415in}{2.575276in}}%
\pgfpathlineto{\pgfqpoint{4.755739in}{2.525271in}}%
\pgfpathlineto{\pgfqpoint{4.758063in}{2.573890in}}%
\pgfpathlineto{\pgfqpoint{4.760387in}{2.591051in}}%
\pgfpathlineto{\pgfqpoint{4.762711in}{2.580755in}}%
\pgfpathlineto{\pgfqpoint{4.765035in}{2.541693in}}%
\pgfpathlineto{\pgfqpoint{4.767359in}{2.559193in}}%
\pgfpathlineto{\pgfqpoint{4.769683in}{2.546419in}}%
\pgfpathlineto{\pgfqpoint{4.772007in}{2.623267in}}%
\pgfpathlineto{\pgfqpoint{4.774331in}{2.620894in}}%
\pgfpathlineto{\pgfqpoint{4.776655in}{2.622871in}}%
\pgfpathlineto{\pgfqpoint{4.781303in}{2.567414in}}%
\pgfpathlineto{\pgfqpoint{4.783627in}{2.592294in}}%
\pgfpathlineto{\pgfqpoint{4.785951in}{2.581768in}}%
\pgfpathlineto{\pgfqpoint{4.788275in}{2.624874in}}%
\pgfpathlineto{\pgfqpoint{4.790599in}{2.554791in}}%
\pgfpathlineto{\pgfqpoint{4.792923in}{2.637099in}}%
\pgfpathlineto{\pgfqpoint{4.797571in}{2.572163in}}%
\pgfpathlineto{\pgfqpoint{4.799895in}{2.599099in}}%
\pgfpathlineto{\pgfqpoint{4.802219in}{2.585178in}}%
\pgfpathlineto{\pgfqpoint{4.804543in}{2.644660in}}%
\pgfpathlineto{\pgfqpoint{4.806867in}{2.614421in}}%
\pgfpathlineto{\pgfqpoint{4.809191in}{2.566745in}}%
\pgfpathlineto{\pgfqpoint{4.811515in}{2.637284in}}%
\pgfpathlineto{\pgfqpoint{4.813838in}{2.627281in}}%
\pgfpathlineto{\pgfqpoint{4.816162in}{2.566162in}}%
\pgfpathlineto{\pgfqpoint{4.818486in}{2.652169in}}%
\pgfpathlineto{\pgfqpoint{4.820810in}{2.624836in}}%
\pgfpathlineto{\pgfqpoint{4.823134in}{2.624175in}}%
\pgfpathlineto{\pgfqpoint{4.825458in}{2.566683in}}%
\pgfpathlineto{\pgfqpoint{4.827782in}{2.626768in}}%
\pgfpathlineto{\pgfqpoint{4.830106in}{2.649788in}}%
\pgfpathlineto{\pgfqpoint{4.832430in}{2.555373in}}%
\pgfpathlineto{\pgfqpoint{4.834754in}{2.658937in}}%
\pgfpathlineto{\pgfqpoint{4.837078in}{2.614835in}}%
\pgfpathlineto{\pgfqpoint{4.839402in}{2.647982in}}%
\pgfpathlineto{\pgfqpoint{4.841726in}{2.624370in}}%
\pgfpathlineto{\pgfqpoint{4.844050in}{2.656932in}}%
\pgfpathlineto{\pgfqpoint{4.846374in}{2.656504in}}%
\pgfpathlineto{\pgfqpoint{4.848698in}{2.675677in}}%
\pgfpathlineto{\pgfqpoint{4.851022in}{2.619922in}}%
\pgfpathlineto{\pgfqpoint{4.855670in}{2.694473in}}%
\pgfpathlineto{\pgfqpoint{4.857994in}{2.618835in}}%
\pgfpathlineto{\pgfqpoint{4.860318in}{2.699162in}}%
\pgfpathlineto{\pgfqpoint{4.862642in}{2.624358in}}%
\pgfpathlineto{\pgfqpoint{4.864966in}{2.622031in}}%
\pgfpathlineto{\pgfqpoint{4.867290in}{2.654301in}}%
\pgfpathlineto{\pgfqpoint{4.869614in}{2.665333in}}%
\pgfpathlineto{\pgfqpoint{4.871938in}{2.634505in}}%
\pgfpathlineto{\pgfqpoint{4.874262in}{2.689165in}}%
\pgfpathlineto{\pgfqpoint{4.876586in}{2.681332in}}%
\pgfpathlineto{\pgfqpoint{4.878910in}{2.653700in}}%
\pgfpathlineto{\pgfqpoint{4.881234in}{2.668049in}}%
\pgfpathlineto{\pgfqpoint{4.883558in}{2.709270in}}%
\pgfpathlineto{\pgfqpoint{4.885881in}{2.701685in}}%
\pgfpathlineto{\pgfqpoint{4.888205in}{2.633180in}}%
\pgfpathlineto{\pgfqpoint{4.890529in}{2.709013in}}%
\pgfpathlineto{\pgfqpoint{4.892853in}{2.670093in}}%
\pgfpathlineto{\pgfqpoint{4.895177in}{2.671715in}}%
\pgfpathlineto{\pgfqpoint{4.897501in}{2.713111in}}%
\pgfpathlineto{\pgfqpoint{4.899825in}{2.651219in}}%
\pgfpathlineto{\pgfqpoint{4.902149in}{2.693160in}}%
\pgfpathlineto{\pgfqpoint{4.904473in}{2.649357in}}%
\pgfpathlineto{\pgfqpoint{4.906797in}{2.709622in}}%
\pgfpathlineto{\pgfqpoint{4.909121in}{2.680623in}}%
\pgfpathlineto{\pgfqpoint{4.911445in}{2.677538in}}%
\pgfpathlineto{\pgfqpoint{4.913769in}{2.732804in}}%
\pgfpathlineto{\pgfqpoint{4.916093in}{2.735546in}}%
\pgfpathlineto{\pgfqpoint{4.918417in}{2.727834in}}%
\pgfpathlineto{\pgfqpoint{4.920741in}{2.693392in}}%
\pgfpathlineto{\pgfqpoint{4.923065in}{2.699301in}}%
\pgfpathlineto{\pgfqpoint{4.925389in}{2.698166in}}%
\pgfpathlineto{\pgfqpoint{4.927713in}{2.754308in}}%
\pgfpathlineto{\pgfqpoint{4.930037in}{2.706052in}}%
\pgfpathlineto{\pgfqpoint{4.932361in}{2.729260in}}%
\pgfpathlineto{\pgfqpoint{4.934685in}{2.679141in}}%
\pgfpathlineto{\pgfqpoint{4.937009in}{2.710623in}}%
\pgfpathlineto{\pgfqpoint{4.939333in}{2.770485in}}%
\pgfpathlineto{\pgfqpoint{4.941657in}{2.649936in}}%
\pgfpathlineto{\pgfqpoint{4.943981in}{2.710297in}}%
\pgfpathlineto{\pgfqpoint{4.946305in}{2.721953in}}%
\pgfpathlineto{\pgfqpoint{4.948629in}{2.690622in}}%
\pgfpathlineto{\pgfqpoint{4.950953in}{2.756826in}}%
\pgfpathlineto{\pgfqpoint{4.953277in}{2.728321in}}%
\pgfpathlineto{\pgfqpoint{4.955600in}{2.726234in}}%
\pgfpathlineto{\pgfqpoint{4.957924in}{2.731557in}}%
\pgfpathlineto{\pgfqpoint{4.960248in}{2.720850in}}%
\pgfpathlineto{\pgfqpoint{4.962572in}{2.683378in}}%
\pgfpathlineto{\pgfqpoint{4.964896in}{2.781574in}}%
\pgfpathlineto{\pgfqpoint{4.967220in}{2.788996in}}%
\pgfpathlineto{\pgfqpoint{4.971868in}{2.700806in}}%
\pgfpathlineto{\pgfqpoint{4.974192in}{2.736669in}}%
\pgfpathlineto{\pgfqpoint{4.976516in}{2.742611in}}%
\pgfpathlineto{\pgfqpoint{4.981164in}{2.762427in}}%
\pgfpathlineto{\pgfqpoint{4.983488in}{2.758405in}}%
\pgfpathlineto{\pgfqpoint{4.985812in}{2.789616in}}%
\pgfpathlineto{\pgfqpoint{4.988136in}{2.753292in}}%
\pgfpathlineto{\pgfqpoint{4.990460in}{2.755014in}}%
\pgfpathlineto{\pgfqpoint{4.992784in}{2.732842in}}%
\pgfpathlineto{\pgfqpoint{4.995108in}{2.777274in}}%
\pgfpathlineto{\pgfqpoint{5.002080in}{2.805167in}}%
\pgfpathlineto{\pgfqpoint{5.004404in}{2.806058in}}%
\pgfpathlineto{\pgfqpoint{5.006728in}{2.761809in}}%
\pgfpathlineto{\pgfqpoint{5.009052in}{2.690238in}}%
\pgfpathlineto{\pgfqpoint{5.011376in}{2.724375in}}%
\pgfpathlineto{\pgfqpoint{5.013700in}{2.701892in}}%
\pgfpathlineto{\pgfqpoint{5.016024in}{2.755228in}}%
\pgfpathlineto{\pgfqpoint{5.018348in}{2.692062in}}%
\pgfpathlineto{\pgfqpoint{5.020672in}{2.712986in}}%
\pgfpathlineto{\pgfqpoint{5.022996in}{2.710531in}}%
\pgfpathlineto{\pgfqpoint{5.025319in}{2.684263in}}%
\pgfpathlineto{\pgfqpoint{5.027643in}{2.721839in}}%
\pgfpathlineto{\pgfqpoint{5.029967in}{2.706941in}}%
\pgfpathlineto{\pgfqpoint{5.034615in}{2.728978in}}%
\pgfpathlineto{\pgfqpoint{5.036939in}{2.682414in}}%
\pgfpathlineto{\pgfqpoint{5.039263in}{2.708770in}}%
\pgfpathlineto{\pgfqpoint{5.041587in}{2.664375in}}%
\pgfpathlineto{\pgfqpoint{5.048559in}{2.731099in}}%
\pgfpathlineto{\pgfqpoint{5.050883in}{2.729688in}}%
\pgfpathlineto{\pgfqpoint{5.053207in}{2.722949in}}%
\pgfpathlineto{\pgfqpoint{5.055531in}{2.732161in}}%
\pgfpathlineto{\pgfqpoint{5.060179in}{2.780509in}}%
\pgfpathlineto{\pgfqpoint{5.062503in}{2.715643in}}%
\pgfpathlineto{\pgfqpoint{5.064827in}{2.790596in}}%
\pgfpathlineto{\pgfqpoint{5.069475in}{2.719794in}}%
\pgfpathlineto{\pgfqpoint{5.071799in}{2.727892in}}%
\pgfpathlineto{\pgfqpoint{5.074123in}{2.713867in}}%
\pgfpathlineto{\pgfqpoint{5.078771in}{2.731958in}}%
\pgfpathlineto{\pgfqpoint{5.081095in}{2.729524in}}%
\pgfpathlineto{\pgfqpoint{5.083419in}{2.734226in}}%
\pgfpathlineto{\pgfqpoint{5.085743in}{2.774460in}}%
\pgfpathlineto{\pgfqpoint{5.088067in}{2.753704in}}%
\pgfpathlineto{\pgfqpoint{5.092715in}{2.746165in}}%
\pgfpathlineto{\pgfqpoint{5.095039in}{2.734995in}}%
\pgfpathlineto{\pgfqpoint{5.097362in}{2.770452in}}%
\pgfpathlineto{\pgfqpoint{5.099686in}{2.737455in}}%
\pgfpathlineto{\pgfqpoint{5.102010in}{2.762719in}}%
\pgfpathlineto{\pgfqpoint{5.104334in}{2.759627in}}%
\pgfpathlineto{\pgfqpoint{5.106658in}{2.731152in}}%
\pgfpathlineto{\pgfqpoint{5.108982in}{2.770769in}}%
\pgfpathlineto{\pgfqpoint{5.111306in}{2.757355in}}%
\pgfpathlineto{\pgfqpoint{5.113630in}{2.802732in}}%
\pgfpathlineto{\pgfqpoint{5.115954in}{2.793586in}}%
\pgfpathlineto{\pgfqpoint{5.118278in}{2.721833in}}%
\pgfpathlineto{\pgfqpoint{5.120602in}{2.772704in}}%
\pgfpathlineto{\pgfqpoint{5.122926in}{2.739690in}}%
\pgfpathlineto{\pgfqpoint{5.125250in}{2.793279in}}%
\pgfpathlineto{\pgfqpoint{5.127574in}{2.790730in}}%
\pgfpathlineto{\pgfqpoint{5.129898in}{2.803803in}}%
\pgfpathlineto{\pgfqpoint{5.132222in}{2.774014in}}%
\pgfpathlineto{\pgfqpoint{5.134546in}{2.799247in}}%
\pgfpathlineto{\pgfqpoint{5.136870in}{2.767797in}}%
\pgfpathlineto{\pgfqpoint{5.139194in}{2.804173in}}%
\pgfpathlineto{\pgfqpoint{5.141518in}{2.795473in}}%
\pgfpathlineto{\pgfqpoint{5.143842in}{2.757106in}}%
\pgfpathlineto{\pgfqpoint{5.146166in}{2.774994in}}%
\pgfpathlineto{\pgfqpoint{5.148490in}{2.836033in}}%
\pgfpathlineto{\pgfqpoint{5.150814in}{2.790330in}}%
\pgfpathlineto{\pgfqpoint{5.155462in}{2.771641in}}%
\pgfpathlineto{\pgfqpoint{5.157786in}{2.792663in}}%
\pgfpathlineto{\pgfqpoint{5.160110in}{2.711309in}}%
\pgfpathlineto{\pgfqpoint{5.164758in}{2.785960in}}%
\pgfpathlineto{\pgfqpoint{5.167081in}{2.808349in}}%
\pgfpathlineto{\pgfqpoint{5.169405in}{2.727369in}}%
\pgfpathlineto{\pgfqpoint{5.171729in}{2.821269in}}%
\pgfpathlineto{\pgfqpoint{5.174053in}{2.758524in}}%
\pgfpathlineto{\pgfqpoint{5.176377in}{2.777065in}}%
\pgfpathlineto{\pgfqpoint{5.178701in}{2.828579in}}%
\pgfpathlineto{\pgfqpoint{5.181025in}{2.739666in}}%
\pgfpathlineto{\pgfqpoint{5.183349in}{2.820861in}}%
\pgfpathlineto{\pgfqpoint{5.185673in}{2.805934in}}%
\pgfpathlineto{\pgfqpoint{5.187997in}{2.804256in}}%
\pgfpathlineto{\pgfqpoint{5.190321in}{2.768003in}}%
\pgfpathlineto{\pgfqpoint{5.192645in}{2.829456in}}%
\pgfpathlineto{\pgfqpoint{5.197293in}{2.844032in}}%
\pgfpathlineto{\pgfqpoint{5.199617in}{2.775780in}}%
\pgfpathlineto{\pgfqpoint{5.204265in}{2.818483in}}%
\pgfpathlineto{\pgfqpoint{5.206589in}{2.784160in}}%
\pgfpathlineto{\pgfqpoint{5.213561in}{2.841383in}}%
\pgfpathlineto{\pgfqpoint{5.215885in}{2.783688in}}%
\pgfpathlineto{\pgfqpoint{5.218209in}{2.792696in}}%
\pgfpathlineto{\pgfqpoint{5.220533in}{2.793299in}}%
\pgfpathlineto{\pgfqpoint{5.222857in}{2.833493in}}%
\pgfpathlineto{\pgfqpoint{5.225181in}{2.772426in}}%
\pgfpathlineto{\pgfqpoint{5.227505in}{2.794542in}}%
\pgfpathlineto{\pgfqpoint{5.229829in}{2.793279in}}%
\pgfpathlineto{\pgfqpoint{5.232153in}{2.853764in}}%
\pgfpathlineto{\pgfqpoint{5.234477in}{2.839874in}}%
\pgfpathlineto{\pgfqpoint{5.236800in}{2.771648in}}%
\pgfpathlineto{\pgfqpoint{5.241448in}{2.891702in}}%
\pgfpathlineto{\pgfqpoint{5.243772in}{2.824193in}}%
\pgfpathlineto{\pgfqpoint{5.246096in}{2.851217in}}%
\pgfpathlineto{\pgfqpoint{5.248420in}{2.799261in}}%
\pgfpathlineto{\pgfqpoint{5.250744in}{2.789633in}}%
\pgfpathlineto{\pgfqpoint{5.253068in}{2.833834in}}%
\pgfpathlineto{\pgfqpoint{5.255392in}{2.833118in}}%
\pgfpathlineto{\pgfqpoint{5.257716in}{2.816645in}}%
\pgfpathlineto{\pgfqpoint{5.260040in}{2.821908in}}%
\pgfpathlineto{\pgfqpoint{5.262364in}{2.821629in}}%
\pgfpathlineto{\pgfqpoint{5.264688in}{2.828913in}}%
\pgfpathlineto{\pgfqpoint{5.267012in}{2.827623in}}%
\pgfpathlineto{\pgfqpoint{5.269336in}{2.806823in}}%
\pgfpathlineto{\pgfqpoint{5.271660in}{2.833232in}}%
\pgfpathlineto{\pgfqpoint{5.273984in}{2.773070in}}%
\pgfpathlineto{\pgfqpoint{5.276308in}{2.857146in}}%
\pgfpathlineto{\pgfqpoint{5.278632in}{2.823597in}}%
\pgfpathlineto{\pgfqpoint{5.280956in}{2.872270in}}%
\pgfpathlineto{\pgfqpoint{5.283280in}{2.870393in}}%
\pgfpathlineto{\pgfqpoint{5.285604in}{2.824841in}}%
\pgfpathlineto{\pgfqpoint{5.290252in}{2.872480in}}%
\pgfpathlineto{\pgfqpoint{5.292576in}{2.837654in}}%
\pgfpathlineto{\pgfqpoint{5.294900in}{2.861903in}}%
\pgfpathlineto{\pgfqpoint{5.297224in}{2.830325in}}%
\pgfpathlineto{\pgfqpoint{5.299548in}{2.512682in}}%
\pgfpathlineto{\pgfqpoint{5.301872in}{2.560825in}}%
\pgfpathlineto{\pgfqpoint{5.304196in}{2.562431in}}%
\pgfpathlineto{\pgfqpoint{5.306519in}{2.555010in}}%
\pgfpathlineto{\pgfqpoint{5.308843in}{2.522215in}}%
\pgfpathlineto{\pgfqpoint{5.311167in}{2.533029in}}%
\pgfpathlineto{\pgfqpoint{5.313491in}{2.550892in}}%
\pgfpathlineto{\pgfqpoint{5.315815in}{2.597096in}}%
\pgfpathlineto{\pgfqpoint{5.318139in}{2.576551in}}%
\pgfpathlineto{\pgfqpoint{5.320463in}{2.584556in}}%
\pgfpathlineto{\pgfqpoint{5.322787in}{2.579042in}}%
\pgfpathlineto{\pgfqpoint{5.325111in}{2.535768in}}%
\pgfpathlineto{\pgfqpoint{5.327435in}{2.551700in}}%
\pgfpathlineto{\pgfqpoint{5.329759in}{2.592884in}}%
\pgfpathlineto{\pgfqpoint{5.332083in}{2.553656in}}%
\pgfpathlineto{\pgfqpoint{5.334407in}{2.490920in}}%
\pgfpathlineto{\pgfqpoint{5.336731in}{2.571045in}}%
\pgfpathlineto{\pgfqpoint{5.339055in}{2.573114in}}%
\pgfpathlineto{\pgfqpoint{5.341379in}{2.612866in}}%
\pgfpathlineto{\pgfqpoint{5.343703in}{2.559500in}}%
\pgfpathlineto{\pgfqpoint{5.346027in}{2.531953in}}%
\pgfpathlineto{\pgfqpoint{5.348351in}{2.562504in}}%
\pgfpathlineto{\pgfqpoint{5.350675in}{2.563111in}}%
\pgfpathlineto{\pgfqpoint{5.352999in}{2.561137in}}%
\pgfpathlineto{\pgfqpoint{5.355323in}{2.593674in}}%
\pgfpathlineto{\pgfqpoint{5.357647in}{2.528208in}}%
\pgfpathlineto{\pgfqpoint{5.359971in}{2.558377in}}%
\pgfpathlineto{\pgfqpoint{5.362295in}{2.561164in}}%
\pgfpathlineto{\pgfqpoint{5.364619in}{2.522700in}}%
\pgfpathlineto{\pgfqpoint{5.366943in}{2.585829in}}%
\pgfpathlineto{\pgfqpoint{5.369267in}{2.557076in}}%
\pgfpathlineto{\pgfqpoint{5.371591in}{2.555085in}}%
\pgfpathlineto{\pgfqpoint{5.373915in}{2.563126in}}%
\pgfpathlineto{\pgfqpoint{5.376239in}{2.561338in}}%
\pgfpathlineto{\pgfqpoint{5.378562in}{2.574248in}}%
\pgfpathlineto{\pgfqpoint{5.380886in}{2.579141in}}%
\pgfpathlineto{\pgfqpoint{5.383210in}{2.558662in}}%
\pgfpathlineto{\pgfqpoint{5.385534in}{2.601875in}}%
\pgfpathlineto{\pgfqpoint{5.387858in}{2.575413in}}%
\pgfpathlineto{\pgfqpoint{5.390182in}{2.571103in}}%
\pgfpathlineto{\pgfqpoint{5.392506in}{2.552275in}}%
\pgfpathlineto{\pgfqpoint{5.394830in}{2.631018in}}%
\pgfpathlineto{\pgfqpoint{5.397154in}{2.593293in}}%
\pgfpathlineto{\pgfqpoint{5.401802in}{2.553406in}}%
\pgfpathlineto{\pgfqpoint{5.404126in}{2.596344in}}%
\pgfpathlineto{\pgfqpoint{5.406450in}{2.539445in}}%
\pgfpathlineto{\pgfqpoint{5.408774in}{2.634292in}}%
\pgfpathlineto{\pgfqpoint{5.411098in}{2.627156in}}%
\pgfpathlineto{\pgfqpoint{5.415746in}{2.511817in}}%
\pgfpathlineto{\pgfqpoint{5.418070in}{2.593382in}}%
\pgfpathlineto{\pgfqpoint{5.420394in}{2.539081in}}%
\pgfpathlineto{\pgfqpoint{5.422718in}{2.571635in}}%
\pgfpathlineto{\pgfqpoint{5.425042in}{2.572774in}}%
\pgfpathlineto{\pgfqpoint{5.427366in}{2.584348in}}%
\pgfpathlineto{\pgfqpoint{5.429690in}{2.612526in}}%
\pgfpathlineto{\pgfqpoint{5.432014in}{2.572554in}}%
\pgfpathlineto{\pgfqpoint{5.434338in}{2.615693in}}%
\pgfpathlineto{\pgfqpoint{5.436662in}{2.530367in}}%
\pgfpathlineto{\pgfqpoint{5.438986in}{2.633266in}}%
\pgfpathlineto{\pgfqpoint{5.441310in}{2.595113in}}%
\pgfpathlineto{\pgfqpoint{5.443634in}{2.579006in}}%
\pgfpathlineto{\pgfqpoint{5.445958in}{2.587340in}}%
\pgfpathlineto{\pgfqpoint{5.448281in}{2.585412in}}%
\pgfpathlineto{\pgfqpoint{5.450605in}{2.631175in}}%
\pgfpathlineto{\pgfqpoint{5.452929in}{2.563297in}}%
\pgfpathlineto{\pgfqpoint{5.455253in}{2.605153in}}%
\pgfpathlineto{\pgfqpoint{5.457577in}{2.564985in}}%
\pgfpathlineto{\pgfqpoint{5.459901in}{2.631429in}}%
\pgfpathlineto{\pgfqpoint{5.462225in}{2.597375in}}%
\pgfpathlineto{\pgfqpoint{5.464549in}{2.590245in}}%
\pgfpathlineto{\pgfqpoint{5.466873in}{2.577873in}}%
\pgfpathlineto{\pgfqpoint{5.469197in}{2.635987in}}%
\pgfpathlineto{\pgfqpoint{5.471521in}{2.595725in}}%
\pgfpathlineto{\pgfqpoint{5.473845in}{2.624807in}}%
\pgfpathlineto{\pgfqpoint{5.476169in}{2.604382in}}%
\pgfpathlineto{\pgfqpoint{5.478493in}{2.603269in}}%
\pgfpathlineto{\pgfqpoint{5.480817in}{2.554695in}}%
\pgfpathlineto{\pgfqpoint{5.485465in}{2.634415in}}%
\pgfpathlineto{\pgfqpoint{5.487789in}{2.589679in}}%
\pgfpathlineto{\pgfqpoint{5.490113in}{2.582649in}}%
\pgfpathlineto{\pgfqpoint{5.492437in}{2.594968in}}%
\pgfpathlineto{\pgfqpoint{5.497085in}{2.561835in}}%
\pgfpathlineto{\pgfqpoint{5.499409in}{2.601584in}}%
\pgfpathlineto{\pgfqpoint{5.501733in}{2.580875in}}%
\pgfpathlineto{\pgfqpoint{5.504057in}{2.594010in}}%
\pgfpathlineto{\pgfqpoint{5.506381in}{2.631116in}}%
\pgfpathlineto{\pgfqpoint{5.511029in}{2.600582in}}%
\pgfpathlineto{\pgfqpoint{5.513353in}{2.610263in}}%
\pgfpathlineto{\pgfqpoint{5.515677in}{2.583577in}}%
\pgfpathlineto{\pgfqpoint{5.518000in}{2.594524in}}%
\pgfpathlineto{\pgfqpoint{5.520324in}{2.641408in}}%
\pgfpathlineto{\pgfqpoint{5.522648in}{2.625746in}}%
\pgfpathlineto{\pgfqpoint{5.524972in}{2.580471in}}%
\pgfpathlineto{\pgfqpoint{5.527296in}{2.590184in}}%
\pgfpathlineto{\pgfqpoint{5.529620in}{2.624572in}}%
\pgfpathlineto{\pgfqpoint{5.531944in}{2.574540in}}%
\pgfpathlineto{\pgfqpoint{5.534268in}{2.613681in}}%
\pgfpathlineto{\pgfqpoint{5.536592in}{2.592370in}}%
\pgfpathlineto{\pgfqpoint{5.538916in}{2.604860in}}%
\pgfpathlineto{\pgfqpoint{5.541240in}{2.586120in}}%
\pgfpathlineto{\pgfqpoint{5.543564in}{2.613305in}}%
\pgfpathlineto{\pgfqpoint{5.545888in}{2.659551in}}%
\pgfpathlineto{\pgfqpoint{5.548212in}{2.611585in}}%
\pgfpathlineto{\pgfqpoint{5.550536in}{2.607857in}}%
\pgfpathlineto{\pgfqpoint{5.552860in}{2.622755in}}%
\pgfpathlineto{\pgfqpoint{5.555184in}{2.613496in}}%
\pgfpathlineto{\pgfqpoint{5.557508in}{2.708468in}}%
\pgfpathlineto{\pgfqpoint{5.559832in}{2.585166in}}%
\pgfpathlineto{\pgfqpoint{5.562156in}{2.590114in}}%
\pgfpathlineto{\pgfqpoint{5.564480in}{2.623279in}}%
\pgfpathlineto{\pgfqpoint{5.566804in}{2.613355in}}%
\pgfpathlineto{\pgfqpoint{5.569128in}{2.623441in}}%
\pgfpathlineto{\pgfqpoint{5.573776in}{2.592059in}}%
\pgfpathlineto{\pgfqpoint{5.576100in}{2.555743in}}%
\pgfpathlineto{\pgfqpoint{5.578424in}{2.636503in}}%
\pgfpathlineto{\pgfqpoint{5.580748in}{2.667239in}}%
\pgfpathlineto{\pgfqpoint{5.583072in}{2.641568in}}%
\pgfpathlineto{\pgfqpoint{5.587720in}{2.545851in}}%
\pgfpathlineto{\pgfqpoint{5.587720in}{2.545851in}}%
\pgfusepath{stroke}%
\end{pgfscope}%
\begin{pgfscope}%
\pgfpathrectangle{\pgfqpoint{0.709829in}{2.192315in}}{\pgfqpoint{5.110171in}{0.887537in}}%
\pgfusepath{clip}%
\pgfsetroundcap%
\pgfsetroundjoin%
\pgfsetlinewidth{1.003750pt}%
\definecolor{currentstroke}{rgb}{0.866667,0.517647,0.321569}%
\pgfsetstrokecolor{currentstroke}%
\pgfsetdash{}{0pt}%
\pgfpathmoveto{\pgfqpoint{0.942110in}{2.760765in}}%
\pgfpathlineto{\pgfqpoint{0.944433in}{2.738792in}}%
\pgfpathlineto{\pgfqpoint{0.946757in}{2.741562in}}%
\pgfpathlineto{\pgfqpoint{0.949081in}{2.765050in}}%
\pgfpathlineto{\pgfqpoint{0.951405in}{2.751268in}}%
\pgfpathlineto{\pgfqpoint{0.953729in}{2.746190in}}%
\pgfpathlineto{\pgfqpoint{0.956053in}{2.702837in}}%
\pgfpathlineto{\pgfqpoint{0.958377in}{2.737607in}}%
\pgfpathlineto{\pgfqpoint{0.960701in}{2.745127in}}%
\pgfpathlineto{\pgfqpoint{0.963025in}{2.771996in}}%
\pgfpathlineto{\pgfqpoint{0.965349in}{2.733875in}}%
\pgfpathlineto{\pgfqpoint{0.969997in}{2.748067in}}%
\pgfpathlineto{\pgfqpoint{0.972321in}{2.720954in}}%
\pgfpathlineto{\pgfqpoint{0.974645in}{2.742813in}}%
\pgfpathlineto{\pgfqpoint{0.976969in}{2.753595in}}%
\pgfpathlineto{\pgfqpoint{0.979293in}{2.730165in}}%
\pgfpathlineto{\pgfqpoint{0.981617in}{2.730221in}}%
\pgfpathlineto{\pgfqpoint{0.983941in}{2.751803in}}%
\pgfpathlineto{\pgfqpoint{0.988589in}{2.735141in}}%
\pgfpathlineto{\pgfqpoint{0.990913in}{2.741186in}}%
\pgfpathlineto{\pgfqpoint{0.993237in}{2.735454in}}%
\pgfpathlineto{\pgfqpoint{0.995561in}{2.756534in}}%
\pgfpathlineto{\pgfqpoint{0.997885in}{2.756411in}}%
\pgfpathlineto{\pgfqpoint{1.000209in}{2.739156in}}%
\pgfpathlineto{\pgfqpoint{1.002533in}{2.735309in}}%
\pgfpathlineto{\pgfqpoint{1.004857in}{2.734793in}}%
\pgfpathlineto{\pgfqpoint{1.007181in}{2.782560in}}%
\pgfpathlineto{\pgfqpoint{1.009505in}{2.765961in}}%
\pgfpathlineto{\pgfqpoint{1.011829in}{2.761896in}}%
\pgfpathlineto{\pgfqpoint{1.014153in}{2.769963in}}%
\pgfpathlineto{\pgfqpoint{1.016476in}{2.764799in}}%
\pgfpathlineto{\pgfqpoint{1.018800in}{2.744643in}}%
\pgfpathlineto{\pgfqpoint{1.021124in}{2.740188in}}%
\pgfpathlineto{\pgfqpoint{1.023448in}{2.749749in}}%
\pgfpathlineto{\pgfqpoint{1.025772in}{2.796265in}}%
\pgfpathlineto{\pgfqpoint{1.030420in}{2.775587in}}%
\pgfpathlineto{\pgfqpoint{1.032744in}{2.775274in}}%
\pgfpathlineto{\pgfqpoint{1.035068in}{2.761343in}}%
\pgfpathlineto{\pgfqpoint{1.037392in}{2.759014in}}%
\pgfpathlineto{\pgfqpoint{1.039716in}{2.776955in}}%
\pgfpathlineto{\pgfqpoint{1.042040in}{2.767707in}}%
\pgfpathlineto{\pgfqpoint{1.044364in}{2.751024in}}%
\pgfpathlineto{\pgfqpoint{1.046688in}{2.763251in}}%
\pgfpathlineto{\pgfqpoint{1.049012in}{2.737661in}}%
\pgfpathlineto{\pgfqpoint{1.051336in}{2.774666in}}%
\pgfpathlineto{\pgfqpoint{1.053660in}{2.759787in}}%
\pgfpathlineto{\pgfqpoint{1.055984in}{2.755396in}}%
\pgfpathlineto{\pgfqpoint{1.058308in}{2.760104in}}%
\pgfpathlineto{\pgfqpoint{1.062956in}{2.747528in}}%
\pgfpathlineto{\pgfqpoint{1.065280in}{2.792194in}}%
\pgfpathlineto{\pgfqpoint{1.067604in}{2.751270in}}%
\pgfpathlineto{\pgfqpoint{1.069928in}{2.742650in}}%
\pgfpathlineto{\pgfqpoint{1.072252in}{2.785661in}}%
\pgfpathlineto{\pgfqpoint{1.074576in}{2.762598in}}%
\pgfpathlineto{\pgfqpoint{1.076900in}{2.756814in}}%
\pgfpathlineto{\pgfqpoint{1.079224in}{2.766107in}}%
\pgfpathlineto{\pgfqpoint{1.083872in}{2.743199in}}%
\pgfpathlineto{\pgfqpoint{1.088519in}{2.774080in}}%
\pgfpathlineto{\pgfqpoint{1.090843in}{2.774942in}}%
\pgfpathlineto{\pgfqpoint{1.093167in}{2.786841in}}%
\pgfpathlineto{\pgfqpoint{1.095491in}{2.777406in}}%
\pgfpathlineto{\pgfqpoint{1.097815in}{2.786338in}}%
\pgfpathlineto{\pgfqpoint{1.100139in}{2.768653in}}%
\pgfpathlineto{\pgfqpoint{1.102463in}{2.771744in}}%
\pgfpathlineto{\pgfqpoint{1.104787in}{2.789592in}}%
\pgfpathlineto{\pgfqpoint{1.107111in}{2.764106in}}%
\pgfpathlineto{\pgfqpoint{1.109435in}{2.810425in}}%
\pgfpathlineto{\pgfqpoint{1.111759in}{2.766993in}}%
\pgfpathlineto{\pgfqpoint{1.114083in}{2.754007in}}%
\pgfpathlineto{\pgfqpoint{1.116407in}{2.780971in}}%
\pgfpathlineto{\pgfqpoint{1.118731in}{2.753644in}}%
\pgfpathlineto{\pgfqpoint{1.123379in}{2.800508in}}%
\pgfpathlineto{\pgfqpoint{1.130351in}{2.752659in}}%
\pgfpathlineto{\pgfqpoint{1.132675in}{2.774753in}}%
\pgfpathlineto{\pgfqpoint{1.134999in}{2.770204in}}%
\pgfpathlineto{\pgfqpoint{1.137323in}{2.790354in}}%
\pgfpathlineto{\pgfqpoint{1.139647in}{2.759723in}}%
\pgfpathlineto{\pgfqpoint{1.141971in}{2.762478in}}%
\pgfpathlineto{\pgfqpoint{1.146619in}{2.786454in}}%
\pgfpathlineto{\pgfqpoint{1.148943in}{2.783660in}}%
\pgfpathlineto{\pgfqpoint{1.151267in}{2.803335in}}%
\pgfpathlineto{\pgfqpoint{1.155914in}{2.755018in}}%
\pgfpathlineto{\pgfqpoint{1.158238in}{2.804819in}}%
\pgfpathlineto{\pgfqpoint{1.160562in}{2.793718in}}%
\pgfpathlineto{\pgfqpoint{1.162886in}{2.801462in}}%
\pgfpathlineto{\pgfqpoint{1.165210in}{2.783500in}}%
\pgfpathlineto{\pgfqpoint{1.167534in}{2.778381in}}%
\pgfpathlineto{\pgfqpoint{1.169858in}{2.792546in}}%
\pgfpathlineto{\pgfqpoint{1.172182in}{2.793050in}}%
\pgfpathlineto{\pgfqpoint{1.174506in}{2.789279in}}%
\pgfpathlineto{\pgfqpoint{1.176830in}{2.782433in}}%
\pgfpathlineto{\pgfqpoint{1.179154in}{2.790910in}}%
\pgfpathlineto{\pgfqpoint{1.181478in}{2.804511in}}%
\pgfpathlineto{\pgfqpoint{1.183802in}{2.778526in}}%
\pgfpathlineto{\pgfqpoint{1.186126in}{2.786009in}}%
\pgfpathlineto{\pgfqpoint{1.188450in}{2.779363in}}%
\pgfpathlineto{\pgfqpoint{1.190774in}{2.776822in}}%
\pgfpathlineto{\pgfqpoint{1.193098in}{2.785129in}}%
\pgfpathlineto{\pgfqpoint{1.195422in}{2.808899in}}%
\pgfpathlineto{\pgfqpoint{1.197746in}{2.801257in}}%
\pgfpathlineto{\pgfqpoint{1.202394in}{2.780377in}}%
\pgfpathlineto{\pgfqpoint{1.204718in}{2.794966in}}%
\pgfpathlineto{\pgfqpoint{1.207042in}{2.781228in}}%
\pgfpathlineto{\pgfqpoint{1.209366in}{2.794745in}}%
\pgfpathlineto{\pgfqpoint{1.211690in}{2.786236in}}%
\pgfpathlineto{\pgfqpoint{1.214014in}{2.786609in}}%
\pgfpathlineto{\pgfqpoint{1.216338in}{2.764693in}}%
\pgfpathlineto{\pgfqpoint{1.218662in}{2.802405in}}%
\pgfpathlineto{\pgfqpoint{1.220986in}{2.812172in}}%
\pgfpathlineto{\pgfqpoint{1.227957in}{2.781258in}}%
\pgfpathlineto{\pgfqpoint{1.230281in}{2.800068in}}%
\pgfpathlineto{\pgfqpoint{1.232605in}{2.480769in}}%
\pgfpathlineto{\pgfqpoint{1.234929in}{2.478438in}}%
\pgfpathlineto{\pgfqpoint{1.237253in}{2.478609in}}%
\pgfpathlineto{\pgfqpoint{1.239577in}{2.480952in}}%
\pgfpathlineto{\pgfqpoint{1.241901in}{2.507168in}}%
\pgfpathlineto{\pgfqpoint{1.246549in}{2.486538in}}%
\pgfpathlineto{\pgfqpoint{1.248873in}{2.531610in}}%
\pgfpathlineto{\pgfqpoint{1.251197in}{2.486980in}}%
\pgfpathlineto{\pgfqpoint{1.253521in}{2.481020in}}%
\pgfpathlineto{\pgfqpoint{1.255845in}{2.509355in}}%
\pgfpathlineto{\pgfqpoint{1.258169in}{2.479895in}}%
\pgfpathlineto{\pgfqpoint{1.265141in}{2.537786in}}%
\pgfpathlineto{\pgfqpoint{1.267465in}{2.479752in}}%
\pgfpathlineto{\pgfqpoint{1.269789in}{2.508096in}}%
\pgfpathlineto{\pgfqpoint{1.272113in}{2.490822in}}%
\pgfpathlineto{\pgfqpoint{1.274437in}{2.485443in}}%
\pgfpathlineto{\pgfqpoint{1.279085in}{2.512343in}}%
\pgfpathlineto{\pgfqpoint{1.281409in}{2.505159in}}%
\pgfpathlineto{\pgfqpoint{1.283733in}{2.508879in}}%
\pgfpathlineto{\pgfqpoint{1.286057in}{2.518078in}}%
\pgfpathlineto{\pgfqpoint{1.288381in}{2.502953in}}%
\pgfpathlineto{\pgfqpoint{1.290705in}{2.517664in}}%
\pgfpathlineto{\pgfqpoint{1.293029in}{2.542731in}}%
\pgfpathlineto{\pgfqpoint{1.295353in}{2.504579in}}%
\pgfpathlineto{\pgfqpoint{1.297676in}{2.494737in}}%
\pgfpathlineto{\pgfqpoint{1.300000in}{2.503862in}}%
\pgfpathlineto{\pgfqpoint{1.302324in}{2.502640in}}%
\pgfpathlineto{\pgfqpoint{1.304648in}{2.477743in}}%
\pgfpathlineto{\pgfqpoint{1.306972in}{2.499908in}}%
\pgfpathlineto{\pgfqpoint{1.309296in}{2.540880in}}%
\pgfpathlineto{\pgfqpoint{1.313944in}{2.506143in}}%
\pgfpathlineto{\pgfqpoint{1.316268in}{2.476896in}}%
\pgfpathlineto{\pgfqpoint{1.318592in}{2.519756in}}%
\pgfpathlineto{\pgfqpoint{1.320916in}{2.506451in}}%
\pgfpathlineto{\pgfqpoint{1.323240in}{2.563100in}}%
\pgfpathlineto{\pgfqpoint{1.325564in}{2.505372in}}%
\pgfpathlineto{\pgfqpoint{1.327888in}{2.518942in}}%
\pgfpathlineto{\pgfqpoint{1.330212in}{2.516922in}}%
\pgfpathlineto{\pgfqpoint{1.332536in}{2.489362in}}%
\pgfpathlineto{\pgfqpoint{1.334860in}{2.528183in}}%
\pgfpathlineto{\pgfqpoint{1.337184in}{2.504925in}}%
\pgfpathlineto{\pgfqpoint{1.339508in}{2.513051in}}%
\pgfpathlineto{\pgfqpoint{1.341832in}{2.527491in}}%
\pgfpathlineto{\pgfqpoint{1.344156in}{2.508282in}}%
\pgfpathlineto{\pgfqpoint{1.346480in}{2.514901in}}%
\pgfpathlineto{\pgfqpoint{1.348804in}{2.541476in}}%
\pgfpathlineto{\pgfqpoint{1.351128in}{2.483466in}}%
\pgfpathlineto{\pgfqpoint{1.353452in}{2.526318in}}%
\pgfpathlineto{\pgfqpoint{1.355776in}{2.510850in}}%
\pgfpathlineto{\pgfqpoint{1.358100in}{2.517577in}}%
\pgfpathlineto{\pgfqpoint{1.360424in}{2.513136in}}%
\pgfpathlineto{\pgfqpoint{1.362748in}{2.516758in}}%
\pgfpathlineto{\pgfqpoint{1.365072in}{2.532060in}}%
\pgfpathlineto{\pgfqpoint{1.369719in}{2.513255in}}%
\pgfpathlineto{\pgfqpoint{1.372043in}{2.516991in}}%
\pgfpathlineto{\pgfqpoint{1.374367in}{2.546475in}}%
\pgfpathlineto{\pgfqpoint{1.376691in}{2.497066in}}%
\pgfpathlineto{\pgfqpoint{1.379015in}{2.527537in}}%
\pgfpathlineto{\pgfqpoint{1.381339in}{2.533026in}}%
\pgfpathlineto{\pgfqpoint{1.383663in}{2.518736in}}%
\pgfpathlineto{\pgfqpoint{1.385987in}{2.522923in}}%
\pgfpathlineto{\pgfqpoint{1.390635in}{2.510489in}}%
\pgfpathlineto{\pgfqpoint{1.392959in}{2.517037in}}%
\pgfpathlineto{\pgfqpoint{1.395283in}{2.530128in}}%
\pgfpathlineto{\pgfqpoint{1.399931in}{2.510167in}}%
\pgfpathlineto{\pgfqpoint{1.402255in}{2.521647in}}%
\pgfpathlineto{\pgfqpoint{1.404579in}{2.519974in}}%
\pgfpathlineto{\pgfqpoint{1.406903in}{2.510628in}}%
\pgfpathlineto{\pgfqpoint{1.409227in}{2.513766in}}%
\pgfpathlineto{\pgfqpoint{1.411551in}{2.522753in}}%
\pgfpathlineto{\pgfqpoint{1.413875in}{2.500938in}}%
\pgfpathlineto{\pgfqpoint{1.416199in}{2.531004in}}%
\pgfpathlineto{\pgfqpoint{1.418523in}{2.535156in}}%
\pgfpathlineto{\pgfqpoint{1.420847in}{2.536287in}}%
\pgfpathlineto{\pgfqpoint{1.423171in}{2.522690in}}%
\pgfpathlineto{\pgfqpoint{1.425495in}{2.526467in}}%
\pgfpathlineto{\pgfqpoint{1.427819in}{2.533424in}}%
\pgfpathlineto{\pgfqpoint{1.430143in}{2.534634in}}%
\pgfpathlineto{\pgfqpoint{1.432467in}{2.533493in}}%
\pgfpathlineto{\pgfqpoint{1.434791in}{2.541285in}}%
\pgfpathlineto{\pgfqpoint{1.437114in}{2.554040in}}%
\pgfpathlineto{\pgfqpoint{1.439438in}{2.511581in}}%
\pgfpathlineto{\pgfqpoint{1.441762in}{2.533226in}}%
\pgfpathlineto{\pgfqpoint{1.444086in}{2.540084in}}%
\pgfpathlineto{\pgfqpoint{1.446410in}{2.534957in}}%
\pgfpathlineto{\pgfqpoint{1.448734in}{2.533239in}}%
\pgfpathlineto{\pgfqpoint{1.451058in}{2.513534in}}%
\pgfpathlineto{\pgfqpoint{1.453382in}{2.532235in}}%
\pgfpathlineto{\pgfqpoint{1.460354in}{2.504106in}}%
\pgfpathlineto{\pgfqpoint{1.465002in}{2.546426in}}%
\pgfpathlineto{\pgfqpoint{1.467326in}{2.530256in}}%
\pgfpathlineto{\pgfqpoint{1.469650in}{2.552862in}}%
\pgfpathlineto{\pgfqpoint{1.471974in}{2.553878in}}%
\pgfpathlineto{\pgfqpoint{1.474298in}{2.566211in}}%
\pgfpathlineto{\pgfqpoint{1.476622in}{2.517067in}}%
\pgfpathlineto{\pgfqpoint{1.478946in}{2.545207in}}%
\pgfpathlineto{\pgfqpoint{1.481270in}{2.534690in}}%
\pgfpathlineto{\pgfqpoint{1.483594in}{2.515021in}}%
\pgfpathlineto{\pgfqpoint{1.485918in}{2.541724in}}%
\pgfpathlineto{\pgfqpoint{1.488242in}{2.544543in}}%
\pgfpathlineto{\pgfqpoint{1.490566in}{2.529944in}}%
\pgfpathlineto{\pgfqpoint{1.492890in}{2.531513in}}%
\pgfpathlineto{\pgfqpoint{1.495214in}{2.527936in}}%
\pgfpathlineto{\pgfqpoint{1.499862in}{2.537357in}}%
\pgfpathlineto{\pgfqpoint{1.502186in}{2.517534in}}%
\pgfpathlineto{\pgfqpoint{1.504510in}{2.552424in}}%
\pgfpathlineto{\pgfqpoint{1.506834in}{2.525567in}}%
\pgfpathlineto{\pgfqpoint{1.509157in}{2.570421in}}%
\pgfpathlineto{\pgfqpoint{1.511481in}{2.516071in}}%
\pgfpathlineto{\pgfqpoint{1.513805in}{2.517339in}}%
\pgfpathlineto{\pgfqpoint{1.516129in}{2.540870in}}%
\pgfpathlineto{\pgfqpoint{1.520777in}{2.538971in}}%
\pgfpathlineto{\pgfqpoint{1.523101in}{2.765102in}}%
\pgfpathlineto{\pgfqpoint{1.525425in}{2.779334in}}%
\pgfpathlineto{\pgfqpoint{1.527749in}{2.762777in}}%
\pgfpathlineto{\pgfqpoint{1.530073in}{2.761809in}}%
\pgfpathlineto{\pgfqpoint{1.532397in}{2.776218in}}%
\pgfpathlineto{\pgfqpoint{1.534721in}{2.761016in}}%
\pgfpathlineto{\pgfqpoint{1.537045in}{2.760352in}}%
\pgfpathlineto{\pgfqpoint{1.539369in}{2.763723in}}%
\pgfpathlineto{\pgfqpoint{1.541693in}{2.755960in}}%
\pgfpathlineto{\pgfqpoint{1.544017in}{2.756752in}}%
\pgfpathlineto{\pgfqpoint{1.546341in}{2.779607in}}%
\pgfpathlineto{\pgfqpoint{1.550989in}{2.747894in}}%
\pgfpathlineto{\pgfqpoint{1.553313in}{2.761778in}}%
\pgfpathlineto{\pgfqpoint{1.557961in}{2.747882in}}%
\pgfpathlineto{\pgfqpoint{1.560285in}{2.782934in}}%
\pgfpathlineto{\pgfqpoint{1.562609in}{2.762149in}}%
\pgfpathlineto{\pgfqpoint{1.564933in}{2.777713in}}%
\pgfpathlineto{\pgfqpoint{1.567257in}{2.758571in}}%
\pgfpathlineto{\pgfqpoint{1.569581in}{2.783025in}}%
\pgfpathlineto{\pgfqpoint{1.571905in}{2.793033in}}%
\pgfpathlineto{\pgfqpoint{1.574229in}{2.759175in}}%
\pgfpathlineto{\pgfqpoint{1.576553in}{2.767589in}}%
\pgfpathlineto{\pgfqpoint{1.578876in}{2.753357in}}%
\pgfpathlineto{\pgfqpoint{1.581200in}{2.802990in}}%
\pgfpathlineto{\pgfqpoint{1.583524in}{2.774262in}}%
\pgfpathlineto{\pgfqpoint{1.585848in}{2.769672in}}%
\pgfpathlineto{\pgfqpoint{1.588172in}{2.759059in}}%
\pgfpathlineto{\pgfqpoint{1.592820in}{2.782374in}}%
\pgfpathlineto{\pgfqpoint{1.595144in}{2.772757in}}%
\pgfpathlineto{\pgfqpoint{1.599792in}{2.742400in}}%
\pgfpathlineto{\pgfqpoint{1.604440in}{2.756261in}}%
\pgfpathlineto{\pgfqpoint{1.606764in}{2.785701in}}%
\pgfpathlineto{\pgfqpoint{1.609088in}{2.771879in}}%
\pgfpathlineto{\pgfqpoint{1.611412in}{2.765097in}}%
\pgfpathlineto{\pgfqpoint{1.613736in}{2.778748in}}%
\pgfpathlineto{\pgfqpoint{1.618384in}{2.755208in}}%
\pgfpathlineto{\pgfqpoint{1.620708in}{2.767715in}}%
\pgfpathlineto{\pgfqpoint{1.623032in}{2.757071in}}%
\pgfpathlineto{\pgfqpoint{1.625356in}{2.759043in}}%
\pgfpathlineto{\pgfqpoint{1.627680in}{2.764705in}}%
\pgfpathlineto{\pgfqpoint{1.630004in}{2.759387in}}%
\pgfpathlineto{\pgfqpoint{1.632328in}{2.773571in}}%
\pgfpathlineto{\pgfqpoint{1.634652in}{2.762916in}}%
\pgfpathlineto{\pgfqpoint{1.636976in}{2.771760in}}%
\pgfpathlineto{\pgfqpoint{1.639300in}{2.771312in}}%
\pgfpathlineto{\pgfqpoint{1.641624in}{2.796590in}}%
\pgfpathlineto{\pgfqpoint{1.643948in}{2.780894in}}%
\pgfpathlineto{\pgfqpoint{1.646272in}{2.751781in}}%
\pgfpathlineto{\pgfqpoint{1.648595in}{2.749186in}}%
\pgfpathlineto{\pgfqpoint{1.650919in}{2.771001in}}%
\pgfpathlineto{\pgfqpoint{1.653243in}{2.753850in}}%
\pgfpathlineto{\pgfqpoint{1.655567in}{2.784231in}}%
\pgfpathlineto{\pgfqpoint{1.657891in}{2.777973in}}%
\pgfpathlineto{\pgfqpoint{1.660215in}{2.796005in}}%
\pgfpathlineto{\pgfqpoint{1.662539in}{2.781043in}}%
\pgfpathlineto{\pgfqpoint{1.664863in}{2.756552in}}%
\pgfpathlineto{\pgfqpoint{1.667187in}{2.765805in}}%
\pgfpathlineto{\pgfqpoint{1.669511in}{2.793329in}}%
\pgfpathlineto{\pgfqpoint{1.671835in}{2.784491in}}%
\pgfpathlineto{\pgfqpoint{1.674159in}{2.797264in}}%
\pgfpathlineto{\pgfqpoint{1.676483in}{2.761039in}}%
\pgfpathlineto{\pgfqpoint{1.681131in}{2.789717in}}%
\pgfpathlineto{\pgfqpoint{1.683455in}{2.762518in}}%
\pgfpathlineto{\pgfqpoint{1.685779in}{2.781936in}}%
\pgfpathlineto{\pgfqpoint{1.688103in}{2.774097in}}%
\pgfpathlineto{\pgfqpoint{1.690427in}{2.791371in}}%
\pgfpathlineto{\pgfqpoint{1.692751in}{2.789230in}}%
\pgfpathlineto{\pgfqpoint{1.695075in}{2.811212in}}%
\pgfpathlineto{\pgfqpoint{1.697399in}{2.760405in}}%
\pgfpathlineto{\pgfqpoint{1.702047in}{2.789975in}}%
\pgfpathlineto{\pgfqpoint{1.704371in}{2.769217in}}%
\pgfpathlineto{\pgfqpoint{1.706695in}{2.778544in}}%
\pgfpathlineto{\pgfqpoint{1.709019in}{2.766051in}}%
\pgfpathlineto{\pgfqpoint{1.711343in}{2.784165in}}%
\pgfpathlineto{\pgfqpoint{1.713667in}{2.759223in}}%
\pgfpathlineto{\pgfqpoint{1.715991in}{2.791090in}}%
\pgfpathlineto{\pgfqpoint{1.718314in}{2.779462in}}%
\pgfpathlineto{\pgfqpoint{1.720638in}{2.752266in}}%
\pgfpathlineto{\pgfqpoint{1.722962in}{2.762826in}}%
\pgfpathlineto{\pgfqpoint{1.725286in}{2.805109in}}%
\pgfpathlineto{\pgfqpoint{1.727610in}{2.775194in}}%
\pgfpathlineto{\pgfqpoint{1.732258in}{2.776355in}}%
\pgfpathlineto{\pgfqpoint{1.734582in}{2.769482in}}%
\pgfpathlineto{\pgfqpoint{1.736906in}{2.787690in}}%
\pgfpathlineto{\pgfqpoint{1.739230in}{2.775749in}}%
\pgfpathlineto{\pgfqpoint{1.743878in}{2.795788in}}%
\pgfpathlineto{\pgfqpoint{1.748526in}{2.797914in}}%
\pgfpathlineto{\pgfqpoint{1.750850in}{2.789979in}}%
\pgfpathlineto{\pgfqpoint{1.753174in}{2.764422in}}%
\pgfpathlineto{\pgfqpoint{1.755498in}{2.771167in}}%
\pgfpathlineto{\pgfqpoint{1.757822in}{2.796037in}}%
\pgfpathlineto{\pgfqpoint{1.762470in}{2.759631in}}%
\pgfpathlineto{\pgfqpoint{1.764794in}{2.780754in}}%
\pgfpathlineto{\pgfqpoint{1.767118in}{2.780620in}}%
\pgfpathlineto{\pgfqpoint{1.771766in}{2.754302in}}%
\pgfpathlineto{\pgfqpoint{1.774090in}{2.792303in}}%
\pgfpathlineto{\pgfqpoint{1.776414in}{2.778712in}}%
\pgfpathlineto{\pgfqpoint{1.778738in}{2.786458in}}%
\pgfpathlineto{\pgfqpoint{1.781062in}{2.763988in}}%
\pgfpathlineto{\pgfqpoint{1.783386in}{2.784269in}}%
\pgfpathlineto{\pgfqpoint{1.785710in}{2.773451in}}%
\pgfpathlineto{\pgfqpoint{1.788034in}{2.805422in}}%
\pgfpathlineto{\pgfqpoint{1.792681in}{2.767275in}}%
\pgfpathlineto{\pgfqpoint{1.797329in}{2.782065in}}%
\pgfpathlineto{\pgfqpoint{1.799653in}{2.794204in}}%
\pgfpathlineto{\pgfqpoint{1.801977in}{2.795883in}}%
\pgfpathlineto{\pgfqpoint{1.804301in}{2.813096in}}%
\pgfpathlineto{\pgfqpoint{1.806625in}{2.801771in}}%
\pgfpathlineto{\pgfqpoint{1.808949in}{2.799914in}}%
\pgfpathlineto{\pgfqpoint{1.811273in}{2.790214in}}%
\pgfpathlineto{\pgfqpoint{1.813597in}{2.485917in}}%
\pgfpathlineto{\pgfqpoint{1.815921in}{2.458108in}}%
\pgfpathlineto{\pgfqpoint{1.818245in}{2.494627in}}%
\pgfpathlineto{\pgfqpoint{1.820569in}{2.485507in}}%
\pgfpathlineto{\pgfqpoint{1.825217in}{2.498026in}}%
\pgfpathlineto{\pgfqpoint{1.827541in}{2.479904in}}%
\pgfpathlineto{\pgfqpoint{1.829865in}{2.501304in}}%
\pgfpathlineto{\pgfqpoint{1.832189in}{2.460714in}}%
\pgfpathlineto{\pgfqpoint{1.834513in}{2.490296in}}%
\pgfpathlineto{\pgfqpoint{1.836837in}{2.492439in}}%
\pgfpathlineto{\pgfqpoint{1.839161in}{2.475889in}}%
\pgfpathlineto{\pgfqpoint{1.841485in}{2.501675in}}%
\pgfpathlineto{\pgfqpoint{1.843809in}{2.512237in}}%
\pgfpathlineto{\pgfqpoint{1.846133in}{2.488947in}}%
\pgfpathlineto{\pgfqpoint{1.848457in}{2.492470in}}%
\pgfpathlineto{\pgfqpoint{1.850781in}{2.490341in}}%
\pgfpathlineto{\pgfqpoint{1.853105in}{2.477697in}}%
\pgfpathlineto{\pgfqpoint{1.855429in}{2.494531in}}%
\pgfpathlineto{\pgfqpoint{1.857753in}{2.454990in}}%
\pgfpathlineto{\pgfqpoint{1.860076in}{2.501589in}}%
\pgfpathlineto{\pgfqpoint{1.862400in}{2.482146in}}%
\pgfpathlineto{\pgfqpoint{1.864724in}{2.480090in}}%
\pgfpathlineto{\pgfqpoint{1.867048in}{2.493279in}}%
\pgfpathlineto{\pgfqpoint{1.869372in}{2.485558in}}%
\pgfpathlineto{\pgfqpoint{1.871696in}{2.499930in}}%
\pgfpathlineto{\pgfqpoint{1.874020in}{2.465212in}}%
\pgfpathlineto{\pgfqpoint{1.876344in}{2.504258in}}%
\pgfpathlineto{\pgfqpoint{1.880992in}{2.483307in}}%
\pgfpathlineto{\pgfqpoint{1.883316in}{2.497826in}}%
\pgfpathlineto{\pgfqpoint{1.885640in}{2.485454in}}%
\pgfpathlineto{\pgfqpoint{1.887964in}{2.486406in}}%
\pgfpathlineto{\pgfqpoint{1.890288in}{2.494129in}}%
\pgfpathlineto{\pgfqpoint{1.894936in}{2.479685in}}%
\pgfpathlineto{\pgfqpoint{1.897260in}{2.462295in}}%
\pgfpathlineto{\pgfqpoint{1.901908in}{2.494989in}}%
\pgfpathlineto{\pgfqpoint{1.904232in}{2.496336in}}%
\pgfpathlineto{\pgfqpoint{1.906556in}{2.485941in}}%
\pgfpathlineto{\pgfqpoint{1.908880in}{2.468682in}}%
\pgfpathlineto{\pgfqpoint{1.911204in}{2.501184in}}%
\pgfpathlineto{\pgfqpoint{1.913528in}{2.440806in}}%
\pgfpathlineto{\pgfqpoint{1.915852in}{2.481825in}}%
\pgfpathlineto{\pgfqpoint{1.918176in}{2.477828in}}%
\pgfpathlineto{\pgfqpoint{1.920500in}{2.500331in}}%
\pgfpathlineto{\pgfqpoint{1.922824in}{2.471523in}}%
\pgfpathlineto{\pgfqpoint{1.925148in}{2.506058in}}%
\pgfpathlineto{\pgfqpoint{1.927472in}{2.495614in}}%
\pgfpathlineto{\pgfqpoint{1.929795in}{2.465233in}}%
\pgfpathlineto{\pgfqpoint{1.934443in}{2.498564in}}%
\pgfpathlineto{\pgfqpoint{1.936767in}{2.490296in}}%
\pgfpathlineto{\pgfqpoint{1.939091in}{2.511516in}}%
\pgfpathlineto{\pgfqpoint{1.941415in}{2.489485in}}%
\pgfpathlineto{\pgfqpoint{1.943739in}{2.484521in}}%
\pgfpathlineto{\pgfqpoint{1.946063in}{2.493352in}}%
\pgfpathlineto{\pgfqpoint{1.948387in}{2.472435in}}%
\pgfpathlineto{\pgfqpoint{1.953035in}{2.477846in}}%
\pgfpathlineto{\pgfqpoint{1.955359in}{2.498790in}}%
\pgfpathlineto{\pgfqpoint{1.957683in}{2.482273in}}%
\pgfpathlineto{\pgfqpoint{1.960007in}{2.478294in}}%
\pgfpathlineto{\pgfqpoint{1.962331in}{2.469223in}}%
\pgfpathlineto{\pgfqpoint{1.964655in}{2.480980in}}%
\pgfpathlineto{\pgfqpoint{1.966979in}{2.483923in}}%
\pgfpathlineto{\pgfqpoint{1.969303in}{2.510663in}}%
\pgfpathlineto{\pgfqpoint{1.971627in}{2.482805in}}%
\pgfpathlineto{\pgfqpoint{1.973951in}{2.482339in}}%
\pgfpathlineto{\pgfqpoint{1.976275in}{2.505507in}}%
\pgfpathlineto{\pgfqpoint{1.978599in}{2.486637in}}%
\pgfpathlineto{\pgfqpoint{1.980923in}{2.481875in}}%
\pgfpathlineto{\pgfqpoint{1.983247in}{2.498503in}}%
\pgfpathlineto{\pgfqpoint{1.985571in}{2.499321in}}%
\pgfpathlineto{\pgfqpoint{1.987895in}{2.507138in}}%
\pgfpathlineto{\pgfqpoint{1.990219in}{2.500392in}}%
\pgfpathlineto{\pgfqpoint{1.992543in}{2.461885in}}%
\pgfpathlineto{\pgfqpoint{1.994867in}{2.513765in}}%
\pgfpathlineto{\pgfqpoint{1.997191in}{2.474677in}}%
\pgfpathlineto{\pgfqpoint{2.001838in}{2.508417in}}%
\pgfpathlineto{\pgfqpoint{2.004162in}{2.518058in}}%
\pgfpathlineto{\pgfqpoint{2.006486in}{2.467153in}}%
\pgfpathlineto{\pgfqpoint{2.011134in}{2.509082in}}%
\pgfpathlineto{\pgfqpoint{2.013458in}{2.515384in}}%
\pgfpathlineto{\pgfqpoint{2.015782in}{2.484352in}}%
\pgfpathlineto{\pgfqpoint{2.020430in}{2.496947in}}%
\pgfpathlineto{\pgfqpoint{2.022754in}{2.462020in}}%
\pgfpathlineto{\pgfqpoint{2.027402in}{2.519703in}}%
\pgfpathlineto{\pgfqpoint{2.029726in}{2.505862in}}%
\pgfpathlineto{\pgfqpoint{2.032050in}{2.501721in}}%
\pgfpathlineto{\pgfqpoint{2.039022in}{2.451362in}}%
\pgfpathlineto{\pgfqpoint{2.045994in}{2.511937in}}%
\pgfpathlineto{\pgfqpoint{2.050642in}{2.491133in}}%
\pgfpathlineto{\pgfqpoint{2.052966in}{2.509604in}}%
\pgfpathlineto{\pgfqpoint{2.055290in}{2.511848in}}%
\pgfpathlineto{\pgfqpoint{2.057614in}{2.505314in}}%
\pgfpathlineto{\pgfqpoint{2.059938in}{2.509933in}}%
\pgfpathlineto{\pgfqpoint{2.062262in}{2.518690in}}%
\pgfpathlineto{\pgfqpoint{2.066910in}{2.499420in}}%
\pgfpathlineto{\pgfqpoint{2.069234in}{2.475742in}}%
\pgfpathlineto{\pgfqpoint{2.071557in}{2.515289in}}%
\pgfpathlineto{\pgfqpoint{2.073881in}{2.490280in}}%
\pgfpathlineto{\pgfqpoint{2.076205in}{2.495567in}}%
\pgfpathlineto{\pgfqpoint{2.078529in}{2.503690in}}%
\pgfpathlineto{\pgfqpoint{2.080853in}{2.502467in}}%
\pgfpathlineto{\pgfqpoint{2.083177in}{2.491514in}}%
\pgfpathlineto{\pgfqpoint{2.085501in}{2.505176in}}%
\pgfpathlineto{\pgfqpoint{2.087825in}{2.488781in}}%
\pgfpathlineto{\pgfqpoint{2.092473in}{2.516584in}}%
\pgfpathlineto{\pgfqpoint{2.094797in}{2.489005in}}%
\pgfpathlineto{\pgfqpoint{2.097121in}{2.501360in}}%
\pgfpathlineto{\pgfqpoint{2.099445in}{2.494445in}}%
\pgfpathlineto{\pgfqpoint{2.101769in}{2.492189in}}%
\pgfpathlineto{\pgfqpoint{2.104093in}{2.726366in}}%
\pgfpathlineto{\pgfqpoint{2.106417in}{2.725269in}}%
\pgfpathlineto{\pgfqpoint{2.108741in}{2.738849in}}%
\pgfpathlineto{\pgfqpoint{2.111065in}{2.735267in}}%
\pgfpathlineto{\pgfqpoint{2.113389in}{2.726058in}}%
\pgfpathlineto{\pgfqpoint{2.115713in}{2.722227in}}%
\pgfpathlineto{\pgfqpoint{2.118037in}{2.742721in}}%
\pgfpathlineto{\pgfqpoint{2.120361in}{2.748126in}}%
\pgfpathlineto{\pgfqpoint{2.122685in}{2.730028in}}%
\pgfpathlineto{\pgfqpoint{2.125009in}{2.722780in}}%
\pgfpathlineto{\pgfqpoint{2.127333in}{2.704707in}}%
\pgfpathlineto{\pgfqpoint{2.129657in}{2.706081in}}%
\pgfpathlineto{\pgfqpoint{2.131981in}{2.743922in}}%
\pgfpathlineto{\pgfqpoint{2.136629in}{2.713559in}}%
\pgfpathlineto{\pgfqpoint{2.141276in}{2.732330in}}%
\pgfpathlineto{\pgfqpoint{2.143600in}{2.736124in}}%
\pgfpathlineto{\pgfqpoint{2.148248in}{2.725736in}}%
\pgfpathlineto{\pgfqpoint{2.150572in}{2.710026in}}%
\pgfpathlineto{\pgfqpoint{2.152896in}{2.728948in}}%
\pgfpathlineto{\pgfqpoint{2.155220in}{2.738681in}}%
\pgfpathlineto{\pgfqpoint{2.157544in}{2.721424in}}%
\pgfpathlineto{\pgfqpoint{2.159868in}{2.744933in}}%
\pgfpathlineto{\pgfqpoint{2.162192in}{2.746703in}}%
\pgfpathlineto{\pgfqpoint{2.164516in}{2.732174in}}%
\pgfpathlineto{\pgfqpoint{2.166840in}{2.693083in}}%
\pgfpathlineto{\pgfqpoint{2.171488in}{2.745302in}}%
\pgfpathlineto{\pgfqpoint{2.173812in}{2.728174in}}%
\pgfpathlineto{\pgfqpoint{2.176136in}{2.749152in}}%
\pgfpathlineto{\pgfqpoint{2.178460in}{2.710338in}}%
\pgfpathlineto{\pgfqpoint{2.180784in}{2.726165in}}%
\pgfpathlineto{\pgfqpoint{2.183108in}{2.699685in}}%
\pgfpathlineto{\pgfqpoint{2.185432in}{2.733079in}}%
\pgfpathlineto{\pgfqpoint{2.187756in}{2.735854in}}%
\pgfpathlineto{\pgfqpoint{2.190080in}{2.711829in}}%
\pgfpathlineto{\pgfqpoint{2.194728in}{2.745042in}}%
\pgfpathlineto{\pgfqpoint{2.197052in}{2.711786in}}%
\pgfpathlineto{\pgfqpoint{2.199376in}{2.726792in}}%
\pgfpathlineto{\pgfqpoint{2.201700in}{2.713174in}}%
\pgfpathlineto{\pgfqpoint{2.204024in}{2.731174in}}%
\pgfpathlineto{\pgfqpoint{2.206348in}{2.727337in}}%
\pgfpathlineto{\pgfqpoint{2.208672in}{2.736689in}}%
\pgfpathlineto{\pgfqpoint{2.210995in}{2.728596in}}%
\pgfpathlineto{\pgfqpoint{2.213319in}{2.725287in}}%
\pgfpathlineto{\pgfqpoint{2.215643in}{2.718117in}}%
\pgfpathlineto{\pgfqpoint{2.217967in}{2.750752in}}%
\pgfpathlineto{\pgfqpoint{2.220291in}{2.708607in}}%
\pgfpathlineto{\pgfqpoint{2.222615in}{2.716955in}}%
\pgfpathlineto{\pgfqpoint{2.224939in}{2.752771in}}%
\pgfpathlineto{\pgfqpoint{2.229587in}{2.725456in}}%
\pgfpathlineto{\pgfqpoint{2.234235in}{2.718913in}}%
\pgfpathlineto{\pgfqpoint{2.236559in}{2.738468in}}%
\pgfpathlineto{\pgfqpoint{2.238883in}{2.723693in}}%
\pgfpathlineto{\pgfqpoint{2.241207in}{2.752456in}}%
\pgfpathlineto{\pgfqpoint{2.243531in}{2.738003in}}%
\pgfpathlineto{\pgfqpoint{2.245855in}{2.742514in}}%
\pgfpathlineto{\pgfqpoint{2.248179in}{2.752147in}}%
\pgfpathlineto{\pgfqpoint{2.255151in}{2.737487in}}%
\pgfpathlineto{\pgfqpoint{2.257475in}{2.715236in}}%
\pgfpathlineto{\pgfqpoint{2.262123in}{2.742323in}}%
\pgfpathlineto{\pgfqpoint{2.264447in}{2.739902in}}%
\pgfpathlineto{\pgfqpoint{2.266771in}{2.721253in}}%
\pgfpathlineto{\pgfqpoint{2.269095in}{2.720114in}}%
\pgfpathlineto{\pgfqpoint{2.271419in}{2.737100in}}%
\pgfpathlineto{\pgfqpoint{2.273743in}{2.715854in}}%
\pgfpathlineto{\pgfqpoint{2.278391in}{2.737406in}}%
\pgfpathlineto{\pgfqpoint{2.280715in}{2.723889in}}%
\pgfpathlineto{\pgfqpoint{2.283038in}{2.777136in}}%
\pgfpathlineto{\pgfqpoint{2.285362in}{2.751824in}}%
\pgfpathlineto{\pgfqpoint{2.287686in}{2.738747in}}%
\pgfpathlineto{\pgfqpoint{2.290010in}{2.746026in}}%
\pgfpathlineto{\pgfqpoint{2.292334in}{2.713586in}}%
\pgfpathlineto{\pgfqpoint{2.294658in}{2.757628in}}%
\pgfpathlineto{\pgfqpoint{2.296982in}{2.761724in}}%
\pgfpathlineto{\pgfqpoint{2.301630in}{2.740851in}}%
\pgfpathlineto{\pgfqpoint{2.303954in}{2.747414in}}%
\pgfpathlineto{\pgfqpoint{2.306278in}{2.723187in}}%
\pgfpathlineto{\pgfqpoint{2.308602in}{2.762473in}}%
\pgfpathlineto{\pgfqpoint{2.310926in}{2.767762in}}%
\pgfpathlineto{\pgfqpoint{2.313250in}{2.736837in}}%
\pgfpathlineto{\pgfqpoint{2.315574in}{2.758564in}}%
\pgfpathlineto{\pgfqpoint{2.317898in}{2.734855in}}%
\pgfpathlineto{\pgfqpoint{2.320222in}{2.771935in}}%
\pgfpathlineto{\pgfqpoint{2.322546in}{2.730815in}}%
\pgfpathlineto{\pgfqpoint{2.324870in}{2.752635in}}%
\pgfpathlineto{\pgfqpoint{2.327194in}{2.732382in}}%
\pgfpathlineto{\pgfqpoint{2.329518in}{2.740540in}}%
\pgfpathlineto{\pgfqpoint{2.331842in}{2.740985in}}%
\pgfpathlineto{\pgfqpoint{2.334166in}{2.748344in}}%
\pgfpathlineto{\pgfqpoint{2.336490in}{2.768213in}}%
\pgfpathlineto{\pgfqpoint{2.338814in}{2.738716in}}%
\pgfpathlineto{\pgfqpoint{2.341138in}{2.736560in}}%
\pgfpathlineto{\pgfqpoint{2.343462in}{2.738992in}}%
\pgfpathlineto{\pgfqpoint{2.345786in}{2.735479in}}%
\pgfpathlineto{\pgfqpoint{2.348110in}{2.742649in}}%
\pgfpathlineto{\pgfqpoint{2.350434in}{2.730089in}}%
\pgfpathlineto{\pgfqpoint{2.352757in}{2.746162in}}%
\pgfpathlineto{\pgfqpoint{2.355081in}{2.741112in}}%
\pgfpathlineto{\pgfqpoint{2.357405in}{2.756062in}}%
\pgfpathlineto{\pgfqpoint{2.359729in}{2.740423in}}%
\pgfpathlineto{\pgfqpoint{2.362053in}{2.732739in}}%
\pgfpathlineto{\pgfqpoint{2.364377in}{2.747645in}}%
\pgfpathlineto{\pgfqpoint{2.366701in}{2.755065in}}%
\pgfpathlineto{\pgfqpoint{2.369025in}{2.731825in}}%
\pgfpathlineto{\pgfqpoint{2.371349in}{2.730155in}}%
\pgfpathlineto{\pgfqpoint{2.373673in}{2.759062in}}%
\pgfpathlineto{\pgfqpoint{2.375997in}{2.753655in}}%
\pgfpathlineto{\pgfqpoint{2.378321in}{2.771551in}}%
\pgfpathlineto{\pgfqpoint{2.380645in}{2.749765in}}%
\pgfpathlineto{\pgfqpoint{2.382969in}{2.738927in}}%
\pgfpathlineto{\pgfqpoint{2.385293in}{2.748718in}}%
\pgfpathlineto{\pgfqpoint{2.387617in}{2.751235in}}%
\pgfpathlineto{\pgfqpoint{2.392265in}{2.750140in}}%
\pgfpathlineto{\pgfqpoint{2.394589in}{2.447160in}}%
\pgfpathlineto{\pgfqpoint{2.399237in}{2.482796in}}%
\pgfpathlineto{\pgfqpoint{2.401561in}{2.431388in}}%
\pgfpathlineto{\pgfqpoint{2.403885in}{2.471345in}}%
\pgfpathlineto{\pgfqpoint{2.406209in}{2.455276in}}%
\pgfpathlineto{\pgfqpoint{2.408533in}{2.456562in}}%
\pgfpathlineto{\pgfqpoint{2.410857in}{2.461246in}}%
\pgfpathlineto{\pgfqpoint{2.413181in}{2.459563in}}%
\pgfpathlineto{\pgfqpoint{2.415505in}{2.455174in}}%
\pgfpathlineto{\pgfqpoint{2.417829in}{2.457396in}}%
\pgfpathlineto{\pgfqpoint{2.420153in}{2.476796in}}%
\pgfpathlineto{\pgfqpoint{2.422476in}{2.450807in}}%
\pgfpathlineto{\pgfqpoint{2.424800in}{2.486978in}}%
\pgfpathlineto{\pgfqpoint{2.427124in}{2.462099in}}%
\pgfpathlineto{\pgfqpoint{2.429448in}{2.456766in}}%
\pgfpathlineto{\pgfqpoint{2.431772in}{2.478623in}}%
\pgfpathlineto{\pgfqpoint{2.434096in}{2.485081in}}%
\pgfpathlineto{\pgfqpoint{2.436420in}{2.451542in}}%
\pgfpathlineto{\pgfqpoint{2.438744in}{2.474522in}}%
\pgfpathlineto{\pgfqpoint{2.441068in}{2.452370in}}%
\pgfpathlineto{\pgfqpoint{2.443392in}{2.476203in}}%
\pgfpathlineto{\pgfqpoint{2.445716in}{2.463835in}}%
\pgfpathlineto{\pgfqpoint{2.448040in}{2.467367in}}%
\pgfpathlineto{\pgfqpoint{2.450364in}{2.452033in}}%
\pgfpathlineto{\pgfqpoint{2.452688in}{2.428581in}}%
\pgfpathlineto{\pgfqpoint{2.455012in}{2.463233in}}%
\pgfpathlineto{\pgfqpoint{2.459660in}{2.493976in}}%
\pgfpathlineto{\pgfqpoint{2.461984in}{2.491856in}}%
\pgfpathlineto{\pgfqpoint{2.464308in}{2.479584in}}%
\pgfpathlineto{\pgfqpoint{2.466632in}{2.486914in}}%
\pgfpathlineto{\pgfqpoint{2.468956in}{2.471597in}}%
\pgfpathlineto{\pgfqpoint{2.471280in}{2.479512in}}%
\pgfpathlineto{\pgfqpoint{2.473604in}{2.450842in}}%
\pgfpathlineto{\pgfqpoint{2.475928in}{2.465348in}}%
\pgfpathlineto{\pgfqpoint{2.478252in}{2.443132in}}%
\pgfpathlineto{\pgfqpoint{2.482900in}{2.464977in}}%
\pgfpathlineto{\pgfqpoint{2.485224in}{2.476281in}}%
\pgfpathlineto{\pgfqpoint{2.487548in}{2.479581in}}%
\pgfpathlineto{\pgfqpoint{2.489872in}{2.441559in}}%
\pgfpathlineto{\pgfqpoint{2.492195in}{2.454224in}}%
\pgfpathlineto{\pgfqpoint{2.494519in}{2.458639in}}%
\pgfpathlineto{\pgfqpoint{2.496843in}{2.488686in}}%
\pgfpathlineto{\pgfqpoint{2.499167in}{2.480595in}}%
\pgfpathlineto{\pgfqpoint{2.501491in}{2.467955in}}%
\pgfpathlineto{\pgfqpoint{2.503815in}{2.431527in}}%
\pgfpathlineto{\pgfqpoint{2.506139in}{2.479363in}}%
\pgfpathlineto{\pgfqpoint{2.508463in}{2.478127in}}%
\pgfpathlineto{\pgfqpoint{2.510787in}{2.488974in}}%
\pgfpathlineto{\pgfqpoint{2.513111in}{2.487060in}}%
\pgfpathlineto{\pgfqpoint{2.515435in}{2.497625in}}%
\pgfpathlineto{\pgfqpoint{2.517759in}{2.479069in}}%
\pgfpathlineto{\pgfqpoint{2.520083in}{2.496469in}}%
\pgfpathlineto{\pgfqpoint{2.522407in}{2.464511in}}%
\pgfpathlineto{\pgfqpoint{2.524731in}{2.480495in}}%
\pgfpathlineto{\pgfqpoint{2.527055in}{2.457255in}}%
\pgfpathlineto{\pgfqpoint{2.529379in}{2.479892in}}%
\pgfpathlineto{\pgfqpoint{2.531703in}{2.482197in}}%
\pgfpathlineto{\pgfqpoint{2.534027in}{2.467269in}}%
\pgfpathlineto{\pgfqpoint{2.536351in}{2.527024in}}%
\pgfpathlineto{\pgfqpoint{2.538675in}{2.462120in}}%
\pgfpathlineto{\pgfqpoint{2.540999in}{2.459098in}}%
\pgfpathlineto{\pgfqpoint{2.543323in}{2.504713in}}%
\pgfpathlineto{\pgfqpoint{2.545647in}{2.493736in}}%
\pgfpathlineto{\pgfqpoint{2.547971in}{2.491965in}}%
\pgfpathlineto{\pgfqpoint{2.550295in}{2.510602in}}%
\pgfpathlineto{\pgfqpoint{2.552619in}{2.488608in}}%
\pgfpathlineto{\pgfqpoint{2.554943in}{2.482688in}}%
\pgfpathlineto{\pgfqpoint{2.559591in}{2.507819in}}%
\pgfpathlineto{\pgfqpoint{2.561915in}{2.483297in}}%
\pgfpathlineto{\pgfqpoint{2.564238in}{2.489442in}}%
\pgfpathlineto{\pgfqpoint{2.566562in}{2.509274in}}%
\pgfpathlineto{\pgfqpoint{2.568886in}{2.494426in}}%
\pgfpathlineto{\pgfqpoint{2.571210in}{2.516507in}}%
\pgfpathlineto{\pgfqpoint{2.573534in}{2.482949in}}%
\pgfpathlineto{\pgfqpoint{2.575858in}{2.514240in}}%
\pgfpathlineto{\pgfqpoint{2.578182in}{2.440453in}}%
\pgfpathlineto{\pgfqpoint{2.580506in}{2.506916in}}%
\pgfpathlineto{\pgfqpoint{2.582830in}{2.509369in}}%
\pgfpathlineto{\pgfqpoint{2.585154in}{2.470397in}}%
\pgfpathlineto{\pgfqpoint{2.592126in}{2.516570in}}%
\pgfpathlineto{\pgfqpoint{2.594450in}{2.480157in}}%
\pgfpathlineto{\pgfqpoint{2.599098in}{2.498683in}}%
\pgfpathlineto{\pgfqpoint{2.601422in}{2.454778in}}%
\pgfpathlineto{\pgfqpoint{2.603746in}{2.503385in}}%
\pgfpathlineto{\pgfqpoint{2.606070in}{2.470033in}}%
\pgfpathlineto{\pgfqpoint{2.608394in}{2.504880in}}%
\pgfpathlineto{\pgfqpoint{2.610718in}{2.474966in}}%
\pgfpathlineto{\pgfqpoint{2.613042in}{2.468816in}}%
\pgfpathlineto{\pgfqpoint{2.615366in}{2.480854in}}%
\pgfpathlineto{\pgfqpoint{2.620014in}{2.488525in}}%
\pgfpathlineto{\pgfqpoint{2.622338in}{2.534758in}}%
\pgfpathlineto{\pgfqpoint{2.626986in}{2.489000in}}%
\pgfpathlineto{\pgfqpoint{2.629310in}{2.477793in}}%
\pgfpathlineto{\pgfqpoint{2.631634in}{2.479012in}}%
\pgfpathlineto{\pgfqpoint{2.633957in}{2.498674in}}%
\pgfpathlineto{\pgfqpoint{2.636281in}{2.496548in}}%
\pgfpathlineto{\pgfqpoint{2.638605in}{2.513393in}}%
\pgfpathlineto{\pgfqpoint{2.640929in}{2.511466in}}%
\pgfpathlineto{\pgfqpoint{2.643253in}{2.490496in}}%
\pgfpathlineto{\pgfqpoint{2.645577in}{2.492950in}}%
\pgfpathlineto{\pgfqpoint{2.647901in}{2.517198in}}%
\pgfpathlineto{\pgfqpoint{2.650225in}{2.494085in}}%
\pgfpathlineto{\pgfqpoint{2.652549in}{2.492202in}}%
\pgfpathlineto{\pgfqpoint{2.654873in}{2.500919in}}%
\pgfpathlineto{\pgfqpoint{2.657197in}{2.521035in}}%
\pgfpathlineto{\pgfqpoint{2.659521in}{2.500115in}}%
\pgfpathlineto{\pgfqpoint{2.661845in}{2.515748in}}%
\pgfpathlineto{\pgfqpoint{2.664169in}{2.511398in}}%
\pgfpathlineto{\pgfqpoint{2.666493in}{2.495763in}}%
\pgfpathlineto{\pgfqpoint{2.668817in}{2.499889in}}%
\pgfpathlineto{\pgfqpoint{2.671141in}{2.488009in}}%
\pgfpathlineto{\pgfqpoint{2.673465in}{2.515073in}}%
\pgfpathlineto{\pgfqpoint{2.675789in}{2.483741in}}%
\pgfpathlineto{\pgfqpoint{2.678113in}{2.529615in}}%
\pgfpathlineto{\pgfqpoint{2.680437in}{2.498174in}}%
\pgfpathlineto{\pgfqpoint{2.682761in}{2.503577in}}%
\pgfpathlineto{\pgfqpoint{2.685085in}{2.736007in}}%
\pgfpathlineto{\pgfqpoint{2.687409in}{2.712273in}}%
\pgfpathlineto{\pgfqpoint{2.689733in}{2.736951in}}%
\pgfpathlineto{\pgfqpoint{2.692057in}{2.730231in}}%
\pgfpathlineto{\pgfqpoint{2.694381in}{2.706543in}}%
\pgfpathlineto{\pgfqpoint{2.699029in}{2.736732in}}%
\pgfpathlineto{\pgfqpoint{2.701353in}{2.752217in}}%
\pgfpathlineto{\pgfqpoint{2.708324in}{2.717160in}}%
\pgfpathlineto{\pgfqpoint{2.715296in}{2.750847in}}%
\pgfpathlineto{\pgfqpoint{2.717620in}{2.777259in}}%
\pgfpathlineto{\pgfqpoint{2.719944in}{2.737388in}}%
\pgfpathlineto{\pgfqpoint{2.722268in}{2.731197in}}%
\pgfpathlineto{\pgfqpoint{2.724592in}{2.744016in}}%
\pgfpathlineto{\pgfqpoint{2.726916in}{2.737010in}}%
\pgfpathlineto{\pgfqpoint{2.729240in}{2.738085in}}%
\pgfpathlineto{\pgfqpoint{2.731564in}{2.734691in}}%
\pgfpathlineto{\pgfqpoint{2.733888in}{2.767333in}}%
\pgfpathlineto{\pgfqpoint{2.736212in}{2.750892in}}%
\pgfpathlineto{\pgfqpoint{2.738536in}{2.743256in}}%
\pgfpathlineto{\pgfqpoint{2.740860in}{2.751301in}}%
\pgfpathlineto{\pgfqpoint{2.743184in}{2.737210in}}%
\pgfpathlineto{\pgfqpoint{2.745508in}{2.743042in}}%
\pgfpathlineto{\pgfqpoint{2.747832in}{2.731668in}}%
\pgfpathlineto{\pgfqpoint{2.752480in}{2.738790in}}%
\pgfpathlineto{\pgfqpoint{2.754804in}{2.726841in}}%
\pgfpathlineto{\pgfqpoint{2.759452in}{2.786407in}}%
\pgfpathlineto{\pgfqpoint{2.761776in}{2.734879in}}%
\pgfpathlineto{\pgfqpoint{2.764100in}{2.721723in}}%
\pgfpathlineto{\pgfqpoint{2.766424in}{2.762631in}}%
\pgfpathlineto{\pgfqpoint{2.768748in}{2.705811in}}%
\pgfpathlineto{\pgfqpoint{2.775719in}{2.772182in}}%
\pgfpathlineto{\pgfqpoint{2.778043in}{2.732430in}}%
\pgfpathlineto{\pgfqpoint{2.780367in}{2.743379in}}%
\pgfpathlineto{\pgfqpoint{2.785015in}{2.777054in}}%
\pgfpathlineto{\pgfqpoint{2.789663in}{2.743195in}}%
\pgfpathlineto{\pgfqpoint{2.791987in}{2.757363in}}%
\pgfpathlineto{\pgfqpoint{2.794311in}{2.756546in}}%
\pgfpathlineto{\pgfqpoint{2.796635in}{2.750854in}}%
\pgfpathlineto{\pgfqpoint{2.801283in}{2.757180in}}%
\pgfpathlineto{\pgfqpoint{2.803607in}{2.769622in}}%
\pgfpathlineto{\pgfqpoint{2.805931in}{2.771014in}}%
\pgfpathlineto{\pgfqpoint{2.808255in}{2.757075in}}%
\pgfpathlineto{\pgfqpoint{2.810579in}{2.756298in}}%
\pgfpathlineto{\pgfqpoint{2.812903in}{2.791431in}}%
\pgfpathlineto{\pgfqpoint{2.817551in}{2.749434in}}%
\pgfpathlineto{\pgfqpoint{2.819875in}{2.752393in}}%
\pgfpathlineto{\pgfqpoint{2.822199in}{2.768835in}}%
\pgfpathlineto{\pgfqpoint{2.824523in}{2.740673in}}%
\pgfpathlineto{\pgfqpoint{2.826847in}{2.732747in}}%
\pgfpathlineto{\pgfqpoint{2.829171in}{2.769691in}}%
\pgfpathlineto{\pgfqpoint{2.831495in}{2.758075in}}%
\pgfpathlineto{\pgfqpoint{2.833819in}{2.762222in}}%
\pgfpathlineto{\pgfqpoint{2.836143in}{2.760996in}}%
\pgfpathlineto{\pgfqpoint{2.838467in}{2.799394in}}%
\pgfpathlineto{\pgfqpoint{2.840791in}{2.751839in}}%
\pgfpathlineto{\pgfqpoint{2.843115in}{2.788178in}}%
\pgfpathlineto{\pgfqpoint{2.845438in}{2.769015in}}%
\pgfpathlineto{\pgfqpoint{2.847762in}{2.773340in}}%
\pgfpathlineto{\pgfqpoint{2.850086in}{2.739165in}}%
\pgfpathlineto{\pgfqpoint{2.854734in}{2.764355in}}%
\pgfpathlineto{\pgfqpoint{2.857058in}{2.738377in}}%
\pgfpathlineto{\pgfqpoint{2.861706in}{2.779297in}}%
\pgfpathlineto{\pgfqpoint{2.864030in}{2.763444in}}%
\pgfpathlineto{\pgfqpoint{2.871002in}{2.802920in}}%
\pgfpathlineto{\pgfqpoint{2.873326in}{2.790732in}}%
\pgfpathlineto{\pgfqpoint{2.875650in}{2.789832in}}%
\pgfpathlineto{\pgfqpoint{2.877974in}{2.774658in}}%
\pgfpathlineto{\pgfqpoint{2.880298in}{2.774055in}}%
\pgfpathlineto{\pgfqpoint{2.882622in}{2.743080in}}%
\pgfpathlineto{\pgfqpoint{2.884946in}{2.770652in}}%
\pgfpathlineto{\pgfqpoint{2.887270in}{2.782219in}}%
\pgfpathlineto{\pgfqpoint{2.889594in}{2.758053in}}%
\pgfpathlineto{\pgfqpoint{2.891918in}{2.757062in}}%
\pgfpathlineto{\pgfqpoint{2.894242in}{2.799918in}}%
\pgfpathlineto{\pgfqpoint{2.896566in}{2.783570in}}%
\pgfpathlineto{\pgfqpoint{2.898890in}{2.779903in}}%
\pgfpathlineto{\pgfqpoint{2.901214in}{2.783505in}}%
\pgfpathlineto{\pgfqpoint{2.903538in}{2.760582in}}%
\pgfpathlineto{\pgfqpoint{2.905862in}{2.798406in}}%
\pgfpathlineto{\pgfqpoint{2.908186in}{2.793132in}}%
\pgfpathlineto{\pgfqpoint{2.910510in}{2.794980in}}%
\pgfpathlineto{\pgfqpoint{2.912834in}{2.799762in}}%
\pgfpathlineto{\pgfqpoint{2.915157in}{2.779788in}}%
\pgfpathlineto{\pgfqpoint{2.917481in}{2.775193in}}%
\pgfpathlineto{\pgfqpoint{2.919805in}{2.792602in}}%
\pgfpathlineto{\pgfqpoint{2.922129in}{2.779510in}}%
\pgfpathlineto{\pgfqpoint{2.924453in}{2.784437in}}%
\pgfpathlineto{\pgfqpoint{2.926777in}{2.775272in}}%
\pgfpathlineto{\pgfqpoint{2.931425in}{2.800560in}}%
\pgfpathlineto{\pgfqpoint{2.933749in}{2.762208in}}%
\pgfpathlineto{\pgfqpoint{2.936073in}{2.785236in}}%
\pgfpathlineto{\pgfqpoint{2.940721in}{2.781045in}}%
\pgfpathlineto{\pgfqpoint{2.943045in}{2.790357in}}%
\pgfpathlineto{\pgfqpoint{2.945369in}{2.787585in}}%
\pgfpathlineto{\pgfqpoint{2.947693in}{2.767254in}}%
\pgfpathlineto{\pgfqpoint{2.950017in}{2.774536in}}%
\pgfpathlineto{\pgfqpoint{2.952341in}{2.794647in}}%
\pgfpathlineto{\pgfqpoint{2.954665in}{2.792265in}}%
\pgfpathlineto{\pgfqpoint{2.956989in}{2.758538in}}%
\pgfpathlineto{\pgfqpoint{2.959313in}{2.773566in}}%
\pgfpathlineto{\pgfqpoint{2.961637in}{2.773392in}}%
\pgfpathlineto{\pgfqpoint{2.966285in}{2.787769in}}%
\pgfpathlineto{\pgfqpoint{2.968609in}{2.764236in}}%
\pgfpathlineto{\pgfqpoint{2.970933in}{2.826810in}}%
\pgfpathlineto{\pgfqpoint{2.973257in}{2.816587in}}%
\pgfpathlineto{\pgfqpoint{2.975581in}{2.511934in}}%
\pgfpathlineto{\pgfqpoint{2.977905in}{2.502379in}}%
\pgfpathlineto{\pgfqpoint{2.980229in}{2.511343in}}%
\pgfpathlineto{\pgfqpoint{2.982553in}{2.496518in}}%
\pgfpathlineto{\pgfqpoint{2.984876in}{2.473868in}}%
\pgfpathlineto{\pgfqpoint{2.987200in}{2.508622in}}%
\pgfpathlineto{\pgfqpoint{2.991848in}{2.469026in}}%
\pgfpathlineto{\pgfqpoint{2.994172in}{2.502668in}}%
\pgfpathlineto{\pgfqpoint{2.996496in}{2.479782in}}%
\pgfpathlineto{\pgfqpoint{2.998820in}{2.503754in}}%
\pgfpathlineto{\pgfqpoint{3.001144in}{2.501218in}}%
\pgfpathlineto{\pgfqpoint{3.003468in}{2.514196in}}%
\pgfpathlineto{\pgfqpoint{3.005792in}{2.506626in}}%
\pgfpathlineto{\pgfqpoint{3.008116in}{2.526608in}}%
\pgfpathlineto{\pgfqpoint{3.010440in}{2.502483in}}%
\pgfpathlineto{\pgfqpoint{3.012764in}{2.497218in}}%
\pgfpathlineto{\pgfqpoint{3.015088in}{2.489101in}}%
\pgfpathlineto{\pgfqpoint{3.017412in}{2.500010in}}%
\pgfpathlineto{\pgfqpoint{3.019736in}{2.488253in}}%
\pgfpathlineto{\pgfqpoint{3.022060in}{2.506490in}}%
\pgfpathlineto{\pgfqpoint{3.026708in}{2.498291in}}%
\pgfpathlineto{\pgfqpoint{3.029032in}{2.516441in}}%
\pgfpathlineto{\pgfqpoint{3.031356in}{2.514890in}}%
\pgfpathlineto{\pgfqpoint{3.033680in}{2.469876in}}%
\pgfpathlineto{\pgfqpoint{3.036004in}{2.510888in}}%
\pgfpathlineto{\pgfqpoint{3.038328in}{2.511841in}}%
\pgfpathlineto{\pgfqpoint{3.040652in}{2.508368in}}%
\pgfpathlineto{\pgfqpoint{3.042976in}{2.494544in}}%
\pgfpathlineto{\pgfqpoint{3.045300in}{2.495175in}}%
\pgfpathlineto{\pgfqpoint{3.047624in}{2.497216in}}%
\pgfpathlineto{\pgfqpoint{3.049948in}{2.504165in}}%
\pgfpathlineto{\pgfqpoint{3.052272in}{2.516365in}}%
\pgfpathlineto{\pgfqpoint{3.054596in}{2.537542in}}%
\pgfpathlineto{\pgfqpoint{3.056919in}{2.502814in}}%
\pgfpathlineto{\pgfqpoint{3.059243in}{2.533478in}}%
\pgfpathlineto{\pgfqpoint{3.061567in}{2.527391in}}%
\pgfpathlineto{\pgfqpoint{3.063891in}{2.509586in}}%
\pgfpathlineto{\pgfqpoint{3.066215in}{2.525608in}}%
\pgfpathlineto{\pgfqpoint{3.068539in}{2.501612in}}%
\pgfpathlineto{\pgfqpoint{3.070863in}{2.491160in}}%
\pgfpathlineto{\pgfqpoint{3.077835in}{2.544925in}}%
\pgfpathlineto{\pgfqpoint{3.080159in}{2.506363in}}%
\pgfpathlineto{\pgfqpoint{3.084807in}{2.492716in}}%
\pgfpathlineto{\pgfqpoint{3.087131in}{2.512034in}}%
\pgfpathlineto{\pgfqpoint{3.089455in}{2.521587in}}%
\pgfpathlineto{\pgfqpoint{3.091779in}{2.524306in}}%
\pgfpathlineto{\pgfqpoint{3.094103in}{2.504381in}}%
\pgfpathlineto{\pgfqpoint{3.096427in}{2.525504in}}%
\pgfpathlineto{\pgfqpoint{3.098751in}{2.497465in}}%
\pgfpathlineto{\pgfqpoint{3.101075in}{2.548424in}}%
\pgfpathlineto{\pgfqpoint{3.105723in}{2.513119in}}%
\pgfpathlineto{\pgfqpoint{3.108047in}{2.498022in}}%
\pgfpathlineto{\pgfqpoint{3.110371in}{2.492182in}}%
\pgfpathlineto{\pgfqpoint{3.112695in}{2.530627in}}%
\pgfpathlineto{\pgfqpoint{3.115019in}{2.499111in}}%
\pgfpathlineto{\pgfqpoint{3.117343in}{2.538918in}}%
\pgfpathlineto{\pgfqpoint{3.119667in}{2.528448in}}%
\pgfpathlineto{\pgfqpoint{3.121991in}{2.506829in}}%
\pgfpathlineto{\pgfqpoint{3.126638in}{2.526925in}}%
\pgfpathlineto{\pgfqpoint{3.128962in}{2.515723in}}%
\pgfpathlineto{\pgfqpoint{3.131286in}{2.537105in}}%
\pgfpathlineto{\pgfqpoint{3.133610in}{2.530970in}}%
\pgfpathlineto{\pgfqpoint{3.135934in}{2.538290in}}%
\pgfpathlineto{\pgfqpoint{3.138258in}{2.511235in}}%
\pgfpathlineto{\pgfqpoint{3.140582in}{2.537439in}}%
\pgfpathlineto{\pgfqpoint{3.142906in}{2.507750in}}%
\pgfpathlineto{\pgfqpoint{3.145230in}{2.541230in}}%
\pgfpathlineto{\pgfqpoint{3.147554in}{2.518506in}}%
\pgfpathlineto{\pgfqpoint{3.149878in}{2.525273in}}%
\pgfpathlineto{\pgfqpoint{3.152202in}{2.515645in}}%
\pgfpathlineto{\pgfqpoint{3.154526in}{2.514094in}}%
\pgfpathlineto{\pgfqpoint{3.156850in}{2.559803in}}%
\pgfpathlineto{\pgfqpoint{3.159174in}{2.548175in}}%
\pgfpathlineto{\pgfqpoint{3.161498in}{2.499023in}}%
\pgfpathlineto{\pgfqpoint{3.163822in}{2.546211in}}%
\pgfpathlineto{\pgfqpoint{3.166146in}{2.522613in}}%
\pgfpathlineto{\pgfqpoint{3.168470in}{2.534612in}}%
\pgfpathlineto{\pgfqpoint{3.170794in}{2.522000in}}%
\pgfpathlineto{\pgfqpoint{3.173118in}{2.526424in}}%
\pgfpathlineto{\pgfqpoint{3.175442in}{2.528075in}}%
\pgfpathlineto{\pgfqpoint{3.180090in}{2.554604in}}%
\pgfpathlineto{\pgfqpoint{3.182414in}{2.558853in}}%
\pgfpathlineto{\pgfqpoint{3.184738in}{2.544225in}}%
\pgfpathlineto{\pgfqpoint{3.187062in}{2.502443in}}%
\pgfpathlineto{\pgfqpoint{3.189386in}{2.527210in}}%
\pgfpathlineto{\pgfqpoint{3.194034in}{2.518109in}}%
\pgfpathlineto{\pgfqpoint{3.196357in}{2.516420in}}%
\pgfpathlineto{\pgfqpoint{3.198681in}{2.522818in}}%
\pgfpathlineto{\pgfqpoint{3.201005in}{2.543583in}}%
\pgfpathlineto{\pgfqpoint{3.203329in}{2.524979in}}%
\pgfpathlineto{\pgfqpoint{3.205653in}{2.540888in}}%
\pgfpathlineto{\pgfqpoint{3.207977in}{2.523379in}}%
\pgfpathlineto{\pgfqpoint{3.210301in}{2.532947in}}%
\pgfpathlineto{\pgfqpoint{3.212625in}{2.536714in}}%
\pgfpathlineto{\pgfqpoint{3.217273in}{2.537319in}}%
\pgfpathlineto{\pgfqpoint{3.219597in}{2.548707in}}%
\pgfpathlineto{\pgfqpoint{3.221921in}{2.519978in}}%
\pgfpathlineto{\pgfqpoint{3.224245in}{2.506540in}}%
\pgfpathlineto{\pgfqpoint{3.226569in}{2.535148in}}%
\pgfpathlineto{\pgfqpoint{3.228893in}{2.525266in}}%
\pgfpathlineto{\pgfqpoint{3.231217in}{2.544875in}}%
\pgfpathlineto{\pgfqpoint{3.233541in}{2.519769in}}%
\pgfpathlineto{\pgfqpoint{3.235865in}{2.545961in}}%
\pgfpathlineto{\pgfqpoint{3.238189in}{2.556997in}}%
\pgfpathlineto{\pgfqpoint{3.240513in}{2.534710in}}%
\pgfpathlineto{\pgfqpoint{3.242837in}{2.532307in}}%
\pgfpathlineto{\pgfqpoint{3.245161in}{2.568126in}}%
\pgfpathlineto{\pgfqpoint{3.247485in}{2.537523in}}%
\pgfpathlineto{\pgfqpoint{3.249809in}{2.549728in}}%
\pgfpathlineto{\pgfqpoint{3.252133in}{2.532522in}}%
\pgfpathlineto{\pgfqpoint{3.254457in}{2.553945in}}%
\pgfpathlineto{\pgfqpoint{3.256781in}{2.528096in}}%
\pgfpathlineto{\pgfqpoint{3.259105in}{2.541940in}}%
\pgfpathlineto{\pgfqpoint{3.261429in}{2.536337in}}%
\pgfpathlineto{\pgfqpoint{3.263753in}{2.535321in}}%
\pgfpathlineto{\pgfqpoint{3.266076in}{2.769186in}}%
\pgfpathlineto{\pgfqpoint{3.270724in}{2.743145in}}%
\pgfpathlineto{\pgfqpoint{3.273048in}{2.761875in}}%
\pgfpathlineto{\pgfqpoint{3.275372in}{2.738296in}}%
\pgfpathlineto{\pgfqpoint{3.277696in}{2.763823in}}%
\pgfpathlineto{\pgfqpoint{3.280020in}{2.725959in}}%
\pgfpathlineto{\pgfqpoint{3.282344in}{2.761019in}}%
\pgfpathlineto{\pgfqpoint{3.284668in}{2.766880in}}%
\pgfpathlineto{\pgfqpoint{3.286992in}{2.775767in}}%
\pgfpathlineto{\pgfqpoint{3.289316in}{2.760424in}}%
\pgfpathlineto{\pgfqpoint{3.291640in}{2.777587in}}%
\pgfpathlineto{\pgfqpoint{3.293964in}{2.742534in}}%
\pgfpathlineto{\pgfqpoint{3.296288in}{2.807235in}}%
\pgfpathlineto{\pgfqpoint{3.298612in}{2.776056in}}%
\pgfpathlineto{\pgfqpoint{3.300936in}{2.799749in}}%
\pgfpathlineto{\pgfqpoint{3.303260in}{2.770592in}}%
\pgfpathlineto{\pgfqpoint{3.305584in}{2.776573in}}%
\pgfpathlineto{\pgfqpoint{3.307908in}{2.758500in}}%
\pgfpathlineto{\pgfqpoint{3.310232in}{2.770291in}}%
\pgfpathlineto{\pgfqpoint{3.312556in}{2.747833in}}%
\pgfpathlineto{\pgfqpoint{3.314880in}{2.774445in}}%
\pgfpathlineto{\pgfqpoint{3.317204in}{2.787978in}}%
\pgfpathlineto{\pgfqpoint{3.321852in}{2.750444in}}%
\pgfpathlineto{\pgfqpoint{3.324176in}{2.800265in}}%
\pgfpathlineto{\pgfqpoint{3.326500in}{2.764693in}}%
\pgfpathlineto{\pgfqpoint{3.328824in}{2.760630in}}%
\pgfpathlineto{\pgfqpoint{3.331148in}{2.764393in}}%
\pgfpathlineto{\pgfqpoint{3.333472in}{2.791675in}}%
\pgfpathlineto{\pgfqpoint{3.338119in}{2.748876in}}%
\pgfpathlineto{\pgfqpoint{3.340443in}{2.779331in}}%
\pgfpathlineto{\pgfqpoint{3.342767in}{2.770894in}}%
\pgfpathlineto{\pgfqpoint{3.345091in}{2.771974in}}%
\pgfpathlineto{\pgfqpoint{3.347415in}{2.767749in}}%
\pgfpathlineto{\pgfqpoint{3.349739in}{2.772230in}}%
\pgfpathlineto{\pgfqpoint{3.354387in}{2.757088in}}%
\pgfpathlineto{\pgfqpoint{3.356711in}{2.822692in}}%
\pgfpathlineto{\pgfqpoint{3.361359in}{2.752907in}}%
\pgfpathlineto{\pgfqpoint{3.363683in}{2.773552in}}%
\pgfpathlineto{\pgfqpoint{3.370655in}{2.758732in}}%
\pgfpathlineto{\pgfqpoint{3.372979in}{2.774994in}}%
\pgfpathlineto{\pgfqpoint{3.375303in}{2.753162in}}%
\pgfpathlineto{\pgfqpoint{3.377627in}{2.769592in}}%
\pgfpathlineto{\pgfqpoint{3.379951in}{2.760950in}}%
\pgfpathlineto{\pgfqpoint{3.382275in}{2.789008in}}%
\pgfpathlineto{\pgfqpoint{3.384599in}{2.774711in}}%
\pgfpathlineto{\pgfqpoint{3.389247in}{2.800386in}}%
\pgfpathlineto{\pgfqpoint{3.391571in}{2.774149in}}%
\pgfpathlineto{\pgfqpoint{3.396219in}{2.762930in}}%
\pgfpathlineto{\pgfqpoint{3.398543in}{2.766692in}}%
\pgfpathlineto{\pgfqpoint{3.400867in}{2.762587in}}%
\pgfpathlineto{\pgfqpoint{3.403191in}{2.767526in}}%
\pgfpathlineto{\pgfqpoint{3.405515in}{2.795358in}}%
\pgfpathlineto{\pgfqpoint{3.410162in}{2.773867in}}%
\pgfpathlineto{\pgfqpoint{3.412486in}{2.784466in}}%
\pgfpathlineto{\pgfqpoint{3.414810in}{2.767875in}}%
\pgfpathlineto{\pgfqpoint{3.417134in}{2.778310in}}%
\pgfpathlineto{\pgfqpoint{3.421782in}{2.736975in}}%
\pgfpathlineto{\pgfqpoint{3.424106in}{2.774991in}}%
\pgfpathlineto{\pgfqpoint{3.426430in}{2.772394in}}%
\pgfpathlineto{\pgfqpoint{3.428754in}{2.789885in}}%
\pgfpathlineto{\pgfqpoint{3.431078in}{2.771104in}}%
\pgfpathlineto{\pgfqpoint{3.433402in}{2.761390in}}%
\pgfpathlineto{\pgfqpoint{3.435726in}{2.790778in}}%
\pgfpathlineto{\pgfqpoint{3.438050in}{2.765728in}}%
\pgfpathlineto{\pgfqpoint{3.442698in}{2.828987in}}%
\pgfpathlineto{\pgfqpoint{3.445022in}{2.799653in}}%
\pgfpathlineto{\pgfqpoint{3.447346in}{2.788361in}}%
\pgfpathlineto{\pgfqpoint{3.449670in}{2.764599in}}%
\pgfpathlineto{\pgfqpoint{3.451994in}{2.786945in}}%
\pgfpathlineto{\pgfqpoint{3.454318in}{2.790368in}}%
\pgfpathlineto{\pgfqpoint{3.456642in}{2.757604in}}%
\pgfpathlineto{\pgfqpoint{3.458966in}{2.792458in}}%
\pgfpathlineto{\pgfqpoint{3.461290in}{2.803368in}}%
\pgfpathlineto{\pgfqpoint{3.465938in}{2.765534in}}%
\pgfpathlineto{\pgfqpoint{3.468262in}{2.764866in}}%
\pgfpathlineto{\pgfqpoint{3.470586in}{2.778338in}}%
\pgfpathlineto{\pgfqpoint{3.472910in}{2.766439in}}%
\pgfpathlineto{\pgfqpoint{3.475234in}{2.795064in}}%
\pgfpathlineto{\pgfqpoint{3.477557in}{2.775489in}}%
\pgfpathlineto{\pgfqpoint{3.479881in}{2.814554in}}%
\pgfpathlineto{\pgfqpoint{3.482205in}{2.785198in}}%
\pgfpathlineto{\pgfqpoint{3.484529in}{2.780941in}}%
\pgfpathlineto{\pgfqpoint{3.486853in}{2.779304in}}%
\pgfpathlineto{\pgfqpoint{3.489177in}{2.761806in}}%
\pgfpathlineto{\pgfqpoint{3.491501in}{2.798062in}}%
\pgfpathlineto{\pgfqpoint{3.498473in}{2.751330in}}%
\pgfpathlineto{\pgfqpoint{3.500797in}{2.776896in}}%
\pgfpathlineto{\pgfqpoint{3.505445in}{2.790729in}}%
\pgfpathlineto{\pgfqpoint{3.507769in}{2.781151in}}%
\pgfpathlineto{\pgfqpoint{3.510093in}{2.745977in}}%
\pgfpathlineto{\pgfqpoint{3.512417in}{2.791222in}}%
\pgfpathlineto{\pgfqpoint{3.514741in}{2.801442in}}%
\pgfpathlineto{\pgfqpoint{3.517065in}{2.784705in}}%
\pgfpathlineto{\pgfqpoint{3.521713in}{2.787918in}}%
\pgfpathlineto{\pgfqpoint{3.524037in}{2.815759in}}%
\pgfpathlineto{\pgfqpoint{3.528685in}{2.781362in}}%
\pgfpathlineto{\pgfqpoint{3.531009in}{2.809980in}}%
\pgfpathlineto{\pgfqpoint{3.533333in}{2.795506in}}%
\pgfpathlineto{\pgfqpoint{3.535657in}{2.806659in}}%
\pgfpathlineto{\pgfqpoint{3.540305in}{2.794358in}}%
\pgfpathlineto{\pgfqpoint{3.542629in}{2.779601in}}%
\pgfpathlineto{\pgfqpoint{3.544953in}{2.802856in}}%
\pgfpathlineto{\pgfqpoint{3.547277in}{2.773737in}}%
\pgfpathlineto{\pgfqpoint{3.549600in}{2.762706in}}%
\pgfpathlineto{\pgfqpoint{3.554248in}{2.784648in}}%
\pgfpathlineto{\pgfqpoint{3.556572in}{2.511654in}}%
\pgfpathlineto{\pgfqpoint{3.558896in}{2.480247in}}%
\pgfpathlineto{\pgfqpoint{3.561220in}{2.503349in}}%
\pgfpathlineto{\pgfqpoint{3.563544in}{2.488696in}}%
\pgfpathlineto{\pgfqpoint{3.565868in}{2.464714in}}%
\pgfpathlineto{\pgfqpoint{3.568192in}{2.491000in}}%
\pgfpathlineto{\pgfqpoint{3.570516in}{2.502781in}}%
\pgfpathlineto{\pgfqpoint{3.572840in}{2.475222in}}%
\pgfpathlineto{\pgfqpoint{3.575164in}{2.475561in}}%
\pgfpathlineto{\pgfqpoint{3.577488in}{2.496790in}}%
\pgfpathlineto{\pgfqpoint{3.579812in}{2.477291in}}%
\pgfpathlineto{\pgfqpoint{3.582136in}{2.482521in}}%
\pgfpathlineto{\pgfqpoint{3.586784in}{2.503023in}}%
\pgfpathlineto{\pgfqpoint{3.589108in}{2.500648in}}%
\pgfpathlineto{\pgfqpoint{3.591432in}{2.489998in}}%
\pgfpathlineto{\pgfqpoint{3.593756in}{2.490154in}}%
\pgfpathlineto{\pgfqpoint{3.596080in}{2.478110in}}%
\pgfpathlineto{\pgfqpoint{3.598404in}{2.484879in}}%
\pgfpathlineto{\pgfqpoint{3.600728in}{2.500037in}}%
\pgfpathlineto{\pgfqpoint{3.603052in}{2.476741in}}%
\pgfpathlineto{\pgfqpoint{3.605376in}{2.467665in}}%
\pgfpathlineto{\pgfqpoint{3.607700in}{2.490869in}}%
\pgfpathlineto{\pgfqpoint{3.610024in}{2.491066in}}%
\pgfpathlineto{\pgfqpoint{3.612348in}{2.528316in}}%
\pgfpathlineto{\pgfqpoint{3.614672in}{2.493773in}}%
\pgfpathlineto{\pgfqpoint{3.616996in}{2.491903in}}%
\pgfpathlineto{\pgfqpoint{3.619319in}{2.498195in}}%
\pgfpathlineto{\pgfqpoint{3.621643in}{2.489463in}}%
\pgfpathlineto{\pgfqpoint{3.623967in}{2.498097in}}%
\pgfpathlineto{\pgfqpoint{3.626291in}{2.531822in}}%
\pgfpathlineto{\pgfqpoint{3.628615in}{2.474053in}}%
\pgfpathlineto{\pgfqpoint{3.630939in}{2.493079in}}%
\pgfpathlineto{\pgfqpoint{3.633263in}{2.488115in}}%
\pgfpathlineto{\pgfqpoint{3.635587in}{2.476422in}}%
\pgfpathlineto{\pgfqpoint{3.637911in}{2.508281in}}%
\pgfpathlineto{\pgfqpoint{3.640235in}{2.463875in}}%
\pgfpathlineto{\pgfqpoint{3.644883in}{2.512200in}}%
\pgfpathlineto{\pgfqpoint{3.647207in}{2.496182in}}%
\pgfpathlineto{\pgfqpoint{3.649531in}{2.508006in}}%
\pgfpathlineto{\pgfqpoint{3.651855in}{2.500018in}}%
\pgfpathlineto{\pgfqpoint{3.654179in}{2.522189in}}%
\pgfpathlineto{\pgfqpoint{3.656503in}{2.524181in}}%
\pgfpathlineto{\pgfqpoint{3.658827in}{2.507646in}}%
\pgfpathlineto{\pgfqpoint{3.661151in}{2.473253in}}%
\pgfpathlineto{\pgfqpoint{3.663475in}{2.511529in}}%
\pgfpathlineto{\pgfqpoint{3.665799in}{2.501362in}}%
\pgfpathlineto{\pgfqpoint{3.668123in}{2.501849in}}%
\pgfpathlineto{\pgfqpoint{3.670447in}{2.469290in}}%
\pgfpathlineto{\pgfqpoint{3.672771in}{2.529905in}}%
\pgfpathlineto{\pgfqpoint{3.675095in}{2.516894in}}%
\pgfpathlineto{\pgfqpoint{3.677419in}{2.487847in}}%
\pgfpathlineto{\pgfqpoint{3.682067in}{2.489405in}}%
\pgfpathlineto{\pgfqpoint{3.684391in}{2.490730in}}%
\pgfpathlineto{\pgfqpoint{3.686715in}{2.462285in}}%
\pgfpathlineto{\pgfqpoint{3.689038in}{2.502687in}}%
\pgfpathlineto{\pgfqpoint{3.691362in}{2.488856in}}%
\pgfpathlineto{\pgfqpoint{3.693686in}{2.486897in}}%
\pgfpathlineto{\pgfqpoint{3.696010in}{2.504229in}}%
\pgfpathlineto{\pgfqpoint{3.698334in}{2.512452in}}%
\pgfpathlineto{\pgfqpoint{3.700658in}{2.513183in}}%
\pgfpathlineto{\pgfqpoint{3.702982in}{2.486194in}}%
\pgfpathlineto{\pgfqpoint{3.705306in}{2.487973in}}%
\pgfpathlineto{\pgfqpoint{3.707630in}{2.474813in}}%
\pgfpathlineto{\pgfqpoint{3.712278in}{2.507645in}}%
\pgfpathlineto{\pgfqpoint{3.714602in}{2.514132in}}%
\pgfpathlineto{\pgfqpoint{3.719250in}{2.475417in}}%
\pgfpathlineto{\pgfqpoint{3.721574in}{2.498442in}}%
\pgfpathlineto{\pgfqpoint{3.723898in}{2.509495in}}%
\pgfpathlineto{\pgfqpoint{3.726222in}{2.505937in}}%
\pgfpathlineto{\pgfqpoint{3.728546in}{2.511817in}}%
\pgfpathlineto{\pgfqpoint{3.733194in}{2.472800in}}%
\pgfpathlineto{\pgfqpoint{3.735518in}{2.502848in}}%
\pgfpathlineto{\pgfqpoint{3.737842in}{2.508498in}}%
\pgfpathlineto{\pgfqpoint{3.740166in}{2.483099in}}%
\pgfpathlineto{\pgfqpoint{3.742490in}{2.518556in}}%
\pgfpathlineto{\pgfqpoint{3.744814in}{2.482873in}}%
\pgfpathlineto{\pgfqpoint{3.747138in}{2.514255in}}%
\pgfpathlineto{\pgfqpoint{3.751786in}{2.479497in}}%
\pgfpathlineto{\pgfqpoint{3.754110in}{2.478124in}}%
\pgfpathlineto{\pgfqpoint{3.758757in}{2.507373in}}%
\pgfpathlineto{\pgfqpoint{3.761081in}{2.497247in}}%
\pgfpathlineto{\pgfqpoint{3.763405in}{2.479246in}}%
\pgfpathlineto{\pgfqpoint{3.765729in}{2.505726in}}%
\pgfpathlineto{\pgfqpoint{3.768053in}{2.505105in}}%
\pgfpathlineto{\pgfqpoint{3.770377in}{2.513216in}}%
\pgfpathlineto{\pgfqpoint{3.772701in}{2.512431in}}%
\pgfpathlineto{\pgfqpoint{3.775025in}{2.503480in}}%
\pgfpathlineto{\pgfqpoint{3.777349in}{2.499724in}}%
\pgfpathlineto{\pgfqpoint{3.781997in}{2.477219in}}%
\pgfpathlineto{\pgfqpoint{3.784321in}{2.485045in}}%
\pgfpathlineto{\pgfqpoint{3.786645in}{2.498717in}}%
\pgfpathlineto{\pgfqpoint{3.791293in}{2.487256in}}%
\pgfpathlineto{\pgfqpoint{3.793617in}{2.499239in}}%
\pgfpathlineto{\pgfqpoint{3.795941in}{2.495220in}}%
\pgfpathlineto{\pgfqpoint{3.800589in}{2.513029in}}%
\pgfpathlineto{\pgfqpoint{3.805237in}{2.485872in}}%
\pgfpathlineto{\pgfqpoint{3.807561in}{2.527024in}}%
\pgfpathlineto{\pgfqpoint{3.809885in}{2.495088in}}%
\pgfpathlineto{\pgfqpoint{3.812209in}{2.507438in}}%
\pgfpathlineto{\pgfqpoint{3.814533in}{2.498683in}}%
\pgfpathlineto{\pgfqpoint{3.816857in}{2.503183in}}%
\pgfpathlineto{\pgfqpoint{3.821505in}{2.515564in}}%
\pgfpathlineto{\pgfqpoint{3.823829in}{2.496196in}}%
\pgfpathlineto{\pgfqpoint{3.826153in}{2.504669in}}%
\pgfpathlineto{\pgfqpoint{3.828477in}{2.530831in}}%
\pgfpathlineto{\pgfqpoint{3.830800in}{2.498088in}}%
\pgfpathlineto{\pgfqpoint{3.833124in}{2.485689in}}%
\pgfpathlineto{\pgfqpoint{3.835448in}{2.506491in}}%
\pgfpathlineto{\pgfqpoint{3.837772in}{2.498475in}}%
\pgfpathlineto{\pgfqpoint{3.840096in}{2.477482in}}%
\pgfpathlineto{\pgfqpoint{3.842420in}{2.511182in}}%
\pgfpathlineto{\pgfqpoint{3.844744in}{2.519376in}}%
\pgfpathlineto{\pgfqpoint{3.847068in}{2.672678in}}%
\pgfpathlineto{\pgfqpoint{3.849392in}{2.717201in}}%
\pgfpathlineto{\pgfqpoint{3.851716in}{2.725190in}}%
\pgfpathlineto{\pgfqpoint{3.854040in}{2.749557in}}%
\pgfpathlineto{\pgfqpoint{3.856364in}{2.712154in}}%
\pgfpathlineto{\pgfqpoint{3.858688in}{2.734547in}}%
\pgfpathlineto{\pgfqpoint{3.861012in}{2.731019in}}%
\pgfpathlineto{\pgfqpoint{3.863336in}{2.720198in}}%
\pgfpathlineto{\pgfqpoint{3.865660in}{2.748897in}}%
\pgfpathlineto{\pgfqpoint{3.867984in}{2.713069in}}%
\pgfpathlineto{\pgfqpoint{3.870308in}{2.717664in}}%
\pgfpathlineto{\pgfqpoint{3.874956in}{2.698294in}}%
\pgfpathlineto{\pgfqpoint{3.877280in}{2.710500in}}%
\pgfpathlineto{\pgfqpoint{3.879604in}{2.743106in}}%
\pgfpathlineto{\pgfqpoint{3.881928in}{2.717272in}}%
\pgfpathlineto{\pgfqpoint{3.884252in}{2.728595in}}%
\pgfpathlineto{\pgfqpoint{3.886576in}{2.718847in}}%
\pgfpathlineto{\pgfqpoint{3.888900in}{2.723610in}}%
\pgfpathlineto{\pgfqpoint{3.891224in}{2.738846in}}%
\pgfpathlineto{\pgfqpoint{3.893548in}{2.718049in}}%
\pgfpathlineto{\pgfqpoint{3.895872in}{2.727264in}}%
\pgfpathlineto{\pgfqpoint{3.898196in}{2.700165in}}%
\pgfpathlineto{\pgfqpoint{3.900519in}{2.727604in}}%
\pgfpathlineto{\pgfqpoint{3.902843in}{2.733791in}}%
\pgfpathlineto{\pgfqpoint{3.905167in}{2.734810in}}%
\pgfpathlineto{\pgfqpoint{3.909815in}{2.720036in}}%
\pgfpathlineto{\pgfqpoint{3.912139in}{2.734415in}}%
\pgfpathlineto{\pgfqpoint{3.914463in}{2.710978in}}%
\pgfpathlineto{\pgfqpoint{3.916787in}{2.722435in}}%
\pgfpathlineto{\pgfqpoint{3.919111in}{2.719313in}}%
\pgfpathlineto{\pgfqpoint{3.921435in}{2.754163in}}%
\pgfpathlineto{\pgfqpoint{3.923759in}{2.767262in}}%
\pgfpathlineto{\pgfqpoint{3.926083in}{2.731483in}}%
\pgfpathlineto{\pgfqpoint{3.928407in}{2.738667in}}%
\pgfpathlineto{\pgfqpoint{3.930731in}{2.740100in}}%
\pgfpathlineto{\pgfqpoint{3.933055in}{2.729096in}}%
\pgfpathlineto{\pgfqpoint{3.935379in}{2.755603in}}%
\pgfpathlineto{\pgfqpoint{3.937703in}{2.733703in}}%
\pgfpathlineto{\pgfqpoint{3.942351in}{2.706313in}}%
\pgfpathlineto{\pgfqpoint{3.944675in}{2.750501in}}%
\pgfpathlineto{\pgfqpoint{3.946999in}{2.746573in}}%
\pgfpathlineto{\pgfqpoint{3.949323in}{2.712375in}}%
\pgfpathlineto{\pgfqpoint{3.951647in}{2.720328in}}%
\pgfpathlineto{\pgfqpoint{3.956295in}{2.755512in}}%
\pgfpathlineto{\pgfqpoint{3.958619in}{2.720474in}}%
\pgfpathlineto{\pgfqpoint{3.960943in}{2.759849in}}%
\pgfpathlineto{\pgfqpoint{3.963267in}{2.729942in}}%
\pgfpathlineto{\pgfqpoint{3.965591in}{2.730436in}}%
\pgfpathlineto{\pgfqpoint{3.967915in}{2.751045in}}%
\pgfpathlineto{\pgfqpoint{3.970238in}{2.732316in}}%
\pgfpathlineto{\pgfqpoint{3.972562in}{2.730847in}}%
\pgfpathlineto{\pgfqpoint{3.974886in}{2.726931in}}%
\pgfpathlineto{\pgfqpoint{3.977210in}{2.742825in}}%
\pgfpathlineto{\pgfqpoint{3.979534in}{2.729129in}}%
\pgfpathlineto{\pgfqpoint{3.981858in}{2.739549in}}%
\pgfpathlineto{\pgfqpoint{3.984182in}{2.758616in}}%
\pgfpathlineto{\pgfqpoint{3.986506in}{2.731931in}}%
\pgfpathlineto{\pgfqpoint{3.988830in}{2.735960in}}%
\pgfpathlineto{\pgfqpoint{3.991154in}{2.749725in}}%
\pgfpathlineto{\pgfqpoint{3.993478in}{2.732503in}}%
\pgfpathlineto{\pgfqpoint{3.995802in}{2.730786in}}%
\pgfpathlineto{\pgfqpoint{3.998126in}{2.727644in}}%
\pgfpathlineto{\pgfqpoint{4.000450in}{2.727921in}}%
\pgfpathlineto{\pgfqpoint{4.002774in}{2.737221in}}%
\pgfpathlineto{\pgfqpoint{4.005098in}{2.728783in}}%
\pgfpathlineto{\pgfqpoint{4.007422in}{2.728950in}}%
\pgfpathlineto{\pgfqpoint{4.009746in}{2.739046in}}%
\pgfpathlineto{\pgfqpoint{4.012070in}{2.743564in}}%
\pgfpathlineto{\pgfqpoint{4.014394in}{2.725085in}}%
\pgfpathlineto{\pgfqpoint{4.016718in}{2.747483in}}%
\pgfpathlineto{\pgfqpoint{4.019042in}{2.756694in}}%
\pgfpathlineto{\pgfqpoint{4.021366in}{2.729403in}}%
\pgfpathlineto{\pgfqpoint{4.023690in}{2.758660in}}%
\pgfpathlineto{\pgfqpoint{4.026014in}{2.737979in}}%
\pgfpathlineto{\pgfqpoint{4.028338in}{2.752046in}}%
\pgfpathlineto{\pgfqpoint{4.030662in}{2.725471in}}%
\pgfpathlineto{\pgfqpoint{4.032986in}{2.740932in}}%
\pgfpathlineto{\pgfqpoint{4.035310in}{2.733713in}}%
\pgfpathlineto{\pgfqpoint{4.037634in}{2.769651in}}%
\pgfpathlineto{\pgfqpoint{4.039957in}{2.769893in}}%
\pgfpathlineto{\pgfqpoint{4.042281in}{2.721380in}}%
\pgfpathlineto{\pgfqpoint{4.044605in}{2.755850in}}%
\pgfpathlineto{\pgfqpoint{4.049253in}{2.713079in}}%
\pgfpathlineto{\pgfqpoint{4.051577in}{2.747031in}}%
\pgfpathlineto{\pgfqpoint{4.053901in}{2.756305in}}%
\pgfpathlineto{\pgfqpoint{4.056225in}{2.735067in}}%
\pgfpathlineto{\pgfqpoint{4.058549in}{2.774882in}}%
\pgfpathlineto{\pgfqpoint{4.060873in}{2.762219in}}%
\pgfpathlineto{\pgfqpoint{4.063197in}{2.767505in}}%
\pgfpathlineto{\pgfqpoint{4.065521in}{2.717257in}}%
\pgfpathlineto{\pgfqpoint{4.067845in}{2.762363in}}%
\pgfpathlineto{\pgfqpoint{4.070169in}{2.760217in}}%
\pgfpathlineto{\pgfqpoint{4.072493in}{2.747781in}}%
\pgfpathlineto{\pgfqpoint{4.074817in}{2.753528in}}%
\pgfpathlineto{\pgfqpoint{4.077141in}{2.766025in}}%
\pgfpathlineto{\pgfqpoint{4.079465in}{2.738576in}}%
\pgfpathlineto{\pgfqpoint{4.081789in}{2.736880in}}%
\pgfpathlineto{\pgfqpoint{4.084113in}{2.746856in}}%
\pgfpathlineto{\pgfqpoint{4.086437in}{2.733760in}}%
\pgfpathlineto{\pgfqpoint{4.088761in}{2.728586in}}%
\pgfpathlineto{\pgfqpoint{4.091085in}{2.748218in}}%
\pgfpathlineto{\pgfqpoint{4.093409in}{2.748427in}}%
\pgfpathlineto{\pgfqpoint{4.095733in}{2.745539in}}%
\pgfpathlineto{\pgfqpoint{4.102705in}{2.711421in}}%
\pgfpathlineto{\pgfqpoint{4.105029in}{2.753812in}}%
\pgfpathlineto{\pgfqpoint{4.107353in}{2.767466in}}%
\pgfpathlineto{\pgfqpoint{4.109677in}{2.733939in}}%
\pgfpathlineto{\pgfqpoint{4.112000in}{2.763344in}}%
\pgfpathlineto{\pgfqpoint{4.114324in}{2.761274in}}%
\pgfpathlineto{\pgfqpoint{4.116648in}{2.719296in}}%
\pgfpathlineto{\pgfqpoint{4.118972in}{2.755386in}}%
\pgfpathlineto{\pgfqpoint{4.121296in}{2.724079in}}%
\pgfpathlineto{\pgfqpoint{4.123620in}{2.749111in}}%
\pgfpathlineto{\pgfqpoint{4.125944in}{2.743449in}}%
\pgfpathlineto{\pgfqpoint{4.130592in}{2.753997in}}%
\pgfpathlineto{\pgfqpoint{4.132916in}{2.722087in}}%
\pgfpathlineto{\pgfqpoint{4.135240in}{2.776719in}}%
\pgfpathlineto{\pgfqpoint{4.137564in}{2.457177in}}%
\pgfpathlineto{\pgfqpoint{4.139888in}{2.453771in}}%
\pgfpathlineto{\pgfqpoint{4.142212in}{2.469707in}}%
\pgfpathlineto{\pgfqpoint{4.144536in}{2.439669in}}%
\pgfpathlineto{\pgfqpoint{4.146860in}{2.431070in}}%
\pgfpathlineto{\pgfqpoint{4.149184in}{2.460516in}}%
\pgfpathlineto{\pgfqpoint{4.151508in}{2.446058in}}%
\pgfpathlineto{\pgfqpoint{4.153832in}{2.441262in}}%
\pgfpathlineto{\pgfqpoint{4.156156in}{2.483430in}}%
\pgfpathlineto{\pgfqpoint{4.160804in}{2.447123in}}%
\pgfpathlineto{\pgfqpoint{4.163128in}{2.474098in}}%
\pgfpathlineto{\pgfqpoint{4.165452in}{2.472754in}}%
\pgfpathlineto{\pgfqpoint{4.170100in}{2.449356in}}%
\pgfpathlineto{\pgfqpoint{4.172424in}{2.485716in}}%
\pgfpathlineto{\pgfqpoint{4.177072in}{2.464266in}}%
\pgfpathlineto{\pgfqpoint{4.179396in}{2.464871in}}%
\pgfpathlineto{\pgfqpoint{4.184043in}{2.462691in}}%
\pgfpathlineto{\pgfqpoint{4.186367in}{2.442434in}}%
\pgfpathlineto{\pgfqpoint{4.188691in}{2.476704in}}%
\pgfpathlineto{\pgfqpoint{4.191015in}{2.473180in}}%
\pgfpathlineto{\pgfqpoint{4.195663in}{2.458295in}}%
\pgfpathlineto{\pgfqpoint{4.197987in}{2.463302in}}%
\pgfpathlineto{\pgfqpoint{4.200311in}{2.454481in}}%
\pgfpathlineto{\pgfqpoint{4.202635in}{2.466421in}}%
\pgfpathlineto{\pgfqpoint{4.204959in}{2.448653in}}%
\pgfpathlineto{\pgfqpoint{4.207283in}{2.454243in}}%
\pgfpathlineto{\pgfqpoint{4.209607in}{2.479574in}}%
\pgfpathlineto{\pgfqpoint{4.211931in}{2.476051in}}%
\pgfpathlineto{\pgfqpoint{4.214255in}{2.451605in}}%
\pgfpathlineto{\pgfqpoint{4.216579in}{2.447359in}}%
\pgfpathlineto{\pgfqpoint{4.218903in}{2.455653in}}%
\pgfpathlineto{\pgfqpoint{4.221227in}{2.452338in}}%
\pgfpathlineto{\pgfqpoint{4.223551in}{2.430718in}}%
\pgfpathlineto{\pgfqpoint{4.225875in}{2.496624in}}%
\pgfpathlineto{\pgfqpoint{4.228199in}{2.496420in}}%
\pgfpathlineto{\pgfqpoint{4.230523in}{2.460896in}}%
\pgfpathlineto{\pgfqpoint{4.232847in}{2.459160in}}%
\pgfpathlineto{\pgfqpoint{4.235171in}{2.476744in}}%
\pgfpathlineto{\pgfqpoint{4.237495in}{2.465685in}}%
\pgfpathlineto{\pgfqpoint{4.239819in}{2.479850in}}%
\pgfpathlineto{\pgfqpoint{4.244467in}{2.458750in}}%
\pgfpathlineto{\pgfqpoint{4.246791in}{2.500226in}}%
\pgfpathlineto{\pgfqpoint{4.251438in}{2.483502in}}%
\pgfpathlineto{\pgfqpoint{4.253762in}{2.458130in}}%
\pgfpathlineto{\pgfqpoint{4.258410in}{2.485006in}}%
\pgfpathlineto{\pgfqpoint{4.260734in}{2.445067in}}%
\pgfpathlineto{\pgfqpoint{4.263058in}{2.484676in}}%
\pgfpathlineto{\pgfqpoint{4.265382in}{2.477210in}}%
\pgfpathlineto{\pgfqpoint{4.267706in}{2.494998in}}%
\pgfpathlineto{\pgfqpoint{4.272354in}{2.476378in}}%
\pgfpathlineto{\pgfqpoint{4.274678in}{2.494886in}}%
\pgfpathlineto{\pgfqpoint{4.277002in}{2.471950in}}%
\pgfpathlineto{\pgfqpoint{4.279326in}{2.465703in}}%
\pgfpathlineto{\pgfqpoint{4.281650in}{2.478741in}}%
\pgfpathlineto{\pgfqpoint{4.286298in}{2.470296in}}%
\pgfpathlineto{\pgfqpoint{4.288622in}{2.492544in}}%
\pgfpathlineto{\pgfqpoint{4.290946in}{2.486228in}}%
\pgfpathlineto{\pgfqpoint{4.293270in}{2.508464in}}%
\pgfpathlineto{\pgfqpoint{4.297918in}{2.487563in}}%
\pgfpathlineto{\pgfqpoint{4.300242in}{2.440660in}}%
\pgfpathlineto{\pgfqpoint{4.302566in}{2.465822in}}%
\pgfpathlineto{\pgfqpoint{4.304890in}{2.468534in}}%
\pgfpathlineto{\pgfqpoint{4.307214in}{2.503028in}}%
\pgfpathlineto{\pgfqpoint{4.309538in}{2.473276in}}%
\pgfpathlineto{\pgfqpoint{4.311862in}{2.477381in}}%
\pgfpathlineto{\pgfqpoint{4.314186in}{2.485511in}}%
\pgfpathlineto{\pgfqpoint{4.316510in}{2.506224in}}%
\pgfpathlineto{\pgfqpoint{4.318834in}{2.498249in}}%
\pgfpathlineto{\pgfqpoint{4.321158in}{2.499480in}}%
\pgfpathlineto{\pgfqpoint{4.323481in}{2.517064in}}%
\pgfpathlineto{\pgfqpoint{4.325805in}{2.488874in}}%
\pgfpathlineto{\pgfqpoint{4.328129in}{2.491946in}}%
\pgfpathlineto{\pgfqpoint{4.330453in}{2.492104in}}%
\pgfpathlineto{\pgfqpoint{4.332777in}{2.494300in}}%
\pgfpathlineto{\pgfqpoint{4.335101in}{2.511143in}}%
\pgfpathlineto{\pgfqpoint{4.337425in}{2.462417in}}%
\pgfpathlineto{\pgfqpoint{4.339749in}{2.472730in}}%
\pgfpathlineto{\pgfqpoint{4.342073in}{2.475480in}}%
\pgfpathlineto{\pgfqpoint{4.344397in}{2.515361in}}%
\pgfpathlineto{\pgfqpoint{4.346721in}{2.454789in}}%
\pgfpathlineto{\pgfqpoint{4.349045in}{2.489939in}}%
\pgfpathlineto{\pgfqpoint{4.351369in}{2.480862in}}%
\pgfpathlineto{\pgfqpoint{4.353693in}{2.480212in}}%
\pgfpathlineto{\pgfqpoint{4.356017in}{2.510967in}}%
\pgfpathlineto{\pgfqpoint{4.358341in}{2.479701in}}%
\pgfpathlineto{\pgfqpoint{4.360665in}{2.463980in}}%
\pgfpathlineto{\pgfqpoint{4.362989in}{2.512274in}}%
\pgfpathlineto{\pgfqpoint{4.365313in}{2.508060in}}%
\pgfpathlineto{\pgfqpoint{4.367637in}{2.497561in}}%
\pgfpathlineto{\pgfqpoint{4.369961in}{2.506567in}}%
\pgfpathlineto{\pgfqpoint{4.372285in}{2.489596in}}%
\pgfpathlineto{\pgfqpoint{4.374609in}{2.502280in}}%
\pgfpathlineto{\pgfqpoint{4.376933in}{2.497468in}}%
\pgfpathlineto{\pgfqpoint{4.379257in}{2.500441in}}%
\pgfpathlineto{\pgfqpoint{4.381581in}{2.469783in}}%
\pgfpathlineto{\pgfqpoint{4.383905in}{2.504402in}}%
\pgfpathlineto{\pgfqpoint{4.386229in}{2.490340in}}%
\pgfpathlineto{\pgfqpoint{4.388553in}{2.508481in}}%
\pgfpathlineto{\pgfqpoint{4.390877in}{2.508694in}}%
\pgfpathlineto{\pgfqpoint{4.393200in}{2.488304in}}%
\pgfpathlineto{\pgfqpoint{4.395524in}{2.501978in}}%
\pgfpathlineto{\pgfqpoint{4.397848in}{2.478416in}}%
\pgfpathlineto{\pgfqpoint{4.400172in}{2.506205in}}%
\pgfpathlineto{\pgfqpoint{4.402496in}{2.457403in}}%
\pgfpathlineto{\pgfqpoint{4.404820in}{2.531703in}}%
\pgfpathlineto{\pgfqpoint{4.407144in}{2.486506in}}%
\pgfpathlineto{\pgfqpoint{4.409468in}{2.512884in}}%
\pgfpathlineto{\pgfqpoint{4.411792in}{2.520360in}}%
\pgfpathlineto{\pgfqpoint{4.414116in}{2.497805in}}%
\pgfpathlineto{\pgfqpoint{4.416440in}{2.458655in}}%
\pgfpathlineto{\pgfqpoint{4.418764in}{2.496627in}}%
\pgfpathlineto{\pgfqpoint{4.421088in}{2.494893in}}%
\pgfpathlineto{\pgfqpoint{4.423412in}{2.500231in}}%
\pgfpathlineto{\pgfqpoint{4.425736in}{2.499554in}}%
\pgfpathlineto{\pgfqpoint{4.428060in}{2.727706in}}%
\pgfpathlineto{\pgfqpoint{4.430384in}{2.742903in}}%
\pgfpathlineto{\pgfqpoint{4.432708in}{2.707009in}}%
\pgfpathlineto{\pgfqpoint{4.435032in}{2.731319in}}%
\pgfpathlineto{\pgfqpoint{4.437356in}{2.727915in}}%
\pgfpathlineto{\pgfqpoint{4.439680in}{2.734534in}}%
\pgfpathlineto{\pgfqpoint{4.442004in}{2.734515in}}%
\pgfpathlineto{\pgfqpoint{4.444328in}{2.729046in}}%
\pgfpathlineto{\pgfqpoint{4.446652in}{2.747172in}}%
\pgfpathlineto{\pgfqpoint{4.448976in}{2.718684in}}%
\pgfpathlineto{\pgfqpoint{4.451300in}{2.735378in}}%
\pgfpathlineto{\pgfqpoint{4.453624in}{2.768895in}}%
\pgfpathlineto{\pgfqpoint{4.458272in}{2.742503in}}%
\pgfpathlineto{\pgfqpoint{4.462919in}{2.737890in}}%
\pgfpathlineto{\pgfqpoint{4.467567in}{2.722182in}}%
\pgfpathlineto{\pgfqpoint{4.469891in}{2.731047in}}%
\pgfpathlineto{\pgfqpoint{4.472215in}{2.772552in}}%
\pgfpathlineto{\pgfqpoint{4.474539in}{2.730561in}}%
\pgfpathlineto{\pgfqpoint{4.476863in}{2.753809in}}%
\pgfpathlineto{\pgfqpoint{4.481511in}{2.727637in}}%
\pgfpathlineto{\pgfqpoint{4.483835in}{2.737430in}}%
\pgfpathlineto{\pgfqpoint{4.486159in}{2.735423in}}%
\pgfpathlineto{\pgfqpoint{4.488483in}{2.726051in}}%
\pgfpathlineto{\pgfqpoint{4.490807in}{2.738084in}}%
\pgfpathlineto{\pgfqpoint{4.493131in}{2.720036in}}%
\pgfpathlineto{\pgfqpoint{4.495455in}{2.718689in}}%
\pgfpathlineto{\pgfqpoint{4.497779in}{2.750694in}}%
\pgfpathlineto{\pgfqpoint{4.500103in}{2.735442in}}%
\pgfpathlineto{\pgfqpoint{4.502427in}{2.729707in}}%
\pgfpathlineto{\pgfqpoint{4.504751in}{2.750522in}}%
\pgfpathlineto{\pgfqpoint{4.507075in}{2.745239in}}%
\pgfpathlineto{\pgfqpoint{4.509399in}{2.760080in}}%
\pgfpathlineto{\pgfqpoint{4.511723in}{2.764116in}}%
\pgfpathlineto{\pgfqpoint{4.514047in}{2.750373in}}%
\pgfpathlineto{\pgfqpoint{4.516371in}{2.749824in}}%
\pgfpathlineto{\pgfqpoint{4.518695in}{2.739779in}}%
\pgfpathlineto{\pgfqpoint{4.521019in}{2.734920in}}%
\pgfpathlineto{\pgfqpoint{4.523343in}{2.752351in}}%
\pgfpathlineto{\pgfqpoint{4.525667in}{2.751876in}}%
\pgfpathlineto{\pgfqpoint{4.527991in}{2.743600in}}%
\pgfpathlineto{\pgfqpoint{4.530315in}{2.743676in}}%
\pgfpathlineto{\pgfqpoint{4.532638in}{2.757976in}}%
\pgfpathlineto{\pgfqpoint{4.534962in}{2.754310in}}%
\pgfpathlineto{\pgfqpoint{4.537286in}{2.745583in}}%
\pgfpathlineto{\pgfqpoint{4.539610in}{2.743523in}}%
\pgfpathlineto{\pgfqpoint{4.541934in}{2.736636in}}%
\pgfpathlineto{\pgfqpoint{4.544258in}{2.765270in}}%
\pgfpathlineto{\pgfqpoint{4.546582in}{2.747033in}}%
\pgfpathlineto{\pgfqpoint{4.548906in}{2.760240in}}%
\pgfpathlineto{\pgfqpoint{4.551230in}{2.733306in}}%
\pgfpathlineto{\pgfqpoint{4.553554in}{2.745488in}}%
\pgfpathlineto{\pgfqpoint{4.555878in}{2.747050in}}%
\pgfpathlineto{\pgfqpoint{4.558202in}{2.756188in}}%
\pgfpathlineto{\pgfqpoint{4.560526in}{2.736659in}}%
\pgfpathlineto{\pgfqpoint{4.562850in}{2.791778in}}%
\pgfpathlineto{\pgfqpoint{4.565174in}{2.761571in}}%
\pgfpathlineto{\pgfqpoint{4.567498in}{2.760976in}}%
\pgfpathlineto{\pgfqpoint{4.572146in}{2.745216in}}%
\pgfpathlineto{\pgfqpoint{4.574470in}{2.762284in}}%
\pgfpathlineto{\pgfqpoint{4.576794in}{2.756236in}}%
\pgfpathlineto{\pgfqpoint{4.579118in}{2.775292in}}%
\pgfpathlineto{\pgfqpoint{4.581442in}{2.738352in}}%
\pgfpathlineto{\pgfqpoint{4.583766in}{2.752287in}}%
\pgfpathlineto{\pgfqpoint{4.586090in}{2.755018in}}%
\pgfpathlineto{\pgfqpoint{4.588414in}{2.750673in}}%
\pgfpathlineto{\pgfqpoint{4.590738in}{2.739104in}}%
\pgfpathlineto{\pgfqpoint{4.593062in}{2.785058in}}%
\pgfpathlineto{\pgfqpoint{4.597710in}{2.762607in}}%
\pgfpathlineto{\pgfqpoint{4.600034in}{2.777669in}}%
\pgfpathlineto{\pgfqpoint{4.602358in}{2.746000in}}%
\pgfpathlineto{\pgfqpoint{4.604681in}{2.787691in}}%
\pgfpathlineto{\pgfqpoint{4.607005in}{2.752090in}}%
\pgfpathlineto{\pgfqpoint{4.609329in}{2.750104in}}%
\pgfpathlineto{\pgfqpoint{4.611653in}{2.780750in}}%
\pgfpathlineto{\pgfqpoint{4.613977in}{2.779218in}}%
\pgfpathlineto{\pgfqpoint{4.616301in}{2.755925in}}%
\pgfpathlineto{\pgfqpoint{4.618625in}{2.779813in}}%
\pgfpathlineto{\pgfqpoint{4.620949in}{2.773521in}}%
\pgfpathlineto{\pgfqpoint{4.623273in}{2.800572in}}%
\pgfpathlineto{\pgfqpoint{4.627921in}{2.743382in}}%
\pgfpathlineto{\pgfqpoint{4.630245in}{2.748845in}}%
\pgfpathlineto{\pgfqpoint{4.632569in}{2.788717in}}%
\pgfpathlineto{\pgfqpoint{4.634893in}{2.762508in}}%
\pgfpathlineto{\pgfqpoint{4.637217in}{2.764368in}}%
\pgfpathlineto{\pgfqpoint{4.644189in}{2.795747in}}%
\pgfpathlineto{\pgfqpoint{4.646513in}{2.753849in}}%
\pgfpathlineto{\pgfqpoint{4.648837in}{2.804108in}}%
\pgfpathlineto{\pgfqpoint{4.651161in}{2.778524in}}%
\pgfpathlineto{\pgfqpoint{4.653485in}{2.738861in}}%
\pgfpathlineto{\pgfqpoint{4.655809in}{2.787919in}}%
\pgfpathlineto{\pgfqpoint{4.658133in}{2.765392in}}%
\pgfpathlineto{\pgfqpoint{4.660457in}{2.779088in}}%
\pgfpathlineto{\pgfqpoint{4.665105in}{2.771818in}}%
\pgfpathlineto{\pgfqpoint{4.667429in}{2.789880in}}%
\pgfpathlineto{\pgfqpoint{4.669753in}{2.796864in}}%
\pgfpathlineto{\pgfqpoint{4.672077in}{2.775687in}}%
\pgfpathlineto{\pgfqpoint{4.674400in}{2.781655in}}%
\pgfpathlineto{\pgfqpoint{4.676724in}{2.784227in}}%
\pgfpathlineto{\pgfqpoint{4.679048in}{2.796811in}}%
\pgfpathlineto{\pgfqpoint{4.681372in}{2.776336in}}%
\pgfpathlineto{\pgfqpoint{4.683696in}{2.773597in}}%
\pgfpathlineto{\pgfqpoint{4.686020in}{2.772942in}}%
\pgfpathlineto{\pgfqpoint{4.688344in}{2.798283in}}%
\pgfpathlineto{\pgfqpoint{4.690668in}{2.798887in}}%
\pgfpathlineto{\pgfqpoint{4.692992in}{2.813173in}}%
\pgfpathlineto{\pgfqpoint{4.695316in}{2.771323in}}%
\pgfpathlineto{\pgfqpoint{4.699964in}{2.805441in}}%
\pgfpathlineto{\pgfqpoint{4.702288in}{2.772672in}}%
\pgfpathlineto{\pgfqpoint{4.704612in}{2.799707in}}%
\pgfpathlineto{\pgfqpoint{4.706936in}{2.787197in}}%
\pgfpathlineto{\pgfqpoint{4.709260in}{2.785366in}}%
\pgfpathlineto{\pgfqpoint{4.711584in}{2.768044in}}%
\pgfpathlineto{\pgfqpoint{4.713908in}{2.787974in}}%
\pgfpathlineto{\pgfqpoint{4.716232in}{2.772666in}}%
\pgfpathlineto{\pgfqpoint{4.718556in}{2.483849in}}%
\pgfpathlineto{\pgfqpoint{4.720880in}{2.469527in}}%
\pgfpathlineto{\pgfqpoint{4.723204in}{2.477003in}}%
\pgfpathlineto{\pgfqpoint{4.725528in}{2.500720in}}%
\pgfpathlineto{\pgfqpoint{4.727852in}{2.484149in}}%
\pgfpathlineto{\pgfqpoint{4.730176in}{2.520991in}}%
\pgfpathlineto{\pgfqpoint{4.732500in}{2.477108in}}%
\pgfpathlineto{\pgfqpoint{4.734824in}{2.500477in}}%
\pgfpathlineto{\pgfqpoint{4.737148in}{2.512134in}}%
\pgfpathlineto{\pgfqpoint{4.741796in}{2.481976in}}%
\pgfpathlineto{\pgfqpoint{4.744119in}{2.471701in}}%
\pgfpathlineto{\pgfqpoint{4.746443in}{2.505449in}}%
\pgfpathlineto{\pgfqpoint{4.748767in}{2.470448in}}%
\pgfpathlineto{\pgfqpoint{4.751091in}{2.469580in}}%
\pgfpathlineto{\pgfqpoint{4.753415in}{2.505915in}}%
\pgfpathlineto{\pgfqpoint{4.755739in}{2.488996in}}%
\pgfpathlineto{\pgfqpoint{4.758063in}{2.503521in}}%
\pgfpathlineto{\pgfqpoint{4.760387in}{2.506291in}}%
\pgfpathlineto{\pgfqpoint{4.762711in}{2.489873in}}%
\pgfpathlineto{\pgfqpoint{4.765035in}{2.487762in}}%
\pgfpathlineto{\pgfqpoint{4.767359in}{2.493500in}}%
\pgfpathlineto{\pgfqpoint{4.769683in}{2.481851in}}%
\pgfpathlineto{\pgfqpoint{4.772007in}{2.514388in}}%
\pgfpathlineto{\pgfqpoint{4.774331in}{2.517186in}}%
\pgfpathlineto{\pgfqpoint{4.776655in}{2.527476in}}%
\pgfpathlineto{\pgfqpoint{4.778979in}{2.513677in}}%
\pgfpathlineto{\pgfqpoint{4.781303in}{2.516858in}}%
\pgfpathlineto{\pgfqpoint{4.783627in}{2.494557in}}%
\pgfpathlineto{\pgfqpoint{4.785951in}{2.489908in}}%
\pgfpathlineto{\pgfqpoint{4.788275in}{2.522471in}}%
\pgfpathlineto{\pgfqpoint{4.790599in}{2.482805in}}%
\pgfpathlineto{\pgfqpoint{4.792923in}{2.533332in}}%
\pgfpathlineto{\pgfqpoint{4.795247in}{2.484545in}}%
\pgfpathlineto{\pgfqpoint{4.797571in}{2.494677in}}%
\pgfpathlineto{\pgfqpoint{4.799895in}{2.529557in}}%
\pgfpathlineto{\pgfqpoint{4.802219in}{2.480828in}}%
\pgfpathlineto{\pgfqpoint{4.804543in}{2.527233in}}%
\pgfpathlineto{\pgfqpoint{4.806867in}{2.515370in}}%
\pgfpathlineto{\pgfqpoint{4.809191in}{2.482436in}}%
\pgfpathlineto{\pgfqpoint{4.811515in}{2.525424in}}%
\pgfpathlineto{\pgfqpoint{4.813838in}{2.516203in}}%
\pgfpathlineto{\pgfqpoint{4.816162in}{2.493947in}}%
\pgfpathlineto{\pgfqpoint{4.818486in}{2.546633in}}%
\pgfpathlineto{\pgfqpoint{4.820810in}{2.512474in}}%
\pgfpathlineto{\pgfqpoint{4.823134in}{2.513795in}}%
\pgfpathlineto{\pgfqpoint{4.825458in}{2.493645in}}%
\pgfpathlineto{\pgfqpoint{4.827782in}{2.520003in}}%
\pgfpathlineto{\pgfqpoint{4.830106in}{2.503933in}}%
\pgfpathlineto{\pgfqpoint{4.832430in}{2.496636in}}%
\pgfpathlineto{\pgfqpoint{4.834754in}{2.526743in}}%
\pgfpathlineto{\pgfqpoint{4.837078in}{2.492073in}}%
\pgfpathlineto{\pgfqpoint{4.841726in}{2.510225in}}%
\pgfpathlineto{\pgfqpoint{4.844050in}{2.511326in}}%
\pgfpathlineto{\pgfqpoint{4.848698in}{2.544977in}}%
\pgfpathlineto{\pgfqpoint{4.851022in}{2.502450in}}%
\pgfpathlineto{\pgfqpoint{4.853346in}{2.505685in}}%
\pgfpathlineto{\pgfqpoint{4.855670in}{2.529216in}}%
\pgfpathlineto{\pgfqpoint{4.857994in}{2.491031in}}%
\pgfpathlineto{\pgfqpoint{4.860318in}{2.537985in}}%
\pgfpathlineto{\pgfqpoint{4.864966in}{2.506458in}}%
\pgfpathlineto{\pgfqpoint{4.867290in}{2.538308in}}%
\pgfpathlineto{\pgfqpoint{4.869614in}{2.511338in}}%
\pgfpathlineto{\pgfqpoint{4.871938in}{2.502159in}}%
\pgfpathlineto{\pgfqpoint{4.874262in}{2.522239in}}%
\pgfpathlineto{\pgfqpoint{4.876586in}{2.532279in}}%
\pgfpathlineto{\pgfqpoint{4.878910in}{2.529000in}}%
\pgfpathlineto{\pgfqpoint{4.881234in}{2.519014in}}%
\pgfpathlineto{\pgfqpoint{4.883558in}{2.540058in}}%
\pgfpathlineto{\pgfqpoint{4.885881in}{2.532758in}}%
\pgfpathlineto{\pgfqpoint{4.888205in}{2.494019in}}%
\pgfpathlineto{\pgfqpoint{4.890529in}{2.523785in}}%
\pgfpathlineto{\pgfqpoint{4.892853in}{2.515372in}}%
\pgfpathlineto{\pgfqpoint{4.895177in}{2.534848in}}%
\pgfpathlineto{\pgfqpoint{4.897501in}{2.532744in}}%
\pgfpathlineto{\pgfqpoint{4.899825in}{2.498178in}}%
\pgfpathlineto{\pgfqpoint{4.902149in}{2.516153in}}%
\pgfpathlineto{\pgfqpoint{4.904473in}{2.503416in}}%
\pgfpathlineto{\pgfqpoint{4.906797in}{2.553151in}}%
\pgfpathlineto{\pgfqpoint{4.909121in}{2.505453in}}%
\pgfpathlineto{\pgfqpoint{4.911445in}{2.535144in}}%
\pgfpathlineto{\pgfqpoint{4.913769in}{2.526440in}}%
\pgfpathlineto{\pgfqpoint{4.916093in}{2.528423in}}%
\pgfpathlineto{\pgfqpoint{4.918417in}{2.533543in}}%
\pgfpathlineto{\pgfqpoint{4.920741in}{2.510619in}}%
\pgfpathlineto{\pgfqpoint{4.923065in}{2.538870in}}%
\pgfpathlineto{\pgfqpoint{4.925389in}{2.519944in}}%
\pgfpathlineto{\pgfqpoint{4.927713in}{2.556498in}}%
\pgfpathlineto{\pgfqpoint{4.930037in}{2.523621in}}%
\pgfpathlineto{\pgfqpoint{4.932361in}{2.546194in}}%
\pgfpathlineto{\pgfqpoint{4.934685in}{2.507736in}}%
\pgfpathlineto{\pgfqpoint{4.937009in}{2.517835in}}%
\pgfpathlineto{\pgfqpoint{4.939333in}{2.549520in}}%
\pgfpathlineto{\pgfqpoint{4.941657in}{2.485062in}}%
\pgfpathlineto{\pgfqpoint{4.943981in}{2.529398in}}%
\pgfpathlineto{\pgfqpoint{4.946305in}{2.542349in}}%
\pgfpathlineto{\pgfqpoint{4.948629in}{2.510603in}}%
\pgfpathlineto{\pgfqpoint{4.950953in}{2.564825in}}%
\pgfpathlineto{\pgfqpoint{4.953277in}{2.528668in}}%
\pgfpathlineto{\pgfqpoint{4.955600in}{2.514507in}}%
\pgfpathlineto{\pgfqpoint{4.957924in}{2.517531in}}%
\pgfpathlineto{\pgfqpoint{4.960248in}{2.533617in}}%
\pgfpathlineto{\pgfqpoint{4.962572in}{2.514511in}}%
\pgfpathlineto{\pgfqpoint{4.964896in}{2.550440in}}%
\pgfpathlineto{\pgfqpoint{4.967220in}{2.542871in}}%
\pgfpathlineto{\pgfqpoint{4.971868in}{2.513029in}}%
\pgfpathlineto{\pgfqpoint{4.974192in}{2.520670in}}%
\pgfpathlineto{\pgfqpoint{4.976516in}{2.538428in}}%
\pgfpathlineto{\pgfqpoint{4.978840in}{2.520771in}}%
\pgfpathlineto{\pgfqpoint{4.981164in}{2.541650in}}%
\pgfpathlineto{\pgfqpoint{4.983488in}{2.537104in}}%
\pgfpathlineto{\pgfqpoint{4.985812in}{2.555766in}}%
\pgfpathlineto{\pgfqpoint{4.988136in}{2.535249in}}%
\pgfpathlineto{\pgfqpoint{4.990460in}{2.529295in}}%
\pgfpathlineto{\pgfqpoint{4.992784in}{2.507883in}}%
\pgfpathlineto{\pgfqpoint{4.995108in}{2.538271in}}%
\pgfpathlineto{\pgfqpoint{4.997432in}{2.549949in}}%
\pgfpathlineto{\pgfqpoint{4.999756in}{2.540772in}}%
\pgfpathlineto{\pgfqpoint{5.004404in}{2.567968in}}%
\pgfpathlineto{\pgfqpoint{5.006728in}{2.527494in}}%
\pgfpathlineto{\pgfqpoint{5.009052in}{2.774188in}}%
\pgfpathlineto{\pgfqpoint{5.011376in}{2.782281in}}%
\pgfpathlineto{\pgfqpoint{5.013700in}{2.765574in}}%
\pgfpathlineto{\pgfqpoint{5.016024in}{2.806587in}}%
\pgfpathlineto{\pgfqpoint{5.018348in}{2.778919in}}%
\pgfpathlineto{\pgfqpoint{5.020672in}{2.789266in}}%
\pgfpathlineto{\pgfqpoint{5.022996in}{2.765136in}}%
\pgfpathlineto{\pgfqpoint{5.025319in}{2.771692in}}%
\pgfpathlineto{\pgfqpoint{5.027643in}{2.795811in}}%
\pgfpathlineto{\pgfqpoint{5.029967in}{2.783807in}}%
\pgfpathlineto{\pgfqpoint{5.034615in}{2.774122in}}%
\pgfpathlineto{\pgfqpoint{5.036939in}{2.762456in}}%
\pgfpathlineto{\pgfqpoint{5.039263in}{2.776800in}}%
\pgfpathlineto{\pgfqpoint{5.041587in}{2.752505in}}%
\pgfpathlineto{\pgfqpoint{5.043911in}{2.749518in}}%
\pgfpathlineto{\pgfqpoint{5.046235in}{2.777591in}}%
\pgfpathlineto{\pgfqpoint{5.048559in}{2.776031in}}%
\pgfpathlineto{\pgfqpoint{5.050883in}{2.792190in}}%
\pgfpathlineto{\pgfqpoint{5.053207in}{2.771982in}}%
\pgfpathlineto{\pgfqpoint{5.057855in}{2.799673in}}%
\pgfpathlineto{\pgfqpoint{5.060179in}{2.801639in}}%
\pgfpathlineto{\pgfqpoint{5.062503in}{2.770292in}}%
\pgfpathlineto{\pgfqpoint{5.064827in}{2.802461in}}%
\pgfpathlineto{\pgfqpoint{5.069475in}{2.756856in}}%
\pgfpathlineto{\pgfqpoint{5.071799in}{2.769878in}}%
\pgfpathlineto{\pgfqpoint{5.074123in}{2.752188in}}%
\pgfpathlineto{\pgfqpoint{5.076447in}{2.778346in}}%
\pgfpathlineto{\pgfqpoint{5.081095in}{2.776763in}}%
\pgfpathlineto{\pgfqpoint{5.083419in}{2.771657in}}%
\pgfpathlineto{\pgfqpoint{5.085743in}{2.777091in}}%
\pgfpathlineto{\pgfqpoint{5.090391in}{2.763560in}}%
\pgfpathlineto{\pgfqpoint{5.092715in}{2.785697in}}%
\pgfpathlineto{\pgfqpoint{5.095039in}{2.768687in}}%
\pgfpathlineto{\pgfqpoint{5.097362in}{2.781805in}}%
\pgfpathlineto{\pgfqpoint{5.099686in}{2.770895in}}%
\pgfpathlineto{\pgfqpoint{5.102010in}{2.770768in}}%
\pgfpathlineto{\pgfqpoint{5.104334in}{2.785563in}}%
\pgfpathlineto{\pgfqpoint{5.108982in}{2.760898in}}%
\pgfpathlineto{\pgfqpoint{5.111306in}{2.783914in}}%
\pgfpathlineto{\pgfqpoint{5.113630in}{2.792510in}}%
\pgfpathlineto{\pgfqpoint{5.115954in}{2.787538in}}%
\pgfpathlineto{\pgfqpoint{5.118278in}{2.762749in}}%
\pgfpathlineto{\pgfqpoint{5.120602in}{2.788765in}}%
\pgfpathlineto{\pgfqpoint{5.122926in}{2.773164in}}%
\pgfpathlineto{\pgfqpoint{5.125250in}{2.791777in}}%
\pgfpathlineto{\pgfqpoint{5.127574in}{2.797664in}}%
\pgfpathlineto{\pgfqpoint{5.129898in}{2.808101in}}%
\pgfpathlineto{\pgfqpoint{5.134546in}{2.790300in}}%
\pgfpathlineto{\pgfqpoint{5.136870in}{2.786613in}}%
\pgfpathlineto{\pgfqpoint{5.139194in}{2.800732in}}%
\pgfpathlineto{\pgfqpoint{5.141518in}{2.775049in}}%
\pgfpathlineto{\pgfqpoint{5.143842in}{2.772045in}}%
\pgfpathlineto{\pgfqpoint{5.146166in}{2.783449in}}%
\pgfpathlineto{\pgfqpoint{5.148490in}{2.814376in}}%
\pgfpathlineto{\pgfqpoint{5.150814in}{2.779530in}}%
\pgfpathlineto{\pgfqpoint{5.153138in}{2.789934in}}%
\pgfpathlineto{\pgfqpoint{5.157786in}{2.778646in}}%
\pgfpathlineto{\pgfqpoint{5.160110in}{2.757751in}}%
\pgfpathlineto{\pgfqpoint{5.162434in}{2.766357in}}%
\pgfpathlineto{\pgfqpoint{5.164758in}{2.781400in}}%
\pgfpathlineto{\pgfqpoint{5.167081in}{2.783634in}}%
\pgfpathlineto{\pgfqpoint{5.169405in}{2.755280in}}%
\pgfpathlineto{\pgfqpoint{5.171729in}{2.817780in}}%
\pgfpathlineto{\pgfqpoint{5.174053in}{2.767966in}}%
\pgfpathlineto{\pgfqpoint{5.176377in}{2.769233in}}%
\pgfpathlineto{\pgfqpoint{5.178701in}{2.789688in}}%
\pgfpathlineto{\pgfqpoint{5.181025in}{2.781747in}}%
\pgfpathlineto{\pgfqpoint{5.185673in}{2.804952in}}%
\pgfpathlineto{\pgfqpoint{5.190321in}{2.777423in}}%
\pgfpathlineto{\pgfqpoint{5.192645in}{2.784745in}}%
\pgfpathlineto{\pgfqpoint{5.194969in}{2.796275in}}%
\pgfpathlineto{\pgfqpoint{5.197293in}{2.797356in}}%
\pgfpathlineto{\pgfqpoint{5.199617in}{2.792925in}}%
\pgfpathlineto{\pgfqpoint{5.201941in}{2.791279in}}%
\pgfpathlineto{\pgfqpoint{5.204265in}{2.791645in}}%
\pgfpathlineto{\pgfqpoint{5.206589in}{2.764537in}}%
\pgfpathlineto{\pgfqpoint{5.208913in}{2.785918in}}%
\pgfpathlineto{\pgfqpoint{5.211237in}{2.780907in}}%
\pgfpathlineto{\pgfqpoint{5.213561in}{2.807891in}}%
\pgfpathlineto{\pgfqpoint{5.215885in}{2.770301in}}%
\pgfpathlineto{\pgfqpoint{5.218209in}{2.776128in}}%
\pgfpathlineto{\pgfqpoint{5.220533in}{2.771170in}}%
\pgfpathlineto{\pgfqpoint{5.222857in}{2.813691in}}%
\pgfpathlineto{\pgfqpoint{5.225181in}{2.773243in}}%
\pgfpathlineto{\pgfqpoint{5.227505in}{2.777660in}}%
\pgfpathlineto{\pgfqpoint{5.229829in}{2.787759in}}%
\pgfpathlineto{\pgfqpoint{5.232153in}{2.807229in}}%
\pgfpathlineto{\pgfqpoint{5.234477in}{2.797640in}}%
\pgfpathlineto{\pgfqpoint{5.236800in}{2.777898in}}%
\pgfpathlineto{\pgfqpoint{5.239124in}{2.816085in}}%
\pgfpathlineto{\pgfqpoint{5.241448in}{2.823939in}}%
\pgfpathlineto{\pgfqpoint{5.243772in}{2.792215in}}%
\pgfpathlineto{\pgfqpoint{5.246096in}{2.788321in}}%
\pgfpathlineto{\pgfqpoint{5.248420in}{2.786823in}}%
\pgfpathlineto{\pgfqpoint{5.250744in}{2.775623in}}%
\pgfpathlineto{\pgfqpoint{5.253068in}{2.804223in}}%
\pgfpathlineto{\pgfqpoint{5.255392in}{2.778955in}}%
\pgfpathlineto{\pgfqpoint{5.257716in}{2.776671in}}%
\pgfpathlineto{\pgfqpoint{5.262364in}{2.791241in}}%
\pgfpathlineto{\pgfqpoint{5.264688in}{2.767746in}}%
\pgfpathlineto{\pgfqpoint{5.267012in}{2.786039in}}%
\pgfpathlineto{\pgfqpoint{5.269336in}{2.765937in}}%
\pgfpathlineto{\pgfqpoint{5.271660in}{2.785004in}}%
\pgfpathlineto{\pgfqpoint{5.273984in}{2.751403in}}%
\pgfpathlineto{\pgfqpoint{5.276308in}{2.802135in}}%
\pgfpathlineto{\pgfqpoint{5.278632in}{2.796292in}}%
\pgfpathlineto{\pgfqpoint{5.280956in}{2.803813in}}%
\pgfpathlineto{\pgfqpoint{5.283280in}{2.815239in}}%
\pgfpathlineto{\pgfqpoint{5.285604in}{2.789690in}}%
\pgfpathlineto{\pgfqpoint{5.290252in}{2.824618in}}%
\pgfpathlineto{\pgfqpoint{5.292576in}{2.781666in}}%
\pgfpathlineto{\pgfqpoint{5.294900in}{2.818764in}}%
\pgfpathlineto{\pgfqpoint{5.297224in}{2.773385in}}%
\pgfpathlineto{\pgfqpoint{5.299548in}{2.482571in}}%
\pgfpathlineto{\pgfqpoint{5.301872in}{2.517982in}}%
\pgfpathlineto{\pgfqpoint{5.304196in}{2.501206in}}%
\pgfpathlineto{\pgfqpoint{5.306519in}{2.504171in}}%
\pgfpathlineto{\pgfqpoint{5.308843in}{2.495375in}}%
\pgfpathlineto{\pgfqpoint{5.311167in}{2.493784in}}%
\pgfpathlineto{\pgfqpoint{5.313491in}{2.499424in}}%
\pgfpathlineto{\pgfqpoint{5.315815in}{2.532811in}}%
\pgfpathlineto{\pgfqpoint{5.318139in}{2.515611in}}%
\pgfpathlineto{\pgfqpoint{5.320463in}{2.519093in}}%
\pgfpathlineto{\pgfqpoint{5.322787in}{2.482053in}}%
\pgfpathlineto{\pgfqpoint{5.325111in}{2.508387in}}%
\pgfpathlineto{\pgfqpoint{5.327435in}{2.507457in}}%
\pgfpathlineto{\pgfqpoint{5.329759in}{2.499452in}}%
\pgfpathlineto{\pgfqpoint{5.332083in}{2.504888in}}%
\pgfpathlineto{\pgfqpoint{5.334407in}{2.461549in}}%
\pgfpathlineto{\pgfqpoint{5.336731in}{2.501788in}}%
\pgfpathlineto{\pgfqpoint{5.339055in}{2.506492in}}%
\pgfpathlineto{\pgfqpoint{5.341379in}{2.539901in}}%
\pgfpathlineto{\pgfqpoint{5.346027in}{2.488816in}}%
\pgfpathlineto{\pgfqpoint{5.348351in}{2.490964in}}%
\pgfpathlineto{\pgfqpoint{5.350675in}{2.521981in}}%
\pgfpathlineto{\pgfqpoint{5.352999in}{2.491786in}}%
\pgfpathlineto{\pgfqpoint{5.355323in}{2.528753in}}%
\pgfpathlineto{\pgfqpoint{5.357647in}{2.486641in}}%
\pgfpathlineto{\pgfqpoint{5.359971in}{2.508735in}}%
\pgfpathlineto{\pgfqpoint{5.362295in}{2.504586in}}%
\pgfpathlineto{\pgfqpoint{5.364619in}{2.463239in}}%
\pgfpathlineto{\pgfqpoint{5.366943in}{2.492568in}}%
\pgfpathlineto{\pgfqpoint{5.369267in}{2.498308in}}%
\pgfpathlineto{\pgfqpoint{5.371591in}{2.495946in}}%
\pgfpathlineto{\pgfqpoint{5.373915in}{2.486710in}}%
\pgfpathlineto{\pgfqpoint{5.378562in}{2.497623in}}%
\pgfpathlineto{\pgfqpoint{5.383210in}{2.512419in}}%
\pgfpathlineto{\pgfqpoint{5.385534in}{2.523576in}}%
\pgfpathlineto{\pgfqpoint{5.387858in}{2.514346in}}%
\pgfpathlineto{\pgfqpoint{5.390182in}{2.482113in}}%
\pgfpathlineto{\pgfqpoint{5.392506in}{2.486348in}}%
\pgfpathlineto{\pgfqpoint{5.394830in}{2.538919in}}%
\pgfpathlineto{\pgfqpoint{5.399478in}{2.487431in}}%
\pgfpathlineto{\pgfqpoint{5.401802in}{2.482267in}}%
\pgfpathlineto{\pgfqpoint{5.404126in}{2.516046in}}%
\pgfpathlineto{\pgfqpoint{5.406450in}{2.475792in}}%
\pgfpathlineto{\pgfqpoint{5.408774in}{2.530809in}}%
\pgfpathlineto{\pgfqpoint{5.411098in}{2.531898in}}%
\pgfpathlineto{\pgfqpoint{5.415746in}{2.478134in}}%
\pgfpathlineto{\pgfqpoint{5.418070in}{2.508108in}}%
\pgfpathlineto{\pgfqpoint{5.420394in}{2.496063in}}%
\pgfpathlineto{\pgfqpoint{5.422718in}{2.505089in}}%
\pgfpathlineto{\pgfqpoint{5.427366in}{2.492213in}}%
\pgfpathlineto{\pgfqpoint{5.432014in}{2.507733in}}%
\pgfpathlineto{\pgfqpoint{5.434338in}{2.517756in}}%
\pgfpathlineto{\pgfqpoint{5.436662in}{2.469614in}}%
\pgfpathlineto{\pgfqpoint{5.438986in}{2.525674in}}%
\pgfpathlineto{\pgfqpoint{5.441310in}{2.509610in}}%
\pgfpathlineto{\pgfqpoint{5.443634in}{2.511315in}}%
\pgfpathlineto{\pgfqpoint{5.445958in}{2.511043in}}%
\pgfpathlineto{\pgfqpoint{5.448281in}{2.509560in}}%
\pgfpathlineto{\pgfqpoint{5.450605in}{2.514914in}}%
\pgfpathlineto{\pgfqpoint{5.452929in}{2.493713in}}%
\pgfpathlineto{\pgfqpoint{5.455253in}{2.527084in}}%
\pgfpathlineto{\pgfqpoint{5.457577in}{2.498245in}}%
\pgfpathlineto{\pgfqpoint{5.459901in}{2.525096in}}%
\pgfpathlineto{\pgfqpoint{5.462225in}{2.504323in}}%
\pgfpathlineto{\pgfqpoint{5.464549in}{2.494270in}}%
\pgfpathlineto{\pgfqpoint{5.466873in}{2.514657in}}%
\pgfpathlineto{\pgfqpoint{5.469197in}{2.523530in}}%
\pgfpathlineto{\pgfqpoint{5.471521in}{2.521109in}}%
\pgfpathlineto{\pgfqpoint{5.473845in}{2.505938in}}%
\pgfpathlineto{\pgfqpoint{5.476169in}{2.511462in}}%
\pgfpathlineto{\pgfqpoint{5.478493in}{2.520685in}}%
\pgfpathlineto{\pgfqpoint{5.480817in}{2.486079in}}%
\pgfpathlineto{\pgfqpoint{5.485465in}{2.521767in}}%
\pgfpathlineto{\pgfqpoint{5.490113in}{2.473373in}}%
\pgfpathlineto{\pgfqpoint{5.492437in}{2.516360in}}%
\pgfpathlineto{\pgfqpoint{5.494761in}{2.494457in}}%
\pgfpathlineto{\pgfqpoint{5.497085in}{2.504217in}}%
\pgfpathlineto{\pgfqpoint{5.499409in}{2.493046in}}%
\pgfpathlineto{\pgfqpoint{5.501733in}{2.488100in}}%
\pgfpathlineto{\pgfqpoint{5.504057in}{2.504436in}}%
\pgfpathlineto{\pgfqpoint{5.506381in}{2.540996in}}%
\pgfpathlineto{\pgfqpoint{5.508705in}{2.506975in}}%
\pgfpathlineto{\pgfqpoint{5.511029in}{2.496941in}}%
\pgfpathlineto{\pgfqpoint{5.513353in}{2.509300in}}%
\pgfpathlineto{\pgfqpoint{5.515677in}{2.486762in}}%
\pgfpathlineto{\pgfqpoint{5.518000in}{2.504725in}}%
\pgfpathlineto{\pgfqpoint{5.520324in}{2.534382in}}%
\pgfpathlineto{\pgfqpoint{5.522648in}{2.515691in}}%
\pgfpathlineto{\pgfqpoint{5.524972in}{2.507006in}}%
\pgfpathlineto{\pgfqpoint{5.527296in}{2.486326in}}%
\pgfpathlineto{\pgfqpoint{5.529620in}{2.526219in}}%
\pgfpathlineto{\pgfqpoint{5.531944in}{2.503661in}}%
\pgfpathlineto{\pgfqpoint{5.534268in}{2.499202in}}%
\pgfpathlineto{\pgfqpoint{5.536592in}{2.505547in}}%
\pgfpathlineto{\pgfqpoint{5.538916in}{2.503520in}}%
\pgfpathlineto{\pgfqpoint{5.541240in}{2.493402in}}%
\pgfpathlineto{\pgfqpoint{5.543564in}{2.501847in}}%
\pgfpathlineto{\pgfqpoint{5.545888in}{2.537610in}}%
\pgfpathlineto{\pgfqpoint{5.548212in}{2.494876in}}%
\pgfpathlineto{\pgfqpoint{5.550536in}{2.514491in}}%
\pgfpathlineto{\pgfqpoint{5.552860in}{2.496187in}}%
\pgfpathlineto{\pgfqpoint{5.555184in}{2.500012in}}%
\pgfpathlineto{\pgfqpoint{5.557508in}{2.538243in}}%
\pgfpathlineto{\pgfqpoint{5.559832in}{2.498205in}}%
\pgfpathlineto{\pgfqpoint{5.562156in}{2.487012in}}%
\pgfpathlineto{\pgfqpoint{5.564480in}{2.513082in}}%
\pgfpathlineto{\pgfqpoint{5.566804in}{2.507235in}}%
\pgfpathlineto{\pgfqpoint{5.569128in}{2.508530in}}%
\pgfpathlineto{\pgfqpoint{5.571452in}{2.505602in}}%
\pgfpathlineto{\pgfqpoint{5.573776in}{2.499035in}}%
\pgfpathlineto{\pgfqpoint{5.576100in}{2.498331in}}%
\pgfpathlineto{\pgfqpoint{5.578424in}{2.520618in}}%
\pgfpathlineto{\pgfqpoint{5.580748in}{2.526019in}}%
\pgfpathlineto{\pgfqpoint{5.583072in}{2.502629in}}%
\pgfpathlineto{\pgfqpoint{5.585396in}{2.509371in}}%
\pgfpathlineto{\pgfqpoint{5.587720in}{2.737664in}}%
\pgfpathlineto{\pgfqpoint{5.587720in}{2.737664in}}%
\pgfusepath{stroke}%
\end{pgfscope}%
\begin{pgfscope}%
\pgfpathrectangle{\pgfqpoint{0.709829in}{2.192315in}}{\pgfqpoint{5.110171in}{0.887537in}}%
\pgfusepath{clip}%
\pgfsetroundcap%
\pgfsetroundjoin%
\pgfsetlinewidth{1.003750pt}%
\definecolor{currentstroke}{rgb}{0.333333,0.658824,0.407843}%
\pgfsetstrokecolor{currentstroke}%
\pgfsetdash{}{0pt}%
\pgfpathmoveto{\pgfqpoint{0.942110in}{2.395881in}}%
\pgfpathlineto{\pgfqpoint{0.944433in}{2.441485in}}%
\pgfpathlineto{\pgfqpoint{0.946757in}{2.378360in}}%
\pgfpathlineto{\pgfqpoint{0.949081in}{2.382500in}}%
\pgfpathlineto{\pgfqpoint{0.951405in}{2.388972in}}%
\pgfpathlineto{\pgfqpoint{0.953729in}{2.421638in}}%
\pgfpathlineto{\pgfqpoint{0.956053in}{2.364986in}}%
\pgfpathlineto{\pgfqpoint{0.958377in}{2.415360in}}%
\pgfpathlineto{\pgfqpoint{0.960701in}{2.378537in}}%
\pgfpathlineto{\pgfqpoint{0.963025in}{2.428369in}}%
\pgfpathlineto{\pgfqpoint{0.965349in}{2.406117in}}%
\pgfpathlineto{\pgfqpoint{0.967673in}{2.365822in}}%
\pgfpathlineto{\pgfqpoint{0.969997in}{2.445394in}}%
\pgfpathlineto{\pgfqpoint{0.972321in}{2.373576in}}%
\pgfpathlineto{\pgfqpoint{0.974645in}{2.435985in}}%
\pgfpathlineto{\pgfqpoint{0.976969in}{2.372737in}}%
\pgfpathlineto{\pgfqpoint{0.981617in}{2.407002in}}%
\pgfpathlineto{\pgfqpoint{0.983941in}{2.428856in}}%
\pgfpathlineto{\pgfqpoint{0.986265in}{2.408791in}}%
\pgfpathlineto{\pgfqpoint{0.988589in}{2.407317in}}%
\pgfpathlineto{\pgfqpoint{0.990913in}{2.411553in}}%
\pgfpathlineto{\pgfqpoint{0.993237in}{2.422827in}}%
\pgfpathlineto{\pgfqpoint{0.995561in}{2.452112in}}%
\pgfpathlineto{\pgfqpoint{0.997885in}{2.418081in}}%
\pgfpathlineto{\pgfqpoint{1.000209in}{2.408956in}}%
\pgfpathlineto{\pgfqpoint{1.007181in}{2.513038in}}%
\pgfpathlineto{\pgfqpoint{1.009505in}{2.421529in}}%
\pgfpathlineto{\pgfqpoint{1.011829in}{2.479826in}}%
\pgfpathlineto{\pgfqpoint{1.014153in}{2.476946in}}%
\pgfpathlineto{\pgfqpoint{1.016476in}{2.504746in}}%
\pgfpathlineto{\pgfqpoint{1.018800in}{2.514690in}}%
\pgfpathlineto{\pgfqpoint{1.021124in}{2.475019in}}%
\pgfpathlineto{\pgfqpoint{1.023448in}{2.480444in}}%
\pgfpathlineto{\pgfqpoint{1.025772in}{2.531012in}}%
\pgfpathlineto{\pgfqpoint{1.028096in}{2.462878in}}%
\pgfpathlineto{\pgfqpoint{1.030420in}{2.494161in}}%
\pgfpathlineto{\pgfqpoint{1.032744in}{2.493629in}}%
\pgfpathlineto{\pgfqpoint{1.035068in}{2.459416in}}%
\pgfpathlineto{\pgfqpoint{1.037392in}{2.483630in}}%
\pgfpathlineto{\pgfqpoint{1.039716in}{2.530311in}}%
\pgfpathlineto{\pgfqpoint{1.042040in}{2.476824in}}%
\pgfpathlineto{\pgfqpoint{1.044364in}{2.487379in}}%
\pgfpathlineto{\pgfqpoint{1.046688in}{2.519794in}}%
\pgfpathlineto{\pgfqpoint{1.049012in}{2.477209in}}%
\pgfpathlineto{\pgfqpoint{1.053660in}{2.557664in}}%
\pgfpathlineto{\pgfqpoint{1.055984in}{2.488652in}}%
\pgfpathlineto{\pgfqpoint{1.058308in}{2.554297in}}%
\pgfpathlineto{\pgfqpoint{1.060632in}{2.502539in}}%
\pgfpathlineto{\pgfqpoint{1.062956in}{2.503371in}}%
\pgfpathlineto{\pgfqpoint{1.065280in}{2.502729in}}%
\pgfpathlineto{\pgfqpoint{1.067604in}{2.536566in}}%
\pgfpathlineto{\pgfqpoint{1.069928in}{2.510711in}}%
\pgfpathlineto{\pgfqpoint{1.072252in}{2.550887in}}%
\pgfpathlineto{\pgfqpoint{1.074576in}{2.557260in}}%
\pgfpathlineto{\pgfqpoint{1.081548in}{2.501568in}}%
\pgfpathlineto{\pgfqpoint{1.083872in}{2.533187in}}%
\pgfpathlineto{\pgfqpoint{1.086195in}{2.534592in}}%
\pgfpathlineto{\pgfqpoint{1.088519in}{2.534295in}}%
\pgfpathlineto{\pgfqpoint{1.090843in}{2.521844in}}%
\pgfpathlineto{\pgfqpoint{1.095491in}{2.616497in}}%
\pgfpathlineto{\pgfqpoint{1.097815in}{2.585333in}}%
\pgfpathlineto{\pgfqpoint{1.100139in}{2.529732in}}%
\pgfpathlineto{\pgfqpoint{1.104787in}{2.587412in}}%
\pgfpathlineto{\pgfqpoint{1.107111in}{2.542280in}}%
\pgfpathlineto{\pgfqpoint{1.109435in}{2.583793in}}%
\pgfpathlineto{\pgfqpoint{1.111759in}{2.552545in}}%
\pgfpathlineto{\pgfqpoint{1.114083in}{2.586767in}}%
\pgfpathlineto{\pgfqpoint{1.116407in}{2.587494in}}%
\pgfpathlineto{\pgfqpoint{1.118731in}{2.572609in}}%
\pgfpathlineto{\pgfqpoint{1.121055in}{2.568905in}}%
\pgfpathlineto{\pgfqpoint{1.123379in}{2.597435in}}%
\pgfpathlineto{\pgfqpoint{1.125703in}{2.580586in}}%
\pgfpathlineto{\pgfqpoint{1.128027in}{2.579238in}}%
\pgfpathlineto{\pgfqpoint{1.130351in}{2.588193in}}%
\pgfpathlineto{\pgfqpoint{1.132675in}{2.642968in}}%
\pgfpathlineto{\pgfqpoint{1.134999in}{2.601083in}}%
\pgfpathlineto{\pgfqpoint{1.137323in}{2.601346in}}%
\pgfpathlineto{\pgfqpoint{1.139647in}{2.600264in}}%
\pgfpathlineto{\pgfqpoint{1.141971in}{2.610234in}}%
\pgfpathlineto{\pgfqpoint{1.144295in}{2.595018in}}%
\pgfpathlineto{\pgfqpoint{1.146619in}{2.600940in}}%
\pgfpathlineto{\pgfqpoint{1.148943in}{2.614641in}}%
\pgfpathlineto{\pgfqpoint{1.151267in}{2.558763in}}%
\pgfpathlineto{\pgfqpoint{1.153591in}{2.614972in}}%
\pgfpathlineto{\pgfqpoint{1.160562in}{2.665827in}}%
\pgfpathlineto{\pgfqpoint{1.162886in}{2.618433in}}%
\pgfpathlineto{\pgfqpoint{1.165210in}{2.681402in}}%
\pgfpathlineto{\pgfqpoint{1.167534in}{2.695777in}}%
\pgfpathlineto{\pgfqpoint{1.172182in}{2.645211in}}%
\pgfpathlineto{\pgfqpoint{1.174506in}{2.677875in}}%
\pgfpathlineto{\pgfqpoint{1.176830in}{2.662775in}}%
\pgfpathlineto{\pgfqpoint{1.179154in}{2.606928in}}%
\pgfpathlineto{\pgfqpoint{1.181478in}{2.653946in}}%
\pgfpathlineto{\pgfqpoint{1.183802in}{2.630230in}}%
\pgfpathlineto{\pgfqpoint{1.186126in}{2.664519in}}%
\pgfpathlineto{\pgfqpoint{1.188450in}{2.663696in}}%
\pgfpathlineto{\pgfqpoint{1.190774in}{2.667227in}}%
\pgfpathlineto{\pgfqpoint{1.193098in}{2.651012in}}%
\pgfpathlineto{\pgfqpoint{1.195422in}{2.683518in}}%
\pgfpathlineto{\pgfqpoint{1.197746in}{2.656300in}}%
\pgfpathlineto{\pgfqpoint{1.200070in}{2.599238in}}%
\pgfpathlineto{\pgfqpoint{1.204718in}{2.731826in}}%
\pgfpathlineto{\pgfqpoint{1.207042in}{2.649358in}}%
\pgfpathlineto{\pgfqpoint{1.211690in}{2.708825in}}%
\pgfpathlineto{\pgfqpoint{1.214014in}{2.713557in}}%
\pgfpathlineto{\pgfqpoint{1.216338in}{2.677613in}}%
\pgfpathlineto{\pgfqpoint{1.218662in}{2.683269in}}%
\pgfpathlineto{\pgfqpoint{1.220986in}{2.711765in}}%
\pgfpathlineto{\pgfqpoint{1.223310in}{2.709648in}}%
\pgfpathlineto{\pgfqpoint{1.225633in}{2.696425in}}%
\pgfpathlineto{\pgfqpoint{1.230281in}{2.717213in}}%
\pgfpathlineto{\pgfqpoint{1.232605in}{2.738199in}}%
\pgfpathlineto{\pgfqpoint{1.234929in}{2.738542in}}%
\pgfpathlineto{\pgfqpoint{1.237253in}{2.668718in}}%
\pgfpathlineto{\pgfqpoint{1.239577in}{2.682304in}}%
\pgfpathlineto{\pgfqpoint{1.241901in}{2.750750in}}%
\pgfpathlineto{\pgfqpoint{1.246549in}{2.705010in}}%
\pgfpathlineto{\pgfqpoint{1.248873in}{2.772569in}}%
\pgfpathlineto{\pgfqpoint{1.251197in}{2.730998in}}%
\pgfpathlineto{\pgfqpoint{1.253521in}{2.756608in}}%
\pgfpathlineto{\pgfqpoint{1.255845in}{2.765870in}}%
\pgfpathlineto{\pgfqpoint{1.258169in}{2.721017in}}%
\pgfpathlineto{\pgfqpoint{1.265141in}{2.791019in}}%
\pgfpathlineto{\pgfqpoint{1.267465in}{2.778913in}}%
\pgfpathlineto{\pgfqpoint{1.269789in}{2.753945in}}%
\pgfpathlineto{\pgfqpoint{1.272113in}{2.699945in}}%
\pgfpathlineto{\pgfqpoint{1.276761in}{2.780155in}}%
\pgfpathlineto{\pgfqpoint{1.279085in}{2.789867in}}%
\pgfpathlineto{\pgfqpoint{1.281409in}{2.779129in}}%
\pgfpathlineto{\pgfqpoint{1.283733in}{2.754519in}}%
\pgfpathlineto{\pgfqpoint{1.286057in}{2.779884in}}%
\pgfpathlineto{\pgfqpoint{1.288381in}{2.743926in}}%
\pgfpathlineto{\pgfqpoint{1.290705in}{2.785763in}}%
\pgfpathlineto{\pgfqpoint{1.293029in}{2.791503in}}%
\pgfpathlineto{\pgfqpoint{1.295353in}{2.720911in}}%
\pgfpathlineto{\pgfqpoint{1.297676in}{2.770696in}}%
\pgfpathlineto{\pgfqpoint{1.300000in}{2.786171in}}%
\pgfpathlineto{\pgfqpoint{1.302324in}{2.792874in}}%
\pgfpathlineto{\pgfqpoint{1.304648in}{2.771034in}}%
\pgfpathlineto{\pgfqpoint{1.306972in}{2.780415in}}%
\pgfpathlineto{\pgfqpoint{1.309296in}{2.841393in}}%
\pgfpathlineto{\pgfqpoint{1.311620in}{2.816529in}}%
\pgfpathlineto{\pgfqpoint{1.313944in}{2.835591in}}%
\pgfpathlineto{\pgfqpoint{1.316268in}{2.819717in}}%
\pgfpathlineto{\pgfqpoint{1.318592in}{2.836767in}}%
\pgfpathlineto{\pgfqpoint{1.320916in}{2.783949in}}%
\pgfpathlineto{\pgfqpoint{1.323240in}{2.855327in}}%
\pgfpathlineto{\pgfqpoint{1.325564in}{2.794428in}}%
\pgfpathlineto{\pgfqpoint{1.327888in}{2.800356in}}%
\pgfpathlineto{\pgfqpoint{1.330212in}{2.843749in}}%
\pgfpathlineto{\pgfqpoint{1.334860in}{2.772819in}}%
\pgfpathlineto{\pgfqpoint{1.337184in}{2.868310in}}%
\pgfpathlineto{\pgfqpoint{1.339508in}{2.785702in}}%
\pgfpathlineto{\pgfqpoint{1.341832in}{2.859847in}}%
\pgfpathlineto{\pgfqpoint{1.344156in}{2.844425in}}%
\pgfpathlineto{\pgfqpoint{1.346480in}{2.789864in}}%
\pgfpathlineto{\pgfqpoint{1.348804in}{2.847443in}}%
\pgfpathlineto{\pgfqpoint{1.351128in}{2.830512in}}%
\pgfpathlineto{\pgfqpoint{1.353452in}{2.874111in}}%
\pgfpathlineto{\pgfqpoint{1.355776in}{2.810912in}}%
\pgfpathlineto{\pgfqpoint{1.358100in}{2.835531in}}%
\pgfpathlineto{\pgfqpoint{1.360424in}{2.816868in}}%
\pgfpathlineto{\pgfqpoint{1.362748in}{2.855046in}}%
\pgfpathlineto{\pgfqpoint{1.365072in}{2.797083in}}%
\pgfpathlineto{\pgfqpoint{1.367395in}{2.848111in}}%
\pgfpathlineto{\pgfqpoint{1.369719in}{2.857670in}}%
\pgfpathlineto{\pgfqpoint{1.372043in}{2.810925in}}%
\pgfpathlineto{\pgfqpoint{1.374367in}{2.822707in}}%
\pgfpathlineto{\pgfqpoint{1.376691in}{2.864168in}}%
\pgfpathlineto{\pgfqpoint{1.379015in}{2.858322in}}%
\pgfpathlineto{\pgfqpoint{1.381339in}{2.871342in}}%
\pgfpathlineto{\pgfqpoint{1.383663in}{2.825040in}}%
\pgfpathlineto{\pgfqpoint{1.385987in}{2.891934in}}%
\pgfpathlineto{\pgfqpoint{1.388311in}{2.832469in}}%
\pgfpathlineto{\pgfqpoint{1.390635in}{2.836047in}}%
\pgfpathlineto{\pgfqpoint{1.392959in}{2.880781in}}%
\pgfpathlineto{\pgfqpoint{1.395283in}{2.865293in}}%
\pgfpathlineto{\pgfqpoint{1.397607in}{2.796714in}}%
\pgfpathlineto{\pgfqpoint{1.399931in}{2.879878in}}%
\pgfpathlineto{\pgfqpoint{1.402255in}{2.896013in}}%
\pgfpathlineto{\pgfqpoint{1.404579in}{2.858293in}}%
\pgfpathlineto{\pgfqpoint{1.406903in}{2.878807in}}%
\pgfpathlineto{\pgfqpoint{1.409227in}{2.855556in}}%
\pgfpathlineto{\pgfqpoint{1.411551in}{2.847039in}}%
\pgfpathlineto{\pgfqpoint{1.413875in}{2.843816in}}%
\pgfpathlineto{\pgfqpoint{1.416199in}{2.923947in}}%
\pgfpathlineto{\pgfqpoint{1.418523in}{2.906594in}}%
\pgfpathlineto{\pgfqpoint{1.420847in}{2.875066in}}%
\pgfpathlineto{\pgfqpoint{1.423171in}{2.889995in}}%
\pgfpathlineto{\pgfqpoint{1.425495in}{2.866687in}}%
\pgfpathlineto{\pgfqpoint{1.427819in}{2.922047in}}%
\pgfpathlineto{\pgfqpoint{1.430143in}{2.894934in}}%
\pgfpathlineto{\pgfqpoint{1.432467in}{2.892044in}}%
\pgfpathlineto{\pgfqpoint{1.437114in}{2.925214in}}%
\pgfpathlineto{\pgfqpoint{1.439438in}{2.900408in}}%
\pgfpathlineto{\pgfqpoint{1.441762in}{2.937882in}}%
\pgfpathlineto{\pgfqpoint{1.444086in}{2.949545in}}%
\pgfpathlineto{\pgfqpoint{1.446410in}{2.900148in}}%
\pgfpathlineto{\pgfqpoint{1.448734in}{2.935042in}}%
\pgfpathlineto{\pgfqpoint{1.451058in}{2.898160in}}%
\pgfpathlineto{\pgfqpoint{1.453382in}{2.881952in}}%
\pgfpathlineto{\pgfqpoint{1.458030in}{2.955435in}}%
\pgfpathlineto{\pgfqpoint{1.462678in}{2.886328in}}%
\pgfpathlineto{\pgfqpoint{1.465002in}{2.914744in}}%
\pgfpathlineto{\pgfqpoint{1.467326in}{2.962302in}}%
\pgfpathlineto{\pgfqpoint{1.469650in}{2.969709in}}%
\pgfpathlineto{\pgfqpoint{1.471974in}{2.859993in}}%
\pgfpathlineto{\pgfqpoint{1.476622in}{2.970705in}}%
\pgfpathlineto{\pgfqpoint{1.478946in}{2.910445in}}%
\pgfpathlineto{\pgfqpoint{1.481270in}{2.922270in}}%
\pgfpathlineto{\pgfqpoint{1.483594in}{2.924231in}}%
\pgfpathlineto{\pgfqpoint{1.488242in}{3.000168in}}%
\pgfpathlineto{\pgfqpoint{1.492890in}{2.945427in}}%
\pgfpathlineto{\pgfqpoint{1.495214in}{2.969740in}}%
\pgfpathlineto{\pgfqpoint{1.497538in}{2.978956in}}%
\pgfpathlineto{\pgfqpoint{1.502186in}{2.916957in}}%
\pgfpathlineto{\pgfqpoint{1.504510in}{2.984199in}}%
\pgfpathlineto{\pgfqpoint{1.509157in}{2.934504in}}%
\pgfpathlineto{\pgfqpoint{1.511481in}{2.956757in}}%
\pgfpathlineto{\pgfqpoint{1.513805in}{2.944145in}}%
\pgfpathlineto{\pgfqpoint{1.516129in}{3.003073in}}%
\pgfpathlineto{\pgfqpoint{1.518453in}{2.978775in}}%
\pgfpathlineto{\pgfqpoint{1.520777in}{2.937572in}}%
\pgfpathlineto{\pgfqpoint{1.523101in}{2.438400in}}%
\pgfpathlineto{\pgfqpoint{1.525425in}{2.492425in}}%
\pgfpathlineto{\pgfqpoint{1.530073in}{2.446595in}}%
\pgfpathlineto{\pgfqpoint{1.532397in}{2.486932in}}%
\pgfpathlineto{\pgfqpoint{1.534721in}{2.464056in}}%
\pgfpathlineto{\pgfqpoint{1.537045in}{2.497225in}}%
\pgfpathlineto{\pgfqpoint{1.541693in}{2.434716in}}%
\pgfpathlineto{\pgfqpoint{1.544017in}{2.440864in}}%
\pgfpathlineto{\pgfqpoint{1.546341in}{2.517944in}}%
\pgfpathlineto{\pgfqpoint{1.548665in}{2.485387in}}%
\pgfpathlineto{\pgfqpoint{1.550989in}{2.507373in}}%
\pgfpathlineto{\pgfqpoint{1.553313in}{2.472436in}}%
\pgfpathlineto{\pgfqpoint{1.555637in}{2.516339in}}%
\pgfpathlineto{\pgfqpoint{1.560285in}{2.470766in}}%
\pgfpathlineto{\pgfqpoint{1.562609in}{2.505926in}}%
\pgfpathlineto{\pgfqpoint{1.564933in}{2.520952in}}%
\pgfpathlineto{\pgfqpoint{1.567257in}{2.458270in}}%
\pgfpathlineto{\pgfqpoint{1.569581in}{2.512863in}}%
\pgfpathlineto{\pgfqpoint{1.571905in}{2.508437in}}%
\pgfpathlineto{\pgfqpoint{1.574229in}{2.540489in}}%
\pgfpathlineto{\pgfqpoint{1.578876in}{2.484016in}}%
\pgfpathlineto{\pgfqpoint{1.581200in}{2.522364in}}%
\pgfpathlineto{\pgfqpoint{1.583524in}{2.514767in}}%
\pgfpathlineto{\pgfqpoint{1.588172in}{2.537120in}}%
\pgfpathlineto{\pgfqpoint{1.592820in}{2.510943in}}%
\pgfpathlineto{\pgfqpoint{1.595144in}{2.554084in}}%
\pgfpathlineto{\pgfqpoint{1.597468in}{2.515550in}}%
\pgfpathlineto{\pgfqpoint{1.599792in}{2.456889in}}%
\pgfpathlineto{\pgfqpoint{1.602116in}{2.484953in}}%
\pgfpathlineto{\pgfqpoint{1.604440in}{2.478017in}}%
\pgfpathlineto{\pgfqpoint{1.606764in}{2.520461in}}%
\pgfpathlineto{\pgfqpoint{1.609088in}{2.530005in}}%
\pgfpathlineto{\pgfqpoint{1.611412in}{2.505810in}}%
\pgfpathlineto{\pgfqpoint{1.613736in}{2.499207in}}%
\pgfpathlineto{\pgfqpoint{1.616060in}{2.528291in}}%
\pgfpathlineto{\pgfqpoint{1.618384in}{2.540473in}}%
\pgfpathlineto{\pgfqpoint{1.620708in}{2.510105in}}%
\pgfpathlineto{\pgfqpoint{1.623032in}{2.548400in}}%
\pgfpathlineto{\pgfqpoint{1.625356in}{2.532477in}}%
\pgfpathlineto{\pgfqpoint{1.627680in}{2.551838in}}%
\pgfpathlineto{\pgfqpoint{1.630004in}{2.522772in}}%
\pgfpathlineto{\pgfqpoint{1.632328in}{2.570251in}}%
\pgfpathlineto{\pgfqpoint{1.634652in}{2.576504in}}%
\pgfpathlineto{\pgfqpoint{1.636976in}{2.591531in}}%
\pgfpathlineto{\pgfqpoint{1.639300in}{2.510939in}}%
\pgfpathlineto{\pgfqpoint{1.641624in}{2.527277in}}%
\pgfpathlineto{\pgfqpoint{1.643948in}{2.511887in}}%
\pgfpathlineto{\pgfqpoint{1.646272in}{2.544057in}}%
\pgfpathlineto{\pgfqpoint{1.648595in}{2.510805in}}%
\pgfpathlineto{\pgfqpoint{1.650919in}{2.561741in}}%
\pgfpathlineto{\pgfqpoint{1.653243in}{2.553419in}}%
\pgfpathlineto{\pgfqpoint{1.655567in}{2.602846in}}%
\pgfpathlineto{\pgfqpoint{1.657891in}{2.615443in}}%
\pgfpathlineto{\pgfqpoint{1.660215in}{2.563259in}}%
\pgfpathlineto{\pgfqpoint{1.662539in}{2.570386in}}%
\pgfpathlineto{\pgfqpoint{1.664863in}{2.535687in}}%
\pgfpathlineto{\pgfqpoint{1.667187in}{2.591690in}}%
\pgfpathlineto{\pgfqpoint{1.669511in}{2.565250in}}%
\pgfpathlineto{\pgfqpoint{1.671835in}{2.638828in}}%
\pgfpathlineto{\pgfqpoint{1.674159in}{2.574718in}}%
\pgfpathlineto{\pgfqpoint{1.676483in}{2.568966in}}%
\pgfpathlineto{\pgfqpoint{1.678807in}{2.605525in}}%
\pgfpathlineto{\pgfqpoint{1.681131in}{2.580896in}}%
\pgfpathlineto{\pgfqpoint{1.683455in}{2.619692in}}%
\pgfpathlineto{\pgfqpoint{1.688103in}{2.569374in}}%
\pgfpathlineto{\pgfqpoint{1.690427in}{2.613292in}}%
\pgfpathlineto{\pgfqpoint{1.692751in}{2.567368in}}%
\pgfpathlineto{\pgfqpoint{1.695075in}{2.613692in}}%
\pgfpathlineto{\pgfqpoint{1.697399in}{2.563316in}}%
\pgfpathlineto{\pgfqpoint{1.702047in}{2.568781in}}%
\pgfpathlineto{\pgfqpoint{1.704371in}{2.580833in}}%
\pgfpathlineto{\pgfqpoint{1.706695in}{2.625490in}}%
\pgfpathlineto{\pgfqpoint{1.709019in}{2.592893in}}%
\pgfpathlineto{\pgfqpoint{1.711343in}{2.594554in}}%
\pgfpathlineto{\pgfqpoint{1.713667in}{2.585702in}}%
\pgfpathlineto{\pgfqpoint{1.715991in}{2.596715in}}%
\pgfpathlineto{\pgfqpoint{1.718314in}{2.585750in}}%
\pgfpathlineto{\pgfqpoint{1.720638in}{2.605229in}}%
\pgfpathlineto{\pgfqpoint{1.722962in}{2.595865in}}%
\pgfpathlineto{\pgfqpoint{1.725286in}{2.615184in}}%
\pgfpathlineto{\pgfqpoint{1.727610in}{2.560612in}}%
\pgfpathlineto{\pgfqpoint{1.729934in}{2.579805in}}%
\pgfpathlineto{\pgfqpoint{1.732258in}{2.624671in}}%
\pgfpathlineto{\pgfqpoint{1.734582in}{2.626773in}}%
\pgfpathlineto{\pgfqpoint{1.736906in}{2.557840in}}%
\pgfpathlineto{\pgfqpoint{1.739230in}{2.573511in}}%
\pgfpathlineto{\pgfqpoint{1.741554in}{2.580222in}}%
\pgfpathlineto{\pgfqpoint{1.743878in}{2.638261in}}%
\pgfpathlineto{\pgfqpoint{1.746202in}{2.561892in}}%
\pgfpathlineto{\pgfqpoint{1.748526in}{2.622138in}}%
\pgfpathlineto{\pgfqpoint{1.750850in}{2.631677in}}%
\pgfpathlineto{\pgfqpoint{1.755498in}{2.575702in}}%
\pgfpathlineto{\pgfqpoint{1.757822in}{2.700217in}}%
\pgfpathlineto{\pgfqpoint{1.760146in}{2.625226in}}%
\pgfpathlineto{\pgfqpoint{1.762470in}{2.607706in}}%
\pgfpathlineto{\pgfqpoint{1.764794in}{2.611599in}}%
\pgfpathlineto{\pgfqpoint{1.769442in}{2.627317in}}%
\pgfpathlineto{\pgfqpoint{1.771766in}{2.596402in}}%
\pgfpathlineto{\pgfqpoint{1.774090in}{2.640513in}}%
\pgfpathlineto{\pgfqpoint{1.776414in}{2.582920in}}%
\pgfpathlineto{\pgfqpoint{1.778738in}{2.605326in}}%
\pgfpathlineto{\pgfqpoint{1.783386in}{2.586423in}}%
\pgfpathlineto{\pgfqpoint{1.785710in}{2.599743in}}%
\pgfpathlineto{\pgfqpoint{1.788034in}{2.584845in}}%
\pgfpathlineto{\pgfqpoint{1.790357in}{2.641037in}}%
\pgfpathlineto{\pgfqpoint{1.792681in}{2.645528in}}%
\pgfpathlineto{\pgfqpoint{1.795005in}{2.665366in}}%
\pgfpathlineto{\pgfqpoint{1.797329in}{2.612545in}}%
\pgfpathlineto{\pgfqpoint{1.799653in}{2.666192in}}%
\pgfpathlineto{\pgfqpoint{1.801977in}{2.660249in}}%
\pgfpathlineto{\pgfqpoint{1.804301in}{2.676680in}}%
\pgfpathlineto{\pgfqpoint{1.806625in}{2.638943in}}%
\pgfpathlineto{\pgfqpoint{1.808949in}{2.648908in}}%
\pgfpathlineto{\pgfqpoint{1.811273in}{2.606497in}}%
\pgfpathlineto{\pgfqpoint{1.813597in}{2.663320in}}%
\pgfpathlineto{\pgfqpoint{1.815921in}{2.581798in}}%
\pgfpathlineto{\pgfqpoint{1.818245in}{2.658187in}}%
\pgfpathlineto{\pgfqpoint{1.820569in}{2.629526in}}%
\pgfpathlineto{\pgfqpoint{1.822893in}{2.675405in}}%
\pgfpathlineto{\pgfqpoint{1.825217in}{2.590650in}}%
\pgfpathlineto{\pgfqpoint{1.827541in}{2.646850in}}%
\pgfpathlineto{\pgfqpoint{1.829865in}{2.636904in}}%
\pgfpathlineto{\pgfqpoint{1.832189in}{2.640269in}}%
\pgfpathlineto{\pgfqpoint{1.834513in}{2.660291in}}%
\pgfpathlineto{\pgfqpoint{1.836837in}{2.663239in}}%
\pgfpathlineto{\pgfqpoint{1.839161in}{2.626672in}}%
\pgfpathlineto{\pgfqpoint{1.841485in}{2.688023in}}%
\pgfpathlineto{\pgfqpoint{1.843809in}{2.666802in}}%
\pgfpathlineto{\pgfqpoint{1.846133in}{2.660382in}}%
\pgfpathlineto{\pgfqpoint{1.848457in}{2.624006in}}%
\pgfpathlineto{\pgfqpoint{1.850781in}{2.696632in}}%
\pgfpathlineto{\pgfqpoint{1.853105in}{2.664325in}}%
\pgfpathlineto{\pgfqpoint{1.855429in}{2.714000in}}%
\pgfpathlineto{\pgfqpoint{1.857753in}{2.642884in}}%
\pgfpathlineto{\pgfqpoint{1.860076in}{2.660733in}}%
\pgfpathlineto{\pgfqpoint{1.862400in}{2.658209in}}%
\pgfpathlineto{\pgfqpoint{1.864724in}{2.667353in}}%
\pgfpathlineto{\pgfqpoint{1.867048in}{2.702192in}}%
\pgfpathlineto{\pgfqpoint{1.869372in}{2.603042in}}%
\pgfpathlineto{\pgfqpoint{1.871696in}{2.601852in}}%
\pgfpathlineto{\pgfqpoint{1.874020in}{2.641762in}}%
\pgfpathlineto{\pgfqpoint{1.876344in}{2.652060in}}%
\pgfpathlineto{\pgfqpoint{1.878668in}{2.685943in}}%
\pgfpathlineto{\pgfqpoint{1.880992in}{2.690775in}}%
\pgfpathlineto{\pgfqpoint{1.883316in}{2.687393in}}%
\pgfpathlineto{\pgfqpoint{1.885640in}{2.641897in}}%
\pgfpathlineto{\pgfqpoint{1.887964in}{2.709677in}}%
\pgfpathlineto{\pgfqpoint{1.890288in}{2.688230in}}%
\pgfpathlineto{\pgfqpoint{1.892612in}{2.650951in}}%
\pgfpathlineto{\pgfqpoint{1.894936in}{2.697088in}}%
\pgfpathlineto{\pgfqpoint{1.897260in}{2.657638in}}%
\pgfpathlineto{\pgfqpoint{1.901908in}{2.722033in}}%
\pgfpathlineto{\pgfqpoint{1.904232in}{2.691537in}}%
\pgfpathlineto{\pgfqpoint{1.906556in}{2.703188in}}%
\pgfpathlineto{\pgfqpoint{1.908880in}{2.632070in}}%
\pgfpathlineto{\pgfqpoint{1.911204in}{2.710171in}}%
\pgfpathlineto{\pgfqpoint{1.913528in}{2.645789in}}%
\pgfpathlineto{\pgfqpoint{1.915852in}{2.681340in}}%
\pgfpathlineto{\pgfqpoint{1.918176in}{2.649924in}}%
\pgfpathlineto{\pgfqpoint{1.920500in}{2.721401in}}%
\pgfpathlineto{\pgfqpoint{1.922824in}{2.724419in}}%
\pgfpathlineto{\pgfqpoint{1.925148in}{2.656213in}}%
\pgfpathlineto{\pgfqpoint{1.927472in}{2.686410in}}%
\pgfpathlineto{\pgfqpoint{1.929795in}{2.675315in}}%
\pgfpathlineto{\pgfqpoint{1.932119in}{2.688948in}}%
\pgfpathlineto{\pgfqpoint{1.934443in}{2.709620in}}%
\pgfpathlineto{\pgfqpoint{1.936767in}{2.653713in}}%
\pgfpathlineto{\pgfqpoint{1.939091in}{2.690258in}}%
\pgfpathlineto{\pgfqpoint{1.941415in}{2.679202in}}%
\pgfpathlineto{\pgfqpoint{1.943739in}{2.725833in}}%
\pgfpathlineto{\pgfqpoint{1.946063in}{2.707608in}}%
\pgfpathlineto{\pgfqpoint{1.948387in}{2.639154in}}%
\pgfpathlineto{\pgfqpoint{1.950711in}{2.699928in}}%
\pgfpathlineto{\pgfqpoint{1.953035in}{2.674962in}}%
\pgfpathlineto{\pgfqpoint{1.955359in}{2.699708in}}%
\pgfpathlineto{\pgfqpoint{1.957683in}{2.689722in}}%
\pgfpathlineto{\pgfqpoint{1.960007in}{2.710957in}}%
\pgfpathlineto{\pgfqpoint{1.964655in}{2.684814in}}%
\pgfpathlineto{\pgfqpoint{1.966979in}{2.686822in}}%
\pgfpathlineto{\pgfqpoint{1.969303in}{2.723509in}}%
\pgfpathlineto{\pgfqpoint{1.971627in}{2.733675in}}%
\pgfpathlineto{\pgfqpoint{1.973951in}{2.715841in}}%
\pgfpathlineto{\pgfqpoint{1.976275in}{2.665145in}}%
\pgfpathlineto{\pgfqpoint{1.978599in}{2.736274in}}%
\pgfpathlineto{\pgfqpoint{1.980923in}{2.720436in}}%
\pgfpathlineto{\pgfqpoint{1.983247in}{2.714148in}}%
\pgfpathlineto{\pgfqpoint{1.985571in}{2.726848in}}%
\pgfpathlineto{\pgfqpoint{1.987895in}{2.774396in}}%
\pgfpathlineto{\pgfqpoint{1.990219in}{2.786644in}}%
\pgfpathlineto{\pgfqpoint{1.992543in}{2.705674in}}%
\pgfpathlineto{\pgfqpoint{1.997191in}{2.736366in}}%
\pgfpathlineto{\pgfqpoint{1.999514in}{2.705879in}}%
\pgfpathlineto{\pgfqpoint{2.001838in}{2.755149in}}%
\pgfpathlineto{\pgfqpoint{2.004162in}{2.739406in}}%
\pgfpathlineto{\pgfqpoint{2.006486in}{2.741537in}}%
\pgfpathlineto{\pgfqpoint{2.008810in}{2.721708in}}%
\pgfpathlineto{\pgfqpoint{2.011134in}{2.746640in}}%
\pgfpathlineto{\pgfqpoint{2.013458in}{2.758595in}}%
\pgfpathlineto{\pgfqpoint{2.015782in}{2.728790in}}%
\pgfpathlineto{\pgfqpoint{2.018106in}{2.756663in}}%
\pgfpathlineto{\pgfqpoint{2.020430in}{2.763409in}}%
\pgfpathlineto{\pgfqpoint{2.022754in}{2.728785in}}%
\pgfpathlineto{\pgfqpoint{2.025078in}{2.736071in}}%
\pgfpathlineto{\pgfqpoint{2.027402in}{2.714713in}}%
\pgfpathlineto{\pgfqpoint{2.029726in}{2.721442in}}%
\pgfpathlineto{\pgfqpoint{2.032050in}{2.811749in}}%
\pgfpathlineto{\pgfqpoint{2.034374in}{2.703983in}}%
\pgfpathlineto{\pgfqpoint{2.036698in}{2.724292in}}%
\pgfpathlineto{\pgfqpoint{2.039022in}{2.728860in}}%
\pgfpathlineto{\pgfqpoint{2.041346in}{2.717476in}}%
\pgfpathlineto{\pgfqpoint{2.043670in}{2.757194in}}%
\pgfpathlineto{\pgfqpoint{2.045994in}{2.763509in}}%
\pgfpathlineto{\pgfqpoint{2.050642in}{2.687436in}}%
\pgfpathlineto{\pgfqpoint{2.052966in}{2.771109in}}%
\pgfpathlineto{\pgfqpoint{2.057614in}{2.732662in}}%
\pgfpathlineto{\pgfqpoint{2.059938in}{2.778089in}}%
\pgfpathlineto{\pgfqpoint{2.062262in}{2.745649in}}%
\pgfpathlineto{\pgfqpoint{2.064586in}{2.769567in}}%
\pgfpathlineto{\pgfqpoint{2.066910in}{2.781349in}}%
\pgfpathlineto{\pgfqpoint{2.069234in}{2.727934in}}%
\pgfpathlineto{\pgfqpoint{2.071557in}{2.720607in}}%
\pgfpathlineto{\pgfqpoint{2.073881in}{2.757930in}}%
\pgfpathlineto{\pgfqpoint{2.076205in}{2.751029in}}%
\pgfpathlineto{\pgfqpoint{2.078529in}{2.775803in}}%
\pgfpathlineto{\pgfqpoint{2.080853in}{2.772454in}}%
\pgfpathlineto{\pgfqpoint{2.083177in}{2.766034in}}%
\pgfpathlineto{\pgfqpoint{2.085501in}{2.769959in}}%
\pgfpathlineto{\pgfqpoint{2.087825in}{2.787415in}}%
\pgfpathlineto{\pgfqpoint{2.090149in}{2.770996in}}%
\pgfpathlineto{\pgfqpoint{2.092473in}{2.810529in}}%
\pgfpathlineto{\pgfqpoint{2.094797in}{2.803424in}}%
\pgfpathlineto{\pgfqpoint{2.097121in}{2.755731in}}%
\pgfpathlineto{\pgfqpoint{2.099445in}{2.788230in}}%
\pgfpathlineto{\pgfqpoint{2.101769in}{2.751298in}}%
\pgfpathlineto{\pgfqpoint{2.104093in}{2.274409in}}%
\pgfpathlineto{\pgfqpoint{2.106417in}{2.249236in}}%
\pgfpathlineto{\pgfqpoint{2.108741in}{2.286315in}}%
\pgfpathlineto{\pgfqpoint{2.111065in}{2.248070in}}%
\pgfpathlineto{\pgfqpoint{2.113389in}{2.232657in}}%
\pgfpathlineto{\pgfqpoint{2.115713in}{2.289919in}}%
\pgfpathlineto{\pgfqpoint{2.120361in}{2.278666in}}%
\pgfpathlineto{\pgfqpoint{2.122685in}{2.345585in}}%
\pgfpathlineto{\pgfqpoint{2.125009in}{2.264398in}}%
\pgfpathlineto{\pgfqpoint{2.127333in}{2.294650in}}%
\pgfpathlineto{\pgfqpoint{2.129657in}{2.289422in}}%
\pgfpathlineto{\pgfqpoint{2.131981in}{2.262944in}}%
\pgfpathlineto{\pgfqpoint{2.134305in}{2.315988in}}%
\pgfpathlineto{\pgfqpoint{2.136629in}{2.261752in}}%
\pgfpathlineto{\pgfqpoint{2.138953in}{2.306158in}}%
\pgfpathlineto{\pgfqpoint{2.141276in}{2.324764in}}%
\pgfpathlineto{\pgfqpoint{2.143600in}{2.319010in}}%
\pgfpathlineto{\pgfqpoint{2.145924in}{2.345345in}}%
\pgfpathlineto{\pgfqpoint{2.148248in}{2.292273in}}%
\pgfpathlineto{\pgfqpoint{2.150572in}{2.309060in}}%
\pgfpathlineto{\pgfqpoint{2.152896in}{2.316214in}}%
\pgfpathlineto{\pgfqpoint{2.155220in}{2.311377in}}%
\pgfpathlineto{\pgfqpoint{2.157544in}{2.276094in}}%
\pgfpathlineto{\pgfqpoint{2.159868in}{2.259579in}}%
\pgfpathlineto{\pgfqpoint{2.162192in}{2.323153in}}%
\pgfpathlineto{\pgfqpoint{2.164516in}{2.296039in}}%
\pgfpathlineto{\pgfqpoint{2.166840in}{2.341064in}}%
\pgfpathlineto{\pgfqpoint{2.169164in}{2.320081in}}%
\pgfpathlineto{\pgfqpoint{2.171488in}{2.373932in}}%
\pgfpathlineto{\pgfqpoint{2.173812in}{2.308583in}}%
\pgfpathlineto{\pgfqpoint{2.176136in}{2.359522in}}%
\pgfpathlineto{\pgfqpoint{2.178460in}{2.350635in}}%
\pgfpathlineto{\pgfqpoint{2.180784in}{2.331969in}}%
\pgfpathlineto{\pgfqpoint{2.183108in}{2.296451in}}%
\pgfpathlineto{\pgfqpoint{2.185432in}{2.368649in}}%
\pgfpathlineto{\pgfqpoint{2.187756in}{2.290335in}}%
\pgfpathlineto{\pgfqpoint{2.190080in}{2.314198in}}%
\pgfpathlineto{\pgfqpoint{2.192404in}{2.313328in}}%
\pgfpathlineto{\pgfqpoint{2.194728in}{2.381488in}}%
\pgfpathlineto{\pgfqpoint{2.197052in}{2.302094in}}%
\pgfpathlineto{\pgfqpoint{2.199376in}{2.356993in}}%
\pgfpathlineto{\pgfqpoint{2.204024in}{2.364322in}}%
\pgfpathlineto{\pgfqpoint{2.206348in}{2.356250in}}%
\pgfpathlineto{\pgfqpoint{2.208672in}{2.322359in}}%
\pgfpathlineto{\pgfqpoint{2.210995in}{2.318599in}}%
\pgfpathlineto{\pgfqpoint{2.215643in}{2.406072in}}%
\pgfpathlineto{\pgfqpoint{2.217967in}{2.363780in}}%
\pgfpathlineto{\pgfqpoint{2.220291in}{2.362698in}}%
\pgfpathlineto{\pgfqpoint{2.222615in}{2.345900in}}%
\pgfpathlineto{\pgfqpoint{2.224939in}{2.349727in}}%
\pgfpathlineto{\pgfqpoint{2.227263in}{2.373102in}}%
\pgfpathlineto{\pgfqpoint{2.229587in}{2.298879in}}%
\pgfpathlineto{\pgfqpoint{2.231911in}{2.373361in}}%
\pgfpathlineto{\pgfqpoint{2.234235in}{2.364576in}}%
\pgfpathlineto{\pgfqpoint{2.236559in}{2.336297in}}%
\pgfpathlineto{\pgfqpoint{2.238883in}{2.400012in}}%
\pgfpathlineto{\pgfqpoint{2.241207in}{2.427879in}}%
\pgfpathlineto{\pgfqpoint{2.243531in}{2.388418in}}%
\pgfpathlineto{\pgfqpoint{2.245855in}{2.425834in}}%
\pgfpathlineto{\pgfqpoint{2.250503in}{2.322127in}}%
\pgfpathlineto{\pgfqpoint{2.252827in}{2.417853in}}%
\pgfpathlineto{\pgfqpoint{2.255151in}{2.374065in}}%
\pgfpathlineto{\pgfqpoint{2.257475in}{2.400899in}}%
\pgfpathlineto{\pgfqpoint{2.259799in}{2.401918in}}%
\pgfpathlineto{\pgfqpoint{2.262123in}{2.428378in}}%
\pgfpathlineto{\pgfqpoint{2.264447in}{2.340678in}}%
\pgfpathlineto{\pgfqpoint{2.266771in}{2.392681in}}%
\pgfpathlineto{\pgfqpoint{2.269095in}{2.372556in}}%
\pgfpathlineto{\pgfqpoint{2.271419in}{2.368088in}}%
\pgfpathlineto{\pgfqpoint{2.278391in}{2.419233in}}%
\pgfpathlineto{\pgfqpoint{2.280715in}{2.398755in}}%
\pgfpathlineto{\pgfqpoint{2.283038in}{2.484993in}}%
\pgfpathlineto{\pgfqpoint{2.287686in}{2.415568in}}%
\pgfpathlineto{\pgfqpoint{2.290010in}{2.377129in}}%
\pgfpathlineto{\pgfqpoint{2.292334in}{2.407002in}}%
\pgfpathlineto{\pgfqpoint{2.294658in}{2.344022in}}%
\pgfpathlineto{\pgfqpoint{2.296982in}{2.417109in}}%
\pgfpathlineto{\pgfqpoint{2.299306in}{2.394903in}}%
\pgfpathlineto{\pgfqpoint{2.301630in}{2.414689in}}%
\pgfpathlineto{\pgfqpoint{2.303954in}{2.415701in}}%
\pgfpathlineto{\pgfqpoint{2.306278in}{2.379639in}}%
\pgfpathlineto{\pgfqpoint{2.308602in}{2.462250in}}%
\pgfpathlineto{\pgfqpoint{2.310926in}{2.415577in}}%
\pgfpathlineto{\pgfqpoint{2.313250in}{2.408318in}}%
\pgfpathlineto{\pgfqpoint{2.315574in}{2.443928in}}%
\pgfpathlineto{\pgfqpoint{2.317898in}{2.458318in}}%
\pgfpathlineto{\pgfqpoint{2.322546in}{2.417073in}}%
\pgfpathlineto{\pgfqpoint{2.324870in}{2.383795in}}%
\pgfpathlineto{\pgfqpoint{2.338814in}{2.460266in}}%
\pgfpathlineto{\pgfqpoint{2.341138in}{2.433580in}}%
\pgfpathlineto{\pgfqpoint{2.343462in}{2.429221in}}%
\pgfpathlineto{\pgfqpoint{2.345786in}{2.453265in}}%
\pgfpathlineto{\pgfqpoint{2.348110in}{2.511957in}}%
\pgfpathlineto{\pgfqpoint{2.350434in}{2.429246in}}%
\pgfpathlineto{\pgfqpoint{2.352757in}{2.484606in}}%
\pgfpathlineto{\pgfqpoint{2.355081in}{2.421599in}}%
\pgfpathlineto{\pgfqpoint{2.357405in}{2.510412in}}%
\pgfpathlineto{\pgfqpoint{2.359729in}{2.445678in}}%
\pgfpathlineto{\pgfqpoint{2.364377in}{2.480443in}}%
\pgfpathlineto{\pgfqpoint{2.366701in}{2.472470in}}%
\pgfpathlineto{\pgfqpoint{2.369025in}{2.497421in}}%
\pgfpathlineto{\pgfqpoint{2.371349in}{2.477276in}}%
\pgfpathlineto{\pgfqpoint{2.373673in}{2.531864in}}%
\pgfpathlineto{\pgfqpoint{2.375997in}{2.494745in}}%
\pgfpathlineto{\pgfqpoint{2.378321in}{2.480647in}}%
\pgfpathlineto{\pgfqpoint{2.380645in}{2.485276in}}%
\pgfpathlineto{\pgfqpoint{2.382969in}{2.471556in}}%
\pgfpathlineto{\pgfqpoint{2.385293in}{2.504256in}}%
\pgfpathlineto{\pgfqpoint{2.387617in}{2.489613in}}%
\pgfpathlineto{\pgfqpoint{2.389941in}{2.512271in}}%
\pgfpathlineto{\pgfqpoint{2.392265in}{2.454626in}}%
\pgfpathlineto{\pgfqpoint{2.396913in}{2.534052in}}%
\pgfpathlineto{\pgfqpoint{2.399237in}{2.516084in}}%
\pgfpathlineto{\pgfqpoint{2.401561in}{2.508433in}}%
\pgfpathlineto{\pgfqpoint{2.403885in}{2.522342in}}%
\pgfpathlineto{\pgfqpoint{2.406209in}{2.543772in}}%
\pgfpathlineto{\pgfqpoint{2.408533in}{2.530920in}}%
\pgfpathlineto{\pgfqpoint{2.410857in}{2.501074in}}%
\pgfpathlineto{\pgfqpoint{2.413181in}{2.535288in}}%
\pgfpathlineto{\pgfqpoint{2.415505in}{2.495081in}}%
\pgfpathlineto{\pgfqpoint{2.417829in}{2.537846in}}%
\pgfpathlineto{\pgfqpoint{2.420153in}{2.519734in}}%
\pgfpathlineto{\pgfqpoint{2.422476in}{2.511560in}}%
\pgfpathlineto{\pgfqpoint{2.424800in}{2.555626in}}%
\pgfpathlineto{\pgfqpoint{2.427124in}{2.526496in}}%
\pgfpathlineto{\pgfqpoint{2.429448in}{2.516742in}}%
\pgfpathlineto{\pgfqpoint{2.431772in}{2.532665in}}%
\pgfpathlineto{\pgfqpoint{2.434096in}{2.594324in}}%
\pgfpathlineto{\pgfqpoint{2.438744in}{2.561119in}}%
\pgfpathlineto{\pgfqpoint{2.441068in}{2.557507in}}%
\pgfpathlineto{\pgfqpoint{2.443392in}{2.546071in}}%
\pgfpathlineto{\pgfqpoint{2.445716in}{2.584476in}}%
\pgfpathlineto{\pgfqpoint{2.448040in}{2.576889in}}%
\pgfpathlineto{\pgfqpoint{2.450364in}{2.564438in}}%
\pgfpathlineto{\pgfqpoint{2.452688in}{2.540022in}}%
\pgfpathlineto{\pgfqpoint{2.455012in}{2.544931in}}%
\pgfpathlineto{\pgfqpoint{2.457336in}{2.544778in}}%
\pgfpathlineto{\pgfqpoint{2.461984in}{2.628819in}}%
\pgfpathlineto{\pgfqpoint{2.464308in}{2.560958in}}%
\pgfpathlineto{\pgfqpoint{2.466632in}{2.580640in}}%
\pgfpathlineto{\pgfqpoint{2.468956in}{2.575887in}}%
\pgfpathlineto{\pgfqpoint{2.471280in}{2.581831in}}%
\pgfpathlineto{\pgfqpoint{2.473604in}{2.573344in}}%
\pgfpathlineto{\pgfqpoint{2.475928in}{2.582103in}}%
\pgfpathlineto{\pgfqpoint{2.478252in}{2.568977in}}%
\pgfpathlineto{\pgfqpoint{2.482900in}{2.633152in}}%
\pgfpathlineto{\pgfqpoint{2.485224in}{2.608582in}}%
\pgfpathlineto{\pgfqpoint{2.487548in}{2.640553in}}%
\pgfpathlineto{\pgfqpoint{2.489872in}{2.560089in}}%
\pgfpathlineto{\pgfqpoint{2.492195in}{2.590320in}}%
\pgfpathlineto{\pgfqpoint{2.494519in}{2.592947in}}%
\pgfpathlineto{\pgfqpoint{2.496843in}{2.626448in}}%
\pgfpathlineto{\pgfqpoint{2.499167in}{2.596155in}}%
\pgfpathlineto{\pgfqpoint{2.501491in}{2.634412in}}%
\pgfpathlineto{\pgfqpoint{2.503815in}{2.640980in}}%
\pgfpathlineto{\pgfqpoint{2.506139in}{2.583776in}}%
\pgfpathlineto{\pgfqpoint{2.508463in}{2.655627in}}%
\pgfpathlineto{\pgfqpoint{2.510787in}{2.615807in}}%
\pgfpathlineto{\pgfqpoint{2.515435in}{2.637416in}}%
\pgfpathlineto{\pgfqpoint{2.517759in}{2.671373in}}%
\pgfpathlineto{\pgfqpoint{2.520083in}{2.687739in}}%
\pgfpathlineto{\pgfqpoint{2.522407in}{2.632401in}}%
\pgfpathlineto{\pgfqpoint{2.527055in}{2.674961in}}%
\pgfpathlineto{\pgfqpoint{2.529379in}{2.612541in}}%
\pgfpathlineto{\pgfqpoint{2.531703in}{2.665515in}}%
\pgfpathlineto{\pgfqpoint{2.534027in}{2.633464in}}%
\pgfpathlineto{\pgfqpoint{2.536351in}{2.676172in}}%
\pgfpathlineto{\pgfqpoint{2.538675in}{2.688944in}}%
\pgfpathlineto{\pgfqpoint{2.540999in}{2.639871in}}%
\pgfpathlineto{\pgfqpoint{2.543323in}{2.674015in}}%
\pgfpathlineto{\pgfqpoint{2.545647in}{2.688165in}}%
\pgfpathlineto{\pgfqpoint{2.547971in}{2.685413in}}%
\pgfpathlineto{\pgfqpoint{2.550295in}{2.686490in}}%
\pgfpathlineto{\pgfqpoint{2.552619in}{2.694420in}}%
\pgfpathlineto{\pgfqpoint{2.554943in}{2.642817in}}%
\pgfpathlineto{\pgfqpoint{2.557267in}{2.647981in}}%
\pgfpathlineto{\pgfqpoint{2.561915in}{2.685927in}}%
\pgfpathlineto{\pgfqpoint{2.564238in}{2.663463in}}%
\pgfpathlineto{\pgfqpoint{2.566562in}{2.697669in}}%
\pgfpathlineto{\pgfqpoint{2.568886in}{2.706450in}}%
\pgfpathlineto{\pgfqpoint{2.571210in}{2.728644in}}%
\pgfpathlineto{\pgfqpoint{2.573534in}{2.712534in}}%
\pgfpathlineto{\pgfqpoint{2.575858in}{2.741213in}}%
\pgfpathlineto{\pgfqpoint{2.578182in}{2.667337in}}%
\pgfpathlineto{\pgfqpoint{2.582830in}{2.718739in}}%
\pgfpathlineto{\pgfqpoint{2.585154in}{2.714242in}}%
\pgfpathlineto{\pgfqpoint{2.587478in}{2.715632in}}%
\pgfpathlineto{\pgfqpoint{2.589802in}{2.722152in}}%
\pgfpathlineto{\pgfqpoint{2.592126in}{2.707084in}}%
\pgfpathlineto{\pgfqpoint{2.594450in}{2.711061in}}%
\pgfpathlineto{\pgfqpoint{2.596774in}{2.761332in}}%
\pgfpathlineto{\pgfqpoint{2.599098in}{2.721027in}}%
\pgfpathlineto{\pgfqpoint{2.601422in}{2.717785in}}%
\pgfpathlineto{\pgfqpoint{2.603746in}{2.719780in}}%
\pgfpathlineto{\pgfqpoint{2.606070in}{2.717201in}}%
\pgfpathlineto{\pgfqpoint{2.608394in}{2.704619in}}%
\pgfpathlineto{\pgfqpoint{2.610718in}{2.707295in}}%
\pgfpathlineto{\pgfqpoint{2.613042in}{2.722019in}}%
\pgfpathlineto{\pgfqpoint{2.615366in}{2.725345in}}%
\pgfpathlineto{\pgfqpoint{2.617690in}{2.771473in}}%
\pgfpathlineto{\pgfqpoint{2.620014in}{2.732380in}}%
\pgfpathlineto{\pgfqpoint{2.622338in}{2.777948in}}%
\pgfpathlineto{\pgfqpoint{2.624662in}{2.745493in}}%
\pgfpathlineto{\pgfqpoint{2.626986in}{2.807299in}}%
\pgfpathlineto{\pgfqpoint{2.629310in}{2.796792in}}%
\pgfpathlineto{\pgfqpoint{2.631634in}{2.730865in}}%
\pgfpathlineto{\pgfqpoint{2.633957in}{2.795432in}}%
\pgfpathlineto{\pgfqpoint{2.636281in}{2.742241in}}%
\pgfpathlineto{\pgfqpoint{2.638605in}{2.815198in}}%
\pgfpathlineto{\pgfqpoint{2.640929in}{2.751631in}}%
\pgfpathlineto{\pgfqpoint{2.643253in}{2.823697in}}%
\pgfpathlineto{\pgfqpoint{2.645577in}{2.774868in}}%
\pgfpathlineto{\pgfqpoint{2.650225in}{2.820297in}}%
\pgfpathlineto{\pgfqpoint{2.652549in}{2.806749in}}%
\pgfpathlineto{\pgfqpoint{2.654873in}{2.800841in}}%
\pgfpathlineto{\pgfqpoint{2.657197in}{2.738476in}}%
\pgfpathlineto{\pgfqpoint{2.659521in}{2.793573in}}%
\pgfpathlineto{\pgfqpoint{2.661845in}{2.802328in}}%
\pgfpathlineto{\pgfqpoint{2.664169in}{2.793049in}}%
\pgfpathlineto{\pgfqpoint{2.668817in}{2.833141in}}%
\pgfpathlineto{\pgfqpoint{2.673465in}{2.838411in}}%
\pgfpathlineto{\pgfqpoint{2.675789in}{2.830023in}}%
\pgfpathlineto{\pgfqpoint{2.678113in}{2.876727in}}%
\pgfpathlineto{\pgfqpoint{2.680437in}{2.785003in}}%
\pgfpathlineto{\pgfqpoint{2.682761in}{2.839320in}}%
\pgfpathlineto{\pgfqpoint{2.685085in}{2.370092in}}%
\pgfpathlineto{\pgfqpoint{2.687409in}{2.279942in}}%
\pgfpathlineto{\pgfqpoint{2.692057in}{2.367116in}}%
\pgfpathlineto{\pgfqpoint{2.694381in}{2.341744in}}%
\pgfpathlineto{\pgfqpoint{2.696705in}{2.351194in}}%
\pgfpathlineto{\pgfqpoint{2.699029in}{2.354555in}}%
\pgfpathlineto{\pgfqpoint{2.701353in}{2.363837in}}%
\pgfpathlineto{\pgfqpoint{2.703676in}{2.388520in}}%
\pgfpathlineto{\pgfqpoint{2.706000in}{2.393382in}}%
\pgfpathlineto{\pgfqpoint{2.708324in}{2.355465in}}%
\pgfpathlineto{\pgfqpoint{2.710648in}{2.360652in}}%
\pgfpathlineto{\pgfqpoint{2.712972in}{2.357145in}}%
\pgfpathlineto{\pgfqpoint{2.715296in}{2.359896in}}%
\pgfpathlineto{\pgfqpoint{2.717620in}{2.373130in}}%
\pgfpathlineto{\pgfqpoint{2.719944in}{2.408598in}}%
\pgfpathlineto{\pgfqpoint{2.722268in}{2.344765in}}%
\pgfpathlineto{\pgfqpoint{2.724592in}{2.411609in}}%
\pgfpathlineto{\pgfqpoint{2.726916in}{2.390765in}}%
\pgfpathlineto{\pgfqpoint{2.729240in}{2.433281in}}%
\pgfpathlineto{\pgfqpoint{2.731564in}{2.375462in}}%
\pgfpathlineto{\pgfqpoint{2.733888in}{2.403706in}}%
\pgfpathlineto{\pgfqpoint{2.736212in}{2.376452in}}%
\pgfpathlineto{\pgfqpoint{2.738536in}{2.403101in}}%
\pgfpathlineto{\pgfqpoint{2.740860in}{2.394462in}}%
\pgfpathlineto{\pgfqpoint{2.743184in}{2.463530in}}%
\pgfpathlineto{\pgfqpoint{2.745508in}{2.426537in}}%
\pgfpathlineto{\pgfqpoint{2.747832in}{2.430382in}}%
\pgfpathlineto{\pgfqpoint{2.750156in}{2.395742in}}%
\pgfpathlineto{\pgfqpoint{2.752480in}{2.386381in}}%
\pgfpathlineto{\pgfqpoint{2.754804in}{2.428419in}}%
\pgfpathlineto{\pgfqpoint{2.757128in}{2.432000in}}%
\pgfpathlineto{\pgfqpoint{2.759452in}{2.475130in}}%
\pgfpathlineto{\pgfqpoint{2.764100in}{2.416231in}}%
\pgfpathlineto{\pgfqpoint{2.766424in}{2.433264in}}%
\pgfpathlineto{\pgfqpoint{2.768748in}{2.403271in}}%
\pgfpathlineto{\pgfqpoint{2.771072in}{2.430315in}}%
\pgfpathlineto{\pgfqpoint{2.773395in}{2.401150in}}%
\pgfpathlineto{\pgfqpoint{2.778043in}{2.483295in}}%
\pgfpathlineto{\pgfqpoint{2.780367in}{2.451921in}}%
\pgfpathlineto{\pgfqpoint{2.782691in}{2.474511in}}%
\pgfpathlineto{\pgfqpoint{2.785015in}{2.464889in}}%
\pgfpathlineto{\pgfqpoint{2.787339in}{2.494819in}}%
\pgfpathlineto{\pgfqpoint{2.789663in}{2.415057in}}%
\pgfpathlineto{\pgfqpoint{2.791987in}{2.487972in}}%
\pgfpathlineto{\pgfqpoint{2.794311in}{2.494416in}}%
\pgfpathlineto{\pgfqpoint{2.798959in}{2.470722in}}%
\pgfpathlineto{\pgfqpoint{2.801283in}{2.473399in}}%
\pgfpathlineto{\pgfqpoint{2.803607in}{2.503919in}}%
\pgfpathlineto{\pgfqpoint{2.805931in}{2.501135in}}%
\pgfpathlineto{\pgfqpoint{2.808255in}{2.483273in}}%
\pgfpathlineto{\pgfqpoint{2.810579in}{2.484867in}}%
\pgfpathlineto{\pgfqpoint{2.812903in}{2.567610in}}%
\pgfpathlineto{\pgfqpoint{2.815227in}{2.483939in}}%
\pgfpathlineto{\pgfqpoint{2.817551in}{2.470758in}}%
\pgfpathlineto{\pgfqpoint{2.822199in}{2.520463in}}%
\pgfpathlineto{\pgfqpoint{2.824523in}{2.494744in}}%
\pgfpathlineto{\pgfqpoint{2.826847in}{2.504259in}}%
\pgfpathlineto{\pgfqpoint{2.829171in}{2.484213in}}%
\pgfpathlineto{\pgfqpoint{2.831495in}{2.537971in}}%
\pgfpathlineto{\pgfqpoint{2.833819in}{2.538870in}}%
\pgfpathlineto{\pgfqpoint{2.836143in}{2.542476in}}%
\pgfpathlineto{\pgfqpoint{2.840791in}{2.573213in}}%
\pgfpathlineto{\pgfqpoint{2.843115in}{2.512092in}}%
\pgfpathlineto{\pgfqpoint{2.845438in}{2.538508in}}%
\pgfpathlineto{\pgfqpoint{2.847762in}{2.538572in}}%
\pgfpathlineto{\pgfqpoint{2.850086in}{2.559008in}}%
\pgfpathlineto{\pgfqpoint{2.852410in}{2.563336in}}%
\pgfpathlineto{\pgfqpoint{2.854734in}{2.519880in}}%
\pgfpathlineto{\pgfqpoint{2.857058in}{2.503207in}}%
\pgfpathlineto{\pgfqpoint{2.859382in}{2.539210in}}%
\pgfpathlineto{\pgfqpoint{2.861706in}{2.544182in}}%
\pgfpathlineto{\pgfqpoint{2.864030in}{2.582710in}}%
\pgfpathlineto{\pgfqpoint{2.866354in}{2.530604in}}%
\pgfpathlineto{\pgfqpoint{2.868678in}{2.591506in}}%
\pgfpathlineto{\pgfqpoint{2.871002in}{2.582522in}}%
\pgfpathlineto{\pgfqpoint{2.873326in}{2.580652in}}%
\pgfpathlineto{\pgfqpoint{2.875650in}{2.610252in}}%
\pgfpathlineto{\pgfqpoint{2.877974in}{2.525285in}}%
\pgfpathlineto{\pgfqpoint{2.880298in}{2.583039in}}%
\pgfpathlineto{\pgfqpoint{2.882622in}{2.601050in}}%
\pgfpathlineto{\pgfqpoint{2.884946in}{2.569329in}}%
\pgfpathlineto{\pgfqpoint{2.887270in}{2.575913in}}%
\pgfpathlineto{\pgfqpoint{2.889594in}{2.559433in}}%
\pgfpathlineto{\pgfqpoint{2.894242in}{2.611691in}}%
\pgfpathlineto{\pgfqpoint{2.896566in}{2.586460in}}%
\pgfpathlineto{\pgfqpoint{2.898890in}{2.610951in}}%
\pgfpathlineto{\pgfqpoint{2.901214in}{2.607585in}}%
\pgfpathlineto{\pgfqpoint{2.903538in}{2.568177in}}%
\pgfpathlineto{\pgfqpoint{2.905862in}{2.626058in}}%
\pgfpathlineto{\pgfqpoint{2.908186in}{2.583792in}}%
\pgfpathlineto{\pgfqpoint{2.910510in}{2.649949in}}%
\pgfpathlineto{\pgfqpoint{2.912834in}{2.640522in}}%
\pgfpathlineto{\pgfqpoint{2.915157in}{2.619758in}}%
\pgfpathlineto{\pgfqpoint{2.917481in}{2.570045in}}%
\pgfpathlineto{\pgfqpoint{2.919805in}{2.654441in}}%
\pgfpathlineto{\pgfqpoint{2.922129in}{2.627047in}}%
\pgfpathlineto{\pgfqpoint{2.924453in}{2.654264in}}%
\pgfpathlineto{\pgfqpoint{2.926777in}{2.620271in}}%
\pgfpathlineto{\pgfqpoint{2.929101in}{2.657692in}}%
\pgfpathlineto{\pgfqpoint{2.931425in}{2.641458in}}%
\pgfpathlineto{\pgfqpoint{2.933749in}{2.670238in}}%
\pgfpathlineto{\pgfqpoint{2.938397in}{2.625655in}}%
\pgfpathlineto{\pgfqpoint{2.940721in}{2.588784in}}%
\pgfpathlineto{\pgfqpoint{2.943045in}{2.673194in}}%
\pgfpathlineto{\pgfqpoint{2.945369in}{2.667176in}}%
\pgfpathlineto{\pgfqpoint{2.947693in}{2.709390in}}%
\pgfpathlineto{\pgfqpoint{2.950017in}{2.639006in}}%
\pgfpathlineto{\pgfqpoint{2.952341in}{2.662985in}}%
\pgfpathlineto{\pgfqpoint{2.954665in}{2.630043in}}%
\pgfpathlineto{\pgfqpoint{2.956989in}{2.678066in}}%
\pgfpathlineto{\pgfqpoint{2.959313in}{2.651994in}}%
\pgfpathlineto{\pgfqpoint{2.961637in}{2.662380in}}%
\pgfpathlineto{\pgfqpoint{2.963961in}{2.689608in}}%
\pgfpathlineto{\pgfqpoint{2.966285in}{2.735867in}}%
\pgfpathlineto{\pgfqpoint{2.968609in}{2.665027in}}%
\pgfpathlineto{\pgfqpoint{2.970933in}{2.712393in}}%
\pgfpathlineto{\pgfqpoint{2.973257in}{2.716364in}}%
\pgfpathlineto{\pgfqpoint{2.975581in}{2.713245in}}%
\pgfpathlineto{\pgfqpoint{2.977905in}{2.733526in}}%
\pgfpathlineto{\pgfqpoint{2.980229in}{2.698778in}}%
\pgfpathlineto{\pgfqpoint{2.982553in}{2.687001in}}%
\pgfpathlineto{\pgfqpoint{2.987200in}{2.730748in}}%
\pgfpathlineto{\pgfqpoint{2.989524in}{2.666275in}}%
\pgfpathlineto{\pgfqpoint{2.991848in}{2.705278in}}%
\pgfpathlineto{\pgfqpoint{2.994172in}{2.707332in}}%
\pgfpathlineto{\pgfqpoint{2.996496in}{2.728961in}}%
\pgfpathlineto{\pgfqpoint{2.998820in}{2.733866in}}%
\pgfpathlineto{\pgfqpoint{3.001144in}{2.743943in}}%
\pgfpathlineto{\pgfqpoint{3.003468in}{2.772962in}}%
\pgfpathlineto{\pgfqpoint{3.005792in}{2.740716in}}%
\pgfpathlineto{\pgfqpoint{3.008116in}{2.724648in}}%
\pgfpathlineto{\pgfqpoint{3.010440in}{2.763666in}}%
\pgfpathlineto{\pgfqpoint{3.012764in}{2.769125in}}%
\pgfpathlineto{\pgfqpoint{3.015088in}{2.725071in}}%
\pgfpathlineto{\pgfqpoint{3.017412in}{2.753733in}}%
\pgfpathlineto{\pgfqpoint{3.019736in}{2.723350in}}%
\pgfpathlineto{\pgfqpoint{3.022060in}{2.749011in}}%
\pgfpathlineto{\pgfqpoint{3.024384in}{2.758021in}}%
\pgfpathlineto{\pgfqpoint{3.026708in}{2.692618in}}%
\pgfpathlineto{\pgfqpoint{3.031356in}{2.801167in}}%
\pgfpathlineto{\pgfqpoint{3.033680in}{2.682089in}}%
\pgfpathlineto{\pgfqpoint{3.036004in}{2.760346in}}%
\pgfpathlineto{\pgfqpoint{3.038328in}{2.741438in}}%
\pgfpathlineto{\pgfqpoint{3.042976in}{2.789943in}}%
\pgfpathlineto{\pgfqpoint{3.045300in}{2.735548in}}%
\pgfpathlineto{\pgfqpoint{3.047624in}{2.800876in}}%
\pgfpathlineto{\pgfqpoint{3.049948in}{2.788289in}}%
\pgfpathlineto{\pgfqpoint{3.052272in}{2.761880in}}%
\pgfpathlineto{\pgfqpoint{3.054596in}{2.750006in}}%
\pgfpathlineto{\pgfqpoint{3.056919in}{2.802165in}}%
\pgfpathlineto{\pgfqpoint{3.059243in}{2.780274in}}%
\pgfpathlineto{\pgfqpoint{3.061567in}{2.776183in}}%
\pgfpathlineto{\pgfqpoint{3.063891in}{2.834807in}}%
\pgfpathlineto{\pgfqpoint{3.066215in}{2.790539in}}%
\pgfpathlineto{\pgfqpoint{3.068539in}{2.816429in}}%
\pgfpathlineto{\pgfqpoint{3.070863in}{2.745893in}}%
\pgfpathlineto{\pgfqpoint{3.073187in}{2.813045in}}%
\pgfpathlineto{\pgfqpoint{3.075511in}{2.797094in}}%
\pgfpathlineto{\pgfqpoint{3.077835in}{2.820942in}}%
\pgfpathlineto{\pgfqpoint{3.080159in}{2.804895in}}%
\pgfpathlineto{\pgfqpoint{3.082483in}{2.800790in}}%
\pgfpathlineto{\pgfqpoint{3.084807in}{2.771726in}}%
\pgfpathlineto{\pgfqpoint{3.087131in}{2.857139in}}%
\pgfpathlineto{\pgfqpoint{3.089455in}{2.844116in}}%
\pgfpathlineto{\pgfqpoint{3.091779in}{2.776641in}}%
\pgfpathlineto{\pgfqpoint{3.094103in}{2.847733in}}%
\pgfpathlineto{\pgfqpoint{3.098751in}{2.849100in}}%
\pgfpathlineto{\pgfqpoint{3.101075in}{2.872763in}}%
\pgfpathlineto{\pgfqpoint{3.103399in}{2.807748in}}%
\pgfpathlineto{\pgfqpoint{3.105723in}{2.832356in}}%
\pgfpathlineto{\pgfqpoint{3.108047in}{2.839376in}}%
\pgfpathlineto{\pgfqpoint{3.110371in}{2.826614in}}%
\pgfpathlineto{\pgfqpoint{3.112695in}{2.856859in}}%
\pgfpathlineto{\pgfqpoint{3.115019in}{2.798932in}}%
\pgfpathlineto{\pgfqpoint{3.117343in}{2.890897in}}%
\pgfpathlineto{\pgfqpoint{3.121991in}{2.832900in}}%
\pgfpathlineto{\pgfqpoint{3.124315in}{2.840946in}}%
\pgfpathlineto{\pgfqpoint{3.126638in}{2.874610in}}%
\pgfpathlineto{\pgfqpoint{3.128962in}{2.845713in}}%
\pgfpathlineto{\pgfqpoint{3.131286in}{2.872429in}}%
\pgfpathlineto{\pgfqpoint{3.133610in}{2.862244in}}%
\pgfpathlineto{\pgfqpoint{3.135934in}{2.863038in}}%
\pgfpathlineto{\pgfqpoint{3.138258in}{2.897522in}}%
\pgfpathlineto{\pgfqpoint{3.140582in}{2.860651in}}%
\pgfpathlineto{\pgfqpoint{3.142906in}{2.860385in}}%
\pgfpathlineto{\pgfqpoint{3.145230in}{2.912874in}}%
\pgfpathlineto{\pgfqpoint{3.147554in}{2.884144in}}%
\pgfpathlineto{\pgfqpoint{3.149878in}{2.890602in}}%
\pgfpathlineto{\pgfqpoint{3.152202in}{2.891234in}}%
\pgfpathlineto{\pgfqpoint{3.154526in}{2.900645in}}%
\pgfpathlineto{\pgfqpoint{3.156850in}{2.879691in}}%
\pgfpathlineto{\pgfqpoint{3.159174in}{2.879399in}}%
\pgfpathlineto{\pgfqpoint{3.161498in}{2.857407in}}%
\pgfpathlineto{\pgfqpoint{3.163822in}{2.928056in}}%
\pgfpathlineto{\pgfqpoint{3.166146in}{2.880072in}}%
\pgfpathlineto{\pgfqpoint{3.168470in}{2.881712in}}%
\pgfpathlineto{\pgfqpoint{3.170794in}{2.880414in}}%
\pgfpathlineto{\pgfqpoint{3.173118in}{2.841887in}}%
\pgfpathlineto{\pgfqpoint{3.175442in}{2.932575in}}%
\pgfpathlineto{\pgfqpoint{3.177766in}{2.906048in}}%
\pgfpathlineto{\pgfqpoint{3.184738in}{2.974724in}}%
\pgfpathlineto{\pgfqpoint{3.187062in}{2.856454in}}%
\pgfpathlineto{\pgfqpoint{3.189386in}{2.905907in}}%
\pgfpathlineto{\pgfqpoint{3.191710in}{2.870903in}}%
\pgfpathlineto{\pgfqpoint{3.194034in}{2.902403in}}%
\pgfpathlineto{\pgfqpoint{3.196357in}{2.906153in}}%
\pgfpathlineto{\pgfqpoint{3.198681in}{2.950017in}}%
\pgfpathlineto{\pgfqpoint{3.201005in}{2.898235in}}%
\pgfpathlineto{\pgfqpoint{3.203329in}{2.981209in}}%
\pgfpathlineto{\pgfqpoint{3.205653in}{2.924294in}}%
\pgfpathlineto{\pgfqpoint{3.207977in}{2.908746in}}%
\pgfpathlineto{\pgfqpoint{3.210301in}{2.958981in}}%
\pgfpathlineto{\pgfqpoint{3.212625in}{2.970228in}}%
\pgfpathlineto{\pgfqpoint{3.214949in}{2.899633in}}%
\pgfpathlineto{\pgfqpoint{3.217273in}{2.896959in}}%
\pgfpathlineto{\pgfqpoint{3.219597in}{2.913810in}}%
\pgfpathlineto{\pgfqpoint{3.221921in}{2.963124in}}%
\pgfpathlineto{\pgfqpoint{3.224245in}{2.927776in}}%
\pgfpathlineto{\pgfqpoint{3.226569in}{2.922961in}}%
\pgfpathlineto{\pgfqpoint{3.228893in}{2.970543in}}%
\pgfpathlineto{\pgfqpoint{3.231217in}{2.925842in}}%
\pgfpathlineto{\pgfqpoint{3.233541in}{2.971092in}}%
\pgfpathlineto{\pgfqpoint{3.235865in}{2.975445in}}%
\pgfpathlineto{\pgfqpoint{3.240513in}{2.945328in}}%
\pgfpathlineto{\pgfqpoint{3.242837in}{2.944984in}}%
\pgfpathlineto{\pgfqpoint{3.247485in}{3.005456in}}%
\pgfpathlineto{\pgfqpoint{3.252133in}{2.924409in}}%
\pgfpathlineto{\pgfqpoint{3.254457in}{3.024624in}}%
\pgfpathlineto{\pgfqpoint{3.256781in}{2.931807in}}%
\pgfpathlineto{\pgfqpoint{3.261429in}{3.039509in}}%
\pgfpathlineto{\pgfqpoint{3.263753in}{2.952221in}}%
\pgfpathlineto{\pgfqpoint{3.266076in}{2.514416in}}%
\pgfpathlineto{\pgfqpoint{3.268400in}{2.455669in}}%
\pgfpathlineto{\pgfqpoint{3.270724in}{2.488723in}}%
\pgfpathlineto{\pgfqpoint{3.273048in}{2.497961in}}%
\pgfpathlineto{\pgfqpoint{3.275372in}{2.451486in}}%
\pgfpathlineto{\pgfqpoint{3.280020in}{2.498019in}}%
\pgfpathlineto{\pgfqpoint{3.284668in}{2.466887in}}%
\pgfpathlineto{\pgfqpoint{3.286992in}{2.500607in}}%
\pgfpathlineto{\pgfqpoint{3.289316in}{2.516515in}}%
\pgfpathlineto{\pgfqpoint{3.291640in}{2.512280in}}%
\pgfpathlineto{\pgfqpoint{3.293964in}{2.464634in}}%
\pgfpathlineto{\pgfqpoint{3.296288in}{2.522996in}}%
\pgfpathlineto{\pgfqpoint{3.298612in}{2.524711in}}%
\pgfpathlineto{\pgfqpoint{3.300936in}{2.496132in}}%
\pgfpathlineto{\pgfqpoint{3.303260in}{2.491722in}}%
\pgfpathlineto{\pgfqpoint{3.305584in}{2.499700in}}%
\pgfpathlineto{\pgfqpoint{3.307908in}{2.493041in}}%
\pgfpathlineto{\pgfqpoint{3.310232in}{2.496806in}}%
\pgfpathlineto{\pgfqpoint{3.312556in}{2.492260in}}%
\pgfpathlineto{\pgfqpoint{3.314880in}{2.464098in}}%
\pgfpathlineto{\pgfqpoint{3.317204in}{2.522344in}}%
\pgfpathlineto{\pgfqpoint{3.319528in}{2.484845in}}%
\pgfpathlineto{\pgfqpoint{3.321852in}{2.488687in}}%
\pgfpathlineto{\pgfqpoint{3.324176in}{2.519371in}}%
\pgfpathlineto{\pgfqpoint{3.326500in}{2.517351in}}%
\pgfpathlineto{\pgfqpoint{3.328824in}{2.609214in}}%
\pgfpathlineto{\pgfqpoint{3.331148in}{2.481046in}}%
\pgfpathlineto{\pgfqpoint{3.333472in}{2.544986in}}%
\pgfpathlineto{\pgfqpoint{3.335796in}{2.523266in}}%
\pgfpathlineto{\pgfqpoint{3.338119in}{2.513712in}}%
\pgfpathlineto{\pgfqpoint{3.340443in}{2.545670in}}%
\pgfpathlineto{\pgfqpoint{3.342767in}{2.557712in}}%
\pgfpathlineto{\pgfqpoint{3.345091in}{2.494042in}}%
\pgfpathlineto{\pgfqpoint{3.347415in}{2.542120in}}%
\pgfpathlineto{\pgfqpoint{3.349739in}{2.534603in}}%
\pgfpathlineto{\pgfqpoint{3.352063in}{2.568388in}}%
\pgfpathlineto{\pgfqpoint{3.354387in}{2.542167in}}%
\pgfpathlineto{\pgfqpoint{3.356711in}{2.644090in}}%
\pgfpathlineto{\pgfqpoint{3.359035in}{2.596650in}}%
\pgfpathlineto{\pgfqpoint{3.361359in}{2.502753in}}%
\pgfpathlineto{\pgfqpoint{3.363683in}{2.547836in}}%
\pgfpathlineto{\pgfqpoint{3.368331in}{2.525821in}}%
\pgfpathlineto{\pgfqpoint{3.370655in}{2.536074in}}%
\pgfpathlineto{\pgfqpoint{3.372979in}{2.568784in}}%
\pgfpathlineto{\pgfqpoint{3.375303in}{2.581170in}}%
\pgfpathlineto{\pgfqpoint{3.377627in}{2.536285in}}%
\pgfpathlineto{\pgfqpoint{3.379951in}{2.562919in}}%
\pgfpathlineto{\pgfqpoint{3.382275in}{2.545182in}}%
\pgfpathlineto{\pgfqpoint{3.384599in}{2.581888in}}%
\pgfpathlineto{\pgfqpoint{3.386923in}{2.579499in}}%
\pgfpathlineto{\pgfqpoint{3.389247in}{2.600360in}}%
\pgfpathlineto{\pgfqpoint{3.391571in}{2.564308in}}%
\pgfpathlineto{\pgfqpoint{3.393895in}{2.607075in}}%
\pgfpathlineto{\pgfqpoint{3.396219in}{2.584915in}}%
\pgfpathlineto{\pgfqpoint{3.400867in}{2.561127in}}%
\pgfpathlineto{\pgfqpoint{3.403191in}{2.574481in}}%
\pgfpathlineto{\pgfqpoint{3.407838in}{2.641285in}}%
\pgfpathlineto{\pgfqpoint{3.410162in}{2.593739in}}%
\pgfpathlineto{\pgfqpoint{3.412486in}{2.632704in}}%
\pgfpathlineto{\pgfqpoint{3.414810in}{2.627454in}}%
\pgfpathlineto{\pgfqpoint{3.417134in}{2.586375in}}%
\pgfpathlineto{\pgfqpoint{3.421782in}{2.561139in}}%
\pgfpathlineto{\pgfqpoint{3.426430in}{2.614219in}}%
\pgfpathlineto{\pgfqpoint{3.428754in}{2.554787in}}%
\pgfpathlineto{\pgfqpoint{3.433402in}{2.561080in}}%
\pgfpathlineto{\pgfqpoint{3.435726in}{2.632092in}}%
\pgfpathlineto{\pgfqpoint{3.438050in}{2.562039in}}%
\pgfpathlineto{\pgfqpoint{3.440374in}{2.636635in}}%
\pgfpathlineto{\pgfqpoint{3.442698in}{2.636682in}}%
\pgfpathlineto{\pgfqpoint{3.445022in}{2.579782in}}%
\pgfpathlineto{\pgfqpoint{3.447346in}{2.612950in}}%
\pgfpathlineto{\pgfqpoint{3.449670in}{2.574943in}}%
\pgfpathlineto{\pgfqpoint{3.454318in}{2.596023in}}%
\pgfpathlineto{\pgfqpoint{3.458966in}{2.665588in}}%
\pgfpathlineto{\pgfqpoint{3.461290in}{2.661159in}}%
\pgfpathlineto{\pgfqpoint{3.463614in}{2.609102in}}%
\pgfpathlineto{\pgfqpoint{3.465938in}{2.658175in}}%
\pgfpathlineto{\pgfqpoint{3.468262in}{2.646750in}}%
\pgfpathlineto{\pgfqpoint{3.470586in}{2.575858in}}%
\pgfpathlineto{\pgfqpoint{3.472910in}{2.580614in}}%
\pgfpathlineto{\pgfqpoint{3.475234in}{2.651878in}}%
\pgfpathlineto{\pgfqpoint{3.477557in}{2.596507in}}%
\pgfpathlineto{\pgfqpoint{3.479881in}{2.638664in}}%
\pgfpathlineto{\pgfqpoint{3.482205in}{2.648015in}}%
\pgfpathlineto{\pgfqpoint{3.484529in}{2.668044in}}%
\pgfpathlineto{\pgfqpoint{3.486853in}{2.619528in}}%
\pgfpathlineto{\pgfqpoint{3.489177in}{2.615262in}}%
\pgfpathlineto{\pgfqpoint{3.491501in}{2.662332in}}%
\pgfpathlineto{\pgfqpoint{3.493825in}{2.609452in}}%
\pgfpathlineto{\pgfqpoint{3.496149in}{2.656674in}}%
\pgfpathlineto{\pgfqpoint{3.498473in}{2.596683in}}%
\pgfpathlineto{\pgfqpoint{3.500797in}{2.667001in}}%
\pgfpathlineto{\pgfqpoint{3.503121in}{2.631417in}}%
\pgfpathlineto{\pgfqpoint{3.505445in}{2.633200in}}%
\pgfpathlineto{\pgfqpoint{3.507769in}{2.624655in}}%
\pgfpathlineto{\pgfqpoint{3.510093in}{2.572974in}}%
\pgfpathlineto{\pgfqpoint{3.512417in}{2.651786in}}%
\pgfpathlineto{\pgfqpoint{3.517065in}{2.667081in}}%
\pgfpathlineto{\pgfqpoint{3.519389in}{2.585545in}}%
\pgfpathlineto{\pgfqpoint{3.521713in}{2.669810in}}%
\pgfpathlineto{\pgfqpoint{3.524037in}{2.706773in}}%
\pgfpathlineto{\pgfqpoint{3.526361in}{2.655620in}}%
\pgfpathlineto{\pgfqpoint{3.528685in}{2.663068in}}%
\pgfpathlineto{\pgfqpoint{3.531009in}{2.641430in}}%
\pgfpathlineto{\pgfqpoint{3.535657in}{2.684258in}}%
\pgfpathlineto{\pgfqpoint{3.537981in}{2.685873in}}%
\pgfpathlineto{\pgfqpoint{3.540305in}{2.685527in}}%
\pgfpathlineto{\pgfqpoint{3.542629in}{2.613145in}}%
\pgfpathlineto{\pgfqpoint{3.544953in}{2.653678in}}%
\pgfpathlineto{\pgfqpoint{3.547277in}{2.653503in}}%
\pgfpathlineto{\pgfqpoint{3.549600in}{2.680182in}}%
\pgfpathlineto{\pgfqpoint{3.551924in}{2.594970in}}%
\pgfpathlineto{\pgfqpoint{3.554248in}{2.651310in}}%
\pgfpathlineto{\pgfqpoint{3.556572in}{2.633973in}}%
\pgfpathlineto{\pgfqpoint{3.561220in}{2.726053in}}%
\pgfpathlineto{\pgfqpoint{3.563544in}{2.659522in}}%
\pgfpathlineto{\pgfqpoint{3.565868in}{2.649361in}}%
\pgfpathlineto{\pgfqpoint{3.568192in}{2.674092in}}%
\pgfpathlineto{\pgfqpoint{3.570516in}{2.648295in}}%
\pgfpathlineto{\pgfqpoint{3.572840in}{2.656111in}}%
\pgfpathlineto{\pgfqpoint{3.575164in}{2.628783in}}%
\pgfpathlineto{\pgfqpoint{3.577488in}{2.703006in}}%
\pgfpathlineto{\pgfqpoint{3.582136in}{2.668555in}}%
\pgfpathlineto{\pgfqpoint{3.584460in}{2.706519in}}%
\pgfpathlineto{\pgfqpoint{3.586784in}{2.684794in}}%
\pgfpathlineto{\pgfqpoint{3.589108in}{2.721909in}}%
\pgfpathlineto{\pgfqpoint{3.591432in}{2.689530in}}%
\pgfpathlineto{\pgfqpoint{3.593756in}{2.709500in}}%
\pgfpathlineto{\pgfqpoint{3.596080in}{2.673727in}}%
\pgfpathlineto{\pgfqpoint{3.603052in}{2.716954in}}%
\pgfpathlineto{\pgfqpoint{3.605376in}{2.682033in}}%
\pgfpathlineto{\pgfqpoint{3.607700in}{2.706594in}}%
\pgfpathlineto{\pgfqpoint{3.610024in}{2.676620in}}%
\pgfpathlineto{\pgfqpoint{3.612348in}{2.749251in}}%
\pgfpathlineto{\pgfqpoint{3.616996in}{2.622276in}}%
\pgfpathlineto{\pgfqpoint{3.619319in}{2.725898in}}%
\pgfpathlineto{\pgfqpoint{3.621643in}{2.719304in}}%
\pgfpathlineto{\pgfqpoint{3.623967in}{2.663324in}}%
\pgfpathlineto{\pgfqpoint{3.626291in}{2.729172in}}%
\pgfpathlineto{\pgfqpoint{3.628615in}{2.730871in}}%
\pgfpathlineto{\pgfqpoint{3.633263in}{2.698120in}}%
\pgfpathlineto{\pgfqpoint{3.635587in}{2.702984in}}%
\pgfpathlineto{\pgfqpoint{3.637911in}{2.695676in}}%
\pgfpathlineto{\pgfqpoint{3.640235in}{2.710088in}}%
\pgfpathlineto{\pgfqpoint{3.642559in}{2.717477in}}%
\pgfpathlineto{\pgfqpoint{3.644883in}{2.696243in}}%
\pgfpathlineto{\pgfqpoint{3.647207in}{2.748052in}}%
\pgfpathlineto{\pgfqpoint{3.649531in}{2.683734in}}%
\pgfpathlineto{\pgfqpoint{3.651855in}{2.715141in}}%
\pgfpathlineto{\pgfqpoint{3.654179in}{2.725299in}}%
\pgfpathlineto{\pgfqpoint{3.656503in}{2.715698in}}%
\pgfpathlineto{\pgfqpoint{3.658827in}{2.716531in}}%
\pgfpathlineto{\pgfqpoint{3.661151in}{2.677592in}}%
\pgfpathlineto{\pgfqpoint{3.663475in}{2.747534in}}%
\pgfpathlineto{\pgfqpoint{3.665799in}{2.734498in}}%
\pgfpathlineto{\pgfqpoint{3.668123in}{2.710724in}}%
\pgfpathlineto{\pgfqpoint{3.670447in}{2.730973in}}%
\pgfpathlineto{\pgfqpoint{3.672771in}{2.738939in}}%
\pgfpathlineto{\pgfqpoint{3.675095in}{2.778247in}}%
\pgfpathlineto{\pgfqpoint{3.677419in}{2.702433in}}%
\pgfpathlineto{\pgfqpoint{3.679743in}{2.710339in}}%
\pgfpathlineto{\pgfqpoint{3.682067in}{2.747083in}}%
\pgfpathlineto{\pgfqpoint{3.684391in}{2.735582in}}%
\pgfpathlineto{\pgfqpoint{3.686715in}{2.767759in}}%
\pgfpathlineto{\pgfqpoint{3.689038in}{2.750652in}}%
\pgfpathlineto{\pgfqpoint{3.691362in}{2.724363in}}%
\pgfpathlineto{\pgfqpoint{3.693686in}{2.729748in}}%
\pgfpathlineto{\pgfqpoint{3.696010in}{2.755690in}}%
\pgfpathlineto{\pgfqpoint{3.698334in}{2.736637in}}%
\pgfpathlineto{\pgfqpoint{3.700658in}{2.784805in}}%
\pgfpathlineto{\pgfqpoint{3.702982in}{2.690490in}}%
\pgfpathlineto{\pgfqpoint{3.705306in}{2.701602in}}%
\pgfpathlineto{\pgfqpoint{3.707630in}{2.725025in}}%
\pgfpathlineto{\pgfqpoint{3.712278in}{2.816926in}}%
\pgfpathlineto{\pgfqpoint{3.716926in}{2.696597in}}%
\pgfpathlineto{\pgfqpoint{3.719250in}{2.758082in}}%
\pgfpathlineto{\pgfqpoint{3.721574in}{2.715859in}}%
\pgfpathlineto{\pgfqpoint{3.723898in}{2.720559in}}%
\pgfpathlineto{\pgfqpoint{3.726222in}{2.752871in}}%
\pgfpathlineto{\pgfqpoint{3.728546in}{2.731941in}}%
\pgfpathlineto{\pgfqpoint{3.730870in}{2.758654in}}%
\pgfpathlineto{\pgfqpoint{3.733194in}{2.738995in}}%
\pgfpathlineto{\pgfqpoint{3.735518in}{2.785209in}}%
\pgfpathlineto{\pgfqpoint{3.737842in}{2.744282in}}%
\pgfpathlineto{\pgfqpoint{3.740166in}{2.743336in}}%
\pgfpathlineto{\pgfqpoint{3.742490in}{2.808465in}}%
\pgfpathlineto{\pgfqpoint{3.744814in}{2.778358in}}%
\pgfpathlineto{\pgfqpoint{3.747138in}{2.774687in}}%
\pgfpathlineto{\pgfqpoint{3.749462in}{2.763058in}}%
\pgfpathlineto{\pgfqpoint{3.751786in}{2.800154in}}%
\pgfpathlineto{\pgfqpoint{3.754110in}{2.717441in}}%
\pgfpathlineto{\pgfqpoint{3.756434in}{2.756053in}}%
\pgfpathlineto{\pgfqpoint{3.758757in}{2.814126in}}%
\pgfpathlineto{\pgfqpoint{3.761081in}{2.784700in}}%
\pgfpathlineto{\pgfqpoint{3.763405in}{2.790231in}}%
\pgfpathlineto{\pgfqpoint{3.765729in}{2.771813in}}%
\pgfpathlineto{\pgfqpoint{3.768053in}{2.775921in}}%
\pgfpathlineto{\pgfqpoint{3.770377in}{2.817245in}}%
\pgfpathlineto{\pgfqpoint{3.772701in}{2.751821in}}%
\pgfpathlineto{\pgfqpoint{3.775025in}{2.836875in}}%
\pgfpathlineto{\pgfqpoint{3.777349in}{2.757230in}}%
\pgfpathlineto{\pgfqpoint{3.779673in}{2.741049in}}%
\pgfpathlineto{\pgfqpoint{3.784321in}{2.769931in}}%
\pgfpathlineto{\pgfqpoint{3.786645in}{2.756212in}}%
\pgfpathlineto{\pgfqpoint{3.788969in}{2.761859in}}%
\pgfpathlineto{\pgfqpoint{3.791293in}{2.677885in}}%
\pgfpathlineto{\pgfqpoint{3.793617in}{2.811400in}}%
\pgfpathlineto{\pgfqpoint{3.795941in}{2.753075in}}%
\pgfpathlineto{\pgfqpoint{3.798265in}{2.756276in}}%
\pgfpathlineto{\pgfqpoint{3.800589in}{2.856394in}}%
\pgfpathlineto{\pgfqpoint{3.802913in}{2.817272in}}%
\pgfpathlineto{\pgfqpoint{3.805237in}{2.757412in}}%
\pgfpathlineto{\pgfqpoint{3.807561in}{2.779817in}}%
\pgfpathlineto{\pgfqpoint{3.809885in}{2.786593in}}%
\pgfpathlineto{\pgfqpoint{3.812209in}{2.765381in}}%
\pgfpathlineto{\pgfqpoint{3.814533in}{2.790790in}}%
\pgfpathlineto{\pgfqpoint{3.816857in}{2.771501in}}%
\pgfpathlineto{\pgfqpoint{3.821505in}{2.831467in}}%
\pgfpathlineto{\pgfqpoint{3.823829in}{2.751750in}}%
\pgfpathlineto{\pgfqpoint{3.826153in}{2.791189in}}%
\pgfpathlineto{\pgfqpoint{3.828477in}{2.802891in}}%
\pgfpathlineto{\pgfqpoint{3.830800in}{2.775858in}}%
\pgfpathlineto{\pgfqpoint{3.833124in}{2.784011in}}%
\pgfpathlineto{\pgfqpoint{3.835448in}{2.765581in}}%
\pgfpathlineto{\pgfqpoint{3.837772in}{2.824602in}}%
\pgfpathlineto{\pgfqpoint{3.840096in}{2.817782in}}%
\pgfpathlineto{\pgfqpoint{3.842420in}{2.829525in}}%
\pgfpathlineto{\pgfqpoint{3.844744in}{2.811562in}}%
\pgfpathlineto{\pgfqpoint{3.847068in}{2.314106in}}%
\pgfpathlineto{\pgfqpoint{3.849392in}{2.344860in}}%
\pgfpathlineto{\pgfqpoint{3.854040in}{2.267682in}}%
\pgfpathlineto{\pgfqpoint{3.856364in}{2.306595in}}%
\pgfpathlineto{\pgfqpoint{3.858688in}{2.307175in}}%
\pgfpathlineto{\pgfqpoint{3.861012in}{2.348231in}}%
\pgfpathlineto{\pgfqpoint{3.863336in}{2.325066in}}%
\pgfpathlineto{\pgfqpoint{3.865660in}{2.332273in}}%
\pgfpathlineto{\pgfqpoint{3.867984in}{2.295077in}}%
\pgfpathlineto{\pgfqpoint{3.870308in}{2.371949in}}%
\pgfpathlineto{\pgfqpoint{3.872632in}{2.323881in}}%
\pgfpathlineto{\pgfqpoint{3.874956in}{2.321001in}}%
\pgfpathlineto{\pgfqpoint{3.877280in}{2.316164in}}%
\pgfpathlineto{\pgfqpoint{3.879604in}{2.339381in}}%
\pgfpathlineto{\pgfqpoint{3.881928in}{2.316918in}}%
\pgfpathlineto{\pgfqpoint{3.884252in}{2.335016in}}%
\pgfpathlineto{\pgfqpoint{3.886576in}{2.271659in}}%
\pgfpathlineto{\pgfqpoint{3.891224in}{2.345505in}}%
\pgfpathlineto{\pgfqpoint{3.893548in}{2.307953in}}%
\pgfpathlineto{\pgfqpoint{3.895872in}{2.318894in}}%
\pgfpathlineto{\pgfqpoint{3.898196in}{2.299098in}}%
\pgfpathlineto{\pgfqpoint{3.900519in}{2.339507in}}%
\pgfpathlineto{\pgfqpoint{3.902843in}{2.336183in}}%
\pgfpathlineto{\pgfqpoint{3.905167in}{2.386645in}}%
\pgfpathlineto{\pgfqpoint{3.909815in}{2.326207in}}%
\pgfpathlineto{\pgfqpoint{3.912139in}{2.353801in}}%
\pgfpathlineto{\pgfqpoint{3.914463in}{2.348712in}}%
\pgfpathlineto{\pgfqpoint{3.916787in}{2.387907in}}%
\pgfpathlineto{\pgfqpoint{3.919111in}{2.332454in}}%
\pgfpathlineto{\pgfqpoint{3.921435in}{2.388824in}}%
\pgfpathlineto{\pgfqpoint{3.923759in}{2.324073in}}%
\pgfpathlineto{\pgfqpoint{3.926083in}{2.298312in}}%
\pgfpathlineto{\pgfqpoint{3.930731in}{2.385785in}}%
\pgfpathlineto{\pgfqpoint{3.933055in}{2.396892in}}%
\pgfpathlineto{\pgfqpoint{3.935379in}{2.370924in}}%
\pgfpathlineto{\pgfqpoint{3.937703in}{2.389413in}}%
\pgfpathlineto{\pgfqpoint{3.940027in}{2.378391in}}%
\pgfpathlineto{\pgfqpoint{3.942351in}{2.294762in}}%
\pgfpathlineto{\pgfqpoint{3.946999in}{2.368672in}}%
\pgfpathlineto{\pgfqpoint{3.949323in}{2.360050in}}%
\pgfpathlineto{\pgfqpoint{3.951647in}{2.359934in}}%
\pgfpathlineto{\pgfqpoint{3.953971in}{2.361653in}}%
\pgfpathlineto{\pgfqpoint{3.956295in}{2.333649in}}%
\pgfpathlineto{\pgfqpoint{3.958619in}{2.382113in}}%
\pgfpathlineto{\pgfqpoint{3.960943in}{2.382149in}}%
\pgfpathlineto{\pgfqpoint{3.963267in}{2.342056in}}%
\pgfpathlineto{\pgfqpoint{3.965591in}{2.353774in}}%
\pgfpathlineto{\pgfqpoint{3.967915in}{2.418029in}}%
\pgfpathlineto{\pgfqpoint{3.970238in}{2.385303in}}%
\pgfpathlineto{\pgfqpoint{3.972562in}{2.331195in}}%
\pgfpathlineto{\pgfqpoint{3.974886in}{2.365354in}}%
\pgfpathlineto{\pgfqpoint{3.979534in}{2.342834in}}%
\pgfpathlineto{\pgfqpoint{3.981858in}{2.387165in}}%
\pgfpathlineto{\pgfqpoint{3.984182in}{2.355028in}}%
\pgfpathlineto{\pgfqpoint{3.986506in}{2.340549in}}%
\pgfpathlineto{\pgfqpoint{3.988830in}{2.427095in}}%
\pgfpathlineto{\pgfqpoint{3.991154in}{2.405658in}}%
\pgfpathlineto{\pgfqpoint{3.993478in}{2.427803in}}%
\pgfpathlineto{\pgfqpoint{3.998126in}{2.371757in}}%
\pgfpathlineto{\pgfqpoint{4.000450in}{2.401430in}}%
\pgfpathlineto{\pgfqpoint{4.002774in}{2.402473in}}%
\pgfpathlineto{\pgfqpoint{4.005098in}{2.358669in}}%
\pgfpathlineto{\pgfqpoint{4.007422in}{2.412797in}}%
\pgfpathlineto{\pgfqpoint{4.009746in}{2.365342in}}%
\pgfpathlineto{\pgfqpoint{4.012070in}{2.367157in}}%
\pgfpathlineto{\pgfqpoint{4.014394in}{2.372729in}}%
\pgfpathlineto{\pgfqpoint{4.016718in}{2.449184in}}%
\pgfpathlineto{\pgfqpoint{4.019042in}{2.467987in}}%
\pgfpathlineto{\pgfqpoint{4.021366in}{2.430689in}}%
\pgfpathlineto{\pgfqpoint{4.023690in}{2.440267in}}%
\pgfpathlineto{\pgfqpoint{4.026014in}{2.394672in}}%
\pgfpathlineto{\pgfqpoint{4.028338in}{2.462821in}}%
\pgfpathlineto{\pgfqpoint{4.030662in}{2.407915in}}%
\pgfpathlineto{\pgfqpoint{4.035310in}{2.421519in}}%
\pgfpathlineto{\pgfqpoint{4.037634in}{2.442241in}}%
\pgfpathlineto{\pgfqpoint{4.039957in}{2.430023in}}%
\pgfpathlineto{\pgfqpoint{4.042281in}{2.457599in}}%
\pgfpathlineto{\pgfqpoint{4.044605in}{2.436463in}}%
\pgfpathlineto{\pgfqpoint{4.046929in}{2.382268in}}%
\pgfpathlineto{\pgfqpoint{4.049253in}{2.426973in}}%
\pgfpathlineto{\pgfqpoint{4.051577in}{2.445315in}}%
\pgfpathlineto{\pgfqpoint{4.053901in}{2.415381in}}%
\pgfpathlineto{\pgfqpoint{4.056225in}{2.409444in}}%
\pgfpathlineto{\pgfqpoint{4.058549in}{2.452930in}}%
\pgfpathlineto{\pgfqpoint{4.060873in}{2.431403in}}%
\pgfpathlineto{\pgfqpoint{4.063197in}{2.442609in}}%
\pgfpathlineto{\pgfqpoint{4.065521in}{2.476443in}}%
\pgfpathlineto{\pgfqpoint{4.067845in}{2.429172in}}%
\pgfpathlineto{\pgfqpoint{4.070169in}{2.463410in}}%
\pgfpathlineto{\pgfqpoint{4.072493in}{2.457292in}}%
\pgfpathlineto{\pgfqpoint{4.074817in}{2.439737in}}%
\pgfpathlineto{\pgfqpoint{4.077141in}{2.479740in}}%
\pgfpathlineto{\pgfqpoint{4.079465in}{2.424705in}}%
\pgfpathlineto{\pgfqpoint{4.081789in}{2.450092in}}%
\pgfpathlineto{\pgfqpoint{4.084113in}{2.441571in}}%
\pgfpathlineto{\pgfqpoint{4.086437in}{2.464401in}}%
\pgfpathlineto{\pgfqpoint{4.088761in}{2.472375in}}%
\pgfpathlineto{\pgfqpoint{4.093409in}{2.474431in}}%
\pgfpathlineto{\pgfqpoint{4.095733in}{2.422555in}}%
\pgfpathlineto{\pgfqpoint{4.098057in}{2.479956in}}%
\pgfpathlineto{\pgfqpoint{4.100381in}{2.436530in}}%
\pgfpathlineto{\pgfqpoint{4.102705in}{2.456994in}}%
\pgfpathlineto{\pgfqpoint{4.105029in}{2.428504in}}%
\pgfpathlineto{\pgfqpoint{4.107353in}{2.459948in}}%
\pgfpathlineto{\pgfqpoint{4.109677in}{2.452462in}}%
\pgfpathlineto{\pgfqpoint{4.114324in}{2.553165in}}%
\pgfpathlineto{\pgfqpoint{4.116648in}{2.445850in}}%
\pgfpathlineto{\pgfqpoint{4.118972in}{2.453379in}}%
\pgfpathlineto{\pgfqpoint{4.121296in}{2.464811in}}%
\pgfpathlineto{\pgfqpoint{4.123620in}{2.534543in}}%
\pgfpathlineto{\pgfqpoint{4.125944in}{2.527963in}}%
\pgfpathlineto{\pgfqpoint{4.128268in}{2.490887in}}%
\pgfpathlineto{\pgfqpoint{4.130592in}{2.413355in}}%
\pgfpathlineto{\pgfqpoint{4.135240in}{2.510345in}}%
\pgfpathlineto{\pgfqpoint{4.137564in}{2.508509in}}%
\pgfpathlineto{\pgfqpoint{4.139888in}{2.500810in}}%
\pgfpathlineto{\pgfqpoint{4.144536in}{2.524201in}}%
\pgfpathlineto{\pgfqpoint{4.146860in}{2.460903in}}%
\pgfpathlineto{\pgfqpoint{4.149184in}{2.534053in}}%
\pgfpathlineto{\pgfqpoint{4.151508in}{2.492883in}}%
\pgfpathlineto{\pgfqpoint{4.153832in}{2.540481in}}%
\pgfpathlineto{\pgfqpoint{4.156156in}{2.525132in}}%
\pgfpathlineto{\pgfqpoint{4.158480in}{2.481086in}}%
\pgfpathlineto{\pgfqpoint{4.160804in}{2.570826in}}%
\pgfpathlineto{\pgfqpoint{4.163128in}{2.540069in}}%
\pgfpathlineto{\pgfqpoint{4.167776in}{2.507507in}}%
\pgfpathlineto{\pgfqpoint{4.170100in}{2.543785in}}%
\pgfpathlineto{\pgfqpoint{4.172424in}{2.516233in}}%
\pgfpathlineto{\pgfqpoint{4.174748in}{2.540982in}}%
\pgfpathlineto{\pgfqpoint{4.177072in}{2.553293in}}%
\pgfpathlineto{\pgfqpoint{4.179396in}{2.501559in}}%
\pgfpathlineto{\pgfqpoint{4.181719in}{2.524233in}}%
\pgfpathlineto{\pgfqpoint{4.184043in}{2.592113in}}%
\pgfpathlineto{\pgfqpoint{4.186367in}{2.517692in}}%
\pgfpathlineto{\pgfqpoint{4.188691in}{2.535859in}}%
\pgfpathlineto{\pgfqpoint{4.191015in}{2.582958in}}%
\pgfpathlineto{\pgfqpoint{4.193339in}{2.527628in}}%
\pgfpathlineto{\pgfqpoint{4.195663in}{2.572420in}}%
\pgfpathlineto{\pgfqpoint{4.197987in}{2.548147in}}%
\pgfpathlineto{\pgfqpoint{4.200311in}{2.588625in}}%
\pgfpathlineto{\pgfqpoint{4.202635in}{2.546845in}}%
\pgfpathlineto{\pgfqpoint{4.204959in}{2.563591in}}%
\pgfpathlineto{\pgfqpoint{4.207283in}{2.499745in}}%
\pgfpathlineto{\pgfqpoint{4.209607in}{2.575623in}}%
\pgfpathlineto{\pgfqpoint{4.211931in}{2.563288in}}%
\pgfpathlineto{\pgfqpoint{4.214255in}{2.561744in}}%
\pgfpathlineto{\pgfqpoint{4.216579in}{2.604967in}}%
\pgfpathlineto{\pgfqpoint{4.218903in}{2.595638in}}%
\pgfpathlineto{\pgfqpoint{4.221227in}{2.575373in}}%
\pgfpathlineto{\pgfqpoint{4.223551in}{2.544996in}}%
\pgfpathlineto{\pgfqpoint{4.225875in}{2.613528in}}%
\pgfpathlineto{\pgfqpoint{4.228199in}{2.629005in}}%
\pgfpathlineto{\pgfqpoint{4.230523in}{2.586724in}}%
\pgfpathlineto{\pgfqpoint{4.232847in}{2.574560in}}%
\pgfpathlineto{\pgfqpoint{4.235171in}{2.575451in}}%
\pgfpathlineto{\pgfqpoint{4.237495in}{2.572080in}}%
\pgfpathlineto{\pgfqpoint{4.239819in}{2.653445in}}%
\pgfpathlineto{\pgfqpoint{4.242143in}{2.582754in}}%
\pgfpathlineto{\pgfqpoint{4.244467in}{2.555965in}}%
\pgfpathlineto{\pgfqpoint{4.249115in}{2.635477in}}%
\pgfpathlineto{\pgfqpoint{4.251438in}{2.619257in}}%
\pgfpathlineto{\pgfqpoint{4.253762in}{2.594541in}}%
\pgfpathlineto{\pgfqpoint{4.256086in}{2.613890in}}%
\pgfpathlineto{\pgfqpoint{4.258410in}{2.617147in}}%
\pgfpathlineto{\pgfqpoint{4.260734in}{2.574481in}}%
\pgfpathlineto{\pgfqpoint{4.263058in}{2.633426in}}%
\pgfpathlineto{\pgfqpoint{4.265382in}{2.638492in}}%
\pgfpathlineto{\pgfqpoint{4.267706in}{2.646659in}}%
\pgfpathlineto{\pgfqpoint{4.270030in}{2.634623in}}%
\pgfpathlineto{\pgfqpoint{4.272354in}{2.613094in}}%
\pgfpathlineto{\pgfqpoint{4.274678in}{2.631590in}}%
\pgfpathlineto{\pgfqpoint{4.277002in}{2.633279in}}%
\pgfpathlineto{\pgfqpoint{4.279326in}{2.645018in}}%
\pgfpathlineto{\pgfqpoint{4.281650in}{2.684734in}}%
\pgfpathlineto{\pgfqpoint{4.286298in}{2.610511in}}%
\pgfpathlineto{\pgfqpoint{4.288622in}{2.698510in}}%
\pgfpathlineto{\pgfqpoint{4.290946in}{2.662112in}}%
\pgfpathlineto{\pgfqpoint{4.293270in}{2.724419in}}%
\pgfpathlineto{\pgfqpoint{4.295594in}{2.649203in}}%
\pgfpathlineto{\pgfqpoint{4.297918in}{2.650853in}}%
\pgfpathlineto{\pgfqpoint{4.300242in}{2.680951in}}%
\pgfpathlineto{\pgfqpoint{4.302566in}{2.661439in}}%
\pgfpathlineto{\pgfqpoint{4.304890in}{2.696514in}}%
\pgfpathlineto{\pgfqpoint{4.307214in}{2.713731in}}%
\pgfpathlineto{\pgfqpoint{4.309538in}{2.657153in}}%
\pgfpathlineto{\pgfqpoint{4.311862in}{2.673988in}}%
\pgfpathlineto{\pgfqpoint{4.314186in}{2.670695in}}%
\pgfpathlineto{\pgfqpoint{4.316510in}{2.673736in}}%
\pgfpathlineto{\pgfqpoint{4.318834in}{2.671851in}}%
\pgfpathlineto{\pgfqpoint{4.323481in}{2.741276in}}%
\pgfpathlineto{\pgfqpoint{4.325805in}{2.684205in}}%
\pgfpathlineto{\pgfqpoint{4.328129in}{2.720148in}}%
\pgfpathlineto{\pgfqpoint{4.330453in}{2.707872in}}%
\pgfpathlineto{\pgfqpoint{4.335101in}{2.700451in}}%
\pgfpathlineto{\pgfqpoint{4.337425in}{2.690743in}}%
\pgfpathlineto{\pgfqpoint{4.339749in}{2.637836in}}%
\pgfpathlineto{\pgfqpoint{4.344397in}{2.732775in}}%
\pgfpathlineto{\pgfqpoint{4.346721in}{2.707965in}}%
\pgfpathlineto{\pgfqpoint{4.349045in}{2.753566in}}%
\pgfpathlineto{\pgfqpoint{4.356017in}{2.709085in}}%
\pgfpathlineto{\pgfqpoint{4.358341in}{2.737208in}}%
\pgfpathlineto{\pgfqpoint{4.360665in}{2.695554in}}%
\pgfpathlineto{\pgfqpoint{4.362989in}{2.757525in}}%
\pgfpathlineto{\pgfqpoint{4.365313in}{2.789240in}}%
\pgfpathlineto{\pgfqpoint{4.369961in}{2.725266in}}%
\pgfpathlineto{\pgfqpoint{4.372285in}{2.754293in}}%
\pgfpathlineto{\pgfqpoint{4.374609in}{2.738817in}}%
\pgfpathlineto{\pgfqpoint{4.376933in}{2.759130in}}%
\pgfpathlineto{\pgfqpoint{4.379257in}{2.765923in}}%
\pgfpathlineto{\pgfqpoint{4.381581in}{2.726136in}}%
\pgfpathlineto{\pgfqpoint{4.383905in}{2.806164in}}%
\pgfpathlineto{\pgfqpoint{4.386229in}{2.701895in}}%
\pgfpathlineto{\pgfqpoint{4.388553in}{2.774081in}}%
\pgfpathlineto{\pgfqpoint{4.390877in}{2.776557in}}%
\pgfpathlineto{\pgfqpoint{4.393200in}{2.742293in}}%
\pgfpathlineto{\pgfqpoint{4.395524in}{2.795615in}}%
\pgfpathlineto{\pgfqpoint{4.397848in}{2.818465in}}%
\pgfpathlineto{\pgfqpoint{4.400172in}{2.761878in}}%
\pgfpathlineto{\pgfqpoint{4.402496in}{2.799912in}}%
\pgfpathlineto{\pgfqpoint{4.404820in}{2.740857in}}%
\pgfpathlineto{\pgfqpoint{4.407144in}{2.728478in}}%
\pgfpathlineto{\pgfqpoint{4.409468in}{2.819445in}}%
\pgfpathlineto{\pgfqpoint{4.414116in}{2.783899in}}%
\pgfpathlineto{\pgfqpoint{4.416440in}{2.744615in}}%
\pgfpathlineto{\pgfqpoint{4.418764in}{2.792534in}}%
\pgfpathlineto{\pgfqpoint{4.421088in}{2.814600in}}%
\pgfpathlineto{\pgfqpoint{4.423412in}{2.778109in}}%
\pgfpathlineto{\pgfqpoint{4.425736in}{2.827824in}}%
\pgfpathlineto{\pgfqpoint{4.428060in}{2.311308in}}%
\pgfpathlineto{\pgfqpoint{4.430384in}{2.323491in}}%
\pgfpathlineto{\pgfqpoint{4.432708in}{2.267730in}}%
\pgfpathlineto{\pgfqpoint{4.435032in}{2.314726in}}%
\pgfpathlineto{\pgfqpoint{4.437356in}{2.327711in}}%
\pgfpathlineto{\pgfqpoint{4.439680in}{2.303768in}}%
\pgfpathlineto{\pgfqpoint{4.444328in}{2.350991in}}%
\pgfpathlineto{\pgfqpoint{4.446652in}{2.330488in}}%
\pgfpathlineto{\pgfqpoint{4.448976in}{2.335573in}}%
\pgfpathlineto{\pgfqpoint{4.451300in}{2.357736in}}%
\pgfpathlineto{\pgfqpoint{4.453624in}{2.352261in}}%
\pgfpathlineto{\pgfqpoint{4.455948in}{2.342292in}}%
\pgfpathlineto{\pgfqpoint{4.458272in}{2.393075in}}%
\pgfpathlineto{\pgfqpoint{4.460596in}{2.345973in}}%
\pgfpathlineto{\pgfqpoint{4.462919in}{2.372848in}}%
\pgfpathlineto{\pgfqpoint{4.465243in}{2.364362in}}%
\pgfpathlineto{\pgfqpoint{4.467567in}{2.315681in}}%
\pgfpathlineto{\pgfqpoint{4.469891in}{2.381173in}}%
\pgfpathlineto{\pgfqpoint{4.472215in}{2.331441in}}%
\pgfpathlineto{\pgfqpoint{4.476863in}{2.372593in}}%
\pgfpathlineto{\pgfqpoint{4.479187in}{2.396098in}}%
\pgfpathlineto{\pgfqpoint{4.481511in}{2.356018in}}%
\pgfpathlineto{\pgfqpoint{4.483835in}{2.368588in}}%
\pgfpathlineto{\pgfqpoint{4.486159in}{2.390590in}}%
\pgfpathlineto{\pgfqpoint{4.488483in}{2.433159in}}%
\pgfpathlineto{\pgfqpoint{4.490807in}{2.333227in}}%
\pgfpathlineto{\pgfqpoint{4.493131in}{2.380719in}}%
\pgfpathlineto{\pgfqpoint{4.495455in}{2.357415in}}%
\pgfpathlineto{\pgfqpoint{4.497779in}{2.415953in}}%
\pgfpathlineto{\pgfqpoint{4.502427in}{2.346203in}}%
\pgfpathlineto{\pgfqpoint{4.504751in}{2.439489in}}%
\pgfpathlineto{\pgfqpoint{4.509399in}{2.376797in}}%
\pgfpathlineto{\pgfqpoint{4.511723in}{2.429920in}}%
\pgfpathlineto{\pgfqpoint{4.514047in}{2.419123in}}%
\pgfpathlineto{\pgfqpoint{4.516371in}{2.395742in}}%
\pgfpathlineto{\pgfqpoint{4.518695in}{2.406245in}}%
\pgfpathlineto{\pgfqpoint{4.521019in}{2.390676in}}%
\pgfpathlineto{\pgfqpoint{4.523343in}{2.441393in}}%
\pgfpathlineto{\pgfqpoint{4.525667in}{2.419126in}}%
\pgfpathlineto{\pgfqpoint{4.527991in}{2.416866in}}%
\pgfpathlineto{\pgfqpoint{4.530315in}{2.463677in}}%
\pgfpathlineto{\pgfqpoint{4.532638in}{2.464325in}}%
\pgfpathlineto{\pgfqpoint{4.534962in}{2.452641in}}%
\pgfpathlineto{\pgfqpoint{4.537286in}{2.431610in}}%
\pgfpathlineto{\pgfqpoint{4.539610in}{2.436892in}}%
\pgfpathlineto{\pgfqpoint{4.541934in}{2.453756in}}%
\pgfpathlineto{\pgfqpoint{4.544258in}{2.440571in}}%
\pgfpathlineto{\pgfqpoint{4.546582in}{2.434052in}}%
\pgfpathlineto{\pgfqpoint{4.548906in}{2.473411in}}%
\pgfpathlineto{\pgfqpoint{4.551230in}{2.448861in}}%
\pgfpathlineto{\pgfqpoint{4.553554in}{2.486471in}}%
\pgfpathlineto{\pgfqpoint{4.555878in}{2.475380in}}%
\pgfpathlineto{\pgfqpoint{4.558202in}{2.426512in}}%
\pgfpathlineto{\pgfqpoint{4.560526in}{2.462634in}}%
\pgfpathlineto{\pgfqpoint{4.562850in}{2.465196in}}%
\pgfpathlineto{\pgfqpoint{4.565174in}{2.495928in}}%
\pgfpathlineto{\pgfqpoint{4.567498in}{2.476974in}}%
\pgfpathlineto{\pgfqpoint{4.569822in}{2.441042in}}%
\pgfpathlineto{\pgfqpoint{4.572146in}{2.456151in}}%
\pgfpathlineto{\pgfqpoint{4.574470in}{2.501666in}}%
\pgfpathlineto{\pgfqpoint{4.576794in}{2.448403in}}%
\pgfpathlineto{\pgfqpoint{4.579118in}{2.555300in}}%
\pgfpathlineto{\pgfqpoint{4.581442in}{2.530888in}}%
\pgfpathlineto{\pgfqpoint{4.583766in}{2.493732in}}%
\pgfpathlineto{\pgfqpoint{4.586090in}{2.553535in}}%
\pgfpathlineto{\pgfqpoint{4.590738in}{2.478232in}}%
\pgfpathlineto{\pgfqpoint{4.595386in}{2.530938in}}%
\pgfpathlineto{\pgfqpoint{4.597710in}{2.527662in}}%
\pgfpathlineto{\pgfqpoint{4.600034in}{2.543715in}}%
\pgfpathlineto{\pgfqpoint{4.602358in}{2.533400in}}%
\pgfpathlineto{\pgfqpoint{4.604681in}{2.539064in}}%
\pgfpathlineto{\pgfqpoint{4.607005in}{2.517809in}}%
\pgfpathlineto{\pgfqpoint{4.609329in}{2.508582in}}%
\pgfpathlineto{\pgfqpoint{4.611653in}{2.514848in}}%
\pgfpathlineto{\pgfqpoint{4.613977in}{2.558073in}}%
\pgfpathlineto{\pgfqpoint{4.616301in}{2.513354in}}%
\pgfpathlineto{\pgfqpoint{4.618625in}{2.558774in}}%
\pgfpathlineto{\pgfqpoint{4.620949in}{2.515791in}}%
\pgfpathlineto{\pgfqpoint{4.623273in}{2.595860in}}%
\pgfpathlineto{\pgfqpoint{4.625597in}{2.554259in}}%
\pgfpathlineto{\pgfqpoint{4.627921in}{2.540580in}}%
\pgfpathlineto{\pgfqpoint{4.630245in}{2.571675in}}%
\pgfpathlineto{\pgfqpoint{4.632569in}{2.585093in}}%
\pgfpathlineto{\pgfqpoint{4.634893in}{2.531829in}}%
\pgfpathlineto{\pgfqpoint{4.637217in}{2.546147in}}%
\pgfpathlineto{\pgfqpoint{4.639541in}{2.526649in}}%
\pgfpathlineto{\pgfqpoint{4.644189in}{2.631123in}}%
\pgfpathlineto{\pgfqpoint{4.646513in}{2.574083in}}%
\pgfpathlineto{\pgfqpoint{4.651161in}{2.586234in}}%
\pgfpathlineto{\pgfqpoint{4.653485in}{2.551450in}}%
\pgfpathlineto{\pgfqpoint{4.655809in}{2.598824in}}%
\pgfpathlineto{\pgfqpoint{4.658133in}{2.558597in}}%
\pgfpathlineto{\pgfqpoint{4.660457in}{2.574207in}}%
\pgfpathlineto{\pgfqpoint{4.662781in}{2.626146in}}%
\pgfpathlineto{\pgfqpoint{4.665105in}{2.634691in}}%
\pgfpathlineto{\pgfqpoint{4.667429in}{2.655813in}}%
\pgfpathlineto{\pgfqpoint{4.672077in}{2.611493in}}%
\pgfpathlineto{\pgfqpoint{4.674400in}{2.628276in}}%
\pgfpathlineto{\pgfqpoint{4.676724in}{2.619066in}}%
\pgfpathlineto{\pgfqpoint{4.681372in}{2.609615in}}%
\pgfpathlineto{\pgfqpoint{4.683696in}{2.594125in}}%
\pgfpathlineto{\pgfqpoint{4.686020in}{2.654703in}}%
\pgfpathlineto{\pgfqpoint{4.688344in}{2.614561in}}%
\pgfpathlineto{\pgfqpoint{4.690668in}{2.629527in}}%
\pgfpathlineto{\pgfqpoint{4.692992in}{2.658466in}}%
\pgfpathlineto{\pgfqpoint{4.697640in}{2.592936in}}%
\pgfpathlineto{\pgfqpoint{4.699964in}{2.653336in}}%
\pgfpathlineto{\pgfqpoint{4.702288in}{2.631147in}}%
\pgfpathlineto{\pgfqpoint{4.704612in}{2.654443in}}%
\pgfpathlineto{\pgfqpoint{4.706936in}{2.647080in}}%
\pgfpathlineto{\pgfqpoint{4.709260in}{2.690658in}}%
\pgfpathlineto{\pgfqpoint{4.711584in}{2.652733in}}%
\pgfpathlineto{\pgfqpoint{4.713908in}{2.683416in}}%
\pgfpathlineto{\pgfqpoint{4.716232in}{2.641912in}}%
\pgfpathlineto{\pgfqpoint{4.718556in}{2.694260in}}%
\pgfpathlineto{\pgfqpoint{4.720880in}{2.671469in}}%
\pgfpathlineto{\pgfqpoint{4.723204in}{2.671651in}}%
\pgfpathlineto{\pgfqpoint{4.725528in}{2.661076in}}%
\pgfpathlineto{\pgfqpoint{4.727852in}{2.656633in}}%
\pgfpathlineto{\pgfqpoint{4.730176in}{2.719049in}}%
\pgfpathlineto{\pgfqpoint{4.732500in}{2.716112in}}%
\pgfpathlineto{\pgfqpoint{4.734824in}{2.735423in}}%
\pgfpathlineto{\pgfqpoint{4.737148in}{2.742390in}}%
\pgfpathlineto{\pgfqpoint{4.739472in}{2.707113in}}%
\pgfpathlineto{\pgfqpoint{4.741796in}{2.725362in}}%
\pgfpathlineto{\pgfqpoint{4.744119in}{2.776227in}}%
\pgfpathlineto{\pgfqpoint{4.746443in}{2.705967in}}%
\pgfpathlineto{\pgfqpoint{4.748767in}{2.681763in}}%
\pgfpathlineto{\pgfqpoint{4.751091in}{2.696438in}}%
\pgfpathlineto{\pgfqpoint{4.753415in}{2.734493in}}%
\pgfpathlineto{\pgfqpoint{4.755739in}{2.683962in}}%
\pgfpathlineto{\pgfqpoint{4.758063in}{2.737531in}}%
\pgfpathlineto{\pgfqpoint{4.762711in}{2.769186in}}%
\pgfpathlineto{\pgfqpoint{4.765035in}{2.712850in}}%
\pgfpathlineto{\pgfqpoint{4.767359in}{2.732645in}}%
\pgfpathlineto{\pgfqpoint{4.769683in}{2.726948in}}%
\pgfpathlineto{\pgfqpoint{4.772007in}{2.793679in}}%
\pgfpathlineto{\pgfqpoint{4.774331in}{2.787032in}}%
\pgfpathlineto{\pgfqpoint{4.778979in}{2.753757in}}%
\pgfpathlineto{\pgfqpoint{4.781303in}{2.707532in}}%
\pgfpathlineto{\pgfqpoint{4.783627in}{2.775534in}}%
\pgfpathlineto{\pgfqpoint{4.785951in}{2.766890in}}%
\pgfpathlineto{\pgfqpoint{4.788275in}{2.783389in}}%
\pgfpathlineto{\pgfqpoint{4.790599in}{2.739602in}}%
\pgfpathlineto{\pgfqpoint{4.792923in}{2.782190in}}%
\pgfpathlineto{\pgfqpoint{4.795247in}{2.802071in}}%
\pgfpathlineto{\pgfqpoint{4.797571in}{2.745909in}}%
\pgfpathlineto{\pgfqpoint{4.799895in}{2.735534in}}%
\pgfpathlineto{\pgfqpoint{4.802219in}{2.790800in}}%
\pgfpathlineto{\pgfqpoint{4.804543in}{2.800748in}}%
\pgfpathlineto{\pgfqpoint{4.809191in}{2.757700in}}%
\pgfpathlineto{\pgfqpoint{4.811515in}{2.798544in}}%
\pgfpathlineto{\pgfqpoint{4.813838in}{2.796696in}}%
\pgfpathlineto{\pgfqpoint{4.816162in}{2.738786in}}%
\pgfpathlineto{\pgfqpoint{4.818486in}{2.791125in}}%
\pgfpathlineto{\pgfqpoint{4.820810in}{2.796289in}}%
\pgfpathlineto{\pgfqpoint{4.823134in}{2.792540in}}%
\pgfpathlineto{\pgfqpoint{4.825458in}{2.743429in}}%
\pgfpathlineto{\pgfqpoint{4.830106in}{2.851411in}}%
\pgfpathlineto{\pgfqpoint{4.832430in}{2.719774in}}%
\pgfpathlineto{\pgfqpoint{4.834754in}{2.825683in}}%
\pgfpathlineto{\pgfqpoint{4.837078in}{2.815641in}}%
\pgfpathlineto{\pgfqpoint{4.839402in}{2.847575in}}%
\pgfpathlineto{\pgfqpoint{4.841726in}{2.806077in}}%
\pgfpathlineto{\pgfqpoint{4.844050in}{2.851059in}}%
\pgfpathlineto{\pgfqpoint{4.846374in}{2.829146in}}%
\pgfpathlineto{\pgfqpoint{4.848698in}{2.828274in}}%
\pgfpathlineto{\pgfqpoint{4.851022in}{2.806687in}}%
\pgfpathlineto{\pgfqpoint{4.853346in}{2.860944in}}%
\pgfpathlineto{\pgfqpoint{4.855670in}{2.876671in}}%
\pgfpathlineto{\pgfqpoint{4.857994in}{2.828301in}}%
\pgfpathlineto{\pgfqpoint{4.860318in}{2.870604in}}%
\pgfpathlineto{\pgfqpoint{4.862642in}{2.790026in}}%
\pgfpathlineto{\pgfqpoint{4.864966in}{2.801983in}}%
\pgfpathlineto{\pgfqpoint{4.867290in}{2.800719in}}%
\pgfpathlineto{\pgfqpoint{4.869614in}{2.859848in}}%
\pgfpathlineto{\pgfqpoint{4.871938in}{2.829005in}}%
\pgfpathlineto{\pgfqpoint{4.874262in}{2.881350in}}%
\pgfpathlineto{\pgfqpoint{4.878910in}{2.815776in}}%
\pgfpathlineto{\pgfqpoint{4.883558in}{2.886478in}}%
\pgfpathlineto{\pgfqpoint{4.885881in}{2.882218in}}%
\pgfpathlineto{\pgfqpoint{4.888205in}{2.841209in}}%
\pgfpathlineto{\pgfqpoint{4.890529in}{2.909025in}}%
\pgfpathlineto{\pgfqpoint{4.895177in}{2.833431in}}%
\pgfpathlineto{\pgfqpoint{4.897501in}{2.901370in}}%
\pgfpathlineto{\pgfqpoint{4.899825in}{2.861932in}}%
\pgfpathlineto{\pgfqpoint{4.902149in}{2.894479in}}%
\pgfpathlineto{\pgfqpoint{4.904473in}{2.852149in}}%
\pgfpathlineto{\pgfqpoint{4.906797in}{2.861713in}}%
\pgfpathlineto{\pgfqpoint{4.909121in}{2.891253in}}%
\pgfpathlineto{\pgfqpoint{4.911445in}{2.840806in}}%
\pgfpathlineto{\pgfqpoint{4.913769in}{2.939926in}}%
\pgfpathlineto{\pgfqpoint{4.916093in}{2.940892in}}%
\pgfpathlineto{\pgfqpoint{4.920741in}{2.904089in}}%
\pgfpathlineto{\pgfqpoint{4.923065in}{2.865350in}}%
\pgfpathlineto{\pgfqpoint{4.927713in}{2.925057in}}%
\pgfpathlineto{\pgfqpoint{4.930037in}{2.904221in}}%
\pgfpathlineto{\pgfqpoint{4.932361in}{2.903220in}}%
\pgfpathlineto{\pgfqpoint{4.934685in}{2.883107in}}%
\pgfpathlineto{\pgfqpoint{4.939333in}{2.961572in}}%
\pgfpathlineto{\pgfqpoint{4.941657in}{2.880308in}}%
\pgfpathlineto{\pgfqpoint{4.943981in}{2.899163in}}%
\pgfpathlineto{\pgfqpoint{4.946305in}{2.897104in}}%
\pgfpathlineto{\pgfqpoint{4.948629in}{2.902759in}}%
\pgfpathlineto{\pgfqpoint{4.957924in}{2.952570in}}%
\pgfpathlineto{\pgfqpoint{4.962572in}{2.882631in}}%
\pgfpathlineto{\pgfqpoint{4.964896in}{2.975243in}}%
\pgfpathlineto{\pgfqpoint{4.967220in}{3.002573in}}%
\pgfpathlineto{\pgfqpoint{4.969544in}{2.935929in}}%
\pgfpathlineto{\pgfqpoint{4.971868in}{2.909265in}}%
\pgfpathlineto{\pgfqpoint{4.974192in}{2.951805in}}%
\pgfpathlineto{\pgfqpoint{4.976516in}{2.936476in}}%
\pgfpathlineto{\pgfqpoint{4.978840in}{2.983533in}}%
\pgfpathlineto{\pgfqpoint{4.981164in}{2.962254in}}%
\pgfpathlineto{\pgfqpoint{4.983488in}{2.963197in}}%
\pgfpathlineto{\pgfqpoint{4.985812in}{2.978324in}}%
\pgfpathlineto{\pgfqpoint{4.988136in}{2.959889in}}%
\pgfpathlineto{\pgfqpoint{4.990460in}{2.967524in}}%
\pgfpathlineto{\pgfqpoint{4.992784in}{2.970532in}}%
\pgfpathlineto{\pgfqpoint{4.995108in}{2.993142in}}%
\pgfpathlineto{\pgfqpoint{4.997432in}{2.988871in}}%
\pgfpathlineto{\pgfqpoint{4.999756in}{3.015678in}}%
\pgfpathlineto{\pgfqpoint{5.004404in}{2.990197in}}%
\pgfpathlineto{\pgfqpoint{5.006728in}{2.984199in}}%
\pgfpathlineto{\pgfqpoint{5.009052in}{2.470465in}}%
\pgfpathlineto{\pgfqpoint{5.011376in}{2.509160in}}%
\pgfpathlineto{\pgfqpoint{5.013700in}{2.501011in}}%
\pgfpathlineto{\pgfqpoint{5.016024in}{2.517186in}}%
\pgfpathlineto{\pgfqpoint{5.018348in}{2.466445in}}%
\pgfpathlineto{\pgfqpoint{5.020672in}{2.479482in}}%
\pgfpathlineto{\pgfqpoint{5.022996in}{2.514475in}}%
\pgfpathlineto{\pgfqpoint{5.025319in}{2.466292in}}%
\pgfpathlineto{\pgfqpoint{5.027643in}{2.481553in}}%
\pgfpathlineto{\pgfqpoint{5.029967in}{2.477764in}}%
\pgfpathlineto{\pgfqpoint{5.034615in}{2.526592in}}%
\pgfpathlineto{\pgfqpoint{5.036939in}{2.478963in}}%
\pgfpathlineto{\pgfqpoint{5.039263in}{2.493995in}}%
\pgfpathlineto{\pgfqpoint{5.041587in}{2.463150in}}%
\pgfpathlineto{\pgfqpoint{5.043911in}{2.503541in}}%
\pgfpathlineto{\pgfqpoint{5.046235in}{2.490123in}}%
\pgfpathlineto{\pgfqpoint{5.048559in}{2.529517in}}%
\pgfpathlineto{\pgfqpoint{5.050883in}{2.503905in}}%
\pgfpathlineto{\pgfqpoint{5.053207in}{2.527136in}}%
\pgfpathlineto{\pgfqpoint{5.055531in}{2.520840in}}%
\pgfpathlineto{\pgfqpoint{5.057855in}{2.533001in}}%
\pgfpathlineto{\pgfqpoint{5.060179in}{2.567085in}}%
\pgfpathlineto{\pgfqpoint{5.062503in}{2.517578in}}%
\pgfpathlineto{\pgfqpoint{5.064827in}{2.576507in}}%
\pgfpathlineto{\pgfqpoint{5.067151in}{2.549774in}}%
\pgfpathlineto{\pgfqpoint{5.069475in}{2.547547in}}%
\pgfpathlineto{\pgfqpoint{5.071799in}{2.531929in}}%
\pgfpathlineto{\pgfqpoint{5.074123in}{2.539541in}}%
\pgfpathlineto{\pgfqpoint{5.076447in}{2.516038in}}%
\pgfpathlineto{\pgfqpoint{5.078771in}{2.532458in}}%
\pgfpathlineto{\pgfqpoint{5.081095in}{2.519942in}}%
\pgfpathlineto{\pgfqpoint{5.083419in}{2.544767in}}%
\pgfpathlineto{\pgfqpoint{5.085743in}{2.594028in}}%
\pgfpathlineto{\pgfqpoint{5.088067in}{2.573856in}}%
\pgfpathlineto{\pgfqpoint{5.090391in}{2.578947in}}%
\pgfpathlineto{\pgfqpoint{5.092715in}{2.541593in}}%
\pgfpathlineto{\pgfqpoint{5.095039in}{2.551669in}}%
\pgfpathlineto{\pgfqpoint{5.097362in}{2.582743in}}%
\pgfpathlineto{\pgfqpoint{5.099686in}{2.550244in}}%
\pgfpathlineto{\pgfqpoint{5.102010in}{2.588906in}}%
\pgfpathlineto{\pgfqpoint{5.106658in}{2.533359in}}%
\pgfpathlineto{\pgfqpoint{5.108982in}{2.610451in}}%
\pgfpathlineto{\pgfqpoint{5.111306in}{2.560975in}}%
\pgfpathlineto{\pgfqpoint{5.113630in}{2.618119in}}%
\pgfpathlineto{\pgfqpoint{5.115954in}{2.608977in}}%
\pgfpathlineto{\pgfqpoint{5.118278in}{2.542246in}}%
\pgfpathlineto{\pgfqpoint{5.120602in}{2.575869in}}%
\pgfpathlineto{\pgfqpoint{5.122926in}{2.550337in}}%
\pgfpathlineto{\pgfqpoint{5.125250in}{2.601121in}}%
\pgfpathlineto{\pgfqpoint{5.127574in}{2.586062in}}%
\pgfpathlineto{\pgfqpoint{5.129898in}{2.591910in}}%
\pgfpathlineto{\pgfqpoint{5.132222in}{2.562688in}}%
\pgfpathlineto{\pgfqpoint{5.134546in}{2.614940in}}%
\pgfpathlineto{\pgfqpoint{5.136870in}{2.575941in}}%
\pgfpathlineto{\pgfqpoint{5.141518in}{2.629897in}}%
\pgfpathlineto{\pgfqpoint{5.143842in}{2.577845in}}%
\pgfpathlineto{\pgfqpoint{5.146166in}{2.589266in}}%
\pgfpathlineto{\pgfqpoint{5.148490in}{2.628420in}}%
\pgfpathlineto{\pgfqpoint{5.150814in}{2.618823in}}%
\pgfpathlineto{\pgfqpoint{5.153138in}{2.583568in}}%
\pgfpathlineto{\pgfqpoint{5.155462in}{2.586346in}}%
\pgfpathlineto{\pgfqpoint{5.157786in}{2.622964in}}%
\pgfpathlineto{\pgfqpoint{5.160110in}{2.533319in}}%
\pgfpathlineto{\pgfqpoint{5.162434in}{2.582786in}}%
\pgfpathlineto{\pgfqpoint{5.167081in}{2.643356in}}%
\pgfpathlineto{\pgfqpoint{5.169405in}{2.565858in}}%
\pgfpathlineto{\pgfqpoint{5.171729in}{2.609924in}}%
\pgfpathlineto{\pgfqpoint{5.174053in}{2.591502in}}%
\pgfpathlineto{\pgfqpoint{5.178701in}{2.661690in}}%
\pgfpathlineto{\pgfqpoint{5.181025in}{2.539005in}}%
\pgfpathlineto{\pgfqpoint{5.183349in}{2.636230in}}%
\pgfpathlineto{\pgfqpoint{5.185673in}{2.604385in}}%
\pgfpathlineto{\pgfqpoint{5.187997in}{2.628168in}}%
\pgfpathlineto{\pgfqpoint{5.190321in}{2.588088in}}%
\pgfpathlineto{\pgfqpoint{5.192645in}{2.669129in}}%
\pgfpathlineto{\pgfqpoint{5.194969in}{2.664684in}}%
\pgfpathlineto{\pgfqpoint{5.197293in}{2.669688in}}%
\pgfpathlineto{\pgfqpoint{5.199617in}{2.578834in}}%
\pgfpathlineto{\pgfqpoint{5.204265in}{2.647464in}}%
\pgfpathlineto{\pgfqpoint{5.208913in}{2.631999in}}%
\pgfpathlineto{\pgfqpoint{5.211237in}{2.656548in}}%
\pgfpathlineto{\pgfqpoint{5.213561in}{2.653249in}}%
\pgfpathlineto{\pgfqpoint{5.215885in}{2.628186in}}%
\pgfpathlineto{\pgfqpoint{5.220533in}{2.642216in}}%
\pgfpathlineto{\pgfqpoint{5.222857in}{2.634716in}}%
\pgfpathlineto{\pgfqpoint{5.225181in}{2.606298in}}%
\pgfpathlineto{\pgfqpoint{5.227505in}{2.631497in}}%
\pgfpathlineto{\pgfqpoint{5.229829in}{2.617073in}}%
\pgfpathlineto{\pgfqpoint{5.232153in}{2.675209in}}%
\pgfpathlineto{\pgfqpoint{5.234477in}{2.671672in}}%
\pgfpathlineto{\pgfqpoint{5.236800in}{2.595874in}}%
\pgfpathlineto{\pgfqpoint{5.241448in}{2.708684in}}%
\pgfpathlineto{\pgfqpoint{5.243772in}{2.661812in}}%
\pgfpathlineto{\pgfqpoint{5.246096in}{2.702052in}}%
\pgfpathlineto{\pgfqpoint{5.248420in}{2.628355in}}%
\pgfpathlineto{\pgfqpoint{5.250744in}{2.631397in}}%
\pgfpathlineto{\pgfqpoint{5.253068in}{2.651348in}}%
\pgfpathlineto{\pgfqpoint{5.255392in}{2.694672in}}%
\pgfpathlineto{\pgfqpoint{5.257716in}{2.670870in}}%
\pgfpathlineto{\pgfqpoint{5.260040in}{2.669700in}}%
\pgfpathlineto{\pgfqpoint{5.262364in}{2.656271in}}%
\pgfpathlineto{\pgfqpoint{5.264688in}{2.707069in}}%
\pgfpathlineto{\pgfqpoint{5.267012in}{2.672886in}}%
\pgfpathlineto{\pgfqpoint{5.271660in}{2.682432in}}%
\pgfpathlineto{\pgfqpoint{5.273984in}{2.647172in}}%
\pgfpathlineto{\pgfqpoint{5.276308in}{2.692364in}}%
\pgfpathlineto{\pgfqpoint{5.278632in}{2.653959in}}%
\pgfpathlineto{\pgfqpoint{5.280956in}{2.714939in}}%
\pgfpathlineto{\pgfqpoint{5.285604in}{2.665198in}}%
\pgfpathlineto{\pgfqpoint{5.287928in}{2.681822in}}%
\pgfpathlineto{\pgfqpoint{5.290252in}{2.681520in}}%
\pgfpathlineto{\pgfqpoint{5.292576in}{2.699325in}}%
\pgfpathlineto{\pgfqpoint{5.294900in}{2.677220in}}%
\pgfpathlineto{\pgfqpoint{5.297224in}{2.699714in}}%
\pgfpathlineto{\pgfqpoint{5.299548in}{2.671583in}}%
\pgfpathlineto{\pgfqpoint{5.301872in}{2.689548in}}%
\pgfpathlineto{\pgfqpoint{5.304196in}{2.717321in}}%
\pgfpathlineto{\pgfqpoint{5.306519in}{2.704958in}}%
\pgfpathlineto{\pgfqpoint{5.308843in}{2.667779in}}%
\pgfpathlineto{\pgfqpoint{5.315815in}{2.724505in}}%
\pgfpathlineto{\pgfqpoint{5.318139in}{2.720643in}}%
\pgfpathlineto{\pgfqpoint{5.320463in}{2.728174in}}%
\pgfpathlineto{\pgfqpoint{5.322787in}{2.779818in}}%
\pgfpathlineto{\pgfqpoint{5.325111in}{2.670650in}}%
\pgfpathlineto{\pgfqpoint{5.327435in}{2.694676in}}%
\pgfpathlineto{\pgfqpoint{5.329759in}{2.774699in}}%
\pgfpathlineto{\pgfqpoint{5.332083in}{2.700101in}}%
\pgfpathlineto{\pgfqpoint{5.334407in}{2.675978in}}%
\pgfpathlineto{\pgfqpoint{5.336731in}{2.736845in}}%
\pgfpathlineto{\pgfqpoint{5.339055in}{2.735739in}}%
\pgfpathlineto{\pgfqpoint{5.341379in}{2.738246in}}%
\pgfpathlineto{\pgfqpoint{5.343703in}{2.700916in}}%
\pgfpathlineto{\pgfqpoint{5.346027in}{2.702115in}}%
\pgfpathlineto{\pgfqpoint{5.348351in}{2.737619in}}%
\pgfpathlineto{\pgfqpoint{5.350675in}{2.691269in}}%
\pgfpathlineto{\pgfqpoint{5.352999in}{2.738660in}}%
\pgfpathlineto{\pgfqpoint{5.355323in}{2.728151in}}%
\pgfpathlineto{\pgfqpoint{5.357647in}{2.698738in}}%
\pgfpathlineto{\pgfqpoint{5.362295in}{2.719944in}}%
\pgfpathlineto{\pgfqpoint{5.364619in}{2.723249in}}%
\pgfpathlineto{\pgfqpoint{5.366943in}{2.778236in}}%
\pgfpathlineto{\pgfqpoint{5.369267in}{2.725208in}}%
\pgfpathlineto{\pgfqpoint{5.371591in}{2.725105in}}%
\pgfpathlineto{\pgfqpoint{5.373915in}{2.750198in}}%
\pgfpathlineto{\pgfqpoint{5.376239in}{2.738414in}}%
\pgfpathlineto{\pgfqpoint{5.378562in}{2.748531in}}%
\pgfpathlineto{\pgfqpoint{5.380886in}{2.748522in}}%
\pgfpathlineto{\pgfqpoint{5.383210in}{2.701113in}}%
\pgfpathlineto{\pgfqpoint{5.385534in}{2.753601in}}%
\pgfpathlineto{\pgfqpoint{5.387858in}{2.722981in}}%
\pgfpathlineto{\pgfqpoint{5.390182in}{2.772040in}}%
\pgfpathlineto{\pgfqpoint{5.392506in}{2.738979in}}%
\pgfpathlineto{\pgfqpoint{5.394830in}{2.776333in}}%
\pgfpathlineto{\pgfqpoint{5.397154in}{2.768218in}}%
\pgfpathlineto{\pgfqpoint{5.399478in}{2.771905in}}%
\pgfpathlineto{\pgfqpoint{5.401802in}{2.741979in}}%
\pgfpathlineto{\pgfqpoint{5.404126in}{2.755251in}}%
\pgfpathlineto{\pgfqpoint{5.406450in}{2.732921in}}%
\pgfpathlineto{\pgfqpoint{5.408774in}{2.790098in}}%
\pgfpathlineto{\pgfqpoint{5.411098in}{2.782564in}}%
\pgfpathlineto{\pgfqpoint{5.413422in}{2.744655in}}%
\pgfpathlineto{\pgfqpoint{5.415746in}{2.687797in}}%
\pgfpathlineto{\pgfqpoint{5.418070in}{2.767309in}}%
\pgfpathlineto{\pgfqpoint{5.420394in}{2.702828in}}%
\pgfpathlineto{\pgfqpoint{5.422718in}{2.738419in}}%
\pgfpathlineto{\pgfqpoint{5.425042in}{2.749311in}}%
\pgfpathlineto{\pgfqpoint{5.429690in}{2.809866in}}%
\pgfpathlineto{\pgfqpoint{5.432014in}{2.738974in}}%
\pgfpathlineto{\pgfqpoint{5.434338in}{2.787270in}}%
\pgfpathlineto{\pgfqpoint{5.436662in}{2.736426in}}%
\pgfpathlineto{\pgfqpoint{5.438986in}{2.802149in}}%
\pgfpathlineto{\pgfqpoint{5.443634in}{2.738260in}}%
\pgfpathlineto{\pgfqpoint{5.445958in}{2.757221in}}%
\pgfpathlineto{\pgfqpoint{5.448281in}{2.754688in}}%
\pgfpathlineto{\pgfqpoint{5.450605in}{2.818683in}}%
\pgfpathlineto{\pgfqpoint{5.452929in}{2.745017in}}%
\pgfpathlineto{\pgfqpoint{5.455253in}{2.761840in}}%
\pgfpathlineto{\pgfqpoint{5.457577in}{2.747635in}}%
\pgfpathlineto{\pgfqpoint{5.459901in}{2.805834in}}%
\pgfpathlineto{\pgfqpoint{5.462225in}{2.781883in}}%
\pgfpathlineto{\pgfqpoint{5.464549in}{2.788576in}}%
\pgfpathlineto{\pgfqpoint{5.466873in}{2.735305in}}%
\pgfpathlineto{\pgfqpoint{5.469197in}{2.810320in}}%
\pgfpathlineto{\pgfqpoint{5.471521in}{2.755600in}}%
\pgfpathlineto{\pgfqpoint{5.473845in}{2.828979in}}%
\pgfpathlineto{\pgfqpoint{5.476169in}{2.785245in}}%
\pgfpathlineto{\pgfqpoint{5.480817in}{2.752590in}}%
\pgfpathlineto{\pgfqpoint{5.485465in}{2.816213in}}%
\pgfpathlineto{\pgfqpoint{5.487789in}{2.793600in}}%
\pgfpathlineto{\pgfqpoint{5.490113in}{2.810890in}}%
\pgfpathlineto{\pgfqpoint{5.492437in}{2.765012in}}%
\pgfpathlineto{\pgfqpoint{5.494761in}{2.769017in}}%
\pgfpathlineto{\pgfqpoint{5.497085in}{2.736709in}}%
\pgfpathlineto{\pgfqpoint{5.499409in}{2.815466in}}%
\pgfpathlineto{\pgfqpoint{5.501733in}{2.788602in}}%
\pgfpathlineto{\pgfqpoint{5.504057in}{2.782719in}}%
\pgfpathlineto{\pgfqpoint{5.506381in}{2.781174in}}%
\pgfpathlineto{\pgfqpoint{5.508705in}{2.812691in}}%
\pgfpathlineto{\pgfqpoint{5.513353in}{2.801903in}}%
\pgfpathlineto{\pgfqpoint{5.515677in}{2.797613in}}%
\pgfpathlineto{\pgfqpoint{5.518000in}{2.785575in}}%
\pgfpathlineto{\pgfqpoint{5.520324in}{2.809759in}}%
\pgfpathlineto{\pgfqpoint{5.522648in}{2.820458in}}%
\pgfpathlineto{\pgfqpoint{5.524972in}{2.759462in}}%
\pgfpathlineto{\pgfqpoint{5.527296in}{2.809858in}}%
\pgfpathlineto{\pgfqpoint{5.529620in}{2.799955in}}%
\pgfpathlineto{\pgfqpoint{5.531944in}{2.757367in}}%
\pgfpathlineto{\pgfqpoint{5.534268in}{2.821544in}}%
\pgfpathlineto{\pgfqpoint{5.536592in}{2.780895in}}%
\pgfpathlineto{\pgfqpoint{5.538916in}{2.805511in}}%
\pgfpathlineto{\pgfqpoint{5.541240in}{2.794015in}}%
\pgfpathlineto{\pgfqpoint{5.545888in}{2.835208in}}%
\pgfpathlineto{\pgfqpoint{5.548212in}{2.828243in}}%
\pgfpathlineto{\pgfqpoint{5.550536in}{2.792301in}}%
\pgfpathlineto{\pgfqpoint{5.552860in}{2.847561in}}%
\pgfpathlineto{\pgfqpoint{5.555184in}{2.826150in}}%
\pgfpathlineto{\pgfqpoint{5.557508in}{2.910313in}}%
\pgfpathlineto{\pgfqpoint{5.559832in}{2.786771in}}%
\pgfpathlineto{\pgfqpoint{5.562156in}{2.811321in}}%
\pgfpathlineto{\pgfqpoint{5.564480in}{2.819599in}}%
\pgfpathlineto{\pgfqpoint{5.566804in}{2.808316in}}%
\pgfpathlineto{\pgfqpoint{5.569128in}{2.828980in}}%
\pgfpathlineto{\pgfqpoint{5.571452in}{2.803243in}}%
\pgfpathlineto{\pgfqpoint{5.573776in}{2.796825in}}%
\pgfpathlineto{\pgfqpoint{5.576100in}{2.744342in}}%
\pgfpathlineto{\pgfqpoint{5.578424in}{2.832265in}}%
\pgfpathlineto{\pgfqpoint{5.580748in}{2.867240in}}%
\pgfpathlineto{\pgfqpoint{5.583072in}{2.868782in}}%
\pgfpathlineto{\pgfqpoint{5.585396in}{2.797651in}}%
\pgfpathlineto{\pgfqpoint{5.587720in}{2.335317in}}%
\pgfpathlineto{\pgfqpoint{5.587720in}{2.335317in}}%
\pgfusepath{stroke}%
\end{pgfscope}%
\begin{pgfscope}%
\pgfsetrectcap%
\pgfsetmiterjoin%
\pgfsetlinewidth{1.254687pt}%
\definecolor{currentstroke}{rgb}{0.800000,0.800000,0.800000}%
\pgfsetstrokecolor{currentstroke}%
\pgfsetdash{}{0pt}%
\pgfpathmoveto{\pgfqpoint{0.709829in}{2.192315in}}%
\pgfpathlineto{\pgfqpoint{0.709829in}{3.079852in}}%
\pgfusepath{stroke}%
\end{pgfscope}%
\begin{pgfscope}%
\pgfsetrectcap%
\pgfsetmiterjoin%
\pgfsetlinewidth{1.254687pt}%
\definecolor{currentstroke}{rgb}{0.800000,0.800000,0.800000}%
\pgfsetstrokecolor{currentstroke}%
\pgfsetdash{}{0pt}%
\pgfpathmoveto{\pgfqpoint{5.820000in}{2.192315in}}%
\pgfpathlineto{\pgfqpoint{5.820000in}{3.079852in}}%
\pgfusepath{stroke}%
\end{pgfscope}%
\begin{pgfscope}%
\pgfsetrectcap%
\pgfsetmiterjoin%
\pgfsetlinewidth{1.254687pt}%
\definecolor{currentstroke}{rgb}{0.800000,0.800000,0.800000}%
\pgfsetstrokecolor{currentstroke}%
\pgfsetdash{}{0pt}%
\pgfpathmoveto{\pgfqpoint{0.709829in}{2.192315in}}%
\pgfpathlineto{\pgfqpoint{5.820000in}{2.192315in}}%
\pgfusepath{stroke}%
\end{pgfscope}%
\begin{pgfscope}%
\pgfsetrectcap%
\pgfsetmiterjoin%
\pgfsetlinewidth{1.254687pt}%
\definecolor{currentstroke}{rgb}{0.800000,0.800000,0.800000}%
\pgfsetstrokecolor{currentstroke}%
\pgfsetdash{}{0pt}%
\pgfpathmoveto{\pgfqpoint{0.709829in}{3.079852in}}%
\pgfpathlineto{\pgfqpoint{5.820000in}{3.079852in}}%
\pgfusepath{stroke}%
\end{pgfscope}%
\begin{pgfscope}%
\definecolor{textcolor}{rgb}{0.150000,0.150000,0.150000}%
\pgfsetstrokecolor{textcolor}%
\pgfsetfillcolor{textcolor}%
\pgftext[x=3.264915in,y=3.163185in,,base]{\color{textcolor}{\sffamily\fontsize{12.000000}{14.400000}\selectfont\catcode`\^=\active\def^{\ifmmode\sp\else\^{}\fi}\catcode`\%=\active\def%{\%}Mixed Signals}}%
\end{pgfscope}%
\begin{pgfscope}%
\pgfsetbuttcap%
\pgfsetmiterjoin%
\definecolor{currentfill}{rgb}{1.000000,1.000000,1.000000}%
\pgfsetfillcolor{currentfill}%
\pgfsetlinewidth{0.000000pt}%
\definecolor{currentstroke}{rgb}{0.000000,0.000000,0.000000}%
\pgfsetstrokecolor{currentstroke}%
\pgfsetstrokeopacity{0.000000}%
\pgfsetdash{}{0pt}%
\pgfpathmoveto{\pgfqpoint{0.709829in}{0.654666in}}%
\pgfpathlineto{\pgfqpoint{5.820000in}{0.654666in}}%
\pgfpathlineto{\pgfqpoint{5.820000in}{1.542204in}}%
\pgfpathlineto{\pgfqpoint{0.709829in}{1.542204in}}%
\pgfpathlineto{\pgfqpoint{0.709829in}{0.654666in}}%
\pgfpathclose%
\pgfusepath{fill}%
\end{pgfscope}%
\begin{pgfscope}%
\pgfpathrectangle{\pgfqpoint{0.709829in}{0.654666in}}{\pgfqpoint{5.110171in}{0.887537in}}%
\pgfusepath{clip}%
\pgfsetroundcap%
\pgfsetroundjoin%
\pgfsetlinewidth{1.003750pt}%
\definecolor{currentstroke}{rgb}{0.800000,0.800000,0.800000}%
\pgfsetstrokecolor{currentstroke}%
\pgfsetdash{}{0pt}%
\pgfpathmoveto{\pgfqpoint{0.942110in}{0.654666in}}%
\pgfpathlineto{\pgfqpoint{0.942110in}{1.542204in}}%
\pgfusepath{stroke}%
\end{pgfscope}%
\begin{pgfscope}%
\definecolor{textcolor}{rgb}{0.150000,0.150000,0.150000}%
\pgfsetstrokecolor{textcolor}%
\pgfsetfillcolor{textcolor}%
\pgftext[x=0.942110in,y=0.522722in,,top]{\color{textcolor}{\sffamily\fontsize{11.000000}{13.200000}\selectfont\catcode`\^=\active\def^{\ifmmode\sp\else\^{}\fi}\catcode`\%=\active\def%{\%}$\mathdefault{0}$}}%
\end{pgfscope}%
\begin{pgfscope}%
\pgfpathrectangle{\pgfqpoint{0.709829in}{0.654666in}}{\pgfqpoint{5.110171in}{0.887537in}}%
\pgfusepath{clip}%
\pgfsetroundcap%
\pgfsetroundjoin%
\pgfsetlinewidth{1.003750pt}%
\definecolor{currentstroke}{rgb}{0.800000,0.800000,0.800000}%
\pgfsetstrokecolor{currentstroke}%
\pgfsetdash{}{0pt}%
\pgfpathmoveto{\pgfqpoint{1.523101in}{0.654666in}}%
\pgfpathlineto{\pgfqpoint{1.523101in}{1.542204in}}%
\pgfusepath{stroke}%
\end{pgfscope}%
\begin{pgfscope}%
\definecolor{textcolor}{rgb}{0.150000,0.150000,0.150000}%
\pgfsetstrokecolor{textcolor}%
\pgfsetfillcolor{textcolor}%
\pgftext[x=1.523101in,y=0.522722in,,top]{\color{textcolor}{\sffamily\fontsize{11.000000}{13.200000}\selectfont\catcode`\^=\active\def^{\ifmmode\sp\else\^{}\fi}\catcode`\%=\active\def%{\%}$\mathdefault{250}$}}%
\end{pgfscope}%
\begin{pgfscope}%
\pgfpathrectangle{\pgfqpoint{0.709829in}{0.654666in}}{\pgfqpoint{5.110171in}{0.887537in}}%
\pgfusepath{clip}%
\pgfsetroundcap%
\pgfsetroundjoin%
\pgfsetlinewidth{1.003750pt}%
\definecolor{currentstroke}{rgb}{0.800000,0.800000,0.800000}%
\pgfsetstrokecolor{currentstroke}%
\pgfsetdash{}{0pt}%
\pgfpathmoveto{\pgfqpoint{2.104093in}{0.654666in}}%
\pgfpathlineto{\pgfqpoint{2.104093in}{1.542204in}}%
\pgfusepath{stroke}%
\end{pgfscope}%
\begin{pgfscope}%
\definecolor{textcolor}{rgb}{0.150000,0.150000,0.150000}%
\pgfsetstrokecolor{textcolor}%
\pgfsetfillcolor{textcolor}%
\pgftext[x=2.104093in,y=0.522722in,,top]{\color{textcolor}{\sffamily\fontsize{11.000000}{13.200000}\selectfont\catcode`\^=\active\def^{\ifmmode\sp\else\^{}\fi}\catcode`\%=\active\def%{\%}$\mathdefault{500}$}}%
\end{pgfscope}%
\begin{pgfscope}%
\pgfpathrectangle{\pgfqpoint{0.709829in}{0.654666in}}{\pgfqpoint{5.110171in}{0.887537in}}%
\pgfusepath{clip}%
\pgfsetroundcap%
\pgfsetroundjoin%
\pgfsetlinewidth{1.003750pt}%
\definecolor{currentstroke}{rgb}{0.800000,0.800000,0.800000}%
\pgfsetstrokecolor{currentstroke}%
\pgfsetdash{}{0pt}%
\pgfpathmoveto{\pgfqpoint{2.685085in}{0.654666in}}%
\pgfpathlineto{\pgfqpoint{2.685085in}{1.542204in}}%
\pgfusepath{stroke}%
\end{pgfscope}%
\begin{pgfscope}%
\definecolor{textcolor}{rgb}{0.150000,0.150000,0.150000}%
\pgfsetstrokecolor{textcolor}%
\pgfsetfillcolor{textcolor}%
\pgftext[x=2.685085in,y=0.522722in,,top]{\color{textcolor}{\sffamily\fontsize{11.000000}{13.200000}\selectfont\catcode`\^=\active\def^{\ifmmode\sp\else\^{}\fi}\catcode`\%=\active\def%{\%}$\mathdefault{750}$}}%
\end{pgfscope}%
\begin{pgfscope}%
\pgfpathrectangle{\pgfqpoint{0.709829in}{0.654666in}}{\pgfqpoint{5.110171in}{0.887537in}}%
\pgfusepath{clip}%
\pgfsetroundcap%
\pgfsetroundjoin%
\pgfsetlinewidth{1.003750pt}%
\definecolor{currentstroke}{rgb}{0.800000,0.800000,0.800000}%
\pgfsetstrokecolor{currentstroke}%
\pgfsetdash{}{0pt}%
\pgfpathmoveto{\pgfqpoint{3.266076in}{0.654666in}}%
\pgfpathlineto{\pgfqpoint{3.266076in}{1.542204in}}%
\pgfusepath{stroke}%
\end{pgfscope}%
\begin{pgfscope}%
\definecolor{textcolor}{rgb}{0.150000,0.150000,0.150000}%
\pgfsetstrokecolor{textcolor}%
\pgfsetfillcolor{textcolor}%
\pgftext[x=3.266076in,y=0.522722in,,top]{\color{textcolor}{\sffamily\fontsize{11.000000}{13.200000}\selectfont\catcode`\^=\active\def^{\ifmmode\sp\else\^{}\fi}\catcode`\%=\active\def%{\%}$\mathdefault{1000}$}}%
\end{pgfscope}%
\begin{pgfscope}%
\pgfpathrectangle{\pgfqpoint{0.709829in}{0.654666in}}{\pgfqpoint{5.110171in}{0.887537in}}%
\pgfusepath{clip}%
\pgfsetroundcap%
\pgfsetroundjoin%
\pgfsetlinewidth{1.003750pt}%
\definecolor{currentstroke}{rgb}{0.800000,0.800000,0.800000}%
\pgfsetstrokecolor{currentstroke}%
\pgfsetdash{}{0pt}%
\pgfpathmoveto{\pgfqpoint{3.847068in}{0.654666in}}%
\pgfpathlineto{\pgfqpoint{3.847068in}{1.542204in}}%
\pgfusepath{stroke}%
\end{pgfscope}%
\begin{pgfscope}%
\definecolor{textcolor}{rgb}{0.150000,0.150000,0.150000}%
\pgfsetstrokecolor{textcolor}%
\pgfsetfillcolor{textcolor}%
\pgftext[x=3.847068in,y=0.522722in,,top]{\color{textcolor}{\sffamily\fontsize{11.000000}{13.200000}\selectfont\catcode`\^=\active\def^{\ifmmode\sp\else\^{}\fi}\catcode`\%=\active\def%{\%}$\mathdefault{1250}$}}%
\end{pgfscope}%
\begin{pgfscope}%
\pgfpathrectangle{\pgfqpoint{0.709829in}{0.654666in}}{\pgfqpoint{5.110171in}{0.887537in}}%
\pgfusepath{clip}%
\pgfsetroundcap%
\pgfsetroundjoin%
\pgfsetlinewidth{1.003750pt}%
\definecolor{currentstroke}{rgb}{0.800000,0.800000,0.800000}%
\pgfsetstrokecolor{currentstroke}%
\pgfsetdash{}{0pt}%
\pgfpathmoveto{\pgfqpoint{4.428060in}{0.654666in}}%
\pgfpathlineto{\pgfqpoint{4.428060in}{1.542204in}}%
\pgfusepath{stroke}%
\end{pgfscope}%
\begin{pgfscope}%
\definecolor{textcolor}{rgb}{0.150000,0.150000,0.150000}%
\pgfsetstrokecolor{textcolor}%
\pgfsetfillcolor{textcolor}%
\pgftext[x=4.428060in,y=0.522722in,,top]{\color{textcolor}{\sffamily\fontsize{11.000000}{13.200000}\selectfont\catcode`\^=\active\def^{\ifmmode\sp\else\^{}\fi}\catcode`\%=\active\def%{\%}$\mathdefault{1500}$}}%
\end{pgfscope}%
\begin{pgfscope}%
\pgfpathrectangle{\pgfqpoint{0.709829in}{0.654666in}}{\pgfqpoint{5.110171in}{0.887537in}}%
\pgfusepath{clip}%
\pgfsetroundcap%
\pgfsetroundjoin%
\pgfsetlinewidth{1.003750pt}%
\definecolor{currentstroke}{rgb}{0.800000,0.800000,0.800000}%
\pgfsetstrokecolor{currentstroke}%
\pgfsetdash{}{0pt}%
\pgfpathmoveto{\pgfqpoint{5.009052in}{0.654666in}}%
\pgfpathlineto{\pgfqpoint{5.009052in}{1.542204in}}%
\pgfusepath{stroke}%
\end{pgfscope}%
\begin{pgfscope}%
\definecolor{textcolor}{rgb}{0.150000,0.150000,0.150000}%
\pgfsetstrokecolor{textcolor}%
\pgfsetfillcolor{textcolor}%
\pgftext[x=5.009052in,y=0.522722in,,top]{\color{textcolor}{\sffamily\fontsize{11.000000}{13.200000}\selectfont\catcode`\^=\active\def^{\ifmmode\sp\else\^{}\fi}\catcode`\%=\active\def%{\%}$\mathdefault{1750}$}}%
\end{pgfscope}%
\begin{pgfscope}%
\pgfpathrectangle{\pgfqpoint{0.709829in}{0.654666in}}{\pgfqpoint{5.110171in}{0.887537in}}%
\pgfusepath{clip}%
\pgfsetroundcap%
\pgfsetroundjoin%
\pgfsetlinewidth{1.003750pt}%
\definecolor{currentstroke}{rgb}{0.800000,0.800000,0.800000}%
\pgfsetstrokecolor{currentstroke}%
\pgfsetdash{}{0pt}%
\pgfpathmoveto{\pgfqpoint{5.590043in}{0.654666in}}%
\pgfpathlineto{\pgfqpoint{5.590043in}{1.542204in}}%
\pgfusepath{stroke}%
\end{pgfscope}%
\begin{pgfscope}%
\definecolor{textcolor}{rgb}{0.150000,0.150000,0.150000}%
\pgfsetstrokecolor{textcolor}%
\pgfsetfillcolor{textcolor}%
\pgftext[x=5.590043in,y=0.522722in,,top]{\color{textcolor}{\sffamily\fontsize{11.000000}{13.200000}\selectfont\catcode`\^=\active\def^{\ifmmode\sp\else\^{}\fi}\catcode`\%=\active\def%{\%}$\mathdefault{2000}$}}%
\end{pgfscope}%
\begin{pgfscope}%
\definecolor{textcolor}{rgb}{0.150000,0.150000,0.150000}%
\pgfsetstrokecolor{textcolor}%
\pgfsetfillcolor{textcolor}%
\pgftext[x=3.264915in,y=0.331500in,,top]{\color{textcolor}{\sffamily\fontsize{12.000000}{14.400000}\selectfont\catcode`\^=\active\def^{\ifmmode\sp\else\^{}\fi}\catcode`\%=\active\def%{\%}Sample Index}}%
\end{pgfscope}%
\begin{pgfscope}%
\pgfpathrectangle{\pgfqpoint{0.709829in}{0.654666in}}{\pgfqpoint{5.110171in}{0.887537in}}%
\pgfusepath{clip}%
\pgfsetroundcap%
\pgfsetroundjoin%
\pgfsetlinewidth{1.003750pt}%
\definecolor{currentstroke}{rgb}{0.800000,0.800000,0.800000}%
\pgfsetstrokecolor{currentstroke}%
\pgfsetdash{}{0pt}%
\pgfpathmoveto{\pgfqpoint{0.709829in}{0.709540in}}%
\pgfpathlineto{\pgfqpoint{5.820000in}{0.709540in}}%
\pgfusepath{stroke}%
\end{pgfscope}%
\begin{pgfscope}%
\definecolor{textcolor}{rgb}{0.150000,0.150000,0.150000}%
\pgfsetstrokecolor{textcolor}%
\pgfsetfillcolor{textcolor}%
\pgftext[x=0.265268in, y=0.656526in, left, base]{\color{textcolor}{\sffamily\fontsize{11.000000}{13.200000}\selectfont\catcode`\^=\active\def^{\ifmmode\sp\else\^{}\fi}\catcode`\%=\active\def%{\%}$\mathdefault{-2.5}$}}%
\end{pgfscope}%
\begin{pgfscope}%
\pgfpathrectangle{\pgfqpoint{0.709829in}{0.654666in}}{\pgfqpoint{5.110171in}{0.887537in}}%
\pgfusepath{clip}%
\pgfsetroundcap%
\pgfsetroundjoin%
\pgfsetlinewidth{1.003750pt}%
\definecolor{currentstroke}{rgb}{0.800000,0.800000,0.800000}%
\pgfsetstrokecolor{currentstroke}%
\pgfsetdash{}{0pt}%
\pgfpathmoveto{\pgfqpoint{0.709829in}{1.091914in}}%
\pgfpathlineto{\pgfqpoint{5.820000in}{1.091914in}}%
\pgfusepath{stroke}%
\end{pgfscope}%
\begin{pgfscope}%
\definecolor{textcolor}{rgb}{0.150000,0.150000,0.150000}%
\pgfsetstrokecolor{textcolor}%
\pgfsetfillcolor{textcolor}%
\pgftext[x=0.383556in, y=1.038901in, left, base]{\color{textcolor}{\sffamily\fontsize{11.000000}{13.200000}\selectfont\catcode`\^=\active\def^{\ifmmode\sp\else\^{}\fi}\catcode`\%=\active\def%{\%}$\mathdefault{0.0}$}}%
\end{pgfscope}%
\begin{pgfscope}%
\pgfpathrectangle{\pgfqpoint{0.709829in}{0.654666in}}{\pgfqpoint{5.110171in}{0.887537in}}%
\pgfusepath{clip}%
\pgfsetroundcap%
\pgfsetroundjoin%
\pgfsetlinewidth{1.003750pt}%
\definecolor{currentstroke}{rgb}{0.800000,0.800000,0.800000}%
\pgfsetstrokecolor{currentstroke}%
\pgfsetdash{}{0pt}%
\pgfpathmoveto{\pgfqpoint{0.709829in}{1.474289in}}%
\pgfpathlineto{\pgfqpoint{5.820000in}{1.474289in}}%
\pgfusepath{stroke}%
\end{pgfscope}%
\begin{pgfscope}%
\definecolor{textcolor}{rgb}{0.150000,0.150000,0.150000}%
\pgfsetstrokecolor{textcolor}%
\pgfsetfillcolor{textcolor}%
\pgftext[x=0.383556in, y=1.421275in, left, base]{\color{textcolor}{\sffamily\fontsize{11.000000}{13.200000}\selectfont\catcode`\^=\active\def^{\ifmmode\sp\else\^{}\fi}\catcode`\%=\active\def%{\%}$\mathdefault{2.5}$}}%
\end{pgfscope}%
\begin{pgfscope}%
\definecolor{textcolor}{rgb}{0.150000,0.150000,0.150000}%
\pgfsetstrokecolor{textcolor}%
\pgfsetfillcolor{textcolor}%
\pgftext[x=0.209713in,y=1.098435in,,bottom,rotate=90.000000]{\color{textcolor}{\sffamily\fontsize{12.000000}{14.400000}\selectfont\catcode`\^=\active\def^{\ifmmode\sp\else\^{}\fi}\catcode`\%=\active\def%{\%}Amplitude}}%
\end{pgfscope}%
\begin{pgfscope}%
\pgfpathrectangle{\pgfqpoint{0.709829in}{0.654666in}}{\pgfqpoint{5.110171in}{0.887537in}}%
\pgfusepath{clip}%
\pgfsetroundcap%
\pgfsetroundjoin%
\pgfsetlinewidth{1.003750pt}%
\definecolor{currentstroke}{rgb}{0.298039,0.447059,0.690196}%
\pgfsetstrokecolor{currentstroke}%
\pgfsetdash{}{0pt}%
\pgfpathmoveto{\pgfqpoint{0.942110in}{0.893321in}}%
\pgfpathlineto{\pgfqpoint{0.944433in}{0.921902in}}%
\pgfpathlineto{\pgfqpoint{0.946757in}{0.845172in}}%
\pgfpathlineto{\pgfqpoint{0.949081in}{0.902760in}}%
\pgfpathlineto{\pgfqpoint{0.951405in}{0.880680in}}%
\pgfpathlineto{\pgfqpoint{0.953729in}{0.936227in}}%
\pgfpathlineto{\pgfqpoint{0.956053in}{0.763266in}}%
\pgfpathlineto{\pgfqpoint{0.958377in}{0.899900in}}%
\pgfpathlineto{\pgfqpoint{0.960701in}{0.810806in}}%
\pgfpathlineto{\pgfqpoint{0.963025in}{0.996388in}}%
\pgfpathlineto{\pgfqpoint{0.965349in}{0.883463in}}%
\pgfpathlineto{\pgfqpoint{0.967673in}{0.874291in}}%
\pgfpathlineto{\pgfqpoint{0.969997in}{0.947438in}}%
\pgfpathlineto{\pgfqpoint{0.972321in}{0.810252in}}%
\pgfpathlineto{\pgfqpoint{0.974645in}{0.985146in}}%
\pgfpathlineto{\pgfqpoint{0.976969in}{0.863926in}}%
\pgfpathlineto{\pgfqpoint{0.979293in}{0.875595in}}%
\pgfpathlineto{\pgfqpoint{0.981617in}{0.860896in}}%
\pgfpathlineto{\pgfqpoint{0.983941in}{0.942986in}}%
\pgfpathlineto{\pgfqpoint{0.986265in}{0.883631in}}%
\pgfpathlineto{\pgfqpoint{0.988589in}{0.888439in}}%
\pgfpathlineto{\pgfqpoint{0.990913in}{0.928125in}}%
\pgfpathlineto{\pgfqpoint{0.993237in}{0.933837in}}%
\pgfpathlineto{\pgfqpoint{0.995561in}{0.980406in}}%
\pgfpathlineto{\pgfqpoint{0.997885in}{0.895100in}}%
\pgfpathlineto{\pgfqpoint{1.000209in}{0.889474in}}%
\pgfpathlineto{\pgfqpoint{1.002533in}{0.925669in}}%
\pgfpathlineto{\pgfqpoint{1.004857in}{0.942852in}}%
\pgfpathlineto{\pgfqpoint{1.007181in}{1.088456in}}%
\pgfpathlineto{\pgfqpoint{1.009505in}{0.938344in}}%
\pgfpathlineto{\pgfqpoint{1.011829in}{1.035251in}}%
\pgfpathlineto{\pgfqpoint{1.014153in}{1.011031in}}%
\pgfpathlineto{\pgfqpoint{1.016476in}{1.070298in}}%
\pgfpathlineto{\pgfqpoint{1.021124in}{1.002578in}}%
\pgfpathlineto{\pgfqpoint{1.023448in}{0.953707in}}%
\pgfpathlineto{\pgfqpoint{1.025772in}{1.118952in}}%
\pgfpathlineto{\pgfqpoint{1.028096in}{1.005399in}}%
\pgfpathlineto{\pgfqpoint{1.030420in}{1.066291in}}%
\pgfpathlineto{\pgfqpoint{1.032744in}{1.060552in}}%
\pgfpathlineto{\pgfqpoint{1.035068in}{0.964038in}}%
\pgfpathlineto{\pgfqpoint{1.037392in}{0.978599in}}%
\pgfpathlineto{\pgfqpoint{1.039716in}{1.127873in}}%
\pgfpathlineto{\pgfqpoint{1.042040in}{0.999068in}}%
\pgfpathlineto{\pgfqpoint{1.044364in}{1.011624in}}%
\pgfpathlineto{\pgfqpoint{1.046688in}{1.056089in}}%
\pgfpathlineto{\pgfqpoint{1.049012in}{0.943672in}}%
\pgfpathlineto{\pgfqpoint{1.051336in}{1.064884in}}%
\pgfpathlineto{\pgfqpoint{1.053660in}{1.071783in}}%
\pgfpathlineto{\pgfqpoint{1.055984in}{0.972266in}}%
\pgfpathlineto{\pgfqpoint{1.058308in}{1.118679in}}%
\pgfpathlineto{\pgfqpoint{1.062956in}{0.979021in}}%
\pgfpathlineto{\pgfqpoint{1.065280in}{1.087836in}}%
\pgfpathlineto{\pgfqpoint{1.069928in}{0.984809in}}%
\pgfpathlineto{\pgfqpoint{1.072252in}{1.118009in}}%
\pgfpathlineto{\pgfqpoint{1.076900in}{1.096981in}}%
\pgfpathlineto{\pgfqpoint{1.079224in}{1.065682in}}%
\pgfpathlineto{\pgfqpoint{1.081548in}{1.007591in}}%
\pgfpathlineto{\pgfqpoint{1.083872in}{1.084256in}}%
\pgfpathlineto{\pgfqpoint{1.086195in}{1.083496in}}%
\pgfpathlineto{\pgfqpoint{1.088519in}{1.050337in}}%
\pgfpathlineto{\pgfqpoint{1.090843in}{1.064581in}}%
\pgfpathlineto{\pgfqpoint{1.093167in}{1.173093in}}%
\pgfpathlineto{\pgfqpoint{1.095491in}{1.179005in}}%
\pgfpathlineto{\pgfqpoint{1.097815in}{1.193330in}}%
\pgfpathlineto{\pgfqpoint{1.100139in}{1.045815in}}%
\pgfpathlineto{\pgfqpoint{1.102463in}{1.140605in}}%
\pgfpathlineto{\pgfqpoint{1.104787in}{1.171792in}}%
\pgfpathlineto{\pgfqpoint{1.107111in}{1.080144in}}%
\pgfpathlineto{\pgfqpoint{1.109435in}{1.219010in}}%
\pgfpathlineto{\pgfqpoint{1.111759in}{1.090126in}}%
\pgfpathlineto{\pgfqpoint{1.116407in}{1.168834in}}%
\pgfpathlineto{\pgfqpoint{1.118731in}{1.110042in}}%
\pgfpathlineto{\pgfqpoint{1.121055in}{1.145840in}}%
\pgfpathlineto{\pgfqpoint{1.123379in}{1.242453in}}%
\pgfpathlineto{\pgfqpoint{1.125703in}{1.206526in}}%
\pgfpathlineto{\pgfqpoint{1.130351in}{1.121126in}}%
\pgfpathlineto{\pgfqpoint{1.132675in}{1.218984in}}%
\pgfpathlineto{\pgfqpoint{1.134999in}{1.196862in}}%
\pgfpathlineto{\pgfqpoint{1.137323in}{1.193823in}}%
\pgfpathlineto{\pgfqpoint{1.139647in}{1.125805in}}%
\pgfpathlineto{\pgfqpoint{1.141971in}{1.197870in}}%
\pgfpathlineto{\pgfqpoint{1.144295in}{1.172304in}}%
\pgfpathlineto{\pgfqpoint{1.146619in}{1.227937in}}%
\pgfpathlineto{\pgfqpoint{1.148943in}{1.199577in}}%
\pgfpathlineto{\pgfqpoint{1.151267in}{1.128348in}}%
\pgfpathlineto{\pgfqpoint{1.153591in}{1.178795in}}%
\pgfpathlineto{\pgfqpoint{1.155914in}{1.168211in}}%
\pgfpathlineto{\pgfqpoint{1.160562in}{1.315941in}}%
\pgfpathlineto{\pgfqpoint{1.162886in}{1.216626in}}%
\pgfpathlineto{\pgfqpoint{1.167534in}{1.332742in}}%
\pgfpathlineto{\pgfqpoint{1.169858in}{1.308485in}}%
\pgfpathlineto{\pgfqpoint{1.172182in}{1.243882in}}%
\pgfpathlineto{\pgfqpoint{1.174506in}{1.335308in}}%
\pgfpathlineto{\pgfqpoint{1.179154in}{1.213852in}}%
\pgfpathlineto{\pgfqpoint{1.181478in}{1.290294in}}%
\pgfpathlineto{\pgfqpoint{1.183802in}{1.225768in}}%
\pgfpathlineto{\pgfqpoint{1.186126in}{1.268795in}}%
\pgfpathlineto{\pgfqpoint{1.188450in}{1.258374in}}%
\pgfpathlineto{\pgfqpoint{1.190774in}{1.275953in}}%
\pgfpathlineto{\pgfqpoint{1.193098in}{1.268766in}}%
\pgfpathlineto{\pgfqpoint{1.195422in}{1.286870in}}%
\pgfpathlineto{\pgfqpoint{1.197746in}{1.259924in}}%
\pgfpathlineto{\pgfqpoint{1.200070in}{1.183063in}}%
\pgfpathlineto{\pgfqpoint{1.202394in}{1.308901in}}%
\pgfpathlineto{\pgfqpoint{1.204718in}{1.355382in}}%
\pgfpathlineto{\pgfqpoint{1.207042in}{1.241773in}}%
\pgfpathlineto{\pgfqpoint{1.209366in}{1.304183in}}%
\pgfpathlineto{\pgfqpoint{1.211690in}{1.322001in}}%
\pgfpathlineto{\pgfqpoint{1.214014in}{1.375378in}}%
\pgfpathlineto{\pgfqpoint{1.216338in}{1.257166in}}%
\pgfpathlineto{\pgfqpoint{1.220986in}{1.364324in}}%
\pgfpathlineto{\pgfqpoint{1.223310in}{1.347244in}}%
\pgfpathlineto{\pgfqpoint{1.225633in}{1.320591in}}%
\pgfpathlineto{\pgfqpoint{1.227957in}{1.338440in}}%
\pgfpathlineto{\pgfqpoint{1.230281in}{1.345782in}}%
\pgfpathlineto{\pgfqpoint{1.232605in}{0.844503in}}%
\pgfpathlineto{\pgfqpoint{1.234929in}{0.856017in}}%
\pgfpathlineto{\pgfqpoint{1.237253in}{0.721469in}}%
\pgfpathlineto{\pgfqpoint{1.239577in}{0.755842in}}%
\pgfpathlineto{\pgfqpoint{1.241901in}{0.951853in}}%
\pgfpathlineto{\pgfqpoint{1.244225in}{0.852616in}}%
\pgfpathlineto{\pgfqpoint{1.246549in}{0.839237in}}%
\pgfpathlineto{\pgfqpoint{1.248873in}{0.975666in}}%
\pgfpathlineto{\pgfqpoint{1.251197in}{0.827117in}}%
\pgfpathlineto{\pgfqpoint{1.253521in}{0.862199in}}%
\pgfpathlineto{\pgfqpoint{1.255845in}{0.934930in}}%
\pgfpathlineto{\pgfqpoint{1.258169in}{0.818740in}}%
\pgfpathlineto{\pgfqpoint{1.265141in}{1.028297in}}%
\pgfpathlineto{\pgfqpoint{1.267465in}{0.889851in}}%
\pgfpathlineto{\pgfqpoint{1.269789in}{0.912577in}}%
\pgfpathlineto{\pgfqpoint{1.272113in}{0.770671in}}%
\pgfpathlineto{\pgfqpoint{1.276761in}{0.919265in}}%
\pgfpathlineto{\pgfqpoint{1.279085in}{0.974447in}}%
\pgfpathlineto{\pgfqpoint{1.281409in}{0.931965in}}%
\pgfpathlineto{\pgfqpoint{1.283733in}{0.939542in}}%
\pgfpathlineto{\pgfqpoint{1.286057in}{0.952195in}}%
\pgfpathlineto{\pgfqpoint{1.288381in}{0.905394in}}%
\pgfpathlineto{\pgfqpoint{1.290705in}{1.004005in}}%
\pgfpathlineto{\pgfqpoint{1.293029in}{1.025790in}}%
\pgfpathlineto{\pgfqpoint{1.295353in}{0.855298in}}%
\pgfpathlineto{\pgfqpoint{1.300000in}{0.964727in}}%
\pgfpathlineto{\pgfqpoint{1.302324in}{0.909034in}}%
\pgfpathlineto{\pgfqpoint{1.306972in}{0.948587in}}%
\pgfpathlineto{\pgfqpoint{1.309296in}{1.071460in}}%
\pgfpathlineto{\pgfqpoint{1.311620in}{1.056948in}}%
\pgfpathlineto{\pgfqpoint{1.313944in}{1.023074in}}%
\pgfpathlineto{\pgfqpoint{1.316268in}{0.972399in}}%
\pgfpathlineto{\pgfqpoint{1.318592in}{1.030500in}}%
\pgfpathlineto{\pgfqpoint{1.320916in}{0.945517in}}%
\pgfpathlineto{\pgfqpoint{1.323240in}{1.144807in}}%
\pgfpathlineto{\pgfqpoint{1.325564in}{0.967492in}}%
\pgfpathlineto{\pgfqpoint{1.330212in}{1.070478in}}%
\pgfpathlineto{\pgfqpoint{1.332536in}{0.954581in}}%
\pgfpathlineto{\pgfqpoint{1.337184in}{1.070138in}}%
\pgfpathlineto{\pgfqpoint{1.339508in}{1.006001in}}%
\pgfpathlineto{\pgfqpoint{1.341832in}{1.102268in}}%
\pgfpathlineto{\pgfqpoint{1.344156in}{1.014406in}}%
\pgfpathlineto{\pgfqpoint{1.346480in}{1.008797in}}%
\pgfpathlineto{\pgfqpoint{1.348804in}{1.135150in}}%
\pgfpathlineto{\pgfqpoint{1.351128in}{1.040282in}}%
\pgfpathlineto{\pgfqpoint{1.353452in}{1.125498in}}%
\pgfpathlineto{\pgfqpoint{1.355776in}{1.027929in}}%
\pgfpathlineto{\pgfqpoint{1.358100in}{1.056750in}}%
\pgfpathlineto{\pgfqpoint{1.360424in}{1.007788in}}%
\pgfpathlineto{\pgfqpoint{1.362748in}{1.050033in}}%
\pgfpathlineto{\pgfqpoint{1.365072in}{0.985904in}}%
\pgfpathlineto{\pgfqpoint{1.367395in}{1.095655in}}%
\pgfpathlineto{\pgfqpoint{1.369719in}{1.061028in}}%
\pgfpathlineto{\pgfqpoint{1.372043in}{0.978558in}}%
\pgfpathlineto{\pgfqpoint{1.374367in}{1.108902in}}%
\pgfpathlineto{\pgfqpoint{1.376691in}{1.066519in}}%
\pgfpathlineto{\pgfqpoint{1.379015in}{1.123758in}}%
\pgfpathlineto{\pgfqpoint{1.381339in}{1.124980in}}%
\pgfpathlineto{\pgfqpoint{1.383663in}{1.024357in}}%
\pgfpathlineto{\pgfqpoint{1.385987in}{1.139668in}}%
\pgfpathlineto{\pgfqpoint{1.390635in}{1.013691in}}%
\pgfpathlineto{\pgfqpoint{1.392959in}{1.153437in}}%
\pgfpathlineto{\pgfqpoint{1.395283in}{1.133863in}}%
\pgfpathlineto{\pgfqpoint{1.397607in}{0.999037in}}%
\pgfpathlineto{\pgfqpoint{1.399931in}{1.104379in}}%
\pgfpathlineto{\pgfqpoint{1.402255in}{1.153774in}}%
\pgfpathlineto{\pgfqpoint{1.404579in}{1.098773in}}%
\pgfpathlineto{\pgfqpoint{1.406903in}{1.148469in}}%
\pgfpathlineto{\pgfqpoint{1.409227in}{1.073278in}}%
\pgfpathlineto{\pgfqpoint{1.411551in}{1.081882in}}%
\pgfpathlineto{\pgfqpoint{1.413875in}{1.045305in}}%
\pgfpathlineto{\pgfqpoint{1.416199in}{1.212579in}}%
\pgfpathlineto{\pgfqpoint{1.418523in}{1.194386in}}%
\pgfpathlineto{\pgfqpoint{1.420847in}{1.122118in}}%
\pgfpathlineto{\pgfqpoint{1.423171in}{1.162605in}}%
\pgfpathlineto{\pgfqpoint{1.425495in}{1.128853in}}%
\pgfpathlineto{\pgfqpoint{1.427819in}{1.190037in}}%
\pgfpathlineto{\pgfqpoint{1.430143in}{1.198930in}}%
\pgfpathlineto{\pgfqpoint{1.432467in}{1.185029in}}%
\pgfpathlineto{\pgfqpoint{1.434791in}{1.234017in}}%
\pgfpathlineto{\pgfqpoint{1.437114in}{1.238541in}}%
\pgfpathlineto{\pgfqpoint{1.439438in}{1.130994in}}%
\pgfpathlineto{\pgfqpoint{1.441762in}{1.263118in}}%
\pgfpathlineto{\pgfqpoint{1.444086in}{1.301900in}}%
\pgfpathlineto{\pgfqpoint{1.446410in}{1.247206in}}%
\pgfpathlineto{\pgfqpoint{1.448734in}{1.247107in}}%
\pgfpathlineto{\pgfqpoint{1.451058in}{1.185189in}}%
\pgfpathlineto{\pgfqpoint{1.453382in}{1.180719in}}%
\pgfpathlineto{\pgfqpoint{1.455706in}{1.199145in}}%
\pgfpathlineto{\pgfqpoint{1.458030in}{1.244850in}}%
\pgfpathlineto{\pgfqpoint{1.460354in}{1.171290in}}%
\pgfpathlineto{\pgfqpoint{1.469650in}{1.323397in}}%
\pgfpathlineto{\pgfqpoint{1.471974in}{1.176683in}}%
\pgfpathlineto{\pgfqpoint{1.474298in}{1.325703in}}%
\pgfpathlineto{\pgfqpoint{1.476622in}{1.283455in}}%
\pgfpathlineto{\pgfqpoint{1.478946in}{1.275733in}}%
\pgfpathlineto{\pgfqpoint{1.481270in}{1.246287in}}%
\pgfpathlineto{\pgfqpoint{1.483594in}{1.184313in}}%
\pgfpathlineto{\pgfqpoint{1.485918in}{1.348872in}}%
\pgfpathlineto{\pgfqpoint{1.488242in}{1.342907in}}%
\pgfpathlineto{\pgfqpoint{1.492890in}{1.260301in}}%
\pgfpathlineto{\pgfqpoint{1.495214in}{1.282053in}}%
\pgfpathlineto{\pgfqpoint{1.497538in}{1.345557in}}%
\pgfpathlineto{\pgfqpoint{1.502186in}{1.213927in}}%
\pgfpathlineto{\pgfqpoint{1.504510in}{1.402416in}}%
\pgfpathlineto{\pgfqpoint{1.506834in}{1.280081in}}%
\pgfpathlineto{\pgfqpoint{1.509157in}{1.339085in}}%
\pgfpathlineto{\pgfqpoint{1.511481in}{1.324842in}}%
\pgfpathlineto{\pgfqpoint{1.513805in}{1.257464in}}%
\pgfpathlineto{\pgfqpoint{1.516129in}{1.423381in}}%
\pgfpathlineto{\pgfqpoint{1.518453in}{1.371638in}}%
\pgfpathlineto{\pgfqpoint{1.520777in}{1.266509in}}%
\pgfpathlineto{\pgfqpoint{1.523101in}{0.790681in}}%
\pgfpathlineto{\pgfqpoint{1.525425in}{0.887971in}}%
\pgfpathlineto{\pgfqpoint{1.527749in}{0.835724in}}%
\pgfpathlineto{\pgfqpoint{1.530073in}{0.854509in}}%
\pgfpathlineto{\pgfqpoint{1.532397in}{0.897586in}}%
\pgfpathlineto{\pgfqpoint{1.534721in}{0.874380in}}%
\pgfpathlineto{\pgfqpoint{1.537045in}{0.864421in}}%
\pgfpathlineto{\pgfqpoint{1.539369in}{0.847666in}}%
\pgfpathlineto{\pgfqpoint{1.541693in}{0.800971in}}%
\pgfpathlineto{\pgfqpoint{1.544017in}{0.812715in}}%
\pgfpathlineto{\pgfqpoint{1.546341in}{0.944580in}}%
\pgfpathlineto{\pgfqpoint{1.548665in}{0.929827in}}%
\pgfpathlineto{\pgfqpoint{1.553313in}{0.873073in}}%
\pgfpathlineto{\pgfqpoint{1.555637in}{0.958541in}}%
\pgfpathlineto{\pgfqpoint{1.557961in}{0.895039in}}%
\pgfpathlineto{\pgfqpoint{1.560285in}{0.917766in}}%
\pgfpathlineto{\pgfqpoint{1.562609in}{0.916522in}}%
\pgfpathlineto{\pgfqpoint{1.564933in}{0.966438in}}%
\pgfpathlineto{\pgfqpoint{1.567257in}{0.848588in}}%
\pgfpathlineto{\pgfqpoint{1.569581in}{0.996902in}}%
\pgfpathlineto{\pgfqpoint{1.571905in}{1.010188in}}%
\pgfpathlineto{\pgfqpoint{1.574229in}{1.008991in}}%
\pgfpathlineto{\pgfqpoint{1.578876in}{0.907613in}}%
\pgfpathlineto{\pgfqpoint{1.581200in}{1.033057in}}%
\pgfpathlineto{\pgfqpoint{1.583524in}{0.948411in}}%
\pgfpathlineto{\pgfqpoint{1.590496in}{1.006031in}}%
\pgfpathlineto{\pgfqpoint{1.592820in}{0.985679in}}%
\pgfpathlineto{\pgfqpoint{1.595144in}{1.001922in}}%
\pgfpathlineto{\pgfqpoint{1.597468in}{0.942140in}}%
\pgfpathlineto{\pgfqpoint{1.599792in}{0.832412in}}%
\pgfpathlineto{\pgfqpoint{1.602116in}{0.900245in}}%
\pgfpathlineto{\pgfqpoint{1.604440in}{0.905458in}}%
\pgfpathlineto{\pgfqpoint{1.606764in}{1.032270in}}%
\pgfpathlineto{\pgfqpoint{1.609088in}{1.010337in}}%
\pgfpathlineto{\pgfqpoint{1.611412in}{0.957012in}}%
\pgfpathlineto{\pgfqpoint{1.616060in}{0.982301in}}%
\pgfpathlineto{\pgfqpoint{1.618384in}{1.028661in}}%
\pgfpathlineto{\pgfqpoint{1.620708in}{0.984012in}}%
\pgfpathlineto{\pgfqpoint{1.623032in}{1.068768in}}%
\pgfpathlineto{\pgfqpoint{1.625356in}{1.003953in}}%
\pgfpathlineto{\pgfqpoint{1.627680in}{1.069741in}}%
\pgfpathlineto{\pgfqpoint{1.630004in}{0.985593in}}%
\pgfpathlineto{\pgfqpoint{1.632328in}{1.088411in}}%
\pgfpathlineto{\pgfqpoint{1.636976in}{1.170265in}}%
\pgfpathlineto{\pgfqpoint{1.639300in}{1.030030in}}%
\pgfpathlineto{\pgfqpoint{1.641624in}{1.065237in}}%
\pgfpathlineto{\pgfqpoint{1.648595in}{0.941071in}}%
\pgfpathlineto{\pgfqpoint{1.650919in}{1.068920in}}%
\pgfpathlineto{\pgfqpoint{1.653243in}{1.040636in}}%
\pgfpathlineto{\pgfqpoint{1.655567in}{1.158168in}}%
\pgfpathlineto{\pgfqpoint{1.657891in}{1.198510in}}%
\pgfpathlineto{\pgfqpoint{1.664863in}{1.060946in}}%
\pgfpathlineto{\pgfqpoint{1.667187in}{1.166345in}}%
\pgfpathlineto{\pgfqpoint{1.669511in}{1.158362in}}%
\pgfpathlineto{\pgfqpoint{1.671835in}{1.248721in}}%
\pgfpathlineto{\pgfqpoint{1.676483in}{1.090606in}}%
\pgfpathlineto{\pgfqpoint{1.678807in}{1.183584in}}%
\pgfpathlineto{\pgfqpoint{1.681131in}{1.162565in}}%
\pgfpathlineto{\pgfqpoint{1.683455in}{1.214812in}}%
\pgfpathlineto{\pgfqpoint{1.685779in}{1.189056in}}%
\pgfpathlineto{\pgfqpoint{1.688103in}{1.113783in}}%
\pgfpathlineto{\pgfqpoint{1.690427in}{1.239023in}}%
\pgfpathlineto{\pgfqpoint{1.692751in}{1.139939in}}%
\pgfpathlineto{\pgfqpoint{1.695075in}{1.271785in}}%
\pgfpathlineto{\pgfqpoint{1.697399in}{1.111889in}}%
\pgfpathlineto{\pgfqpoint{1.702047in}{1.171177in}}%
\pgfpathlineto{\pgfqpoint{1.704371in}{1.176112in}}%
\pgfpathlineto{\pgfqpoint{1.706695in}{1.232970in}}%
\pgfpathlineto{\pgfqpoint{1.709019in}{1.163455in}}%
\pgfpathlineto{\pgfqpoint{1.711343in}{1.203331in}}%
\pgfpathlineto{\pgfqpoint{1.713667in}{1.157087in}}%
\pgfpathlineto{\pgfqpoint{1.715991in}{1.235635in}}%
\pgfpathlineto{\pgfqpoint{1.720638in}{1.174689in}}%
\pgfpathlineto{\pgfqpoint{1.722962in}{1.183851in}}%
\pgfpathlineto{\pgfqpoint{1.725286in}{1.251866in}}%
\pgfpathlineto{\pgfqpoint{1.727610in}{1.164194in}}%
\pgfpathlineto{\pgfqpoint{1.729934in}{1.219483in}}%
\pgfpathlineto{\pgfqpoint{1.732258in}{1.239052in}}%
\pgfpathlineto{\pgfqpoint{1.734582in}{1.280468in}}%
\pgfpathlineto{\pgfqpoint{1.736906in}{1.193111in}}%
\pgfpathlineto{\pgfqpoint{1.739230in}{1.189525in}}%
\pgfpathlineto{\pgfqpoint{1.741554in}{1.213363in}}%
\pgfpathlineto{\pgfqpoint{1.743878in}{1.302842in}}%
\pgfpathlineto{\pgfqpoint{1.746202in}{1.190504in}}%
\pgfpathlineto{\pgfqpoint{1.748526in}{1.299321in}}%
\pgfpathlineto{\pgfqpoint{1.750850in}{1.278321in}}%
\pgfpathlineto{\pgfqpoint{1.753174in}{1.226792in}}%
\pgfpathlineto{\pgfqpoint{1.755498in}{1.201880in}}%
\pgfpathlineto{\pgfqpoint{1.757822in}{1.420678in}}%
\pgfpathlineto{\pgfqpoint{1.762470in}{1.241211in}}%
\pgfpathlineto{\pgfqpoint{1.764794in}{1.265667in}}%
\pgfpathlineto{\pgfqpoint{1.767118in}{1.275513in}}%
\pgfpathlineto{\pgfqpoint{1.769442in}{1.269994in}}%
\pgfpathlineto{\pgfqpoint{1.771766in}{1.236981in}}%
\pgfpathlineto{\pgfqpoint{1.774090in}{1.345221in}}%
\pgfpathlineto{\pgfqpoint{1.776414in}{1.250185in}}%
\pgfpathlineto{\pgfqpoint{1.778738in}{1.289135in}}%
\pgfpathlineto{\pgfqpoint{1.781062in}{1.252308in}}%
\pgfpathlineto{\pgfqpoint{1.783386in}{1.270846in}}%
\pgfpathlineto{\pgfqpoint{1.785710in}{1.252341in}}%
\pgfpathlineto{\pgfqpoint{1.788034in}{1.321201in}}%
\pgfpathlineto{\pgfqpoint{1.790357in}{1.344592in}}%
\pgfpathlineto{\pgfqpoint{1.792681in}{1.319365in}}%
\pgfpathlineto{\pgfqpoint{1.795005in}{1.376240in}}%
\pgfpathlineto{\pgfqpoint{1.797329in}{1.311957in}}%
\pgfpathlineto{\pgfqpoint{1.799653in}{1.402242in}}%
\pgfpathlineto{\pgfqpoint{1.801977in}{1.409927in}}%
\pgfpathlineto{\pgfqpoint{1.804301in}{1.501861in}}%
\pgfpathlineto{\pgfqpoint{1.806625in}{1.389196in}}%
\pgfpathlineto{\pgfqpoint{1.808949in}{1.405096in}}%
\pgfpathlineto{\pgfqpoint{1.811273in}{1.336287in}}%
\pgfpathlineto{\pgfqpoint{1.813597in}{0.883878in}}%
\pgfpathlineto{\pgfqpoint{1.815921in}{0.714644in}}%
\pgfpathlineto{\pgfqpoint{1.818245in}{0.902839in}}%
\pgfpathlineto{\pgfqpoint{1.820569in}{0.843627in}}%
\pgfpathlineto{\pgfqpoint{1.822893in}{0.929850in}}%
\pgfpathlineto{\pgfqpoint{1.825217in}{0.832165in}}%
\pgfpathlineto{\pgfqpoint{1.827541in}{0.892978in}}%
\pgfpathlineto{\pgfqpoint{1.829865in}{0.891437in}}%
\pgfpathlineto{\pgfqpoint{1.832189in}{0.830370in}}%
\pgfpathlineto{\pgfqpoint{1.834513in}{0.908095in}}%
\pgfpathlineto{\pgfqpoint{1.836837in}{0.891667in}}%
\pgfpathlineto{\pgfqpoint{1.839161in}{0.796519in}}%
\pgfpathlineto{\pgfqpoint{1.841485in}{0.953138in}}%
\pgfpathlineto{\pgfqpoint{1.843809in}{0.949796in}}%
\pgfpathlineto{\pgfqpoint{1.848457in}{0.860399in}}%
\pgfpathlineto{\pgfqpoint{1.850781in}{0.991158in}}%
\pgfpathlineto{\pgfqpoint{1.853105in}{0.903208in}}%
\pgfpathlineto{\pgfqpoint{1.855429in}{1.006460in}}%
\pgfpathlineto{\pgfqpoint{1.857753in}{0.847479in}}%
\pgfpathlineto{\pgfqpoint{1.860076in}{0.956743in}}%
\pgfpathlineto{\pgfqpoint{1.862400in}{0.895288in}}%
\pgfpathlineto{\pgfqpoint{1.864724in}{0.905653in}}%
\pgfpathlineto{\pgfqpoint{1.867048in}{0.983952in}}%
\pgfpathlineto{\pgfqpoint{1.869372in}{0.846578in}}%
\pgfpathlineto{\pgfqpoint{1.871696in}{0.877691in}}%
\pgfpathlineto{\pgfqpoint{1.874020in}{0.854168in}}%
\pgfpathlineto{\pgfqpoint{1.878668in}{0.986772in}}%
\pgfpathlineto{\pgfqpoint{1.880992in}{0.985308in}}%
\pgfpathlineto{\pgfqpoint{1.883316in}{1.008421in}}%
\pgfpathlineto{\pgfqpoint{1.885640in}{0.939673in}}%
\pgfpathlineto{\pgfqpoint{1.887964in}{1.042175in}}%
\pgfpathlineto{\pgfqpoint{1.890288in}{1.014586in}}%
\pgfpathlineto{\pgfqpoint{1.892612in}{0.941457in}}%
\pgfpathlineto{\pgfqpoint{1.894936in}{0.981627in}}%
\pgfpathlineto{\pgfqpoint{1.897260in}{0.869935in}}%
\pgfpathlineto{\pgfqpoint{1.901908in}{1.104523in}}%
\pgfpathlineto{\pgfqpoint{1.904232in}{1.019037in}}%
\pgfpathlineto{\pgfqpoint{1.906556in}{1.023059in}}%
\pgfpathlineto{\pgfqpoint{1.908880in}{0.891596in}}%
\pgfpathlineto{\pgfqpoint{1.911204in}{1.070742in}}%
\pgfpathlineto{\pgfqpoint{1.913528in}{0.872399in}}%
\pgfpathlineto{\pgfqpoint{1.915852in}{0.971621in}}%
\pgfpathlineto{\pgfqpoint{1.918176in}{0.935497in}}%
\pgfpathlineto{\pgfqpoint{1.920500in}{1.056061in}}%
\pgfpathlineto{\pgfqpoint{1.922824in}{1.095575in}}%
\pgfpathlineto{\pgfqpoint{1.925148in}{0.959213in}}%
\pgfpathlineto{\pgfqpoint{1.927472in}{1.002133in}}%
\pgfpathlineto{\pgfqpoint{1.929795in}{0.957199in}}%
\pgfpathlineto{\pgfqpoint{1.932119in}{1.060377in}}%
\pgfpathlineto{\pgfqpoint{1.934443in}{1.101860in}}%
\pgfpathlineto{\pgfqpoint{1.936767in}{0.997415in}}%
\pgfpathlineto{\pgfqpoint{1.939091in}{1.093425in}}%
\pgfpathlineto{\pgfqpoint{1.941415in}{1.014979in}}%
\pgfpathlineto{\pgfqpoint{1.943739in}{1.121618in}}%
\pgfpathlineto{\pgfqpoint{1.946063in}{1.080700in}}%
\pgfpathlineto{\pgfqpoint{1.948387in}{0.977744in}}%
\pgfpathlineto{\pgfqpoint{1.950711in}{1.037994in}}%
\pgfpathlineto{\pgfqpoint{1.953035in}{1.038336in}}%
\pgfpathlineto{\pgfqpoint{1.955359in}{1.076281in}}%
\pgfpathlineto{\pgfqpoint{1.957683in}{1.057679in}}%
\pgfpathlineto{\pgfqpoint{1.960007in}{1.093465in}}%
\pgfpathlineto{\pgfqpoint{1.962331in}{1.050572in}}%
\pgfpathlineto{\pgfqpoint{1.964655in}{1.045975in}}%
\pgfpathlineto{\pgfqpoint{1.966979in}{1.050134in}}%
\pgfpathlineto{\pgfqpoint{1.969303in}{1.129680in}}%
\pgfpathlineto{\pgfqpoint{1.971627in}{1.123175in}}%
\pgfpathlineto{\pgfqpoint{1.976275in}{1.073487in}}%
\pgfpathlineto{\pgfqpoint{1.978599in}{1.169031in}}%
\pgfpathlineto{\pgfqpoint{1.980923in}{1.132753in}}%
\pgfpathlineto{\pgfqpoint{1.983247in}{1.119492in}}%
\pgfpathlineto{\pgfqpoint{1.985571in}{1.179222in}}%
\pgfpathlineto{\pgfqpoint{1.987895in}{1.275081in}}%
\pgfpathlineto{\pgfqpoint{1.990219in}{1.213632in}}%
\pgfpathlineto{\pgfqpoint{1.992543in}{1.071153in}}%
\pgfpathlineto{\pgfqpoint{1.994867in}{1.192357in}}%
\pgfpathlineto{\pgfqpoint{1.999514in}{1.135762in}}%
\pgfpathlineto{\pgfqpoint{2.001838in}{1.222510in}}%
\pgfpathlineto{\pgfqpoint{2.004162in}{1.218224in}}%
\pgfpathlineto{\pgfqpoint{2.006486in}{1.153761in}}%
\pgfpathlineto{\pgfqpoint{2.008810in}{1.142297in}}%
\pgfpathlineto{\pgfqpoint{2.011134in}{1.202048in}}%
\pgfpathlineto{\pgfqpoint{2.013458in}{1.225140in}}%
\pgfpathlineto{\pgfqpoint{2.015782in}{1.163142in}}%
\pgfpathlineto{\pgfqpoint{2.018106in}{1.190805in}}%
\pgfpathlineto{\pgfqpoint{2.020430in}{1.236486in}}%
\pgfpathlineto{\pgfqpoint{2.022754in}{1.143736in}}%
\pgfpathlineto{\pgfqpoint{2.029726in}{1.199714in}}%
\pgfpathlineto{\pgfqpoint{2.032050in}{1.322969in}}%
\pgfpathlineto{\pgfqpoint{2.034374in}{1.125907in}}%
\pgfpathlineto{\pgfqpoint{2.036698in}{1.144444in}}%
\pgfpathlineto{\pgfqpoint{2.039022in}{1.077013in}}%
\pgfpathlineto{\pgfqpoint{2.043670in}{1.219659in}}%
\pgfpathlineto{\pgfqpoint{2.045994in}{1.268160in}}%
\pgfpathlineto{\pgfqpoint{2.050642in}{1.175554in}}%
\pgfpathlineto{\pgfqpoint{2.052966in}{1.306243in}}%
\pgfpathlineto{\pgfqpoint{2.055290in}{1.292457in}}%
\pgfpathlineto{\pgfqpoint{2.057614in}{1.221197in}}%
\pgfpathlineto{\pgfqpoint{2.062262in}{1.340304in}}%
\pgfpathlineto{\pgfqpoint{2.064586in}{1.287235in}}%
\pgfpathlineto{\pgfqpoint{2.066910in}{1.296264in}}%
\pgfpathlineto{\pgfqpoint{2.069234in}{1.190135in}}%
\pgfpathlineto{\pgfqpoint{2.071557in}{1.259987in}}%
\pgfpathlineto{\pgfqpoint{2.073881in}{1.234873in}}%
\pgfpathlineto{\pgfqpoint{2.076205in}{1.247826in}}%
\pgfpathlineto{\pgfqpoint{2.078529in}{1.290645in}}%
\pgfpathlineto{\pgfqpoint{2.080853in}{1.281612in}}%
\pgfpathlineto{\pgfqpoint{2.083177in}{1.280228in}}%
\pgfpathlineto{\pgfqpoint{2.085501in}{1.316533in}}%
\pgfpathlineto{\pgfqpoint{2.087825in}{1.304661in}}%
\pgfpathlineto{\pgfqpoint{2.090149in}{1.338482in}}%
\pgfpathlineto{\pgfqpoint{2.092473in}{1.396602in}}%
\pgfpathlineto{\pgfqpoint{2.094797in}{1.278990in}}%
\pgfpathlineto{\pgfqpoint{2.097121in}{1.296479in}}%
\pgfpathlineto{\pgfqpoint{2.099445in}{1.374058in}}%
\pgfpathlineto{\pgfqpoint{2.101769in}{1.264435in}}%
\pgfpathlineto{\pgfqpoint{2.104093in}{0.836496in}}%
\pgfpathlineto{\pgfqpoint{2.106417in}{0.806040in}}%
\pgfpathlineto{\pgfqpoint{2.108741in}{0.878720in}}%
\pgfpathlineto{\pgfqpoint{2.113389in}{0.802483in}}%
\pgfpathlineto{\pgfqpoint{2.118037in}{0.887351in}}%
\pgfpathlineto{\pgfqpoint{2.120361in}{0.898253in}}%
\pgfpathlineto{\pgfqpoint{2.122685in}{0.974270in}}%
\pgfpathlineto{\pgfqpoint{2.125009in}{0.792635in}}%
\pgfpathlineto{\pgfqpoint{2.127333in}{0.881918in}}%
\pgfpathlineto{\pgfqpoint{2.129657in}{0.834000in}}%
\pgfpathlineto{\pgfqpoint{2.131981in}{0.896995in}}%
\pgfpathlineto{\pgfqpoint{2.134305in}{0.928842in}}%
\pgfpathlineto{\pgfqpoint{2.136629in}{0.788718in}}%
\pgfpathlineto{\pgfqpoint{2.141276in}{0.961014in}}%
\pgfpathlineto{\pgfqpoint{2.143600in}{0.947803in}}%
\pgfpathlineto{\pgfqpoint{2.145924in}{0.980727in}}%
\pgfpathlineto{\pgfqpoint{2.148248in}{0.880624in}}%
\pgfpathlineto{\pgfqpoint{2.150572in}{0.854090in}}%
\pgfpathlineto{\pgfqpoint{2.155220in}{0.945786in}}%
\pgfpathlineto{\pgfqpoint{2.157544in}{0.874410in}}%
\pgfpathlineto{\pgfqpoint{2.159868in}{0.904289in}}%
\pgfpathlineto{\pgfqpoint{2.162192in}{0.964311in}}%
\pgfpathlineto{\pgfqpoint{2.166840in}{0.902515in}}%
\pgfpathlineto{\pgfqpoint{2.171488in}{1.012568in}}%
\pgfpathlineto{\pgfqpoint{2.173812in}{0.954307in}}%
\pgfpathlineto{\pgfqpoint{2.176136in}{1.027908in}}%
\pgfpathlineto{\pgfqpoint{2.178460in}{0.993669in}}%
\pgfpathlineto{\pgfqpoint{2.180784in}{0.980462in}}%
\pgfpathlineto{\pgfqpoint{2.183108in}{0.885482in}}%
\pgfpathlineto{\pgfqpoint{2.185432in}{1.032866in}}%
\pgfpathlineto{\pgfqpoint{2.190080in}{0.866184in}}%
\pgfpathlineto{\pgfqpoint{2.194728in}{1.053814in}}%
\pgfpathlineto{\pgfqpoint{2.197052in}{0.913833in}}%
\pgfpathlineto{\pgfqpoint{2.199376in}{0.983332in}}%
\pgfpathlineto{\pgfqpoint{2.201700in}{1.006722in}}%
\pgfpathlineto{\pgfqpoint{2.206348in}{1.009976in}}%
\pgfpathlineto{\pgfqpoint{2.208672in}{0.983847in}}%
\pgfpathlineto{\pgfqpoint{2.210995in}{0.970506in}}%
\pgfpathlineto{\pgfqpoint{2.213319in}{1.003272in}}%
\pgfpathlineto{\pgfqpoint{2.215643in}{1.083498in}}%
\pgfpathlineto{\pgfqpoint{2.217967in}{1.067261in}}%
\pgfpathlineto{\pgfqpoint{2.220291in}{0.980319in}}%
\pgfpathlineto{\pgfqpoint{2.222615in}{0.970302in}}%
\pgfpathlineto{\pgfqpoint{2.227263in}{1.067026in}}%
\pgfpathlineto{\pgfqpoint{2.229587in}{0.960622in}}%
\pgfpathlineto{\pgfqpoint{2.231911in}{1.034831in}}%
\pgfpathlineto{\pgfqpoint{2.234235in}{1.026210in}}%
\pgfpathlineto{\pgfqpoint{2.236559in}{0.988707in}}%
\pgfpathlineto{\pgfqpoint{2.241207in}{1.192051in}}%
\pgfpathlineto{\pgfqpoint{2.243531in}{1.099793in}}%
\pgfpathlineto{\pgfqpoint{2.245855in}{1.125085in}}%
\pgfpathlineto{\pgfqpoint{2.248179in}{1.090050in}}%
\pgfpathlineto{\pgfqpoint{2.250503in}{1.036908in}}%
\pgfpathlineto{\pgfqpoint{2.252827in}{1.116221in}}%
\pgfpathlineto{\pgfqpoint{2.255151in}{1.063442in}}%
\pgfpathlineto{\pgfqpoint{2.257475in}{1.086126in}}%
\pgfpathlineto{\pgfqpoint{2.259799in}{1.122543in}}%
\pgfpathlineto{\pgfqpoint{2.262123in}{1.203701in}}%
\pgfpathlineto{\pgfqpoint{2.264447in}{1.039229in}}%
\pgfpathlineto{\pgfqpoint{2.266771in}{1.068459in}}%
\pgfpathlineto{\pgfqpoint{2.269095in}{1.048458in}}%
\pgfpathlineto{\pgfqpoint{2.271419in}{1.071744in}}%
\pgfpathlineto{\pgfqpoint{2.273743in}{1.046457in}}%
\pgfpathlineto{\pgfqpoint{2.276067in}{1.064990in}}%
\pgfpathlineto{\pgfqpoint{2.278391in}{1.153974in}}%
\pgfpathlineto{\pgfqpoint{2.280715in}{1.090193in}}%
\pgfpathlineto{\pgfqpoint{2.283038in}{1.308017in}}%
\pgfpathlineto{\pgfqpoint{2.290010in}{1.123957in}}%
\pgfpathlineto{\pgfqpoint{2.292334in}{1.139843in}}%
\pgfpathlineto{\pgfqpoint{2.294658in}{1.112634in}}%
\pgfpathlineto{\pgfqpoint{2.296982in}{1.202168in}}%
\pgfpathlineto{\pgfqpoint{2.299306in}{1.120157in}}%
\pgfpathlineto{\pgfqpoint{2.301630in}{1.153932in}}%
\pgfpathlineto{\pgfqpoint{2.303954in}{1.142104in}}%
\pgfpathlineto{\pgfqpoint{2.306278in}{1.067837in}}%
\pgfpathlineto{\pgfqpoint{2.308602in}{1.272717in}}%
\pgfpathlineto{\pgfqpoint{2.313250in}{1.135489in}}%
\pgfpathlineto{\pgfqpoint{2.315574in}{1.218500in}}%
\pgfpathlineto{\pgfqpoint{2.317898in}{1.198160in}}%
\pgfpathlineto{\pgfqpoint{2.320222in}{1.251840in}}%
\pgfpathlineto{\pgfqpoint{2.324870in}{1.117142in}}%
\pgfpathlineto{\pgfqpoint{2.327194in}{1.112750in}}%
\pgfpathlineto{\pgfqpoint{2.329518in}{1.139666in}}%
\pgfpathlineto{\pgfqpoint{2.331842in}{1.193281in}}%
\pgfpathlineto{\pgfqpoint{2.334166in}{1.209865in}}%
\pgfpathlineto{\pgfqpoint{2.336490in}{1.253966in}}%
\pgfpathlineto{\pgfqpoint{2.338814in}{1.198972in}}%
\pgfpathlineto{\pgfqpoint{2.341138in}{1.219884in}}%
\pgfpathlineto{\pgfqpoint{2.343462in}{1.160988in}}%
\pgfpathlineto{\pgfqpoint{2.345786in}{1.191957in}}%
\pgfpathlineto{\pgfqpoint{2.348110in}{1.359881in}}%
\pgfpathlineto{\pgfqpoint{2.350434in}{1.192256in}}%
\pgfpathlineto{\pgfqpoint{2.352757in}{1.291534in}}%
\pgfpathlineto{\pgfqpoint{2.355081in}{1.188784in}}%
\pgfpathlineto{\pgfqpoint{2.357405in}{1.345300in}}%
\pgfpathlineto{\pgfqpoint{2.359729in}{1.209898in}}%
\pgfpathlineto{\pgfqpoint{2.362053in}{1.230779in}}%
\pgfpathlineto{\pgfqpoint{2.366701in}{1.316657in}}%
\pgfpathlineto{\pgfqpoint{2.369025in}{1.299899in}}%
\pgfpathlineto{\pgfqpoint{2.371349in}{1.243431in}}%
\pgfpathlineto{\pgfqpoint{2.373673in}{1.381682in}}%
\pgfpathlineto{\pgfqpoint{2.375997in}{1.322827in}}%
\pgfpathlineto{\pgfqpoint{2.378321in}{1.317497in}}%
\pgfpathlineto{\pgfqpoint{2.380645in}{1.278204in}}%
\pgfpathlineto{\pgfqpoint{2.382969in}{1.262736in}}%
\pgfpathlineto{\pgfqpoint{2.385293in}{1.344277in}}%
\pgfpathlineto{\pgfqpoint{2.387617in}{1.284610in}}%
\pgfpathlineto{\pgfqpoint{2.389941in}{1.351254in}}%
\pgfpathlineto{\pgfqpoint{2.392265in}{1.250520in}}%
\pgfpathlineto{\pgfqpoint{2.394589in}{0.808143in}}%
\pgfpathlineto{\pgfqpoint{2.396913in}{0.866793in}}%
\pgfpathlineto{\pgfqpoint{2.399237in}{0.896552in}}%
\pgfpathlineto{\pgfqpoint{2.401561in}{0.800053in}}%
\pgfpathlineto{\pgfqpoint{2.403885in}{0.841080in}}%
\pgfpathlineto{\pgfqpoint{2.406209in}{0.920757in}}%
\pgfpathlineto{\pgfqpoint{2.410857in}{0.778894in}}%
\pgfpathlineto{\pgfqpoint{2.413181in}{0.851965in}}%
\pgfpathlineto{\pgfqpoint{2.415505in}{0.798356in}}%
\pgfpathlineto{\pgfqpoint{2.417829in}{0.860409in}}%
\pgfpathlineto{\pgfqpoint{2.420153in}{0.884285in}}%
\pgfpathlineto{\pgfqpoint{2.422476in}{0.861366in}}%
\pgfpathlineto{\pgfqpoint{2.424800in}{0.918124in}}%
\pgfpathlineto{\pgfqpoint{2.427124in}{0.905829in}}%
\pgfpathlineto{\pgfqpoint{2.429448in}{0.799614in}}%
\pgfpathlineto{\pgfqpoint{2.434096in}{1.019376in}}%
\pgfpathlineto{\pgfqpoint{2.436420in}{0.897985in}}%
\pgfpathlineto{\pgfqpoint{2.438744in}{0.944674in}}%
\pgfpathlineto{\pgfqpoint{2.441068in}{0.860532in}}%
\pgfpathlineto{\pgfqpoint{2.448040in}{0.945774in}}%
\pgfpathlineto{\pgfqpoint{2.450364in}{0.848473in}}%
\pgfpathlineto{\pgfqpoint{2.452688in}{0.837436in}}%
\pgfpathlineto{\pgfqpoint{2.455012in}{0.896257in}}%
\pgfpathlineto{\pgfqpoint{2.457336in}{0.893592in}}%
\pgfpathlineto{\pgfqpoint{2.461984in}{1.068316in}}%
\pgfpathlineto{\pgfqpoint{2.464308in}{0.969366in}}%
\pgfpathlineto{\pgfqpoint{2.466632in}{0.978778in}}%
\pgfpathlineto{\pgfqpoint{2.468956in}{0.960297in}}%
\pgfpathlineto{\pgfqpoint{2.471280in}{0.974580in}}%
\pgfpathlineto{\pgfqpoint{2.473604in}{0.892744in}}%
\pgfpathlineto{\pgfqpoint{2.475928in}{0.959280in}}%
\pgfpathlineto{\pgfqpoint{2.478252in}{0.898752in}}%
\pgfpathlineto{\pgfqpoint{2.480576in}{0.939416in}}%
\pgfpathlineto{\pgfqpoint{2.482900in}{1.025183in}}%
\pgfpathlineto{\pgfqpoint{2.485224in}{0.971925in}}%
\pgfpathlineto{\pgfqpoint{2.487548in}{1.059225in}}%
\pgfpathlineto{\pgfqpoint{2.489872in}{0.882655in}}%
\pgfpathlineto{\pgfqpoint{2.492195in}{0.968239in}}%
\pgfpathlineto{\pgfqpoint{2.494519in}{0.942293in}}%
\pgfpathlineto{\pgfqpoint{2.496843in}{1.059275in}}%
\pgfpathlineto{\pgfqpoint{2.499167in}{0.972851in}}%
\pgfpathlineto{\pgfqpoint{2.501491in}{1.008982in}}%
\pgfpathlineto{\pgfqpoint{2.503815in}{0.956640in}}%
\pgfpathlineto{\pgfqpoint{2.506139in}{0.939868in}}%
\pgfpathlineto{\pgfqpoint{2.508463in}{1.078626in}}%
\pgfpathlineto{\pgfqpoint{2.510787in}{1.043882in}}%
\pgfpathlineto{\pgfqpoint{2.513111in}{1.043095in}}%
\pgfpathlineto{\pgfqpoint{2.515435in}{1.068727in}}%
\pgfpathlineto{\pgfqpoint{2.517759in}{1.064022in}}%
\pgfpathlineto{\pgfqpoint{2.520083in}{1.122037in}}%
\pgfpathlineto{\pgfqpoint{2.522407in}{1.032346in}}%
\pgfpathlineto{\pgfqpoint{2.524731in}{1.064601in}}%
\pgfpathlineto{\pgfqpoint{2.527055in}{1.074161in}}%
\pgfpathlineto{\pgfqpoint{2.529379in}{0.976568in}}%
\pgfpathlineto{\pgfqpoint{2.531703in}{1.107232in}}%
\pgfpathlineto{\pgfqpoint{2.534027in}{0.999555in}}%
\pgfpathlineto{\pgfqpoint{2.536351in}{1.174735in}}%
\pgfpathlineto{\pgfqpoint{2.540999in}{1.028566in}}%
\pgfpathlineto{\pgfqpoint{2.543323in}{1.100148in}}%
\pgfpathlineto{\pgfqpoint{2.545647in}{1.104522in}}%
\pgfpathlineto{\pgfqpoint{2.550295in}{1.164208in}}%
\pgfpathlineto{\pgfqpoint{2.554943in}{1.058742in}}%
\pgfpathlineto{\pgfqpoint{2.557267in}{1.099648in}}%
\pgfpathlineto{\pgfqpoint{2.559591in}{1.072985in}}%
\pgfpathlineto{\pgfqpoint{2.561915in}{1.111886in}}%
\pgfpathlineto{\pgfqpoint{2.564238in}{1.052978in}}%
\pgfpathlineto{\pgfqpoint{2.566562in}{1.190252in}}%
\pgfpathlineto{\pgfqpoint{2.568886in}{1.157293in}}%
\pgfpathlineto{\pgfqpoint{2.571210in}{1.249900in}}%
\pgfpathlineto{\pgfqpoint{2.573534in}{1.129480in}}%
\pgfpathlineto{\pgfqpoint{2.575858in}{1.260936in}}%
\pgfpathlineto{\pgfqpoint{2.578182in}{0.993542in}}%
\pgfpathlineto{\pgfqpoint{2.580506in}{1.175980in}}%
\pgfpathlineto{\pgfqpoint{2.582830in}{1.210095in}}%
\pgfpathlineto{\pgfqpoint{2.585154in}{1.116762in}}%
\pgfpathlineto{\pgfqpoint{2.587478in}{1.174743in}}%
\pgfpathlineto{\pgfqpoint{2.589802in}{1.179866in}}%
\pgfpathlineto{\pgfqpoint{2.592126in}{1.199368in}}%
\pgfpathlineto{\pgfqpoint{2.594450in}{1.142991in}}%
\pgfpathlineto{\pgfqpoint{2.596774in}{1.216675in}}%
\pgfpathlineto{\pgfqpoint{2.599098in}{1.180397in}}%
\pgfpathlineto{\pgfqpoint{2.601422in}{1.105405in}}%
\pgfpathlineto{\pgfqpoint{2.603746in}{1.204727in}}%
\pgfpathlineto{\pgfqpoint{2.606070in}{1.103909in}}%
\pgfpathlineto{\pgfqpoint{2.608394in}{1.142796in}}%
\pgfpathlineto{\pgfqpoint{2.610718in}{1.097368in}}%
\pgfpathlineto{\pgfqpoint{2.617690in}{1.196386in}}%
\pgfpathlineto{\pgfqpoint{2.620014in}{1.166482in}}%
\pgfpathlineto{\pgfqpoint{2.622338in}{1.294897in}}%
\pgfpathlineto{\pgfqpoint{2.624662in}{1.202395in}}%
\pgfpathlineto{\pgfqpoint{2.626986in}{1.259676in}}%
\pgfpathlineto{\pgfqpoint{2.629310in}{1.248638in}}%
\pgfpathlineto{\pgfqpoint{2.631634in}{1.143363in}}%
\pgfpathlineto{\pgfqpoint{2.633957in}{1.298234in}}%
\pgfpathlineto{\pgfqpoint{2.636281in}{1.197864in}}%
\pgfpathlineto{\pgfqpoint{2.638605in}{1.331478in}}%
\pgfpathlineto{\pgfqpoint{2.640929in}{1.249087in}}%
\pgfpathlineto{\pgfqpoint{2.643253in}{1.317618in}}%
\pgfpathlineto{\pgfqpoint{2.645577in}{1.208901in}}%
\pgfpathlineto{\pgfqpoint{2.647901in}{1.284222in}}%
\pgfpathlineto{\pgfqpoint{2.650225in}{1.288110in}}%
\pgfpathlineto{\pgfqpoint{2.652549in}{1.275597in}}%
\pgfpathlineto{\pgfqpoint{2.654873in}{1.236673in}}%
\pgfpathlineto{\pgfqpoint{2.657197in}{1.221419in}}%
\pgfpathlineto{\pgfqpoint{2.659521in}{1.260922in}}%
\pgfpathlineto{\pgfqpoint{2.661845in}{1.264201in}}%
\pgfpathlineto{\pgfqpoint{2.664169in}{1.280833in}}%
\pgfpathlineto{\pgfqpoint{2.666493in}{1.257489in}}%
\pgfpathlineto{\pgfqpoint{2.668817in}{1.343823in}}%
\pgfpathlineto{\pgfqpoint{2.671141in}{1.341952in}}%
\pgfpathlineto{\pgfqpoint{2.673465in}{1.331242in}}%
\pgfpathlineto{\pgfqpoint{2.675789in}{1.292519in}}%
\pgfpathlineto{\pgfqpoint{2.678113in}{1.423002in}}%
\pgfpathlineto{\pgfqpoint{2.680437in}{1.268522in}}%
\pgfpathlineto{\pgfqpoint{2.682761in}{1.312917in}}%
\pgfpathlineto{\pgfqpoint{2.685085in}{0.895722in}}%
\pgfpathlineto{\pgfqpoint{2.687409in}{0.695009in}}%
\pgfpathlineto{\pgfqpoint{2.689733in}{0.834194in}}%
\pgfpathlineto{\pgfqpoint{2.692057in}{0.866024in}}%
\pgfpathlineto{\pgfqpoint{2.694381in}{0.837565in}}%
\pgfpathlineto{\pgfqpoint{2.696705in}{0.824866in}}%
\pgfpathlineto{\pgfqpoint{2.701353in}{0.915673in}}%
\pgfpathlineto{\pgfqpoint{2.703676in}{0.925782in}}%
\pgfpathlineto{\pgfqpoint{2.708324in}{0.859015in}}%
\pgfpathlineto{\pgfqpoint{2.710648in}{0.882765in}}%
\pgfpathlineto{\pgfqpoint{2.715296in}{0.898638in}}%
\pgfpathlineto{\pgfqpoint{2.717620in}{0.952290in}}%
\pgfpathlineto{\pgfqpoint{2.719944in}{0.947062in}}%
\pgfpathlineto{\pgfqpoint{2.722268in}{0.859880in}}%
\pgfpathlineto{\pgfqpoint{2.724592in}{0.975607in}}%
\pgfpathlineto{\pgfqpoint{2.726916in}{0.904224in}}%
\pgfpathlineto{\pgfqpoint{2.729240in}{0.987176in}}%
\pgfpathlineto{\pgfqpoint{2.731564in}{0.863362in}}%
\pgfpathlineto{\pgfqpoint{2.733888in}{0.983212in}}%
\pgfpathlineto{\pgfqpoint{2.736212in}{0.907592in}}%
\pgfpathlineto{\pgfqpoint{2.743184in}{1.006915in}}%
\pgfpathlineto{\pgfqpoint{2.745508in}{0.960538in}}%
\pgfpathlineto{\pgfqpoint{2.747832in}{0.948672in}}%
\pgfpathlineto{\pgfqpoint{2.750156in}{0.896085in}}%
\pgfpathlineto{\pgfqpoint{2.752480in}{0.930490in}}%
\pgfpathlineto{\pgfqpoint{2.754804in}{0.928169in}}%
\pgfpathlineto{\pgfqpoint{2.757128in}{0.978450in}}%
\pgfpathlineto{\pgfqpoint{2.759452in}{1.097237in}}%
\pgfpathlineto{\pgfqpoint{2.764100in}{0.879587in}}%
\pgfpathlineto{\pgfqpoint{2.766424in}{0.982131in}}%
\pgfpathlineto{\pgfqpoint{2.768748in}{0.841282in}}%
\pgfpathlineto{\pgfqpoint{2.771072in}{0.963304in}}%
\pgfpathlineto{\pgfqpoint{2.773395in}{0.929774in}}%
\pgfpathlineto{\pgfqpoint{2.775719in}{1.015199in}}%
\pgfpathlineto{\pgfqpoint{2.778043in}{0.992755in}}%
\pgfpathlineto{\pgfqpoint{2.780367in}{1.014433in}}%
\pgfpathlineto{\pgfqpoint{2.782691in}{1.073135in}}%
\pgfpathlineto{\pgfqpoint{2.785015in}{1.035817in}}%
\pgfpathlineto{\pgfqpoint{2.787339in}{1.101023in}}%
\pgfpathlineto{\pgfqpoint{2.789663in}{0.937777in}}%
\pgfpathlineto{\pgfqpoint{2.791987in}{1.072196in}}%
\pgfpathlineto{\pgfqpoint{2.794311in}{1.084102in}}%
\pgfpathlineto{\pgfqpoint{2.796635in}{1.056144in}}%
\pgfpathlineto{\pgfqpoint{2.798959in}{1.010535in}}%
\pgfpathlineto{\pgfqpoint{2.801283in}{1.030292in}}%
\pgfpathlineto{\pgfqpoint{2.803607in}{1.087321in}}%
\pgfpathlineto{\pgfqpoint{2.805931in}{1.115952in}}%
\pgfpathlineto{\pgfqpoint{2.808255in}{1.022453in}}%
\pgfpathlineto{\pgfqpoint{2.810579in}{1.030484in}}%
\pgfpathlineto{\pgfqpoint{2.812903in}{1.202031in}}%
\pgfpathlineto{\pgfqpoint{2.815227in}{1.037506in}}%
\pgfpathlineto{\pgfqpoint{2.817551in}{1.011562in}}%
\pgfpathlineto{\pgfqpoint{2.819875in}{1.040977in}}%
\pgfpathlineto{\pgfqpoint{2.822199in}{1.100730in}}%
\pgfpathlineto{\pgfqpoint{2.824523in}{1.021986in}}%
\pgfpathlineto{\pgfqpoint{2.826847in}{1.018610in}}%
\pgfpathlineto{\pgfqpoint{2.831495in}{1.111612in}}%
\pgfpathlineto{\pgfqpoint{2.833819in}{1.113715in}}%
\pgfpathlineto{\pgfqpoint{2.836143in}{1.142912in}}%
\pgfpathlineto{\pgfqpoint{2.838467in}{1.215914in}}%
\pgfpathlineto{\pgfqpoint{2.840791in}{1.128854in}}%
\pgfpathlineto{\pgfqpoint{2.845438in}{1.148298in}}%
\pgfpathlineto{\pgfqpoint{2.847762in}{1.159377in}}%
\pgfpathlineto{\pgfqpoint{2.850086in}{1.141880in}}%
\pgfpathlineto{\pgfqpoint{2.852410in}{1.157266in}}%
\pgfpathlineto{\pgfqpoint{2.854734in}{1.077507in}}%
\pgfpathlineto{\pgfqpoint{2.857058in}{1.063062in}}%
\pgfpathlineto{\pgfqpoint{2.859382in}{1.113813in}}%
\pgfpathlineto{\pgfqpoint{2.864030in}{1.148470in}}%
\pgfpathlineto{\pgfqpoint{2.866354in}{1.145089in}}%
\pgfpathlineto{\pgfqpoint{2.868678in}{1.235534in}}%
\pgfpathlineto{\pgfqpoint{2.871002in}{1.244043in}}%
\pgfpathlineto{\pgfqpoint{2.873326in}{1.189477in}}%
\pgfpathlineto{\pgfqpoint{2.875650in}{1.256095in}}%
\pgfpathlineto{\pgfqpoint{2.877974in}{1.117327in}}%
\pgfpathlineto{\pgfqpoint{2.884946in}{1.210644in}}%
\pgfpathlineto{\pgfqpoint{2.887270in}{1.199660in}}%
\pgfpathlineto{\pgfqpoint{2.889594in}{1.126892in}}%
\pgfpathlineto{\pgfqpoint{2.891918in}{1.150142in}}%
\pgfpathlineto{\pgfqpoint{2.894242in}{1.273418in}}%
\pgfpathlineto{\pgfqpoint{2.896566in}{1.197775in}}%
\pgfpathlineto{\pgfqpoint{2.898890in}{1.220598in}}%
\pgfpathlineto{\pgfqpoint{2.901214in}{1.229887in}}%
\pgfpathlineto{\pgfqpoint{2.903538in}{1.150254in}}%
\pgfpathlineto{\pgfqpoint{2.905862in}{1.275269in}}%
\pgfpathlineto{\pgfqpoint{2.908186in}{1.210550in}}%
\pgfpathlineto{\pgfqpoint{2.910510in}{1.322400in}}%
\pgfpathlineto{\pgfqpoint{2.915157in}{1.220645in}}%
\pgfpathlineto{\pgfqpoint{2.917481in}{1.178783in}}%
\pgfpathlineto{\pgfqpoint{2.919805in}{1.301392in}}%
\pgfpathlineto{\pgfqpoint{2.922129in}{1.228677in}}%
\pgfpathlineto{\pgfqpoint{2.924453in}{1.285305in}}%
\pgfpathlineto{\pgfqpoint{2.926777in}{1.196299in}}%
\pgfpathlineto{\pgfqpoint{2.929101in}{1.316413in}}%
\pgfpathlineto{\pgfqpoint{2.931425in}{1.331356in}}%
\pgfpathlineto{\pgfqpoint{2.933749in}{1.268776in}}%
\pgfpathlineto{\pgfqpoint{2.936073in}{1.309554in}}%
\pgfpathlineto{\pgfqpoint{2.938397in}{1.228442in}}%
\pgfpathlineto{\pgfqpoint{2.940721in}{1.204100in}}%
\pgfpathlineto{\pgfqpoint{2.943045in}{1.313510in}}%
\pgfpathlineto{\pgfqpoint{2.945369in}{1.351375in}}%
\pgfpathlineto{\pgfqpoint{2.947693in}{1.308297in}}%
\pgfpathlineto{\pgfqpoint{2.950017in}{1.225132in}}%
\pgfpathlineto{\pgfqpoint{2.952341in}{1.322094in}}%
\pgfpathlineto{\pgfqpoint{2.954665in}{1.277255in}}%
\pgfpathlineto{\pgfqpoint{2.956989in}{1.278274in}}%
\pgfpathlineto{\pgfqpoint{2.959313in}{1.261210in}}%
\pgfpathlineto{\pgfqpoint{2.961637in}{1.252513in}}%
\pgfpathlineto{\pgfqpoint{2.966285in}{1.393878in}}%
\pgfpathlineto{\pgfqpoint{2.968609in}{1.263684in}}%
\pgfpathlineto{\pgfqpoint{2.970933in}{1.438257in}}%
\pgfpathlineto{\pgfqpoint{2.973257in}{1.453485in}}%
\pgfpathlineto{\pgfqpoint{2.975581in}{0.902542in}}%
\pgfpathlineto{\pgfqpoint{2.977905in}{0.879650in}}%
\pgfpathlineto{\pgfqpoint{2.980229in}{0.897973in}}%
\pgfpathlineto{\pgfqpoint{2.982553in}{0.825894in}}%
\pgfpathlineto{\pgfqpoint{2.984876in}{0.837107in}}%
\pgfpathlineto{\pgfqpoint{2.987200in}{0.912400in}}%
\pgfpathlineto{\pgfqpoint{2.989524in}{0.783726in}}%
\pgfpathlineto{\pgfqpoint{2.991848in}{0.786813in}}%
\pgfpathlineto{\pgfqpoint{2.994172in}{0.905326in}}%
\pgfpathlineto{\pgfqpoint{2.996496in}{0.847379in}}%
\pgfpathlineto{\pgfqpoint{2.998820in}{0.934046in}}%
\pgfpathlineto{\pgfqpoint{3.001144in}{0.912594in}}%
\pgfpathlineto{\pgfqpoint{3.003468in}{1.002176in}}%
\pgfpathlineto{\pgfqpoint{3.005792in}{0.931393in}}%
\pgfpathlineto{\pgfqpoint{3.008116in}{0.982463in}}%
\pgfpathlineto{\pgfqpoint{3.010440in}{0.976229in}}%
\pgfpathlineto{\pgfqpoint{3.012764in}{0.951147in}}%
\pgfpathlineto{\pgfqpoint{3.015088in}{0.838648in}}%
\pgfpathlineto{\pgfqpoint{3.017412in}{0.947073in}}%
\pgfpathlineto{\pgfqpoint{3.019736in}{0.894435in}}%
\pgfpathlineto{\pgfqpoint{3.022060in}{0.917646in}}%
\pgfpathlineto{\pgfqpoint{3.024384in}{0.952238in}}%
\pgfpathlineto{\pgfqpoint{3.026708in}{0.813318in}}%
\pgfpathlineto{\pgfqpoint{3.029032in}{0.951783in}}%
\pgfpathlineto{\pgfqpoint{3.031356in}{1.017225in}}%
\pgfpathlineto{\pgfqpoint{3.033680in}{0.794073in}}%
\pgfpathlineto{\pgfqpoint{3.036004in}{0.955389in}}%
\pgfpathlineto{\pgfqpoint{3.038328in}{0.936625in}}%
\pgfpathlineto{\pgfqpoint{3.042976in}{0.965351in}}%
\pgfpathlineto{\pgfqpoint{3.045300in}{0.897019in}}%
\pgfpathlineto{\pgfqpoint{3.047624in}{0.977444in}}%
\pgfpathlineto{\pgfqpoint{3.049948in}{0.973357in}}%
\pgfpathlineto{\pgfqpoint{3.052272in}{0.991051in}}%
\pgfpathlineto{\pgfqpoint{3.054596in}{1.022815in}}%
\pgfpathlineto{\pgfqpoint{3.056919in}{0.988364in}}%
\pgfpathlineto{\pgfqpoint{3.059243in}{1.032508in}}%
\pgfpathlineto{\pgfqpoint{3.061567in}{0.988209in}}%
\pgfpathlineto{\pgfqpoint{3.063891in}{1.075184in}}%
\pgfpathlineto{\pgfqpoint{3.068539in}{0.997916in}}%
\pgfpathlineto{\pgfqpoint{3.070863in}{0.892469in}}%
\pgfpathlineto{\pgfqpoint{3.073187in}{1.032009in}}%
\pgfpathlineto{\pgfqpoint{3.075511in}{1.024732in}}%
\pgfpathlineto{\pgfqpoint{3.077835in}{1.119828in}}%
\pgfpathlineto{\pgfqpoint{3.080159in}{0.994483in}}%
\pgfpathlineto{\pgfqpoint{3.082483in}{0.959186in}}%
\pgfpathlineto{\pgfqpoint{3.084807in}{0.949405in}}%
\pgfpathlineto{\pgfqpoint{3.087131in}{1.084571in}}%
\pgfpathlineto{\pgfqpoint{3.089455in}{1.090782in}}%
\pgfpathlineto{\pgfqpoint{3.091779in}{1.004569in}}%
\pgfpathlineto{\pgfqpoint{3.096427in}{1.101922in}}%
\pgfpathlineto{\pgfqpoint{3.098751in}{1.076715in}}%
\pgfpathlineto{\pgfqpoint{3.101075in}{1.165761in}}%
\pgfpathlineto{\pgfqpoint{3.103399in}{1.043912in}}%
\pgfpathlineto{\pgfqpoint{3.105723in}{1.071300in}}%
\pgfpathlineto{\pgfqpoint{3.108047in}{1.061559in}}%
\pgfpathlineto{\pgfqpoint{3.110371in}{0.995065in}}%
\pgfpathlineto{\pgfqpoint{3.112695in}{1.126641in}}%
\pgfpathlineto{\pgfqpoint{3.115019in}{0.952675in}}%
\pgfpathlineto{\pgfqpoint{3.117343in}{1.194619in}}%
\pgfpathlineto{\pgfqpoint{3.119667in}{1.086511in}}%
\pgfpathlineto{\pgfqpoint{3.121991in}{1.065953in}}%
\pgfpathlineto{\pgfqpoint{3.124315in}{1.090071in}}%
\pgfpathlineto{\pgfqpoint{3.126638in}{1.145533in}}%
\pgfpathlineto{\pgfqpoint{3.128962in}{1.084311in}}%
\pgfpathlineto{\pgfqpoint{3.131286in}{1.168653in}}%
\pgfpathlineto{\pgfqpoint{3.133610in}{1.101056in}}%
\pgfpathlineto{\pgfqpoint{3.135934in}{1.174798in}}%
\pgfpathlineto{\pgfqpoint{3.138258in}{1.159425in}}%
\pgfpathlineto{\pgfqpoint{3.140582in}{1.156732in}}%
\pgfpathlineto{\pgfqpoint{3.142906in}{1.094490in}}%
\pgfpathlineto{\pgfqpoint{3.145230in}{1.227565in}}%
\pgfpathlineto{\pgfqpoint{3.147554in}{1.127622in}}%
\pgfpathlineto{\pgfqpoint{3.149878in}{1.150280in}}%
\pgfpathlineto{\pgfqpoint{3.152202in}{1.129509in}}%
\pgfpathlineto{\pgfqpoint{3.154526in}{1.142421in}}%
\pgfpathlineto{\pgfqpoint{3.156850in}{1.229588in}}%
\pgfpathlineto{\pgfqpoint{3.159174in}{1.192553in}}%
\pgfpathlineto{\pgfqpoint{3.161498in}{1.062845in}}%
\pgfpathlineto{\pgfqpoint{3.163822in}{1.235694in}}%
\pgfpathlineto{\pgfqpoint{3.166146in}{1.141434in}}%
\pgfpathlineto{\pgfqpoint{3.168470in}{1.159989in}}%
\pgfpathlineto{\pgfqpoint{3.170794in}{1.167525in}}%
\pgfpathlineto{\pgfqpoint{3.173118in}{1.095429in}}%
\pgfpathlineto{\pgfqpoint{3.175442in}{1.255818in}}%
\pgfpathlineto{\pgfqpoint{3.177766in}{1.233952in}}%
\pgfpathlineto{\pgfqpoint{3.182414in}{1.327350in}}%
\pgfpathlineto{\pgfqpoint{3.184738in}{1.303573in}}%
\pgfpathlineto{\pgfqpoint{3.187062in}{1.072157in}}%
\pgfpathlineto{\pgfqpoint{3.189386in}{1.199890in}}%
\pgfpathlineto{\pgfqpoint{3.191710in}{1.142868in}}%
\pgfpathlineto{\pgfqpoint{3.198681in}{1.229088in}}%
\pgfpathlineto{\pgfqpoint{3.201005in}{1.209919in}}%
\pgfpathlineto{\pgfqpoint{3.203329in}{1.283482in}}%
\pgfpathlineto{\pgfqpoint{3.207977in}{1.209738in}}%
\pgfpathlineto{\pgfqpoint{3.210301in}{1.286312in}}%
\pgfpathlineto{\pgfqpoint{3.212625in}{1.300498in}}%
\pgfpathlineto{\pgfqpoint{3.214949in}{1.222139in}}%
\pgfpathlineto{\pgfqpoint{3.217273in}{1.200889in}}%
\pgfpathlineto{\pgfqpoint{3.221921in}{1.255672in}}%
\pgfpathlineto{\pgfqpoint{3.224245in}{1.222503in}}%
\pgfpathlineto{\pgfqpoint{3.226569in}{1.222061in}}%
\pgfpathlineto{\pgfqpoint{3.228893in}{1.303006in}}%
\pgfpathlineto{\pgfqpoint{3.231217in}{1.246665in}}%
\pgfpathlineto{\pgfqpoint{3.235865in}{1.329102in}}%
\pgfpathlineto{\pgfqpoint{3.238189in}{1.330109in}}%
\pgfpathlineto{\pgfqpoint{3.240513in}{1.273007in}}%
\pgfpathlineto{\pgfqpoint{3.242837in}{1.286428in}}%
\pgfpathlineto{\pgfqpoint{3.245161in}{1.378483in}}%
\pgfpathlineto{\pgfqpoint{3.247485in}{1.365444in}}%
\pgfpathlineto{\pgfqpoint{3.252133in}{1.279053in}}%
\pgfpathlineto{\pgfqpoint{3.254457in}{1.445037in}}%
\pgfpathlineto{\pgfqpoint{3.256781in}{1.251016in}}%
\pgfpathlineto{\pgfqpoint{3.259105in}{1.368896in}}%
\pgfpathlineto{\pgfqpoint{3.261429in}{1.429030in}}%
\pgfpathlineto{\pgfqpoint{3.263753in}{1.288607in}}%
\pgfpathlineto{\pgfqpoint{3.266076in}{0.892073in}}%
\pgfpathlineto{\pgfqpoint{3.268400in}{0.797763in}}%
\pgfpathlineto{\pgfqpoint{3.273048in}{0.880768in}}%
\pgfpathlineto{\pgfqpoint{3.275372in}{0.768061in}}%
\pgfpathlineto{\pgfqpoint{3.277696in}{0.850350in}}%
\pgfpathlineto{\pgfqpoint{3.280020in}{0.809051in}}%
\pgfpathlineto{\pgfqpoint{3.282344in}{0.861491in}}%
\pgfpathlineto{\pgfqpoint{3.284668in}{0.824500in}}%
\pgfpathlineto{\pgfqpoint{3.286992in}{0.883013in}}%
\pgfpathlineto{\pgfqpoint{3.289316in}{0.898376in}}%
\pgfpathlineto{\pgfqpoint{3.291640in}{0.958703in}}%
\pgfpathlineto{\pgfqpoint{3.293964in}{0.818945in}}%
\pgfpathlineto{\pgfqpoint{3.296288in}{0.974918in}}%
\pgfpathlineto{\pgfqpoint{3.303260in}{0.899839in}}%
\pgfpathlineto{\pgfqpoint{3.305584in}{0.897291in}}%
\pgfpathlineto{\pgfqpoint{3.307908in}{0.896474in}}%
\pgfpathlineto{\pgfqpoint{3.310232in}{0.890086in}}%
\pgfpathlineto{\pgfqpoint{3.312556in}{0.869572in}}%
\pgfpathlineto{\pgfqpoint{3.314880in}{0.886394in}}%
\pgfpathlineto{\pgfqpoint{3.317204in}{0.988728in}}%
\pgfpathlineto{\pgfqpoint{3.319528in}{0.912593in}}%
\pgfpathlineto{\pgfqpoint{3.321852in}{0.903713in}}%
\pgfpathlineto{\pgfqpoint{3.324176in}{0.993418in}}%
\pgfpathlineto{\pgfqpoint{3.326500in}{0.916139in}}%
\pgfpathlineto{\pgfqpoint{3.328824in}{1.097400in}}%
\pgfpathlineto{\pgfqpoint{3.331148in}{0.870942in}}%
\pgfpathlineto{\pgfqpoint{3.333472in}{0.989759in}}%
\pgfpathlineto{\pgfqpoint{3.338119in}{0.920773in}}%
\pgfpathlineto{\pgfqpoint{3.342767in}{1.042187in}}%
\pgfpathlineto{\pgfqpoint{3.345091in}{0.917166in}}%
\pgfpathlineto{\pgfqpoint{3.347415in}{0.950549in}}%
\pgfpathlineto{\pgfqpoint{3.349739in}{1.012785in}}%
\pgfpathlineto{\pgfqpoint{3.352063in}{1.018624in}}%
\pgfpathlineto{\pgfqpoint{3.354387in}{0.970710in}}%
\pgfpathlineto{\pgfqpoint{3.356711in}{1.267107in}}%
\pgfpathlineto{\pgfqpoint{3.361359in}{0.904346in}}%
\pgfpathlineto{\pgfqpoint{3.363683in}{1.034674in}}%
\pgfpathlineto{\pgfqpoint{3.366007in}{0.999561in}}%
\pgfpathlineto{\pgfqpoint{3.368331in}{0.995511in}}%
\pgfpathlineto{\pgfqpoint{3.370655in}{1.002262in}}%
\pgfpathlineto{\pgfqpoint{3.372979in}{1.031035in}}%
\pgfpathlineto{\pgfqpoint{3.375303in}{1.034363in}}%
\pgfpathlineto{\pgfqpoint{3.377627in}{1.004102in}}%
\pgfpathlineto{\pgfqpoint{3.382275in}{1.058744in}}%
\pgfpathlineto{\pgfqpoint{3.384599in}{1.077438in}}%
\pgfpathlineto{\pgfqpoint{3.389247in}{1.163004in}}%
\pgfpathlineto{\pgfqpoint{3.391571in}{1.095127in}}%
\pgfpathlineto{\pgfqpoint{3.393895in}{1.114424in}}%
\pgfpathlineto{\pgfqpoint{3.396219in}{1.112553in}}%
\pgfpathlineto{\pgfqpoint{3.398543in}{1.080891in}}%
\pgfpathlineto{\pgfqpoint{3.400867in}{1.072370in}}%
\pgfpathlineto{\pgfqpoint{3.403191in}{1.055279in}}%
\pgfpathlineto{\pgfqpoint{3.405515in}{1.176334in}}%
\pgfpathlineto{\pgfqpoint{3.407838in}{1.234813in}}%
\pgfpathlineto{\pgfqpoint{3.410162in}{1.142160in}}%
\pgfpathlineto{\pgfqpoint{3.412486in}{1.153661in}}%
\pgfpathlineto{\pgfqpoint{3.414810in}{1.130529in}}%
\pgfpathlineto{\pgfqpoint{3.419458in}{1.046917in}}%
\pgfpathlineto{\pgfqpoint{3.421782in}{0.994214in}}%
\pgfpathlineto{\pgfqpoint{3.424106in}{1.140678in}}%
\pgfpathlineto{\pgfqpoint{3.426430in}{1.128098in}}%
\pgfpathlineto{\pgfqpoint{3.428754in}{1.095244in}}%
\pgfpathlineto{\pgfqpoint{3.431078in}{1.091737in}}%
\pgfpathlineto{\pgfqpoint{3.433402in}{1.049896in}}%
\pgfpathlineto{\pgfqpoint{3.435726in}{1.217731in}}%
\pgfpathlineto{\pgfqpoint{3.438050in}{1.062010in}}%
\pgfpathlineto{\pgfqpoint{3.440374in}{1.237274in}}%
\pgfpathlineto{\pgfqpoint{3.442698in}{1.297354in}}%
\pgfpathlineto{\pgfqpoint{3.445022in}{1.167038in}}%
\pgfpathlineto{\pgfqpoint{3.447346in}{1.187472in}}%
\pgfpathlineto{\pgfqpoint{3.449670in}{1.102836in}}%
\pgfpathlineto{\pgfqpoint{3.451994in}{1.122224in}}%
\pgfpathlineto{\pgfqpoint{3.454318in}{1.183508in}}%
\pgfpathlineto{\pgfqpoint{3.456642in}{1.158072in}}%
\pgfpathlineto{\pgfqpoint{3.458966in}{1.261462in}}%
\pgfpathlineto{\pgfqpoint{3.461290in}{1.309088in}}%
\pgfpathlineto{\pgfqpoint{3.463614in}{1.179258in}}%
\pgfpathlineto{\pgfqpoint{3.465938in}{1.215689in}}%
\pgfpathlineto{\pgfqpoint{3.468262in}{1.212336in}}%
\pgfpathlineto{\pgfqpoint{3.470586in}{1.124990in}}%
\pgfpathlineto{\pgfqpoint{3.472910in}{1.161384in}}%
\pgfpathlineto{\pgfqpoint{3.475234in}{1.303353in}}%
\pgfpathlineto{\pgfqpoint{3.477557in}{1.168870in}}%
\pgfpathlineto{\pgfqpoint{3.479881in}{1.323835in}}%
\pgfpathlineto{\pgfqpoint{3.482205in}{1.244167in}}%
\pgfpathlineto{\pgfqpoint{3.484529in}{1.305894in}}%
\pgfpathlineto{\pgfqpoint{3.486853in}{1.176330in}}%
\pgfpathlineto{\pgfqpoint{3.489177in}{1.214745in}}%
\pgfpathlineto{\pgfqpoint{3.491501in}{1.282663in}}%
\pgfpathlineto{\pgfqpoint{3.493825in}{1.198531in}}%
\pgfpathlineto{\pgfqpoint{3.496149in}{1.231053in}}%
\pgfpathlineto{\pgfqpoint{3.498473in}{1.173804in}}%
\pgfpathlineto{\pgfqpoint{3.500797in}{1.279391in}}%
\pgfpathlineto{\pgfqpoint{3.503121in}{1.236143in}}%
\pgfpathlineto{\pgfqpoint{3.507769in}{1.275080in}}%
\pgfpathlineto{\pgfqpoint{3.510093in}{1.135657in}}%
\pgfpathlineto{\pgfqpoint{3.512417in}{1.294516in}}%
\pgfpathlineto{\pgfqpoint{3.514741in}{1.364281in}}%
\pgfpathlineto{\pgfqpoint{3.517065in}{1.339599in}}%
\pgfpathlineto{\pgfqpoint{3.519389in}{1.226412in}}%
\pgfpathlineto{\pgfqpoint{3.524037in}{1.442221in}}%
\pgfpathlineto{\pgfqpoint{3.526361in}{1.334595in}}%
\pgfpathlineto{\pgfqpoint{3.528685in}{1.300158in}}%
\pgfpathlineto{\pgfqpoint{3.531009in}{1.330533in}}%
\pgfpathlineto{\pgfqpoint{3.533333in}{1.318419in}}%
\pgfpathlineto{\pgfqpoint{3.535657in}{1.405580in}}%
\pgfpathlineto{\pgfqpoint{3.540305in}{1.359672in}}%
\pgfpathlineto{\pgfqpoint{3.542629in}{1.250951in}}%
\pgfpathlineto{\pgfqpoint{3.544953in}{1.365833in}}%
\pgfpathlineto{\pgfqpoint{3.547277in}{1.296820in}}%
\pgfpathlineto{\pgfqpoint{3.549600in}{1.361958in}}%
\pgfpathlineto{\pgfqpoint{3.551924in}{1.221492in}}%
\pgfpathlineto{\pgfqpoint{3.554248in}{1.328222in}}%
\pgfpathlineto{\pgfqpoint{3.556572in}{0.845401in}}%
\pgfpathlineto{\pgfqpoint{3.558896in}{0.873688in}}%
\pgfpathlineto{\pgfqpoint{3.561220in}{0.938253in}}%
\pgfpathlineto{\pgfqpoint{3.565868in}{0.774553in}}%
\pgfpathlineto{\pgfqpoint{3.568192in}{0.865451in}}%
\pgfpathlineto{\pgfqpoint{3.570516in}{0.823156in}}%
\pgfpathlineto{\pgfqpoint{3.572840in}{0.831424in}}%
\pgfpathlineto{\pgfqpoint{3.575164in}{0.767708in}}%
\pgfpathlineto{\pgfqpoint{3.577488in}{0.947320in}}%
\pgfpathlineto{\pgfqpoint{3.579812in}{0.859826in}}%
\pgfpathlineto{\pgfqpoint{3.582136in}{0.834770in}}%
\pgfpathlineto{\pgfqpoint{3.584460in}{0.935241in}}%
\pgfpathlineto{\pgfqpoint{3.586784in}{0.887506in}}%
\pgfpathlineto{\pgfqpoint{3.589108in}{0.983316in}}%
\pgfpathlineto{\pgfqpoint{3.591432in}{0.953638in}}%
\pgfpathlineto{\pgfqpoint{3.593756in}{0.960092in}}%
\pgfpathlineto{\pgfqpoint{3.596080in}{0.853490in}}%
\pgfpathlineto{\pgfqpoint{3.598404in}{0.864826in}}%
\pgfpathlineto{\pgfqpoint{3.600728in}{1.008682in}}%
\pgfpathlineto{\pgfqpoint{3.605376in}{0.882836in}}%
\pgfpathlineto{\pgfqpoint{3.607700in}{0.924564in}}%
\pgfpathlineto{\pgfqpoint{3.610024in}{0.899312in}}%
\pgfpathlineto{\pgfqpoint{3.612348in}{1.017580in}}%
\pgfpathlineto{\pgfqpoint{3.616996in}{0.848047in}}%
\pgfpathlineto{\pgfqpoint{3.619319in}{0.994356in}}%
\pgfpathlineto{\pgfqpoint{3.621643in}{0.939596in}}%
\pgfpathlineto{\pgfqpoint{3.623967in}{0.972430in}}%
\pgfpathlineto{\pgfqpoint{3.626291in}{1.077038in}}%
\pgfpathlineto{\pgfqpoint{3.628615in}{0.987344in}}%
\pgfpathlineto{\pgfqpoint{3.635587in}{0.940231in}}%
\pgfpathlineto{\pgfqpoint{3.637911in}{0.978789in}}%
\pgfpathlineto{\pgfqpoint{3.640235in}{0.939946in}}%
\pgfpathlineto{\pgfqpoint{3.642559in}{1.015843in}}%
\pgfpathlineto{\pgfqpoint{3.644883in}{1.013344in}}%
\pgfpathlineto{\pgfqpoint{3.647207in}{1.056660in}}%
\pgfpathlineto{\pgfqpoint{3.649531in}{0.971339in}}%
\pgfpathlineto{\pgfqpoint{3.654179in}{1.042881in}}%
\pgfpathlineto{\pgfqpoint{3.656503in}{1.083569in}}%
\pgfpathlineto{\pgfqpoint{3.658827in}{1.050433in}}%
\pgfpathlineto{\pgfqpoint{3.661151in}{0.918480in}}%
\pgfpathlineto{\pgfqpoint{3.663475in}{1.089894in}}%
\pgfpathlineto{\pgfqpoint{3.665799in}{1.039082in}}%
\pgfpathlineto{\pgfqpoint{3.668123in}{1.061078in}}%
\pgfpathlineto{\pgfqpoint{3.670447in}{1.021115in}}%
\pgfpathlineto{\pgfqpoint{3.672771in}{1.109791in}}%
\pgfpathlineto{\pgfqpoint{3.675095in}{1.142224in}}%
\pgfpathlineto{\pgfqpoint{3.677419in}{0.981117in}}%
\pgfpathlineto{\pgfqpoint{3.679743in}{0.970289in}}%
\pgfpathlineto{\pgfqpoint{3.682067in}{1.111186in}}%
\pgfpathlineto{\pgfqpoint{3.684391in}{1.028306in}}%
\pgfpathlineto{\pgfqpoint{3.686715in}{1.043143in}}%
\pgfpathlineto{\pgfqpoint{3.689038in}{1.103067in}}%
\pgfpathlineto{\pgfqpoint{3.691362in}{1.061066in}}%
\pgfpathlineto{\pgfqpoint{3.693686in}{1.049147in}}%
\pgfpathlineto{\pgfqpoint{3.696010in}{1.126323in}}%
\pgfpathlineto{\pgfqpoint{3.698334in}{1.125538in}}%
\pgfpathlineto{\pgfqpoint{3.700658in}{1.163000in}}%
\pgfpathlineto{\pgfqpoint{3.702982in}{1.015110in}}%
\pgfpathlineto{\pgfqpoint{3.705306in}{1.048172in}}%
\pgfpathlineto{\pgfqpoint{3.707630in}{1.046992in}}%
\pgfpathlineto{\pgfqpoint{3.712278in}{1.237670in}}%
\pgfpathlineto{\pgfqpoint{3.716926in}{1.061370in}}%
\pgfpathlineto{\pgfqpoint{3.719250in}{1.082981in}}%
\pgfpathlineto{\pgfqpoint{3.721574in}{1.089945in}}%
\pgfpathlineto{\pgfqpoint{3.723898in}{1.117292in}}%
\pgfpathlineto{\pgfqpoint{3.726222in}{1.121246in}}%
\pgfpathlineto{\pgfqpoint{3.728546in}{1.169805in}}%
\pgfpathlineto{\pgfqpoint{3.733194in}{1.091060in}}%
\pgfpathlineto{\pgfqpoint{3.735518in}{1.172656in}}%
\pgfpathlineto{\pgfqpoint{3.737842in}{1.163983in}}%
\pgfpathlineto{\pgfqpoint{3.740166in}{1.083882in}}%
\pgfpathlineto{\pgfqpoint{3.742490in}{1.234271in}}%
\pgfpathlineto{\pgfqpoint{3.744814in}{1.173481in}}%
\pgfpathlineto{\pgfqpoint{3.747138in}{1.184590in}}%
\pgfpathlineto{\pgfqpoint{3.749462in}{1.157652in}}%
\pgfpathlineto{\pgfqpoint{3.751786in}{1.179797in}}%
\pgfpathlineto{\pgfqpoint{3.754110in}{1.102543in}}%
\pgfpathlineto{\pgfqpoint{3.758757in}{1.257016in}}%
\pgfpathlineto{\pgfqpoint{3.761081in}{1.179935in}}%
\pgfpathlineto{\pgfqpoint{3.768053in}{1.229857in}}%
\pgfpathlineto{\pgfqpoint{3.770377in}{1.321741in}}%
\pgfpathlineto{\pgfqpoint{3.772701in}{1.203385in}}%
\pgfpathlineto{\pgfqpoint{3.775025in}{1.326720in}}%
\pgfpathlineto{\pgfqpoint{3.777349in}{1.215115in}}%
\pgfpathlineto{\pgfqpoint{3.781997in}{1.155612in}}%
\pgfpathlineto{\pgfqpoint{3.784321in}{1.171795in}}%
\pgfpathlineto{\pgfqpoint{3.786645in}{1.207507in}}%
\pgfpathlineto{\pgfqpoint{3.788969in}{1.174771in}}%
\pgfpathlineto{\pgfqpoint{3.791293in}{1.069506in}}%
\pgfpathlineto{\pgfqpoint{3.793617in}{1.256992in}}%
\pgfpathlineto{\pgfqpoint{3.795941in}{1.183452in}}%
\pgfpathlineto{\pgfqpoint{3.798265in}{1.215983in}}%
\pgfpathlineto{\pgfqpoint{3.800589in}{1.338638in}}%
\pgfpathlineto{\pgfqpoint{3.802913in}{1.284918in}}%
\pgfpathlineto{\pgfqpoint{3.805237in}{1.169217in}}%
\pgfpathlineto{\pgfqpoint{3.807561in}{1.288833in}}%
\pgfpathlineto{\pgfqpoint{3.809885in}{1.238437in}}%
\pgfpathlineto{\pgfqpoint{3.812209in}{1.234978in}}%
\pgfpathlineto{\pgfqpoint{3.814533in}{1.300872in}}%
\pgfpathlineto{\pgfqpoint{3.816857in}{1.236358in}}%
\pgfpathlineto{\pgfqpoint{3.819181in}{1.273160in}}%
\pgfpathlineto{\pgfqpoint{3.821505in}{1.339391in}}%
\pgfpathlineto{\pgfqpoint{3.823829in}{1.227528in}}%
\pgfpathlineto{\pgfqpoint{3.828477in}{1.370307in}}%
\pgfpathlineto{\pgfqpoint{3.833124in}{1.222778in}}%
\pgfpathlineto{\pgfqpoint{3.835448in}{1.302838in}}%
\pgfpathlineto{\pgfqpoint{3.837772in}{1.319395in}}%
\pgfpathlineto{\pgfqpoint{3.840096in}{1.281510in}}%
\pgfpathlineto{\pgfqpoint{3.842420in}{1.404561in}}%
\pgfpathlineto{\pgfqpoint{3.844744in}{1.392401in}}%
\pgfpathlineto{\pgfqpoint{3.847068in}{0.776051in}}%
\pgfpathlineto{\pgfqpoint{3.849392in}{0.874610in}}%
\pgfpathlineto{\pgfqpoint{3.851716in}{0.827674in}}%
\pgfpathlineto{\pgfqpoint{3.856364in}{0.833574in}}%
\pgfpathlineto{\pgfqpoint{3.858688in}{0.890681in}}%
\pgfpathlineto{\pgfqpoint{3.861012in}{0.913586in}}%
\pgfpathlineto{\pgfqpoint{3.863336in}{0.886586in}}%
\pgfpathlineto{\pgfqpoint{3.865660in}{0.919598in}}%
\pgfpathlineto{\pgfqpoint{3.867984in}{0.800371in}}%
\pgfpathlineto{\pgfqpoint{3.870308in}{0.940814in}}%
\pgfpathlineto{\pgfqpoint{3.872632in}{0.857116in}}%
\pgfpathlineto{\pgfqpoint{3.874956in}{0.828525in}}%
\pgfpathlineto{\pgfqpoint{3.877280in}{0.816505in}}%
\pgfpathlineto{\pgfqpoint{3.879604in}{0.940746in}}%
\pgfpathlineto{\pgfqpoint{3.881928in}{0.877607in}}%
\pgfpathlineto{\pgfqpoint{3.884252in}{0.950433in}}%
\pgfpathlineto{\pgfqpoint{3.886576in}{0.800376in}}%
\pgfpathlineto{\pgfqpoint{3.891224in}{0.970528in}}%
\pgfpathlineto{\pgfqpoint{3.893548in}{0.836015in}}%
\pgfpathlineto{\pgfqpoint{3.895872in}{0.898366in}}%
\pgfpathlineto{\pgfqpoint{3.898196in}{0.845202in}}%
\pgfpathlineto{\pgfqpoint{3.900519in}{0.956375in}}%
\pgfpathlineto{\pgfqpoint{3.902843in}{0.908241in}}%
\pgfpathlineto{\pgfqpoint{3.905167in}{1.059873in}}%
\pgfpathlineto{\pgfqpoint{3.907491in}{0.947514in}}%
\pgfpathlineto{\pgfqpoint{3.909815in}{0.935669in}}%
\pgfpathlineto{\pgfqpoint{3.912139in}{0.982419in}}%
\pgfpathlineto{\pgfqpoint{3.914463in}{0.922877in}}%
\pgfpathlineto{\pgfqpoint{3.916787in}{0.988176in}}%
\pgfpathlineto{\pgfqpoint{3.919111in}{0.908617in}}%
\pgfpathlineto{\pgfqpoint{3.921435in}{1.078023in}}%
\pgfpathlineto{\pgfqpoint{3.926083in}{0.865767in}}%
\pgfpathlineto{\pgfqpoint{3.928407in}{1.005381in}}%
\pgfpathlineto{\pgfqpoint{3.930731in}{1.074269in}}%
\pgfpathlineto{\pgfqpoint{3.933055in}{1.033712in}}%
\pgfpathlineto{\pgfqpoint{3.935379in}{1.077236in}}%
\pgfpathlineto{\pgfqpoint{3.937703in}{1.083466in}}%
\pgfpathlineto{\pgfqpoint{3.940027in}{0.999778in}}%
\pgfpathlineto{\pgfqpoint{3.942351in}{0.864163in}}%
\pgfpathlineto{\pgfqpoint{3.944675in}{1.037290in}}%
\pgfpathlineto{\pgfqpoint{3.946999in}{1.043527in}}%
\pgfpathlineto{\pgfqpoint{3.949323in}{0.955432in}}%
\pgfpathlineto{\pgfqpoint{3.953971in}{1.020397in}}%
\pgfpathlineto{\pgfqpoint{3.956295in}{1.020211in}}%
\pgfpathlineto{\pgfqpoint{3.958619in}{1.026260in}}%
\pgfpathlineto{\pgfqpoint{3.960943in}{1.103671in}}%
\pgfpathlineto{\pgfqpoint{3.963267in}{0.992414in}}%
\pgfpathlineto{\pgfqpoint{3.965591in}{1.032584in}}%
\pgfpathlineto{\pgfqpoint{3.967915in}{1.128597in}}%
\pgfpathlineto{\pgfqpoint{3.970238in}{1.040822in}}%
\pgfpathlineto{\pgfqpoint{3.972562in}{1.007958in}}%
\pgfpathlineto{\pgfqpoint{3.974886in}{1.053213in}}%
\pgfpathlineto{\pgfqpoint{3.977210in}{1.051244in}}%
\pgfpathlineto{\pgfqpoint{3.979534in}{0.962676in}}%
\pgfpathlineto{\pgfqpoint{3.981858in}{1.057171in}}%
\pgfpathlineto{\pgfqpoint{3.984182in}{1.089715in}}%
\pgfpathlineto{\pgfqpoint{3.986506in}{1.004123in}}%
\pgfpathlineto{\pgfqpoint{3.988830in}{1.125429in}}%
\pgfpathlineto{\pgfqpoint{3.991154in}{1.125613in}}%
\pgfpathlineto{\pgfqpoint{3.993478in}{1.152727in}}%
\pgfpathlineto{\pgfqpoint{3.995802in}{1.074566in}}%
\pgfpathlineto{\pgfqpoint{3.998126in}{1.051616in}}%
\pgfpathlineto{\pgfqpoint{4.000450in}{1.112595in}}%
\pgfpathlineto{\pgfqpoint{4.002774in}{1.121906in}}%
\pgfpathlineto{\pgfqpoint{4.005098in}{1.051033in}}%
\pgfpathlineto{\pgfqpoint{4.007422in}{1.126268in}}%
\pgfpathlineto{\pgfqpoint{4.009746in}{1.071280in}}%
\pgfpathlineto{\pgfqpoint{4.014394in}{1.020305in}}%
\pgfpathlineto{\pgfqpoint{4.016718in}{1.178213in}}%
\pgfpathlineto{\pgfqpoint{4.019042in}{1.248064in}}%
\pgfpathlineto{\pgfqpoint{4.021366in}{1.176900in}}%
\pgfpathlineto{\pgfqpoint{4.023690in}{1.193829in}}%
\pgfpathlineto{\pgfqpoint{4.026014in}{1.068216in}}%
\pgfpathlineto{\pgfqpoint{4.028338in}{1.201215in}}%
\pgfpathlineto{\pgfqpoint{4.030662in}{1.129095in}}%
\pgfpathlineto{\pgfqpoint{4.032986in}{1.180630in}}%
\pgfpathlineto{\pgfqpoint{4.035310in}{1.162199in}}%
\pgfpathlineto{\pgfqpoint{4.039957in}{1.240626in}}%
\pgfpathlineto{\pgfqpoint{4.042281in}{1.181508in}}%
\pgfpathlineto{\pgfqpoint{4.044605in}{1.194508in}}%
\pgfpathlineto{\pgfqpoint{4.046929in}{1.053701in}}%
\pgfpathlineto{\pgfqpoint{4.051577in}{1.237044in}}%
\pgfpathlineto{\pgfqpoint{4.056225in}{1.172739in}}%
\pgfpathlineto{\pgfqpoint{4.058549in}{1.270204in}}%
\pgfpathlineto{\pgfqpoint{4.060873in}{1.203643in}}%
\pgfpathlineto{\pgfqpoint{4.063197in}{1.264416in}}%
\pgfpathlineto{\pgfqpoint{4.065521in}{1.193696in}}%
\pgfpathlineto{\pgfqpoint{4.067845in}{1.216173in}}%
\pgfpathlineto{\pgfqpoint{4.070169in}{1.250646in}}%
\pgfpathlineto{\pgfqpoint{4.074817in}{1.256339in}}%
\pgfpathlineto{\pgfqpoint{4.077141in}{1.281860in}}%
\pgfpathlineto{\pgfqpoint{4.079465in}{1.206747in}}%
\pgfpathlineto{\pgfqpoint{4.081789in}{1.216215in}}%
\pgfpathlineto{\pgfqpoint{4.084113in}{1.194709in}}%
\pgfpathlineto{\pgfqpoint{4.086437in}{1.227077in}}%
\pgfpathlineto{\pgfqpoint{4.088761in}{1.202273in}}%
\pgfpathlineto{\pgfqpoint{4.091085in}{1.250502in}}%
\pgfpathlineto{\pgfqpoint{4.093409in}{1.274045in}}%
\pgfpathlineto{\pgfqpoint{4.095733in}{1.195247in}}%
\pgfpathlineto{\pgfqpoint{4.098057in}{1.292979in}}%
\pgfpathlineto{\pgfqpoint{4.100381in}{1.182603in}}%
\pgfpathlineto{\pgfqpoint{4.102705in}{1.197117in}}%
\pgfpathlineto{\pgfqpoint{4.107353in}{1.310084in}}%
\pgfpathlineto{\pgfqpoint{4.109677in}{1.240698in}}%
\pgfpathlineto{\pgfqpoint{4.114324in}{1.374762in}}%
\pgfpathlineto{\pgfqpoint{4.116648in}{1.179338in}}%
\pgfpathlineto{\pgfqpoint{4.118972in}{1.250507in}}%
\pgfpathlineto{\pgfqpoint{4.121296in}{1.233743in}}%
\pgfpathlineto{\pgfqpoint{4.123620in}{1.331047in}}%
\pgfpathlineto{\pgfqpoint{4.125944in}{1.354471in}}%
\pgfpathlineto{\pgfqpoint{4.128268in}{1.262697in}}%
\pgfpathlineto{\pgfqpoint{4.130592in}{1.215666in}}%
\pgfpathlineto{\pgfqpoint{4.132916in}{1.195414in}}%
\pgfpathlineto{\pgfqpoint{4.135240in}{1.358394in}}%
\pgfpathlineto{\pgfqpoint{4.137564in}{0.821960in}}%
\pgfpathlineto{\pgfqpoint{4.139888in}{0.829834in}}%
\pgfpathlineto{\pgfqpoint{4.142212in}{0.845565in}}%
\pgfpathlineto{\pgfqpoint{4.144536in}{0.812623in}}%
\pgfpathlineto{\pgfqpoint{4.146860in}{0.713208in}}%
\pgfpathlineto{\pgfqpoint{4.149184in}{0.869311in}}%
\pgfpathlineto{\pgfqpoint{4.151508in}{0.796948in}}%
\pgfpathlineto{\pgfqpoint{4.156156in}{0.904692in}}%
\pgfpathlineto{\pgfqpoint{4.158480in}{0.842157in}}%
\pgfpathlineto{\pgfqpoint{4.160804in}{0.950963in}}%
\pgfpathlineto{\pgfqpoint{4.167776in}{0.859146in}}%
\pgfpathlineto{\pgfqpoint{4.170100in}{0.848025in}}%
\pgfpathlineto{\pgfqpoint{4.172424in}{0.869057in}}%
\pgfpathlineto{\pgfqpoint{4.174748in}{0.941557in}}%
\pgfpathlineto{\pgfqpoint{4.179396in}{0.872273in}}%
\pgfpathlineto{\pgfqpoint{4.181719in}{0.857686in}}%
\pgfpathlineto{\pgfqpoint{4.184043in}{0.981005in}}%
\pgfpathlineto{\pgfqpoint{4.186367in}{0.835283in}}%
\pgfpathlineto{\pgfqpoint{4.191015in}{1.002027in}}%
\pgfpathlineto{\pgfqpoint{4.193339in}{0.871062in}}%
\pgfpathlineto{\pgfqpoint{4.195663in}{0.942511in}}%
\pgfpathlineto{\pgfqpoint{4.197987in}{0.942748in}}%
\pgfpathlineto{\pgfqpoint{4.200311in}{0.958680in}}%
\pgfpathlineto{\pgfqpoint{4.202635in}{0.966025in}}%
\pgfpathlineto{\pgfqpoint{4.207283in}{0.856923in}}%
\pgfpathlineto{\pgfqpoint{4.209607in}{0.951511in}}%
\pgfpathlineto{\pgfqpoint{4.211931in}{0.968531in}}%
\pgfpathlineto{\pgfqpoint{4.214255in}{0.907233in}}%
\pgfpathlineto{\pgfqpoint{4.216579in}{0.979154in}}%
\pgfpathlineto{\pgfqpoint{4.221227in}{0.933885in}}%
\pgfpathlineto{\pgfqpoint{4.223551in}{0.855897in}}%
\pgfpathlineto{\pgfqpoint{4.225875in}{1.070260in}}%
\pgfpathlineto{\pgfqpoint{4.228199in}{1.124629in}}%
\pgfpathlineto{\pgfqpoint{4.230523in}{0.933162in}}%
\pgfpathlineto{\pgfqpoint{4.232847in}{0.984758in}}%
\pgfpathlineto{\pgfqpoint{4.235171in}{0.968107in}}%
\pgfpathlineto{\pgfqpoint{4.237495in}{0.938711in}}%
\pgfpathlineto{\pgfqpoint{4.239819in}{1.103876in}}%
\pgfpathlineto{\pgfqpoint{4.244467in}{0.907944in}}%
\pgfpathlineto{\pgfqpoint{4.246791in}{1.052432in}}%
\pgfpathlineto{\pgfqpoint{4.249115in}{1.116764in}}%
\pgfpathlineto{\pgfqpoint{4.253762in}{0.977796in}}%
\pgfpathlineto{\pgfqpoint{4.256086in}{1.012800in}}%
\pgfpathlineto{\pgfqpoint{4.258410in}{1.018784in}}%
\pgfpathlineto{\pgfqpoint{4.260734in}{0.920331in}}%
\pgfpathlineto{\pgfqpoint{4.263058in}{1.088065in}}%
\pgfpathlineto{\pgfqpoint{4.265382in}{1.077609in}}%
\pgfpathlineto{\pgfqpoint{4.267706in}{1.121697in}}%
\pgfpathlineto{\pgfqpoint{4.270030in}{1.054040in}}%
\pgfpathlineto{\pgfqpoint{4.274678in}{1.083693in}}%
\pgfpathlineto{\pgfqpoint{4.279326in}{1.035414in}}%
\pgfpathlineto{\pgfqpoint{4.281650in}{1.151147in}}%
\pgfpathlineto{\pgfqpoint{4.283974in}{1.057740in}}%
\pgfpathlineto{\pgfqpoint{4.286298in}{1.017195in}}%
\pgfpathlineto{\pgfqpoint{4.288622in}{1.177014in}}%
\pgfpathlineto{\pgfqpoint{4.290946in}{1.137317in}}%
\pgfpathlineto{\pgfqpoint{4.293270in}{1.231586in}}%
\pgfpathlineto{\pgfqpoint{4.295594in}{1.118816in}}%
\pgfpathlineto{\pgfqpoint{4.297918in}{1.110329in}}%
\pgfpathlineto{\pgfqpoint{4.300242in}{1.045561in}}%
\pgfpathlineto{\pgfqpoint{4.302566in}{1.070382in}}%
\pgfpathlineto{\pgfqpoint{4.307214in}{1.193677in}}%
\pgfpathlineto{\pgfqpoint{4.309538in}{1.098807in}}%
\pgfpathlineto{\pgfqpoint{4.314186in}{1.126525in}}%
\pgfpathlineto{\pgfqpoint{4.316510in}{1.199402in}}%
\pgfpathlineto{\pgfqpoint{4.318834in}{1.119794in}}%
\pgfpathlineto{\pgfqpoint{4.321158in}{1.152833in}}%
\pgfpathlineto{\pgfqpoint{4.323481in}{1.241652in}}%
\pgfpathlineto{\pgfqpoint{4.325805in}{1.170931in}}%
\pgfpathlineto{\pgfqpoint{4.330453in}{1.219934in}}%
\pgfpathlineto{\pgfqpoint{4.332777in}{1.213303in}}%
\pgfpathlineto{\pgfqpoint{4.335101in}{1.195427in}}%
\pgfpathlineto{\pgfqpoint{4.337425in}{1.093678in}}%
\pgfpathlineto{\pgfqpoint{4.339749in}{1.049154in}}%
\pgfpathlineto{\pgfqpoint{4.342073in}{1.105581in}}%
\pgfpathlineto{\pgfqpoint{4.344397in}{1.230649in}}%
\pgfpathlineto{\pgfqpoint{4.346721in}{1.122752in}}%
\pgfpathlineto{\pgfqpoint{4.349045in}{1.239406in}}%
\pgfpathlineto{\pgfqpoint{4.353693in}{1.199209in}}%
\pgfpathlineto{\pgfqpoint{4.356017in}{1.197512in}}%
\pgfpathlineto{\pgfqpoint{4.358341in}{1.190424in}}%
\pgfpathlineto{\pgfqpoint{4.360665in}{1.129333in}}%
\pgfpathlineto{\pgfqpoint{4.362989in}{1.262965in}}%
\pgfpathlineto{\pgfqpoint{4.365313in}{1.300163in}}%
\pgfpathlineto{\pgfqpoint{4.372285in}{1.210412in}}%
\pgfpathlineto{\pgfqpoint{4.374609in}{1.221663in}}%
\pgfpathlineto{\pgfqpoint{4.376933in}{1.223633in}}%
\pgfpathlineto{\pgfqpoint{4.379257in}{1.233346in}}%
\pgfpathlineto{\pgfqpoint{4.381581in}{1.177483in}}%
\pgfpathlineto{\pgfqpoint{4.383905in}{1.285591in}}%
\pgfpathlineto{\pgfqpoint{4.386229in}{1.170591in}}%
\pgfpathlineto{\pgfqpoint{4.388553in}{1.303979in}}%
\pgfpathlineto{\pgfqpoint{4.390877in}{1.281008in}}%
\pgfpathlineto{\pgfqpoint{4.393200in}{1.217899in}}%
\pgfpathlineto{\pgfqpoint{4.395524in}{1.305118in}}%
\pgfpathlineto{\pgfqpoint{4.397848in}{1.313048in}}%
\pgfpathlineto{\pgfqpoint{4.400172in}{1.297120in}}%
\pgfpathlineto{\pgfqpoint{4.402496in}{1.254530in}}%
\pgfpathlineto{\pgfqpoint{4.404820in}{1.325801in}}%
\pgfpathlineto{\pgfqpoint{4.407144in}{1.173166in}}%
\pgfpathlineto{\pgfqpoint{4.409468in}{1.359574in}}%
\pgfpathlineto{\pgfqpoint{4.411792in}{1.365457in}}%
\pgfpathlineto{\pgfqpoint{4.414116in}{1.296149in}}%
\pgfpathlineto{\pgfqpoint{4.416440in}{1.186220in}}%
\pgfpathlineto{\pgfqpoint{4.418764in}{1.291570in}}%
\pgfpathlineto{\pgfqpoint{4.421088in}{1.327853in}}%
\pgfpathlineto{\pgfqpoint{4.423412in}{1.245147in}}%
\pgfpathlineto{\pgfqpoint{4.425736in}{1.357015in}}%
\pgfpathlineto{\pgfqpoint{4.428060in}{0.827203in}}%
\pgfpathlineto{\pgfqpoint{4.430384in}{0.900160in}}%
\pgfpathlineto{\pgfqpoint{4.432708in}{0.751568in}}%
\pgfpathlineto{\pgfqpoint{4.435032in}{0.855079in}}%
\pgfpathlineto{\pgfqpoint{4.437356in}{0.890270in}}%
\pgfpathlineto{\pgfqpoint{4.439680in}{0.879259in}}%
\pgfpathlineto{\pgfqpoint{4.442004in}{0.876927in}}%
\pgfpathlineto{\pgfqpoint{4.444328in}{0.935060in}}%
\pgfpathlineto{\pgfqpoint{4.446652in}{0.871933in}}%
\pgfpathlineto{\pgfqpoint{4.448976in}{0.858316in}}%
\pgfpathlineto{\pgfqpoint{4.453624in}{0.947583in}}%
\pgfpathlineto{\pgfqpoint{4.455948in}{0.908788in}}%
\pgfpathlineto{\pgfqpoint{4.458272in}{0.995700in}}%
\pgfpathlineto{\pgfqpoint{4.460596in}{0.912078in}}%
\pgfpathlineto{\pgfqpoint{4.462919in}{0.915978in}}%
\pgfpathlineto{\pgfqpoint{4.465243in}{0.929978in}}%
\pgfpathlineto{\pgfqpoint{4.467567in}{0.848865in}}%
\pgfpathlineto{\pgfqpoint{4.469891in}{0.951626in}}%
\pgfpathlineto{\pgfqpoint{4.472215in}{0.905028in}}%
\pgfpathlineto{\pgfqpoint{4.474539in}{0.923839in}}%
\pgfpathlineto{\pgfqpoint{4.476863in}{0.985600in}}%
\pgfpathlineto{\pgfqpoint{4.479187in}{0.992149in}}%
\pgfpathlineto{\pgfqpoint{4.481511in}{0.915341in}}%
\pgfpathlineto{\pgfqpoint{4.483835in}{0.917514in}}%
\pgfpathlineto{\pgfqpoint{4.488483in}{1.040727in}}%
\pgfpathlineto{\pgfqpoint{4.490807in}{0.851478in}}%
\pgfpathlineto{\pgfqpoint{4.493131in}{0.908141in}}%
\pgfpathlineto{\pgfqpoint{4.495455in}{0.887640in}}%
\pgfpathlineto{\pgfqpoint{4.497779in}{1.043472in}}%
\pgfpathlineto{\pgfqpoint{4.500103in}{0.909287in}}%
\pgfpathlineto{\pgfqpoint{4.502427in}{0.901813in}}%
\pgfpathlineto{\pgfqpoint{4.504751in}{1.066644in}}%
\pgfpathlineto{\pgfqpoint{4.507075in}{1.003079in}}%
\pgfpathlineto{\pgfqpoint{4.509399in}{0.986403in}}%
\pgfpathlineto{\pgfqpoint{4.511723in}{1.029052in}}%
\pgfpathlineto{\pgfqpoint{4.514047in}{0.977575in}}%
\pgfpathlineto{\pgfqpoint{4.516371in}{0.966756in}}%
\pgfpathlineto{\pgfqpoint{4.518695in}{0.996896in}}%
\pgfpathlineto{\pgfqpoint{4.521019in}{0.924376in}}%
\pgfpathlineto{\pgfqpoint{4.523343in}{1.006884in}}%
\pgfpathlineto{\pgfqpoint{4.525667in}{1.023714in}}%
\pgfpathlineto{\pgfqpoint{4.527991in}{0.999543in}}%
\pgfpathlineto{\pgfqpoint{4.530315in}{1.061169in}}%
\pgfpathlineto{\pgfqpoint{4.532638in}{1.073336in}}%
\pgfpathlineto{\pgfqpoint{4.537286in}{1.014408in}}%
\pgfpathlineto{\pgfqpoint{4.539610in}{1.056432in}}%
\pgfpathlineto{\pgfqpoint{4.541934in}{1.063113in}}%
\pgfpathlineto{\pgfqpoint{4.544258in}{1.034676in}}%
\pgfpathlineto{\pgfqpoint{4.546582in}{0.967014in}}%
\pgfpathlineto{\pgfqpoint{4.548906in}{1.076952in}}%
\pgfpathlineto{\pgfqpoint{4.551230in}{0.987831in}}%
\pgfpathlineto{\pgfqpoint{4.553554in}{1.108567in}}%
\pgfpathlineto{\pgfqpoint{4.555878in}{1.108898in}}%
\pgfpathlineto{\pgfqpoint{4.558202in}{1.013804in}}%
\pgfpathlineto{\pgfqpoint{4.560526in}{1.017977in}}%
\pgfpathlineto{\pgfqpoint{4.562850in}{1.170271in}}%
\pgfpathlineto{\pgfqpoint{4.569822in}{1.030896in}}%
\pgfpathlineto{\pgfqpoint{4.572146in}{1.022994in}}%
\pgfpathlineto{\pgfqpoint{4.574470in}{1.138932in}}%
\pgfpathlineto{\pgfqpoint{4.576794in}{1.030818in}}%
\pgfpathlineto{\pgfqpoint{4.579118in}{1.230077in}}%
\pgfpathlineto{\pgfqpoint{4.581442in}{1.114242in}}%
\pgfpathlineto{\pgfqpoint{4.583766in}{1.095854in}}%
\pgfpathlineto{\pgfqpoint{4.586090in}{1.179620in}}%
\pgfpathlineto{\pgfqpoint{4.590738in}{1.086316in}}%
\pgfpathlineto{\pgfqpoint{4.593062in}{1.216207in}}%
\pgfpathlineto{\pgfqpoint{4.595386in}{1.169606in}}%
\pgfpathlineto{\pgfqpoint{4.597710in}{1.160027in}}%
\pgfpathlineto{\pgfqpoint{4.600034in}{1.226830in}}%
\pgfpathlineto{\pgfqpoint{4.602358in}{1.076058in}}%
\pgfpathlineto{\pgfqpoint{4.604681in}{1.184882in}}%
\pgfpathlineto{\pgfqpoint{4.607005in}{1.070497in}}%
\pgfpathlineto{\pgfqpoint{4.613977in}{1.248339in}}%
\pgfpathlineto{\pgfqpoint{4.616301in}{1.109191in}}%
\pgfpathlineto{\pgfqpoint{4.618625in}{1.176785in}}%
\pgfpathlineto{\pgfqpoint{4.620949in}{1.144533in}}%
\pgfpathlineto{\pgfqpoint{4.623273in}{1.283384in}}%
\pgfpathlineto{\pgfqpoint{4.627921in}{1.129333in}}%
\pgfpathlineto{\pgfqpoint{4.630245in}{1.162375in}}%
\pgfpathlineto{\pgfqpoint{4.632569in}{1.315937in}}%
\pgfpathlineto{\pgfqpoint{4.634893in}{1.127739in}}%
\pgfpathlineto{\pgfqpoint{4.637217in}{1.185866in}}%
\pgfpathlineto{\pgfqpoint{4.639541in}{1.163328in}}%
\pgfpathlineto{\pgfqpoint{4.641865in}{1.276251in}}%
\pgfpathlineto{\pgfqpoint{4.644189in}{1.332215in}}%
\pgfpathlineto{\pgfqpoint{4.646513in}{1.175208in}}%
\pgfpathlineto{\pgfqpoint{4.648837in}{1.269109in}}%
\pgfpathlineto{\pgfqpoint{4.651161in}{1.256501in}}%
\pgfpathlineto{\pgfqpoint{4.653485in}{1.149053in}}%
\pgfpathlineto{\pgfqpoint{4.655809in}{1.273051in}}%
\pgfpathlineto{\pgfqpoint{4.658133in}{1.186138in}}%
\pgfpathlineto{\pgfqpoint{4.660457in}{1.208868in}}%
\pgfpathlineto{\pgfqpoint{4.662781in}{1.289495in}}%
\pgfpathlineto{\pgfqpoint{4.665105in}{1.304661in}}%
\pgfpathlineto{\pgfqpoint{4.667429in}{1.372970in}}%
\pgfpathlineto{\pgfqpoint{4.672077in}{1.257001in}}%
\pgfpathlineto{\pgfqpoint{4.674400in}{1.310529in}}%
\pgfpathlineto{\pgfqpoint{4.679048in}{1.269334in}}%
\pgfpathlineto{\pgfqpoint{4.681372in}{1.268561in}}%
\pgfpathlineto{\pgfqpoint{4.683696in}{1.256478in}}%
\pgfpathlineto{\pgfqpoint{4.686020in}{1.340958in}}%
\pgfpathlineto{\pgfqpoint{4.688344in}{1.303491in}}%
\pgfpathlineto{\pgfqpoint{4.690668in}{1.355246in}}%
\pgfpathlineto{\pgfqpoint{4.692992in}{1.365520in}}%
\pgfpathlineto{\pgfqpoint{4.695316in}{1.271695in}}%
\pgfpathlineto{\pgfqpoint{4.697640in}{1.237518in}}%
\pgfpathlineto{\pgfqpoint{4.699964in}{1.342405in}}%
\pgfpathlineto{\pgfqpoint{4.702288in}{1.261475in}}%
\pgfpathlineto{\pgfqpoint{4.704612in}{1.370546in}}%
\pgfpathlineto{\pgfqpoint{4.706936in}{1.308143in}}%
\pgfpathlineto{\pgfqpoint{4.709260in}{1.407617in}}%
\pgfpathlineto{\pgfqpoint{4.711584in}{1.298656in}}%
\pgfpathlineto{\pgfqpoint{4.713908in}{1.384162in}}%
\pgfpathlineto{\pgfqpoint{4.716232in}{1.320431in}}%
\pgfpathlineto{\pgfqpoint{4.718556in}{0.878103in}}%
\pgfpathlineto{\pgfqpoint{4.720880in}{0.817448in}}%
\pgfpathlineto{\pgfqpoint{4.723204in}{0.817365in}}%
\pgfpathlineto{\pgfqpoint{4.725528in}{0.876404in}}%
\pgfpathlineto{\pgfqpoint{4.727852in}{0.797279in}}%
\pgfpathlineto{\pgfqpoint{4.730176in}{0.966855in}}%
\pgfpathlineto{\pgfqpoint{4.732500in}{0.890870in}}%
\pgfpathlineto{\pgfqpoint{4.734824in}{0.955790in}}%
\pgfpathlineto{\pgfqpoint{4.737148in}{0.966450in}}%
\pgfpathlineto{\pgfqpoint{4.739472in}{0.867071in}}%
\pgfpathlineto{\pgfqpoint{4.741796in}{0.879138in}}%
\pgfpathlineto{\pgfqpoint{4.744119in}{0.981689in}}%
\pgfpathlineto{\pgfqpoint{4.748767in}{0.795444in}}%
\pgfpathlineto{\pgfqpoint{4.751091in}{0.856113in}}%
\pgfpathlineto{\pgfqpoint{4.753415in}{0.947594in}}%
\pgfpathlineto{\pgfqpoint{4.755739in}{0.839836in}}%
\pgfpathlineto{\pgfqpoint{4.758063in}{0.959613in}}%
\pgfpathlineto{\pgfqpoint{4.760387in}{0.957024in}}%
\pgfpathlineto{\pgfqpoint{4.762711in}{0.978614in}}%
\pgfpathlineto{\pgfqpoint{4.765035in}{0.894754in}}%
\pgfpathlineto{\pgfqpoint{4.767359in}{0.950691in}}%
\pgfpathlineto{\pgfqpoint{4.769683in}{0.884021in}}%
\pgfpathlineto{\pgfqpoint{4.772007in}{1.039817in}}%
\pgfpathlineto{\pgfqpoint{4.774331in}{1.046942in}}%
\pgfpathlineto{\pgfqpoint{4.778979in}{0.985463in}}%
\pgfpathlineto{\pgfqpoint{4.781303in}{0.947330in}}%
\pgfpathlineto{\pgfqpoint{4.783627in}{0.962427in}}%
\pgfpathlineto{\pgfqpoint{4.785951in}{0.943104in}}%
\pgfpathlineto{\pgfqpoint{4.788275in}{1.038870in}}%
\pgfpathlineto{\pgfqpoint{4.790599in}{0.915362in}}%
\pgfpathlineto{\pgfqpoint{4.792923in}{1.032135in}}%
\pgfpathlineto{\pgfqpoint{4.795247in}{1.022137in}}%
\pgfpathlineto{\pgfqpoint{4.797571in}{0.931608in}}%
\pgfpathlineto{\pgfqpoint{4.799895in}{1.005577in}}%
\pgfpathlineto{\pgfqpoint{4.802219in}{0.999158in}}%
\pgfpathlineto{\pgfqpoint{4.804543in}{1.026040in}}%
\pgfpathlineto{\pgfqpoint{4.809191in}{0.934179in}}%
\pgfpathlineto{\pgfqpoint{4.811515in}{1.072746in}}%
\pgfpathlineto{\pgfqpoint{4.813838in}{1.045039in}}%
\pgfpathlineto{\pgfqpoint{4.816162in}{0.927817in}}%
\pgfpathlineto{\pgfqpoint{4.818486in}{1.126140in}}%
\pgfpathlineto{\pgfqpoint{4.820810in}{1.016398in}}%
\pgfpathlineto{\pgfqpoint{4.823134in}{1.007699in}}%
\pgfpathlineto{\pgfqpoint{4.825458in}{0.962657in}}%
\pgfpathlineto{\pgfqpoint{4.827782in}{1.019030in}}%
\pgfpathlineto{\pgfqpoint{4.830106in}{1.110662in}}%
\pgfpathlineto{\pgfqpoint{4.832430in}{0.919837in}}%
\pgfpathlineto{\pgfqpoint{4.834754in}{1.080818in}}%
\pgfpathlineto{\pgfqpoint{4.837078in}{1.030376in}}%
\pgfpathlineto{\pgfqpoint{4.841726in}{1.085797in}}%
\pgfpathlineto{\pgfqpoint{4.844050in}{1.126028in}}%
\pgfpathlineto{\pgfqpoint{4.846374in}{1.123560in}}%
\pgfpathlineto{\pgfqpoint{4.848698in}{1.164599in}}%
\pgfpathlineto{\pgfqpoint{4.851022in}{1.032109in}}%
\pgfpathlineto{\pgfqpoint{4.855670in}{1.162834in}}%
\pgfpathlineto{\pgfqpoint{4.857994in}{1.088470in}}%
\pgfpathlineto{\pgfqpoint{4.860318in}{1.174139in}}%
\pgfpathlineto{\pgfqpoint{4.864966in}{1.018506in}}%
\pgfpathlineto{\pgfqpoint{4.869614in}{1.104294in}}%
\pgfpathlineto{\pgfqpoint{4.871938in}{1.059874in}}%
\pgfpathlineto{\pgfqpoint{4.874262in}{1.172904in}}%
\pgfpathlineto{\pgfqpoint{4.876586in}{1.154512in}}%
\pgfpathlineto{\pgfqpoint{4.878910in}{1.084584in}}%
\pgfpathlineto{\pgfqpoint{4.883558in}{1.231835in}}%
\pgfpathlineto{\pgfqpoint{4.885881in}{1.177730in}}%
\pgfpathlineto{\pgfqpoint{4.888205in}{1.075337in}}%
\pgfpathlineto{\pgfqpoint{4.890529in}{1.213218in}}%
\pgfpathlineto{\pgfqpoint{4.892853in}{1.116689in}}%
\pgfpathlineto{\pgfqpoint{4.895177in}{1.114190in}}%
\pgfpathlineto{\pgfqpoint{4.897501in}{1.219290in}}%
\pgfpathlineto{\pgfqpoint{4.899825in}{1.109601in}}%
\pgfpathlineto{\pgfqpoint{4.902149in}{1.159387in}}%
\pgfpathlineto{\pgfqpoint{4.904473in}{1.115783in}}%
\pgfpathlineto{\pgfqpoint{4.906797in}{1.179043in}}%
\pgfpathlineto{\pgfqpoint{4.909121in}{1.128588in}}%
\pgfpathlineto{\pgfqpoint{4.911445in}{1.116222in}}%
\pgfpathlineto{\pgfqpoint{4.913769in}{1.251077in}}%
\pgfpathlineto{\pgfqpoint{4.916093in}{1.255005in}}%
\pgfpathlineto{\pgfqpoint{4.918417in}{1.269484in}}%
\pgfpathlineto{\pgfqpoint{4.920741in}{1.168239in}}%
\pgfpathlineto{\pgfqpoint{4.923065in}{1.133451in}}%
\pgfpathlineto{\pgfqpoint{4.925389in}{1.188399in}}%
\pgfpathlineto{\pgfqpoint{4.927713in}{1.278090in}}%
\pgfpathlineto{\pgfqpoint{4.930037in}{1.201486in}}%
\pgfpathlineto{\pgfqpoint{4.932361in}{1.231020in}}%
\pgfpathlineto{\pgfqpoint{4.934685in}{1.101398in}}%
\pgfpathlineto{\pgfqpoint{4.939333in}{1.325156in}}%
\pgfpathlineto{\pgfqpoint{4.941657in}{1.110618in}}%
\pgfpathlineto{\pgfqpoint{4.943981in}{1.183331in}}%
\pgfpathlineto{\pgfqpoint{4.946305in}{1.207109in}}%
\pgfpathlineto{\pgfqpoint{4.948629in}{1.190815in}}%
\pgfpathlineto{\pgfqpoint{4.950953in}{1.280866in}}%
\pgfpathlineto{\pgfqpoint{4.953277in}{1.261499in}}%
\pgfpathlineto{\pgfqpoint{4.955600in}{1.228932in}}%
\pgfpathlineto{\pgfqpoint{4.957924in}{1.258277in}}%
\pgfpathlineto{\pgfqpoint{4.960248in}{1.244526in}}%
\pgfpathlineto{\pgfqpoint{4.962572in}{1.144024in}}%
\pgfpathlineto{\pgfqpoint{4.964896in}{1.330705in}}%
\pgfpathlineto{\pgfqpoint{4.967220in}{1.391415in}}%
\pgfpathlineto{\pgfqpoint{4.971868in}{1.159675in}}%
\pgfpathlineto{\pgfqpoint{4.978840in}{1.344442in}}%
\pgfpathlineto{\pgfqpoint{4.981164in}{1.317734in}}%
\pgfpathlineto{\pgfqpoint{4.983488in}{1.310684in}}%
\pgfpathlineto{\pgfqpoint{4.985812in}{1.337263in}}%
\pgfpathlineto{\pgfqpoint{4.988136in}{1.316268in}}%
\pgfpathlineto{\pgfqpoint{4.990460in}{1.279691in}}%
\pgfpathlineto{\pgfqpoint{4.992784in}{1.274222in}}%
\pgfpathlineto{\pgfqpoint{4.995108in}{1.380602in}}%
\pgfpathlineto{\pgfqpoint{4.997432in}{1.378861in}}%
\pgfpathlineto{\pgfqpoint{4.999756in}{1.364293in}}%
\pgfpathlineto{\pgfqpoint{5.002080in}{1.408293in}}%
\pgfpathlineto{\pgfqpoint{5.004404in}{1.425980in}}%
\pgfpathlineto{\pgfqpoint{5.006728in}{1.329864in}}%
\pgfpathlineto{\pgfqpoint{5.009052in}{0.884298in}}%
\pgfpathlineto{\pgfqpoint{5.011376in}{0.947324in}}%
\pgfpathlineto{\pgfqpoint{5.013700in}{0.906098in}}%
\pgfpathlineto{\pgfqpoint{5.016024in}{0.993320in}}%
\pgfpathlineto{\pgfqpoint{5.018348in}{0.892177in}}%
\pgfpathlineto{\pgfqpoint{5.022996in}{0.920948in}}%
\pgfpathlineto{\pgfqpoint{5.025319in}{0.882800in}}%
\pgfpathlineto{\pgfqpoint{5.027643in}{0.910449in}}%
\pgfpathlineto{\pgfqpoint{5.029967in}{0.885036in}}%
\pgfpathlineto{\pgfqpoint{5.032291in}{0.895739in}}%
\pgfpathlineto{\pgfqpoint{5.034615in}{0.937323in}}%
\pgfpathlineto{\pgfqpoint{5.036939in}{0.893162in}}%
\pgfpathlineto{\pgfqpoint{5.039263in}{0.916659in}}%
\pgfpathlineto{\pgfqpoint{5.041587in}{0.820084in}}%
\pgfpathlineto{\pgfqpoint{5.043911in}{0.885462in}}%
\pgfpathlineto{\pgfqpoint{5.046235in}{0.905077in}}%
\pgfpathlineto{\pgfqpoint{5.048559in}{0.967712in}}%
\pgfpathlineto{\pgfqpoint{5.050883in}{0.975811in}}%
\pgfpathlineto{\pgfqpoint{5.057855in}{1.011183in}}%
\pgfpathlineto{\pgfqpoint{5.060179in}{1.085723in}}%
\pgfpathlineto{\pgfqpoint{5.062503in}{0.963289in}}%
\pgfpathlineto{\pgfqpoint{5.064827in}{1.060571in}}%
\pgfpathlineto{\pgfqpoint{5.067151in}{1.046194in}}%
\pgfpathlineto{\pgfqpoint{5.069475in}{1.004054in}}%
\pgfpathlineto{\pgfqpoint{5.071799in}{0.940326in}}%
\pgfpathlineto{\pgfqpoint{5.074123in}{0.930916in}}%
\pgfpathlineto{\pgfqpoint{5.076447in}{0.938275in}}%
\pgfpathlineto{\pgfqpoint{5.078771in}{0.999689in}}%
\pgfpathlineto{\pgfqpoint{5.081095in}{0.904031in}}%
\pgfpathlineto{\pgfqpoint{5.083419in}{1.012503in}}%
\pgfpathlineto{\pgfqpoint{5.085743in}{1.061002in}}%
\pgfpathlineto{\pgfqpoint{5.088067in}{1.033218in}}%
\pgfpathlineto{\pgfqpoint{5.090391in}{1.020725in}}%
\pgfpathlineto{\pgfqpoint{5.092715in}{1.037984in}}%
\pgfpathlineto{\pgfqpoint{5.095039in}{1.026200in}}%
\pgfpathlineto{\pgfqpoint{5.097362in}{1.072271in}}%
\pgfpathlineto{\pgfqpoint{5.099686in}{1.013418in}}%
\pgfpathlineto{\pgfqpoint{5.102010in}{1.067220in}}%
\pgfpathlineto{\pgfqpoint{5.104334in}{1.056781in}}%
\pgfpathlineto{\pgfqpoint{5.106658in}{0.974283in}}%
\pgfpathlineto{\pgfqpoint{5.111306in}{1.058126in}}%
\pgfpathlineto{\pgfqpoint{5.113630in}{1.165945in}}%
\pgfpathlineto{\pgfqpoint{5.115954in}{1.118837in}}%
\pgfpathlineto{\pgfqpoint{5.118278in}{1.013960in}}%
\pgfpathlineto{\pgfqpoint{5.120602in}{1.079881in}}%
\pgfpathlineto{\pgfqpoint{5.122926in}{1.019596in}}%
\pgfpathlineto{\pgfqpoint{5.125250in}{1.108834in}}%
\pgfpathlineto{\pgfqpoint{5.127574in}{1.081298in}}%
\pgfpathlineto{\pgfqpoint{5.129898in}{1.127233in}}%
\pgfpathlineto{\pgfqpoint{5.132222in}{1.075364in}}%
\pgfpathlineto{\pgfqpoint{5.134546in}{1.146244in}}%
\pgfpathlineto{\pgfqpoint{5.136870in}{1.111912in}}%
\pgfpathlineto{\pgfqpoint{5.139194in}{1.150195in}}%
\pgfpathlineto{\pgfqpoint{5.141518in}{1.113381in}}%
\pgfpathlineto{\pgfqpoint{5.143842in}{1.049587in}}%
\pgfpathlineto{\pgfqpoint{5.148490in}{1.165819in}}%
\pgfpathlineto{\pgfqpoint{5.150814in}{1.137982in}}%
\pgfpathlineto{\pgfqpoint{5.153138in}{1.094433in}}%
\pgfpathlineto{\pgfqpoint{5.155462in}{1.105990in}}%
\pgfpathlineto{\pgfqpoint{5.157786in}{1.135428in}}%
\pgfpathlineto{\pgfqpoint{5.160110in}{0.986509in}}%
\pgfpathlineto{\pgfqpoint{5.167081in}{1.209569in}}%
\pgfpathlineto{\pgfqpoint{5.169405in}{1.063741in}}%
\pgfpathlineto{\pgfqpoint{5.171729in}{1.225533in}}%
\pgfpathlineto{\pgfqpoint{5.174053in}{1.105243in}}%
\pgfpathlineto{\pgfqpoint{5.176377in}{1.181572in}}%
\pgfpathlineto{\pgfqpoint{5.178701in}{1.220211in}}%
\pgfpathlineto{\pgfqpoint{5.181025in}{1.037062in}}%
\pgfpathlineto{\pgfqpoint{5.183349in}{1.156749in}}%
\pgfpathlineto{\pgfqpoint{5.185673in}{1.175931in}}%
\pgfpathlineto{\pgfqpoint{5.187997in}{1.176028in}}%
\pgfpathlineto{\pgfqpoint{5.190321in}{1.091000in}}%
\pgfpathlineto{\pgfqpoint{5.192645in}{1.208000in}}%
\pgfpathlineto{\pgfqpoint{5.194969in}{1.255399in}}%
\pgfpathlineto{\pgfqpoint{5.197293in}{1.214633in}}%
\pgfpathlineto{\pgfqpoint{5.199617in}{1.132302in}}%
\pgfpathlineto{\pgfqpoint{5.201941in}{1.170863in}}%
\pgfpathlineto{\pgfqpoint{5.204265in}{1.239742in}}%
\pgfpathlineto{\pgfqpoint{5.206589in}{1.177609in}}%
\pgfpathlineto{\pgfqpoint{5.208913in}{1.181342in}}%
\pgfpathlineto{\pgfqpoint{5.211237in}{1.198115in}}%
\pgfpathlineto{\pgfqpoint{5.213561in}{1.245016in}}%
\pgfpathlineto{\pgfqpoint{5.215885in}{1.177922in}}%
\pgfpathlineto{\pgfqpoint{5.218209in}{1.213033in}}%
\pgfpathlineto{\pgfqpoint{5.220533in}{1.205524in}}%
\pgfpathlineto{\pgfqpoint{5.222857in}{1.251147in}}%
\pgfpathlineto{\pgfqpoint{5.225181in}{1.151315in}}%
\pgfpathlineto{\pgfqpoint{5.229829in}{1.206719in}}%
\pgfpathlineto{\pgfqpoint{5.232153in}{1.292278in}}%
\pgfpathlineto{\pgfqpoint{5.234477in}{1.292802in}}%
\pgfpathlineto{\pgfqpoint{5.236800in}{1.128191in}}%
\pgfpathlineto{\pgfqpoint{5.241448in}{1.384275in}}%
\pgfpathlineto{\pgfqpoint{5.243772in}{1.316770in}}%
\pgfpathlineto{\pgfqpoint{5.246096in}{1.306326in}}%
\pgfpathlineto{\pgfqpoint{5.248420in}{1.226893in}}%
\pgfpathlineto{\pgfqpoint{5.250744in}{1.213018in}}%
\pgfpathlineto{\pgfqpoint{5.255392in}{1.326790in}}%
\pgfpathlineto{\pgfqpoint{5.257716in}{1.269894in}}%
\pgfpathlineto{\pgfqpoint{5.260040in}{1.292780in}}%
\pgfpathlineto{\pgfqpoint{5.262364in}{1.280281in}}%
\pgfpathlineto{\pgfqpoint{5.264688in}{1.335229in}}%
\pgfpathlineto{\pgfqpoint{5.267012in}{1.288810in}}%
\pgfpathlineto{\pgfqpoint{5.269336in}{1.295337in}}%
\pgfpathlineto{\pgfqpoint{5.271660in}{1.295086in}}%
\pgfpathlineto{\pgfqpoint{5.273984in}{1.219030in}}%
\pgfpathlineto{\pgfqpoint{5.276308in}{1.341823in}}%
\pgfpathlineto{\pgfqpoint{5.278632in}{1.308093in}}%
\pgfpathlineto{\pgfqpoint{5.280956in}{1.394892in}}%
\pgfpathlineto{\pgfqpoint{5.283280in}{1.375823in}}%
\pgfpathlineto{\pgfqpoint{5.285604in}{1.303793in}}%
\pgfpathlineto{\pgfqpoint{5.287928in}{1.350504in}}%
\pgfpathlineto{\pgfqpoint{5.290252in}{1.374565in}}%
\pgfpathlineto{\pgfqpoint{5.292576in}{1.355059in}}%
\pgfpathlineto{\pgfqpoint{5.294900in}{1.380895in}}%
\pgfpathlineto{\pgfqpoint{5.297224in}{1.329028in}}%
\pgfpathlineto{\pgfqpoint{5.299548in}{0.783358in}}%
\pgfpathlineto{\pgfqpoint{5.301872in}{0.871207in}}%
\pgfpathlineto{\pgfqpoint{5.304196in}{0.872701in}}%
\pgfpathlineto{\pgfqpoint{5.306519in}{0.891200in}}%
\pgfpathlineto{\pgfqpoint{5.308843in}{0.815160in}}%
\pgfpathlineto{\pgfqpoint{5.311167in}{0.836368in}}%
\pgfpathlineto{\pgfqpoint{5.315815in}{0.971181in}}%
\pgfpathlineto{\pgfqpoint{5.318139in}{0.939941in}}%
\pgfpathlineto{\pgfqpoint{5.322787in}{0.988332in}}%
\pgfpathlineto{\pgfqpoint{5.325111in}{0.864207in}}%
\pgfpathlineto{\pgfqpoint{5.327435in}{0.881745in}}%
\pgfpathlineto{\pgfqpoint{5.329759in}{1.020513in}}%
\pgfpathlineto{\pgfqpoint{5.334407in}{0.792231in}}%
\pgfpathlineto{\pgfqpoint{5.336731in}{0.963601in}}%
\pgfpathlineto{\pgfqpoint{5.339055in}{0.996850in}}%
\pgfpathlineto{\pgfqpoint{5.341379in}{1.009886in}}%
\pgfpathlineto{\pgfqpoint{5.343703in}{0.913369in}}%
\pgfpathlineto{\pgfqpoint{5.346027in}{0.930716in}}%
\pgfpathlineto{\pgfqpoint{5.348351in}{0.918683in}}%
\pgfpathlineto{\pgfqpoint{5.350675in}{0.919093in}}%
\pgfpathlineto{\pgfqpoint{5.355323in}{0.990596in}}%
\pgfpathlineto{\pgfqpoint{5.357647in}{0.912941in}}%
\pgfpathlineto{\pgfqpoint{5.359971in}{0.966087in}}%
\pgfpathlineto{\pgfqpoint{5.362295in}{0.966456in}}%
\pgfpathlineto{\pgfqpoint{5.364619in}{0.874342in}}%
\pgfpathlineto{\pgfqpoint{5.366943in}{1.043365in}}%
\pgfpathlineto{\pgfqpoint{5.369267in}{0.977018in}}%
\pgfpathlineto{\pgfqpoint{5.371591in}{0.966175in}}%
\pgfpathlineto{\pgfqpoint{5.373915in}{0.971346in}}%
\pgfpathlineto{\pgfqpoint{5.376239in}{0.953969in}}%
\pgfpathlineto{\pgfqpoint{5.378562in}{0.975132in}}%
\pgfpathlineto{\pgfqpoint{5.380886in}{1.018454in}}%
\pgfpathlineto{\pgfqpoint{5.383210in}{0.930224in}}%
\pgfpathlineto{\pgfqpoint{5.385534in}{1.058667in}}%
\pgfpathlineto{\pgfqpoint{5.387858in}{0.958864in}}%
\pgfpathlineto{\pgfqpoint{5.390182in}{1.015222in}}%
\pgfpathlineto{\pgfqpoint{5.392506in}{0.995534in}}%
\pgfpathlineto{\pgfqpoint{5.394830in}{1.137590in}}%
\pgfpathlineto{\pgfqpoint{5.401802in}{0.948888in}}%
\pgfpathlineto{\pgfqpoint{5.404126in}{1.033141in}}%
\pgfpathlineto{\pgfqpoint{5.406450in}{0.942539in}}%
\pgfpathlineto{\pgfqpoint{5.408774in}{1.109173in}}%
\pgfpathlineto{\pgfqpoint{5.411098in}{1.143622in}}%
\pgfpathlineto{\pgfqpoint{5.415746in}{0.888753in}}%
\pgfpathlineto{\pgfqpoint{5.418070in}{1.071515in}}%
\pgfpathlineto{\pgfqpoint{5.420394in}{0.954690in}}%
\pgfpathlineto{\pgfqpoint{5.422718in}{1.021536in}}%
\pgfpathlineto{\pgfqpoint{5.425042in}{1.031712in}}%
\pgfpathlineto{\pgfqpoint{5.427366in}{1.057712in}}%
\pgfpathlineto{\pgfqpoint{5.429690in}{1.109127in}}%
\pgfpathlineto{\pgfqpoint{5.432014in}{1.055112in}}%
\pgfpathlineto{\pgfqpoint{5.434338in}{1.125634in}}%
\pgfpathlineto{\pgfqpoint{5.436662in}{1.002085in}}%
\pgfpathlineto{\pgfqpoint{5.438986in}{1.164708in}}%
\pgfpathlineto{\pgfqpoint{5.441310in}{1.114446in}}%
\pgfpathlineto{\pgfqpoint{5.443634in}{1.018213in}}%
\pgfpathlineto{\pgfqpoint{5.445958in}{1.094230in}}%
\pgfpathlineto{\pgfqpoint{5.448281in}{1.071692in}}%
\pgfpathlineto{\pgfqpoint{5.450605in}{1.193571in}}%
\pgfpathlineto{\pgfqpoint{5.452929in}{1.023567in}}%
\pgfpathlineto{\pgfqpoint{5.455253in}{1.151035in}}%
\pgfpathlineto{\pgfqpoint{5.457577in}{1.096213in}}%
\pgfpathlineto{\pgfqpoint{5.459901in}{1.216539in}}%
\pgfpathlineto{\pgfqpoint{5.462225in}{1.107229in}}%
\pgfpathlineto{\pgfqpoint{5.464549in}{1.114495in}}%
\pgfpathlineto{\pgfqpoint{5.466873in}{1.054145in}}%
\pgfpathlineto{\pgfqpoint{5.469197in}{1.178134in}}%
\pgfpathlineto{\pgfqpoint{5.471521in}{1.121277in}}%
\pgfpathlineto{\pgfqpoint{5.473845in}{1.243053in}}%
\pgfpathlineto{\pgfqpoint{5.476169in}{1.157487in}}%
\pgfpathlineto{\pgfqpoint{5.478493in}{1.163945in}}%
\pgfpathlineto{\pgfqpoint{5.480817in}{1.095358in}}%
\pgfpathlineto{\pgfqpoint{5.485465in}{1.230540in}}%
\pgfpathlineto{\pgfqpoint{5.487789in}{1.155021in}}%
\pgfpathlineto{\pgfqpoint{5.490113in}{1.119126in}}%
\pgfpathlineto{\pgfqpoint{5.492437in}{1.152909in}}%
\pgfpathlineto{\pgfqpoint{5.494761in}{1.129877in}}%
\pgfpathlineto{\pgfqpoint{5.497085in}{1.118915in}}%
\pgfpathlineto{\pgfqpoint{5.499409in}{1.215862in}}%
\pgfpathlineto{\pgfqpoint{5.501733in}{1.143354in}}%
\pgfpathlineto{\pgfqpoint{5.504057in}{1.161074in}}%
\pgfpathlineto{\pgfqpoint{5.506381in}{1.215918in}}%
\pgfpathlineto{\pgfqpoint{5.508705in}{1.237238in}}%
\pgfpathlineto{\pgfqpoint{5.511029in}{1.195359in}}%
\pgfpathlineto{\pgfqpoint{5.513353in}{1.213922in}}%
\pgfpathlineto{\pgfqpoint{5.515677in}{1.177576in}}%
\pgfpathlineto{\pgfqpoint{5.518000in}{1.187161in}}%
\pgfpathlineto{\pgfqpoint{5.520324in}{1.266189in}}%
\pgfpathlineto{\pgfqpoint{5.522648in}{1.293670in}}%
\pgfpathlineto{\pgfqpoint{5.524972in}{1.144857in}}%
\pgfpathlineto{\pgfqpoint{5.529620in}{1.264308in}}%
\pgfpathlineto{\pgfqpoint{5.531944in}{1.150605in}}%
\pgfpathlineto{\pgfqpoint{5.534268in}{1.212004in}}%
\pgfpathlineto{\pgfqpoint{5.536592in}{1.181073in}}%
\pgfpathlineto{\pgfqpoint{5.538916in}{1.232681in}}%
\pgfpathlineto{\pgfqpoint{5.541240in}{1.209175in}}%
\pgfpathlineto{\pgfqpoint{5.543564in}{1.250412in}}%
\pgfpathlineto{\pgfqpoint{5.545888in}{1.332020in}}%
\pgfpathlineto{\pgfqpoint{5.548212in}{1.240461in}}%
\pgfpathlineto{\pgfqpoint{5.550536in}{1.228285in}}%
\pgfpathlineto{\pgfqpoint{5.552860in}{1.307052in}}%
\pgfpathlineto{\pgfqpoint{5.555184in}{1.272497in}}%
\pgfpathlineto{\pgfqpoint{5.557508in}{1.451917in}}%
\pgfpathlineto{\pgfqpoint{5.559832in}{1.222204in}}%
\pgfpathlineto{\pgfqpoint{5.562156in}{1.232623in}}%
\pgfpathlineto{\pgfqpoint{5.564480in}{1.277750in}}%
\pgfpathlineto{\pgfqpoint{5.566804in}{1.206218in}}%
\pgfpathlineto{\pgfqpoint{5.569128in}{1.300083in}}%
\pgfpathlineto{\pgfqpoint{5.571452in}{1.248469in}}%
\pgfpathlineto{\pgfqpoint{5.573776in}{1.245056in}}%
\pgfpathlineto{\pgfqpoint{5.576100in}{1.185091in}}%
\pgfpathlineto{\pgfqpoint{5.578424in}{1.345654in}}%
\pgfpathlineto{\pgfqpoint{5.580748in}{1.375183in}}%
\pgfpathlineto{\pgfqpoint{5.583072in}{1.370751in}}%
\pgfpathlineto{\pgfqpoint{5.585396in}{1.270292in}}%
\pgfpathlineto{\pgfqpoint{5.587720in}{0.866807in}}%
\pgfpathlineto{\pgfqpoint{5.587720in}{0.866807in}}%
\pgfusepath{stroke}%
\end{pgfscope}%
\begin{pgfscope}%
\pgfpathrectangle{\pgfqpoint{0.709829in}{0.654666in}}{\pgfqpoint{5.110171in}{0.887537in}}%
\pgfusepath{clip}%
\pgfsetroundcap%
\pgfsetroundjoin%
\pgfsetlinewidth{1.003750pt}%
\definecolor{currentstroke}{rgb}{0.866667,0.517647,0.321569}%
\pgfsetstrokecolor{currentstroke}%
\pgfsetdash{}{0pt}%
\pgfpathmoveto{\pgfqpoint{0.942110in}{1.130149in}}%
\pgfpathlineto{\pgfqpoint{0.944433in}{1.138133in}}%
\pgfpathlineto{\pgfqpoint{0.949081in}{1.105761in}}%
\pgfpathlineto{\pgfqpoint{0.951405in}{1.115211in}}%
\pgfpathlineto{\pgfqpoint{0.953729in}{1.103438in}}%
\pgfpathlineto{\pgfqpoint{0.956053in}{1.111971in}}%
\pgfpathlineto{\pgfqpoint{0.958377in}{1.115435in}}%
\pgfpathlineto{\pgfqpoint{0.960701in}{1.158994in}}%
\pgfpathlineto{\pgfqpoint{0.963025in}{1.097621in}}%
\pgfpathlineto{\pgfqpoint{0.965349in}{1.110533in}}%
\pgfpathlineto{\pgfqpoint{0.967673in}{1.067295in}}%
\pgfpathlineto{\pgfqpoint{0.969997in}{1.134492in}}%
\pgfpathlineto{\pgfqpoint{0.972321in}{1.109404in}}%
\pgfpathlineto{\pgfqpoint{0.974645in}{1.071112in}}%
\pgfpathlineto{\pgfqpoint{0.976969in}{1.109582in}}%
\pgfpathlineto{\pgfqpoint{0.979293in}{1.082641in}}%
\pgfpathlineto{\pgfqpoint{0.981617in}{1.128815in}}%
\pgfpathlineto{\pgfqpoint{0.983941in}{1.118183in}}%
\pgfpathlineto{\pgfqpoint{0.986265in}{1.130152in}}%
\pgfpathlineto{\pgfqpoint{0.990913in}{1.086382in}}%
\pgfpathlineto{\pgfqpoint{0.993237in}{1.089226in}}%
\pgfpathlineto{\pgfqpoint{0.997885in}{1.157497in}}%
\pgfpathlineto{\pgfqpoint{1.000209in}{1.118227in}}%
\pgfpathlineto{\pgfqpoint{1.002533in}{1.125229in}}%
\pgfpathlineto{\pgfqpoint{1.004857in}{1.152061in}}%
\pgfpathlineto{\pgfqpoint{1.007181in}{1.161548in}}%
\pgfpathlineto{\pgfqpoint{1.009505in}{1.135396in}}%
\pgfpathlineto{\pgfqpoint{1.011829in}{1.125342in}}%
\pgfpathlineto{\pgfqpoint{1.014153in}{1.159430in}}%
\pgfpathlineto{\pgfqpoint{1.016476in}{1.135642in}}%
\pgfpathlineto{\pgfqpoint{1.018800in}{1.145673in}}%
\pgfpathlineto{\pgfqpoint{1.021124in}{1.113245in}}%
\pgfpathlineto{\pgfqpoint{1.023448in}{1.188963in}}%
\pgfpathlineto{\pgfqpoint{1.025772in}{1.183773in}}%
\pgfpathlineto{\pgfqpoint{1.028096in}{1.169935in}}%
\pgfpathlineto{\pgfqpoint{1.030420in}{1.140964in}}%
\pgfpathlineto{\pgfqpoint{1.032744in}{1.145428in}}%
\pgfpathlineto{\pgfqpoint{1.037392in}{1.184768in}}%
\pgfpathlineto{\pgfqpoint{1.039716in}{1.139948in}}%
\pgfpathlineto{\pgfqpoint{1.042040in}{1.167589in}}%
\pgfpathlineto{\pgfqpoint{1.044364in}{1.143215in}}%
\pgfpathlineto{\pgfqpoint{1.046688in}{1.172432in}}%
\pgfpathlineto{\pgfqpoint{1.049012in}{1.172931in}}%
\pgfpathlineto{\pgfqpoint{1.051336in}{1.171080in}}%
\pgfpathlineto{\pgfqpoint{1.053660in}{1.212966in}}%
\pgfpathlineto{\pgfqpoint{1.055984in}{1.193296in}}%
\pgfpathlineto{\pgfqpoint{1.058308in}{1.159822in}}%
\pgfpathlineto{\pgfqpoint{1.060632in}{1.149988in}}%
\pgfpathlineto{\pgfqpoint{1.062956in}{1.197061in}}%
\pgfpathlineto{\pgfqpoint{1.065280in}{1.161849in}}%
\pgfpathlineto{\pgfqpoint{1.067604in}{1.192447in}}%
\pgfpathlineto{\pgfqpoint{1.069928in}{1.194799in}}%
\pgfpathlineto{\pgfqpoint{1.072252in}{1.199219in}}%
\pgfpathlineto{\pgfqpoint{1.076900in}{1.152590in}}%
\pgfpathlineto{\pgfqpoint{1.081548in}{1.176177in}}%
\pgfpathlineto{\pgfqpoint{1.083872in}{1.130884in}}%
\pgfpathlineto{\pgfqpoint{1.086195in}{1.158169in}}%
\pgfpathlineto{\pgfqpoint{1.088519in}{1.221117in}}%
\pgfpathlineto{\pgfqpoint{1.090843in}{1.187398in}}%
\pgfpathlineto{\pgfqpoint{1.093167in}{1.179962in}}%
\pgfpathlineto{\pgfqpoint{1.095491in}{1.230843in}}%
\pgfpathlineto{\pgfqpoint{1.097815in}{1.180086in}}%
\pgfpathlineto{\pgfqpoint{1.100139in}{1.208789in}}%
\pgfpathlineto{\pgfqpoint{1.102463in}{1.174299in}}%
\pgfpathlineto{\pgfqpoint{1.104787in}{1.211276in}}%
\pgfpathlineto{\pgfqpoint{1.107111in}{1.186429in}}%
\pgfpathlineto{\pgfqpoint{1.111759in}{1.198179in}}%
\pgfpathlineto{\pgfqpoint{1.114083in}{1.186166in}}%
\pgfpathlineto{\pgfqpoint{1.116407in}{1.199484in}}%
\pgfpathlineto{\pgfqpoint{1.118731in}{1.187768in}}%
\pgfpathlineto{\pgfqpoint{1.121055in}{1.191754in}}%
\pgfpathlineto{\pgfqpoint{1.125703in}{1.156720in}}%
\pgfpathlineto{\pgfqpoint{1.128027in}{1.191819in}}%
\pgfpathlineto{\pgfqpoint{1.130351in}{1.200466in}}%
\pgfpathlineto{\pgfqpoint{1.132675in}{1.229006in}}%
\pgfpathlineto{\pgfqpoint{1.134999in}{1.174520in}}%
\pgfpathlineto{\pgfqpoint{1.137323in}{1.213042in}}%
\pgfpathlineto{\pgfqpoint{1.139647in}{1.227895in}}%
\pgfpathlineto{\pgfqpoint{1.141971in}{1.175216in}}%
\pgfpathlineto{\pgfqpoint{1.144295in}{1.203194in}}%
\pgfpathlineto{\pgfqpoint{1.146619in}{1.170595in}}%
\pgfpathlineto{\pgfqpoint{1.148943in}{1.217515in}}%
\pgfpathlineto{\pgfqpoint{1.151267in}{1.232311in}}%
\pgfpathlineto{\pgfqpoint{1.153591in}{1.227380in}}%
\pgfpathlineto{\pgfqpoint{1.155914in}{1.229460in}}%
\pgfpathlineto{\pgfqpoint{1.158238in}{1.264187in}}%
\pgfpathlineto{\pgfqpoint{1.160562in}{1.200240in}}%
\pgfpathlineto{\pgfqpoint{1.162886in}{1.237187in}}%
\pgfpathlineto{\pgfqpoint{1.165210in}{1.233393in}}%
\pgfpathlineto{\pgfqpoint{1.167534in}{1.205929in}}%
\pgfpathlineto{\pgfqpoint{1.169858in}{1.217702in}}%
\pgfpathlineto{\pgfqpoint{1.172182in}{1.238920in}}%
\pgfpathlineto{\pgfqpoint{1.174506in}{1.192594in}}%
\pgfpathlineto{\pgfqpoint{1.176830in}{1.210279in}}%
\pgfpathlineto{\pgfqpoint{1.179154in}{1.202688in}}%
\pgfpathlineto{\pgfqpoint{1.181478in}{1.225633in}}%
\pgfpathlineto{\pgfqpoint{1.183802in}{1.207518in}}%
\pgfpathlineto{\pgfqpoint{1.186126in}{1.233077in}}%
\pgfpathlineto{\pgfqpoint{1.188450in}{1.230877in}}%
\pgfpathlineto{\pgfqpoint{1.190774in}{1.214268in}}%
\pgfpathlineto{\pgfqpoint{1.193098in}{1.209234in}}%
\pgfpathlineto{\pgfqpoint{1.195422in}{1.285684in}}%
\pgfpathlineto{\pgfqpoint{1.202394in}{1.197499in}}%
\pgfpathlineto{\pgfqpoint{1.204718in}{1.271116in}}%
\pgfpathlineto{\pgfqpoint{1.207042in}{1.227428in}}%
\pgfpathlineto{\pgfqpoint{1.209366in}{1.234511in}}%
\pgfpathlineto{\pgfqpoint{1.211690in}{1.252170in}}%
\pgfpathlineto{\pgfqpoint{1.214014in}{1.205872in}}%
\pgfpathlineto{\pgfqpoint{1.216338in}{1.229681in}}%
\pgfpathlineto{\pgfqpoint{1.218662in}{1.233631in}}%
\pgfpathlineto{\pgfqpoint{1.220986in}{1.258613in}}%
\pgfpathlineto{\pgfqpoint{1.225633in}{1.242123in}}%
\pgfpathlineto{\pgfqpoint{1.227957in}{1.220517in}}%
\pgfpathlineto{\pgfqpoint{1.230281in}{1.265648in}}%
\pgfpathlineto{\pgfqpoint{1.232605in}{1.260653in}}%
\pgfpathlineto{\pgfqpoint{1.234929in}{1.245361in}}%
\pgfpathlineto{\pgfqpoint{1.237253in}{1.268214in}}%
\pgfpathlineto{\pgfqpoint{1.239577in}{1.259483in}}%
\pgfpathlineto{\pgfqpoint{1.241901in}{1.217069in}}%
\pgfpathlineto{\pgfqpoint{1.244225in}{1.267961in}}%
\pgfpathlineto{\pgfqpoint{1.246549in}{1.221162in}}%
\pgfpathlineto{\pgfqpoint{1.248873in}{1.271147in}}%
\pgfpathlineto{\pgfqpoint{1.251197in}{1.277360in}}%
\pgfpathlineto{\pgfqpoint{1.253521in}{1.273386in}}%
\pgfpathlineto{\pgfqpoint{1.258169in}{1.257148in}}%
\pgfpathlineto{\pgfqpoint{1.262817in}{1.295370in}}%
\pgfpathlineto{\pgfqpoint{1.265141in}{1.258378in}}%
\pgfpathlineto{\pgfqpoint{1.267465in}{1.279709in}}%
\pgfpathlineto{\pgfqpoint{1.269789in}{1.264270in}}%
\pgfpathlineto{\pgfqpoint{1.272113in}{1.290578in}}%
\pgfpathlineto{\pgfqpoint{1.274437in}{1.269780in}}%
\pgfpathlineto{\pgfqpoint{1.276761in}{1.281049in}}%
\pgfpathlineto{\pgfqpoint{1.279085in}{1.267578in}}%
\pgfpathlineto{\pgfqpoint{1.281409in}{1.280943in}}%
\pgfpathlineto{\pgfqpoint{1.283733in}{1.238907in}}%
\pgfpathlineto{\pgfqpoint{1.286057in}{1.283851in}}%
\pgfpathlineto{\pgfqpoint{1.288381in}{1.246142in}}%
\pgfpathlineto{\pgfqpoint{1.290705in}{1.239691in}}%
\pgfpathlineto{\pgfqpoint{1.293029in}{1.270334in}}%
\pgfpathlineto{\pgfqpoint{1.295353in}{1.262295in}}%
\pgfpathlineto{\pgfqpoint{1.297676in}{1.267042in}}%
\pgfpathlineto{\pgfqpoint{1.300000in}{1.256721in}}%
\pgfpathlineto{\pgfqpoint{1.302324in}{1.322841in}}%
\pgfpathlineto{\pgfqpoint{1.304648in}{1.217066in}}%
\pgfpathlineto{\pgfqpoint{1.309296in}{1.302792in}}%
\pgfpathlineto{\pgfqpoint{1.311620in}{1.245554in}}%
\pgfpathlineto{\pgfqpoint{1.313944in}{1.282562in}}%
\pgfpathlineto{\pgfqpoint{1.316268in}{1.257546in}}%
\pgfpathlineto{\pgfqpoint{1.318592in}{1.300511in}}%
\pgfpathlineto{\pgfqpoint{1.320916in}{1.277251in}}%
\pgfpathlineto{\pgfqpoint{1.323240in}{1.289137in}}%
\pgfpathlineto{\pgfqpoint{1.327888in}{1.247802in}}%
\pgfpathlineto{\pgfqpoint{1.330212in}{1.266120in}}%
\pgfpathlineto{\pgfqpoint{1.332536in}{1.264715in}}%
\pgfpathlineto{\pgfqpoint{1.334860in}{1.239350in}}%
\pgfpathlineto{\pgfqpoint{1.337184in}{1.286282in}}%
\pgfpathlineto{\pgfqpoint{1.339508in}{1.229536in}}%
\pgfpathlineto{\pgfqpoint{1.344156in}{1.309783in}}%
\pgfpathlineto{\pgfqpoint{1.346480in}{1.236762in}}%
\pgfpathlineto{\pgfqpoint{1.348804in}{1.248477in}}%
\pgfpathlineto{\pgfqpoint{1.351128in}{1.217145in}}%
\pgfpathlineto{\pgfqpoint{1.353452in}{1.276194in}}%
\pgfpathlineto{\pgfqpoint{1.355776in}{1.244920in}}%
\pgfpathlineto{\pgfqpoint{1.358100in}{1.267749in}}%
\pgfpathlineto{\pgfqpoint{1.360424in}{1.279411in}}%
\pgfpathlineto{\pgfqpoint{1.362748in}{1.305506in}}%
\pgfpathlineto{\pgfqpoint{1.365072in}{1.301976in}}%
\pgfpathlineto{\pgfqpoint{1.367395in}{1.256687in}}%
\pgfpathlineto{\pgfqpoint{1.369719in}{1.292485in}}%
\pgfpathlineto{\pgfqpoint{1.372043in}{1.306263in}}%
\pgfpathlineto{\pgfqpoint{1.374367in}{1.243164in}}%
\pgfpathlineto{\pgfqpoint{1.376691in}{1.269504in}}%
\pgfpathlineto{\pgfqpoint{1.379015in}{1.253975in}}%
\pgfpathlineto{\pgfqpoint{1.381339in}{1.283786in}}%
\pgfpathlineto{\pgfqpoint{1.383663in}{1.285645in}}%
\pgfpathlineto{\pgfqpoint{1.385987in}{1.285246in}}%
\pgfpathlineto{\pgfqpoint{1.388311in}{1.255710in}}%
\pgfpathlineto{\pgfqpoint{1.390635in}{1.300488in}}%
\pgfpathlineto{\pgfqpoint{1.392959in}{1.242453in}}%
\pgfpathlineto{\pgfqpoint{1.397607in}{1.269504in}}%
\pgfpathlineto{\pgfqpoint{1.399931in}{1.279377in}}%
\pgfpathlineto{\pgfqpoint{1.402255in}{1.275306in}}%
\pgfpathlineto{\pgfqpoint{1.404579in}{1.266440in}}%
\pgfpathlineto{\pgfqpoint{1.406903in}{1.233162in}}%
\pgfpathlineto{\pgfqpoint{1.409227in}{1.277302in}}%
\pgfpathlineto{\pgfqpoint{1.413875in}{1.264324in}}%
\pgfpathlineto{\pgfqpoint{1.416199in}{1.277411in}}%
\pgfpathlineto{\pgfqpoint{1.418523in}{1.274578in}}%
\pgfpathlineto{\pgfqpoint{1.420847in}{1.298543in}}%
\pgfpathlineto{\pgfqpoint{1.423171in}{1.258097in}}%
\pgfpathlineto{\pgfqpoint{1.425495in}{1.260728in}}%
\pgfpathlineto{\pgfqpoint{1.427819in}{1.301602in}}%
\pgfpathlineto{\pgfqpoint{1.430143in}{1.249718in}}%
\pgfpathlineto{\pgfqpoint{1.432467in}{1.257224in}}%
\pgfpathlineto{\pgfqpoint{1.434791in}{1.242613in}}%
\pgfpathlineto{\pgfqpoint{1.437114in}{1.292843in}}%
\pgfpathlineto{\pgfqpoint{1.439438in}{1.288485in}}%
\pgfpathlineto{\pgfqpoint{1.441762in}{1.252457in}}%
\pgfpathlineto{\pgfqpoint{1.444086in}{1.243857in}}%
\pgfpathlineto{\pgfqpoint{1.446410in}{1.209365in}}%
\pgfpathlineto{\pgfqpoint{1.448734in}{1.264214in}}%
\pgfpathlineto{\pgfqpoint{1.451058in}{1.232541in}}%
\pgfpathlineto{\pgfqpoint{1.453382in}{1.242767in}}%
\pgfpathlineto{\pgfqpoint{1.455706in}{1.262529in}}%
\pgfpathlineto{\pgfqpoint{1.458030in}{1.255034in}}%
\pgfpathlineto{\pgfqpoint{1.460354in}{1.267985in}}%
\pgfpathlineto{\pgfqpoint{1.462678in}{1.217251in}}%
\pgfpathlineto{\pgfqpoint{1.465002in}{1.267650in}}%
\pgfpathlineto{\pgfqpoint{1.467326in}{1.285889in}}%
\pgfpathlineto{\pgfqpoint{1.469650in}{1.277331in}}%
\pgfpathlineto{\pgfqpoint{1.474298in}{1.218222in}}%
\pgfpathlineto{\pgfqpoint{1.476622in}{1.257848in}}%
\pgfpathlineto{\pgfqpoint{1.478946in}{1.214915in}}%
\pgfpathlineto{\pgfqpoint{1.483594in}{1.279151in}}%
\pgfpathlineto{\pgfqpoint{1.485918in}{1.229660in}}%
\pgfpathlineto{\pgfqpoint{1.488242in}{1.293264in}}%
\pgfpathlineto{\pgfqpoint{1.492890in}{1.264859in}}%
\pgfpathlineto{\pgfqpoint{1.495214in}{1.276553in}}%
\pgfpathlineto{\pgfqpoint{1.497538in}{1.233173in}}%
\pgfpathlineto{\pgfqpoint{1.502186in}{1.241088in}}%
\pgfpathlineto{\pgfqpoint{1.504510in}{1.219454in}}%
\pgfpathlineto{\pgfqpoint{1.506834in}{1.257348in}}%
\pgfpathlineto{\pgfqpoint{1.509157in}{1.233462in}}%
\pgfpathlineto{\pgfqpoint{1.511481in}{1.190574in}}%
\pgfpathlineto{\pgfqpoint{1.513805in}{1.241051in}}%
\pgfpathlineto{\pgfqpoint{1.516129in}{1.209115in}}%
\pgfpathlineto{\pgfqpoint{1.518453in}{1.219899in}}%
\pgfpathlineto{\pgfqpoint{1.525425in}{1.327586in}}%
\pgfpathlineto{\pgfqpoint{1.527749in}{1.306406in}}%
\pgfpathlineto{\pgfqpoint{1.530073in}{1.255690in}}%
\pgfpathlineto{\pgfqpoint{1.532397in}{1.303224in}}%
\pgfpathlineto{\pgfqpoint{1.534721in}{1.262810in}}%
\pgfpathlineto{\pgfqpoint{1.537045in}{1.326751in}}%
\pgfpathlineto{\pgfqpoint{1.539369in}{1.287691in}}%
\pgfpathlineto{\pgfqpoint{1.541693in}{1.280829in}}%
\pgfpathlineto{\pgfqpoint{1.544017in}{1.280323in}}%
\pgfpathlineto{\pgfqpoint{1.546341in}{1.312187in}}%
\pgfpathlineto{\pgfqpoint{1.548665in}{1.249443in}}%
\pgfpathlineto{\pgfqpoint{1.550989in}{1.289696in}}%
\pgfpathlineto{\pgfqpoint{1.553313in}{1.279338in}}%
\pgfpathlineto{\pgfqpoint{1.555637in}{1.253510in}}%
\pgfpathlineto{\pgfqpoint{1.557961in}{1.256054in}}%
\pgfpathlineto{\pgfqpoint{1.560285in}{1.267427in}}%
\pgfpathlineto{\pgfqpoint{1.562609in}{1.290798in}}%
\pgfpathlineto{\pgfqpoint{1.564933in}{1.291443in}}%
\pgfpathlineto{\pgfqpoint{1.569581in}{1.256018in}}%
\pgfpathlineto{\pgfqpoint{1.574229in}{1.247926in}}%
\pgfpathlineto{\pgfqpoint{1.576553in}{1.266632in}}%
\pgfpathlineto{\pgfqpoint{1.578876in}{1.248436in}}%
\pgfpathlineto{\pgfqpoint{1.583524in}{1.293723in}}%
\pgfpathlineto{\pgfqpoint{1.585848in}{1.282742in}}%
\pgfpathlineto{\pgfqpoint{1.588172in}{1.258455in}}%
\pgfpathlineto{\pgfqpoint{1.590496in}{1.249837in}}%
\pgfpathlineto{\pgfqpoint{1.592820in}{1.263230in}}%
\pgfpathlineto{\pgfqpoint{1.595144in}{1.301243in}}%
\pgfpathlineto{\pgfqpoint{1.597468in}{1.272707in}}%
\pgfpathlineto{\pgfqpoint{1.602116in}{1.248347in}}%
\pgfpathlineto{\pgfqpoint{1.604440in}{1.245762in}}%
\pgfpathlineto{\pgfqpoint{1.606764in}{1.236938in}}%
\pgfpathlineto{\pgfqpoint{1.609088in}{1.251248in}}%
\pgfpathlineto{\pgfqpoint{1.613736in}{1.255381in}}%
\pgfpathlineto{\pgfqpoint{1.616060in}{1.268879in}}%
\pgfpathlineto{\pgfqpoint{1.618384in}{1.220829in}}%
\pgfpathlineto{\pgfqpoint{1.620708in}{1.238112in}}%
\pgfpathlineto{\pgfqpoint{1.623032in}{1.196020in}}%
\pgfpathlineto{\pgfqpoint{1.625356in}{1.239612in}}%
\pgfpathlineto{\pgfqpoint{1.627680in}{1.213957in}}%
\pgfpathlineto{\pgfqpoint{1.630004in}{1.242994in}}%
\pgfpathlineto{\pgfqpoint{1.632328in}{1.240646in}}%
\pgfpathlineto{\pgfqpoint{1.634652in}{1.194710in}}%
\pgfpathlineto{\pgfqpoint{1.636976in}{1.188711in}}%
\pgfpathlineto{\pgfqpoint{1.639300in}{1.198512in}}%
\pgfpathlineto{\pgfqpoint{1.641624in}{1.233281in}}%
\pgfpathlineto{\pgfqpoint{1.643948in}{1.217349in}}%
\pgfpathlineto{\pgfqpoint{1.646272in}{1.255652in}}%
\pgfpathlineto{\pgfqpoint{1.648595in}{1.251181in}}%
\pgfpathlineto{\pgfqpoint{1.650919in}{1.242110in}}%
\pgfpathlineto{\pgfqpoint{1.653243in}{1.227603in}}%
\pgfpathlineto{\pgfqpoint{1.655567in}{1.241486in}}%
\pgfpathlineto{\pgfqpoint{1.657891in}{1.210069in}}%
\pgfpathlineto{\pgfqpoint{1.660215in}{1.207305in}}%
\pgfpathlineto{\pgfqpoint{1.662539in}{1.248273in}}%
\pgfpathlineto{\pgfqpoint{1.664863in}{1.182115in}}%
\pgfpathlineto{\pgfqpoint{1.667187in}{1.182662in}}%
\pgfpathlineto{\pgfqpoint{1.671835in}{1.208540in}}%
\pgfpathlineto{\pgfqpoint{1.674159in}{1.209086in}}%
\pgfpathlineto{\pgfqpoint{1.676483in}{1.214519in}}%
\pgfpathlineto{\pgfqpoint{1.678807in}{1.202478in}}%
\pgfpathlineto{\pgfqpoint{1.681131in}{1.210180in}}%
\pgfpathlineto{\pgfqpoint{1.683455in}{1.173547in}}%
\pgfpathlineto{\pgfqpoint{1.685779in}{1.185305in}}%
\pgfpathlineto{\pgfqpoint{1.688103in}{1.214072in}}%
\pgfpathlineto{\pgfqpoint{1.690427in}{1.188186in}}%
\pgfpathlineto{\pgfqpoint{1.692751in}{1.210173in}}%
\pgfpathlineto{\pgfqpoint{1.695075in}{1.189660in}}%
\pgfpathlineto{\pgfqpoint{1.697399in}{1.182231in}}%
\pgfpathlineto{\pgfqpoint{1.699723in}{1.183727in}}%
\pgfpathlineto{\pgfqpoint{1.702047in}{1.181747in}}%
\pgfpathlineto{\pgfqpoint{1.704371in}{1.160599in}}%
\pgfpathlineto{\pgfqpoint{1.706695in}{1.192317in}}%
\pgfpathlineto{\pgfqpoint{1.709019in}{1.188045in}}%
\pgfpathlineto{\pgfqpoint{1.711343in}{1.181308in}}%
\pgfpathlineto{\pgfqpoint{1.715991in}{1.163752in}}%
\pgfpathlineto{\pgfqpoint{1.718314in}{1.162034in}}%
\pgfpathlineto{\pgfqpoint{1.720638in}{1.173000in}}%
\pgfpathlineto{\pgfqpoint{1.722962in}{1.166435in}}%
\pgfpathlineto{\pgfqpoint{1.725286in}{1.201978in}}%
\pgfpathlineto{\pgfqpoint{1.727610in}{1.149748in}}%
\pgfpathlineto{\pgfqpoint{1.729934in}{1.125200in}}%
\pgfpathlineto{\pgfqpoint{1.732258in}{1.180923in}}%
\pgfpathlineto{\pgfqpoint{1.734582in}{1.129970in}}%
\pgfpathlineto{\pgfqpoint{1.741554in}{1.150667in}}%
\pgfpathlineto{\pgfqpoint{1.743878in}{1.171669in}}%
\pgfpathlineto{\pgfqpoint{1.748526in}{1.152298in}}%
\pgfpathlineto{\pgfqpoint{1.750850in}{1.175858in}}%
\pgfpathlineto{\pgfqpoint{1.753174in}{1.130968in}}%
\pgfpathlineto{\pgfqpoint{1.755498in}{1.129052in}}%
\pgfpathlineto{\pgfqpoint{1.757822in}{1.153676in}}%
\pgfpathlineto{\pgfqpoint{1.760146in}{1.105147in}}%
\pgfpathlineto{\pgfqpoint{1.764794in}{1.139619in}}%
\pgfpathlineto{\pgfqpoint{1.767118in}{1.140740in}}%
\pgfpathlineto{\pgfqpoint{1.769442in}{1.135893in}}%
\pgfpathlineto{\pgfqpoint{1.771766in}{1.098010in}}%
\pgfpathlineto{\pgfqpoint{1.774090in}{1.125857in}}%
\pgfpathlineto{\pgfqpoint{1.776414in}{1.104517in}}%
\pgfpathlineto{\pgfqpoint{1.778738in}{1.115057in}}%
\pgfpathlineto{\pgfqpoint{1.781062in}{1.100863in}}%
\pgfpathlineto{\pgfqpoint{1.783386in}{1.098753in}}%
\pgfpathlineto{\pgfqpoint{1.785710in}{1.121005in}}%
\pgfpathlineto{\pgfqpoint{1.788034in}{1.081178in}}%
\pgfpathlineto{\pgfqpoint{1.790357in}{1.109194in}}%
\pgfpathlineto{\pgfqpoint{1.792681in}{1.117252in}}%
\pgfpathlineto{\pgfqpoint{1.797329in}{1.095956in}}%
\pgfpathlineto{\pgfqpoint{1.799653in}{1.113125in}}%
\pgfpathlineto{\pgfqpoint{1.801977in}{1.098319in}}%
\pgfpathlineto{\pgfqpoint{1.804301in}{1.061033in}}%
\pgfpathlineto{\pgfqpoint{1.806625in}{1.094564in}}%
\pgfpathlineto{\pgfqpoint{1.808949in}{1.091511in}}%
\pgfpathlineto{\pgfqpoint{1.811273in}{1.075125in}}%
\pgfpathlineto{\pgfqpoint{1.813597in}{1.105303in}}%
\pgfpathlineto{\pgfqpoint{1.815921in}{1.095842in}}%
\pgfpathlineto{\pgfqpoint{1.822893in}{1.087558in}}%
\pgfpathlineto{\pgfqpoint{1.825217in}{1.059162in}}%
\pgfpathlineto{\pgfqpoint{1.827541in}{1.058282in}}%
\pgfpathlineto{\pgfqpoint{1.829865in}{1.080547in}}%
\pgfpathlineto{\pgfqpoint{1.832189in}{1.078340in}}%
\pgfpathlineto{\pgfqpoint{1.834513in}{1.083040in}}%
\pgfpathlineto{\pgfqpoint{1.836837in}{1.108495in}}%
\pgfpathlineto{\pgfqpoint{1.839161in}{1.116918in}}%
\pgfpathlineto{\pgfqpoint{1.846133in}{1.077736in}}%
\pgfpathlineto{\pgfqpoint{1.848457in}{1.075731in}}%
\pgfpathlineto{\pgfqpoint{1.850781in}{1.058001in}}%
\pgfpathlineto{\pgfqpoint{1.853105in}{1.072863in}}%
\pgfpathlineto{\pgfqpoint{1.855429in}{1.078303in}}%
\pgfpathlineto{\pgfqpoint{1.857753in}{1.055175in}}%
\pgfpathlineto{\pgfqpoint{1.860076in}{1.053448in}}%
\pgfpathlineto{\pgfqpoint{1.862400in}{1.078596in}}%
\pgfpathlineto{\pgfqpoint{1.867048in}{1.079692in}}%
\pgfpathlineto{\pgfqpoint{1.869372in}{1.043235in}}%
\pgfpathlineto{\pgfqpoint{1.871696in}{1.034281in}}%
\pgfpathlineto{\pgfqpoint{1.874020in}{1.064193in}}%
\pgfpathlineto{\pgfqpoint{1.876344in}{1.065780in}}%
\pgfpathlineto{\pgfqpoint{1.880992in}{1.042107in}}%
\pgfpathlineto{\pgfqpoint{1.883316in}{1.037990in}}%
\pgfpathlineto{\pgfqpoint{1.885640in}{1.011802in}}%
\pgfpathlineto{\pgfqpoint{1.890288in}{1.026634in}}%
\pgfpathlineto{\pgfqpoint{1.892612in}{1.029572in}}%
\pgfpathlineto{\pgfqpoint{1.897260in}{1.069216in}}%
\pgfpathlineto{\pgfqpoint{1.899584in}{1.045095in}}%
\pgfpathlineto{\pgfqpoint{1.901908in}{0.991756in}}%
\pgfpathlineto{\pgfqpoint{1.904232in}{1.031367in}}%
\pgfpathlineto{\pgfqpoint{1.906556in}{1.028474in}}%
\pgfpathlineto{\pgfqpoint{1.908880in}{1.015773in}}%
\pgfpathlineto{\pgfqpoint{1.911204in}{1.017548in}}%
\pgfpathlineto{\pgfqpoint{1.913528in}{1.009794in}}%
\pgfpathlineto{\pgfqpoint{1.915852in}{1.037972in}}%
\pgfpathlineto{\pgfqpoint{1.918176in}{1.016132in}}%
\pgfpathlineto{\pgfqpoint{1.920500in}{1.049712in}}%
\pgfpathlineto{\pgfqpoint{1.922824in}{0.964163in}}%
\pgfpathlineto{\pgfqpoint{1.925148in}{1.051192in}}%
\pgfpathlineto{\pgfqpoint{1.927472in}{1.038978in}}%
\pgfpathlineto{\pgfqpoint{1.929795in}{1.014010in}}%
\pgfpathlineto{\pgfqpoint{1.932119in}{0.967002in}}%
\pgfpathlineto{\pgfqpoint{1.934443in}{0.980156in}}%
\pgfpathlineto{\pgfqpoint{1.936767in}{0.980499in}}%
\pgfpathlineto{\pgfqpoint{1.939091in}{0.979259in}}%
\pgfpathlineto{\pgfqpoint{1.941415in}{1.003233in}}%
\pgfpathlineto{\pgfqpoint{1.943739in}{0.962333in}}%
\pgfpathlineto{\pgfqpoint{1.946063in}{0.989497in}}%
\pgfpathlineto{\pgfqpoint{1.948387in}{0.945602in}}%
\pgfpathlineto{\pgfqpoint{1.950711in}{0.989715in}}%
\pgfpathlineto{\pgfqpoint{1.953035in}{0.952052in}}%
\pgfpathlineto{\pgfqpoint{1.955359in}{0.990401in}}%
\pgfpathlineto{\pgfqpoint{1.957683in}{0.964305in}}%
\pgfpathlineto{\pgfqpoint{1.960007in}{0.955806in}}%
\pgfpathlineto{\pgfqpoint{1.964655in}{0.965952in}}%
\pgfpathlineto{\pgfqpoint{1.966979in}{0.970112in}}%
\pgfpathlineto{\pgfqpoint{1.969303in}{0.995584in}}%
\pgfpathlineto{\pgfqpoint{1.971627in}{0.970730in}}%
\pgfpathlineto{\pgfqpoint{1.973951in}{0.965475in}}%
\pgfpathlineto{\pgfqpoint{1.978599in}{0.934624in}}%
\pgfpathlineto{\pgfqpoint{1.980923in}{0.937385in}}%
\pgfpathlineto{\pgfqpoint{1.983247in}{0.969447in}}%
\pgfpathlineto{\pgfqpoint{1.985571in}{0.930583in}}%
\pgfpathlineto{\pgfqpoint{1.987895in}{0.924441in}}%
\pgfpathlineto{\pgfqpoint{1.990219in}{0.996056in}}%
\pgfpathlineto{\pgfqpoint{1.992543in}{0.941485in}}%
\pgfpathlineto{\pgfqpoint{1.994867in}{0.930859in}}%
\pgfpathlineto{\pgfqpoint{1.997191in}{0.926091in}}%
\pgfpathlineto{\pgfqpoint{1.999514in}{0.932030in}}%
\pgfpathlineto{\pgfqpoint{2.001838in}{0.948767in}}%
\pgfpathlineto{\pgfqpoint{2.004162in}{0.943852in}}%
\pgfpathlineto{\pgfqpoint{2.006486in}{0.925186in}}%
\pgfpathlineto{\pgfqpoint{2.008810in}{0.946953in}}%
\pgfpathlineto{\pgfqpoint{2.013458in}{0.963860in}}%
\pgfpathlineto{\pgfqpoint{2.015782in}{0.924319in}}%
\pgfpathlineto{\pgfqpoint{2.018106in}{0.951276in}}%
\pgfpathlineto{\pgfqpoint{2.022754in}{0.905469in}}%
\pgfpathlineto{\pgfqpoint{2.025078in}{0.942643in}}%
\pgfpathlineto{\pgfqpoint{2.027402in}{0.945587in}}%
\pgfpathlineto{\pgfqpoint{2.029726in}{0.911961in}}%
\pgfpathlineto{\pgfqpoint{2.032050in}{0.927693in}}%
\pgfpathlineto{\pgfqpoint{2.034374in}{0.919613in}}%
\pgfpathlineto{\pgfqpoint{2.036698in}{0.902053in}}%
\pgfpathlineto{\pgfqpoint{2.039022in}{0.955569in}}%
\pgfpathlineto{\pgfqpoint{2.041346in}{0.888892in}}%
\pgfpathlineto{\pgfqpoint{2.043670in}{0.935609in}}%
\pgfpathlineto{\pgfqpoint{2.045994in}{0.921861in}}%
\pgfpathlineto{\pgfqpoint{2.050642in}{0.854936in}}%
\pgfpathlineto{\pgfqpoint{2.052966in}{0.891306in}}%
\pgfpathlineto{\pgfqpoint{2.055290in}{0.875518in}}%
\pgfpathlineto{\pgfqpoint{2.057614in}{0.907529in}}%
\pgfpathlineto{\pgfqpoint{2.059938in}{0.912140in}}%
\pgfpathlineto{\pgfqpoint{2.062262in}{0.830000in}}%
\pgfpathlineto{\pgfqpoint{2.064586in}{0.905725in}}%
\pgfpathlineto{\pgfqpoint{2.066910in}{0.900812in}}%
\pgfpathlineto{\pgfqpoint{2.071557in}{0.865090in}}%
\pgfpathlineto{\pgfqpoint{2.073881in}{0.909205in}}%
\pgfpathlineto{\pgfqpoint{2.076205in}{0.893672in}}%
\pgfpathlineto{\pgfqpoint{2.078529in}{0.904814in}}%
\pgfpathlineto{\pgfqpoint{2.080853in}{0.906421in}}%
\pgfpathlineto{\pgfqpoint{2.083177in}{0.878210in}}%
\pgfpathlineto{\pgfqpoint{2.085501in}{0.871158in}}%
\pgfpathlineto{\pgfqpoint{2.087825in}{0.883768in}}%
\pgfpathlineto{\pgfqpoint{2.090149in}{0.852756in}}%
\pgfpathlineto{\pgfqpoint{2.092473in}{0.875907in}}%
\pgfpathlineto{\pgfqpoint{2.094797in}{0.936986in}}%
\pgfpathlineto{\pgfqpoint{2.097121in}{0.861576in}}%
\pgfpathlineto{\pgfqpoint{2.099445in}{0.823730in}}%
\pgfpathlineto{\pgfqpoint{2.104093in}{0.927805in}}%
\pgfpathlineto{\pgfqpoint{2.106417in}{0.915510in}}%
\pgfpathlineto{\pgfqpoint{2.108741in}{0.925836in}}%
\pgfpathlineto{\pgfqpoint{2.111065in}{0.903063in}}%
\pgfpathlineto{\pgfqpoint{2.113389in}{0.893103in}}%
\pgfpathlineto{\pgfqpoint{2.115713in}{0.925378in}}%
\pgfpathlineto{\pgfqpoint{2.118037in}{0.919817in}}%
\pgfpathlineto{\pgfqpoint{2.120361in}{0.909237in}}%
\pgfpathlineto{\pgfqpoint{2.122685in}{0.910530in}}%
\pgfpathlineto{\pgfqpoint{2.125009in}{0.950030in}}%
\pgfpathlineto{\pgfqpoint{2.127333in}{0.877092in}}%
\pgfpathlineto{\pgfqpoint{2.129657in}{0.920001in}}%
\pgfpathlineto{\pgfqpoint{2.131981in}{0.877222in}}%
\pgfpathlineto{\pgfqpoint{2.136629in}{0.933672in}}%
\pgfpathlineto{\pgfqpoint{2.138953in}{0.929032in}}%
\pgfpathlineto{\pgfqpoint{2.141276in}{0.893683in}}%
\pgfpathlineto{\pgfqpoint{2.143600in}{0.904304in}}%
\pgfpathlineto{\pgfqpoint{2.145924in}{0.904256in}}%
\pgfpathlineto{\pgfqpoint{2.148248in}{0.910981in}}%
\pgfpathlineto{\pgfqpoint{2.150572in}{0.938720in}}%
\pgfpathlineto{\pgfqpoint{2.152896in}{0.922927in}}%
\pgfpathlineto{\pgfqpoint{2.157544in}{0.883109in}}%
\pgfpathlineto{\pgfqpoint{2.159868in}{0.865926in}}%
\pgfpathlineto{\pgfqpoint{2.162192in}{0.912577in}}%
\pgfpathlineto{\pgfqpoint{2.164516in}{0.873227in}}%
\pgfpathlineto{\pgfqpoint{2.166840in}{0.912569in}}%
\pgfpathlineto{\pgfqpoint{2.169164in}{0.870527in}}%
\pgfpathlineto{\pgfqpoint{2.171488in}{0.944632in}}%
\pgfpathlineto{\pgfqpoint{2.173812in}{0.866584in}}%
\pgfpathlineto{\pgfqpoint{2.176136in}{0.911733in}}%
\pgfpathlineto{\pgfqpoint{2.178460in}{0.864806in}}%
\pgfpathlineto{\pgfqpoint{2.180784in}{0.874946in}}%
\pgfpathlineto{\pgfqpoint{2.183108in}{0.867700in}}%
\pgfpathlineto{\pgfqpoint{2.185432in}{0.893850in}}%
\pgfpathlineto{\pgfqpoint{2.187756in}{0.880419in}}%
\pgfpathlineto{\pgfqpoint{2.190080in}{0.937937in}}%
\pgfpathlineto{\pgfqpoint{2.192404in}{0.869877in}}%
\pgfpathlineto{\pgfqpoint{2.194728in}{0.914354in}}%
\pgfpathlineto{\pgfqpoint{2.197052in}{0.868942in}}%
\pgfpathlineto{\pgfqpoint{2.199376in}{0.914487in}}%
\pgfpathlineto{\pgfqpoint{2.201700in}{0.873298in}}%
\pgfpathlineto{\pgfqpoint{2.204024in}{0.909124in}}%
\pgfpathlineto{\pgfqpoint{2.208672in}{0.873836in}}%
\pgfpathlineto{\pgfqpoint{2.210995in}{0.867263in}}%
\pgfpathlineto{\pgfqpoint{2.213319in}{0.904589in}}%
\pgfpathlineto{\pgfqpoint{2.215643in}{0.877838in}}%
\pgfpathlineto{\pgfqpoint{2.217967in}{0.881169in}}%
\pgfpathlineto{\pgfqpoint{2.220291in}{0.895458in}}%
\pgfpathlineto{\pgfqpoint{2.222615in}{0.892436in}}%
\pgfpathlineto{\pgfqpoint{2.224939in}{0.898885in}}%
\pgfpathlineto{\pgfqpoint{2.227263in}{0.876962in}}%
\pgfpathlineto{\pgfqpoint{2.229587in}{0.839332in}}%
\pgfpathlineto{\pgfqpoint{2.231911in}{0.881540in}}%
\pgfpathlineto{\pgfqpoint{2.234235in}{0.869359in}}%
\pgfpathlineto{\pgfqpoint{2.236559in}{0.894997in}}%
\pgfpathlineto{\pgfqpoint{2.238883in}{0.872755in}}%
\pgfpathlineto{\pgfqpoint{2.241207in}{0.862111in}}%
\pgfpathlineto{\pgfqpoint{2.243531in}{0.866421in}}%
\pgfpathlineto{\pgfqpoint{2.245855in}{0.910195in}}%
\pgfpathlineto{\pgfqpoint{2.250503in}{0.837722in}}%
\pgfpathlineto{\pgfqpoint{2.252827in}{0.909051in}}%
\pgfpathlineto{\pgfqpoint{2.257475in}{0.861584in}}%
\pgfpathlineto{\pgfqpoint{2.262123in}{0.833396in}}%
\pgfpathlineto{\pgfqpoint{2.266771in}{0.876560in}}%
\pgfpathlineto{\pgfqpoint{2.269095in}{0.861812in}}%
\pgfpathlineto{\pgfqpoint{2.271419in}{0.860005in}}%
\pgfpathlineto{\pgfqpoint{2.276067in}{0.895331in}}%
\pgfpathlineto{\pgfqpoint{2.278391in}{0.860763in}}%
\pgfpathlineto{\pgfqpoint{2.283038in}{0.880425in}}%
\pgfpathlineto{\pgfqpoint{2.285362in}{0.854183in}}%
\pgfpathlineto{\pgfqpoint{2.290010in}{0.836871in}}%
\pgfpathlineto{\pgfqpoint{2.292334in}{0.813692in}}%
\pgfpathlineto{\pgfqpoint{2.294658in}{0.813854in}}%
\pgfpathlineto{\pgfqpoint{2.299306in}{0.878900in}}%
\pgfpathlineto{\pgfqpoint{2.301630in}{0.859268in}}%
\pgfpathlineto{\pgfqpoint{2.303954in}{0.884470in}}%
\pgfpathlineto{\pgfqpoint{2.306278in}{0.858978in}}%
\pgfpathlineto{\pgfqpoint{2.308602in}{0.853580in}}%
\pgfpathlineto{\pgfqpoint{2.310926in}{0.868279in}}%
\pgfpathlineto{\pgfqpoint{2.313250in}{0.860688in}}%
\pgfpathlineto{\pgfqpoint{2.315574in}{0.872121in}}%
\pgfpathlineto{\pgfqpoint{2.317898in}{0.875655in}}%
\pgfpathlineto{\pgfqpoint{2.320222in}{0.854791in}}%
\pgfpathlineto{\pgfqpoint{2.322546in}{0.817161in}}%
\pgfpathlineto{\pgfqpoint{2.324870in}{0.866362in}}%
\pgfpathlineto{\pgfqpoint{2.327194in}{0.857743in}}%
\pgfpathlineto{\pgfqpoint{2.329518in}{0.869794in}}%
\pgfpathlineto{\pgfqpoint{2.331842in}{0.830172in}}%
\pgfpathlineto{\pgfqpoint{2.334166in}{0.856888in}}%
\pgfpathlineto{\pgfqpoint{2.338814in}{0.884744in}}%
\pgfpathlineto{\pgfqpoint{2.341138in}{0.815394in}}%
\pgfpathlineto{\pgfqpoint{2.343462in}{0.872843in}}%
\pgfpathlineto{\pgfqpoint{2.345786in}{0.874744in}}%
\pgfpathlineto{\pgfqpoint{2.348110in}{0.811962in}}%
\pgfpathlineto{\pgfqpoint{2.352757in}{0.842947in}}%
\pgfpathlineto{\pgfqpoint{2.355081in}{0.835388in}}%
\pgfpathlineto{\pgfqpoint{2.357405in}{0.847647in}}%
\pgfpathlineto{\pgfqpoint{2.359729in}{0.852361in}}%
\pgfpathlineto{\pgfqpoint{2.362053in}{0.843121in}}%
\pgfpathlineto{\pgfqpoint{2.364377in}{0.863714in}}%
\pgfpathlineto{\pgfqpoint{2.366701in}{0.812541in}}%
\pgfpathlineto{\pgfqpoint{2.371349in}{0.852406in}}%
\pgfpathlineto{\pgfqpoint{2.373673in}{0.851007in}}%
\pgfpathlineto{\pgfqpoint{2.375997in}{0.840611in}}%
\pgfpathlineto{\pgfqpoint{2.380645in}{0.863987in}}%
\pgfpathlineto{\pgfqpoint{2.382969in}{0.838356in}}%
\pgfpathlineto{\pgfqpoint{2.385293in}{0.825766in}}%
\pgfpathlineto{\pgfqpoint{2.387617in}{0.867140in}}%
\pgfpathlineto{\pgfqpoint{2.389941in}{0.835625in}}%
\pgfpathlineto{\pgfqpoint{2.392265in}{0.842342in}}%
\pgfpathlineto{\pgfqpoint{2.394589in}{0.842626in}}%
\pgfpathlineto{\pgfqpoint{2.396913in}{0.877373in}}%
\pgfpathlineto{\pgfqpoint{2.399237in}{0.843297in}}%
\pgfpathlineto{\pgfqpoint{2.401561in}{0.840452in}}%
\pgfpathlineto{\pgfqpoint{2.403885in}{0.890703in}}%
\pgfpathlineto{\pgfqpoint{2.406209in}{0.816506in}}%
\pgfpathlineto{\pgfqpoint{2.410857in}{0.901806in}}%
\pgfpathlineto{\pgfqpoint{2.415505in}{0.861383in}}%
\pgfpathlineto{\pgfqpoint{2.417829in}{0.872310in}}%
\pgfpathlineto{\pgfqpoint{2.420153in}{0.851512in}}%
\pgfpathlineto{\pgfqpoint{2.422476in}{0.816407in}}%
\pgfpathlineto{\pgfqpoint{2.424800in}{0.893836in}}%
\pgfpathlineto{\pgfqpoint{2.427124in}{0.815086in}}%
\pgfpathlineto{\pgfqpoint{2.429448in}{0.898691in}}%
\pgfpathlineto{\pgfqpoint{2.431772in}{0.866514in}}%
\pgfpathlineto{\pgfqpoint{2.434096in}{0.850660in}}%
\pgfpathlineto{\pgfqpoint{2.436420in}{0.884168in}}%
\pgfpathlineto{\pgfqpoint{2.438744in}{0.854060in}}%
\pgfpathlineto{\pgfqpoint{2.441068in}{0.895987in}}%
\pgfpathlineto{\pgfqpoint{2.443392in}{0.887375in}}%
\pgfpathlineto{\pgfqpoint{2.445716in}{0.891859in}}%
\pgfpathlineto{\pgfqpoint{2.448040in}{0.866604in}}%
\pgfpathlineto{\pgfqpoint{2.450364in}{0.919243in}}%
\pgfpathlineto{\pgfqpoint{2.452688in}{0.849477in}}%
\pgfpathlineto{\pgfqpoint{2.455012in}{0.857373in}}%
\pgfpathlineto{\pgfqpoint{2.459660in}{0.910111in}}%
\pgfpathlineto{\pgfqpoint{2.464308in}{0.837241in}}%
\pgfpathlineto{\pgfqpoint{2.466632in}{0.872864in}}%
\pgfpathlineto{\pgfqpoint{2.468956in}{0.857383in}}%
\pgfpathlineto{\pgfqpoint{2.471280in}{0.866296in}}%
\pgfpathlineto{\pgfqpoint{2.473604in}{0.886481in}}%
\pgfpathlineto{\pgfqpoint{2.475928in}{0.857867in}}%
\pgfpathlineto{\pgfqpoint{2.478252in}{0.859710in}}%
\pgfpathlineto{\pgfqpoint{2.480576in}{0.886039in}}%
\pgfpathlineto{\pgfqpoint{2.482900in}{0.874048in}}%
\pgfpathlineto{\pgfqpoint{2.485224in}{0.907670in}}%
\pgfpathlineto{\pgfqpoint{2.489872in}{0.858794in}}%
\pgfpathlineto{\pgfqpoint{2.492195in}{0.842961in}}%
\pgfpathlineto{\pgfqpoint{2.494519in}{0.881597in}}%
\pgfpathlineto{\pgfqpoint{2.496843in}{0.869117in}}%
\pgfpathlineto{\pgfqpoint{2.499167in}{0.893649in}}%
\pgfpathlineto{\pgfqpoint{2.501491in}{0.897924in}}%
\pgfpathlineto{\pgfqpoint{2.503815in}{0.899287in}}%
\pgfpathlineto{\pgfqpoint{2.506139in}{0.904877in}}%
\pgfpathlineto{\pgfqpoint{2.508463in}{0.879208in}}%
\pgfpathlineto{\pgfqpoint{2.510787in}{0.867809in}}%
\pgfpathlineto{\pgfqpoint{2.513111in}{0.888612in}}%
\pgfpathlineto{\pgfqpoint{2.515435in}{0.893075in}}%
\pgfpathlineto{\pgfqpoint{2.517759in}{0.921878in}}%
\pgfpathlineto{\pgfqpoint{2.520083in}{0.919615in}}%
\pgfpathlineto{\pgfqpoint{2.522407in}{0.864646in}}%
\pgfpathlineto{\pgfqpoint{2.524731in}{0.888212in}}%
\pgfpathlineto{\pgfqpoint{2.527055in}{0.879554in}}%
\pgfpathlineto{\pgfqpoint{2.529379in}{0.915722in}}%
\pgfpathlineto{\pgfqpoint{2.531703in}{0.873273in}}%
\pgfpathlineto{\pgfqpoint{2.534027in}{0.904841in}}%
\pgfpathlineto{\pgfqpoint{2.536351in}{0.899427in}}%
\pgfpathlineto{\pgfqpoint{2.538675in}{0.900541in}}%
\pgfpathlineto{\pgfqpoint{2.540999in}{0.871489in}}%
\pgfpathlineto{\pgfqpoint{2.543323in}{0.933681in}}%
\pgfpathlineto{\pgfqpoint{2.545647in}{0.933553in}}%
\pgfpathlineto{\pgfqpoint{2.547971in}{0.899395in}}%
\pgfpathlineto{\pgfqpoint{2.550295in}{0.898801in}}%
\pgfpathlineto{\pgfqpoint{2.552619in}{0.919011in}}%
\pgfpathlineto{\pgfqpoint{2.554943in}{0.886334in}}%
\pgfpathlineto{\pgfqpoint{2.557267in}{0.878633in}}%
\pgfpathlineto{\pgfqpoint{2.559591in}{0.954087in}}%
\pgfpathlineto{\pgfqpoint{2.561915in}{0.904175in}}%
\pgfpathlineto{\pgfqpoint{2.564238in}{0.938128in}}%
\pgfpathlineto{\pgfqpoint{2.566562in}{0.888263in}}%
\pgfpathlineto{\pgfqpoint{2.568886in}{0.910847in}}%
\pgfpathlineto{\pgfqpoint{2.571210in}{0.890852in}}%
\pgfpathlineto{\pgfqpoint{2.573534in}{0.929534in}}%
\pgfpathlineto{\pgfqpoint{2.575858in}{0.896384in}}%
\pgfpathlineto{\pgfqpoint{2.578182in}{0.920512in}}%
\pgfpathlineto{\pgfqpoint{2.580506in}{0.894185in}}%
\pgfpathlineto{\pgfqpoint{2.582830in}{0.902924in}}%
\pgfpathlineto{\pgfqpoint{2.585154in}{0.923627in}}%
\pgfpathlineto{\pgfqpoint{2.587478in}{0.893827in}}%
\pgfpathlineto{\pgfqpoint{2.589802in}{0.925177in}}%
\pgfpathlineto{\pgfqpoint{2.592126in}{0.907147in}}%
\pgfpathlineto{\pgfqpoint{2.594450in}{0.908389in}}%
\pgfpathlineto{\pgfqpoint{2.596774in}{0.932915in}}%
\pgfpathlineto{\pgfqpoint{2.599098in}{0.918640in}}%
\pgfpathlineto{\pgfqpoint{2.601422in}{0.914040in}}%
\pgfpathlineto{\pgfqpoint{2.603746in}{0.899769in}}%
\pgfpathlineto{\pgfqpoint{2.606070in}{0.941083in}}%
\pgfpathlineto{\pgfqpoint{2.610718in}{0.939968in}}%
\pgfpathlineto{\pgfqpoint{2.613042in}{0.921640in}}%
\pgfpathlineto{\pgfqpoint{2.615366in}{0.919203in}}%
\pgfpathlineto{\pgfqpoint{2.617690in}{0.961972in}}%
\pgfpathlineto{\pgfqpoint{2.620014in}{0.934074in}}%
\pgfpathlineto{\pgfqpoint{2.622338in}{0.957913in}}%
\pgfpathlineto{\pgfqpoint{2.624662in}{0.952073in}}%
\pgfpathlineto{\pgfqpoint{2.626986in}{0.963205in}}%
\pgfpathlineto{\pgfqpoint{2.629310in}{0.937702in}}%
\pgfpathlineto{\pgfqpoint{2.631634in}{0.938783in}}%
\pgfpathlineto{\pgfqpoint{2.633957in}{0.920794in}}%
\pgfpathlineto{\pgfqpoint{2.638605in}{0.944926in}}%
\pgfpathlineto{\pgfqpoint{2.640929in}{0.920966in}}%
\pgfpathlineto{\pgfqpoint{2.643253in}{0.933471in}}%
\pgfpathlineto{\pgfqpoint{2.645577in}{0.968512in}}%
\pgfpathlineto{\pgfqpoint{2.647901in}{0.978913in}}%
\pgfpathlineto{\pgfqpoint{2.652549in}{0.951516in}}%
\pgfpathlineto{\pgfqpoint{2.654873in}{0.996812in}}%
\pgfpathlineto{\pgfqpoint{2.657197in}{0.944201in}}%
\pgfpathlineto{\pgfqpoint{2.659521in}{0.958508in}}%
\pgfpathlineto{\pgfqpoint{2.661845in}{0.996758in}}%
\pgfpathlineto{\pgfqpoint{2.664169in}{0.956791in}}%
\pgfpathlineto{\pgfqpoint{2.666493in}{0.989096in}}%
\pgfpathlineto{\pgfqpoint{2.668817in}{0.938505in}}%
\pgfpathlineto{\pgfqpoint{2.671141in}{0.924314in}}%
\pgfpathlineto{\pgfqpoint{2.673465in}{0.986485in}}%
\pgfpathlineto{\pgfqpoint{2.675789in}{0.957970in}}%
\pgfpathlineto{\pgfqpoint{2.678113in}{0.980947in}}%
\pgfpathlineto{\pgfqpoint{2.680437in}{0.933164in}}%
\pgfpathlineto{\pgfqpoint{2.685085in}{1.042055in}}%
\pgfpathlineto{\pgfqpoint{2.687409in}{1.057695in}}%
\pgfpathlineto{\pgfqpoint{2.689733in}{1.029709in}}%
\pgfpathlineto{\pgfqpoint{2.692057in}{1.057585in}}%
\pgfpathlineto{\pgfqpoint{2.694381in}{1.003706in}}%
\pgfpathlineto{\pgfqpoint{2.696705in}{1.054799in}}%
\pgfpathlineto{\pgfqpoint{2.701353in}{1.039365in}}%
\pgfpathlineto{\pgfqpoint{2.703676in}{1.051972in}}%
\pgfpathlineto{\pgfqpoint{2.706000in}{1.079381in}}%
\pgfpathlineto{\pgfqpoint{2.708324in}{1.022817in}}%
\pgfpathlineto{\pgfqpoint{2.712972in}{1.029710in}}%
\pgfpathlineto{\pgfqpoint{2.717620in}{1.060622in}}%
\pgfpathlineto{\pgfqpoint{2.719944in}{1.055470in}}%
\pgfpathlineto{\pgfqpoint{2.722268in}{1.028589in}}%
\pgfpathlineto{\pgfqpoint{2.724592in}{1.042661in}}%
\pgfpathlineto{\pgfqpoint{2.726916in}{1.069273in}}%
\pgfpathlineto{\pgfqpoint{2.729240in}{1.056351in}}%
\pgfpathlineto{\pgfqpoint{2.731564in}{1.081862in}}%
\pgfpathlineto{\pgfqpoint{2.733888in}{1.062247in}}%
\pgfpathlineto{\pgfqpoint{2.736212in}{1.066228in}}%
\pgfpathlineto{\pgfqpoint{2.738536in}{1.057810in}}%
\pgfpathlineto{\pgfqpoint{2.740860in}{1.023847in}}%
\pgfpathlineto{\pgfqpoint{2.743184in}{1.084618in}}%
\pgfpathlineto{\pgfqpoint{2.745508in}{1.081132in}}%
\pgfpathlineto{\pgfqpoint{2.747832in}{1.079930in}}%
\pgfpathlineto{\pgfqpoint{2.750156in}{1.082234in}}%
\pgfpathlineto{\pgfqpoint{2.752480in}{1.038155in}}%
\pgfpathlineto{\pgfqpoint{2.754804in}{1.089347in}}%
\pgfpathlineto{\pgfqpoint{2.757128in}{1.097611in}}%
\pgfpathlineto{\pgfqpoint{2.759452in}{1.096500in}}%
\pgfpathlineto{\pgfqpoint{2.761776in}{1.079262in}}%
\pgfpathlineto{\pgfqpoint{2.764100in}{1.110155in}}%
\pgfpathlineto{\pgfqpoint{2.768748in}{1.100410in}}%
\pgfpathlineto{\pgfqpoint{2.771072in}{1.063824in}}%
\pgfpathlineto{\pgfqpoint{2.775719in}{1.093729in}}%
\pgfpathlineto{\pgfqpoint{2.778043in}{1.123552in}}%
\pgfpathlineto{\pgfqpoint{2.780367in}{1.068404in}}%
\pgfpathlineto{\pgfqpoint{2.782691in}{1.081711in}}%
\pgfpathlineto{\pgfqpoint{2.785015in}{1.126355in}}%
\pgfpathlineto{\pgfqpoint{2.787339in}{1.081562in}}%
\pgfpathlineto{\pgfqpoint{2.789663in}{1.085762in}}%
\pgfpathlineto{\pgfqpoint{2.791987in}{1.093038in}}%
\pgfpathlineto{\pgfqpoint{2.794311in}{1.090063in}}%
\pgfpathlineto{\pgfqpoint{2.796635in}{1.089971in}}%
\pgfpathlineto{\pgfqpoint{2.798959in}{1.121893in}}%
\pgfpathlineto{\pgfqpoint{2.801283in}{1.111614in}}%
\pgfpathlineto{\pgfqpoint{2.803607in}{1.125173in}}%
\pgfpathlineto{\pgfqpoint{2.805931in}{1.093603in}}%
\pgfpathlineto{\pgfqpoint{2.808255in}{1.135811in}}%
\pgfpathlineto{\pgfqpoint{2.810579in}{1.128858in}}%
\pgfpathlineto{\pgfqpoint{2.812903in}{1.150676in}}%
\pgfpathlineto{\pgfqpoint{2.815227in}{1.139569in}}%
\pgfpathlineto{\pgfqpoint{2.817551in}{1.113022in}}%
\pgfpathlineto{\pgfqpoint{2.822199in}{1.137417in}}%
\pgfpathlineto{\pgfqpoint{2.824523in}{1.126802in}}%
\pgfpathlineto{\pgfqpoint{2.826847in}{1.132254in}}%
\pgfpathlineto{\pgfqpoint{2.829171in}{1.119527in}}%
\pgfpathlineto{\pgfqpoint{2.831495in}{1.136542in}}%
\pgfpathlineto{\pgfqpoint{2.833819in}{1.143069in}}%
\pgfpathlineto{\pgfqpoint{2.836143in}{1.116944in}}%
\pgfpathlineto{\pgfqpoint{2.838467in}{1.134029in}}%
\pgfpathlineto{\pgfqpoint{2.840791in}{1.166330in}}%
\pgfpathlineto{\pgfqpoint{2.843115in}{1.119996in}}%
\pgfpathlineto{\pgfqpoint{2.845438in}{1.118770in}}%
\pgfpathlineto{\pgfqpoint{2.847762in}{1.115012in}}%
\pgfpathlineto{\pgfqpoint{2.850086in}{1.107468in}}%
\pgfpathlineto{\pgfqpoint{2.852410in}{1.123998in}}%
\pgfpathlineto{\pgfqpoint{2.854734in}{1.152510in}}%
\pgfpathlineto{\pgfqpoint{2.857058in}{1.094667in}}%
\pgfpathlineto{\pgfqpoint{2.861706in}{1.165199in}}%
\pgfpathlineto{\pgfqpoint{2.864030in}{1.182051in}}%
\pgfpathlineto{\pgfqpoint{2.866354in}{1.122738in}}%
\pgfpathlineto{\pgfqpoint{2.868678in}{1.149945in}}%
\pgfpathlineto{\pgfqpoint{2.871002in}{1.152172in}}%
\pgfpathlineto{\pgfqpoint{2.873326in}{1.183923in}}%
\pgfpathlineto{\pgfqpoint{2.877974in}{1.138468in}}%
\pgfpathlineto{\pgfqpoint{2.880298in}{1.200380in}}%
\pgfpathlineto{\pgfqpoint{2.882622in}{1.132171in}}%
\pgfpathlineto{\pgfqpoint{2.884946in}{1.108621in}}%
\pgfpathlineto{\pgfqpoint{2.887270in}{1.150859in}}%
\pgfpathlineto{\pgfqpoint{2.889594in}{1.156331in}}%
\pgfpathlineto{\pgfqpoint{2.891918in}{1.184542in}}%
\pgfpathlineto{\pgfqpoint{2.894242in}{1.165075in}}%
\pgfpathlineto{\pgfqpoint{2.898890in}{1.183318in}}%
\pgfpathlineto{\pgfqpoint{2.901214in}{1.174470in}}%
\pgfpathlineto{\pgfqpoint{2.903538in}{1.151211in}}%
\pgfpathlineto{\pgfqpoint{2.905862in}{1.184318in}}%
\pgfpathlineto{\pgfqpoint{2.908186in}{1.171657in}}%
\pgfpathlineto{\pgfqpoint{2.910510in}{1.169533in}}%
\pgfpathlineto{\pgfqpoint{2.912834in}{1.223482in}}%
\pgfpathlineto{\pgfqpoint{2.915157in}{1.197640in}}%
\pgfpathlineto{\pgfqpoint{2.917481in}{1.150382in}}%
\pgfpathlineto{\pgfqpoint{2.919805in}{1.194397in}}%
\pgfpathlineto{\pgfqpoint{2.922129in}{1.200974in}}%
\pgfpathlineto{\pgfqpoint{2.924453in}{1.196441in}}%
\pgfpathlineto{\pgfqpoint{2.926777in}{1.215637in}}%
\pgfpathlineto{\pgfqpoint{2.929101in}{1.175445in}}%
\pgfpathlineto{\pgfqpoint{2.931425in}{1.155977in}}%
\pgfpathlineto{\pgfqpoint{2.933749in}{1.201253in}}%
\pgfpathlineto{\pgfqpoint{2.936073in}{1.166674in}}%
\pgfpathlineto{\pgfqpoint{2.938397in}{1.204181in}}%
\pgfpathlineto{\pgfqpoint{2.940721in}{1.165559in}}%
\pgfpathlineto{\pgfqpoint{2.943045in}{1.209088in}}%
\pgfpathlineto{\pgfqpoint{2.945369in}{1.155465in}}%
\pgfpathlineto{\pgfqpoint{2.947693in}{1.234225in}}%
\pgfpathlineto{\pgfqpoint{2.954665in}{1.178218in}}%
\pgfpathlineto{\pgfqpoint{2.956989in}{1.198087in}}%
\pgfpathlineto{\pgfqpoint{2.959313in}{1.198547in}}%
\pgfpathlineto{\pgfqpoint{2.961637in}{1.224350in}}%
\pgfpathlineto{\pgfqpoint{2.963961in}{1.193452in}}%
\pgfpathlineto{\pgfqpoint{2.966285in}{1.225808in}}%
\pgfpathlineto{\pgfqpoint{2.968609in}{1.201378in}}%
\pgfpathlineto{\pgfqpoint{2.970933in}{1.209185in}}%
\pgfpathlineto{\pgfqpoint{2.973257in}{1.182387in}}%
\pgfpathlineto{\pgfqpoint{2.977905in}{1.254359in}}%
\pgfpathlineto{\pgfqpoint{2.980229in}{1.193626in}}%
\pgfpathlineto{\pgfqpoint{2.982553in}{1.222380in}}%
\pgfpathlineto{\pgfqpoint{2.984876in}{1.212518in}}%
\pgfpathlineto{\pgfqpoint{2.987200in}{1.226989in}}%
\pgfpathlineto{\pgfqpoint{2.989524in}{1.223127in}}%
\pgfpathlineto{\pgfqpoint{2.991848in}{1.245013in}}%
\pgfpathlineto{\pgfqpoint{2.994172in}{1.185174in}}%
\pgfpathlineto{\pgfqpoint{2.996496in}{1.240706in}}%
\pgfpathlineto{\pgfqpoint{2.998820in}{1.201486in}}%
\pgfpathlineto{\pgfqpoint{3.001144in}{1.235770in}}%
\pgfpathlineto{\pgfqpoint{3.003468in}{1.214371in}}%
\pgfpathlineto{\pgfqpoint{3.005792in}{1.220525in}}%
\pgfpathlineto{\pgfqpoint{3.008116in}{1.176211in}}%
\pgfpathlineto{\pgfqpoint{3.015088in}{1.259404in}}%
\pgfpathlineto{\pgfqpoint{3.017412in}{1.214487in}}%
\pgfpathlineto{\pgfqpoint{3.019736in}{1.197834in}}%
\pgfpathlineto{\pgfqpoint{3.022060in}{1.248119in}}%
\pgfpathlineto{\pgfqpoint{3.024384in}{1.221887in}}%
\pgfpathlineto{\pgfqpoint{3.026708in}{1.247656in}}%
\pgfpathlineto{\pgfqpoint{3.029032in}{1.243208in}}%
\pgfpathlineto{\pgfqpoint{3.031356in}{1.246793in}}%
\pgfpathlineto{\pgfqpoint{3.033680in}{1.200674in}}%
\pgfpathlineto{\pgfqpoint{3.036004in}{1.235771in}}%
\pgfpathlineto{\pgfqpoint{3.038328in}{1.225401in}}%
\pgfpathlineto{\pgfqpoint{3.040652in}{1.252896in}}%
\pgfpathlineto{\pgfqpoint{3.042976in}{1.246149in}}%
\pgfpathlineto{\pgfqpoint{3.045300in}{1.227377in}}%
\pgfpathlineto{\pgfqpoint{3.047624in}{1.256463in}}%
\pgfpathlineto{\pgfqpoint{3.049948in}{1.251893in}}%
\pgfpathlineto{\pgfqpoint{3.052272in}{1.211219in}}%
\pgfpathlineto{\pgfqpoint{3.054596in}{1.195731in}}%
\pgfpathlineto{\pgfqpoint{3.056919in}{1.257105in}}%
\pgfpathlineto{\pgfqpoint{3.059243in}{1.228805in}}%
\pgfpathlineto{\pgfqpoint{3.061567in}{1.256931in}}%
\pgfpathlineto{\pgfqpoint{3.063891in}{1.233766in}}%
\pgfpathlineto{\pgfqpoint{3.066215in}{1.223098in}}%
\pgfpathlineto{\pgfqpoint{3.068539in}{1.268816in}}%
\pgfpathlineto{\pgfqpoint{3.070863in}{1.242192in}}%
\pgfpathlineto{\pgfqpoint{3.073187in}{1.242690in}}%
\pgfpathlineto{\pgfqpoint{3.075511in}{1.248066in}}%
\pgfpathlineto{\pgfqpoint{3.077835in}{1.226342in}}%
\pgfpathlineto{\pgfqpoint{3.082483in}{1.280061in}}%
\pgfpathlineto{\pgfqpoint{3.084807in}{1.229203in}}%
\pgfpathlineto{\pgfqpoint{3.087131in}{1.265326in}}%
\pgfpathlineto{\pgfqpoint{3.089455in}{1.253987in}}%
\pgfpathlineto{\pgfqpoint{3.091779in}{1.235547in}}%
\pgfpathlineto{\pgfqpoint{3.094103in}{1.262143in}}%
\pgfpathlineto{\pgfqpoint{3.096427in}{1.256484in}}%
\pgfpathlineto{\pgfqpoint{3.098751in}{1.234804in}}%
\pgfpathlineto{\pgfqpoint{3.101075in}{1.271011in}}%
\pgfpathlineto{\pgfqpoint{3.103399in}{1.250039in}}%
\pgfpathlineto{\pgfqpoint{3.105723in}{1.239828in}}%
\pgfpathlineto{\pgfqpoint{3.108047in}{1.235237in}}%
\pgfpathlineto{\pgfqpoint{3.110371in}{1.272225in}}%
\pgfpathlineto{\pgfqpoint{3.112695in}{1.253958in}}%
\pgfpathlineto{\pgfqpoint{3.115019in}{1.281953in}}%
\pgfpathlineto{\pgfqpoint{3.117343in}{1.254899in}}%
\pgfpathlineto{\pgfqpoint{3.119667in}{1.300684in}}%
\pgfpathlineto{\pgfqpoint{3.121991in}{1.235300in}}%
\pgfpathlineto{\pgfqpoint{3.126638in}{1.257512in}}%
\pgfpathlineto{\pgfqpoint{3.131286in}{1.247847in}}%
\pgfpathlineto{\pgfqpoint{3.133610in}{1.289719in}}%
\pgfpathlineto{\pgfqpoint{3.135934in}{1.228061in}}%
\pgfpathlineto{\pgfqpoint{3.138258in}{1.253933in}}%
\pgfpathlineto{\pgfqpoint{3.140582in}{1.241173in}}%
\pgfpathlineto{\pgfqpoint{3.147554in}{1.277055in}}%
\pgfpathlineto{\pgfqpoint{3.149878in}{1.276231in}}%
\pgfpathlineto{\pgfqpoint{3.152202in}{1.281882in}}%
\pgfpathlineto{\pgfqpoint{3.154526in}{1.281511in}}%
\pgfpathlineto{\pgfqpoint{3.156850in}{1.236721in}}%
\pgfpathlineto{\pgfqpoint{3.163822in}{1.286880in}}%
\pgfpathlineto{\pgfqpoint{3.166146in}{1.263273in}}%
\pgfpathlineto{\pgfqpoint{3.168470in}{1.267769in}}%
\pgfpathlineto{\pgfqpoint{3.170794in}{1.236002in}}%
\pgfpathlineto{\pgfqpoint{3.173118in}{1.253923in}}%
\pgfpathlineto{\pgfqpoint{3.175442in}{1.242231in}}%
\pgfpathlineto{\pgfqpoint{3.177766in}{1.250329in}}%
\pgfpathlineto{\pgfqpoint{3.180090in}{1.247046in}}%
\pgfpathlineto{\pgfqpoint{3.182414in}{1.279229in}}%
\pgfpathlineto{\pgfqpoint{3.184738in}{1.290985in}}%
\pgfpathlineto{\pgfqpoint{3.187062in}{1.260288in}}%
\pgfpathlineto{\pgfqpoint{3.191710in}{1.245486in}}%
\pgfpathlineto{\pgfqpoint{3.194034in}{1.260947in}}%
\pgfpathlineto{\pgfqpoint{3.196357in}{1.243266in}}%
\pgfpathlineto{\pgfqpoint{3.198681in}{1.289395in}}%
\pgfpathlineto{\pgfqpoint{3.201005in}{1.259433in}}%
\pgfpathlineto{\pgfqpoint{3.203329in}{1.288927in}}%
\pgfpathlineto{\pgfqpoint{3.207977in}{1.241946in}}%
\pgfpathlineto{\pgfqpoint{3.210301in}{1.263079in}}%
\pgfpathlineto{\pgfqpoint{3.212625in}{1.273666in}}%
\pgfpathlineto{\pgfqpoint{3.214949in}{1.237496in}}%
\pgfpathlineto{\pgfqpoint{3.219597in}{1.267716in}}%
\pgfpathlineto{\pgfqpoint{3.221921in}{1.278869in}}%
\pgfpathlineto{\pgfqpoint{3.224245in}{1.231107in}}%
\pgfpathlineto{\pgfqpoint{3.226569in}{1.273242in}}%
\pgfpathlineto{\pgfqpoint{3.228893in}{1.251744in}}%
\pgfpathlineto{\pgfqpoint{3.231217in}{1.269638in}}%
\pgfpathlineto{\pgfqpoint{3.233541in}{1.255671in}}%
\pgfpathlineto{\pgfqpoint{3.235865in}{1.268991in}}%
\pgfpathlineto{\pgfqpoint{3.238189in}{1.265891in}}%
\pgfpathlineto{\pgfqpoint{3.240513in}{1.257206in}}%
\pgfpathlineto{\pgfqpoint{3.242837in}{1.238691in}}%
\pgfpathlineto{\pgfqpoint{3.245161in}{1.248386in}}%
\pgfpathlineto{\pgfqpoint{3.247485in}{1.266701in}}%
\pgfpathlineto{\pgfqpoint{3.249809in}{1.245021in}}%
\pgfpathlineto{\pgfqpoint{3.252133in}{1.212595in}}%
\pgfpathlineto{\pgfqpoint{3.254457in}{1.245235in}}%
\pgfpathlineto{\pgfqpoint{3.256781in}{1.245925in}}%
\pgfpathlineto{\pgfqpoint{3.259105in}{1.225281in}}%
\pgfpathlineto{\pgfqpoint{3.261429in}{1.255729in}}%
\pgfpathlineto{\pgfqpoint{3.263753in}{1.253660in}}%
\pgfpathlineto{\pgfqpoint{3.266076in}{1.342145in}}%
\pgfpathlineto{\pgfqpoint{3.268400in}{1.317662in}}%
\pgfpathlineto{\pgfqpoint{3.270724in}{1.309850in}}%
\pgfpathlineto{\pgfqpoint{3.273048in}{1.313837in}}%
\pgfpathlineto{\pgfqpoint{3.275372in}{1.311690in}}%
\pgfpathlineto{\pgfqpoint{3.277696in}{1.311248in}}%
\pgfpathlineto{\pgfqpoint{3.280020in}{1.325199in}}%
\pgfpathlineto{\pgfqpoint{3.282344in}{1.303339in}}%
\pgfpathlineto{\pgfqpoint{3.284668in}{1.328856in}}%
\pgfpathlineto{\pgfqpoint{3.286992in}{1.340018in}}%
\pgfpathlineto{\pgfqpoint{3.289316in}{1.323944in}}%
\pgfpathlineto{\pgfqpoint{3.291640in}{1.284817in}}%
\pgfpathlineto{\pgfqpoint{3.293964in}{1.288579in}}%
\pgfpathlineto{\pgfqpoint{3.296288in}{1.337357in}}%
\pgfpathlineto{\pgfqpoint{3.298612in}{1.311420in}}%
\pgfpathlineto{\pgfqpoint{3.300936in}{1.324206in}}%
\pgfpathlineto{\pgfqpoint{3.303260in}{1.299073in}}%
\pgfpathlineto{\pgfqpoint{3.305584in}{1.325265in}}%
\pgfpathlineto{\pgfqpoint{3.307908in}{1.283723in}}%
\pgfpathlineto{\pgfqpoint{3.310232in}{1.316968in}}%
\pgfpathlineto{\pgfqpoint{3.314880in}{1.273855in}}%
\pgfpathlineto{\pgfqpoint{3.317204in}{1.288687in}}%
\pgfpathlineto{\pgfqpoint{3.321852in}{1.255110in}}%
\pgfpathlineto{\pgfqpoint{3.324176in}{1.300280in}}%
\pgfpathlineto{\pgfqpoint{3.326500in}{1.314507in}}%
\pgfpathlineto{\pgfqpoint{3.328824in}{1.273425in}}%
\pgfpathlineto{\pgfqpoint{3.333472in}{1.331504in}}%
\pgfpathlineto{\pgfqpoint{3.335796in}{1.287744in}}%
\pgfpathlineto{\pgfqpoint{3.338119in}{1.276283in}}%
\pgfpathlineto{\pgfqpoint{3.340443in}{1.320089in}}%
\pgfpathlineto{\pgfqpoint{3.342767in}{1.262692in}}%
\pgfpathlineto{\pgfqpoint{3.345091in}{1.287527in}}%
\pgfpathlineto{\pgfqpoint{3.347415in}{1.325477in}}%
\pgfpathlineto{\pgfqpoint{3.349739in}{1.256951in}}%
\pgfpathlineto{\pgfqpoint{3.352063in}{1.295186in}}%
\pgfpathlineto{\pgfqpoint{3.354387in}{1.286366in}}%
\pgfpathlineto{\pgfqpoint{3.356711in}{1.264673in}}%
\pgfpathlineto{\pgfqpoint{3.361359in}{1.282006in}}%
\pgfpathlineto{\pgfqpoint{3.363683in}{1.258674in}}%
\pgfpathlineto{\pgfqpoint{3.366007in}{1.264191in}}%
\pgfpathlineto{\pgfqpoint{3.368331in}{1.245694in}}%
\pgfpathlineto{\pgfqpoint{3.370655in}{1.246759in}}%
\pgfpathlineto{\pgfqpoint{3.372979in}{1.299569in}}%
\pgfpathlineto{\pgfqpoint{3.377627in}{1.264066in}}%
\pgfpathlineto{\pgfqpoint{3.379951in}{1.256451in}}%
\pgfpathlineto{\pgfqpoint{3.382275in}{1.256407in}}%
\pgfpathlineto{\pgfqpoint{3.384599in}{1.273137in}}%
\pgfpathlineto{\pgfqpoint{3.386923in}{1.254719in}}%
\pgfpathlineto{\pgfqpoint{3.389247in}{1.260446in}}%
\pgfpathlineto{\pgfqpoint{3.391571in}{1.224926in}}%
\pgfpathlineto{\pgfqpoint{3.393895in}{1.266740in}}%
\pgfpathlineto{\pgfqpoint{3.396219in}{1.221667in}}%
\pgfpathlineto{\pgfqpoint{3.398543in}{1.243468in}}%
\pgfpathlineto{\pgfqpoint{3.400867in}{1.222951in}}%
\pgfpathlineto{\pgfqpoint{3.403191in}{1.271155in}}%
\pgfpathlineto{\pgfqpoint{3.407838in}{1.221752in}}%
\pgfpathlineto{\pgfqpoint{3.410162in}{1.224862in}}%
\pgfpathlineto{\pgfqpoint{3.412486in}{1.295915in}}%
\pgfpathlineto{\pgfqpoint{3.419458in}{1.253118in}}%
\pgfpathlineto{\pgfqpoint{3.421782in}{1.258725in}}%
\pgfpathlineto{\pgfqpoint{3.424106in}{1.225464in}}%
\pgfpathlineto{\pgfqpoint{3.426430in}{1.270615in}}%
\pgfpathlineto{\pgfqpoint{3.431078in}{1.213129in}}%
\pgfpathlineto{\pgfqpoint{3.433402in}{1.243860in}}%
\pgfpathlineto{\pgfqpoint{3.435726in}{1.240109in}}%
\pgfpathlineto{\pgfqpoint{3.438050in}{1.240542in}}%
\pgfpathlineto{\pgfqpoint{3.440374in}{1.242905in}}%
\pgfpathlineto{\pgfqpoint{3.445022in}{1.220993in}}%
\pgfpathlineto{\pgfqpoint{3.447346in}{1.235299in}}%
\pgfpathlineto{\pgfqpoint{3.449670in}{1.218036in}}%
\pgfpathlineto{\pgfqpoint{3.451994in}{1.259024in}}%
\pgfpathlineto{\pgfqpoint{3.454318in}{1.214844in}}%
\pgfpathlineto{\pgfqpoint{3.456642in}{1.242466in}}%
\pgfpathlineto{\pgfqpoint{3.458966in}{1.253563in}}%
\pgfpathlineto{\pgfqpoint{3.461290in}{1.216295in}}%
\pgfpathlineto{\pgfqpoint{3.463614in}{1.220369in}}%
\pgfpathlineto{\pgfqpoint{3.465938in}{1.241545in}}%
\pgfpathlineto{\pgfqpoint{3.468262in}{1.224926in}}%
\pgfpathlineto{\pgfqpoint{3.470586in}{1.220659in}}%
\pgfpathlineto{\pgfqpoint{3.472910in}{1.170528in}}%
\pgfpathlineto{\pgfqpoint{3.477557in}{1.204846in}}%
\pgfpathlineto{\pgfqpoint{3.479881in}{1.183362in}}%
\pgfpathlineto{\pgfqpoint{3.482205in}{1.229639in}}%
\pgfpathlineto{\pgfqpoint{3.484529in}{1.192043in}}%
\pgfpathlineto{\pgfqpoint{3.486853in}{1.241895in}}%
\pgfpathlineto{\pgfqpoint{3.489177in}{1.165052in}}%
\pgfpathlineto{\pgfqpoint{3.491501in}{1.236147in}}%
\pgfpathlineto{\pgfqpoint{3.493825in}{1.208851in}}%
\pgfpathlineto{\pgfqpoint{3.496149in}{1.238144in}}%
\pgfpathlineto{\pgfqpoint{3.498473in}{1.158148in}}%
\pgfpathlineto{\pgfqpoint{3.500797in}{1.210495in}}%
\pgfpathlineto{\pgfqpoint{3.503121in}{1.207949in}}%
\pgfpathlineto{\pgfqpoint{3.505445in}{1.200486in}}%
\pgfpathlineto{\pgfqpoint{3.507769in}{1.152248in}}%
\pgfpathlineto{\pgfqpoint{3.510093in}{1.148779in}}%
\pgfpathlineto{\pgfqpoint{3.512417in}{1.194663in}}%
\pgfpathlineto{\pgfqpoint{3.514741in}{1.151738in}}%
\pgfpathlineto{\pgfqpoint{3.517065in}{1.162395in}}%
\pgfpathlineto{\pgfqpoint{3.519389in}{1.146749in}}%
\pgfpathlineto{\pgfqpoint{3.521713in}{1.174265in}}%
\pgfpathlineto{\pgfqpoint{3.524037in}{1.176643in}}%
\pgfpathlineto{\pgfqpoint{3.526361in}{1.164970in}}%
\pgfpathlineto{\pgfqpoint{3.528685in}{1.190428in}}%
\pgfpathlineto{\pgfqpoint{3.531009in}{1.173125in}}%
\pgfpathlineto{\pgfqpoint{3.533333in}{1.197995in}}%
\pgfpathlineto{\pgfqpoint{3.535657in}{1.161201in}}%
\pgfpathlineto{\pgfqpoint{3.540305in}{1.189064in}}%
\pgfpathlineto{\pgfqpoint{3.542629in}{1.155277in}}%
\pgfpathlineto{\pgfqpoint{3.544953in}{1.144799in}}%
\pgfpathlineto{\pgfqpoint{3.547277in}{1.164797in}}%
\pgfpathlineto{\pgfqpoint{3.549600in}{1.122945in}}%
\pgfpathlineto{\pgfqpoint{3.551924in}{1.142481in}}%
\pgfpathlineto{\pgfqpoint{3.554248in}{1.147878in}}%
\pgfpathlineto{\pgfqpoint{3.556572in}{1.140902in}}%
\pgfpathlineto{\pgfqpoint{3.558896in}{1.144859in}}%
\pgfpathlineto{\pgfqpoint{3.561220in}{1.183540in}}%
\pgfpathlineto{\pgfqpoint{3.563544in}{1.126688in}}%
\pgfpathlineto{\pgfqpoint{3.565868in}{1.157600in}}%
\pgfpathlineto{\pgfqpoint{3.568192in}{1.150854in}}%
\pgfpathlineto{\pgfqpoint{3.570516in}{1.172025in}}%
\pgfpathlineto{\pgfqpoint{3.572840in}{1.128644in}}%
\pgfpathlineto{\pgfqpoint{3.575164in}{1.149408in}}%
\pgfpathlineto{\pgfqpoint{3.577488in}{1.124724in}}%
\pgfpathlineto{\pgfqpoint{3.579812in}{1.153306in}}%
\pgfpathlineto{\pgfqpoint{3.582136in}{1.158464in}}%
\pgfpathlineto{\pgfqpoint{3.584460in}{1.135330in}}%
\pgfpathlineto{\pgfqpoint{3.586784in}{1.166793in}}%
\pgfpathlineto{\pgfqpoint{3.591432in}{1.084158in}}%
\pgfpathlineto{\pgfqpoint{3.598404in}{1.164700in}}%
\pgfpathlineto{\pgfqpoint{3.600728in}{1.078473in}}%
\pgfpathlineto{\pgfqpoint{3.603052in}{1.121393in}}%
\pgfpathlineto{\pgfqpoint{3.605376in}{1.105655in}}%
\pgfpathlineto{\pgfqpoint{3.607700in}{1.143737in}}%
\pgfpathlineto{\pgfqpoint{3.610024in}{1.120403in}}%
\pgfpathlineto{\pgfqpoint{3.612348in}{1.183841in}}%
\pgfpathlineto{\pgfqpoint{3.614672in}{1.103536in}}%
\pgfpathlineto{\pgfqpoint{3.616996in}{1.084559in}}%
\pgfpathlineto{\pgfqpoint{3.621643in}{1.146898in}}%
\pgfpathlineto{\pgfqpoint{3.623967in}{1.035576in}}%
\pgfpathlineto{\pgfqpoint{3.626291in}{1.095693in}}%
\pgfpathlineto{\pgfqpoint{3.628615in}{1.090292in}}%
\pgfpathlineto{\pgfqpoint{3.630939in}{1.117844in}}%
\pgfpathlineto{\pgfqpoint{3.633263in}{1.090437in}}%
\pgfpathlineto{\pgfqpoint{3.637911in}{1.100245in}}%
\pgfpathlineto{\pgfqpoint{3.640235in}{1.086886in}}%
\pgfpathlineto{\pgfqpoint{3.642559in}{1.055493in}}%
\pgfpathlineto{\pgfqpoint{3.647207in}{1.085988in}}%
\pgfpathlineto{\pgfqpoint{3.649531in}{1.087658in}}%
\pgfpathlineto{\pgfqpoint{3.651855in}{1.079863in}}%
\pgfpathlineto{\pgfqpoint{3.654179in}{1.107621in}}%
\pgfpathlineto{\pgfqpoint{3.656503in}{1.053439in}}%
\pgfpathlineto{\pgfqpoint{3.658827in}{1.060126in}}%
\pgfpathlineto{\pgfqpoint{3.665799in}{1.090592in}}%
\pgfpathlineto{\pgfqpoint{3.668123in}{1.029535in}}%
\pgfpathlineto{\pgfqpoint{3.670447in}{1.047540in}}%
\pgfpathlineto{\pgfqpoint{3.672771in}{1.074913in}}%
\pgfpathlineto{\pgfqpoint{3.675095in}{1.084081in}}%
\pgfpathlineto{\pgfqpoint{3.677419in}{1.073574in}}%
\pgfpathlineto{\pgfqpoint{3.679743in}{1.098221in}}%
\pgfpathlineto{\pgfqpoint{3.682067in}{1.016669in}}%
\pgfpathlineto{\pgfqpoint{3.684391in}{1.084992in}}%
\pgfpathlineto{\pgfqpoint{3.686715in}{1.073636in}}%
\pgfpathlineto{\pgfqpoint{3.691362in}{1.029562in}}%
\pgfpathlineto{\pgfqpoint{3.693686in}{1.047303in}}%
\pgfpathlineto{\pgfqpoint{3.696010in}{1.041101in}}%
\pgfpathlineto{\pgfqpoint{3.698334in}{1.024658in}}%
\pgfpathlineto{\pgfqpoint{3.700658in}{1.067170in}}%
\pgfpathlineto{\pgfqpoint{3.702982in}{1.016063in}}%
\pgfpathlineto{\pgfqpoint{3.705306in}{1.003605in}}%
\pgfpathlineto{\pgfqpoint{3.709954in}{1.030294in}}%
\pgfpathlineto{\pgfqpoint{3.714602in}{1.037575in}}%
\pgfpathlineto{\pgfqpoint{3.716926in}{0.987703in}}%
\pgfpathlineto{\pgfqpoint{3.719250in}{1.039535in}}%
\pgfpathlineto{\pgfqpoint{3.721574in}{1.002494in}}%
\pgfpathlineto{\pgfqpoint{3.723898in}{1.001388in}}%
\pgfpathlineto{\pgfqpoint{3.726222in}{1.044610in}}%
\pgfpathlineto{\pgfqpoint{3.728546in}{0.970357in}}%
\pgfpathlineto{\pgfqpoint{3.730870in}{1.021657in}}%
\pgfpathlineto{\pgfqpoint{3.733194in}{0.995125in}}%
\pgfpathlineto{\pgfqpoint{3.735518in}{1.039993in}}%
\pgfpathlineto{\pgfqpoint{3.737842in}{0.990990in}}%
\pgfpathlineto{\pgfqpoint{3.740166in}{1.027544in}}%
\pgfpathlineto{\pgfqpoint{3.742490in}{1.042497in}}%
\pgfpathlineto{\pgfqpoint{3.744814in}{0.993147in}}%
\pgfpathlineto{\pgfqpoint{3.747138in}{1.030136in}}%
\pgfpathlineto{\pgfqpoint{3.749462in}{1.009331in}}%
\pgfpathlineto{\pgfqpoint{3.751786in}{1.016867in}}%
\pgfpathlineto{\pgfqpoint{3.754110in}{0.956923in}}%
\pgfpathlineto{\pgfqpoint{3.756434in}{0.963828in}}%
\pgfpathlineto{\pgfqpoint{3.758757in}{1.009115in}}%
\pgfpathlineto{\pgfqpoint{3.761081in}{1.021961in}}%
\pgfpathlineto{\pgfqpoint{3.763405in}{0.982177in}}%
\pgfpathlineto{\pgfqpoint{3.768053in}{0.969844in}}%
\pgfpathlineto{\pgfqpoint{3.770377in}{0.957995in}}%
\pgfpathlineto{\pgfqpoint{3.772701in}{0.969854in}}%
\pgfpathlineto{\pgfqpoint{3.775025in}{0.968465in}}%
\pgfpathlineto{\pgfqpoint{3.779673in}{0.924635in}}%
\pgfpathlineto{\pgfqpoint{3.784321in}{0.984706in}}%
\pgfpathlineto{\pgfqpoint{3.786645in}{0.949087in}}%
\pgfpathlineto{\pgfqpoint{3.788969in}{0.979994in}}%
\pgfpathlineto{\pgfqpoint{3.791293in}{0.941232in}}%
\pgfpathlineto{\pgfqpoint{3.793617in}{0.990516in}}%
\pgfpathlineto{\pgfqpoint{3.795941in}{0.962514in}}%
\pgfpathlineto{\pgfqpoint{3.798265in}{0.948145in}}%
\pgfpathlineto{\pgfqpoint{3.800589in}{1.005098in}}%
\pgfpathlineto{\pgfqpoint{3.802913in}{0.968271in}}%
\pgfpathlineto{\pgfqpoint{3.805237in}{0.968074in}}%
\pgfpathlineto{\pgfqpoint{3.807561in}{0.953809in}}%
\pgfpathlineto{\pgfqpoint{3.809885in}{0.961311in}}%
\pgfpathlineto{\pgfqpoint{3.812209in}{0.951202in}}%
\pgfpathlineto{\pgfqpoint{3.814533in}{0.910424in}}%
\pgfpathlineto{\pgfqpoint{3.819181in}{0.992736in}}%
\pgfpathlineto{\pgfqpoint{3.821505in}{0.967486in}}%
\pgfpathlineto{\pgfqpoint{3.823829in}{0.916786in}}%
\pgfpathlineto{\pgfqpoint{3.826153in}{0.923275in}}%
\pgfpathlineto{\pgfqpoint{3.830800in}{0.909213in}}%
\pgfpathlineto{\pgfqpoint{3.833124in}{0.956797in}}%
\pgfpathlineto{\pgfqpoint{3.835448in}{0.880251in}}%
\pgfpathlineto{\pgfqpoint{3.837772in}{0.946991in}}%
\pgfpathlineto{\pgfqpoint{3.840096in}{0.938154in}}%
\pgfpathlineto{\pgfqpoint{3.842420in}{0.889792in}}%
\pgfpathlineto{\pgfqpoint{3.844744in}{0.886774in}}%
\pgfpathlineto{\pgfqpoint{3.847068in}{0.962337in}}%
\pgfpathlineto{\pgfqpoint{3.849392in}{0.989340in}}%
\pgfpathlineto{\pgfqpoint{3.854040in}{0.962220in}}%
\pgfpathlineto{\pgfqpoint{3.856364in}{0.959390in}}%
\pgfpathlineto{\pgfqpoint{3.858688in}{0.940606in}}%
\pgfpathlineto{\pgfqpoint{3.861012in}{0.978901in}}%
\pgfpathlineto{\pgfqpoint{3.863336in}{0.949505in}}%
\pgfpathlineto{\pgfqpoint{3.865660in}{0.977357in}}%
\pgfpathlineto{\pgfqpoint{3.867984in}{0.975995in}}%
\pgfpathlineto{\pgfqpoint{3.872632in}{0.954778in}}%
\pgfpathlineto{\pgfqpoint{3.874956in}{0.964348in}}%
\pgfpathlineto{\pgfqpoint{3.877280in}{0.989864in}}%
\pgfpathlineto{\pgfqpoint{3.881928in}{0.940162in}}%
\pgfpathlineto{\pgfqpoint{3.884252in}{0.915019in}}%
\pgfpathlineto{\pgfqpoint{3.886576in}{0.947274in}}%
\pgfpathlineto{\pgfqpoint{3.888900in}{0.925877in}}%
\pgfpathlineto{\pgfqpoint{3.891224in}{0.929541in}}%
\pgfpathlineto{\pgfqpoint{3.893548in}{0.969360in}}%
\pgfpathlineto{\pgfqpoint{3.898196in}{0.914246in}}%
\pgfpathlineto{\pgfqpoint{3.900519in}{0.914632in}}%
\pgfpathlineto{\pgfqpoint{3.902843in}{0.969265in}}%
\pgfpathlineto{\pgfqpoint{3.905167in}{0.898911in}}%
\pgfpathlineto{\pgfqpoint{3.907491in}{0.948673in}}%
\pgfpathlineto{\pgfqpoint{3.909815in}{0.900742in}}%
\pgfpathlineto{\pgfqpoint{3.914463in}{0.935382in}}%
\pgfpathlineto{\pgfqpoint{3.916787in}{0.953099in}}%
\pgfpathlineto{\pgfqpoint{3.921435in}{0.917476in}}%
\pgfpathlineto{\pgfqpoint{3.923759in}{0.937705in}}%
\pgfpathlineto{\pgfqpoint{3.926083in}{0.946194in}}%
\pgfpathlineto{\pgfqpoint{3.928407in}{0.906831in}}%
\pgfpathlineto{\pgfqpoint{3.930731in}{0.891897in}}%
\pgfpathlineto{\pgfqpoint{3.933055in}{0.932793in}}%
\pgfpathlineto{\pgfqpoint{3.935379in}{0.891171in}}%
\pgfpathlineto{\pgfqpoint{3.937703in}{0.877360in}}%
\pgfpathlineto{\pgfqpoint{3.940027in}{0.925703in}}%
\pgfpathlineto{\pgfqpoint{3.944675in}{0.876617in}}%
\pgfpathlineto{\pgfqpoint{3.949323in}{0.923156in}}%
\pgfpathlineto{\pgfqpoint{3.951647in}{0.896172in}}%
\pgfpathlineto{\pgfqpoint{3.953971in}{0.912957in}}%
\pgfpathlineto{\pgfqpoint{3.956295in}{0.887864in}}%
\pgfpathlineto{\pgfqpoint{3.958619in}{0.901030in}}%
\pgfpathlineto{\pgfqpoint{3.960943in}{0.889981in}}%
\pgfpathlineto{\pgfqpoint{3.963267in}{0.885923in}}%
\pgfpathlineto{\pgfqpoint{3.965591in}{0.864945in}}%
\pgfpathlineto{\pgfqpoint{3.967915in}{0.908483in}}%
\pgfpathlineto{\pgfqpoint{3.970238in}{0.911913in}}%
\pgfpathlineto{\pgfqpoint{3.972562in}{0.853575in}}%
\pgfpathlineto{\pgfqpoint{3.974886in}{0.856849in}}%
\pgfpathlineto{\pgfqpoint{3.977210in}{0.871223in}}%
\pgfpathlineto{\pgfqpoint{3.979534in}{0.916316in}}%
\pgfpathlineto{\pgfqpoint{3.981858in}{0.910769in}}%
\pgfpathlineto{\pgfqpoint{3.984182in}{0.857296in}}%
\pgfpathlineto{\pgfqpoint{3.988830in}{0.900553in}}%
\pgfpathlineto{\pgfqpoint{3.991154in}{0.888789in}}%
\pgfpathlineto{\pgfqpoint{3.993478in}{0.867713in}}%
\pgfpathlineto{\pgfqpoint{3.995802in}{0.899230in}}%
\pgfpathlineto{\pgfqpoint{3.998126in}{0.870318in}}%
\pgfpathlineto{\pgfqpoint{4.000450in}{0.857313in}}%
\pgfpathlineto{\pgfqpoint{4.002774in}{0.865624in}}%
\pgfpathlineto{\pgfqpoint{4.005098in}{0.851241in}}%
\pgfpathlineto{\pgfqpoint{4.007422in}{0.863873in}}%
\pgfpathlineto{\pgfqpoint{4.009746in}{0.859315in}}%
\pgfpathlineto{\pgfqpoint{4.012070in}{0.891982in}}%
\pgfpathlineto{\pgfqpoint{4.014394in}{0.899616in}}%
\pgfpathlineto{\pgfqpoint{4.016718in}{0.902927in}}%
\pgfpathlineto{\pgfqpoint{4.021366in}{0.842301in}}%
\pgfpathlineto{\pgfqpoint{4.023690in}{0.891547in}}%
\pgfpathlineto{\pgfqpoint{4.026014in}{0.909129in}}%
\pgfpathlineto{\pgfqpoint{4.028338in}{0.909802in}}%
\pgfpathlineto{\pgfqpoint{4.030662in}{0.846859in}}%
\pgfpathlineto{\pgfqpoint{4.032986in}{0.834682in}}%
\pgfpathlineto{\pgfqpoint{4.035310in}{0.849697in}}%
\pgfpathlineto{\pgfqpoint{4.037634in}{0.905509in}}%
\pgfpathlineto{\pgfqpoint{4.039957in}{0.846072in}}%
\pgfpathlineto{\pgfqpoint{4.044605in}{0.879682in}}%
\pgfpathlineto{\pgfqpoint{4.046929in}{0.889476in}}%
\pgfpathlineto{\pgfqpoint{4.049253in}{0.848271in}}%
\pgfpathlineto{\pgfqpoint{4.051577in}{0.835371in}}%
\pgfpathlineto{\pgfqpoint{4.053901in}{0.838237in}}%
\pgfpathlineto{\pgfqpoint{4.056225in}{0.821254in}}%
\pgfpathlineto{\pgfqpoint{4.058549in}{0.862275in}}%
\pgfpathlineto{\pgfqpoint{4.060873in}{0.872988in}}%
\pgfpathlineto{\pgfqpoint{4.063197in}{0.838336in}}%
\pgfpathlineto{\pgfqpoint{4.065521in}{0.879680in}}%
\pgfpathlineto{\pgfqpoint{4.067845in}{0.856689in}}%
\pgfpathlineto{\pgfqpoint{4.070169in}{0.874231in}}%
\pgfpathlineto{\pgfqpoint{4.074817in}{0.817618in}}%
\pgfpathlineto{\pgfqpoint{4.077141in}{0.879297in}}%
\pgfpathlineto{\pgfqpoint{4.079465in}{0.817691in}}%
\pgfpathlineto{\pgfqpoint{4.084113in}{0.872316in}}%
\pgfpathlineto{\pgfqpoint{4.086437in}{0.854142in}}%
\pgfpathlineto{\pgfqpoint{4.088761in}{0.883810in}}%
\pgfpathlineto{\pgfqpoint{4.095733in}{0.838021in}}%
\pgfpathlineto{\pgfqpoint{4.098057in}{0.813997in}}%
\pgfpathlineto{\pgfqpoint{4.100381in}{0.834492in}}%
\pgfpathlineto{\pgfqpoint{4.102705in}{0.833866in}}%
\pgfpathlineto{\pgfqpoint{4.105029in}{0.797493in}}%
\pgfpathlineto{\pgfqpoint{4.107353in}{0.820090in}}%
\pgfpathlineto{\pgfqpoint{4.109677in}{0.820726in}}%
\pgfpathlineto{\pgfqpoint{4.114324in}{0.897187in}}%
\pgfpathlineto{\pgfqpoint{4.116648in}{0.847342in}}%
\pgfpathlineto{\pgfqpoint{4.118972in}{0.849396in}}%
\pgfpathlineto{\pgfqpoint{4.121296in}{0.831180in}}%
\pgfpathlineto{\pgfqpoint{4.123620in}{0.890130in}}%
\pgfpathlineto{\pgfqpoint{4.125944in}{0.845383in}}%
\pgfpathlineto{\pgfqpoint{4.128268in}{0.890311in}}%
\pgfpathlineto{\pgfqpoint{4.130592in}{0.816523in}}%
\pgfpathlineto{\pgfqpoint{4.132916in}{0.852379in}}%
\pgfpathlineto{\pgfqpoint{4.135240in}{0.869947in}}%
\pgfpathlineto{\pgfqpoint{4.137564in}{0.862852in}}%
\pgfpathlineto{\pgfqpoint{4.139888in}{0.836123in}}%
\pgfpathlineto{\pgfqpoint{4.142212in}{0.866952in}}%
\pgfpathlineto{\pgfqpoint{4.144536in}{0.868009in}}%
\pgfpathlineto{\pgfqpoint{4.146860in}{0.850388in}}%
\pgfpathlineto{\pgfqpoint{4.149184in}{0.862314in}}%
\pgfpathlineto{\pgfqpoint{4.151508in}{0.843370in}}%
\pgfpathlineto{\pgfqpoint{4.153832in}{0.867515in}}%
\pgfpathlineto{\pgfqpoint{4.156156in}{0.851014in}}%
\pgfpathlineto{\pgfqpoint{4.158480in}{0.811036in}}%
\pgfpathlineto{\pgfqpoint{4.160804in}{0.816116in}}%
\pgfpathlineto{\pgfqpoint{4.165452in}{0.863921in}}%
\pgfpathlineto{\pgfqpoint{4.167776in}{0.832061in}}%
\pgfpathlineto{\pgfqpoint{4.170100in}{0.880890in}}%
\pgfpathlineto{\pgfqpoint{4.172424in}{0.876826in}}%
\pgfpathlineto{\pgfqpoint{4.174748in}{0.823398in}}%
\pgfpathlineto{\pgfqpoint{4.177072in}{0.863491in}}%
\pgfpathlineto{\pgfqpoint{4.179396in}{0.813074in}}%
\pgfpathlineto{\pgfqpoint{4.181719in}{0.864589in}}%
\pgfpathlineto{\pgfqpoint{4.184043in}{0.847522in}}%
\pgfpathlineto{\pgfqpoint{4.186367in}{0.838786in}}%
\pgfpathlineto{\pgfqpoint{4.188691in}{0.857335in}}%
\pgfpathlineto{\pgfqpoint{4.191015in}{0.829007in}}%
\pgfpathlineto{\pgfqpoint{4.193339in}{0.856469in}}%
\pgfpathlineto{\pgfqpoint{4.195663in}{0.846815in}}%
\pgfpathlineto{\pgfqpoint{4.197987in}{0.815105in}}%
\pgfpathlineto{\pgfqpoint{4.200311in}{0.850412in}}%
\pgfpathlineto{\pgfqpoint{4.202635in}{0.794478in}}%
\pgfpathlineto{\pgfqpoint{4.204959in}{0.837494in}}%
\pgfpathlineto{\pgfqpoint{4.207283in}{0.807383in}}%
\pgfpathlineto{\pgfqpoint{4.209607in}{0.879807in}}%
\pgfpathlineto{\pgfqpoint{4.211931in}{0.835820in}}%
\pgfpathlineto{\pgfqpoint{4.214255in}{0.853745in}}%
\pgfpathlineto{\pgfqpoint{4.216579in}{0.844079in}}%
\pgfpathlineto{\pgfqpoint{4.218903in}{0.867684in}}%
\pgfpathlineto{\pgfqpoint{4.221227in}{0.850214in}}%
\pgfpathlineto{\pgfqpoint{4.223551in}{0.842471in}}%
\pgfpathlineto{\pgfqpoint{4.225875in}{0.850246in}}%
\pgfpathlineto{\pgfqpoint{4.228199in}{0.819702in}}%
\pgfpathlineto{\pgfqpoint{4.230523in}{0.884587in}}%
\pgfpathlineto{\pgfqpoint{4.232847in}{0.808503in}}%
\pgfpathlineto{\pgfqpoint{4.235171in}{0.857580in}}%
\pgfpathlineto{\pgfqpoint{4.237495in}{0.862978in}}%
\pgfpathlineto{\pgfqpoint{4.239819in}{0.852675in}}%
\pgfpathlineto{\pgfqpoint{4.242143in}{0.833222in}}%
\pgfpathlineto{\pgfqpoint{4.244467in}{0.855855in}}%
\pgfpathlineto{\pgfqpoint{4.246791in}{0.838821in}}%
\pgfpathlineto{\pgfqpoint{4.249115in}{0.831364in}}%
\pgfpathlineto{\pgfqpoint{4.251438in}{0.869307in}}%
\pgfpathlineto{\pgfqpoint{4.253762in}{0.846915in}}%
\pgfpathlineto{\pgfqpoint{4.256086in}{0.862094in}}%
\pgfpathlineto{\pgfqpoint{4.258410in}{0.888895in}}%
\pgfpathlineto{\pgfqpoint{4.260734in}{0.850028in}}%
\pgfpathlineto{\pgfqpoint{4.263058in}{0.844158in}}%
\pgfpathlineto{\pgfqpoint{4.265382in}{0.850311in}}%
\pgfpathlineto{\pgfqpoint{4.267706in}{0.849449in}}%
\pgfpathlineto{\pgfqpoint{4.270030in}{0.881498in}}%
\pgfpathlineto{\pgfqpoint{4.272354in}{0.811856in}}%
\pgfpathlineto{\pgfqpoint{4.274678in}{0.863327in}}%
\pgfpathlineto{\pgfqpoint{4.277002in}{0.858626in}}%
\pgfpathlineto{\pgfqpoint{4.279326in}{0.884438in}}%
\pgfpathlineto{\pgfqpoint{4.281650in}{0.854005in}}%
\pgfpathlineto{\pgfqpoint{4.283974in}{0.876355in}}%
\pgfpathlineto{\pgfqpoint{4.286298in}{0.854019in}}%
\pgfpathlineto{\pgfqpoint{4.288622in}{0.874206in}}%
\pgfpathlineto{\pgfqpoint{4.290946in}{0.843768in}}%
\pgfpathlineto{\pgfqpoint{4.293270in}{0.888697in}}%
\pgfpathlineto{\pgfqpoint{4.295594in}{0.856920in}}%
\pgfpathlineto{\pgfqpoint{4.297918in}{0.855151in}}%
\pgfpathlineto{\pgfqpoint{4.300242in}{0.890011in}}%
\pgfpathlineto{\pgfqpoint{4.304890in}{0.865536in}}%
\pgfpathlineto{\pgfqpoint{4.307214in}{0.900483in}}%
\pgfpathlineto{\pgfqpoint{4.309538in}{0.852604in}}%
\pgfpathlineto{\pgfqpoint{4.311862in}{0.870602in}}%
\pgfpathlineto{\pgfqpoint{4.314186in}{0.867797in}}%
\pgfpathlineto{\pgfqpoint{4.316510in}{0.833987in}}%
\pgfpathlineto{\pgfqpoint{4.318834in}{0.898722in}}%
\pgfpathlineto{\pgfqpoint{4.321158in}{0.924107in}}%
\pgfpathlineto{\pgfqpoint{4.323481in}{0.921178in}}%
\pgfpathlineto{\pgfqpoint{4.325805in}{0.850413in}}%
\pgfpathlineto{\pgfqpoint{4.328129in}{0.885927in}}%
\pgfpathlineto{\pgfqpoint{4.330453in}{0.844887in}}%
\pgfpathlineto{\pgfqpoint{4.332777in}{0.849548in}}%
\pgfpathlineto{\pgfqpoint{4.335101in}{0.890799in}}%
\pgfpathlineto{\pgfqpoint{4.337425in}{0.894592in}}%
\pgfpathlineto{\pgfqpoint{4.339749in}{0.870652in}}%
\pgfpathlineto{\pgfqpoint{4.342073in}{0.887751in}}%
\pgfpathlineto{\pgfqpoint{4.344397in}{0.915451in}}%
\pgfpathlineto{\pgfqpoint{4.346721in}{0.880012in}}%
\pgfpathlineto{\pgfqpoint{4.349045in}{0.896742in}}%
\pgfpathlineto{\pgfqpoint{4.351369in}{0.875901in}}%
\pgfpathlineto{\pgfqpoint{4.353693in}{0.867986in}}%
\pgfpathlineto{\pgfqpoint{4.356017in}{0.902639in}}%
\pgfpathlineto{\pgfqpoint{4.358341in}{0.902177in}}%
\pgfpathlineto{\pgfqpoint{4.360665in}{0.868674in}}%
\pgfpathlineto{\pgfqpoint{4.362989in}{0.917878in}}%
\pgfpathlineto{\pgfqpoint{4.365313in}{0.924860in}}%
\pgfpathlineto{\pgfqpoint{4.367637in}{0.880586in}}%
\pgfpathlineto{\pgfqpoint{4.369961in}{0.888473in}}%
\pgfpathlineto{\pgfqpoint{4.372285in}{0.927104in}}%
\pgfpathlineto{\pgfqpoint{4.374609in}{0.911966in}}%
\pgfpathlineto{\pgfqpoint{4.376933in}{0.935200in}}%
\pgfpathlineto{\pgfqpoint{4.379257in}{0.941630in}}%
\pgfpathlineto{\pgfqpoint{4.381581in}{0.879927in}}%
\pgfpathlineto{\pgfqpoint{4.383905in}{0.961464in}}%
\pgfpathlineto{\pgfqpoint{4.386229in}{0.882573in}}%
\pgfpathlineto{\pgfqpoint{4.388553in}{0.896595in}}%
\pgfpathlineto{\pgfqpoint{4.390877in}{0.924634in}}%
\pgfpathlineto{\pgfqpoint{4.393200in}{0.897325in}}%
\pgfpathlineto{\pgfqpoint{4.395524in}{0.919765in}}%
\pgfpathlineto{\pgfqpoint{4.397848in}{0.908542in}}%
\pgfpathlineto{\pgfqpoint{4.400172in}{0.879495in}}%
\pgfpathlineto{\pgfqpoint{4.402496in}{0.901432in}}%
\pgfpathlineto{\pgfqpoint{4.404820in}{0.859538in}}%
\pgfpathlineto{\pgfqpoint{4.407144in}{0.917255in}}%
\pgfpathlineto{\pgfqpoint{4.409468in}{0.922234in}}%
\pgfpathlineto{\pgfqpoint{4.411792in}{0.905467in}}%
\pgfpathlineto{\pgfqpoint{4.414116in}{0.902346in}}%
\pgfpathlineto{\pgfqpoint{4.416440in}{0.882222in}}%
\pgfpathlineto{\pgfqpoint{4.418764in}{0.919286in}}%
\pgfpathlineto{\pgfqpoint{4.421088in}{0.915550in}}%
\pgfpathlineto{\pgfqpoint{4.423412in}{0.949316in}}%
\pgfpathlineto{\pgfqpoint{4.425736in}{0.915590in}}%
\pgfpathlineto{\pgfqpoint{4.428060in}{1.000713in}}%
\pgfpathlineto{\pgfqpoint{4.430384in}{0.972372in}}%
\pgfpathlineto{\pgfqpoint{4.432708in}{0.970315in}}%
\pgfpathlineto{\pgfqpoint{4.435032in}{0.984032in}}%
\pgfpathlineto{\pgfqpoint{4.439680in}{0.946668in}}%
\pgfpathlineto{\pgfqpoint{4.442004in}{0.986939in}}%
\pgfpathlineto{\pgfqpoint{4.444328in}{0.958006in}}%
\pgfpathlineto{\pgfqpoint{4.446652in}{1.020324in}}%
\pgfpathlineto{\pgfqpoint{4.448976in}{0.993272in}}%
\pgfpathlineto{\pgfqpoint{4.451300in}{1.014563in}}%
\pgfpathlineto{\pgfqpoint{4.453624in}{1.016412in}}%
\pgfpathlineto{\pgfqpoint{4.455948in}{1.014992in}}%
\pgfpathlineto{\pgfqpoint{4.458272in}{0.988754in}}%
\pgfpathlineto{\pgfqpoint{4.460596in}{0.992643in}}%
\pgfpathlineto{\pgfqpoint{4.462919in}{1.029096in}}%
\pgfpathlineto{\pgfqpoint{4.465243in}{0.984423in}}%
\pgfpathlineto{\pgfqpoint{4.467567in}{0.976132in}}%
\pgfpathlineto{\pgfqpoint{4.469891in}{0.994412in}}%
\pgfpathlineto{\pgfqpoint{4.472215in}{1.031985in}}%
\pgfpathlineto{\pgfqpoint{4.474539in}{0.967658in}}%
\pgfpathlineto{\pgfqpoint{4.479187in}{0.998262in}}%
\pgfpathlineto{\pgfqpoint{4.481511in}{0.984115in}}%
\pgfpathlineto{\pgfqpoint{4.483835in}{1.019673in}}%
\pgfpathlineto{\pgfqpoint{4.486159in}{0.982982in}}%
\pgfpathlineto{\pgfqpoint{4.488483in}{0.980310in}}%
\pgfpathlineto{\pgfqpoint{4.490807in}{1.030073in}}%
\pgfpathlineto{\pgfqpoint{4.493131in}{1.019177in}}%
\pgfpathlineto{\pgfqpoint{4.495455in}{0.999323in}}%
\pgfpathlineto{\pgfqpoint{4.497779in}{0.991795in}}%
\pgfpathlineto{\pgfqpoint{4.500103in}{1.050407in}}%
\pgfpathlineto{\pgfqpoint{4.502427in}{0.985350in}}%
\pgfpathlineto{\pgfqpoint{4.504751in}{1.006653in}}%
\pgfpathlineto{\pgfqpoint{4.507075in}{1.015423in}}%
\pgfpathlineto{\pgfqpoint{4.509399in}{1.001868in}}%
\pgfpathlineto{\pgfqpoint{4.511723in}{1.052994in}}%
\pgfpathlineto{\pgfqpoint{4.514047in}{1.064106in}}%
\pgfpathlineto{\pgfqpoint{4.518695in}{1.004588in}}%
\pgfpathlineto{\pgfqpoint{4.523343in}{1.074304in}}%
\pgfpathlineto{\pgfqpoint{4.525667in}{1.019372in}}%
\pgfpathlineto{\pgfqpoint{4.527991in}{1.026074in}}%
\pgfpathlineto{\pgfqpoint{4.534962in}{1.063442in}}%
\pgfpathlineto{\pgfqpoint{4.537286in}{1.038652in}}%
\pgfpathlineto{\pgfqpoint{4.539610in}{1.000692in}}%
\pgfpathlineto{\pgfqpoint{4.541934in}{1.009783in}}%
\pgfpathlineto{\pgfqpoint{4.544258in}{1.066843in}}%
\pgfpathlineto{\pgfqpoint{4.546582in}{1.093845in}}%
\pgfpathlineto{\pgfqpoint{4.548906in}{1.069060in}}%
\pgfpathlineto{\pgfqpoint{4.551230in}{1.073159in}}%
\pgfpathlineto{\pgfqpoint{4.553554in}{1.032624in}}%
\pgfpathlineto{\pgfqpoint{4.555878in}{1.016646in}}%
\pgfpathlineto{\pgfqpoint{4.560526in}{1.070829in}}%
\pgfpathlineto{\pgfqpoint{4.562850in}{1.014439in}}%
\pgfpathlineto{\pgfqpoint{4.565174in}{1.066533in}}%
\pgfpathlineto{\pgfqpoint{4.567498in}{1.069311in}}%
\pgfpathlineto{\pgfqpoint{4.569822in}{1.048703in}}%
\pgfpathlineto{\pgfqpoint{4.572146in}{1.069806in}}%
\pgfpathlineto{\pgfqpoint{4.574470in}{1.055750in}}%
\pgfpathlineto{\pgfqpoint{4.576794in}{1.068083in}}%
\pgfpathlineto{\pgfqpoint{4.579118in}{1.073521in}}%
\pgfpathlineto{\pgfqpoint{4.581442in}{1.087897in}}%
\pgfpathlineto{\pgfqpoint{4.583766in}{1.069480in}}%
\pgfpathlineto{\pgfqpoint{4.586090in}{1.087195in}}%
\pgfpathlineto{\pgfqpoint{4.588414in}{1.070117in}}%
\pgfpathlineto{\pgfqpoint{4.590738in}{1.030746in}}%
\pgfpathlineto{\pgfqpoint{4.593062in}{1.032254in}}%
\pgfpathlineto{\pgfqpoint{4.595386in}{1.089393in}}%
\pgfpathlineto{\pgfqpoint{4.600034in}{1.061814in}}%
\pgfpathlineto{\pgfqpoint{4.602358in}{1.144511in}}%
\pgfpathlineto{\pgfqpoint{4.604681in}{1.114557in}}%
\pgfpathlineto{\pgfqpoint{4.607005in}{1.134993in}}%
\pgfpathlineto{\pgfqpoint{4.609329in}{1.052071in}}%
\pgfpathlineto{\pgfqpoint{4.611653in}{1.046738in}}%
\pgfpathlineto{\pgfqpoint{4.616301in}{1.094571in}}%
\pgfpathlineto{\pgfqpoint{4.618625in}{1.141802in}}%
\pgfpathlineto{\pgfqpoint{4.620949in}{1.092871in}}%
\pgfpathlineto{\pgfqpoint{4.623273in}{1.129792in}}%
\pgfpathlineto{\pgfqpoint{4.625597in}{1.101770in}}%
\pgfpathlineto{\pgfqpoint{4.627921in}{1.097176in}}%
\pgfpathlineto{\pgfqpoint{4.630245in}{1.124191in}}%
\pgfpathlineto{\pgfqpoint{4.632569in}{1.057999in}}%
\pgfpathlineto{\pgfqpoint{4.634893in}{1.117525in}}%
\pgfpathlineto{\pgfqpoint{4.637217in}{1.084790in}}%
\pgfpathlineto{\pgfqpoint{4.639541in}{1.095842in}}%
\pgfpathlineto{\pgfqpoint{4.644189in}{1.129646in}}%
\pgfpathlineto{\pgfqpoint{4.646513in}{1.123689in}}%
\pgfpathlineto{\pgfqpoint{4.648837in}{1.126985in}}%
\pgfpathlineto{\pgfqpoint{4.653485in}{1.087074in}}%
\pgfpathlineto{\pgfqpoint{4.655809in}{1.123341in}}%
\pgfpathlineto{\pgfqpoint{4.658133in}{1.106885in}}%
\pgfpathlineto{\pgfqpoint{4.660457in}{1.133153in}}%
\pgfpathlineto{\pgfqpoint{4.662781in}{1.130794in}}%
\pgfpathlineto{\pgfqpoint{4.665105in}{1.122299in}}%
\pgfpathlineto{\pgfqpoint{4.667429in}{1.118489in}}%
\pgfpathlineto{\pgfqpoint{4.669753in}{1.146312in}}%
\pgfpathlineto{\pgfqpoint{4.672077in}{1.139542in}}%
\pgfpathlineto{\pgfqpoint{4.674400in}{1.122734in}}%
\pgfpathlineto{\pgfqpoint{4.676724in}{1.130461in}}%
\pgfpathlineto{\pgfqpoint{4.679048in}{1.169253in}}%
\pgfpathlineto{\pgfqpoint{4.681372in}{1.125698in}}%
\pgfpathlineto{\pgfqpoint{4.683696in}{1.107719in}}%
\pgfpathlineto{\pgfqpoint{4.688344in}{1.136124in}}%
\pgfpathlineto{\pgfqpoint{4.690668in}{1.108822in}}%
\pgfpathlineto{\pgfqpoint{4.692992in}{1.170946in}}%
\pgfpathlineto{\pgfqpoint{4.695316in}{1.123419in}}%
\pgfpathlineto{\pgfqpoint{4.699964in}{1.172761in}}%
\pgfpathlineto{\pgfqpoint{4.702288in}{1.162233in}}%
\pgfpathlineto{\pgfqpoint{4.704612in}{1.135762in}}%
\pgfpathlineto{\pgfqpoint{4.706936in}{1.165909in}}%
\pgfpathlineto{\pgfqpoint{4.709260in}{1.132747in}}%
\pgfpathlineto{\pgfqpoint{4.711584in}{1.151759in}}%
\pgfpathlineto{\pgfqpoint{4.713908in}{1.149362in}}%
\pgfpathlineto{\pgfqpoint{4.716232in}{1.119532in}}%
\pgfpathlineto{\pgfqpoint{4.718556in}{1.158828in}}%
\pgfpathlineto{\pgfqpoint{4.720880in}{1.158511in}}%
\pgfpathlineto{\pgfqpoint{4.723204in}{1.171873in}}%
\pgfpathlineto{\pgfqpoint{4.725528in}{1.134951in}}%
\pgfpathlineto{\pgfqpoint{4.727852in}{1.180041in}}%
\pgfpathlineto{\pgfqpoint{4.730176in}{1.173216in}}%
\pgfpathlineto{\pgfqpoint{4.732500in}{1.170179in}}%
\pgfpathlineto{\pgfqpoint{4.734824in}{1.176060in}}%
\pgfpathlineto{\pgfqpoint{4.739472in}{1.209788in}}%
\pgfpathlineto{\pgfqpoint{4.741796in}{1.205968in}}%
\pgfpathlineto{\pgfqpoint{4.744119in}{1.167049in}}%
\pgfpathlineto{\pgfqpoint{4.746443in}{1.206385in}}%
\pgfpathlineto{\pgfqpoint{4.748767in}{1.199720in}}%
\pgfpathlineto{\pgfqpoint{4.751091in}{1.160235in}}%
\pgfpathlineto{\pgfqpoint{4.753415in}{1.192372in}}%
\pgfpathlineto{\pgfqpoint{4.755739in}{1.189991in}}%
\pgfpathlineto{\pgfqpoint{4.758063in}{1.180908in}}%
\pgfpathlineto{\pgfqpoint{4.760387in}{1.218122in}}%
\pgfpathlineto{\pgfqpoint{4.762711in}{1.190093in}}%
\pgfpathlineto{\pgfqpoint{4.767359in}{1.164589in}}%
\pgfpathlineto{\pgfqpoint{4.769683in}{1.203363in}}%
\pgfpathlineto{\pgfqpoint{4.772007in}{1.210352in}}%
\pgfpathlineto{\pgfqpoint{4.774331in}{1.196900in}}%
\pgfpathlineto{\pgfqpoint{4.776655in}{1.214972in}}%
\pgfpathlineto{\pgfqpoint{4.778979in}{1.198849in}}%
\pgfpathlineto{\pgfqpoint{4.781303in}{1.167030in}}%
\pgfpathlineto{\pgfqpoint{4.783627in}{1.225335in}}%
\pgfpathlineto{\pgfqpoint{4.785951in}{1.222795in}}%
\pgfpathlineto{\pgfqpoint{4.790599in}{1.193794in}}%
\pgfpathlineto{\pgfqpoint{4.792923in}{1.232103in}}%
\pgfpathlineto{\pgfqpoint{4.795247in}{1.190588in}}%
\pgfpathlineto{\pgfqpoint{4.797571in}{1.208159in}}%
\pgfpathlineto{\pgfqpoint{4.799895in}{1.175620in}}%
\pgfpathlineto{\pgfqpoint{4.802219in}{1.189071in}}%
\pgfpathlineto{\pgfqpoint{4.804543in}{1.258476in}}%
\pgfpathlineto{\pgfqpoint{4.809191in}{1.203782in}}%
\pgfpathlineto{\pgfqpoint{4.811515in}{1.203761in}}%
\pgfpathlineto{\pgfqpoint{4.813838in}{1.213134in}}%
\pgfpathlineto{\pgfqpoint{4.816162in}{1.198999in}}%
\pgfpathlineto{\pgfqpoint{4.818486in}{1.173501in}}%
\pgfpathlineto{\pgfqpoint{4.820810in}{1.235381in}}%
\pgfpathlineto{\pgfqpoint{4.823134in}{1.240398in}}%
\pgfpathlineto{\pgfqpoint{4.825458in}{1.170402in}}%
\pgfpathlineto{\pgfqpoint{4.827782in}{1.231417in}}%
\pgfpathlineto{\pgfqpoint{4.830106in}{1.215017in}}%
\pgfpathlineto{\pgfqpoint{4.832430in}{1.180400in}}%
\pgfpathlineto{\pgfqpoint{4.834754in}{1.242665in}}%
\pgfpathlineto{\pgfqpoint{4.837078in}{1.217647in}}%
\pgfpathlineto{\pgfqpoint{4.839402in}{1.261111in}}%
\pgfpathlineto{\pgfqpoint{4.841726in}{1.176454in}}%
\pgfpathlineto{\pgfqpoint{4.844050in}{1.211496in}}%
\pgfpathlineto{\pgfqpoint{4.848698in}{1.192619in}}%
\pgfpathlineto{\pgfqpoint{4.853346in}{1.241491in}}%
\pgfpathlineto{\pgfqpoint{4.855670in}{1.247144in}}%
\pgfpathlineto{\pgfqpoint{4.857994in}{1.177167in}}%
\pgfpathlineto{\pgfqpoint{4.860318in}{1.240724in}}%
\pgfpathlineto{\pgfqpoint{4.862642in}{1.173417in}}%
\pgfpathlineto{\pgfqpoint{4.864966in}{1.232196in}}%
\pgfpathlineto{\pgfqpoint{4.867290in}{1.234761in}}%
\pgfpathlineto{\pgfqpoint{4.869614in}{1.248362in}}%
\pgfpathlineto{\pgfqpoint{4.871938in}{1.226989in}}%
\pgfpathlineto{\pgfqpoint{4.874262in}{1.232444in}}%
\pgfpathlineto{\pgfqpoint{4.876586in}{1.223457in}}%
\pgfpathlineto{\pgfqpoint{4.878910in}{1.226326in}}%
\pgfpathlineto{\pgfqpoint{4.881234in}{1.203267in}}%
\pgfpathlineto{\pgfqpoint{4.883558in}{1.211375in}}%
\pgfpathlineto{\pgfqpoint{4.885881in}{1.247182in}}%
\pgfpathlineto{\pgfqpoint{4.888205in}{1.217184in}}%
\pgfpathlineto{\pgfqpoint{4.890529in}{1.239542in}}%
\pgfpathlineto{\pgfqpoint{4.892853in}{1.244859in}}%
\pgfpathlineto{\pgfqpoint{4.895177in}{1.235302in}}%
\pgfpathlineto{\pgfqpoint{4.897501in}{1.236194in}}%
\pgfpathlineto{\pgfqpoint{4.899825in}{1.223525in}}%
\pgfpathlineto{\pgfqpoint{4.902149in}{1.257472in}}%
\pgfpathlineto{\pgfqpoint{4.904473in}{1.210085in}}%
\pgfpathlineto{\pgfqpoint{4.906797in}{1.247302in}}%
\pgfpathlineto{\pgfqpoint{4.909121in}{1.265173in}}%
\pgfpathlineto{\pgfqpoint{4.911445in}{1.245931in}}%
\pgfpathlineto{\pgfqpoint{4.913769in}{1.256420in}}%
\pgfpathlineto{\pgfqpoint{4.916093in}{1.257428in}}%
\pgfpathlineto{\pgfqpoint{4.918417in}{1.224283in}}%
\pgfpathlineto{\pgfqpoint{4.923065in}{1.275322in}}%
\pgfpathlineto{\pgfqpoint{4.925389in}{1.240402in}}%
\pgfpathlineto{\pgfqpoint{4.927713in}{1.256264in}}%
\pgfpathlineto{\pgfqpoint{4.930037in}{1.243348in}}%
\pgfpathlineto{\pgfqpoint{4.932361in}{1.250559in}}%
\pgfpathlineto{\pgfqpoint{4.934685in}{1.283562in}}%
\pgfpathlineto{\pgfqpoint{4.937009in}{1.223992in}}%
\pgfpathlineto{\pgfqpoint{4.939333in}{1.256265in}}%
\pgfpathlineto{\pgfqpoint{4.941657in}{1.230120in}}%
\pgfpathlineto{\pgfqpoint{4.943981in}{1.263636in}}%
\pgfpathlineto{\pgfqpoint{4.946305in}{1.258306in}}%
\pgfpathlineto{\pgfqpoint{4.948629in}{1.229288in}}%
\pgfpathlineto{\pgfqpoint{4.950953in}{1.252822in}}%
\pgfpathlineto{\pgfqpoint{4.953277in}{1.236013in}}%
\pgfpathlineto{\pgfqpoint{4.955600in}{1.270561in}}%
\pgfpathlineto{\pgfqpoint{4.957924in}{1.254489in}}%
\pgfpathlineto{\pgfqpoint{4.960248in}{1.230566in}}%
\pgfpathlineto{\pgfqpoint{4.964896in}{1.274783in}}%
\pgfpathlineto{\pgfqpoint{4.967220in}{1.244563in}}%
\pgfpathlineto{\pgfqpoint{4.969544in}{1.248893in}}%
\pgfpathlineto{\pgfqpoint{4.971868in}{1.276219in}}%
\pgfpathlineto{\pgfqpoint{4.974192in}{1.286039in}}%
\pgfpathlineto{\pgfqpoint{4.978840in}{1.222914in}}%
\pgfpathlineto{\pgfqpoint{4.981164in}{1.251351in}}%
\pgfpathlineto{\pgfqpoint{4.983488in}{1.252259in}}%
\pgfpathlineto{\pgfqpoint{4.985812in}{1.282394in}}%
\pgfpathlineto{\pgfqpoint{4.988136in}{1.237835in}}%
\pgfpathlineto{\pgfqpoint{4.990460in}{1.277669in}}%
\pgfpathlineto{\pgfqpoint{4.995108in}{1.232074in}}%
\pgfpathlineto{\pgfqpoint{4.997432in}{1.247060in}}%
\pgfpathlineto{\pgfqpoint{4.999756in}{1.290434in}}%
\pgfpathlineto{\pgfqpoint{5.004404in}{1.232171in}}%
\pgfpathlineto{\pgfqpoint{5.006728in}{1.250643in}}%
\pgfpathlineto{\pgfqpoint{5.009052in}{1.286094in}}%
\pgfpathlineto{\pgfqpoint{5.011376in}{1.299479in}}%
\pgfpathlineto{\pgfqpoint{5.013700in}{1.299308in}}%
\pgfpathlineto{\pgfqpoint{5.016024in}{1.307735in}}%
\pgfpathlineto{\pgfqpoint{5.018348in}{1.279569in}}%
\pgfpathlineto{\pgfqpoint{5.020672in}{1.304862in}}%
\pgfpathlineto{\pgfqpoint{5.022996in}{1.305584in}}%
\pgfpathlineto{\pgfqpoint{5.025319in}{1.276394in}}%
\pgfpathlineto{\pgfqpoint{5.027643in}{1.315127in}}%
\pgfpathlineto{\pgfqpoint{5.029967in}{1.314104in}}%
\pgfpathlineto{\pgfqpoint{5.032291in}{1.330260in}}%
\pgfpathlineto{\pgfqpoint{5.034615in}{1.324420in}}%
\pgfpathlineto{\pgfqpoint{5.036939in}{1.270697in}}%
\pgfpathlineto{\pgfqpoint{5.039263in}{1.296346in}}%
\pgfpathlineto{\pgfqpoint{5.041587in}{1.302259in}}%
\pgfpathlineto{\pgfqpoint{5.043911in}{1.296806in}}%
\pgfpathlineto{\pgfqpoint{5.046235in}{1.303198in}}%
\pgfpathlineto{\pgfqpoint{5.048559in}{1.301389in}}%
\pgfpathlineto{\pgfqpoint{5.050883in}{1.278747in}}%
\pgfpathlineto{\pgfqpoint{5.053207in}{1.268733in}}%
\pgfpathlineto{\pgfqpoint{5.055531in}{1.271562in}}%
\pgfpathlineto{\pgfqpoint{5.057855in}{1.303570in}}%
\pgfpathlineto{\pgfqpoint{5.062503in}{1.276215in}}%
\pgfpathlineto{\pgfqpoint{5.064827in}{1.329702in}}%
\pgfpathlineto{\pgfqpoint{5.067151in}{1.265264in}}%
\pgfpathlineto{\pgfqpoint{5.069475in}{1.260646in}}%
\pgfpathlineto{\pgfqpoint{5.071799in}{1.322803in}}%
\pgfpathlineto{\pgfqpoint{5.074123in}{1.314353in}}%
\pgfpathlineto{\pgfqpoint{5.076447in}{1.313314in}}%
\pgfpathlineto{\pgfqpoint{5.078771in}{1.275034in}}%
\pgfpathlineto{\pgfqpoint{5.081095in}{1.352166in}}%
\pgfpathlineto{\pgfqpoint{5.083419in}{1.273062in}}%
\pgfpathlineto{\pgfqpoint{5.085743in}{1.314218in}}%
\pgfpathlineto{\pgfqpoint{5.088067in}{1.298197in}}%
\pgfpathlineto{\pgfqpoint{5.090391in}{1.307118in}}%
\pgfpathlineto{\pgfqpoint{5.092715in}{1.266028in}}%
\pgfpathlineto{\pgfqpoint{5.095039in}{1.265269in}}%
\pgfpathlineto{\pgfqpoint{5.097362in}{1.292166in}}%
\pgfpathlineto{\pgfqpoint{5.099686in}{1.279861in}}%
\pgfpathlineto{\pgfqpoint{5.102010in}{1.288391in}}%
\pgfpathlineto{\pgfqpoint{5.104334in}{1.279114in}}%
\pgfpathlineto{\pgfqpoint{5.106658in}{1.297452in}}%
\pgfpathlineto{\pgfqpoint{5.108982in}{1.347791in}}%
\pgfpathlineto{\pgfqpoint{5.111306in}{1.274328in}}%
\pgfpathlineto{\pgfqpoint{5.113630in}{1.273140in}}%
\pgfpathlineto{\pgfqpoint{5.115954in}{1.297731in}}%
\pgfpathlineto{\pgfqpoint{5.118278in}{1.251936in}}%
\pgfpathlineto{\pgfqpoint{5.120602in}{1.285062in}}%
\pgfpathlineto{\pgfqpoint{5.122926in}{1.277613in}}%
\pgfpathlineto{\pgfqpoint{5.127574in}{1.315915in}}%
\pgfpathlineto{\pgfqpoint{5.132222in}{1.284365in}}%
\pgfpathlineto{\pgfqpoint{5.134546in}{1.284262in}}%
\pgfpathlineto{\pgfqpoint{5.136870in}{1.248576in}}%
\pgfpathlineto{\pgfqpoint{5.141518in}{1.316264in}}%
\pgfpathlineto{\pgfqpoint{5.146166in}{1.274368in}}%
\pgfpathlineto{\pgfqpoint{5.148490in}{1.328260in}}%
\pgfpathlineto{\pgfqpoint{5.150814in}{1.280475in}}%
\pgfpathlineto{\pgfqpoint{5.153138in}{1.284896in}}%
\pgfpathlineto{\pgfqpoint{5.155462in}{1.265057in}}%
\pgfpathlineto{\pgfqpoint{5.157786in}{1.288413in}}%
\pgfpathlineto{\pgfqpoint{5.160110in}{1.256662in}}%
\pgfpathlineto{\pgfqpoint{5.162434in}{1.266442in}}%
\pgfpathlineto{\pgfqpoint{5.164758in}{1.288550in}}%
\pgfpathlineto{\pgfqpoint{5.169405in}{1.226952in}}%
\pgfpathlineto{\pgfqpoint{5.171729in}{1.242289in}}%
\pgfpathlineto{\pgfqpoint{5.174053in}{1.248803in}}%
\pgfpathlineto{\pgfqpoint{5.176377in}{1.222318in}}%
\pgfpathlineto{\pgfqpoint{5.178701in}{1.284638in}}%
\pgfpathlineto{\pgfqpoint{5.181025in}{1.255838in}}%
\pgfpathlineto{\pgfqpoint{5.183349in}{1.317888in}}%
\pgfpathlineto{\pgfqpoint{5.185673in}{1.261765in}}%
\pgfpathlineto{\pgfqpoint{5.187997in}{1.271385in}}%
\pgfpathlineto{\pgfqpoint{5.190321in}{1.274184in}}%
\pgfpathlineto{\pgfqpoint{5.192645in}{1.300899in}}%
\pgfpathlineto{\pgfqpoint{5.194969in}{1.264913in}}%
\pgfpathlineto{\pgfqpoint{5.197293in}{1.316901in}}%
\pgfpathlineto{\pgfqpoint{5.199617in}{1.243393in}}%
\pgfpathlineto{\pgfqpoint{5.201941in}{1.254667in}}%
\pgfpathlineto{\pgfqpoint{5.204265in}{1.244457in}}%
\pgfpathlineto{\pgfqpoint{5.206589in}{1.244756in}}%
\pgfpathlineto{\pgfqpoint{5.211237in}{1.283567in}}%
\pgfpathlineto{\pgfqpoint{5.213561in}{1.276809in}}%
\pgfpathlineto{\pgfqpoint{5.215885in}{1.238962in}}%
\pgfpathlineto{\pgfqpoint{5.218209in}{1.223721in}}%
\pgfpathlineto{\pgfqpoint{5.222857in}{1.249923in}}%
\pgfpathlineto{\pgfqpoint{5.225181in}{1.235162in}}%
\pgfpathlineto{\pgfqpoint{5.227505in}{1.254136in}}%
\pgfpathlineto{\pgfqpoint{5.229829in}{1.221324in}}%
\pgfpathlineto{\pgfqpoint{5.232153in}{1.263491in}}%
\pgfpathlineto{\pgfqpoint{5.234477in}{1.240460in}}%
\pgfpathlineto{\pgfqpoint{5.236800in}{1.249724in}}%
\pgfpathlineto{\pgfqpoint{5.239124in}{1.263772in}}%
\pgfpathlineto{\pgfqpoint{5.241448in}{1.253463in}}%
\pgfpathlineto{\pgfqpoint{5.243772in}{1.190138in}}%
\pgfpathlineto{\pgfqpoint{5.246096in}{1.260670in}}%
\pgfpathlineto{\pgfqpoint{5.248420in}{1.217659in}}%
\pgfpathlineto{\pgfqpoint{5.250744in}{1.217496in}}%
\pgfpathlineto{\pgfqpoint{5.253068in}{1.242819in}}%
\pgfpathlineto{\pgfqpoint{5.255392in}{1.211205in}}%
\pgfpathlineto{\pgfqpoint{5.257716in}{1.226250in}}%
\pgfpathlineto{\pgfqpoint{5.260040in}{1.212646in}}%
\pgfpathlineto{\pgfqpoint{5.262364in}{1.216725in}}%
\pgfpathlineto{\pgfqpoint{5.264688in}{1.203603in}}%
\pgfpathlineto{\pgfqpoint{5.267012in}{1.226433in}}%
\pgfpathlineto{\pgfqpoint{5.269336in}{1.191495in}}%
\pgfpathlineto{\pgfqpoint{5.271660in}{1.233988in}}%
\pgfpathlineto{\pgfqpoint{5.273984in}{1.195390in}}%
\pgfpathlineto{\pgfqpoint{5.276308in}{1.232189in}}%
\pgfpathlineto{\pgfqpoint{5.278632in}{1.193126in}}%
\pgfpathlineto{\pgfqpoint{5.280956in}{1.217988in}}%
\pgfpathlineto{\pgfqpoint{5.283280in}{1.221393in}}%
\pgfpathlineto{\pgfqpoint{5.285604in}{1.204674in}}%
\pgfpathlineto{\pgfqpoint{5.290252in}{1.219666in}}%
\pgfpathlineto{\pgfqpoint{5.292576in}{1.194599in}}%
\pgfpathlineto{\pgfqpoint{5.294900in}{1.195897in}}%
\pgfpathlineto{\pgfqpoint{5.299548in}{1.216320in}}%
\pgfpathlineto{\pgfqpoint{5.301872in}{1.217343in}}%
\pgfpathlineto{\pgfqpoint{5.304196in}{1.232643in}}%
\pgfpathlineto{\pgfqpoint{5.306519in}{1.198353in}}%
\pgfpathlineto{\pgfqpoint{5.308843in}{1.199610in}}%
\pgfpathlineto{\pgfqpoint{5.311167in}{1.206029in}}%
\pgfpathlineto{\pgfqpoint{5.313491in}{1.178848in}}%
\pgfpathlineto{\pgfqpoint{5.315815in}{1.198317in}}%
\pgfpathlineto{\pgfqpoint{5.322787in}{1.184138in}}%
\pgfpathlineto{\pgfqpoint{5.325111in}{1.176611in}}%
\pgfpathlineto{\pgfqpoint{5.327435in}{1.196749in}}%
\pgfpathlineto{\pgfqpoint{5.329759in}{1.172840in}}%
\pgfpathlineto{\pgfqpoint{5.332083in}{1.211920in}}%
\pgfpathlineto{\pgfqpoint{5.334407in}{1.178002in}}%
\pgfpathlineto{\pgfqpoint{5.336731in}{1.172673in}}%
\pgfpathlineto{\pgfqpoint{5.339055in}{1.144886in}}%
\pgfpathlineto{\pgfqpoint{5.341379in}{1.193637in}}%
\pgfpathlineto{\pgfqpoint{5.343703in}{1.182773in}}%
\pgfpathlineto{\pgfqpoint{5.346027in}{1.126451in}}%
\pgfpathlineto{\pgfqpoint{5.348351in}{1.201262in}}%
\pgfpathlineto{\pgfqpoint{5.352999in}{1.162754in}}%
\pgfpathlineto{\pgfqpoint{5.355323in}{1.177383in}}%
\pgfpathlineto{\pgfqpoint{5.357647in}{1.135329in}}%
\pgfpathlineto{\pgfqpoint{5.359971in}{1.137681in}}%
\pgfpathlineto{\pgfqpoint{5.362295in}{1.146638in}}%
\pgfpathlineto{\pgfqpoint{5.364619in}{1.174878in}}%
\pgfpathlineto{\pgfqpoint{5.366943in}{1.143294in}}%
\pgfpathlineto{\pgfqpoint{5.369267in}{1.133612in}}%
\pgfpathlineto{\pgfqpoint{5.371591in}{1.140471in}}%
\pgfpathlineto{\pgfqpoint{5.373915in}{1.160649in}}%
\pgfpathlineto{\pgfqpoint{5.378562in}{1.172945in}}%
\pgfpathlineto{\pgfqpoint{5.380886in}{1.140756in}}%
\pgfpathlineto{\pgfqpoint{5.383210in}{1.166260in}}%
\pgfpathlineto{\pgfqpoint{5.385534in}{1.140648in}}%
\pgfpathlineto{\pgfqpoint{5.387858in}{1.176392in}}%
\pgfpathlineto{\pgfqpoint{5.392506in}{1.116640in}}%
\pgfpathlineto{\pgfqpoint{5.397154in}{1.128123in}}%
\pgfpathlineto{\pgfqpoint{5.399478in}{1.152481in}}%
\pgfpathlineto{\pgfqpoint{5.401802in}{1.162387in}}%
\pgfpathlineto{\pgfqpoint{5.404126in}{1.156507in}}%
\pgfpathlineto{\pgfqpoint{5.406450in}{1.142681in}}%
\pgfpathlineto{\pgfqpoint{5.408774in}{1.161752in}}%
\pgfpathlineto{\pgfqpoint{5.411098in}{1.115827in}}%
\pgfpathlineto{\pgfqpoint{5.413422in}{1.117246in}}%
\pgfpathlineto{\pgfqpoint{5.415746in}{1.127288in}}%
\pgfpathlineto{\pgfqpoint{5.418070in}{1.123299in}}%
\pgfpathlineto{\pgfqpoint{5.420394in}{1.115605in}}%
\pgfpathlineto{\pgfqpoint{5.422718in}{1.121552in}}%
\pgfpathlineto{\pgfqpoint{5.425042in}{1.119571in}}%
\pgfpathlineto{\pgfqpoint{5.429690in}{1.139451in}}%
\pgfpathlineto{\pgfqpoint{5.432014in}{1.092605in}}%
\pgfpathlineto{\pgfqpoint{5.434338in}{1.117528in}}%
\pgfpathlineto{\pgfqpoint{5.436662in}{1.076651in}}%
\pgfpathlineto{\pgfqpoint{5.438986in}{1.115788in}}%
\pgfpathlineto{\pgfqpoint{5.441310in}{1.088956in}}%
\pgfpathlineto{\pgfqpoint{5.443634in}{1.135505in}}%
\pgfpathlineto{\pgfqpoint{5.445958in}{1.088392in}}%
\pgfpathlineto{\pgfqpoint{5.448281in}{1.104757in}}%
\pgfpathlineto{\pgfqpoint{5.450605in}{1.094848in}}%
\pgfpathlineto{\pgfqpoint{5.452929in}{1.110642in}}%
\pgfpathlineto{\pgfqpoint{5.455253in}{1.065579in}}%
\pgfpathlineto{\pgfqpoint{5.457577in}{1.048288in}}%
\pgfpathlineto{\pgfqpoint{5.459901in}{1.067692in}}%
\pgfpathlineto{\pgfqpoint{5.462225in}{1.104191in}}%
\pgfpathlineto{\pgfqpoint{5.464549in}{1.090362in}}%
\pgfpathlineto{\pgfqpoint{5.469197in}{1.111807in}}%
\pgfpathlineto{\pgfqpoint{5.473845in}{1.045525in}}%
\pgfpathlineto{\pgfqpoint{5.476169in}{1.070567in}}%
\pgfpathlineto{\pgfqpoint{5.480817in}{1.036250in}}%
\pgfpathlineto{\pgfqpoint{5.483141in}{1.060854in}}%
\pgfpathlineto{\pgfqpoint{5.485465in}{1.064719in}}%
\pgfpathlineto{\pgfqpoint{5.487789in}{1.055156in}}%
\pgfpathlineto{\pgfqpoint{5.490113in}{1.086260in}}%
\pgfpathlineto{\pgfqpoint{5.492437in}{1.050294in}}%
\pgfpathlineto{\pgfqpoint{5.494761in}{1.042542in}}%
\pgfpathlineto{\pgfqpoint{5.497085in}{1.017282in}}%
\pgfpathlineto{\pgfqpoint{5.499409in}{1.028701in}}%
\pgfpathlineto{\pgfqpoint{5.501733in}{1.050081in}}%
\pgfpathlineto{\pgfqpoint{5.504057in}{1.050515in}}%
\pgfpathlineto{\pgfqpoint{5.506381in}{1.055126in}}%
\pgfpathlineto{\pgfqpoint{5.508705in}{1.026346in}}%
\pgfpathlineto{\pgfqpoint{5.511029in}{1.041118in}}%
\pgfpathlineto{\pgfqpoint{5.513353in}{1.036461in}}%
\pgfpathlineto{\pgfqpoint{5.515677in}{1.027548in}}%
\pgfpathlineto{\pgfqpoint{5.518000in}{1.028972in}}%
\pgfpathlineto{\pgfqpoint{5.520324in}{1.039350in}}%
\pgfpathlineto{\pgfqpoint{5.522648in}{0.996413in}}%
\pgfpathlineto{\pgfqpoint{5.524972in}{1.033141in}}%
\pgfpathlineto{\pgfqpoint{5.527296in}{1.017085in}}%
\pgfpathlineto{\pgfqpoint{5.529620in}{1.010893in}}%
\pgfpathlineto{\pgfqpoint{5.531944in}{1.017971in}}%
\pgfpathlineto{\pgfqpoint{5.534268in}{1.053399in}}%
\pgfpathlineto{\pgfqpoint{5.538916in}{1.013149in}}%
\pgfpathlineto{\pgfqpoint{5.541240in}{1.000693in}}%
\pgfpathlineto{\pgfqpoint{5.543564in}{1.017492in}}%
\pgfpathlineto{\pgfqpoint{5.545888in}{1.019498in}}%
\pgfpathlineto{\pgfqpoint{5.548212in}{1.027771in}}%
\pgfpathlineto{\pgfqpoint{5.550536in}{1.014845in}}%
\pgfpathlineto{\pgfqpoint{5.552860in}{0.993670in}}%
\pgfpathlineto{\pgfqpoint{5.555184in}{1.000348in}}%
\pgfpathlineto{\pgfqpoint{5.557508in}{1.021811in}}%
\pgfpathlineto{\pgfqpoint{5.559832in}{0.983673in}}%
\pgfpathlineto{\pgfqpoint{5.564480in}{1.006799in}}%
\pgfpathlineto{\pgfqpoint{5.566804in}{1.051393in}}%
\pgfpathlineto{\pgfqpoint{5.569128in}{0.991501in}}%
\pgfpathlineto{\pgfqpoint{5.571452in}{0.996808in}}%
\pgfpathlineto{\pgfqpoint{5.576100in}{0.951785in}}%
\pgfpathlineto{\pgfqpoint{5.578424in}{0.971151in}}%
\pgfpathlineto{\pgfqpoint{5.580748in}{1.008082in}}%
\pgfpathlineto{\pgfqpoint{5.583072in}{0.974591in}}%
\pgfpathlineto{\pgfqpoint{5.585396in}{0.971703in}}%
\pgfpathlineto{\pgfqpoint{5.587720in}{1.017072in}}%
\pgfpathlineto{\pgfqpoint{5.587720in}{1.017072in}}%
\pgfusepath{stroke}%
\end{pgfscope}%
\begin{pgfscope}%
\pgfpathrectangle{\pgfqpoint{0.709829in}{0.654666in}}{\pgfqpoint{5.110171in}{0.887537in}}%
\pgfusepath{clip}%
\pgfsetroundcap%
\pgfsetroundjoin%
\pgfsetlinewidth{1.003750pt}%
\definecolor{currentstroke}{rgb}{0.333333,0.658824,0.407843}%
\pgfsetstrokecolor{currentstroke}%
\pgfsetdash{}{0pt}%
\pgfpathmoveto{\pgfqpoint{0.942110in}{0.917038in}}%
\pgfpathlineto{\pgfqpoint{0.944433in}{0.949156in}}%
\pgfpathlineto{\pgfqpoint{0.949081in}{0.908031in}}%
\pgfpathlineto{\pgfqpoint{0.953729in}{0.934926in}}%
\pgfpathlineto{\pgfqpoint{0.956053in}{0.962677in}}%
\pgfpathlineto{\pgfqpoint{0.960701in}{0.929673in}}%
\pgfpathlineto{\pgfqpoint{0.963025in}{0.911895in}}%
\pgfpathlineto{\pgfqpoint{0.965349in}{0.943287in}}%
\pgfpathlineto{\pgfqpoint{0.967673in}{0.922906in}}%
\pgfpathlineto{\pgfqpoint{0.969997in}{0.941121in}}%
\pgfpathlineto{\pgfqpoint{0.972321in}{0.947543in}}%
\pgfpathlineto{\pgfqpoint{0.974645in}{0.939164in}}%
\pgfpathlineto{\pgfqpoint{0.976969in}{0.916724in}}%
\pgfpathlineto{\pgfqpoint{0.979293in}{0.940457in}}%
\pgfpathlineto{\pgfqpoint{0.981617in}{0.948266in}}%
\pgfpathlineto{\pgfqpoint{0.983941in}{0.932464in}}%
\pgfpathlineto{\pgfqpoint{0.988589in}{0.942319in}}%
\pgfpathlineto{\pgfqpoint{0.990913in}{0.935961in}}%
\pgfpathlineto{\pgfqpoint{0.993237in}{0.944246in}}%
\pgfpathlineto{\pgfqpoint{0.997885in}{0.928437in}}%
\pgfpathlineto{\pgfqpoint{1.004857in}{0.960055in}}%
\pgfpathlineto{\pgfqpoint{1.007181in}{0.926918in}}%
\pgfpathlineto{\pgfqpoint{1.009505in}{0.918684in}}%
\pgfpathlineto{\pgfqpoint{1.011829in}{0.935714in}}%
\pgfpathlineto{\pgfqpoint{1.014153in}{0.929948in}}%
\pgfpathlineto{\pgfqpoint{1.016476in}{0.939706in}}%
\pgfpathlineto{\pgfqpoint{1.018800in}{0.961735in}}%
\pgfpathlineto{\pgfqpoint{1.021124in}{0.954050in}}%
\pgfpathlineto{\pgfqpoint{1.023448in}{0.951915in}}%
\pgfpathlineto{\pgfqpoint{1.025772in}{0.919985in}}%
\pgfpathlineto{\pgfqpoint{1.028096in}{0.912209in}}%
\pgfpathlineto{\pgfqpoint{1.030420in}{0.927439in}}%
\pgfpathlineto{\pgfqpoint{1.032744in}{0.927932in}}%
\pgfpathlineto{\pgfqpoint{1.035068in}{0.934154in}}%
\pgfpathlineto{\pgfqpoint{1.037392in}{0.943676in}}%
\pgfpathlineto{\pgfqpoint{1.039716in}{0.934732in}}%
\pgfpathlineto{\pgfqpoint{1.042040in}{0.932632in}}%
\pgfpathlineto{\pgfqpoint{1.044364in}{0.949031in}}%
\pgfpathlineto{\pgfqpoint{1.046688in}{0.947451in}}%
\pgfpathlineto{\pgfqpoint{1.049012in}{0.961311in}}%
\pgfpathlineto{\pgfqpoint{1.051336in}{0.934885in}}%
\pgfpathlineto{\pgfqpoint{1.053660in}{0.962729in}}%
\pgfpathlineto{\pgfqpoint{1.055984in}{0.948896in}}%
\pgfpathlineto{\pgfqpoint{1.058308in}{0.957739in}}%
\pgfpathlineto{\pgfqpoint{1.060632in}{0.951176in}}%
\pgfpathlineto{\pgfqpoint{1.062956in}{0.960078in}}%
\pgfpathlineto{\pgfqpoint{1.065280in}{0.915432in}}%
\pgfpathlineto{\pgfqpoint{1.067604in}{0.964173in}}%
\pgfpathlineto{\pgfqpoint{1.069928in}{0.966246in}}%
\pgfpathlineto{\pgfqpoint{1.072252in}{0.935821in}}%
\pgfpathlineto{\pgfqpoint{1.074576in}{0.957604in}}%
\pgfpathlineto{\pgfqpoint{1.076900in}{0.956856in}}%
\pgfpathlineto{\pgfqpoint{1.079224in}{0.943785in}}%
\pgfpathlineto{\pgfqpoint{1.081548in}{0.951945in}}%
\pgfpathlineto{\pgfqpoint{1.083872in}{0.966435in}}%
\pgfpathlineto{\pgfqpoint{1.090843in}{0.938065in}}%
\pgfpathlineto{\pgfqpoint{1.093167in}{0.938400in}}%
\pgfpathlineto{\pgfqpoint{1.095491in}{0.961582in}}%
\pgfpathlineto{\pgfqpoint{1.097815in}{0.942030in}}%
\pgfpathlineto{\pgfqpoint{1.102463in}{0.950302in}}%
\pgfpathlineto{\pgfqpoint{1.104787in}{0.941755in}}%
\pgfpathlineto{\pgfqpoint{1.107111in}{0.953054in}}%
\pgfpathlineto{\pgfqpoint{1.109435in}{0.920001in}}%
\pgfpathlineto{\pgfqpoint{1.114083in}{0.973158in}}%
\pgfpathlineto{\pgfqpoint{1.116407in}{0.949002in}}%
\pgfpathlineto{\pgfqpoint{1.118731in}{0.970228in}}%
\pgfpathlineto{\pgfqpoint{1.121055in}{0.944299in}}%
\pgfpathlineto{\pgfqpoint{1.123379in}{0.931204in}}%
\pgfpathlineto{\pgfqpoint{1.128027in}{0.952979in}}%
\pgfpathlineto{\pgfqpoint{1.130351in}{0.975810in}}%
\pgfpathlineto{\pgfqpoint{1.132675in}{0.970272in}}%
\pgfpathlineto{\pgfqpoint{1.134999in}{0.960532in}}%
\pgfpathlineto{\pgfqpoint{1.137323in}{0.944503in}}%
\pgfpathlineto{\pgfqpoint{1.139647in}{0.974076in}}%
\pgfpathlineto{\pgfqpoint{1.141971in}{0.970023in}}%
\pgfpathlineto{\pgfqpoint{1.146619in}{0.944962in}}%
\pgfpathlineto{\pgfqpoint{1.148943in}{0.954294in}}%
\pgfpathlineto{\pgfqpoint{1.151267in}{0.923541in}}%
\pgfpathlineto{\pgfqpoint{1.155914in}{0.986308in}}%
\pgfpathlineto{\pgfqpoint{1.158238in}{0.947179in}}%
\pgfpathlineto{\pgfqpoint{1.160562in}{0.955841in}}%
\pgfpathlineto{\pgfqpoint{1.162886in}{0.939935in}}%
\pgfpathlineto{\pgfqpoint{1.165210in}{0.971582in}}%
\pgfpathlineto{\pgfqpoint{1.167534in}{0.977817in}}%
\pgfpathlineto{\pgfqpoint{1.169858in}{0.959931in}}%
\pgfpathlineto{\pgfqpoint{1.172182in}{0.954363in}}%
\pgfpathlineto{\pgfqpoint{1.176830in}{0.966435in}}%
\pgfpathlineto{\pgfqpoint{1.179154in}{0.944560in}}%
\pgfpathlineto{\pgfqpoint{1.181478in}{0.944727in}}%
\pgfpathlineto{\pgfqpoint{1.183802in}{0.962112in}}%
\pgfpathlineto{\pgfqpoint{1.186126in}{0.965164in}}%
\pgfpathlineto{\pgfqpoint{1.188450in}{0.971043in}}%
\pgfpathlineto{\pgfqpoint{1.190774in}{0.973058in}}%
\pgfpathlineto{\pgfqpoint{1.195422in}{0.952078in}}%
\pgfpathlineto{\pgfqpoint{1.197746in}{0.950495in}}%
\pgfpathlineto{\pgfqpoint{1.200070in}{0.945713in}}%
\pgfpathlineto{\pgfqpoint{1.202394in}{0.970067in}}%
\pgfpathlineto{\pgfqpoint{1.204718in}{0.975648in}}%
\pgfpathlineto{\pgfqpoint{1.209366in}{0.960339in}}%
\pgfpathlineto{\pgfqpoint{1.211690in}{0.976950in}}%
\pgfpathlineto{\pgfqpoint{1.214014in}{0.974346in}}%
\pgfpathlineto{\pgfqpoint{1.216338in}{0.988053in}}%
\pgfpathlineto{\pgfqpoint{1.218662in}{0.954305in}}%
\pgfpathlineto{\pgfqpoint{1.220986in}{0.953786in}}%
\pgfpathlineto{\pgfqpoint{1.223310in}{0.964069in}}%
\pgfpathlineto{\pgfqpoint{1.225633in}{0.968370in}}%
\pgfpathlineto{\pgfqpoint{1.227957in}{0.978417in}}%
\pgfpathlineto{\pgfqpoint{1.230281in}{0.966963in}}%
\pgfpathlineto{\pgfqpoint{1.232605in}{1.271259in}}%
\pgfpathlineto{\pgfqpoint{1.234929in}{1.272410in}}%
\pgfpathlineto{\pgfqpoint{1.237253in}{1.257211in}}%
\pgfpathlineto{\pgfqpoint{1.239577in}{1.257625in}}%
\pgfpathlineto{\pgfqpoint{1.241901in}{1.246301in}}%
\pgfpathlineto{\pgfqpoint{1.244225in}{1.253729in}}%
\pgfpathlineto{\pgfqpoint{1.246549in}{1.255015in}}%
\pgfpathlineto{\pgfqpoint{1.248873in}{1.232541in}}%
\pgfpathlineto{\pgfqpoint{1.251197in}{1.264929in}}%
\pgfpathlineto{\pgfqpoint{1.253521in}{1.276360in}}%
\pgfpathlineto{\pgfqpoint{1.255845in}{1.251245in}}%
\pgfpathlineto{\pgfqpoint{1.258169in}{1.267711in}}%
\pgfpathlineto{\pgfqpoint{1.260493in}{1.254189in}}%
\pgfpathlineto{\pgfqpoint{1.262817in}{1.250223in}}%
\pgfpathlineto{\pgfqpoint{1.265141in}{1.230226in}}%
\pgfpathlineto{\pgfqpoint{1.267465in}{1.283351in}}%
\pgfpathlineto{\pgfqpoint{1.269789in}{1.249648in}}%
\pgfpathlineto{\pgfqpoint{1.272113in}{1.254859in}}%
\pgfpathlineto{\pgfqpoint{1.274437in}{1.270835in}}%
\pgfpathlineto{\pgfqpoint{1.276761in}{1.267810in}}%
\pgfpathlineto{\pgfqpoint{1.279085in}{1.254501in}}%
\pgfpathlineto{\pgfqpoint{1.281409in}{1.259649in}}%
\pgfpathlineto{\pgfqpoint{1.283733in}{1.247195in}}%
\pgfpathlineto{\pgfqpoint{1.286057in}{1.247920in}}%
\pgfpathlineto{\pgfqpoint{1.288381in}{1.250750in}}%
\pgfpathlineto{\pgfqpoint{1.290705in}{1.246485in}}%
\pgfpathlineto{\pgfqpoint{1.293029in}{1.226576in}}%
\pgfpathlineto{\pgfqpoint{1.297676in}{1.266395in}}%
\pgfpathlineto{\pgfqpoint{1.300000in}{1.260780in}}%
\pgfpathlineto{\pgfqpoint{1.304648in}{1.278768in}}%
\pgfpathlineto{\pgfqpoint{1.309296in}{1.242627in}}%
\pgfpathlineto{\pgfqpoint{1.311620in}{1.249250in}}%
\pgfpathlineto{\pgfqpoint{1.316268in}{1.294175in}}%
\pgfpathlineto{\pgfqpoint{1.318592in}{1.261176in}}%
\pgfpathlineto{\pgfqpoint{1.320916in}{1.259320in}}%
\pgfpathlineto{\pgfqpoint{1.323240in}{1.224111in}}%
\pgfpathlineto{\pgfqpoint{1.325564in}{1.262322in}}%
\pgfpathlineto{\pgfqpoint{1.327888in}{1.249371in}}%
\pgfpathlineto{\pgfqpoint{1.332536in}{1.278263in}}%
\pgfpathlineto{\pgfqpoint{1.334860in}{1.233491in}}%
\pgfpathlineto{\pgfqpoint{1.337184in}{1.281602in}}%
\pgfpathlineto{\pgfqpoint{1.339508in}{1.250056in}}%
\pgfpathlineto{\pgfqpoint{1.341832in}{1.257828in}}%
\pgfpathlineto{\pgfqpoint{1.344156in}{1.274455in}}%
\pgfpathlineto{\pgfqpoint{1.346480in}{1.249845in}}%
\pgfpathlineto{\pgfqpoint{1.348804in}{1.239542in}}%
\pgfpathlineto{\pgfqpoint{1.351128in}{1.287637in}}%
\pgfpathlineto{\pgfqpoint{1.353452in}{1.262176in}}%
\pgfpathlineto{\pgfqpoint{1.355776in}{1.259281in}}%
\pgfpathlineto{\pgfqpoint{1.360424in}{1.261083in}}%
\pgfpathlineto{\pgfqpoint{1.362748in}{1.268729in}}%
\pgfpathlineto{\pgfqpoint{1.365072in}{1.240240in}}%
\pgfpathlineto{\pgfqpoint{1.369719in}{1.271692in}}%
\pgfpathlineto{\pgfqpoint{1.372043in}{1.258008in}}%
\pgfpathlineto{\pgfqpoint{1.374367in}{1.228543in}}%
\pgfpathlineto{\pgfqpoint{1.376691in}{1.286758in}}%
\pgfpathlineto{\pgfqpoint{1.379015in}{1.255628in}}%
\pgfpathlineto{\pgfqpoint{1.381339in}{1.255773in}}%
\pgfpathlineto{\pgfqpoint{1.383663in}{1.258239in}}%
\pgfpathlineto{\pgfqpoint{1.385987in}{1.270289in}}%
\pgfpathlineto{\pgfqpoint{1.388311in}{1.260005in}}%
\pgfpathlineto{\pgfqpoint{1.390635in}{1.269699in}}%
\pgfpathlineto{\pgfqpoint{1.392959in}{1.270014in}}%
\pgfpathlineto{\pgfqpoint{1.395283in}{1.255280in}}%
\pgfpathlineto{\pgfqpoint{1.397607in}{1.247718in}}%
\pgfpathlineto{\pgfqpoint{1.399931in}{1.278946in}}%
\pgfpathlineto{\pgfqpoint{1.404579in}{1.263630in}}%
\pgfpathlineto{\pgfqpoint{1.406903in}{1.274876in}}%
\pgfpathlineto{\pgfqpoint{1.409227in}{1.269596in}}%
\pgfpathlineto{\pgfqpoint{1.411551in}{1.258588in}}%
\pgfpathlineto{\pgfqpoint{1.413875in}{1.277875in}}%
\pgfpathlineto{\pgfqpoint{1.420847in}{1.254684in}}%
\pgfpathlineto{\pgfqpoint{1.423171in}{1.268058in}}%
\pgfpathlineto{\pgfqpoint{1.425495in}{1.259128in}}%
\pgfpathlineto{\pgfqpoint{1.427819in}{1.268837in}}%
\pgfpathlineto{\pgfqpoint{1.430143in}{1.257419in}}%
\pgfpathlineto{\pgfqpoint{1.432467in}{1.258347in}}%
\pgfpathlineto{\pgfqpoint{1.437114in}{1.249608in}}%
\pgfpathlineto{\pgfqpoint{1.439438in}{1.283198in}}%
\pgfpathlineto{\pgfqpoint{1.441762in}{1.269218in}}%
\pgfpathlineto{\pgfqpoint{1.444086in}{1.264944in}}%
\pgfpathlineto{\pgfqpoint{1.446410in}{1.255412in}}%
\pgfpathlineto{\pgfqpoint{1.451058in}{1.276735in}}%
\pgfpathlineto{\pgfqpoint{1.453382in}{1.256054in}}%
\pgfpathlineto{\pgfqpoint{1.458030in}{1.298055in}}%
\pgfpathlineto{\pgfqpoint{1.460354in}{1.293606in}}%
\pgfpathlineto{\pgfqpoint{1.462678in}{1.258626in}}%
\pgfpathlineto{\pgfqpoint{1.465002in}{1.252405in}}%
\pgfpathlineto{\pgfqpoint{1.467326in}{1.280295in}}%
\pgfpathlineto{\pgfqpoint{1.471974in}{1.230883in}}%
\pgfpathlineto{\pgfqpoint{1.474298in}{1.231817in}}%
\pgfpathlineto{\pgfqpoint{1.476622in}{1.292631in}}%
\pgfpathlineto{\pgfqpoint{1.478946in}{1.248663in}}%
\pgfpathlineto{\pgfqpoint{1.483594in}{1.284988in}}%
\pgfpathlineto{\pgfqpoint{1.485918in}{1.266873in}}%
\pgfpathlineto{\pgfqpoint{1.490566in}{1.282523in}}%
\pgfpathlineto{\pgfqpoint{1.492890in}{1.273538in}}%
\pgfpathlineto{\pgfqpoint{1.495214in}{1.283569in}}%
\pgfpathlineto{\pgfqpoint{1.497538in}{1.279064in}}%
\pgfpathlineto{\pgfqpoint{1.499862in}{1.265502in}}%
\pgfpathlineto{\pgfqpoint{1.502186in}{1.278104in}}%
\pgfpathlineto{\pgfqpoint{1.504510in}{1.259872in}}%
\pgfpathlineto{\pgfqpoint{1.506834in}{1.281910in}}%
\pgfpathlineto{\pgfqpoint{1.509157in}{1.232115in}}%
\pgfpathlineto{\pgfqpoint{1.511481in}{1.285307in}}%
\pgfpathlineto{\pgfqpoint{1.513805in}{1.284791in}}%
\pgfpathlineto{\pgfqpoint{1.516129in}{1.274475in}}%
\pgfpathlineto{\pgfqpoint{1.518453in}{1.270593in}}%
\pgfpathlineto{\pgfqpoint{1.520777in}{1.264190in}}%
\pgfpathlineto{\pgfqpoint{1.523101in}{0.936543in}}%
\pgfpathlineto{\pgfqpoint{1.525425in}{0.937157in}}%
\pgfpathlineto{\pgfqpoint{1.527749in}{0.944483in}}%
\pgfpathlineto{\pgfqpoint{1.530073in}{0.937377in}}%
\pgfpathlineto{\pgfqpoint{1.532397in}{0.936985in}}%
\pgfpathlineto{\pgfqpoint{1.534721in}{0.942821in}}%
\pgfpathlineto{\pgfqpoint{1.537045in}{0.956058in}}%
\pgfpathlineto{\pgfqpoint{1.539369in}{0.941055in}}%
\pgfpathlineto{\pgfqpoint{1.544017in}{0.942553in}}%
\pgfpathlineto{\pgfqpoint{1.546341in}{0.941882in}}%
\pgfpathlineto{\pgfqpoint{1.548665in}{0.942491in}}%
\pgfpathlineto{\pgfqpoint{1.550989in}{0.967468in}}%
\pgfpathlineto{\pgfqpoint{1.553313in}{0.945320in}}%
\pgfpathlineto{\pgfqpoint{1.555637in}{0.959751in}}%
\pgfpathlineto{\pgfqpoint{1.557961in}{0.960039in}}%
\pgfpathlineto{\pgfqpoint{1.560285in}{0.924195in}}%
\pgfpathlineto{\pgfqpoint{1.562609in}{0.953825in}}%
\pgfpathlineto{\pgfqpoint{1.564933in}{0.942862in}}%
\pgfpathlineto{\pgfqpoint{1.567257in}{0.944656in}}%
\pgfpathlineto{\pgfqpoint{1.571905in}{0.922637in}}%
\pgfpathlineto{\pgfqpoint{1.574229in}{0.961750in}}%
\pgfpathlineto{\pgfqpoint{1.576553in}{0.947264in}}%
\pgfpathlineto{\pgfqpoint{1.578876in}{0.953734in}}%
\pgfpathlineto{\pgfqpoint{1.581200in}{0.917858in}}%
\pgfpathlineto{\pgfqpoint{1.583524in}{0.944787in}}%
\pgfpathlineto{\pgfqpoint{1.585848in}{0.950868in}}%
\pgfpathlineto{\pgfqpoint{1.588172in}{0.961824in}}%
\pgfpathlineto{\pgfqpoint{1.592820in}{0.934029in}}%
\pgfpathlineto{\pgfqpoint{1.595144in}{0.956159in}}%
\pgfpathlineto{\pgfqpoint{1.597468in}{0.958982in}}%
\pgfpathlineto{\pgfqpoint{1.602116in}{0.958926in}}%
\pgfpathlineto{\pgfqpoint{1.604440in}{0.949377in}}%
\pgfpathlineto{\pgfqpoint{1.606764in}{0.931261in}}%
\pgfpathlineto{\pgfqpoint{1.609088in}{0.947563in}}%
\pgfpathlineto{\pgfqpoint{1.611412in}{0.948351in}}%
\pgfpathlineto{\pgfqpoint{1.613736in}{0.934049in}}%
\pgfpathlineto{\pgfqpoint{1.618384in}{0.963488in}}%
\pgfpathlineto{\pgfqpoint{1.620708in}{0.945748in}}%
\pgfpathlineto{\pgfqpoint{1.623032in}{0.961821in}}%
\pgfpathlineto{\pgfqpoint{1.627680in}{0.956793in}}%
\pgfpathlineto{\pgfqpoint{1.630004in}{0.956951in}}%
\pgfpathlineto{\pgfqpoint{1.632328in}{0.954831in}}%
\pgfpathlineto{\pgfqpoint{1.634652in}{0.962966in}}%
\pgfpathlineto{\pgfqpoint{1.636976in}{0.957824in}}%
\pgfpathlineto{\pgfqpoint{1.641624in}{0.922407in}}%
\pgfpathlineto{\pgfqpoint{1.643948in}{0.932288in}}%
\pgfpathlineto{\pgfqpoint{1.646272in}{0.970109in}}%
\pgfpathlineto{\pgfqpoint{1.648595in}{0.964259in}}%
\pgfpathlineto{\pgfqpoint{1.650919in}{0.955313in}}%
\pgfpathlineto{\pgfqpoint{1.653243in}{0.968355in}}%
\pgfpathlineto{\pgfqpoint{1.655567in}{0.952689in}}%
\pgfpathlineto{\pgfqpoint{1.657891in}{0.959278in}}%
\pgfpathlineto{\pgfqpoint{1.660215in}{0.929666in}}%
\pgfpathlineto{\pgfqpoint{1.664863in}{0.958249in}}%
\pgfpathlineto{\pgfqpoint{1.667187in}{0.963009in}}%
\pgfpathlineto{\pgfqpoint{1.669511in}{0.931745in}}%
\pgfpathlineto{\pgfqpoint{1.671835in}{0.958646in}}%
\pgfpathlineto{\pgfqpoint{1.674159in}{0.931358in}}%
\pgfpathlineto{\pgfqpoint{1.676483in}{0.964373in}}%
\pgfpathlineto{\pgfqpoint{1.678807in}{0.959864in}}%
\pgfpathlineto{\pgfqpoint{1.681131in}{0.939998in}}%
\pgfpathlineto{\pgfqpoint{1.683455in}{0.972127in}}%
\pgfpathlineto{\pgfqpoint{1.685779in}{0.947768in}}%
\pgfpathlineto{\pgfqpoint{1.688103in}{0.952183in}}%
\pgfpathlineto{\pgfqpoint{1.695075in}{0.926174in}}%
\pgfpathlineto{\pgfqpoint{1.697399in}{0.961254in}}%
\pgfpathlineto{\pgfqpoint{1.702047in}{0.934777in}}%
\pgfpathlineto{\pgfqpoint{1.704371in}{0.955594in}}%
\pgfpathlineto{\pgfqpoint{1.709019in}{0.963459in}}%
\pgfpathlineto{\pgfqpoint{1.711343in}{0.946365in}}%
\pgfpathlineto{\pgfqpoint{1.713667in}{0.966887in}}%
\pgfpathlineto{\pgfqpoint{1.715991in}{0.939099in}}%
\pgfpathlineto{\pgfqpoint{1.718314in}{0.947261in}}%
\pgfpathlineto{\pgfqpoint{1.720638in}{0.978247in}}%
\pgfpathlineto{\pgfqpoint{1.722962in}{0.965616in}}%
\pgfpathlineto{\pgfqpoint{1.725286in}{0.933159in}}%
\pgfpathlineto{\pgfqpoint{1.727610in}{0.944352in}}%
\pgfpathlineto{\pgfqpoint{1.729934in}{0.946914in}}%
\pgfpathlineto{\pgfqpoint{1.732258in}{0.960873in}}%
\pgfpathlineto{\pgfqpoint{1.734582in}{0.964099in}}%
\pgfpathlineto{\pgfqpoint{1.736906in}{0.931043in}}%
\pgfpathlineto{\pgfqpoint{1.739230in}{0.946648in}}%
\pgfpathlineto{\pgfqpoint{1.741554in}{0.938874in}}%
\pgfpathlineto{\pgfqpoint{1.743878in}{0.945211in}}%
\pgfpathlineto{\pgfqpoint{1.746202in}{0.924851in}}%
\pgfpathlineto{\pgfqpoint{1.753174in}{0.962367in}}%
\pgfpathlineto{\pgfqpoint{1.755498in}{0.950229in}}%
\pgfpathlineto{\pgfqpoint{1.757822in}{0.958488in}}%
\pgfpathlineto{\pgfqpoint{1.760146in}{0.955995in}}%
\pgfpathlineto{\pgfqpoint{1.762470in}{0.968170in}}%
\pgfpathlineto{\pgfqpoint{1.764794in}{0.950595in}}%
\pgfpathlineto{\pgfqpoint{1.767118in}{0.952461in}}%
\pgfpathlineto{\pgfqpoint{1.769442in}{0.967759in}}%
\pgfpathlineto{\pgfqpoint{1.771766in}{0.968738in}}%
\pgfpathlineto{\pgfqpoint{1.774090in}{0.945670in}}%
\pgfpathlineto{\pgfqpoint{1.778738in}{0.941945in}}%
\pgfpathlineto{\pgfqpoint{1.781062in}{0.960115in}}%
\pgfpathlineto{\pgfqpoint{1.783386in}{0.938283in}}%
\pgfpathlineto{\pgfqpoint{1.785710in}{0.953250in}}%
\pgfpathlineto{\pgfqpoint{1.788034in}{0.916769in}}%
\pgfpathlineto{\pgfqpoint{1.790357in}{0.954406in}}%
\pgfpathlineto{\pgfqpoint{1.792681in}{0.969728in}}%
\pgfpathlineto{\pgfqpoint{1.795005in}{0.966752in}}%
\pgfpathlineto{\pgfqpoint{1.797329in}{0.946398in}}%
\pgfpathlineto{\pgfqpoint{1.799653in}{0.949101in}}%
\pgfpathlineto{\pgfqpoint{1.801977in}{0.945019in}}%
\pgfpathlineto{\pgfqpoint{1.804301in}{0.930071in}}%
\pgfpathlineto{\pgfqpoint{1.808949in}{0.938025in}}%
\pgfpathlineto{\pgfqpoint{1.811273in}{0.935780in}}%
\pgfpathlineto{\pgfqpoint{1.813597in}{1.237146in}}%
\pgfpathlineto{\pgfqpoint{1.815921in}{1.243042in}}%
\pgfpathlineto{\pgfqpoint{1.818245in}{1.226805in}}%
\pgfpathlineto{\pgfqpoint{1.820569in}{1.228323in}}%
\pgfpathlineto{\pgfqpoint{1.822893in}{1.233636in}}%
\pgfpathlineto{\pgfqpoint{1.825217in}{1.205017in}}%
\pgfpathlineto{\pgfqpoint{1.827541in}{1.235409in}}%
\pgfpathlineto{\pgfqpoint{1.829865in}{1.214574in}}%
\pgfpathlineto{\pgfqpoint{1.832189in}{1.253309in}}%
\pgfpathlineto{\pgfqpoint{1.834513in}{1.230684in}}%
\pgfpathlineto{\pgfqpoint{1.836837in}{1.231239in}}%
\pgfpathlineto{\pgfqpoint{1.839161in}{1.238636in}}%
\pgfpathlineto{\pgfqpoint{1.841485in}{1.228061in}}%
\pgfpathlineto{\pgfqpoint{1.843809in}{1.212094in}}%
\pgfpathlineto{\pgfqpoint{1.846133in}{1.231583in}}%
\pgfpathlineto{\pgfqpoint{1.848457in}{1.219425in}}%
\pgfpathlineto{\pgfqpoint{1.850781in}{1.237506in}}%
\pgfpathlineto{\pgfqpoint{1.853105in}{1.242728in}}%
\pgfpathlineto{\pgfqpoint{1.855429in}{1.239215in}}%
\pgfpathlineto{\pgfqpoint{1.857753in}{1.257612in}}%
\pgfpathlineto{\pgfqpoint{1.860076in}{1.218026in}}%
\pgfpathlineto{\pgfqpoint{1.862400in}{1.237508in}}%
\pgfpathlineto{\pgfqpoint{1.864724in}{1.241694in}}%
\pgfpathlineto{\pgfqpoint{1.867048in}{1.237666in}}%
\pgfpathlineto{\pgfqpoint{1.871696in}{1.204091in}}%
\pgfpathlineto{\pgfqpoint{1.874020in}{1.248410in}}%
\pgfpathlineto{\pgfqpoint{1.876344in}{1.214349in}}%
\pgfpathlineto{\pgfqpoint{1.880992in}{1.241543in}}%
\pgfpathlineto{\pgfqpoint{1.883316in}{1.226808in}}%
\pgfpathlineto{\pgfqpoint{1.885640in}{1.225615in}}%
\pgfpathlineto{\pgfqpoint{1.887964in}{1.241571in}}%
\pgfpathlineto{\pgfqpoint{1.890288in}{1.229646in}}%
\pgfpathlineto{\pgfqpoint{1.892612in}{1.226581in}}%
\pgfpathlineto{\pgfqpoint{1.894936in}{1.247033in}}%
\pgfpathlineto{\pgfqpoint{1.897260in}{1.255315in}}%
\pgfpathlineto{\pgfqpoint{1.901908in}{1.234378in}}%
\pgfpathlineto{\pgfqpoint{1.904232in}{1.228713in}}%
\pgfpathlineto{\pgfqpoint{1.906556in}{1.241045in}}%
\pgfpathlineto{\pgfqpoint{1.908880in}{1.239293in}}%
\pgfpathlineto{\pgfqpoint{1.911204in}{1.227613in}}%
\pgfpathlineto{\pgfqpoint{1.913528in}{1.268298in}}%
\pgfpathlineto{\pgfqpoint{1.915852in}{1.240374in}}%
\pgfpathlineto{\pgfqpoint{1.918176in}{1.235009in}}%
\pgfpathlineto{\pgfqpoint{1.920500in}{1.233452in}}%
\pgfpathlineto{\pgfqpoint{1.922824in}{1.254951in}}%
\pgfpathlineto{\pgfqpoint{1.925148in}{1.212587in}}%
\pgfpathlineto{\pgfqpoint{1.927472in}{1.228720in}}%
\pgfpathlineto{\pgfqpoint{1.929795in}{1.252749in}}%
\pgfpathlineto{\pgfqpoint{1.932119in}{1.232935in}}%
\pgfpathlineto{\pgfqpoint{1.936767in}{1.221609in}}%
\pgfpathlineto{\pgfqpoint{1.939091in}{1.210350in}}%
\pgfpathlineto{\pgfqpoint{1.943739in}{1.242959in}}%
\pgfpathlineto{\pgfqpoint{1.946063in}{1.232297in}}%
\pgfpathlineto{\pgfqpoint{1.948387in}{1.232333in}}%
\pgfpathlineto{\pgfqpoint{1.950711in}{1.247184in}}%
\pgfpathlineto{\pgfqpoint{1.955359in}{1.225369in}}%
\pgfpathlineto{\pgfqpoint{1.960007in}{1.244764in}}%
\pgfpathlineto{\pgfqpoint{1.962331in}{1.250495in}}%
\pgfpathlineto{\pgfqpoint{1.964655in}{1.236730in}}%
\pgfpathlineto{\pgfqpoint{1.966979in}{1.234753in}}%
\pgfpathlineto{\pgfqpoint{1.969303in}{1.220302in}}%
\pgfpathlineto{\pgfqpoint{1.971627in}{1.247060in}}%
\pgfpathlineto{\pgfqpoint{1.973951in}{1.242845in}}%
\pgfpathlineto{\pgfqpoint{1.976275in}{1.207675in}}%
\pgfpathlineto{\pgfqpoint{1.978599in}{1.241444in}}%
\pgfpathlineto{\pgfqpoint{1.980923in}{1.242326in}}%
\pgfpathlineto{\pgfqpoint{1.983247in}{1.227562in}}%
\pgfpathlineto{\pgfqpoint{1.985571in}{1.226990in}}%
\pgfpathlineto{\pgfqpoint{1.987895in}{1.230584in}}%
\pgfpathlineto{\pgfqpoint{1.992543in}{1.257852in}}%
\pgfpathlineto{\pgfqpoint{1.994867in}{1.211820in}}%
\pgfpathlineto{\pgfqpoint{1.997191in}{1.252066in}}%
\pgfpathlineto{\pgfqpoint{1.999514in}{1.226654in}}%
\pgfpathlineto{\pgfqpoint{2.001838in}{1.226557in}}%
\pgfpathlineto{\pgfqpoint{2.004162in}{1.213382in}}%
\pgfpathlineto{\pgfqpoint{2.006486in}{1.260298in}}%
\pgfpathlineto{\pgfqpoint{2.008810in}{1.233998in}}%
\pgfpathlineto{\pgfqpoint{2.011134in}{1.224487in}}%
\pgfpathlineto{\pgfqpoint{2.013458in}{1.221947in}}%
\pgfpathlineto{\pgfqpoint{2.015782in}{1.241044in}}%
\pgfpathlineto{\pgfqpoint{2.018106in}{1.244694in}}%
\pgfpathlineto{\pgfqpoint{2.020430in}{1.237792in}}%
\pgfpathlineto{\pgfqpoint{2.022754in}{1.260621in}}%
\pgfpathlineto{\pgfqpoint{2.025078in}{1.241799in}}%
\pgfpathlineto{\pgfqpoint{2.027402in}{1.206057in}}%
\pgfpathlineto{\pgfqpoint{2.029726in}{1.218197in}}%
\pgfpathlineto{\pgfqpoint{2.032050in}{1.244844in}}%
\pgfpathlineto{\pgfqpoint{2.034374in}{1.235779in}}%
\pgfpathlineto{\pgfqpoint{2.039022in}{1.274305in}}%
\pgfpathlineto{\pgfqpoint{2.041346in}{1.247181in}}%
\pgfpathlineto{\pgfqpoint{2.045994in}{1.223286in}}%
\pgfpathlineto{\pgfqpoint{2.048318in}{1.222218in}}%
\pgfpathlineto{\pgfqpoint{2.050642in}{1.219709in}}%
\pgfpathlineto{\pgfqpoint{2.052966in}{1.225060in}}%
\pgfpathlineto{\pgfqpoint{2.055290in}{1.216906in}}%
\pgfpathlineto{\pgfqpoint{2.057614in}{1.221072in}}%
\pgfpathlineto{\pgfqpoint{2.059938in}{1.227945in}}%
\pgfpathlineto{\pgfqpoint{2.062262in}{1.205956in}}%
\pgfpathlineto{\pgfqpoint{2.064586in}{1.227116in}}%
\pgfpathlineto{\pgfqpoint{2.069234in}{1.245697in}}%
\pgfpathlineto{\pgfqpoint{2.071557in}{1.205722in}}%
\pgfpathlineto{\pgfqpoint{2.073881in}{1.241350in}}%
\pgfpathlineto{\pgfqpoint{2.076205in}{1.233600in}}%
\pgfpathlineto{\pgfqpoint{2.078529in}{1.232721in}}%
\pgfpathlineto{\pgfqpoint{2.080853in}{1.233184in}}%
\pgfpathlineto{\pgfqpoint{2.083177in}{1.239864in}}%
\pgfpathlineto{\pgfqpoint{2.085501in}{1.227466in}}%
\pgfpathlineto{\pgfqpoint{2.087825in}{1.247951in}}%
\pgfpathlineto{\pgfqpoint{2.090149in}{1.225067in}}%
\pgfpathlineto{\pgfqpoint{2.092473in}{1.226817in}}%
\pgfpathlineto{\pgfqpoint{2.094797in}{1.255465in}}%
\pgfpathlineto{\pgfqpoint{2.097121in}{1.226941in}}%
\pgfpathlineto{\pgfqpoint{2.099445in}{1.238439in}}%
\pgfpathlineto{\pgfqpoint{2.101769in}{1.235191in}}%
\pgfpathlineto{\pgfqpoint{2.104093in}{0.905450in}}%
\pgfpathlineto{\pgfqpoint{2.111065in}{0.888984in}}%
\pgfpathlineto{\pgfqpoint{2.113389in}{0.893209in}}%
\pgfpathlineto{\pgfqpoint{2.115713in}{0.912868in}}%
\pgfpathlineto{\pgfqpoint{2.118037in}{0.891805in}}%
\pgfpathlineto{\pgfqpoint{2.120361in}{0.884690in}}%
\pgfpathlineto{\pgfqpoint{2.122685in}{0.917783in}}%
\pgfpathlineto{\pgfqpoint{2.125009in}{0.908045in}}%
\pgfpathlineto{\pgfqpoint{2.127333in}{0.926909in}}%
\pgfpathlineto{\pgfqpoint{2.129657in}{0.927508in}}%
\pgfpathlineto{\pgfqpoint{2.131981in}{0.882531in}}%
\pgfpathlineto{\pgfqpoint{2.134305in}{0.913844in}}%
\pgfpathlineto{\pgfqpoint{2.136629in}{0.914868in}}%
\pgfpathlineto{\pgfqpoint{2.138953in}{0.914522in}}%
\pgfpathlineto{\pgfqpoint{2.143600in}{0.905247in}}%
\pgfpathlineto{\pgfqpoint{2.145924in}{0.916861in}}%
\pgfpathlineto{\pgfqpoint{2.148248in}{0.909086in}}%
\pgfpathlineto{\pgfqpoint{2.150572in}{0.929875in}}%
\pgfpathlineto{\pgfqpoint{2.155220in}{0.900574in}}%
\pgfpathlineto{\pgfqpoint{2.157544in}{0.907222in}}%
\pgfpathlineto{\pgfqpoint{2.159868in}{0.879951in}}%
\pgfpathlineto{\pgfqpoint{2.162192in}{0.896916in}}%
\pgfpathlineto{\pgfqpoint{2.164516in}{0.901183in}}%
\pgfpathlineto{\pgfqpoint{2.166840in}{0.951521in}}%
\pgfpathlineto{\pgfqpoint{2.169164in}{0.917090in}}%
\pgfpathlineto{\pgfqpoint{2.176136in}{0.903259in}}%
\pgfpathlineto{\pgfqpoint{2.178460in}{0.934124in}}%
\pgfpathlineto{\pgfqpoint{2.180784in}{0.915547in}}%
\pgfpathlineto{\pgfqpoint{2.183108in}{0.931367in}}%
\pgfpathlineto{\pgfqpoint{2.185432in}{0.919219in}}%
\pgfpathlineto{\pgfqpoint{2.187756in}{0.896892in}}%
\pgfpathlineto{\pgfqpoint{2.190080in}{0.929355in}}%
\pgfpathlineto{\pgfqpoint{2.192404in}{0.908249in}}%
\pgfpathlineto{\pgfqpoint{2.194728in}{0.912565in}}%
\pgfpathlineto{\pgfqpoint{2.197052in}{0.921452in}}%
\pgfpathlineto{\pgfqpoint{2.199376in}{0.923839in}}%
\pgfpathlineto{\pgfqpoint{2.201700in}{0.934538in}}%
\pgfpathlineto{\pgfqpoint{2.204024in}{0.921088in}}%
\pgfpathlineto{\pgfqpoint{2.206348in}{0.921129in}}%
\pgfpathlineto{\pgfqpoint{2.208672in}{0.903290in}}%
\pgfpathlineto{\pgfqpoint{2.210995in}{0.909504in}}%
\pgfpathlineto{\pgfqpoint{2.215643in}{0.941044in}}%
\pgfpathlineto{\pgfqpoint{2.217967in}{0.900541in}}%
\pgfpathlineto{\pgfqpoint{2.220291in}{0.940878in}}%
\pgfpathlineto{\pgfqpoint{2.222615in}{0.928802in}}%
\pgfpathlineto{\pgfqpoint{2.224939in}{0.896579in}}%
\pgfpathlineto{\pgfqpoint{2.227263in}{0.913210in}}%
\pgfpathlineto{\pgfqpoint{2.229587in}{0.905689in}}%
\pgfpathlineto{\pgfqpoint{2.231911in}{0.929227in}}%
\pgfpathlineto{\pgfqpoint{2.234235in}{0.929746in}}%
\pgfpathlineto{\pgfqpoint{2.236559in}{0.906504in}}%
\pgfpathlineto{\pgfqpoint{2.238883in}{0.933989in}}%
\pgfpathlineto{\pgfqpoint{2.241207in}{0.912888in}}%
\pgfpathlineto{\pgfqpoint{2.243531in}{0.917323in}}%
\pgfpathlineto{\pgfqpoint{2.245855in}{0.925245in}}%
\pgfpathlineto{\pgfqpoint{2.248179in}{0.900054in}}%
\pgfpathlineto{\pgfqpoint{2.250503in}{0.890470in}}%
\pgfpathlineto{\pgfqpoint{2.252827in}{0.921650in}}%
\pgfpathlineto{\pgfqpoint{2.255151in}{0.915298in}}%
\pgfpathlineto{\pgfqpoint{2.257475in}{0.941321in}}%
\pgfpathlineto{\pgfqpoint{2.264447in}{0.903126in}}%
\pgfpathlineto{\pgfqpoint{2.266771in}{0.934803in}}%
\pgfpathlineto{\pgfqpoint{2.269095in}{0.929977in}}%
\pgfpathlineto{\pgfqpoint{2.271419in}{0.912836in}}%
\pgfpathlineto{\pgfqpoint{2.273743in}{0.937574in}}%
\pgfpathlineto{\pgfqpoint{2.278391in}{0.924844in}}%
\pgfpathlineto{\pgfqpoint{2.280715in}{0.933222in}}%
\pgfpathlineto{\pgfqpoint{2.283038in}{0.904734in}}%
\pgfpathlineto{\pgfqpoint{2.285362in}{0.919486in}}%
\pgfpathlineto{\pgfqpoint{2.287686in}{0.921513in}}%
\pgfpathlineto{\pgfqpoint{2.290010in}{0.904931in}}%
\pgfpathlineto{\pgfqpoint{2.292334in}{0.940827in}}%
\pgfpathlineto{\pgfqpoint{2.294658in}{0.884437in}}%
\pgfpathlineto{\pgfqpoint{2.296982in}{0.900727in}}%
\pgfpathlineto{\pgfqpoint{2.299306in}{0.907540in}}%
\pgfpathlineto{\pgfqpoint{2.301630in}{0.920414in}}%
\pgfpathlineto{\pgfqpoint{2.303954in}{0.916340in}}%
\pgfpathlineto{\pgfqpoint{2.306278in}{0.928581in}}%
\pgfpathlineto{\pgfqpoint{2.310926in}{0.896031in}}%
\pgfpathlineto{\pgfqpoint{2.313250in}{0.922759in}}%
\pgfpathlineto{\pgfqpoint{2.315574in}{0.911728in}}%
\pgfpathlineto{\pgfqpoint{2.317898in}{0.937680in}}%
\pgfpathlineto{\pgfqpoint{2.320222in}{0.896999in}}%
\pgfpathlineto{\pgfqpoint{2.322546in}{0.927323in}}%
\pgfpathlineto{\pgfqpoint{2.324870in}{0.902481in}}%
\pgfpathlineto{\pgfqpoint{2.327194in}{0.924042in}}%
\pgfpathlineto{\pgfqpoint{2.329518in}{0.920958in}}%
\pgfpathlineto{\pgfqpoint{2.334166in}{0.919285in}}%
\pgfpathlineto{\pgfqpoint{2.336490in}{0.905023in}}%
\pgfpathlineto{\pgfqpoint{2.338814in}{0.935187in}}%
\pgfpathlineto{\pgfqpoint{2.341138in}{0.925752in}}%
\pgfpathlineto{\pgfqpoint{2.343462in}{0.926629in}}%
\pgfpathlineto{\pgfqpoint{2.345786in}{0.935818in}}%
\pgfpathlineto{\pgfqpoint{2.348110in}{0.938543in}}%
\pgfpathlineto{\pgfqpoint{2.350434in}{0.931515in}}%
\pgfpathlineto{\pgfqpoint{2.352757in}{0.930967in}}%
\pgfpathlineto{\pgfqpoint{2.355081in}{0.920075in}}%
\pgfpathlineto{\pgfqpoint{2.357405in}{0.928196in}}%
\pgfpathlineto{\pgfqpoint{2.359729in}{0.927726in}}%
\pgfpathlineto{\pgfqpoint{2.362053in}{0.937953in}}%
\pgfpathlineto{\pgfqpoint{2.364377in}{0.930099in}}%
\pgfpathlineto{\pgfqpoint{2.366701in}{0.917484in}}%
\pgfpathlineto{\pgfqpoint{2.369025in}{0.946591in}}%
\pgfpathlineto{\pgfqpoint{2.371349in}{0.944926in}}%
\pgfpathlineto{\pgfqpoint{2.373673in}{0.930761in}}%
\pgfpathlineto{\pgfqpoint{2.375997in}{0.926192in}}%
\pgfpathlineto{\pgfqpoint{2.378321in}{0.906990in}}%
\pgfpathlineto{\pgfqpoint{2.380645in}{0.929285in}}%
\pgfpathlineto{\pgfqpoint{2.382969in}{0.934298in}}%
\pgfpathlineto{\pgfqpoint{2.387617in}{0.929175in}}%
\pgfpathlineto{\pgfqpoint{2.389941in}{0.932621in}}%
\pgfpathlineto{\pgfqpoint{2.392265in}{0.920016in}}%
\pgfpathlineto{\pgfqpoint{2.394589in}{1.214794in}}%
\pgfpathlineto{\pgfqpoint{2.396913in}{1.206629in}}%
\pgfpathlineto{\pgfqpoint{2.399237in}{1.185676in}}%
\pgfpathlineto{\pgfqpoint{2.401561in}{1.231880in}}%
\pgfpathlineto{\pgfqpoint{2.403885in}{1.201387in}}%
\pgfpathlineto{\pgfqpoint{2.406209in}{1.216168in}}%
\pgfpathlineto{\pgfqpoint{2.408533in}{1.215894in}}%
\pgfpathlineto{\pgfqpoint{2.410857in}{1.206587in}}%
\pgfpathlineto{\pgfqpoint{2.413181in}{1.214795in}}%
\pgfpathlineto{\pgfqpoint{2.415505in}{1.207894in}}%
\pgfpathlineto{\pgfqpoint{2.417829in}{1.216842in}}%
\pgfpathlineto{\pgfqpoint{2.420153in}{1.192781in}}%
\pgfpathlineto{\pgfqpoint{2.422476in}{1.212646in}}%
\pgfpathlineto{\pgfqpoint{2.424800in}{1.194911in}}%
\pgfpathlineto{\pgfqpoint{2.429448in}{1.214312in}}%
\pgfpathlineto{\pgfqpoint{2.431772in}{1.195259in}}%
\pgfpathlineto{\pgfqpoint{2.434096in}{1.202794in}}%
\pgfpathlineto{\pgfqpoint{2.436420in}{1.231966in}}%
\pgfpathlineto{\pgfqpoint{2.438744in}{1.205005in}}%
\pgfpathlineto{\pgfqpoint{2.441068in}{1.227996in}}%
\pgfpathlineto{\pgfqpoint{2.443392in}{1.202264in}}%
\pgfpathlineto{\pgfqpoint{2.445716in}{1.223388in}}%
\pgfpathlineto{\pgfqpoint{2.448040in}{1.216411in}}%
\pgfpathlineto{\pgfqpoint{2.452688in}{1.242734in}}%
\pgfpathlineto{\pgfqpoint{2.455012in}{1.211967in}}%
\pgfpathlineto{\pgfqpoint{2.457336in}{1.201550in}}%
\pgfpathlineto{\pgfqpoint{2.459660in}{1.197637in}}%
\pgfpathlineto{\pgfqpoint{2.461984in}{1.206051in}}%
\pgfpathlineto{\pgfqpoint{2.464308in}{1.198986in}}%
\pgfpathlineto{\pgfqpoint{2.466632in}{1.199423in}}%
\pgfpathlineto{\pgfqpoint{2.468956in}{1.211527in}}%
\pgfpathlineto{\pgfqpoint{2.471280in}{1.206174in}}%
\pgfpathlineto{\pgfqpoint{2.473604in}{1.232524in}}%
\pgfpathlineto{\pgfqpoint{2.475928in}{1.218914in}}%
\pgfpathlineto{\pgfqpoint{2.478252in}{1.236756in}}%
\pgfpathlineto{\pgfqpoint{2.480576in}{1.237275in}}%
\pgfpathlineto{\pgfqpoint{2.482900in}{1.232662in}}%
\pgfpathlineto{\pgfqpoint{2.485224in}{1.218634in}}%
\pgfpathlineto{\pgfqpoint{2.487548in}{1.220923in}}%
\pgfpathlineto{\pgfqpoint{2.489872in}{1.236038in}}%
\pgfpathlineto{\pgfqpoint{2.492195in}{1.230229in}}%
\pgfpathlineto{\pgfqpoint{2.494519in}{1.229541in}}%
\pgfpathlineto{\pgfqpoint{2.496843in}{1.208448in}}%
\pgfpathlineto{\pgfqpoint{2.499167in}{1.210586in}}%
\pgfpathlineto{\pgfqpoint{2.501491in}{1.231916in}}%
\pgfpathlineto{\pgfqpoint{2.503815in}{1.267772in}}%
\pgfpathlineto{\pgfqpoint{2.506139in}{1.209601in}}%
\pgfpathlineto{\pgfqpoint{2.508463in}{1.226078in}}%
\pgfpathlineto{\pgfqpoint{2.510787in}{1.205536in}}%
\pgfpathlineto{\pgfqpoint{2.513111in}{1.212227in}}%
\pgfpathlineto{\pgfqpoint{2.515435in}{1.204437in}}%
\pgfpathlineto{\pgfqpoint{2.517759in}{1.232083in}}%
\pgfpathlineto{\pgfqpoint{2.520083in}{1.219506in}}%
\pgfpathlineto{\pgfqpoint{2.522407in}{1.232233in}}%
\pgfpathlineto{\pgfqpoint{2.524731in}{1.223141in}}%
\pgfpathlineto{\pgfqpoint{2.527055in}{1.250317in}}%
\pgfpathlineto{\pgfqpoint{2.529379in}{1.216781in}}%
\pgfpathlineto{\pgfqpoint{2.534027in}{1.232839in}}%
\pgfpathlineto{\pgfqpoint{2.536351in}{1.186587in}}%
\pgfpathlineto{\pgfqpoint{2.538675in}{1.250633in}}%
\pgfpathlineto{\pgfqpoint{2.540999in}{1.239600in}}%
\pgfpathlineto{\pgfqpoint{2.543323in}{1.209513in}}%
\pgfpathlineto{\pgfqpoint{2.545647in}{1.223191in}}%
\pgfpathlineto{\pgfqpoint{2.547971in}{1.221697in}}%
\pgfpathlineto{\pgfqpoint{2.550295in}{1.204421in}}%
\pgfpathlineto{\pgfqpoint{2.552619in}{1.228437in}}%
\pgfpathlineto{\pgfqpoint{2.557267in}{1.206011in}}%
\pgfpathlineto{\pgfqpoint{2.559591in}{1.206231in}}%
\pgfpathlineto{\pgfqpoint{2.561915in}{1.230303in}}%
\pgfpathlineto{\pgfqpoint{2.564238in}{1.221644in}}%
\pgfpathlineto{\pgfqpoint{2.566562in}{1.207572in}}%
\pgfpathlineto{\pgfqpoint{2.568886in}{1.225259in}}%
\pgfpathlineto{\pgfqpoint{2.571210in}{1.208386in}}%
\pgfpathlineto{\pgfqpoint{2.573534in}{1.238852in}}%
\pgfpathlineto{\pgfqpoint{2.575858in}{1.213926in}}%
\pgfpathlineto{\pgfqpoint{2.578182in}{1.267256in}}%
\pgfpathlineto{\pgfqpoint{2.580506in}{1.209548in}}%
\pgfpathlineto{\pgfqpoint{2.582830in}{1.213597in}}%
\pgfpathlineto{\pgfqpoint{2.585154in}{1.250608in}}%
\pgfpathlineto{\pgfqpoint{2.592126in}{1.204359in}}%
\pgfpathlineto{\pgfqpoint{2.594450in}{1.239573in}}%
\pgfpathlineto{\pgfqpoint{2.596774in}{1.244219in}}%
\pgfpathlineto{\pgfqpoint{2.599098in}{1.225323in}}%
\pgfpathlineto{\pgfqpoint{2.601422in}{1.265412in}}%
\pgfpathlineto{\pgfqpoint{2.603746in}{1.219231in}}%
\pgfpathlineto{\pgfqpoint{2.606070in}{1.252934in}}%
\pgfpathlineto{\pgfqpoint{2.608394in}{1.217204in}}%
\pgfpathlineto{\pgfqpoint{2.610718in}{1.245853in}}%
\pgfpathlineto{\pgfqpoint{2.613042in}{1.253808in}}%
\pgfpathlineto{\pgfqpoint{2.615366in}{1.243128in}}%
\pgfpathlineto{\pgfqpoint{2.617690in}{1.253389in}}%
\pgfpathlineto{\pgfqpoint{2.620014in}{1.238701in}}%
\pgfpathlineto{\pgfqpoint{2.622338in}{1.207963in}}%
\pgfpathlineto{\pgfqpoint{2.624662in}{1.225224in}}%
\pgfpathlineto{\pgfqpoint{2.626986in}{1.258315in}}%
\pgfpathlineto{\pgfqpoint{2.629310in}{1.264452in}}%
\pgfpathlineto{\pgfqpoint{2.631634in}{1.247610in}}%
\pgfpathlineto{\pgfqpoint{2.633957in}{1.243294in}}%
\pgfpathlineto{\pgfqpoint{2.636281in}{1.233388in}}%
\pgfpathlineto{\pgfqpoint{2.638605in}{1.235976in}}%
\pgfpathlineto{\pgfqpoint{2.640929in}{1.220820in}}%
\pgfpathlineto{\pgfqpoint{2.643253in}{1.258660in}}%
\pgfpathlineto{\pgfqpoint{2.647901in}{1.231839in}}%
\pgfpathlineto{\pgfqpoint{2.650225in}{1.256737in}}%
\pgfpathlineto{\pgfqpoint{2.652549in}{1.254323in}}%
\pgfpathlineto{\pgfqpoint{2.654873in}{1.248042in}}%
\pgfpathlineto{\pgfqpoint{2.657197in}{1.210392in}}%
\pgfpathlineto{\pgfqpoint{2.659521in}{1.244256in}}%
\pgfpathlineto{\pgfqpoint{2.661845in}{1.234478in}}%
\pgfpathlineto{\pgfqpoint{2.664169in}{1.233417in}}%
\pgfpathlineto{\pgfqpoint{2.666493in}{1.255585in}}%
\pgfpathlineto{\pgfqpoint{2.668817in}{1.252473in}}%
\pgfpathlineto{\pgfqpoint{2.671141in}{1.263236in}}%
\pgfpathlineto{\pgfqpoint{2.673465in}{1.242995in}}%
\pgfpathlineto{\pgfqpoint{2.675789in}{1.268305in}}%
\pgfpathlineto{\pgfqpoint{2.678113in}{1.238113in}}%
\pgfpathlineto{\pgfqpoint{2.680437in}{1.242172in}}%
\pgfpathlineto{\pgfqpoint{2.682761in}{1.254027in}}%
\pgfpathlineto{\pgfqpoint{2.685085in}{0.927657in}}%
\pgfpathlineto{\pgfqpoint{2.687409in}{0.929500in}}%
\pgfpathlineto{\pgfqpoint{2.689733in}{0.914712in}}%
\pgfpathlineto{\pgfqpoint{2.694381in}{0.945718in}}%
\pgfpathlineto{\pgfqpoint{2.696705in}{0.939588in}}%
\pgfpathlineto{\pgfqpoint{2.701353in}{0.910752in}}%
\pgfpathlineto{\pgfqpoint{2.706000in}{0.935887in}}%
\pgfpathlineto{\pgfqpoint{2.708324in}{0.940436in}}%
\pgfpathlineto{\pgfqpoint{2.715296in}{0.911721in}}%
\pgfpathlineto{\pgfqpoint{2.717620in}{0.891030in}}%
\pgfpathlineto{\pgfqpoint{2.719944in}{0.936557in}}%
\pgfpathlineto{\pgfqpoint{2.722268in}{0.925125in}}%
\pgfpathlineto{\pgfqpoint{2.729240in}{0.941874in}}%
\pgfpathlineto{\pgfqpoint{2.731564in}{0.933089in}}%
\pgfpathlineto{\pgfqpoint{2.733888in}{0.907781in}}%
\pgfpathlineto{\pgfqpoint{2.738536in}{0.929905in}}%
\pgfpathlineto{\pgfqpoint{2.740860in}{0.917805in}}%
\pgfpathlineto{\pgfqpoint{2.743184in}{0.952002in}}%
\pgfpathlineto{\pgfqpoint{2.745508in}{0.937421in}}%
\pgfpathlineto{\pgfqpoint{2.747832in}{0.948927in}}%
\pgfpathlineto{\pgfqpoint{2.752480in}{0.928658in}}%
\pgfpathlineto{\pgfqpoint{2.754804in}{0.953675in}}%
\pgfpathlineto{\pgfqpoint{2.759452in}{0.909479in}}%
\pgfpathlineto{\pgfqpoint{2.761776in}{0.947659in}}%
\pgfpathlineto{\pgfqpoint{2.764100in}{0.957084in}}%
\pgfpathlineto{\pgfqpoint{2.766424in}{0.922328in}}%
\pgfpathlineto{\pgfqpoint{2.768748in}{0.968202in}}%
\pgfpathlineto{\pgfqpoint{2.775719in}{0.913617in}}%
\pgfpathlineto{\pgfqpoint{2.778043in}{0.964065in}}%
\pgfpathlineto{\pgfqpoint{2.782691in}{0.929067in}}%
\pgfpathlineto{\pgfqpoint{2.785015in}{0.917989in}}%
\pgfpathlineto{\pgfqpoint{2.787339in}{0.936688in}}%
\pgfpathlineto{\pgfqpoint{2.789663in}{0.934869in}}%
\pgfpathlineto{\pgfqpoint{2.791987in}{0.939555in}}%
\pgfpathlineto{\pgfqpoint{2.796635in}{0.944246in}}%
\pgfpathlineto{\pgfqpoint{2.805931in}{0.929935in}}%
\pgfpathlineto{\pgfqpoint{2.808255in}{0.941829in}}%
\pgfpathlineto{\pgfqpoint{2.810579in}{0.942431in}}%
\pgfpathlineto{\pgfqpoint{2.812903in}{0.930858in}}%
\pgfpathlineto{\pgfqpoint{2.815227in}{0.932473in}}%
\pgfpathlineto{\pgfqpoint{2.817551in}{0.944337in}}%
\pgfpathlineto{\pgfqpoint{2.819875in}{0.947438in}}%
\pgfpathlineto{\pgfqpoint{2.822199in}{0.939810in}}%
\pgfpathlineto{\pgfqpoint{2.826847in}{0.969420in}}%
\pgfpathlineto{\pgfqpoint{2.829171in}{0.929023in}}%
\pgfpathlineto{\pgfqpoint{2.831495in}{0.954034in}}%
\pgfpathlineto{\pgfqpoint{2.833819in}{0.950834in}}%
\pgfpathlineto{\pgfqpoint{2.836143in}{0.950937in}}%
\pgfpathlineto{\pgfqpoint{2.838467in}{0.919821in}}%
\pgfpathlineto{\pgfqpoint{2.840791in}{0.970498in}}%
\pgfpathlineto{\pgfqpoint{2.843115in}{0.918381in}}%
\pgfpathlineto{\pgfqpoint{2.845438in}{0.942596in}}%
\pgfpathlineto{\pgfqpoint{2.847762in}{0.938277in}}%
\pgfpathlineto{\pgfqpoint{2.850086in}{0.974687in}}%
\pgfpathlineto{\pgfqpoint{2.852410in}{0.963327in}}%
\pgfpathlineto{\pgfqpoint{2.854734in}{0.944979in}}%
\pgfpathlineto{\pgfqpoint{2.857058in}{0.961136in}}%
\pgfpathlineto{\pgfqpoint{2.859382in}{0.954269in}}%
\pgfpathlineto{\pgfqpoint{2.861706in}{0.937701in}}%
\pgfpathlineto{\pgfqpoint{2.864030in}{0.963030in}}%
\pgfpathlineto{\pgfqpoint{2.866354in}{0.933558in}}%
\pgfpathlineto{\pgfqpoint{2.868678in}{0.939704in}}%
\pgfpathlineto{\pgfqpoint{2.871002in}{0.923754in}}%
\pgfpathlineto{\pgfqpoint{2.873326in}{0.937066in}}%
\pgfpathlineto{\pgfqpoint{2.875650in}{0.943462in}}%
\pgfpathlineto{\pgfqpoint{2.877974in}{0.935576in}}%
\pgfpathlineto{\pgfqpoint{2.880298in}{0.954491in}}%
\pgfpathlineto{\pgfqpoint{2.882622in}{0.982881in}}%
\pgfpathlineto{\pgfqpoint{2.884946in}{0.947693in}}%
\pgfpathlineto{\pgfqpoint{2.887270in}{0.941503in}}%
\pgfpathlineto{\pgfqpoint{2.891918in}{0.971412in}}%
\pgfpathlineto{\pgfqpoint{2.894242in}{0.934495in}}%
\pgfpathlineto{\pgfqpoint{2.898890in}{0.954436in}}%
\pgfpathlineto{\pgfqpoint{2.901214in}{0.949602in}}%
\pgfpathlineto{\pgfqpoint{2.903538in}{0.959982in}}%
\pgfpathlineto{\pgfqpoint{2.905862in}{0.940760in}}%
\pgfpathlineto{\pgfqpoint{2.908186in}{0.934668in}}%
\pgfpathlineto{\pgfqpoint{2.910510in}{0.948611in}}%
\pgfpathlineto{\pgfqpoint{2.912834in}{0.945814in}}%
\pgfpathlineto{\pgfqpoint{2.915157in}{0.957699in}}%
\pgfpathlineto{\pgfqpoint{2.917481in}{0.946658in}}%
\pgfpathlineto{\pgfqpoint{2.922129in}{0.959948in}}%
\pgfpathlineto{\pgfqpoint{2.926777in}{0.963376in}}%
\pgfpathlineto{\pgfqpoint{2.929101in}{0.957948in}}%
\pgfpathlineto{\pgfqpoint{2.931425in}{0.940351in}}%
\pgfpathlineto{\pgfqpoint{2.933749in}{0.986540in}}%
\pgfpathlineto{\pgfqpoint{2.936073in}{0.957671in}}%
\pgfpathlineto{\pgfqpoint{2.938397in}{0.957000in}}%
\pgfpathlineto{\pgfqpoint{2.940721in}{0.946760in}}%
\pgfpathlineto{\pgfqpoint{2.943045in}{0.961405in}}%
\pgfpathlineto{\pgfqpoint{2.945369in}{0.958644in}}%
\pgfpathlineto{\pgfqpoint{2.947693in}{0.993586in}}%
\pgfpathlineto{\pgfqpoint{2.952341in}{0.953601in}}%
\pgfpathlineto{\pgfqpoint{2.954665in}{0.947030in}}%
\pgfpathlineto{\pgfqpoint{2.956989in}{0.991627in}}%
\pgfpathlineto{\pgfqpoint{2.959313in}{0.971318in}}%
\pgfpathlineto{\pgfqpoint{2.961637in}{0.975854in}}%
\pgfpathlineto{\pgfqpoint{2.963961in}{0.974216in}}%
\pgfpathlineto{\pgfqpoint{2.968609in}{0.983399in}}%
\pgfpathlineto{\pgfqpoint{2.970933in}{0.936584in}}%
\pgfpathlineto{\pgfqpoint{2.973257in}{0.945168in}}%
\pgfpathlineto{\pgfqpoint{2.975581in}{1.232622in}}%
\pgfpathlineto{\pgfqpoint{2.977905in}{1.249401in}}%
\pgfpathlineto{\pgfqpoint{2.980229in}{1.228231in}}%
\pgfpathlineto{\pgfqpoint{2.982553in}{1.241429in}}%
\pgfpathlineto{\pgfqpoint{2.984876in}{1.267887in}}%
\pgfpathlineto{\pgfqpoint{2.987200in}{1.240876in}}%
\pgfpathlineto{\pgfqpoint{2.989524in}{1.240977in}}%
\pgfpathlineto{\pgfqpoint{2.991848in}{1.273257in}}%
\pgfpathlineto{\pgfqpoint{2.994172in}{1.237801in}}%
\pgfpathlineto{\pgfqpoint{2.996496in}{1.268515in}}%
\pgfpathlineto{\pgfqpoint{2.998820in}{1.244324in}}%
\pgfpathlineto{\pgfqpoint{3.001144in}{1.251623in}}%
\pgfpathlineto{\pgfqpoint{3.003468in}{1.244824in}}%
\pgfpathlineto{\pgfqpoint{3.005792in}{1.244662in}}%
\pgfpathlineto{\pgfqpoint{3.008116in}{1.218823in}}%
\pgfpathlineto{\pgfqpoint{3.010440in}{1.252927in}}%
\pgfpathlineto{\pgfqpoint{3.012764in}{1.261050in}}%
\pgfpathlineto{\pgfqpoint{3.015088in}{1.260207in}}%
\pgfpathlineto{\pgfqpoint{3.017412in}{1.253543in}}%
\pgfpathlineto{\pgfqpoint{3.019736in}{1.256088in}}%
\pgfpathlineto{\pgfqpoint{3.022060in}{1.248793in}}%
\pgfpathlineto{\pgfqpoint{3.024384in}{1.252079in}}%
\pgfpathlineto{\pgfqpoint{3.026708in}{1.242957in}}%
\pgfpathlineto{\pgfqpoint{3.029032in}{1.240953in}}%
\pgfpathlineto{\pgfqpoint{3.033680in}{1.263667in}}%
\pgfpathlineto{\pgfqpoint{3.036004in}{1.246475in}}%
\pgfpathlineto{\pgfqpoint{3.038328in}{1.240297in}}%
\pgfpathlineto{\pgfqpoint{3.042976in}{1.269654in}}%
\pgfpathlineto{\pgfqpoint{3.045300in}{1.254672in}}%
\pgfpathlineto{\pgfqpoint{3.047624in}{1.270517in}}%
\pgfpathlineto{\pgfqpoint{3.049948in}{1.260650in}}%
\pgfpathlineto{\pgfqpoint{3.054596in}{1.216058in}}%
\pgfpathlineto{\pgfqpoint{3.056919in}{1.265620in}}%
\pgfpathlineto{\pgfqpoint{3.059243in}{1.229535in}}%
\pgfpathlineto{\pgfqpoint{3.061567in}{1.236325in}}%
\pgfpathlineto{\pgfqpoint{3.063891in}{1.265369in}}%
\pgfpathlineto{\pgfqpoint{3.066215in}{1.238959in}}%
\pgfpathlineto{\pgfqpoint{3.068539in}{1.271018in}}%
\pgfpathlineto{\pgfqpoint{3.070863in}{1.261999in}}%
\pgfpathlineto{\pgfqpoint{3.073187in}{1.260478in}}%
\pgfpathlineto{\pgfqpoint{3.077835in}{1.228345in}}%
\pgfpathlineto{\pgfqpoint{3.080159in}{1.263264in}}%
\pgfpathlineto{\pgfqpoint{3.082483in}{1.269517in}}%
\pgfpathlineto{\pgfqpoint{3.084807in}{1.265770in}}%
\pgfpathlineto{\pgfqpoint{3.087131in}{1.270724in}}%
\pgfpathlineto{\pgfqpoint{3.089455in}{1.257814in}}%
\pgfpathlineto{\pgfqpoint{3.091779in}{1.237766in}}%
\pgfpathlineto{\pgfqpoint{3.094103in}{1.275422in}}%
\pgfpathlineto{\pgfqpoint{3.096427in}{1.255352in}}%
\pgfpathlineto{\pgfqpoint{3.098751in}{1.280240in}}%
\pgfpathlineto{\pgfqpoint{3.101075in}{1.240729in}}%
\pgfpathlineto{\pgfqpoint{3.103399in}{1.244430in}}%
\pgfpathlineto{\pgfqpoint{3.108047in}{1.277422in}}%
\pgfpathlineto{\pgfqpoint{3.110371in}{1.282553in}}%
\pgfpathlineto{\pgfqpoint{3.112695in}{1.252378in}}%
\pgfpathlineto{\pgfqpoint{3.115019in}{1.270139in}}%
\pgfpathlineto{\pgfqpoint{3.117343in}{1.252812in}}%
\pgfpathlineto{\pgfqpoint{3.121991in}{1.267612in}}%
\pgfpathlineto{\pgfqpoint{3.124315in}{1.258513in}}%
\pgfpathlineto{\pgfqpoint{3.126638in}{1.260360in}}%
\pgfpathlineto{\pgfqpoint{3.128962in}{1.263633in}}%
\pgfpathlineto{\pgfqpoint{3.131286in}{1.249577in}}%
\pgfpathlineto{\pgfqpoint{3.133610in}{1.255960in}}%
\pgfpathlineto{\pgfqpoint{3.135934in}{1.244771in}}%
\pgfpathlineto{\pgfqpoint{3.138258in}{1.280304in}}%
\pgfpathlineto{\pgfqpoint{3.140582in}{1.245957in}}%
\pgfpathlineto{\pgfqpoint{3.142906in}{1.274626in}}%
\pgfpathlineto{\pgfqpoint{3.145230in}{1.256381in}}%
\pgfpathlineto{\pgfqpoint{3.147554in}{1.271971in}}%
\pgfpathlineto{\pgfqpoint{3.149878in}{1.267106in}}%
\pgfpathlineto{\pgfqpoint{3.152202in}{1.276706in}}%
\pgfpathlineto{\pgfqpoint{3.154526in}{1.280386in}}%
\pgfpathlineto{\pgfqpoint{3.156850in}{1.229201in}}%
\pgfpathlineto{\pgfqpoint{3.159174in}{1.241311in}}%
\pgfpathlineto{\pgfqpoint{3.161498in}{1.283009in}}%
\pgfpathlineto{\pgfqpoint{3.163822in}{1.257199in}}%
\pgfpathlineto{\pgfqpoint{3.166146in}{1.266135in}}%
\pgfpathlineto{\pgfqpoint{3.168470in}{1.255595in}}%
\pgfpathlineto{\pgfqpoint{3.170794in}{1.264796in}}%
\pgfpathlineto{\pgfqpoint{3.173118in}{1.252736in}}%
\pgfpathlineto{\pgfqpoint{3.175442in}{1.272034in}}%
\pgfpathlineto{\pgfqpoint{3.177766in}{1.250302in}}%
\pgfpathlineto{\pgfqpoint{3.180090in}{1.247184in}}%
\pgfpathlineto{\pgfqpoint{3.182414in}{1.254101in}}%
\pgfpathlineto{\pgfqpoint{3.187062in}{1.279192in}}%
\pgfpathlineto{\pgfqpoint{3.189386in}{1.267325in}}%
\pgfpathlineto{\pgfqpoint{3.191710in}{1.263259in}}%
\pgfpathlineto{\pgfqpoint{3.194034in}{1.275535in}}%
\pgfpathlineto{\pgfqpoint{3.196357in}{1.276724in}}%
\pgfpathlineto{\pgfqpoint{3.198681in}{1.284591in}}%
\pgfpathlineto{\pgfqpoint{3.201005in}{1.250522in}}%
\pgfpathlineto{\pgfqpoint{3.203329in}{1.289993in}}%
\pgfpathlineto{\pgfqpoint{3.205653in}{1.259408in}}%
\pgfpathlineto{\pgfqpoint{3.207977in}{1.270718in}}%
\pgfpathlineto{\pgfqpoint{3.210301in}{1.275306in}}%
\pgfpathlineto{\pgfqpoint{3.212625in}{1.275236in}}%
\pgfpathlineto{\pgfqpoint{3.214949in}{1.255584in}}%
\pgfpathlineto{\pgfqpoint{3.217273in}{1.255822in}}%
\pgfpathlineto{\pgfqpoint{3.219597in}{1.250046in}}%
\pgfpathlineto{\pgfqpoint{3.221921in}{1.289623in}}%
\pgfpathlineto{\pgfqpoint{3.224245in}{1.290281in}}%
\pgfpathlineto{\pgfqpoint{3.226569in}{1.265364in}}%
\pgfpathlineto{\pgfqpoint{3.228893in}{1.284452in}}%
\pgfpathlineto{\pgfqpoint{3.231217in}{1.256662in}}%
\pgfpathlineto{\pgfqpoint{3.233541in}{1.290029in}}%
\pgfpathlineto{\pgfqpoint{3.238189in}{1.253813in}}%
\pgfpathlineto{\pgfqpoint{3.240513in}{1.269954in}}%
\pgfpathlineto{\pgfqpoint{3.242837in}{1.270773in}}%
\pgfpathlineto{\pgfqpoint{3.245161in}{1.243943in}}%
\pgfpathlineto{\pgfqpoint{3.247485in}{1.282399in}}%
\pgfpathlineto{\pgfqpoint{3.249809in}{1.257819in}}%
\pgfpathlineto{\pgfqpoint{3.254457in}{1.270004in}}%
\pgfpathlineto{\pgfqpoint{3.256781in}{1.272101in}}%
\pgfpathlineto{\pgfqpoint{3.259105in}{1.268636in}}%
\pgfpathlineto{\pgfqpoint{3.261429in}{1.290858in}}%
\pgfpathlineto{\pgfqpoint{3.263753in}{1.270771in}}%
\pgfpathlineto{\pgfqpoint{3.266076in}{0.953008in}}%
\pgfpathlineto{\pgfqpoint{3.268400in}{0.950178in}}%
\pgfpathlineto{\pgfqpoint{3.270724in}{0.968935in}}%
\pgfpathlineto{\pgfqpoint{3.273048in}{0.953861in}}%
\pgfpathlineto{\pgfqpoint{3.275372in}{0.964709in}}%
\pgfpathlineto{\pgfqpoint{3.277696in}{0.946464in}}%
\pgfpathlineto{\pgfqpoint{3.280020in}{0.988410in}}%
\pgfpathlineto{\pgfqpoint{3.282344in}{0.949732in}}%
\pgfpathlineto{\pgfqpoint{3.284668in}{0.942826in}}%
\pgfpathlineto{\pgfqpoint{3.286992in}{0.943372in}}%
\pgfpathlineto{\pgfqpoint{3.289316in}{0.960402in}}%
\pgfpathlineto{\pgfqpoint{3.291640in}{0.940420in}}%
\pgfpathlineto{\pgfqpoint{3.293964in}{0.962188in}}%
\pgfpathlineto{\pgfqpoint{3.296288in}{0.919005in}}%
\pgfpathlineto{\pgfqpoint{3.298612in}{0.946778in}}%
\pgfpathlineto{\pgfqpoint{3.300936in}{0.918640in}}%
\pgfpathlineto{\pgfqpoint{3.303260in}{0.943107in}}%
\pgfpathlineto{\pgfqpoint{3.305584in}{0.941320in}}%
\pgfpathlineto{\pgfqpoint{3.307908in}{0.953648in}}%
\pgfpathlineto{\pgfqpoint{3.310232in}{0.945915in}}%
\pgfpathlineto{\pgfqpoint{3.312556in}{0.964042in}}%
\pgfpathlineto{\pgfqpoint{3.314880in}{0.931037in}}%
\pgfpathlineto{\pgfqpoint{3.317204in}{0.933361in}}%
\pgfpathlineto{\pgfqpoint{3.319528in}{0.941845in}}%
\pgfpathlineto{\pgfqpoint{3.321852in}{0.958073in}}%
\pgfpathlineto{\pgfqpoint{3.324176in}{0.921966in}}%
\pgfpathlineto{\pgfqpoint{3.328824in}{0.978697in}}%
\pgfpathlineto{\pgfqpoint{3.331148in}{0.946460in}}%
\pgfpathlineto{\pgfqpoint{3.333472in}{0.938441in}}%
\pgfpathlineto{\pgfqpoint{3.338119in}{0.967082in}}%
\pgfpathlineto{\pgfqpoint{3.340443in}{0.949354in}}%
\pgfpathlineto{\pgfqpoint{3.342767in}{0.955955in}}%
\pgfpathlineto{\pgfqpoint{3.345091in}{0.941521in}}%
\pgfpathlineto{\pgfqpoint{3.347415in}{0.959767in}}%
\pgfpathlineto{\pgfqpoint{3.349739in}{0.948751in}}%
\pgfpathlineto{\pgfqpoint{3.352063in}{0.965941in}}%
\pgfpathlineto{\pgfqpoint{3.354387in}{0.966922in}}%
\pgfpathlineto{\pgfqpoint{3.356711in}{0.928162in}}%
\pgfpathlineto{\pgfqpoint{3.359035in}{0.957106in}}%
\pgfpathlineto{\pgfqpoint{3.361359in}{0.961095in}}%
\pgfpathlineto{\pgfqpoint{3.363683in}{0.950803in}}%
\pgfpathlineto{\pgfqpoint{3.368331in}{0.953643in}}%
\pgfpathlineto{\pgfqpoint{3.370655in}{0.961025in}}%
\pgfpathlineto{\pgfqpoint{3.372979in}{0.957454in}}%
\pgfpathlineto{\pgfqpoint{3.375303in}{0.979384in}}%
\pgfpathlineto{\pgfqpoint{3.377627in}{0.952150in}}%
\pgfpathlineto{\pgfqpoint{3.379951in}{0.966076in}}%
\pgfpathlineto{\pgfqpoint{3.382275in}{0.935498in}}%
\pgfpathlineto{\pgfqpoint{3.384599in}{0.958922in}}%
\pgfpathlineto{\pgfqpoint{3.386923in}{0.943007in}}%
\pgfpathlineto{\pgfqpoint{3.389247in}{0.938321in}}%
\pgfpathlineto{\pgfqpoint{3.393895in}{0.969949in}}%
\pgfpathlineto{\pgfqpoint{3.400867in}{0.961661in}}%
\pgfpathlineto{\pgfqpoint{3.403191in}{0.963747in}}%
\pgfpathlineto{\pgfqpoint{3.405515in}{0.943409in}}%
\pgfpathlineto{\pgfqpoint{3.407838in}{0.962972in}}%
\pgfpathlineto{\pgfqpoint{3.410162in}{0.959019in}}%
\pgfpathlineto{\pgfqpoint{3.412486in}{0.963595in}}%
\pgfpathlineto{\pgfqpoint{3.414810in}{0.976903in}}%
\pgfpathlineto{\pgfqpoint{3.417134in}{0.956364in}}%
\pgfpathlineto{\pgfqpoint{3.421782in}{0.988315in}}%
\pgfpathlineto{\pgfqpoint{3.424106in}{0.957594in}}%
\pgfpathlineto{\pgfqpoint{3.426430in}{0.968650in}}%
\pgfpathlineto{\pgfqpoint{3.428754in}{0.935507in}}%
\pgfpathlineto{\pgfqpoint{3.433402in}{0.964302in}}%
\pgfpathlineto{\pgfqpoint{3.435726in}{0.953443in}}%
\pgfpathlineto{\pgfqpoint{3.438050in}{0.960218in}}%
\pgfpathlineto{\pgfqpoint{3.440374in}{0.946443in}}%
\pgfpathlineto{\pgfqpoint{3.442698in}{0.918114in}}%
\pgfpathlineto{\pgfqpoint{3.445022in}{0.931198in}}%
\pgfpathlineto{\pgfqpoint{3.449670in}{0.962720in}}%
\pgfpathlineto{\pgfqpoint{3.451994in}{0.947945in}}%
\pgfpathlineto{\pgfqpoint{3.454318in}{0.943349in}}%
\pgfpathlineto{\pgfqpoint{3.456642in}{0.984560in}}%
\pgfpathlineto{\pgfqpoint{3.461290in}{0.946843in}}%
\pgfpathlineto{\pgfqpoint{3.463614in}{0.957954in}}%
\pgfpathlineto{\pgfqpoint{3.465938in}{0.983480in}}%
\pgfpathlineto{\pgfqpoint{3.468262in}{0.980157in}}%
\pgfpathlineto{\pgfqpoint{3.470586in}{0.950238in}}%
\pgfpathlineto{\pgfqpoint{3.472910in}{0.958875in}}%
\pgfpathlineto{\pgfqpoint{3.475234in}{0.950667in}}%
\pgfpathlineto{\pgfqpoint{3.477557in}{0.956695in}}%
\pgfpathlineto{\pgfqpoint{3.479881in}{0.928553in}}%
\pgfpathlineto{\pgfqpoint{3.482205in}{0.961724in}}%
\pgfpathlineto{\pgfqpoint{3.484529in}{0.967762in}}%
\pgfpathlineto{\pgfqpoint{3.486853in}{0.961335in}}%
\pgfpathlineto{\pgfqpoint{3.489177in}{0.971114in}}%
\pgfpathlineto{\pgfqpoint{3.491501in}{0.953554in}}%
\pgfpathlineto{\pgfqpoint{3.493825in}{0.953042in}}%
\pgfpathlineto{\pgfqpoint{3.496149in}{0.974843in}}%
\pgfpathlineto{\pgfqpoint{3.498473in}{0.975994in}}%
\pgfpathlineto{\pgfqpoint{3.500797in}{0.972658in}}%
\pgfpathlineto{\pgfqpoint{3.503121in}{0.957498in}}%
\pgfpathlineto{\pgfqpoint{3.505445in}{0.950856in}}%
\pgfpathlineto{\pgfqpoint{3.507769in}{0.954271in}}%
\pgfpathlineto{\pgfqpoint{3.510093in}{0.974658in}}%
\pgfpathlineto{\pgfqpoint{3.514741in}{0.943266in}}%
\pgfpathlineto{\pgfqpoint{3.517065in}{0.961831in}}%
\pgfpathlineto{\pgfqpoint{3.519389in}{0.939501in}}%
\pgfpathlineto{\pgfqpoint{3.521713in}{0.960337in}}%
\pgfpathlineto{\pgfqpoint{3.524037in}{0.943229in}}%
\pgfpathlineto{\pgfqpoint{3.526361in}{0.950408in}}%
\pgfpathlineto{\pgfqpoint{3.528685in}{0.966058in}}%
\pgfpathlineto{\pgfqpoint{3.531009in}{0.932759in}}%
\pgfpathlineto{\pgfqpoint{3.533333in}{0.953593in}}%
\pgfpathlineto{\pgfqpoint{3.535657in}{0.945252in}}%
\pgfpathlineto{\pgfqpoint{3.540305in}{0.959137in}}%
\pgfpathlineto{\pgfqpoint{3.542629in}{0.953192in}}%
\pgfpathlineto{\pgfqpoint{3.544953in}{0.940303in}}%
\pgfpathlineto{\pgfqpoint{3.547277in}{0.969050in}}%
\pgfpathlineto{\pgfqpoint{3.549600in}{0.982725in}}%
\pgfpathlineto{\pgfqpoint{3.551924in}{0.954914in}}%
\pgfpathlineto{\pgfqpoint{3.554248in}{0.957049in}}%
\pgfpathlineto{\pgfqpoint{3.556572in}{1.208575in}}%
\pgfpathlineto{\pgfqpoint{3.558896in}{1.250993in}}%
\pgfpathlineto{\pgfqpoint{3.563544in}{1.235194in}}%
\pgfpathlineto{\pgfqpoint{3.565868in}{1.257529in}}%
\pgfpathlineto{\pgfqpoint{3.570516in}{1.222605in}}%
\pgfpathlineto{\pgfqpoint{3.572840in}{1.247165in}}%
\pgfpathlineto{\pgfqpoint{3.577488in}{1.237860in}}%
\pgfpathlineto{\pgfqpoint{3.579812in}{1.254291in}}%
\pgfpathlineto{\pgfqpoint{3.582136in}{1.245474in}}%
\pgfpathlineto{\pgfqpoint{3.584460in}{1.243587in}}%
\pgfpathlineto{\pgfqpoint{3.586784in}{1.230729in}}%
\pgfpathlineto{\pgfqpoint{3.589108in}{1.238839in}}%
\pgfpathlineto{\pgfqpoint{3.591432in}{1.238042in}}%
\pgfpathlineto{\pgfqpoint{3.596080in}{1.249512in}}%
\pgfpathlineto{\pgfqpoint{3.598404in}{1.248489in}}%
\pgfpathlineto{\pgfqpoint{3.600728in}{1.233033in}}%
\pgfpathlineto{\pgfqpoint{3.603052in}{1.259769in}}%
\pgfpathlineto{\pgfqpoint{3.605376in}{1.258778in}}%
\pgfpathlineto{\pgfqpoint{3.612348in}{1.223666in}}%
\pgfpathlineto{\pgfqpoint{3.614672in}{1.231718in}}%
\pgfpathlineto{\pgfqpoint{3.616996in}{1.220190in}}%
\pgfpathlineto{\pgfqpoint{3.619319in}{1.241438in}}%
\pgfpathlineto{\pgfqpoint{3.621643in}{1.250258in}}%
\pgfpathlineto{\pgfqpoint{3.623967in}{1.220617in}}%
\pgfpathlineto{\pgfqpoint{3.626291in}{1.209123in}}%
\pgfpathlineto{\pgfqpoint{3.628615in}{1.263347in}}%
\pgfpathlineto{\pgfqpoint{3.630939in}{1.244069in}}%
\pgfpathlineto{\pgfqpoint{3.633263in}{1.242324in}}%
\pgfpathlineto{\pgfqpoint{3.635587in}{1.254913in}}%
\pgfpathlineto{\pgfqpoint{3.637911in}{1.223532in}}%
\pgfpathlineto{\pgfqpoint{3.640235in}{1.267676in}}%
\pgfpathlineto{\pgfqpoint{3.642559in}{1.248618in}}%
\pgfpathlineto{\pgfqpoint{3.644883in}{1.217963in}}%
\pgfpathlineto{\pgfqpoint{3.647207in}{1.246377in}}%
\pgfpathlineto{\pgfqpoint{3.649531in}{1.220012in}}%
\pgfpathlineto{\pgfqpoint{3.651855in}{1.234453in}}%
\pgfpathlineto{\pgfqpoint{3.654179in}{1.218109in}}%
\pgfpathlineto{\pgfqpoint{3.656503in}{1.209979in}}%
\pgfpathlineto{\pgfqpoint{3.661151in}{1.249968in}}%
\pgfpathlineto{\pgfqpoint{3.663475in}{1.231242in}}%
\pgfpathlineto{\pgfqpoint{3.665799in}{1.238609in}}%
\pgfpathlineto{\pgfqpoint{3.668123in}{1.227997in}}%
\pgfpathlineto{\pgfqpoint{3.670447in}{1.264714in}}%
\pgfpathlineto{\pgfqpoint{3.672771in}{1.211739in}}%
\pgfpathlineto{\pgfqpoint{3.675095in}{1.234027in}}%
\pgfpathlineto{\pgfqpoint{3.677419in}{1.242374in}}%
\pgfpathlineto{\pgfqpoint{3.682067in}{1.247435in}}%
\pgfpathlineto{\pgfqpoint{3.684391in}{1.248437in}}%
\pgfpathlineto{\pgfqpoint{3.686715in}{1.281999in}}%
\pgfpathlineto{\pgfqpoint{3.689038in}{1.238552in}}%
\pgfpathlineto{\pgfqpoint{3.693686in}{1.247881in}}%
\pgfpathlineto{\pgfqpoint{3.698334in}{1.223891in}}%
\pgfpathlineto{\pgfqpoint{3.700658in}{1.237842in}}%
\pgfpathlineto{\pgfqpoint{3.705306in}{1.236940in}}%
\pgfpathlineto{\pgfqpoint{3.707630in}{1.256147in}}%
\pgfpathlineto{\pgfqpoint{3.709954in}{1.249796in}}%
\pgfpathlineto{\pgfqpoint{3.712278in}{1.248305in}}%
\pgfpathlineto{\pgfqpoint{3.714602in}{1.227610in}}%
\pgfpathlineto{\pgfqpoint{3.716926in}{1.231377in}}%
\pgfpathlineto{\pgfqpoint{3.719250in}{1.264866in}}%
\pgfpathlineto{\pgfqpoint{3.721574in}{1.230445in}}%
\pgfpathlineto{\pgfqpoint{3.723898in}{1.221116in}}%
\pgfpathlineto{\pgfqpoint{3.726222in}{1.235350in}}%
\pgfpathlineto{\pgfqpoint{3.728546in}{1.219392in}}%
\pgfpathlineto{\pgfqpoint{3.730870in}{1.246291in}}%
\pgfpathlineto{\pgfqpoint{3.733194in}{1.259507in}}%
\pgfpathlineto{\pgfqpoint{3.737842in}{1.226969in}}%
\pgfpathlineto{\pgfqpoint{3.740166in}{1.253252in}}%
\pgfpathlineto{\pgfqpoint{3.742490in}{1.236657in}}%
\pgfpathlineto{\pgfqpoint{3.744814in}{1.259329in}}%
\pgfpathlineto{\pgfqpoint{3.747138in}{1.231706in}}%
\pgfpathlineto{\pgfqpoint{3.749462in}{1.243196in}}%
\pgfpathlineto{\pgfqpoint{3.751786in}{1.269447in}}%
\pgfpathlineto{\pgfqpoint{3.754110in}{1.246558in}}%
\pgfpathlineto{\pgfqpoint{3.756434in}{1.244572in}}%
\pgfpathlineto{\pgfqpoint{3.758757in}{1.246065in}}%
\pgfpathlineto{\pgfqpoint{3.761081in}{1.249465in}}%
\pgfpathlineto{\pgfqpoint{3.763405in}{1.264771in}}%
\pgfpathlineto{\pgfqpoint{3.765729in}{1.235048in}}%
\pgfpathlineto{\pgfqpoint{3.770377in}{1.237589in}}%
\pgfpathlineto{\pgfqpoint{3.772701in}{1.223536in}}%
\pgfpathlineto{\pgfqpoint{3.775025in}{1.252189in}}%
\pgfpathlineto{\pgfqpoint{3.777349in}{1.234917in}}%
\pgfpathlineto{\pgfqpoint{3.779673in}{1.240084in}}%
\pgfpathlineto{\pgfqpoint{3.781997in}{1.256652in}}%
\pgfpathlineto{\pgfqpoint{3.784321in}{1.254656in}}%
\pgfpathlineto{\pgfqpoint{3.786645in}{1.235938in}}%
\pgfpathlineto{\pgfqpoint{3.788969in}{1.246055in}}%
\pgfpathlineto{\pgfqpoint{3.791293in}{1.227374in}}%
\pgfpathlineto{\pgfqpoint{3.793617in}{1.251685in}}%
\pgfpathlineto{\pgfqpoint{3.798265in}{1.231748in}}%
\pgfpathlineto{\pgfqpoint{3.800589in}{1.250578in}}%
\pgfpathlineto{\pgfqpoint{3.802913in}{1.253248in}}%
\pgfpathlineto{\pgfqpoint{3.805237in}{1.249668in}}%
\pgfpathlineto{\pgfqpoint{3.807561in}{1.215368in}}%
\pgfpathlineto{\pgfqpoint{3.809885in}{1.247508in}}%
\pgfpathlineto{\pgfqpoint{3.812209in}{1.230102in}}%
\pgfpathlineto{\pgfqpoint{3.814533in}{1.241416in}}%
\pgfpathlineto{\pgfqpoint{3.816857in}{1.235656in}}%
\pgfpathlineto{\pgfqpoint{3.819181in}{1.240113in}}%
\pgfpathlineto{\pgfqpoint{3.821505in}{1.239483in}}%
\pgfpathlineto{\pgfqpoint{3.823829in}{1.234873in}}%
\pgfpathlineto{\pgfqpoint{3.826153in}{1.236834in}}%
\pgfpathlineto{\pgfqpoint{3.828477in}{1.214477in}}%
\pgfpathlineto{\pgfqpoint{3.833124in}{1.255380in}}%
\pgfpathlineto{\pgfqpoint{3.835448in}{1.225850in}}%
\pgfpathlineto{\pgfqpoint{3.840096in}{1.269799in}}%
\pgfpathlineto{\pgfqpoint{3.842420in}{1.237447in}}%
\pgfpathlineto{\pgfqpoint{3.844744in}{1.225239in}}%
\pgfpathlineto{\pgfqpoint{3.847068in}{0.967858in}}%
\pgfpathlineto{\pgfqpoint{3.854040in}{0.884594in}}%
\pgfpathlineto{\pgfqpoint{3.856364in}{0.928799in}}%
\pgfpathlineto{\pgfqpoint{3.858688in}{0.906550in}}%
\pgfpathlineto{\pgfqpoint{3.861012in}{0.922487in}}%
\pgfpathlineto{\pgfqpoint{3.863336in}{0.924948in}}%
\pgfpathlineto{\pgfqpoint{3.865660in}{0.901778in}}%
\pgfpathlineto{\pgfqpoint{3.870308in}{0.939826in}}%
\pgfpathlineto{\pgfqpoint{3.872632in}{0.937492in}}%
\pgfpathlineto{\pgfqpoint{3.874956in}{0.945616in}}%
\pgfpathlineto{\pgfqpoint{3.877280in}{0.934871in}}%
\pgfpathlineto{\pgfqpoint{3.879604in}{0.907451in}}%
\pgfpathlineto{\pgfqpoint{3.881928in}{0.925061in}}%
\pgfpathlineto{\pgfqpoint{3.884252in}{0.916926in}}%
\pgfpathlineto{\pgfqpoint{3.886576in}{0.913272in}}%
\pgfpathlineto{\pgfqpoint{3.888900in}{0.916750in}}%
\pgfpathlineto{\pgfqpoint{3.891224in}{0.910879in}}%
\pgfpathlineto{\pgfqpoint{3.893548in}{0.924321in}}%
\pgfpathlineto{\pgfqpoint{3.895872in}{0.916106in}}%
\pgfpathlineto{\pgfqpoint{3.898196in}{0.934954in}}%
\pgfpathlineto{\pgfqpoint{3.900519in}{0.918902in}}%
\pgfpathlineto{\pgfqpoint{3.902843in}{0.916297in}}%
\pgfpathlineto{\pgfqpoint{3.907491in}{0.927200in}}%
\pgfpathlineto{\pgfqpoint{3.912139in}{0.916571in}}%
\pgfpathlineto{\pgfqpoint{3.914463in}{0.938226in}}%
\pgfpathlineto{\pgfqpoint{3.916787in}{0.938149in}}%
\pgfpathlineto{\pgfqpoint{3.919111in}{0.926674in}}%
\pgfpathlineto{\pgfqpoint{3.923759in}{0.879681in}}%
\pgfpathlineto{\pgfqpoint{3.926083in}{0.907713in}}%
\pgfpathlineto{\pgfqpoint{3.930731in}{0.916589in}}%
\pgfpathlineto{\pgfqpoint{3.933055in}{0.932564in}}%
\pgfpathlineto{\pgfqpoint{3.935379in}{0.898430in}}%
\pgfpathlineto{\pgfqpoint{3.937703in}{0.922396in}}%
\pgfpathlineto{\pgfqpoint{3.940027in}{0.933732in}}%
\pgfpathlineto{\pgfqpoint{3.942351in}{0.926979in}}%
\pgfpathlineto{\pgfqpoint{3.944675in}{0.895399in}}%
\pgfpathlineto{\pgfqpoint{3.946999in}{0.907476in}}%
\pgfpathlineto{\pgfqpoint{3.949323in}{0.938734in}}%
\pgfpathlineto{\pgfqpoint{3.951647in}{0.929269in}}%
\pgfpathlineto{\pgfqpoint{3.956295in}{0.889354in}}%
\pgfpathlineto{\pgfqpoint{3.958619in}{0.934795in}}%
\pgfpathlineto{\pgfqpoint{3.960943in}{0.897045in}}%
\pgfpathlineto{\pgfqpoint{3.963267in}{0.915219in}}%
\pgfpathlineto{\pgfqpoint{3.965591in}{0.916025in}}%
\pgfpathlineto{\pgfqpoint{3.967915in}{0.915246in}}%
\pgfpathlineto{\pgfqpoint{3.970238in}{0.925242in}}%
\pgfpathlineto{\pgfqpoint{3.972562in}{0.909405in}}%
\pgfpathlineto{\pgfqpoint{3.974886in}{0.921493in}}%
\pgfpathlineto{\pgfqpoint{3.977210in}{0.905426in}}%
\pgfpathlineto{\pgfqpoint{3.979534in}{0.918392in}}%
\pgfpathlineto{\pgfqpoint{3.981858in}{0.918816in}}%
\pgfpathlineto{\pgfqpoint{3.984182in}{0.889321in}}%
\pgfpathlineto{\pgfqpoint{3.988830in}{0.930993in}}%
\pgfpathlineto{\pgfqpoint{3.991154in}{0.912084in}}%
\pgfpathlineto{\pgfqpoint{3.993478in}{0.932004in}}%
\pgfpathlineto{\pgfqpoint{3.995802in}{0.929308in}}%
\pgfpathlineto{\pgfqpoint{3.998126in}{0.923342in}}%
\pgfpathlineto{\pgfqpoint{4.000450in}{0.929231in}}%
\pgfpathlineto{\pgfqpoint{4.002774in}{0.921362in}}%
\pgfpathlineto{\pgfqpoint{4.005098in}{0.917746in}}%
\pgfpathlineto{\pgfqpoint{4.007422in}{0.931466in}}%
\pgfpathlineto{\pgfqpoint{4.009746in}{0.910302in}}%
\pgfpathlineto{\pgfqpoint{4.012070in}{0.908886in}}%
\pgfpathlineto{\pgfqpoint{4.014394in}{0.928119in}}%
\pgfpathlineto{\pgfqpoint{4.016718in}{0.925638in}}%
\pgfpathlineto{\pgfqpoint{4.019042in}{0.919695in}}%
\pgfpathlineto{\pgfqpoint{4.021366in}{0.933745in}}%
\pgfpathlineto{\pgfqpoint{4.023690in}{0.912184in}}%
\pgfpathlineto{\pgfqpoint{4.026014in}{0.921966in}}%
\pgfpathlineto{\pgfqpoint{4.028338in}{0.925122in}}%
\pgfpathlineto{\pgfqpoint{4.030662in}{0.932318in}}%
\pgfpathlineto{\pgfqpoint{4.032986in}{0.918925in}}%
\pgfpathlineto{\pgfqpoint{4.035310in}{0.928046in}}%
\pgfpathlineto{\pgfqpoint{4.037634in}{0.903362in}}%
\pgfpathlineto{\pgfqpoint{4.039957in}{0.895865in}}%
\pgfpathlineto{\pgfqpoint{4.042281in}{0.949605in}}%
\pgfpathlineto{\pgfqpoint{4.044605in}{0.913042in}}%
\pgfpathlineto{\pgfqpoint{4.046929in}{0.925145in}}%
\pgfpathlineto{\pgfqpoint{4.049253in}{0.948611in}}%
\pgfpathlineto{\pgfqpoint{4.053901in}{0.904539in}}%
\pgfpathlineto{\pgfqpoint{4.056225in}{0.921805in}}%
\pgfpathlineto{\pgfqpoint{4.058549in}{0.897849in}}%
\pgfpathlineto{\pgfqpoint{4.060873in}{0.905365in}}%
\pgfpathlineto{\pgfqpoint{4.063197in}{0.900552in}}%
\pgfpathlineto{\pgfqpoint{4.065521in}{0.958826in}}%
\pgfpathlineto{\pgfqpoint{4.067845in}{0.903504in}}%
\pgfpathlineto{\pgfqpoint{4.072493in}{0.922702in}}%
\pgfpathlineto{\pgfqpoint{4.074817in}{0.911465in}}%
\pgfpathlineto{\pgfqpoint{4.077141in}{0.913822in}}%
\pgfpathlineto{\pgfqpoint{4.081789in}{0.931718in}}%
\pgfpathlineto{\pgfqpoint{4.084113in}{0.922166in}}%
\pgfpathlineto{\pgfqpoint{4.088761in}{0.947523in}}%
\pgfpathlineto{\pgfqpoint{4.091085in}{0.928169in}}%
\pgfpathlineto{\pgfqpoint{4.093409in}{0.926776in}}%
\pgfpathlineto{\pgfqpoint{4.095733in}{0.916343in}}%
\pgfpathlineto{\pgfqpoint{4.098057in}{0.938430in}}%
\pgfpathlineto{\pgfqpoint{4.100381in}{0.940855in}}%
\pgfpathlineto{\pgfqpoint{4.102705in}{0.956298in}}%
\pgfpathlineto{\pgfqpoint{4.105029in}{0.907038in}}%
\pgfpathlineto{\pgfqpoint{4.107353in}{0.903403in}}%
\pgfpathlineto{\pgfqpoint{4.109677in}{0.933120in}}%
\pgfpathlineto{\pgfqpoint{4.112000in}{0.920927in}}%
\pgfpathlineto{\pgfqpoint{4.116648in}{0.947226in}}%
\pgfpathlineto{\pgfqpoint{4.118972in}{0.915310in}}%
\pgfpathlineto{\pgfqpoint{4.121296in}{0.946094in}}%
\pgfpathlineto{\pgfqpoint{4.123620in}{0.943602in}}%
\pgfpathlineto{\pgfqpoint{4.125944in}{0.944067in}}%
\pgfpathlineto{\pgfqpoint{4.128268in}{0.931944in}}%
\pgfpathlineto{\pgfqpoint{4.130592in}{0.904632in}}%
\pgfpathlineto{\pgfqpoint{4.132916in}{0.947390in}}%
\pgfpathlineto{\pgfqpoint{4.135240in}{0.910427in}}%
\pgfpathlineto{\pgfqpoint{4.137564in}{1.209336in}}%
\pgfpathlineto{\pgfqpoint{4.139888in}{1.208734in}}%
\pgfpathlineto{\pgfqpoint{4.142212in}{1.198828in}}%
\pgfpathlineto{\pgfqpoint{4.144536in}{1.229897in}}%
\pgfpathlineto{\pgfqpoint{4.149184in}{1.212276in}}%
\pgfpathlineto{\pgfqpoint{4.151508in}{1.214606in}}%
\pgfpathlineto{\pgfqpoint{4.153832in}{1.232263in}}%
\pgfpathlineto{\pgfqpoint{4.156156in}{1.187811in}}%
\pgfpathlineto{\pgfqpoint{4.158480in}{1.191262in}}%
\pgfpathlineto{\pgfqpoint{4.160804in}{1.230264in}}%
\pgfpathlineto{\pgfqpoint{4.163128in}{1.199773in}}%
\pgfpathlineto{\pgfqpoint{4.165452in}{1.198588in}}%
\pgfpathlineto{\pgfqpoint{4.167776in}{1.201962in}}%
\pgfpathlineto{\pgfqpoint{4.170100in}{1.226436in}}%
\pgfpathlineto{\pgfqpoint{4.172424in}{1.185424in}}%
\pgfpathlineto{\pgfqpoint{4.177072in}{1.213447in}}%
\pgfpathlineto{\pgfqpoint{4.179396in}{1.196811in}}%
\pgfpathlineto{\pgfqpoint{4.181719in}{1.206527in}}%
\pgfpathlineto{\pgfqpoint{4.184043in}{1.223046in}}%
\pgfpathlineto{\pgfqpoint{4.186367in}{1.223608in}}%
\pgfpathlineto{\pgfqpoint{4.188691in}{1.197152in}}%
\pgfpathlineto{\pgfqpoint{4.191015in}{1.209658in}}%
\pgfpathlineto{\pgfqpoint{4.193339in}{1.206751in}}%
\pgfpathlineto{\pgfqpoint{4.195663in}{1.222408in}}%
\pgfpathlineto{\pgfqpoint{4.197987in}{1.209580in}}%
\pgfpathlineto{\pgfqpoint{4.200311in}{1.230128in}}%
\pgfpathlineto{\pgfqpoint{4.202635in}{1.204833in}}%
\pgfpathlineto{\pgfqpoint{4.204959in}{1.228662in}}%
\pgfpathlineto{\pgfqpoint{4.207283in}{1.205934in}}%
\pgfpathlineto{\pgfqpoint{4.209607in}{1.205618in}}%
\pgfpathlineto{\pgfqpoint{4.211931in}{1.202755in}}%
\pgfpathlineto{\pgfqpoint{4.216579in}{1.240259in}}%
\pgfpathlineto{\pgfqpoint{4.218903in}{1.231970in}}%
\pgfpathlineto{\pgfqpoint{4.221227in}{1.228953in}}%
\pgfpathlineto{\pgfqpoint{4.223551in}{1.241406in}}%
\pgfpathlineto{\pgfqpoint{4.225875in}{1.196526in}}%
\pgfpathlineto{\pgfqpoint{4.228199in}{1.198189in}}%
\pgfpathlineto{\pgfqpoint{4.230523in}{1.226152in}}%
\pgfpathlineto{\pgfqpoint{4.232847in}{1.219306in}}%
\pgfpathlineto{\pgfqpoint{4.235171in}{1.206608in}}%
\pgfpathlineto{\pgfqpoint{4.239819in}{1.221998in}}%
\pgfpathlineto{\pgfqpoint{4.242143in}{1.216103in}}%
\pgfpathlineto{\pgfqpoint{4.244467in}{1.218704in}}%
\pgfpathlineto{\pgfqpoint{4.246791in}{1.187108in}}%
\pgfpathlineto{\pgfqpoint{4.249115in}{1.204437in}}%
\pgfpathlineto{\pgfqpoint{4.251438in}{1.211606in}}%
\pgfpathlineto{\pgfqpoint{4.253762in}{1.227864in}}%
\pgfpathlineto{\pgfqpoint{4.256086in}{1.223281in}}%
\pgfpathlineto{\pgfqpoint{4.258410in}{1.211122in}}%
\pgfpathlineto{\pgfqpoint{4.260734in}{1.235550in}}%
\pgfpathlineto{\pgfqpoint{4.263058in}{1.212055in}}%
\pgfpathlineto{\pgfqpoint{4.265382in}{1.220725in}}%
\pgfpathlineto{\pgfqpoint{4.267706in}{1.205923in}}%
\pgfpathlineto{\pgfqpoint{4.270030in}{1.214833in}}%
\pgfpathlineto{\pgfqpoint{4.272354in}{1.212615in}}%
\pgfpathlineto{\pgfqpoint{4.274678in}{1.203437in}}%
\pgfpathlineto{\pgfqpoint{4.277002in}{1.225021in}}%
\pgfpathlineto{\pgfqpoint{4.279326in}{1.235581in}}%
\pgfpathlineto{\pgfqpoint{4.286298in}{1.220788in}}%
\pgfpathlineto{\pgfqpoint{4.288622in}{1.222445in}}%
\pgfpathlineto{\pgfqpoint{4.295594in}{1.206913in}}%
\pgfpathlineto{\pgfqpoint{4.297918in}{1.214321in}}%
\pgfpathlineto{\pgfqpoint{4.300242in}{1.268089in}}%
\pgfpathlineto{\pgfqpoint{4.302566in}{1.238777in}}%
\pgfpathlineto{\pgfqpoint{4.304890in}{1.243865in}}%
\pgfpathlineto{\pgfqpoint{4.307214in}{1.218171in}}%
\pgfpathlineto{\pgfqpoint{4.309538in}{1.229050in}}%
\pgfpathlineto{\pgfqpoint{4.311862in}{1.230542in}}%
\pgfpathlineto{\pgfqpoint{4.314186in}{1.221920in}}%
\pgfpathlineto{\pgfqpoint{4.316510in}{1.200737in}}%
\pgfpathlineto{\pgfqpoint{4.321158in}{1.221473in}}%
\pgfpathlineto{\pgfqpoint{4.323481in}{1.213105in}}%
\pgfpathlineto{\pgfqpoint{4.328129in}{1.229041in}}%
\pgfpathlineto{\pgfqpoint{4.330453in}{1.222954in}}%
\pgfpathlineto{\pgfqpoint{4.332777in}{1.220373in}}%
\pgfpathlineto{\pgfqpoint{4.335101in}{1.206670in}}%
\pgfpathlineto{\pgfqpoint{4.337425in}{1.250349in}}%
\pgfpathlineto{\pgfqpoint{4.339749in}{1.226259in}}%
\pgfpathlineto{\pgfqpoint{4.342073in}{1.235089in}}%
\pgfpathlineto{\pgfqpoint{4.344397in}{1.212250in}}%
\pgfpathlineto{\pgfqpoint{4.346721in}{1.260562in}}%
\pgfpathlineto{\pgfqpoint{4.349045in}{1.239714in}}%
\pgfpathlineto{\pgfqpoint{4.351369in}{1.242702in}}%
\pgfpathlineto{\pgfqpoint{4.353693in}{1.239053in}}%
\pgfpathlineto{\pgfqpoint{4.356017in}{1.209766in}}%
\pgfpathlineto{\pgfqpoint{4.358341in}{1.245804in}}%
\pgfpathlineto{\pgfqpoint{4.360665in}{1.248138in}}%
\pgfpathlineto{\pgfqpoint{4.362989in}{1.221247in}}%
\pgfpathlineto{\pgfqpoint{4.365313in}{1.233302in}}%
\pgfpathlineto{\pgfqpoint{4.367637in}{1.231190in}}%
\pgfpathlineto{\pgfqpoint{4.369961in}{1.216732in}}%
\pgfpathlineto{\pgfqpoint{4.372285in}{1.242430in}}%
\pgfpathlineto{\pgfqpoint{4.374609in}{1.225716in}}%
\pgfpathlineto{\pgfqpoint{4.376933in}{1.236792in}}%
\pgfpathlineto{\pgfqpoint{4.379257in}{1.236098in}}%
\pgfpathlineto{\pgfqpoint{4.381581in}{1.250834in}}%
\pgfpathlineto{\pgfqpoint{4.383905in}{1.243462in}}%
\pgfpathlineto{\pgfqpoint{4.386229in}{1.225936in}}%
\pgfpathlineto{\pgfqpoint{4.390877in}{1.229655in}}%
\pgfpathlineto{\pgfqpoint{4.393200in}{1.238592in}}%
\pgfpathlineto{\pgfqpoint{4.395524in}{1.240162in}}%
\pgfpathlineto{\pgfqpoint{4.397848in}{1.266921in}}%
\pgfpathlineto{\pgfqpoint{4.400172in}{1.225177in}}%
\pgfpathlineto{\pgfqpoint{4.402496in}{1.281680in}}%
\pgfpathlineto{\pgfqpoint{4.404820in}{1.194758in}}%
\pgfpathlineto{\pgfqpoint{4.407144in}{1.238432in}}%
\pgfpathlineto{\pgfqpoint{4.409468in}{1.235811in}}%
\pgfpathlineto{\pgfqpoint{4.411792in}{1.224141in}}%
\pgfpathlineto{\pgfqpoint{4.414116in}{1.240000in}}%
\pgfpathlineto{\pgfqpoint{4.416440in}{1.265867in}}%
\pgfpathlineto{\pgfqpoint{4.418764in}{1.244411in}}%
\pgfpathlineto{\pgfqpoint{4.421088in}{1.251045in}}%
\pgfpathlineto{\pgfqpoint{4.423412in}{1.239773in}}%
\pgfpathlineto{\pgfqpoint{4.425736in}{1.249839in}}%
\pgfpathlineto{\pgfqpoint{4.428060in}{0.918355in}}%
\pgfpathlineto{\pgfqpoint{4.430384in}{0.904937in}}%
\pgfpathlineto{\pgfqpoint{4.432708in}{0.925127in}}%
\pgfpathlineto{\pgfqpoint{4.435032in}{0.914563in}}%
\pgfpathlineto{\pgfqpoint{4.437356in}{0.919363in}}%
\pgfpathlineto{\pgfqpoint{4.439680in}{0.906191in}}%
\pgfpathlineto{\pgfqpoint{4.444328in}{0.923471in}}%
\pgfpathlineto{\pgfqpoint{4.446652in}{0.906112in}}%
\pgfpathlineto{\pgfqpoint{4.448976in}{0.932084in}}%
\pgfpathlineto{\pgfqpoint{4.451300in}{0.923279in}}%
\pgfpathlineto{\pgfqpoint{4.453624in}{0.890651in}}%
\pgfpathlineto{\pgfqpoint{4.455948in}{0.901539in}}%
\pgfpathlineto{\pgfqpoint{4.458272in}{0.923162in}}%
\pgfpathlineto{\pgfqpoint{4.460596in}{0.914337in}}%
\pgfpathlineto{\pgfqpoint{4.462919in}{0.925601in}}%
\pgfpathlineto{\pgfqpoint{4.465243in}{0.929100in}}%
\pgfpathlineto{\pgfqpoint{4.467567in}{0.922788in}}%
\pgfpathlineto{\pgfqpoint{4.469891in}{0.931478in}}%
\pgfpathlineto{\pgfqpoint{4.472215in}{0.883376in}}%
\pgfpathlineto{\pgfqpoint{4.474539in}{0.922105in}}%
\pgfpathlineto{\pgfqpoint{4.476863in}{0.907366in}}%
\pgfpathlineto{\pgfqpoint{4.479187in}{0.924114in}}%
\pgfpathlineto{\pgfqpoint{4.481511in}{0.927905in}}%
\pgfpathlineto{\pgfqpoint{4.483835in}{0.924325in}}%
\pgfpathlineto{\pgfqpoint{4.486159in}{0.928789in}}%
\pgfpathlineto{\pgfqpoint{4.488483in}{0.947576in}}%
\pgfpathlineto{\pgfqpoint{4.490807in}{0.916009in}}%
\pgfpathlineto{\pgfqpoint{4.493131in}{0.943514in}}%
\pgfpathlineto{\pgfqpoint{4.495455in}{0.937749in}}%
\pgfpathlineto{\pgfqpoint{4.497779in}{0.921173in}}%
\pgfpathlineto{\pgfqpoint{4.500103in}{0.932162in}}%
\pgfpathlineto{\pgfqpoint{4.502427in}{0.923704in}}%
\pgfpathlineto{\pgfqpoint{4.504751in}{0.928054in}}%
\pgfpathlineto{\pgfqpoint{4.507075in}{0.926811in}}%
\pgfpathlineto{\pgfqpoint{4.509399in}{0.903731in}}%
\pgfpathlineto{\pgfqpoint{4.514047in}{0.927522in}}%
\pgfpathlineto{\pgfqpoint{4.516371in}{0.920355in}}%
\pgfpathlineto{\pgfqpoint{4.518695in}{0.930028in}}%
\pgfpathlineto{\pgfqpoint{4.521019in}{0.933805in}}%
\pgfpathlineto{\pgfqpoint{4.523343in}{0.931741in}}%
\pgfpathlineto{\pgfqpoint{4.525667in}{0.922840in}}%
\pgfpathlineto{\pgfqpoint{4.527991in}{0.930556in}}%
\pgfpathlineto{\pgfqpoint{4.530315in}{0.942735in}}%
\pgfpathlineto{\pgfqpoint{4.532638in}{0.930451in}}%
\pgfpathlineto{\pgfqpoint{4.539610in}{0.933564in}}%
\pgfpathlineto{\pgfqpoint{4.541934in}{0.944728in}}%
\pgfpathlineto{\pgfqpoint{4.544258in}{0.918875in}}%
\pgfpathlineto{\pgfqpoint{4.546582in}{0.936404in}}%
\pgfpathlineto{\pgfqpoint{4.548906in}{0.931616in}}%
\pgfpathlineto{\pgfqpoint{4.551230in}{0.951317in}}%
\pgfpathlineto{\pgfqpoint{4.555878in}{0.940632in}}%
\pgfpathlineto{\pgfqpoint{4.558202in}{0.922746in}}%
\pgfpathlineto{\pgfqpoint{4.560526in}{0.951296in}}%
\pgfpathlineto{\pgfqpoint{4.562850in}{0.896059in}}%
\pgfpathlineto{\pgfqpoint{4.565174in}{0.935571in}}%
\pgfpathlineto{\pgfqpoint{4.569822in}{0.929990in}}%
\pgfpathlineto{\pgfqpoint{4.572146in}{0.941639in}}%
\pgfpathlineto{\pgfqpoint{4.576794in}{0.929317in}}%
\pgfpathlineto{\pgfqpoint{4.579118in}{0.937414in}}%
\pgfpathlineto{\pgfqpoint{4.581442in}{0.967289in}}%
\pgfpathlineto{\pgfqpoint{4.583766in}{0.943973in}}%
\pgfpathlineto{\pgfqpoint{4.586090in}{0.957017in}}%
\pgfpathlineto{\pgfqpoint{4.588414in}{0.950459in}}%
\pgfpathlineto{\pgfqpoint{4.590738in}{0.949802in}}%
\pgfpathlineto{\pgfqpoint{4.593062in}{0.914751in}}%
\pgfpathlineto{\pgfqpoint{4.595386in}{0.935924in}}%
\pgfpathlineto{\pgfqpoint{4.597710in}{0.943007in}}%
\pgfpathlineto{\pgfqpoint{4.600034in}{0.931554in}}%
\pgfpathlineto{\pgfqpoint{4.602358in}{0.964854in}}%
\pgfpathlineto{\pgfqpoint{4.604681in}{0.924895in}}%
\pgfpathlineto{\pgfqpoint{4.607005in}{0.954711in}}%
\pgfpathlineto{\pgfqpoint{4.609329in}{0.948302in}}%
\pgfpathlineto{\pgfqpoint{4.611653in}{0.920652in}}%
\pgfpathlineto{\pgfqpoint{4.616301in}{0.947090in}}%
\pgfpathlineto{\pgfqpoint{4.620949in}{0.931036in}}%
\pgfpathlineto{\pgfqpoint{4.623273in}{0.927512in}}%
\pgfpathlineto{\pgfqpoint{4.625597in}{0.937405in}}%
\pgfpathlineto{\pgfqpoint{4.627921in}{0.965566in}}%
\pgfpathlineto{\pgfqpoint{4.630245in}{0.969857in}}%
\pgfpathlineto{\pgfqpoint{4.632569in}{0.930810in}}%
\pgfpathlineto{\pgfqpoint{4.634893in}{0.947013in}}%
\pgfpathlineto{\pgfqpoint{4.637217in}{0.946299in}}%
\pgfpathlineto{\pgfqpoint{4.639541in}{0.931529in}}%
\pgfpathlineto{\pgfqpoint{4.641865in}{0.932844in}}%
\pgfpathlineto{\pgfqpoint{4.644189in}{0.940469in}}%
\pgfpathlineto{\pgfqpoint{4.646513in}{0.965701in}}%
\pgfpathlineto{\pgfqpoint{4.648837in}{0.920576in}}%
\pgfpathlineto{\pgfqpoint{4.653485in}{0.971672in}}%
\pgfpathlineto{\pgfqpoint{4.655809in}{0.939624in}}%
\pgfpathlineto{\pgfqpoint{4.658133in}{0.949934in}}%
\pgfpathlineto{\pgfqpoint{4.660457in}{0.942738in}}%
\pgfpathlineto{\pgfqpoint{4.662781in}{0.957993in}}%
\pgfpathlineto{\pgfqpoint{4.665105in}{0.963240in}}%
\pgfpathlineto{\pgfqpoint{4.669753in}{0.939668in}}%
\pgfpathlineto{\pgfqpoint{4.672077in}{0.955320in}}%
\pgfpathlineto{\pgfqpoint{4.676724in}{0.948453in}}%
\pgfpathlineto{\pgfqpoint{4.679048in}{0.938495in}}%
\pgfpathlineto{\pgfqpoint{4.681372in}{0.953249in}}%
\pgfpathlineto{\pgfqpoint{4.683696in}{0.950797in}}%
\pgfpathlineto{\pgfqpoint{4.686020in}{0.966814in}}%
\pgfpathlineto{\pgfqpoint{4.688344in}{0.934599in}}%
\pgfpathlineto{\pgfqpoint{4.690668in}{0.935616in}}%
\pgfpathlineto{\pgfqpoint{4.692992in}{0.933680in}}%
\pgfpathlineto{\pgfqpoint{4.695316in}{0.959180in}}%
\pgfpathlineto{\pgfqpoint{4.697640in}{0.937593in}}%
\pgfpathlineto{\pgfqpoint{4.699964in}{0.939841in}}%
\pgfpathlineto{\pgfqpoint{4.702288in}{0.964512in}}%
\pgfpathlineto{\pgfqpoint{4.704612in}{0.942781in}}%
\pgfpathlineto{\pgfqpoint{4.709260in}{0.964684in}}%
\pgfpathlineto{\pgfqpoint{4.711584in}{0.973254in}}%
\pgfpathlineto{\pgfqpoint{4.713908in}{0.961719in}}%
\pgfpathlineto{\pgfqpoint{4.716232in}{0.963971in}}%
\pgfpathlineto{\pgfqpoint{4.718556in}{1.250405in}}%
\pgfpathlineto{\pgfqpoint{4.720880in}{1.258369in}}%
\pgfpathlineto{\pgfqpoint{4.723204in}{1.252374in}}%
\pgfpathlineto{\pgfqpoint{4.725528in}{1.224886in}}%
\pgfpathlineto{\pgfqpoint{4.727852in}{1.242672in}}%
\pgfpathlineto{\pgfqpoint{4.730176in}{1.222536in}}%
\pgfpathlineto{\pgfqpoint{4.732500in}{1.262791in}}%
\pgfpathlineto{\pgfqpoint{4.734824in}{1.245913in}}%
\pgfpathlineto{\pgfqpoint{4.737148in}{1.238164in}}%
\pgfpathlineto{\pgfqpoint{4.739472in}{1.247272in}}%
\pgfpathlineto{\pgfqpoint{4.744119in}{1.282023in}}%
\pgfpathlineto{\pgfqpoint{4.746443in}{1.236416in}}%
\pgfpathlineto{\pgfqpoint{4.748767in}{1.262982in}}%
\pgfpathlineto{\pgfqpoint{4.751091in}{1.264420in}}%
\pgfpathlineto{\pgfqpoint{4.753415in}{1.241780in}}%
\pgfpathlineto{\pgfqpoint{4.755739in}{1.245391in}}%
\pgfpathlineto{\pgfqpoint{4.758063in}{1.243915in}}%
\pgfpathlineto{\pgfqpoint{4.760387in}{1.248341in}}%
\pgfpathlineto{\pgfqpoint{4.762711in}{1.264970in}}%
\pgfpathlineto{\pgfqpoint{4.765035in}{1.252679in}}%
\pgfpathlineto{\pgfqpoint{4.767359in}{1.250956in}}%
\pgfpathlineto{\pgfqpoint{4.769683in}{1.263361in}}%
\pgfpathlineto{\pgfqpoint{4.772007in}{1.249308in}}%
\pgfpathlineto{\pgfqpoint{4.774331in}{1.244107in}}%
\pgfpathlineto{\pgfqpoint{4.776655in}{1.232035in}}%
\pgfpathlineto{\pgfqpoint{4.778979in}{1.239580in}}%
\pgfpathlineto{\pgfqpoint{4.781303in}{1.223205in}}%
\pgfpathlineto{\pgfqpoint{4.783627in}{1.264671in}}%
\pgfpathlineto{\pgfqpoint{4.785951in}{1.266779in}}%
\pgfpathlineto{\pgfqpoint{4.788275in}{1.239112in}}%
\pgfpathlineto{\pgfqpoint{4.790599in}{1.264793in}}%
\pgfpathlineto{\pgfqpoint{4.792923in}{1.230372in}}%
\pgfpathlineto{\pgfqpoint{4.795247in}{1.277876in}}%
\pgfpathlineto{\pgfqpoint{4.797571in}{1.256213in}}%
\pgfpathlineto{\pgfqpoint{4.799895in}{1.218618in}}%
\pgfpathlineto{\pgfqpoint{4.802219in}{1.278556in}}%
\pgfpathlineto{\pgfqpoint{4.804543in}{1.242465in}}%
\pgfpathlineto{\pgfqpoint{4.806867in}{1.245815in}}%
\pgfpathlineto{\pgfqpoint{4.809191in}{1.270201in}}%
\pgfpathlineto{\pgfqpoint{4.811515in}{1.239634in}}%
\pgfpathlineto{\pgfqpoint{4.816162in}{1.254523in}}%
\pgfpathlineto{\pgfqpoint{4.818486in}{1.215742in}}%
\pgfpathlineto{\pgfqpoint{4.820810in}{1.253559in}}%
\pgfpathlineto{\pgfqpoint{4.823134in}{1.251789in}}%
\pgfpathlineto{\pgfqpoint{4.825458in}{1.253826in}}%
\pgfpathlineto{\pgfqpoint{4.827782in}{1.244131in}}%
\pgfpathlineto{\pgfqpoint{4.830106in}{1.273276in}}%
\pgfpathlineto{\pgfqpoint{4.832430in}{1.246090in}}%
\pgfpathlineto{\pgfqpoint{4.834754in}{1.247735in}}%
\pgfpathlineto{\pgfqpoint{4.837078in}{1.276038in}}%
\pgfpathlineto{\pgfqpoint{4.839402in}{1.278274in}}%
\pgfpathlineto{\pgfqpoint{4.841726in}{1.253702in}}%
\pgfpathlineto{\pgfqpoint{4.844050in}{1.265996in}}%
\pgfpathlineto{\pgfqpoint{4.848698in}{1.227584in}}%
\pgfpathlineto{\pgfqpoint{4.851022in}{1.264261in}}%
\pgfpathlineto{\pgfqpoint{4.853346in}{1.275850in}}%
\pgfpathlineto{\pgfqpoint{4.855670in}{1.257944in}}%
\pgfpathlineto{\pgfqpoint{4.857994in}{1.277085in}}%
\pgfpathlineto{\pgfqpoint{4.860318in}{1.247794in}}%
\pgfpathlineto{\pgfqpoint{4.862642in}{1.240102in}}%
\pgfpathlineto{\pgfqpoint{4.864966in}{1.260334in}}%
\pgfpathlineto{\pgfqpoint{4.867290in}{1.230331in}}%
\pgfpathlineto{\pgfqpoint{4.869614in}{1.270784in}}%
\pgfpathlineto{\pgfqpoint{4.871938in}{1.270454in}}%
\pgfpathlineto{\pgfqpoint{4.874262in}{1.264536in}}%
\pgfpathlineto{\pgfqpoint{4.876586in}{1.247907in}}%
\pgfpathlineto{\pgfqpoint{4.878910in}{1.242052in}}%
\pgfpathlineto{\pgfqpoint{4.881234in}{1.259597in}}%
\pgfpathlineto{\pgfqpoint{4.883558in}{1.247501in}}%
\pgfpathlineto{\pgfqpoint{4.885881in}{1.255951in}}%
\pgfpathlineto{\pgfqpoint{4.888205in}{1.280297in}}%
\pgfpathlineto{\pgfqpoint{4.890529in}{1.270227in}}%
\pgfpathlineto{\pgfqpoint{4.892853in}{1.267064in}}%
\pgfpathlineto{\pgfqpoint{4.895177in}{1.241445in}}%
\pgfpathlineto{\pgfqpoint{4.899825in}{1.281816in}}%
\pgfpathlineto{\pgfqpoint{4.902149in}{1.275220in}}%
\pgfpathlineto{\pgfqpoint{4.904473in}{1.273577in}}%
\pgfpathlineto{\pgfqpoint{4.906797in}{1.231915in}}%
\pgfpathlineto{\pgfqpoint{4.909121in}{1.285053in}}%
\pgfpathlineto{\pgfqpoint{4.911445in}{1.243710in}}%
\pgfpathlineto{\pgfqpoint{4.913769in}{1.276365in}}%
\pgfpathlineto{\pgfqpoint{4.916093in}{1.274809in}}%
\pgfpathlineto{\pgfqpoint{4.918417in}{1.263650in}}%
\pgfpathlineto{\pgfqpoint{4.920741in}{1.282509in}}%
\pgfpathlineto{\pgfqpoint{4.923065in}{1.248237in}}%
\pgfpathlineto{\pgfqpoint{4.925389in}{1.271300in}}%
\pgfpathlineto{\pgfqpoint{4.927713in}{1.244588in}}%
\pgfpathlineto{\pgfqpoint{4.930037in}{1.269510in}}%
\pgfpathlineto{\pgfqpoint{4.932361in}{1.248614in}}%
\pgfpathlineto{\pgfqpoint{4.934685in}{1.282305in}}%
\pgfpathlineto{\pgfqpoint{4.937009in}{1.278007in}}%
\pgfpathlineto{\pgfqpoint{4.939333in}{1.259875in}}%
\pgfpathlineto{\pgfqpoint{4.941657in}{1.299005in}}%
\pgfpathlineto{\pgfqpoint{4.943981in}{1.264362in}}%
\pgfpathlineto{\pgfqpoint{4.946305in}{1.251326in}}%
\pgfpathlineto{\pgfqpoint{4.948629in}{1.280348in}}%
\pgfpathlineto{\pgfqpoint{4.950953in}{1.234346in}}%
\pgfpathlineto{\pgfqpoint{4.955600in}{1.290352in}}%
\pgfpathlineto{\pgfqpoint{4.957924in}{1.287610in}}%
\pgfpathlineto{\pgfqpoint{4.960248in}{1.261225in}}%
\pgfpathlineto{\pgfqpoint{4.962572in}{1.273437in}}%
\pgfpathlineto{\pgfqpoint{4.964896in}{1.263638in}}%
\pgfpathlineto{\pgfqpoint{4.967220in}{1.275071in}}%
\pgfpathlineto{\pgfqpoint{4.969544in}{1.273441in}}%
\pgfpathlineto{\pgfqpoint{4.971868in}{1.283061in}}%
\pgfpathlineto{\pgfqpoint{4.974192in}{1.286789in}}%
\pgfpathlineto{\pgfqpoint{4.976516in}{1.263935in}}%
\pgfpathlineto{\pgfqpoint{4.978840in}{1.289669in}}%
\pgfpathlineto{\pgfqpoint{4.981164in}{1.267065in}}%
\pgfpathlineto{\pgfqpoint{4.983488in}{1.271622in}}%
\pgfpathlineto{\pgfqpoint{4.985812in}{1.259934in}}%
\pgfpathlineto{\pgfqpoint{4.990460in}{1.281845in}}%
\pgfpathlineto{\pgfqpoint{4.992784in}{1.300714in}}%
\pgfpathlineto{\pgfqpoint{4.995108in}{1.276217in}}%
\pgfpathlineto{\pgfqpoint{4.997432in}{1.265332in}}%
\pgfpathlineto{\pgfqpoint{4.999756in}{1.283532in}}%
\pgfpathlineto{\pgfqpoint{5.004404in}{1.247651in}}%
\pgfpathlineto{\pgfqpoint{5.006728in}{1.285548in}}%
\pgfpathlineto{\pgfqpoint{5.009052in}{0.933697in}}%
\pgfpathlineto{\pgfqpoint{5.011376in}{0.936341in}}%
\pgfpathlineto{\pgfqpoint{5.013700in}{0.950057in}}%
\pgfpathlineto{\pgfqpoint{5.016024in}{0.916056in}}%
\pgfpathlineto{\pgfqpoint{5.018348in}{0.927817in}}%
\pgfpathlineto{\pgfqpoint{5.020672in}{0.923078in}}%
\pgfpathlineto{\pgfqpoint{5.022996in}{0.954148in}}%
\pgfpathlineto{\pgfqpoint{5.027643in}{0.918182in}}%
\pgfpathlineto{\pgfqpoint{5.034615in}{0.949994in}}%
\pgfpathlineto{\pgfqpoint{5.036939in}{0.945613in}}%
\pgfpathlineto{\pgfqpoint{5.039263in}{0.937626in}}%
\pgfpathlineto{\pgfqpoint{5.043911in}{0.965546in}}%
\pgfpathlineto{\pgfqpoint{5.046235in}{0.936459in}}%
\pgfpathlineto{\pgfqpoint{5.048559in}{0.947217in}}%
\pgfpathlineto{\pgfqpoint{5.050883in}{0.924268in}}%
\pgfpathlineto{\pgfqpoint{5.053207in}{0.948058in}}%
\pgfpathlineto{\pgfqpoint{5.057855in}{0.926024in}}%
\pgfpathlineto{\pgfqpoint{5.060179in}{0.931115in}}%
\pgfpathlineto{\pgfqpoint{5.062503in}{0.947905in}}%
\pgfpathlineto{\pgfqpoint{5.064827in}{0.935731in}}%
\pgfpathlineto{\pgfqpoint{5.067151in}{0.943527in}}%
\pgfpathlineto{\pgfqpoint{5.069475in}{0.966546in}}%
\pgfpathlineto{\pgfqpoint{5.071799in}{0.955135in}}%
\pgfpathlineto{\pgfqpoint{5.074123in}{0.972940in}}%
\pgfpathlineto{\pgfqpoint{5.076447in}{0.942692in}}%
\pgfpathlineto{\pgfqpoint{5.081095in}{0.947953in}}%
\pgfpathlineto{\pgfqpoint{5.083419in}{0.952900in}}%
\pgfpathlineto{\pgfqpoint{5.085743in}{0.962599in}}%
\pgfpathlineto{\pgfqpoint{5.088067in}{0.962635in}}%
\pgfpathlineto{\pgfqpoint{5.090391in}{0.971169in}}%
\pgfpathlineto{\pgfqpoint{5.092715in}{0.938450in}}%
\pgfpathlineto{\pgfqpoint{5.095039in}{0.956769in}}%
\pgfpathlineto{\pgfqpoint{5.097362in}{0.953862in}}%
\pgfpathlineto{\pgfqpoint{5.099686in}{0.955423in}}%
\pgfpathlineto{\pgfqpoint{5.102010in}{0.965418in}}%
\pgfpathlineto{\pgfqpoint{5.104334in}{0.944250in}}%
\pgfpathlineto{\pgfqpoint{5.106658in}{0.949768in}}%
\pgfpathlineto{\pgfqpoint{5.108982in}{0.984181in}}%
\pgfpathlineto{\pgfqpoint{5.111306in}{0.945369in}}%
\pgfpathlineto{\pgfqpoint{5.118278in}{0.959110in}}%
\pgfpathlineto{\pgfqpoint{5.120602in}{0.945166in}}%
\pgfpathlineto{\pgfqpoint{5.122926in}{0.953151in}}%
\pgfpathlineto{\pgfqpoint{5.125250in}{0.949647in}}%
\pgfpathlineto{\pgfqpoint{5.129898in}{0.931700in}}%
\pgfpathlineto{\pgfqpoint{5.132222in}{0.933053in}}%
\pgfpathlineto{\pgfqpoint{5.134546in}{0.953017in}}%
\pgfpathlineto{\pgfqpoint{5.136870in}{0.944534in}}%
\pgfpathlineto{\pgfqpoint{5.139194in}{0.941002in}}%
\pgfpathlineto{\pgfqpoint{5.141518in}{0.973248in}}%
\pgfpathlineto{\pgfqpoint{5.148490in}{0.936867in}}%
\pgfpathlineto{\pgfqpoint{5.150814in}{0.963776in}}%
\pgfpathlineto{\pgfqpoint{5.153138in}{0.945898in}}%
\pgfpathlineto{\pgfqpoint{5.155462in}{0.951912in}}%
\pgfpathlineto{\pgfqpoint{5.157786in}{0.966176in}}%
\pgfpathlineto{\pgfqpoint{5.160110in}{0.962011in}}%
\pgfpathlineto{\pgfqpoint{5.162434in}{0.966487in}}%
\pgfpathlineto{\pgfqpoint{5.164758in}{0.960031in}}%
\pgfpathlineto{\pgfqpoint{5.169405in}{0.969943in}}%
\pgfpathlineto{\pgfqpoint{5.171729in}{0.922958in}}%
\pgfpathlineto{\pgfqpoint{5.174053in}{0.965773in}}%
\pgfpathlineto{\pgfqpoint{5.176377in}{0.969828in}}%
\pgfpathlineto{\pgfqpoint{5.178701in}{0.964806in}}%
\pgfpathlineto{\pgfqpoint{5.181025in}{0.940792in}}%
\pgfpathlineto{\pgfqpoint{5.183349in}{0.955604in}}%
\pgfpathlineto{\pgfqpoint{5.185673in}{0.935096in}}%
\pgfpathlineto{\pgfqpoint{5.187997in}{0.957510in}}%
\pgfpathlineto{\pgfqpoint{5.190321in}{0.957938in}}%
\pgfpathlineto{\pgfqpoint{5.192645in}{0.972414in}}%
\pgfpathlineto{\pgfqpoint{5.194969in}{0.957899in}}%
\pgfpathlineto{\pgfqpoint{5.197293in}{0.961884in}}%
\pgfpathlineto{\pgfqpoint{5.199617in}{0.938924in}}%
\pgfpathlineto{\pgfqpoint{5.204265in}{0.956626in}}%
\pgfpathlineto{\pgfqpoint{5.206589in}{0.979712in}}%
\pgfpathlineto{\pgfqpoint{5.208913in}{0.960085in}}%
\pgfpathlineto{\pgfqpoint{5.211237in}{0.971738in}}%
\pgfpathlineto{\pgfqpoint{5.213561in}{0.945131in}}%
\pgfpathlineto{\pgfqpoint{5.215885in}{0.971641in}}%
\pgfpathlineto{\pgfqpoint{5.218209in}{0.966602in}}%
\pgfpathlineto{\pgfqpoint{5.220533in}{0.973919in}}%
\pgfpathlineto{\pgfqpoint{5.222857in}{0.933287in}}%
\pgfpathlineto{\pgfqpoint{5.225181in}{0.963364in}}%
\pgfpathlineto{\pgfqpoint{5.227505in}{0.966637in}}%
\pgfpathlineto{\pgfqpoint{5.229829in}{0.951308in}}%
\pgfpathlineto{\pgfqpoint{5.232153in}{0.950034in}}%
\pgfpathlineto{\pgfqpoint{5.234477in}{0.956501in}}%
\pgfpathlineto{\pgfqpoint{5.236800in}{0.957566in}}%
\pgfpathlineto{\pgfqpoint{5.239124in}{0.935616in}}%
\pgfpathlineto{\pgfqpoint{5.241448in}{0.941629in}}%
\pgfpathlineto{\pgfqpoint{5.246096in}{0.973995in}}%
\pgfpathlineto{\pgfqpoint{5.248420in}{0.954619in}}%
\pgfpathlineto{\pgfqpoint{5.250744in}{0.965845in}}%
\pgfpathlineto{\pgfqpoint{5.253068in}{0.945633in}}%
\pgfpathlineto{\pgfqpoint{5.255392in}{0.977400in}}%
\pgfpathlineto{\pgfqpoint{5.257716in}{0.974948in}}%
\pgfpathlineto{\pgfqpoint{5.262364in}{0.957084in}}%
\pgfpathlineto{\pgfqpoint{5.264688in}{0.990329in}}%
\pgfpathlineto{\pgfqpoint{5.267012in}{0.966652in}}%
\pgfpathlineto{\pgfqpoint{5.269336in}{0.983922in}}%
\pgfpathlineto{\pgfqpoint{5.271660in}{0.970460in}}%
\pgfpathlineto{\pgfqpoint{5.273984in}{0.990732in}}%
\pgfpathlineto{\pgfqpoint{5.276308in}{0.956629in}}%
\pgfpathlineto{\pgfqpoint{5.278632in}{0.950065in}}%
\pgfpathlineto{\pgfqpoint{5.280956in}{0.959419in}}%
\pgfpathlineto{\pgfqpoint{5.283280in}{0.943738in}}%
\pgfpathlineto{\pgfqpoint{5.285604in}{0.959795in}}%
\pgfpathlineto{\pgfqpoint{5.287928in}{0.949574in}}%
\pgfpathlineto{\pgfqpoint{5.290252in}{0.932019in}}%
\pgfpathlineto{\pgfqpoint{5.292576in}{0.974755in}}%
\pgfpathlineto{\pgfqpoint{5.294900in}{0.934746in}}%
\pgfpathlineto{\pgfqpoint{5.297224in}{0.983568in}}%
\pgfpathlineto{\pgfqpoint{5.299548in}{1.250383in}}%
\pgfpathlineto{\pgfqpoint{5.301872in}{1.221527in}}%
\pgfpathlineto{\pgfqpoint{5.304196in}{1.245036in}}%
\pgfpathlineto{\pgfqpoint{5.306519in}{1.236786in}}%
\pgfpathlineto{\pgfqpoint{5.308843in}{1.236236in}}%
\pgfpathlineto{\pgfqpoint{5.311167in}{1.242674in}}%
\pgfpathlineto{\pgfqpoint{5.313491in}{1.240653in}}%
\pgfpathlineto{\pgfqpoint{5.315815in}{1.214585in}}%
\pgfpathlineto{\pgfqpoint{5.318139in}{1.229496in}}%
\pgfpathlineto{\pgfqpoint{5.320463in}{1.227628in}}%
\pgfpathlineto{\pgfqpoint{5.322787in}{1.274417in}}%
\pgfpathlineto{\pgfqpoint{5.325111in}{1.223029in}}%
\pgfpathlineto{\pgfqpoint{5.327435in}{1.231125in}}%
\pgfpathlineto{\pgfqpoint{5.329759in}{1.256038in}}%
\pgfpathlineto{\pgfqpoint{5.332083in}{1.235943in}}%
\pgfpathlineto{\pgfqpoint{5.334407in}{1.268360in}}%
\pgfpathlineto{\pgfqpoint{5.336731in}{1.244775in}}%
\pgfpathlineto{\pgfqpoint{5.339055in}{1.238064in}}%
\pgfpathlineto{\pgfqpoint{5.341379in}{1.210877in}}%
\pgfpathlineto{\pgfqpoint{5.348351in}{1.257209in}}%
\pgfpathlineto{\pgfqpoint{5.350675in}{1.215307in}}%
\pgfpathlineto{\pgfqpoint{5.352999in}{1.253870in}}%
\pgfpathlineto{\pgfqpoint{5.355323in}{1.217734in}}%
\pgfpathlineto{\pgfqpoint{5.357647in}{1.247139in}}%
\pgfpathlineto{\pgfqpoint{5.359971in}{1.229260in}}%
\pgfpathlineto{\pgfqpoint{5.362295in}{1.236200in}}%
\pgfpathlineto{\pgfqpoint{5.364619in}{1.277858in}}%
\pgfpathlineto{\pgfqpoint{5.369267in}{1.242399in}}%
\pgfpathlineto{\pgfqpoint{5.371591in}{1.245093in}}%
\pgfpathlineto{\pgfqpoint{5.373915in}{1.261241in}}%
\pgfpathlineto{\pgfqpoint{5.376239in}{1.254193in}}%
\pgfpathlineto{\pgfqpoint{5.378562in}{1.251500in}}%
\pgfpathlineto{\pgfqpoint{5.380886in}{1.242502in}}%
\pgfpathlineto{\pgfqpoint{5.383210in}{1.225779in}}%
\pgfpathlineto{\pgfqpoint{5.385534in}{1.225996in}}%
\pgfpathlineto{\pgfqpoint{5.387858in}{1.229944in}}%
\pgfpathlineto{\pgfqpoint{5.390182in}{1.269548in}}%
\pgfpathlineto{\pgfqpoint{5.392506in}{1.255676in}}%
\pgfpathlineto{\pgfqpoint{5.394830in}{1.215812in}}%
\pgfpathlineto{\pgfqpoint{5.399478in}{1.265161in}}%
\pgfpathlineto{\pgfqpoint{5.401802in}{1.263571in}}%
\pgfpathlineto{\pgfqpoint{5.404126in}{1.234617in}}%
\pgfpathlineto{\pgfqpoint{5.406450in}{1.266037in}}%
\pgfpathlineto{\pgfqpoint{5.408774in}{1.229486in}}%
\pgfpathlineto{\pgfqpoint{5.411098in}{1.223302in}}%
\pgfpathlineto{\pgfqpoint{5.413422in}{1.242362in}}%
\pgfpathlineto{\pgfqpoint{5.415746in}{1.251916in}}%
\pgfpathlineto{\pgfqpoint{5.418070in}{1.242523in}}%
\pgfpathlineto{\pgfqpoint{5.420394in}{1.237833in}}%
\pgfpathlineto{\pgfqpoint{5.422718in}{1.238315in}}%
\pgfpathlineto{\pgfqpoint{5.425042in}{1.245945in}}%
\pgfpathlineto{\pgfqpoint{5.427366in}{1.260411in}}%
\pgfpathlineto{\pgfqpoint{5.429690in}{1.262354in}}%
\pgfpathlineto{\pgfqpoint{5.432014in}{1.233849in}}%
\pgfpathlineto{\pgfqpoint{5.434338in}{1.237824in}}%
\pgfpathlineto{\pgfqpoint{5.436662in}{1.267845in}}%
\pgfpathlineto{\pgfqpoint{5.438986in}{1.233827in}}%
\pgfpathlineto{\pgfqpoint{5.441310in}{1.239629in}}%
\pgfpathlineto{\pgfqpoint{5.443634in}{1.233454in}}%
\pgfpathlineto{\pgfqpoint{5.448281in}{1.236784in}}%
\pgfpathlineto{\pgfqpoint{5.450605in}{1.246349in}}%
\pgfpathlineto{\pgfqpoint{5.452929in}{1.249771in}}%
\pgfpathlineto{\pgfqpoint{5.455253in}{1.219185in}}%
\pgfpathlineto{\pgfqpoint{5.457577in}{1.241584in}}%
\pgfpathlineto{\pgfqpoint{5.459901in}{1.231732in}}%
\pgfpathlineto{\pgfqpoint{5.464549in}{1.258189in}}%
\pgfpathlineto{\pgfqpoint{5.466873in}{1.226980in}}%
\pgfpathlineto{\pgfqpoint{5.469197in}{1.237503in}}%
\pgfpathlineto{\pgfqpoint{5.471521in}{1.224019in}}%
\pgfpathlineto{\pgfqpoint{5.473845in}{1.253629in}}%
\pgfpathlineto{\pgfqpoint{5.478493in}{1.226495in}}%
\pgfpathlineto{\pgfqpoint{5.480817in}{1.253306in}}%
\pgfpathlineto{\pgfqpoint{5.483141in}{1.248304in}}%
\pgfpathlineto{\pgfqpoint{5.485465in}{1.237123in}}%
\pgfpathlineto{\pgfqpoint{5.490113in}{1.282840in}}%
\pgfpathlineto{\pgfqpoint{5.492437in}{1.228889in}}%
\pgfpathlineto{\pgfqpoint{5.494761in}{1.249835in}}%
\pgfpathlineto{\pgfqpoint{5.497085in}{1.231097in}}%
\pgfpathlineto{\pgfqpoint{5.499409in}{1.261263in}}%
\pgfpathlineto{\pgfqpoint{5.501733in}{1.261039in}}%
\pgfpathlineto{\pgfqpoint{5.504057in}{1.244333in}}%
\pgfpathlineto{\pgfqpoint{5.506381in}{1.209992in}}%
\pgfpathlineto{\pgfqpoint{5.508705in}{1.247355in}}%
\pgfpathlineto{\pgfqpoint{5.511029in}{1.256291in}}%
\pgfpathlineto{\pgfqpoint{5.513353in}{1.243331in}}%
\pgfpathlineto{\pgfqpoint{5.515677in}{1.262803in}}%
\pgfpathlineto{\pgfqpoint{5.520324in}{1.221885in}}%
\pgfpathlineto{\pgfqpoint{5.522648in}{1.238845in}}%
\pgfpathlineto{\pgfqpoint{5.524972in}{1.235085in}}%
\pgfpathlineto{\pgfqpoint{5.527296in}{1.265377in}}%
\pgfpathlineto{\pgfqpoint{5.529620in}{1.225118in}}%
\pgfpathlineto{\pgfqpoint{5.531944in}{1.236613in}}%
\pgfpathlineto{\pgfqpoint{5.534268in}{1.258747in}}%
\pgfpathlineto{\pgfqpoint{5.536592in}{1.241273in}}%
\pgfpathlineto{\pgfqpoint{5.541240in}{1.253747in}}%
\pgfpathlineto{\pgfqpoint{5.543564in}{1.253481in}}%
\pgfpathlineto{\pgfqpoint{5.545888in}{1.223494in}}%
\pgfpathlineto{\pgfqpoint{5.548212in}{1.262535in}}%
\pgfpathlineto{\pgfqpoint{5.550536in}{1.234581in}}%
\pgfpathlineto{\pgfqpoint{5.552860in}{1.263434in}}%
\pgfpathlineto{\pgfqpoint{5.555184in}{1.255208in}}%
\pgfpathlineto{\pgfqpoint{5.557508in}{1.241042in}}%
\pgfpathlineto{\pgfqpoint{5.559832in}{1.246261in}}%
\pgfpathlineto{\pgfqpoint{5.562156in}{1.263408in}}%
\pgfpathlineto{\pgfqpoint{5.564480in}{1.241848in}}%
\pgfpathlineto{\pgfqpoint{5.566804in}{1.247896in}}%
\pgfpathlineto{\pgfqpoint{5.569128in}{1.247245in}}%
\pgfpathlineto{\pgfqpoint{5.571452in}{1.244222in}}%
\pgfpathlineto{\pgfqpoint{5.573776in}{1.247495in}}%
\pgfpathlineto{\pgfqpoint{5.576100in}{1.233656in}}%
\pgfpathlineto{\pgfqpoint{5.578424in}{1.235199in}}%
\pgfpathlineto{\pgfqpoint{5.580748in}{1.241202in}}%
\pgfpathlineto{\pgfqpoint{5.583072in}{1.261071in}}%
\pgfpathlineto{\pgfqpoint{5.585396in}{1.237511in}}%
\pgfpathlineto{\pgfqpoint{5.587720in}{0.915953in}}%
\pgfpathlineto{\pgfqpoint{5.587720in}{0.915953in}}%
\pgfusepath{stroke}%
\end{pgfscope}%
\begin{pgfscope}%
\pgfsetrectcap%
\pgfsetmiterjoin%
\pgfsetlinewidth{1.254687pt}%
\definecolor{currentstroke}{rgb}{0.800000,0.800000,0.800000}%
\pgfsetstrokecolor{currentstroke}%
\pgfsetdash{}{0pt}%
\pgfpathmoveto{\pgfqpoint{0.709829in}{0.654666in}}%
\pgfpathlineto{\pgfqpoint{0.709829in}{1.542204in}}%
\pgfusepath{stroke}%
\end{pgfscope}%
\begin{pgfscope}%
\pgfsetrectcap%
\pgfsetmiterjoin%
\pgfsetlinewidth{1.254687pt}%
\definecolor{currentstroke}{rgb}{0.800000,0.800000,0.800000}%
\pgfsetstrokecolor{currentstroke}%
\pgfsetdash{}{0pt}%
\pgfpathmoveto{\pgfqpoint{5.820000in}{0.654666in}}%
\pgfpathlineto{\pgfqpoint{5.820000in}{1.542204in}}%
\pgfusepath{stroke}%
\end{pgfscope}%
\begin{pgfscope}%
\pgfsetrectcap%
\pgfsetmiterjoin%
\pgfsetlinewidth{1.254687pt}%
\definecolor{currentstroke}{rgb}{0.800000,0.800000,0.800000}%
\pgfsetstrokecolor{currentstroke}%
\pgfsetdash{}{0pt}%
\pgfpathmoveto{\pgfqpoint{0.709829in}{0.654666in}}%
\pgfpathlineto{\pgfqpoint{5.820000in}{0.654666in}}%
\pgfusepath{stroke}%
\end{pgfscope}%
\begin{pgfscope}%
\pgfsetrectcap%
\pgfsetmiterjoin%
\pgfsetlinewidth{1.254687pt}%
\definecolor{currentstroke}{rgb}{0.800000,0.800000,0.800000}%
\pgfsetstrokecolor{currentstroke}%
\pgfsetdash{}{0pt}%
\pgfpathmoveto{\pgfqpoint{0.709829in}{1.542204in}}%
\pgfpathlineto{\pgfqpoint{5.820000in}{1.542204in}}%
\pgfusepath{stroke}%
\end{pgfscope}%
\begin{pgfscope}%
\definecolor{textcolor}{rgb}{0.150000,0.150000,0.150000}%
\pgfsetstrokecolor{textcolor}%
\pgfsetfillcolor{textcolor}%
\pgftext[x=3.264915in,y=1.625537in,,base]{\color{textcolor}{\sffamily\fontsize{12.000000}{14.400000}\selectfont\catcode`\^=\active\def^{\ifmmode\sp\else\^{}\fi}\catcode`\%=\active\def%{\%}Recovered Signals}}%
\end{pgfscope}%
\end{pgfpicture}%
\makeatother%
\endgroup%
}}
    \caption{FastICA recovered signals}
    \label{fig:fastica}
\end{figure}

可视化后所得结果如下图. 可见该算法能清晰得分辨三种信号并将其分离,对原信号的振幅、频率、波形等参数做了很好的还原. \np
对于真实的 \texttt{.wav} 格式的文件,可以使用 Python 的 \texttt{scipy.io.wavfile} 库函数进行读入并进行类似的处理,能够分离出一段音频中的人声和其它声音,也能分离出男声和女声. 同时,若将音频中提取出的环境噪音消去,则能实现降噪的效果.

\section{遇到的问题与展望}

\begin{itemize}
    \item \textbf{遇到的问题与参考说明} \ 
    在查阅不同资料(包括网站博客与相关论文原文)时,发现不同文章中白化方式、迭代处理等方面存在一定的差异,对某些矩阵的定义有一定出入。文\cite{1}中分别介绍了单点迭代和多点迭代两种算法的不同与相应继承关系,定义较为明确,故采用之。应用场景参考了文\cite{3}, 进行了盲源音频分离的模拟. \np
    在 AI 使用上,本文在信号生成与可视化上借助了 ChatGPT-o4 进行生成,减小了代码工作量。 \np
    \item \textbf{展望与改进点} 
    \begin{enumerate}
        \item 在信号采样率很高时,矩阵的维度会很大,上述算法中大量的矩阵分解等运算计算复杂度较高. 可以使用随机 SVD 降维等方式降低计算量.
        \item ICA 假设信号为瞬时线性混合,但是现实中由于不同信号的传播路径不同,接受到信号具有\textbf{时间延迟}和\textbf{滤波效应}. 此时借鉴文\cite{2}采用卷积来混合源信号,能够消除上述干扰.
        \item ICA 严格依赖原信号的非高斯性,对亚高斯信号分离能力差,需要使用更复杂的算法实现对信号的处理.
        \item 音频处理不仅时基于现有数据的静态数据,也有需要实时处理的动态数据例如降噪耳机的算法,需要对该算法进行一些扩展.
    \end{enumerate}
    \item \textbf{本人的一些新发现} 
    \begin{enumerate}
        \item 发现基于特征值分解的白化方法可以代替施密特正交化来将一个矩阵在保持列空间不变的条件下转化为正交阵,改方法计算远远方便于施密特正交化.
        \item 发现在矩阵、向量的多元函数中仍然能够使用牛顿法获得方程的近似解.
    \end{enumerate}
\end{itemize}



\begin{thebibliography}{99}
    \bibitem{1} HYVÄRINEN A, OJA E. Independent component analysis: algorithms and applications[J]. Neural networks, Elsevier, 2000, 13(4–5): 411–430.
    \bibitem{2} 
    SW\_孙维. 
    \textit{盲源分离(BSS)的学习总结(PCA、ICA)}[EB/OL]. 
    CSDN博客, 
    2025. 
    \url{https://wenku.csdn.net/column/3tjk93vie4}.    
    \bibitem{3} hh867308122. \textit{【ICA独立成分分析】数学原理+python代码实现} [EB/OL]. CSDN博客, 2024. https://blog.csdn.net/hh867308122/article/details/144175594.
\end{thebibliography}

\printbibliography

\end{document}