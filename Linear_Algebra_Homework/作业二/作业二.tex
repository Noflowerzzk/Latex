\documentclass{article}
\usepackage{amsmath}  % 数学符号包
\usepackage{amssymb}  % 更多数学符号
\usepackage{enumitem} % 列表样式
\usepackage{fancyhdr} % 页眉设置
\usepackage{geometry} % 页面设置
\usepackage[UTF8]{ctex}
\usepackage{bm}
\usepackage{amsthm}
\everymath{\displaystyle}  % 让所有数学模式都使用 \displaystyle
\newcommand{\lb}{\left\llbracket}
\newcommand{\rb}{\right\rrbracket}
\newcommand{\bn}{\boldsymbol{n}}


\geometry{a4paper, margin=1in}


\pagestyle{fancy}
\fancyhf{}
\fancyhead[C]{线代作业二}
\fancyhead[R]{2025.3.2}


\title{线代作业二}
\author{Noflowerzzk}
\date{2025.3.2}


\begin{document}
\maketitle

\section{}

\begin{proof}
    \begin{itemize}
        \item 当 $b = c = 0$ 时, $T(\alpha(x, y, z)) = \alpha (2x - 4y + 3z, 6x) = \alpha T(x, y, z)$. \\
        且 $T((x_1, y_1, z_1) + (x_2, y_2, z_2)) = (2(x_1 + x_2) - 4(y_1 + y_2) + 3(z_1 + z_2), 6(x_1 + x_2)) = (2x_1 - 4y_1 + 3z_1, 6x_1) + (2x_2 - 4y_2 + 3z_2, 6x_2) = T(x_1, y_1, z_1) + T(x_2, y_2, z_2)$. \\
        故 $T$ 是线性的.
        \item 若 $T$ 是线性变换,则 $T(0, 0, 0) = (0, 0) \Rightarrow b = 0$. \\
        $2T(1, 1, 1) = T(2, 2, 2) \Rightarrow c = 0$.    
    \end{itemize}
\end{proof}

\section{}

\begin{proof} \quad
    \begin{itemize}
        \item 若 $T$ 是单的,则若 $T(u) = T(v) = 0 = T(0)$, 有 $u = v = 0$. 故 $\ker{T} = \{0\}$
        \item 若 $\ker{T} = \{0\}$, 有若 $T(u) = a, T(v) = a$, 且 $u \neq v$, 则 $T(u - v) = 0$, $u - v \in \ker T$, 有 $u - v = 0, i.e. u = v$, $T$ 是单的.
    \end{itemize}
\end{proof}

\section{}

\begin{itemize}
    \item [(1)] $P = (\eta_1, \eta_2, \eta_3)$, 有 $M_\varepsilon P = PA_\eta$, 得 $$M_\varepsilon = \begin{pmatrix}
        -1 & 1 & 2 \\
        2 & 2 & 0 \\
        3 & 0 & 2
    \end{pmatrix}$$
    \item [(2)] $P = (\eta_1, \eta_2, \eta_3)$. 
    \begin{align*}
        P^{-1} = \frac{1}{7}\begin{pmatrix}
            -1 & 3 & 3 \\
            2 & 6 & -1 \\
            2 & -1 & 1
        \end{pmatrix}
    \end{align*}
    即为 $\varepsilon_i$ 在基 $\eta_1, \eta_2, \eta_3$ 下的坐标. 故计算 $\mathcal{A}(\varepsilon_i)$ 用 $\eta_1$ 的坐标展开即得矩阵
    \begin{align*}
        M_\varepsilon = \frac{1}{7}\begin{pmatrix}
            -5 & 20 & 20 \\
            -4 & -5 & -2 \\
            27 & 18 & 24
        \end{pmatrix}
    \end{align*}
\end{itemize}

\section{}

\begin{proof}
    有 $\forall \alpha, \beta \in V, (T(\alpha), \beta) = (\alpha, T^*(\beta)) $
    \begin{itemize}
        \item [(1)] $\forall \alpha \in \ker T^*, T^*(\alpha) = 0$. 故 $(\beta, T^*(\alpha)) = 0 = (T(\beta), \alpha)$,即 $\alpha \in (\mathrm{Im} T)^\perp , \ker T^* \subseteq (\mathrm{Im} T)^\perp$ \\
        $\forall \alpha \in (\mathrm{Im} T)^\perp$, $(T(\beta), \alpha) = 0 = (\beta, T^*(\alpha)), \alpha \in \ker T^*, (\mathrm{Im} T)^\perp \subseteq \ker T^*$. \\
        综上, $(\mathrm{Im} T)^\perp = \ker T^*$
        \item [(2)] 由于 $(\mathrm{Im} T)^\perp = \ker T^*$, $\mathrm{Im} T = (\ker T^*)^\perp$, 即 $\mathrm{Im} T^* = (\ker T)^\perp$
    \end{itemize}
\end{proof}

\section{}

\begin{proof}
    \begin{itemize}
        \item 由于若 $Ax = 0$, 则 $(E - A)x = x, x \in \mathrm{ker} A, x \in \mathrm{Im}A$; 若 $(E - A)x = 0$, 则 $Ax = x, x \in \mathrm{Im}A, x \in \ker (E - A)$. 因此 $\mathrm{Im}(E - A) = \ker A$. 又 $\dim\mathrm{Im}A + \dim \ker A = n$, 故 $r(A) + r(E - A) = n$.
        \item 若 $r(E - A) + r(A) = n$, 则 $\mathrm{Im}A \cap \ker A = \{0\}$. $\forall \alpha \in V$, 可唯一分解为 $\alpha = \beta + \gamma, \beta \in \mathrm{Im}A, \gamma \in \mathrm{Im}(E - A)$. 由于 $A\alpha \in \mathrm{Im}A$, $A\beta \in \mathrm{Im}A$, 因此 $A\gamma \in \mathrm{Im}A$. 又存在 $\eta \in V$, $\gamma = (E - A)\eta$. 故 $(A - A^2)\eta \in \mathrm{Im}A$ 对任意 $\alpha$ 成立. 故 $A - A^2 = 0$, $A$ 是幂等矩阵.
    \end{itemize}
\end{proof}

\section{}

\begin{proof}
    由定义只用证明它是一个线性变换. \\
    $\forall \alpha, \beta \in V$, 令 $\gamma = A(\alpha + \beta) - A\alpha - A\beta$. 有 $(\gamma, \gamma) = (\alpha + \beta) - 2(\alpha + \beta, \alpha) - 2(\alpha + \beta, \beta) + (\alpha, \alpha) + 2(\alpha, \beta) + (\beta, \beta) = 0$. 所以 $\gamma = 0, A(\alpha + \beta) = A\alpha + A\beta$. \\
    
\end{proof}

\section{}

\begin{itemize}
    \item [(1)] 设其矩阵为 $H$. 取镜面的法向量 $\boldsymbol{n} = \alpha - \beta$, 镜面 $\boldsymbol{n}^Tx = 0$ 由此构造 HouseHolder 矩阵 $H = E - 2\frac{\bn\bn^T}{\bn^T\bn}$, 是为镜像变换 $\mathcal{A}$ 的矩阵. \\
    下面验证之. \\
    \begin{align*}
        H\alpha = \alpha - \frac{2}{\bn^T\bn}\alpha\bn\bn^T = \alpha - \frac{2}{\bn^T\bn}(\alpha^T\bn) \bn = \alpha - 2\bn = \beta
    \end{align*}
    \item [(2)] 取 $\bn_1 = \left(-\sin \frac{\alpha}{4}, \cos \frac{\alpha}{4}\right)^T, \bn_2 = \left(-\sin \frac{3\alpha}{4}, \cos \frac{3\alpha}{4}\right)^T$, 由此取 $H_1 = E - 2\frac{\bn_1\bn_1^T}{\bn_1^T\bn_1}, H_2 = E - 2\frac{\bn_2\bn_2^T}{\bn_2^T\bn_2}$. 此时
    \begin{align*}
        H_1H_2 &= \left(E - 2\bn_1\bn_1^T\right)\left(E - \bn_2\bn_2^T\right) \\
        &= \begin{pmatrix}
            \cos^2 \frac{\alpha}{4} & -\sin \frac{\alpha}{4} \cos \frac{\alpha}{4} \\
            -\sin \frac{\alpha}{4} \cos \frac{\alpha}{4} & \sin^2 \frac{\alpha}{4}
        \end{pmatrix}
        \begin{pmatrix}
            \cos^2 \frac{3\alpha}{4} & -\sin \frac{3\alpha}{4} \cos \frac{3\alpha}{4} \\
            -\sin \frac{3\alpha}{4} \cos \frac{3\alpha}{4} & \sin^2 \frac{3\alpha}{4}
        \end{pmatrix} \\
        &= B
    \end{align*}
    \item [(3)] ...
\end{itemize}

\section{}

\begin{itemize}
    \item HouseHolder 变换:取 $\alpha = (-1, 1, 2)^T$ 有 
    \begin{align*}
        H &= E - 2\frac{\alpha^T\alpha}{\alpha\alpha^T} \\
        &= \begin{pmatrix}
            0 & 1 & 1 \\
            1 & 0 & -1 \\
            2 & 2 & -3
        \end{pmatrix}
    \end{align*}
    此时 $H(2, 1, 2)^T = (3, 0, 0)^T$. \\
    \item Givens 变换:取 $G = \begin{pmatrix}
        \cos \theta & -\sin \theta \\
        \sin \theta & \cos \theta
    \end{pmatrix}$, 由 $G(2, 1, 2)^T = (3, 0, 0)^T$, 解得 $\cos \theta = \frac{2}{3}$. 故 $$G = \begin{pmatrix}
        \frac{2}{3} & -\frac{\sqrt{5}}{3} \\
        \frac{\sqrt{5}}{3} & \frac{2}{3}
    \end{pmatrix}$$
    \end{itemize}
    

\end{document}